\documentclass[UTF8]{ctexart}

\usepackage{enumerate}
\usepackage{amsmath, amssymb}
\usepackage{cite}
\usepackage{graphicx}
\usepackage{subfigure}
\usepackage{physics}
\usepackage[colorlinks, linkcolor=black, anchorcolor=black, citecolor=black]{hyperref}

\newcommand*{\res}{\mathrm{res}\;}
\newcommand*{\natnums}{\mathbb{N}}
\newcommand*{\reals}{\mathbb{R}}
\newcommand*{\complexes}{\mathbb{C}}

\title{复变函数}
\author{wujinq}
\date{\today}

\begin{document}

\maketitle

\begin{abstract}
    关于复变函数的一些东西。
\end{abstract}

\section{复变函数的基本概念}

设$f$是定义在复平面上的一个区域上的函数,其实部为$u$,虚部为$v$,自变量$z = x + y \mathrm{i}$,其中$x,y$为实变量。又使用极坐标表示法,记$z = \rho e^{\mathrm{i}\phi}$。取$z_0$为闭复平面上一个点。

可以按照一般的方式定义$f$是不是连续,定义$f$的导数、极限并且判断它的可导条件。可以证明$f$在某一点的极限正是$u$的极限加上$v\mathrm{i}$的极限,$f$连续当且仅当$u,v$连续。

$f$在无穷远点的定义可以使用链式法则: \[
\frac{\mathrm{d} f(z)}{\mathrm{d} z} \bigg |_{\infty} = \lim_{\zeta \rightarrow 0} \left( - \zeta^2 \frac{\mathrm{d}}{\mathrm{d} \zeta} f \left( \frac{1}{\zeta} \right) \right)
\]

\hypertarget{c-rux6761ux4ef6ux4e0eux53efux5bfcux6027}{%
\subsection{C-R条件与可导性}\label{c-rux6761ux4ef6ux4e0eux53efux5bfcux6027}}

我们说$f$在$z_0$满足C-R条件,当且仅当在这一点有 \[
\frac{\partial u}{\partial x} = \frac{\partial v}{\partial y}, \; \frac{\partial u}{\partial y} + \frac{\partial v}{\partial x} = 0.
\] 也即, \[
\frac{\partial u}{\partial \rho} = \frac{1}{\rho} \frac{\partial v}{\partial \phi}, \; \frac{\partial v}{\partial \rho} = - \frac{1}{\rho} \frac{\partial u}{\partial \phi}.
\]

如果$f$在$z_0$可导,那么它必定在这一点满足C-R条件,但是\textbf{反之不然}。

不过,如果在$z_0$处$\partial_x u, \partial_y u, \partial_x v, \partial_y v$存在且连续,且C-R条件成立,则$f$在这一点可导。

因此,$f$在$z_0$可导,当且仅当,在$z_0$处$\partial_x u, \partial_y u, \partial_x v, \partial_y v$存在且连续,且C-R条件成立。

\hypertarget{ux89e3ux6790ux51fdux6570}{%
\subsection{解析函数}\label{ux89e3ux6790ux51fdux6570}}

$f$在集合$D$内解析,当且仅当$f$在$D$内处处可导。$f$的定义域内不可导的点称为奇点。

如果一个函数在某个区域内是解析的,那么它一定在这个区域内连续。并且,如果一个函数在一个区域内是连续的,那么它在这个区域内解析的充要条件是C-R条件处处成立。因此,一个函数在一个区域内解析的充要条件是,它在这个区域内连续且C-R条件处处成立。

若$f$在$D$内解析,则$D$内$u,v$的等值线正交,并且$u, v$都是调和函数。事实上,$u, v$是共轭调和函数,当且仅当
\[
\frac{\partial u}{\partial x} = \frac{\partial v}{\partial y}, \; \frac{\partial u}{\partial y} + \frac{\partial v}{\partial x} = 0,
\] 因此,$u, v$是共轭调和函数,当且仅当$f$是解析函数。

\emph{在得出上面的结论的时候我们用到了$\partial_x, \partial_y$可以交换的性质。这要求$u, v$具有比较好的性质。}

因此,如果$f$是解析函数,那么我们有: \[
u = \int_L (\frac{\partial v}{\partial y} \mathrm{d}x - \frac{\partial v}{\partial x} \mathrm{d}y) + C_1, \; 
v = \int_L (-\frac{\partial u}{\partial y}\mathrm{d}x + \frac{\partial u}{\partial x} \mathrm{d}y) + C_2,
\]
其中积分路径要让积分有定义,$C_1, C_2$在不同的子区域内保持不变,但未必在整个区域上取固定的值。(例如,它们也许在上半空间取0,在下半空间取$2\pi$)在$f$保持解析的区域是单连通的时候,$C_1, C_2$一定是真正的常数。

\hypertarget{ux591aux503cux51fdux6570}{%
\subsection{多值函数}\label{ux591aux503cux51fdux6570}}

\hypertarget{ux5b9aux4e49}{%
\subsubsection{定义}\label{ux5b9aux4e49}}

设$f, g$是某个区域$D$上的解析函数。对于$z$,我们考虑满足下面条件的$w$:
\[
f(w) = g(z)
\]
不失一般性地要求$f(0)=0$。形式地,我们可以使用反函数解出$w$,但显然$f$有可能将不同的复数映射到同一个结果上。因此,对同一个$z$,我们会有不同的$w$。我们称从$z$到所有可能的$w$有一个\textbf{多值函数}。多值函数不是一个从$\mathbb{C}$到$\mathbb{C}$的函数。

\subsubsection{支点}

我们首先考虑一个简单的情况:$g$是一个多项式函数。在这种情况下我们总是可以把$g$做因式分解,得到
\[
g(z) = a (z - z_1)^{i_1} \ldots (z - z_n)^{i_n},
\] 当然不失一般性地可以将$a$移到左边并重新定义$f$,从而 \[
f(w) = g(z) = (z - z_1)^{i_1} \ldots (z - z_n)^{i_n}.
\]
我们现在讨论某个$z_k$。在它的一个很小的邻域内取一点$z=z_k + \rho e^{\mathrm{i}\phi}$。我们有
\[
(z_k + \rho e^{\mathrm{i}\phi} - z_1)^{i_1} \cdots \rho^{i_k} e^{\mathrm{i}i_k \phi} \cdots (z_k + \rho e^{\mathrm{i}\phi} - z_n)^{i_n} = f(w)
\] 使用小量近似,令$z \rightarrow z_k$,有 \[
(z_k - z)^{i_1} \cdots (z_k - z_n)^{i_n} \rho^{i_k} e^{\mathrm{i} i_k \phi} \approx f(w)
\]

现在我们让$z$绕某个自己不交叉的闭合路径$L$(直观来看就是一个不规则的圆圈)运动,即取$z=z(t)$,$t$为实数,$z(t)$是一个连续的周期函数;我们同时要求$w$沿着某条轨道\textbf{连续地}运动,即$w=w(t)$,且要让$f(w)=g(z)$始终成立。显然,对同一个$t$,$w(t)$给出了$w$在$z=z(t)$时的\textbf{一个}(未必是全部)取值。使用这样的方式我们实际上使用了一个局域上的$g$的反函数,它总是存在的。

在$z$的运动轨迹的内部没有包含$z_k$的时候,$z$绕$L$一周,运动开始和结束时$f(w)$的取值并不会有变化,因为$\phi$随着$t$或是先变大后减小,或是先减小后变大。但是当$L$的内部有$z_k$时,在运动开始和结束的时候$\phi$相差了$2\pi$,因此在运动开始、停止的点$z_0 = r_0 e^{\mathrm{i}\phi_0}=z(t_0)=z(t_1)$处同时有
\[
g(w(t_0)) = g(w(t_1)) = f(r_0 e^{\mathrm{i}\phi_0})
\]
如果不管怎么指定$w(t)$都有$w(t_0)=w(t_1)$,那么$g$就是可逆的,此时$w$就可以写成$z$的一个单值函数。但如果$w(t_1) \neq w(t_0)$,那么$g$是不可逆的,此时$w$要写成$z$的一个多值函数。

上面是以$w(t_0)$为一开始$w$的取值,转一圈以后的结果。我们当然可以一直这样转下去,不断增大$t$,然后得到一连串$w(t_n)$。我们看到,如果把$w(t)$在$z(t)=z_0$时的值视为$z_0$处$w$的取值,那么$w$是有可能随着转圈数目增大而变化的。

这种``使转圈以后$z_0$处的$w$会变化''的$z_k$就称为\textbf{支点}。

还有一种情况是,取$z \rightarrow \infty$,此时 \[
f(w) \approx \rho^{i_1 + \cdots + i_n} e^{\mathrm{i}(i_1 + \cdots + i_n)\phi}
\]
我们同样,让$z$绕着无穷远点转圈,也就是让$\rho$取很大,而让$\phi$转一整圈。设$z_0$为运动开始和结束的点,同样可以得到
\[
g(w(t_0)) = g(w(t_1)) = f(r_0 e^{\mathrm{i}\phi_0})
\]
如果转圈之后$z_0$处的$w$会变化,那么我们说$\infty$也是一个支点。

容易看出,如果一个点是支点,那么任意一条非常接近它的轨道上任意取一个$z_k$,$z_k$都对应多个$w$。事实上我们可以说,支点是满足若$w$存在则必定唯一的所有点。

复平面上的同伦曲线理论(又或者别的不知道什么东西)保证了 -
只要我们选择的轨道$L$(也就是$z(t)$)和$w(t)$性质足够良好,就能够保证不同轨道给出的$w(t_0), w(t_1), \ldots$最终都能够覆盖对应$z_0=z(t_0)=z(t_1)=\ldots$的全部$w$,也就是说,多值函数在这个点的所有取值都可以通过``转圈''获得(要同时考虑顺时针转和逆时针转);而显然,通过``转圈''产生的所有$w(t_i)$都是多值函数在这个点的取值;
-
如果某一组$z(t), w(t)$下有$z(t_n)=z(t_0)$,也就是转了$n$圈以后$w$的值恢复,那么所有$z(t), w(t)$下$w$都是转$n$圈以后恢复的;如果某一组$z(t), w(t)$下$w$的值永远不会恢复,那么所有$z(t), w(t)$下都是如此。

因此,若在$z_0$处绕$n$圈之后$w$的值恢复,则称$z_0$是一个$n-1$\textbf{阶支点}。所有整数阶支点统称\textbf{代数支点};永远恢复不了的支点称为\textbf{超越支点}。

在实际判断支点的时候,比较方便的一种做法是取$t=\phi$,并作出一个可行的$w(\phi)$(只需要一个就够了!)使$f(w(\phi))=g(z_0 + \rho e^{\mathrm{i}\phi})$。有可能是支点的点只有诸$z_k$和$\infty$。在讨论诸$z_k$的时候我们取一个充分小并且保持不变的$\rho$,尝试构造一个$w(\phi)$使
\[
(z_k - z)^{i_1} \cdots (z_k - z)^{i_{k-1}} (z_k - z)^{i_{k+1}} \cdots (z_k - z)^{i_k} \rho^{i_k} e^{\mathrm{i}i_k \phi} \approx f(w(\phi))
\]
然后看满足$w(\phi+2\pi n) = w(\phi)$的最小$n$,$n-1$就是$z_k$的阶数($n=1$那这就不是支点)。
在讨论$\infty$的时候则构造一个$w(\phi)$使 \[
\rho^{i_1 + \cdots + i_n} e^{\mathrm{i}(i_1 + \cdots + i_n)\phi} \approx f(w(\phi))
\]
然后看满足$w(\phi+2\pi n) = w(\phi)$的最小$n$,$n-1$就是$z_k$的阶数($n=1$那这就不是支点)。

上面的这一套说法应该也能够推广到$g$是无穷级数的情况。(大概吧\ldots{}\ldots{})

\subsubsection{关于记号的注记}

为了避免冗长,我们形式地引入记号$f^{-1}$(虽然并不能求反函数)并规定
\[
P(f^{-1}(z_1), \ldots, f^{-1}(z_k)) \Leftrightarrow \exists w_1, \ldots w_k (z_1 = f(w_1) \land \ldots z_k = f(w_k) \land P(w_1, \ldots, w_k))
\] 因此,从复平面上也成立的 \[
e^{a+b} = e^a e^b
\] 就可以直接写出 \[
\ln z \equiv \exp^{-1} z, \; \ln (z_1 z_2) = \ln z_1 + \ln z_2,
\] 虽然对数函数并不是单值的。

这种记号是非常好用的。比如说在判断支点的时候就要处理下面形式的方程: \[
f(w(\phi);a) = \ldots
\] 然后我们就可以形式上写出 \[
w(\phi) = f^{-1}(\ldots;a) 
\] 使用一些像$\ln(z_1z_2)$展开的公式就可以得到 \[
w(\phi) = f^{-1}(\ldots;a) = F(h^{-1}(\phi), f^{-1}(a))
\]
之类的公式。$h^{-1}$也许有定义,也许没有,但是因为我们只需要\textbf{找到一个}符合条件的$w(\phi)$,我们只需要\textbf{随便找一个能称为$h^{-1}$的函数}就已经把$w(\phi)$构造出来了。举例:
\[
w(\phi) = \ln (a\rho e^{\mathrm{i}\phi}) = \ln \rho + \ln a + \phi \mathrm{i}
\]
当然这没有考虑到多值性,但是因为我们只需要\textbf{一个}$w(\phi)$,我们实际上已经构造出我们需要的东西了。上式的严格形式大概是
\[
\exists  z \in \mathrm{C} (a = e^z \land w(\phi) = \ln \rho + z + \phi \mathrm{i})
\]

\subsubsection{割线}

TODO:多个支点的情况 

\subsubsection{单值分支的选取}
为了方便起见将所有的割线都选为直线。设$z_0$是一个支点,则某个单值分支内,$z_0$的一个邻域内的自变量的相角将受到一定的限制:
\[
z = z_0 + \rho e^{\mathrm{i} \phi}, \quad \phi \in [\phi_0 + 2\pi k, \phi_0 + 2 \pi (k+1)) \backslash \{ \phi_1, \ldots, \phi_n \}
\]
被去掉的$\phi_1, \ldots, \phi_n$是为了避免$z$撞上某一条割线。只允许$\phi$转动$2\pi$是因为如果它转动超过了$2\pi$那就一定会撞上一条割线。

我们总是可以求出一个 \[
h_0: (\rho, \phi) \longrightarrow w, 
\]
使$f(h_0(\rho, \phi)) = g(z_0 + \rho e^{\mathrm{i}\phi})$。只要有\textbf{一个}这样的$h_0$,就可以获得\textbf{所有的}单值分支,因为只需要求出$z$满足
\[
z = z_0 + \rho e^{\mathrm{i} \phi}, \quad \phi \in [\phi_0 + 2\pi k, \phi_0 + 2 \pi (k+1)) \backslash \{ \phi_1, \ldots, \phi_n \}
\]
的形式就可以将$\rho, \phi$代入$h_0$中来得到对应的$w$,从而求出$z_0$的邻域内的单值分支值,然后延拓到整个复平面。
(但是为什么是所有的单值分支呢?)

在$h_0$已经给定了的情况下,只需要给定一个点$z_1$的值就可以确定一个单值分支。首先合理安排$z_0, \phi_0, \phi_1, \ldots, \phi_n$使$z_1$不在任何一条割线上并且在$z_0$的某个合适的邻域内。然后从下面的方程求解出$k$即可
\[
f(h_0(|z_1|, \arg z_1 + 2\pi k)) = g(z_1)
\]

TODO:上面的好像写错了……

但总之确定了围绕一个支点的各点的辐角就能够写出一个单值分支。

\subsubsection{涉及多值函数的积分}

要计算积分 \[
\int f(z) \mathrm{d}z
\]
如果有一个多值函数在它的某一支上面的导数就是$f$,那么$f$的一切原函数都是多值的。此时如果积分路径环绕支点就有可能产生比较复杂的情况。比较好的一种做法还是选取一个$z=z(t)$使得随着实数参量$t$的\textbf{连续变化}$z$可以正好扫过整个积分路径,同时选取一个在整条积分路径上都有定义的$f$的原函数的单值化$w(t)$,也就是要求
\[
\mathrm{d} w(t) = f(z(t)) \mathrm{d} z(t)
\] 那么我们就有 \[
\int_{z(t_1)}^{z(t_2)} f(z) \mathrm{d}z = w(t) \big |_{t_1}^{t_2}
\]
为了获得$w$,只需要取$\int f(z) \mathrm{d}z$的一个单值分支就可以。

举例: \[
\int \frac{1}{z} \mathrm{d}z = \ln z + C
\] 但是对数函数是多值的,那么实际做积分的时候就取 \[
w(t) = \ln \rho (t) + \phi (t) \mathrm{i}, \\
\int_{z_1}^{z_2} \frac{1}{z} \mathrm{d}z = (\ln \rho (t) + \phi (t) \mathrm{i}) \big |_{t_1}^{t_2}
\]

在大多数情况下,考虑到积分路径可以固定首尾而连续变形这一事实,我们总是可以将积分路径化成一个圆形,这时直接取$t=\phi$最为方便。

例如,计算一个环绕原点的$1/z$的环路积分: 
\[
\begin{aligned}
    \oint \frac{1}{z} \mathrm{d}z &= \oint_{|z|=1} \frac{1}{z} \mathrm{d}z \\
    &= (\ln 1 + \phi \mathrm{i}) \big |_0^{2\pi} = 2 \pi \mathrm{i}.
\end{aligned}
\]
我们的计算给出了正确的结果。注意,我们要求参数$t$——这里是$\phi$——做连续的变动,因此积分上下限是0和$2\pi$,虽然它们代表同一个点。正是这个连续性的要求让我们得以获得正确的结果。

\section{复变函数的积分}

\subsection{柯西定理与柯西公式等}

\textbf{柯西定理}
设$f$在闭单连通区域$D$上解析,则$f$围$D$的环路积分为零。从而,设$f$在闭复连通区域的所有境界线上取正方向的环路积分为零。

推论:
\begin{itemize}
    \item $f$沿闭复连通区域的外境界线逆时针的积分等于$f$沿各内境界线逆时针方向的积分之和;
    \item 闭区域内的积分,固定开头结尾连续变形积分路径不会改变积分值。
\end{itemize}

微积分基本定理在复变函数的情况下仍然成立。不过要注意有些函数的不定积分有可能是多值的,这个时候必须要取一个单值分支。

在计算具体的积分的时候有下面的定理:

\textbf{有界闭区域的柯西公式}
设$f$在闭区域$D$上解析,且$l$为$D$取正方向的全部境界线(在$D$复连通的时候包含好多条路径),$n=0, 1, 2, \ldots$,那么
\[
f^{(n)}(z) = \frac{n!}{2\pi \mathrm{i}} \oint_l \frac{f(\zeta)}{(\zeta - z)^{n+1}} \mathrm{d}\zeta, \; z \in D
\]

推论:
\begin{itemize}
    \item 区域上的复变函数的一阶导数的存在性就保证了任何高阶导数的存在性,并且对解析函数而言,边界上的值确定了内部的值
    \item 若函数在闭区域上解析,那么它的模在边界上取最大值(在区域内部取最大值的充要条件是函数为常函数)
    \item \textbf{刘维尔定理}
    如果$f$在全平面上解析且$z \rightarrow \infty$时有界,那么$f$是常数。因此,在闭平面上解析的复变函数只有常函数。例如,在全平面上解析的函数在无穷远处一定发散。
    \item 设$|f|$在圆周$|z-\zeta|=R$上的上界为$M$,则
    \[
      |f^{(n)}(z)| \leq \frac{n! M}{R^n}
    \]
\end{itemize}

在无界区域上也有类似的结论。

\textbf{无界区域的柯西公式}
设$f$在闭曲线$l$上以及其外的无界区域上解析,$z$是$l$外一点,那么
\[
f(z) = \frac{1}{2\pi \mathrm{i}} \oint_{l^-} \frac{f(\zeta)}{\zeta - z} \mathrm{d}\zeta + f(\infty  )
\] 且若$n=1, 2, \ldots$,有 \[
f^{(n)}(z) = \frac{n!}{2 \pi \mathrm{i}} \oint_{l^-} \frac{f(\zeta)}{(\zeta - z)^{n+1}}\mathrm{d}\zeta
\] 注意积分路径都是顺时针的。

\subsection{留数}

设$b$是$f$的一个孤立奇点,则$f$在$b$处的留数$\res (b)$定义为$f$在$b$处的洛朗级数的-1次幂项的系数。则我们有

\textbf{留数定理}
设$f$在闭曲线$l$包围的区域内除了有限个奇点$b_k, k=1, 2, \ldots m$以外是单值解析的,且在$l$上也是解析的,那么
\[
\oint_l f(z) \mathrm{d}z = 2 \pi \mathrm{i} \sum_{k=1}^m \res f(b_k).
\]

那么下面就需要讨论计算留数的方法。可以证明

\begin{itemize}
\item
  可去奇点的留数为零。
\item
  若$b$为$m$阶极点,那么 \[
  \res f(b) = \frac{1}{(m-1)!} \lim_{z \to b} \frac{\mathrm{d}^{m-1}}{\mathrm{d} z^{m-1}} \left( (z-b)^m f(z) \right)
  \]
\item
  本性奇点的留数一般需要直接计算洛朗级数
\end{itemize}

今后,常使用 
\[
\sum_D \mathrm{res} f(z)
\] 
表示区域$D$内$f$的所有奇点的留数之和。

\subsection{与实函数定积分的联系}

\subsubsection{\texorpdfstring{$(0, 2\pi)$上的三角函数积分}{(0, 2\textbackslash{}pi)上的三角函数积分}}

考虑积分 \[
\int_0^{2\pi} R(\cos x, \sin x) \mathrm{d}x
\] 设$z=e^{\mathrm{i}x}$,则 \[
R(\cos x, \sin x) \mathrm{d}x = R \left(\frac{z + z^{-1}}{2}, \frac{z - z^{-1}}{2 \mathrm{i}}\right) \frac{\mathrm{d} z}{\mathrm{i}z} \equiv F(z) \mathrm{d}z,
\] 于是 \[
\int_0^{2\pi} R(\cos x, \sin x) \mathrm{d}x = 2 \pi \mathrm{i} \sum_{|z|=1} \mathrm{res} F(z)
\] 其中 \[
F(z) = \frac{1}{\mathrm{i}z} R \left(\frac{z + z^{-1}}{2}, \frac{z - z^{-1}}{2 \mathrm{i}}\right).
\]

\hypertarget{ux6cbfux6574ux6761ux5b9eux6570ux8f74ux7684ux5b9aux79efux5206}{%
\subsubsection{沿整条实数轴的定积分}\label{ux6cbfux6574ux6761ux5b9eux6570ux8f74ux7684ux5b9aux79efux5206}}

\[
\mathrm{P} \int_{-\infty}^{+\infty} f(x) \mathrm{d}x = 2 \pi  \mathrm{i} \sum_{\text{upper half-plane}} \mathrm{res} f(z) + \pi \mathrm{i} \sum_{\text{real axis}} \mathrm{res} f(z)
\]
使用这个公式,对在上半平面处处解析且在$z \to \infty$时一致趋于0的函数$f$,以及$\alpha \in \reals$,有\textbf{希尔伯特变换}
\[
\begin{aligned}
    \mathrm{Re} f(\alpha) &= \frac{1}{\pi} \mathrm{P} \int_{-\infty}^{+\infty} \frac{\mathrm{Im} f(x)}{x - \alpha} \mathrm{d} x \\
    \mathrm{Im} f(\alpha) &= - \frac{1}{\pi} \mathrm{P} \int_{-\infty}^{+\infty} \frac{\mathrm{Re} f(x)}{x - \alpha} \mathrm{d} x 
\end{aligned}
\]

\hypertarget{ux65e0ux7a77ux5c0fux5206ux6bcdux865aux90e8}{%
\subsubsection{无穷小分母虚部}\label{ux65e0ux7a77ux5c0fux5206ux6bcdux865aux90e8}}

设$f$在实轴上没有奇点。考虑积分 \[
\int_a^b \frac{f(x)}{x-x_0 - \mathrm{i}\epsilon} \mathrm{d}x
\]
其中$\epsilon$是一个很小的正数。需要计算的是这个积分在$\epsilon\to 0$时的极限。我们可以让积分路径在$x_0 - \mathrm{i}\epsilon$附近沿着一个小半圆转一下,然后让小半圆的半径趋于零。接着我们交换两个极限的顺序,这样我们要计算的极限就是:极点位于实轴上,而积分路径转过一个半径趋于零的小半圆。从而可得
\[
\int_a^b \frac{f(x)}{x-x_0 - \mathrm{i}\epsilon} \mathrm{d}x = \mathrm{P} \int_a^b \frac{f(x)}{x-x_0} \mathrm{d}x + \mathrm{i} \pi f(x_0)
\] 这个结论可以推广为 \[
\int_a^b \frac{f(x)}{x-x_0 \mp \mathrm{i}\epsilon} \mathrm{d}x = \mathrm{P} \int_a^b \frac{f(x)}{x-x_0} \mathrm{d}x \pm \mathrm{i} \pi f(x_0)
\] 考虑到$f$的任意性,上式又可以写成 \[
\frac{1}{x - x_0 \mp \mathrm{i}\epsilon} = \mathrm{P} \frac{1}{x - x_0} \pm \mathrm{i} \pi \delta(x - x_0)
\]

需要注意的是积分主值是包括$\pi \ii$乘以实轴上的极点的留数这一项的;分母上$+\ii \epsilon$相当于将实轴上的极点全部打到了下半平面,从而不再需要考虑它们,而分母上$- \ii \epsilon$相当于将实轴上的极点全部打到了上半平面,它们对积分的贡献就是$2\pi \ii$乘以留数而不是$\ii \pi$。
表面上看我们似乎直接丢弃了下半平面的数据,从而理论上应该等价的$+ \ii \epsilon$和$- \ii \epsilon$导致的结果不同,实则不然——如果无穷远点是解析的,那么全体有限极点的留数之和一定是零。
因此,当我们对上半平面的极点积分时的确已经考虑了下半平面提供的信息。

\subsubsection{涉及多值函数的定积分}


设$Q$在全平面上除了有限个奇点以外都是解析的,且当$z \to 0$或$z \to \infty$时$z^\alpha Q(x)$一致趋于零,则
\[
\int_0^{+\infty} x^{\alpha - 1} Q(x)
\] TODO

\hypertarget{ux65e0ux7a77ux7ea7ux6570}{%
\section{无穷级数}\label{ux65e0ux7a77ux7ea7ux6570}}

在复变函数的情况下,柯西收敛准则也是成立的。

无穷级数基本性质:
\begin{itemize}
    \item 绝对收敛的级数收敛,并且其和与各项排列顺序无关
    \item 绝对收敛的级数的乘积也绝对收敛
    \item 一个函数级数的各项在一个区域内连续,而级数在这个区域内一致收敛,则级数的和在该区域内连续
    \item 各项均连续的一致收敛级数可以逐项积分 
    \item 若$|u_k(z)| \leq a_k$而$\sum_k a_k$在$D$上收敛,那么$\sum_k u_k(z)$在$D$上一致收敛且绝对收敛
\end{itemize}

\hypertarget{ux6cf0ux52d2ux7ea7ux6570ux6536ux655bux5706}{%
\subsection{泰勒级数、收敛圆}\label{ux6cf0ux52d2ux7ea7ux6570ux6536ux655bux5706}}

\hypertarget{ux6d1bux6717ux7ea7ux6570}{%
\subsection{洛朗级数}\label{ux6d1bux6717ux7ea7ux6570}}

\textbf{洛朗定理}
设$f$在环形区域$R_2 < |z-b| < R_1$内单值解析,则$f$可以在这个区域内展开为绝对收敛且一致收敛的级数
\[
f(z) = \sum_{k=-\infty}^\infty a_k (z-b)^k,
\] 其中 \[
a_k = \frac{1}{2\pi \mathrm{i}} \oint \frac{f(\zeta)}{(\zeta - b)^{k+1}}\mathrm{d}\zeta
\] 因此这个展开是唯一的。

如果洛朗级数中有负幂,那么$f$在$|z-b|\leq R_2$上一定有奇点,否则洛朗级数就是泰勒级数了,也就不应该出现负幂。但是这个奇点却不一定就是$b$,这是因为$f$并不一定在$|z-b|\leq R_2$内解析,所以也不能够将洛朗级数延拓到$b$附近。当然,如果我们能够将$R_2$取为零,那么$b$就是$f$的一个奇点。

\hypertarget{ux5b64ux7acbux5947ux70b9ux4e0eux6d1bux6717ux7ea7ux6570ux7684ux8054ux7cfb}{%
\subsection{孤立奇点与洛朗级数的联系}\label{ux5b64ux7acbux5947ux70b9ux4e0eux6d1bux6717ux7ea7ux6570ux7684ux8054ux7cfb}}

设$b$是$f$的一个非无穷远的孤立奇点,我们在这个奇点附近对它做洛朗展开,
\[
f(z) = \sum_{k=-\infty}^\infty a_k (z - b)^k
\]
称这个级数的正幂部分($k\geq 0$的部分)为解析部分,复幂部分($k<0$的部分)为主要部分。(两个级数在$b$以外都是解析的!)那么,奇点$b$的性质、$f$在$z\rightarrow b$时的性质和洛朗级数的性质有着很重要的联系。
\#\#\# 可去奇点
$b$是一个可去奇点,当且仅当,级数的主要部分不存在,即$k \geq 0$的时候才可能有$a_k \neq 0$,当且仅当,$\lim_{z\rightarrow b} f(z)$为有限复数。

我们规定 \[
F(z) =
\begin{cases}
    f(z) & \text{when $z \neq b$} \\
    \lim_{z\rightarrow b} f(z) & \text{when $z=b$}
\end{cases}
\] 则$F$在$b$处可导,此时 \[
\sum_{k=-\infty}^\infty a_k (z - b)^k = \sum_{k=0}^\infty a_k (z - b)^k
\] 就是$F$在$b$处的泰勒级数。

\hypertarget{ux6781ux70b9}{%
\subsubsection{极点}\label{ux6781ux70b9}}

$b$是一个$m$阶极点,当且仅当,只有$k \geq -m$时才可能有$a_k \neq 0$,当且仅当,$\lim_{z\rightarrow b}(z-b)^m f(z)$为有限复数。

事实上,$b$是一个$m$阶极点,当且仅当 \[
\lim_{z\rightarrow b} (z-b)^n f(z) = 
\begin{cases}
    \infty & \text{when $n<m$} \\
    a_{-m} \neq 0 & \text{when $n=m$} \\
    0 & \text{when $n>m$}
\end{cases}
\] 因此$b$是一个极点当且仅当$\lim_{z\rightarrow b}f(z) = \infty$。

\hypertarget{ux672cux6027ux5947ux70b9}{%
\subsubsection{本性奇点}\label{ux672cux6027ux5947ux70b9}}

$b$是一个本性奇点,当且仅当级数有无穷个负幂项(也即,使$a_k$非零的$k$无下界),当且仅当$\lim_{z\rightarrow b} f(z)$不存在。

注意:在考虑了无穷远点的情况下,$\lim_{z\rightarrow b} f(z)$不存在不包括$f(z)$趋于$\infty$的情况。$\lim_{z\rightarrow b} f(z)$不存在通常意味着从不同方向接近$b$会得到不同的极限值,或者得到正无穷和负无穷等。

\hypertarget{ux89e3ux6790ux5ef6ux62d3}{%
\section{解析延拓}\label{ux89e3ux6790ux5ef6ux62d3}}

\textbf{解析函数的唯一性定理}
设$f_1$和$f_2$是区域$D$内的两个解析函数,$D$内的点列$\{z_k\}$有至少一个极限点,且在整个$\{z_k\}$上$f_1(z)=f_2(z)$,则在整个$D$上都有$f_1(z)=f_2(z)$。

推论: 
\begin{itemize}
    \item 若两个函数在某一区域中某一点的邻域或者某一曲线段上相等,那么它们在整个区域上都相等;
    \item 若两个函数在某一区域上解析,并且在某一点上的函数值和各阶导数都相同,那么它们在整个区域上都相等
\end{itemize}

设函数$f$在闭平面上有一系列奇点$z_i$,我们在$z_0$处对$f$做泰勒展开,得到的级数的收敛半径就是$z_0$到离它最近的奇点的距离。考虑到任何一个不是常函数的复变函数都必定有不可去的奇点,我们得出结论:不可能使用单独一个泰勒级数覆盖整个闭平面上的非常函数的复变函数。(覆盖全平面还是可以做到的,比如指数函数和三角函数)

\hypertarget{ux79efux5206ux53d8ux6362}{%
\section{积分变换}\label{ux79efux5206ux53d8ux6362}}

\hypertarget{ux5085ux91ccux53f6ux53d8ux6362}{%
\subsection{傅里叶变换}\label{ux5085ux91ccux53f6ux53d8ux6362}}


\end{document}