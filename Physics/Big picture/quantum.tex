\documentclass[UTF8, a4paper]{ctexart}

\usepackage{geometry}
\usepackage{titling}
\usepackage{titlesec}
\usepackage{paralist}
\usepackage{footnote}
\usepackage{enumerate}
\usepackage{amsmath, amssymb, amsthm}
\usepackage{cite}
\usepackage{graphicx}
\usepackage{subfigure}
\usepackage{physics}
\usepackage{tikz}
\usepackage[colorlinks, linkcolor=black, anchorcolor=black, citecolor=black]{hyperref}

\geometry{left=3.18cm,right=3.18cm,top=2.54cm,bottom=2.54cm}
\titlespacing{\paragraph}{0pt}{1pt}{10pt}[20pt]
\setlength{\droptitle}{-5em}
\preauthor{\vspace{-10pt}\begin{center}}
\postauthor{\par\end{center}}

\DeclareMathOperator{\timeorder}{T}
\DeclareMathOperator{\diag}{diag}
\newcommand*{\ii}{\mathrm{i}}
\newcommand*{\ee}{\mathrm{e}}
\newcommand*{\diff}{\mathop{}\!\mathrm{d}}
\newcommand*{\st}{\quad \text{s.t.} \quad}
\newcommand*{\const}{\mathrm{const}}
\newcommand*{\comment}{\paragraph{注记}}
\newcommand*{\scheq}{Schr\"odinger's Equation}
\newcommand*{\reals}{\mathbb{R}}

\newenvironment{bigcase}{\left\{\quad\begin{aligned}}{\end{aligned}\right.}

\title{量子物理基本概念}
\author{wujinq}

\begin{document}

\maketitle

% 总之我在这几篇文章中把“可观察量”一词玩坏了,它可以表示厄米算符也可以表示其期望……
% 类似的还有“多粒子态”……
% TODO:LSZ约化公式

本文将尽可能一般地讨论量子力学(包括量子场论)的基本理论框架,但应该指出,大部分物理上有意义的量子理论都是高度受限的。
我们通常在一个性质良好的(如可定向,可以定义各种直觉上合理的结构)流形上工作,或者在这个流形的格点化结果上工作,并且要求整个系统的希尔伯特空间可以写成每一点的希尔伯特空间的直积。
我们还要求物理定律都是局域的。实际上,刚才的“每一点可以定义希尔伯特空间”也是一个局域性条件。局域性的重要之处在于,如果允许非常非局域的东西存在,维数就不是良定义的了,这和直觉不符。
作用在系统上的操作或许可以有局域版本(如$U(1)$等,此时这会自然地诱导出一个规范场理论),或许虽然要求全局的操作但是伴生有一些局域的操作,从而某种意义上可以在不同的流形之间移植(如空间旋转),或许本质上是全局的,从而有可能不能对任何流形都有定义(如拓扑变换;局域性意味着我们可以把局部的操作“粘起来”来构造一个全局的操作,从而总是可以在不同流形之间移植的)。
具有一定局域性的操作一般称为对称性操作,它们有局域定义这件事意味着它们的代数结构在希尔伯特空间上的表示是物理意义明确的算符(如动量、角动量);全局性的操作,如拓扑变换,虽然也有对应的代数结构,但通常它们对应的算符并没有太多意义。

\section{抽象代数}

事实证明,在量子理论中抽象代数具有非常重要的作用,因此我们将专门用一节介绍它。

\subsection{指标升降}

符号约定:$\vb*{A}^2$代表$\vb*{A}$的模长平方,$A^2$则表示分量。

\subsection{算符和态}

\[
    \comm{\hat{A}}{\hat{B}}^\dagger = \comm{\hat{B}}{\hat{A}},
\]
因此两个算符对易当且仅当它们的共轭转置对易。

% TODO:明确地点出动力学变量这个概念
% 自由度、模式、等等

% TODO:分析三种绘景下的态
% 两个态表示了同样的物理状态,当且仅当,
% $\ket{\psi}$和$\hat{A}$组成的系统和$\hat{Q}\ket{\psi}$和$\hat{Q} \hat{A} \hat{Q}^\dagger$组成的系统等价,其中$\hat{Q}$是一个幺正算符;反之,如果两个长度等价的向量描述等价的系统,
% TODO: 设算符$\hat{A}$是CSCO,且它在幺正变换$\hat{P}$下不变,那么对任何一个本征值$A_i$,有一个单位复数,使得$\hat{P} \ket{A_i} = c \ket{A_i}$.
% TODO:虽然描写一个态空间可以需要不止一个算符(或者说这个空间的CSCO的大小不为1),但往往可以将这些CSCO拼凑成一个:
% \hat{A} \ket{a_1 a_2 \cdots} = \pmqty{a_1 & a_2 & \cdots} \ket{a_1 a_2 \cdots}
% 只要推导中不涉及本征值的乘除,这样做就没有任何问题。
% 因此下文中将常常这么写。
% 谱结构和对易关系之间有什么联系?

\textbf{表象}指的是态空间的一组正交完备基。由于通常这样一组基是某个CSCO$\hat{M}$的本征态,我们通常使用对应的$CSCO$来标记表象。例如,我们有坐标表象、动量表象,等等。
表象变换公式
\begin{equation}
    \braket{A_i}{\psi} = \sum_j \braket{A_i}{B_j} \braket{B_j}{\psi}
\end{equation}
是基的完备性的推论。

变换
\[
    \braket{A_i}{\psi} \longrightarrow \mel{A_i}{\hat{B}}{\psi}
\]
称为算符$\hat{B}$在$A$表象下的表示。显然,$\hat{A}$在$A$表象下的表示就是
\[
    \braket{A_i}{\psi} \longrightarrow A_i \braket{A_i}{\psi}.
\]

在离散谱的情况下,归一化条件相当简单:
\[
    \braket{A_i}{A_j} = \delta_{ij}
\]
在连续谱的情况下,需要使用积分代替求和,使用$\delta$函数代替$\delta$符号。
多分量算符的本征值有可能不是按照$\reals^n$的方式分布的,而是分布在一个弯曲的空间上(例如,分布在一个球面上)。此时通常需要使用类似于
\[
    \int \dd[n]{x} \delta(f(x))
\]
这样的测度,其中$f(x)$为描述弯曲空间的方程。其结果是,即使两个表象中的态矢量能够做到一一对应,由于使用的测度不同,它们仍然可以差一个模长不为1的系数。
换而言之,相同的态在不同的表象中会以不同的内积被归一化。

\subsection{李群和李代数,以及它们的表示}
% TODO:形如$\exp(\phi_1 G_1 + \phi_2 G_2 + \ldots)$的映射是不是一定可以写成$\exp (\phi_1' G_1) \exp (\phi_2' G_2) \ldots$?
在讨论对称性和守恒量的联系的时候

李群$U(t,t_0)$对应的无穷小生成元$H(t)$定义为
\begin{equation}
    H(t) = \lim_{\Delta t \to 0} \frac{U(t+\Delta t, t)}{\Delta t}
\end{equation}
使用编时算符$\timeorder$可以写出形式解
\begin{eqnarray}
    U()
\end{eqnarray}

如果$U(a+b)=U(a)U(b)$则生成元是常量。

有必要分析一下升降算符的东西:作用在一个本征态上得到另一个本征态??
可以看成一种平移:
\[
    U(\epsilon) \ket{x} = \ket{x + \epsilon}
\]

李代数中的$XY$和$YX$都未必是李代数的元素,但是$[X,Y]$一定是。

需要注意的是即使一个群的生成元是
\[
    \det \ee^{A} = \ee^{\trace A}
\]
只要知道了李括号就完全确定了整个李代数的结构。

生成元会变的情况??编时算符。

特别的,如果一个变换不改变哈密顿量,或者说“不改变物理规律”,且这个变换的生成元不显含时间,那么其生成元就是守恒量。(也就是说哈密顿量是这个群的卡西米尔算符??)

不可约表示中的卡西米尔算符的表示一定是恒等矩阵的倍数。例如,$\laplacian$是空间平移群的卡西米尔算符,而空间平移群在形如$A\exp (\ii \vb*{k} \cdot \vb*{r})$这样的平面波组成的线性空间上的表示是不可约表示(它自己就是一个不变空间,没有更小的不变子空间),那么$\laplacian u = - k^2 u$,可见确实是恒等变换的倍数。

幺正的李群按照$\exp(\theta J)$的形式得到的生成元是反厄米的,按照$\exp(\ii \theta J)$的形式得到的生成元是厄米的。

交换左右手坐标系的坐标变换行列式是负的,否则是正的。

\subsection{常用公式罗列}

使用算符代数的时候需要特别小心,因为不对易性很容易让我们习以为常的公式失效。


我们有
\[
    \dv{t} \ee^{t A} = A \ee^{t A} = \ee^{t A} A
\]
然而,
\[
    \dv{t} \ee^{A(t)} = \dv{A}{t} \ee^{A(t)} = \ee^{A(t)} \dv{A}{t}
\]
并不一般成立。

回顾经典物理,我们会发现对任何一种系统我们都尝试使用一系列固定的物理量描述它,例如一个粒子有位置、动量,一个场有各点的场量,等等。
有时也可以使用另一些物理量描述它,例如我们可以在速度和动量之间切换,可以使用不同的坐标系,等等。
因此,虽然实际计算中常常使用由一系列物理量的值组成的列表描述物理系统(举例:“粒子质量多少多少、位于$x$坐标多少多少、$y$坐标多少多少的位置”),
但观念上我们使用了两种对象:其一是系统的状态,它是某种流形上的一个点(例如在经典哈密顿力学中它是辛流形上的点),
其二是物理量,它是从这个流形到实数、复数、矢量、张量等“量”的映射。

在经典体系中“态”能够做的运算无非是从一个态转移到另一个态;态和态之间是完全孤立的。
然而,无论是理论上的推广(如将经典力学看成某个波动方程的程函方程)还是实验上的发现(如双缝干涉实验)都表明,这并不是完美描述自然界的正确方式。
实际上,态是可以像矢量一样叠加的(例如,干涉条纹意味着电子在空间中的分布可以看成某种场,
这种场满足叠加原理,那么电子在空间中的分布就是$\delta$函数为基底张成的向量空间中的元素)。
因此,一个\textbf{量子系统}指的是其状态可以使用某个希尔伯特空间$\mathcal{H}$中的向量$\ket{\psi(t)}$来描述、并且可以做叠加、数乘等运算的系统。

我们还需要一个额外的假设:一个态矢量数乘上一个复数得到的态矢量和原态矢量代表了同一个态。因此,我们将使用归一化的态矢量$\ket{\psi}$,并简称它们为“态”。
长度为零的态矢量不能归一化,我们认为它是非物理的,仅仅用于保证正确的代数结构,而不起多大作用。
同时我们也假定态矢量如果随时间发生演化,那么它一直是归一化的;进一步,任何作用于态之上的可逆变换都应该保持态的归一化。
或者说,任何作用于态上的可逆变换都应该是\textbf{幺正}的。

下一个问题是,我们怎样“诊断”或者说“读取”这系统的状态,也就是说怎样构造量子体系中的物理量。
实验上的观察(如双缝干涉实验中如果输入电子束密度足够低,是能够捕捉到单个电子的,但是其位置不固定)表明,
一个态$\ket{\psi}$并不对应着一个固定的物理量取值(刚才的例子表明一个态通常并不对应一个固定的位置)。
但是注意到一旦物理量的定义给定了(比如,假定我们接下来要测定位置),的确有\textbf{一些}态能够毫无疑义地确定物理量的取值
(例如,$\delta(x-x_0)$当然就对应一个位于$x_0$处的尖峰)。
因此量子物理中的物理量应该是这种“能够确定物理量取值”的态连同对应的物理量打包而成的结构。
至于那些不能明确确定物理量取值的态,可以把它写成能够明确确定物理量取值的态的线性组合来判断它对应哪些可能的物理量取值,这些物理量取值占比多少。
什么样的结构可以用来做这件事?一个自然的想法是\textbf{算符}:设诸$A_i$为可能的物理量$A$的取值,$\ket{A_i}$为这些取值对应的一组非零态,定义
\begin{equation}
    \hat{A} = \sum_i A_i \dyad{A_i}
\end{equation}
为该物理量对应的算符%
\footnote{一个细节:如果同一个$A_i$对应多个可能的$\ket{A_i}$,则容易看出这个$A_i$对应的所有态矢量对应一个向量空间。
此时需要写出这个向量空间的一组基矢量$\ket*{A_i^{(1)}}, \ket*{A_i^{(2)}}, \ldots$,然后用
\[
    A_i \left(\ket{A_i^{(1)}} + \ket{A_i^{(2)}} + \cdots\right)
\]
代替$A_i \ket{A_i}$。}
。这样一来,“物理量的取值能够确定”的态就是算符$\hat{A}$的本征态,于是我们就可以使用线性代数来处理有关的问题。
物理量随着时间演化就意味着我们有
\begin{equation}
    \hat{A}(t) = \sum_i A_i(t) \dyad{A_i(t)},
\end{equation}
也就是说每一个时间点都对应一个算符。
由于量子物理中大部分有意义的物理量都对应算符,接下来我们将经常混用物理量和算符这两个词%
\footnote{当然的确有一些物理量和算符关系不大,比如质量等,但因为它们总是被当成常数处理因此无大碍。}
。

需要注意的是如果一个态和另一个态的内积$\braket{\psi}{\phi}$不为零,那
%TODO:正交性的意义
因此我们之后均认定对应不同$A_i$的$\ket{A_i}$彼此正交,从而可以毫无顾虑地使用bra-ket记号。
另一方面有意义的物理量值都是实数(有时引入复数单纯是为了方便计算,如表示相位,等等),这就意味着物理量对应的算符都是\textbf{厄米算符}。

\subsection{对系统的等价描述}

设两个希尔伯特空间$\mathcal{H}$和$\mathcal{H}'$,它们使用一个可逆算符$\hat{A}$相关联,当然$\hat{A}$是幺正的。也就是说
\[
    \ket{\psi'} = \hat{A} \ket{\psi}, \quad \ket{\psi} \in \mathcal{H},  \ket{\psi}' \in \mathcal{H}'
\]
$\hat{A}$是一个同构。容易看出,若$\hat{O}$是$\mathcal{H}$中的一个算符,
那么
\begin{equation}
    \hat{O}' = \hat{A} \hat{O} \hat{A}^{-1} = \hat{A} \hat{O} \hat{A}^\dagger
\end{equation}
就是对应的$\mathcal{H}'$中保持代数结构不变的算符,这是下面几个式子的结果:
\[
    \begin{split}
        \ket{\psi'} = \hat{A} \ket{\psi}, \quad \ket{\phi'} = \hat{A} \ket{\phi}, \\
        \ket{\phi} = \hat{O} \ket{\psi}, \quad \ket{\phi'} = \hat{O}' \ket{\psi'}
    \end{split}
\]
一种常见的情况是,$\mathcal{H}$与$\mathcal{H}'$实际上是同一个空间,
则$\hat{O}$在变换$\hat{A}$下不变的充要条件是$\hat{O}'=\hat{O}$,也就是说$\hat{O}$与$\hat{A}$对易。
进一步,如果算符$\hat{O}$在一个李群作用下不变,那么它和每个群元都对易,这又等价于它和这个李群的所有生成元都对易。

上面我们讨论了对希尔伯特空间做一个变换会导致其上的算符做对应的变换。
现在我们讨论反过来的问题:如果两个算符的代数结构彼此对应,那么它们作用的希尔伯特空间之间会有什么样的关系。

% TODO:写串词
设有态矢量$\ket{\psi}$、算符$\hat{O}$,以及$\ket{\psi'}$和$\hat{O}'$,
若$\hat{O}$和$\hat{O}'$的谱结构相同(不变子空间同构,对应的本征值相同),且两个态矢量中含有的可观察量的各本征态的占比一致
则认为两系统等价,因为它们的代数结构不可区分。这时可以证明
\begin{equation}
    \mel{\psi}{A}{\psi} = \mel{\psi'}{A'}{\psi'}
\end{equation}
实际上,像这样的等价系统能够且只能够使用下面的方式产生:
\begin{equation}
    A' = U A U^\dagger, \quad \ket{\psi'} = U \ket{\psi}
\end{equation}
其中$U$为酉算符。要求$U$是酉算符是为了确保变换之后的$A'$的本征态的正交性,从而确保它确实是可观察量。
(由此也可以看出,要求使用复希尔伯特空间来描述系统而又一定要求可观察量的取值为实数实际上是很强的条件)

\subsubsection{从李群到李代数}

本文中我们将不对李群的流形结构进行正式的分析,而仅仅满足于使用一定的群参数把一个李群完整地表示出来。
一个李群中的成员可以一般地写成
\begin{equation}
    g = \exp(\ii \theta_i \sigma_i) \equiv \exp (\ii \theta^i \sigma_i) = \exp (\ii \vb*{\theta} \vb*{\sigma}),
    \label{eq:lie-group-element}
\end{equation}
其中$\theta_i$指的是群参数,而$\sigma_i$指的是生成元。
通常要求群参数为实数。
$\ii$是一个无关紧要的系数,加上它和不加上它唯一的区别就是$\sigma$需不需要乘上一个$\ii$。
为了方便,常常将诸$\theta$排成行向量,$\sigma$排成列向量。由于没有度规,无需区分上下指标。
对应的,设$\theta$是一个群参数,对应的生成元为
\begin{equation}
    \sigma = \frac{1}{\ii} \dv{g}{\theta}.
\end{equation}
需注意\eqref{eq:lie-group-element}假定了
\[
    g(\theta_1) g(\theta_2) = g(\theta_1 + \theta_2),
\]
这又等价于,无论$\theta$取什么值,$g$对$\theta$求导都会得到完全相同的结果。
在大多数情况下可以不失一般性地要求这个性质成立,因为群参数到底是什么并不重要
——我们总是可以巧妙地定义$\theta$使得$g$对$\theta$求导的结果与$\theta$无关%
\footnote{这是来自常微分方程的基本结论:设$X$是一个生成元,那么必定可以找到李群的一个单参数子群$c(t)$,使得
\[
    \dv{t} c(t) = c(t) \cdot X,
\]
从而可以定义指数映射。这是解析映射,因此可以使用诸如求导等运算。},
% 但是真的如此吗?时间演化一定构成李群吗?
% 一种可能的质疑是:在球面上随意画一条闭合轨迹,它显然描述了起点位于球心,终点位于球上面的矢量的一个连续变换,
% 然而它却不能使用$\exp (\alpha G)$的形式表示出来。
% 但这个质疑本身不成立,因为通常的李群总是可以作用在线性空间上的,然而上述变换显然没有线性性。
% 感觉还是很奇怪。
但是有一个重要的例外:时间演化。
我们关注的是“正常人眼中的时间”,而不能随意定义时间流逝的速率,
因此并没有什么能够保证不同$t$处时间演化算符对$t$求导的结果都是$t=0$(也就是恒等映射附近)时间演化算符对$t$求导的结果。
记$U(t, t_0)$为从$t_0$演化到$t$的算符,也即
\[
    U(t, t_0) U(t_0) = U(t),
\]
由于$t$不再能够任意选取,我们不能够写出\eqref{eq:lie-group-element}这样的指数映射,但是可以证明,一定存在一个$H(t)$使得
\begin{equation}
    U(t, t_0) = T \exp \left( \int_{t_0}^t \dd{t} H(t) \right).
    \label{eq:time-dependent-lie-group}
\end{equation}
这里我们略去了\autoref{sec:time-evolution}中的公式中的因子$- \ii /\hbar$,不过这无关紧要。$T$为编时算符。
在不同时刻的$H(t)$彼此对易的情况下可以把$T$去掉,因为此时重排各算符顺序不会产生任何影响。

\eqref{eq:lie-group-element}和\eqref{eq:time-dependent-lie-group}的区别体现在很多地方。
\eqref{eq:lie-group-element}意味着
\[
    g^{-1}(\theta) = g(-\theta),
\]
或者说
\[
    \left( \exp(\theta \sigma) \right)^{-1} = \exp(- \theta \sigma),
\]
但是在不同时刻的$H(t)$彼此不对易时,
\[
    T \exp(\int \dd{t} H(t))^{-1} \neq T \exp(- \int \dd{t} H(t)).
\]
相应的,
\[
    \dv{t} \left(T \exp(\int \dd{t} H(t))^{-1}\right) \neq -H.
\]
这就是\autoref{sec:time-evolution}中做绘景变换时不同绘景下的哈密顿算符不相等的根本原因。

李代数是李群在单位元附近的切空间,也就是说,是$g$在$\theta=0$附近沿着任意方向对$\theta$求导之后得到的结果组成的代数。
接下来我们将讨论\eqref{eq:lie-group-element}的李群,因为“不同点处求导结果不同”基本上只会在处理时间演化时用到,
而此时只有一个生成元(就是哈密顿量),没有必要讨论李代数。
由于李代数的封闭性,设$g_1, g_2, \ldots$是一组相互独立的生成元,它们中任意两个的李括号$\comm*{g_1}{g_2}$一定也是一个生成元,
这意味着它可以使用$g_1, g_2, \ldots$线性表示。
从而我们有
\begin{equation}
    \comm*{g_i}{g_j} = f_{ij}^k g_k.
    \label{eq:structure-of-lie-algebra}
\end{equation}
如果我们只讨论抽象的李代数的性质而不考虑它作用在某些对象上产生的结果,那么\eqref{eq:structure-of-lie-algebra}就完全刻画了一个李代数的结构。
因此,称$f_{ij}^k$为\textbf{结构常数}。

\subsubsection{李代数的具体计算}

% TODO:把前面用到这一节的内容的部分写得更加简洁一些
若
\[
    \comm*{\hat{q}}{\hat{p}} = c,
\]
则
\[
    \comm*{\hat{q}}{\hat{p}^n} = n c \hat{p}^{n-1}.
\]

\subsubsection{表示论}\label{sec:rep-th}

接下来需要讨论李群和李代数的表示。
通常考虑两种表示,其一是李群和李代数在向量空间上的作用,
也就是说,我们在李群、李代数和向量空间上的算符组成的群(以算符的复合为乘法)之间建立一个同态,
一旦建立起这个同态,我们实际上就得到了李群或李代数的一个表示。
比较方便的做法是,先讨论李代数在特定向量空间上的表示,然后使用指数映射获得对应的李群的表示。
第二种表示是,李群和李代数在向量空间上的算符构成的向量空间上的作用。
这种表示和第一种表示是紧密相关的。
设李群$G$在向量空间$V$上的表示为$G_V$,则$G_V \subset GL(V)$。这就自然地诱导出了李群在$GL(V)$上的表示。
算符$\hat{B} \in GL(V)$建立起了这样的关系:
\[
    \phi = \hat{B} \psi,
\]
现在我们把$\hat{A} \in G_V$作用在$\phi$和$\psi$上面,就得到
\[
    \phi' = \hat{A} \phi, \quad \psi' = \hat{A} \psi,
\]
如果我们还是希望在$\phi'$和$\psi'$之间建立关系
\[
    \phi' = \hat{B}' \psi',
\]
应该怎么选取$\hat{B}$?
考虑到$\phi$和$\psi$的任意性,容易看出,
\[
    \hat{B}' = \hat{A} \hat{B} \hat{A}^{-1}.
\]
我们没有规定$\hat{B}$是什么——它是完全任意选取的。这样一来,$G_V$中的每一个元素$\hat{A}$都对应到下面的映射:
\begin{equation}
    \hat{B} \longrightarrow \hat{A} \hat{B} \hat{A}^{-1},
    \label{eq:group-action-on-operators}
\end{equation}
\eqref{eq:group-action-on-operators}是一个从$GL(V)$到$GL(V)$的映射,也就是满足封闭性。
请注意该映射是$GL(GL(V))$的成员,而不是$GL(V)$的成员——它作用在$V$上的算符上而不是$V$中的向量上。
因此,我们通常只讨论简单的向量空间上的群表示,因为这些向量空间上的算符组成的向量空间上的群表示可以使用前者按照\eqref{eq:group-action-on-operators}写出。
另外注意,不同的$\hat{A}$可能对应着同一个\eqref{eq:group-action-on-operators}型的从算符到算符的映射。
这一点在处理旋转群时体现得很明显。

李群和李代数通常被作用在几类向量空间上。
首先是有有限个分量的向量空间。李群在其上的作用形如
\[
    v \longrightarrow v', \quad (v')^\mu = R_{\nu}^\mu (\Lambda) v^\nu.
\]
其中$\Lambda$指抽象的李群。
在有限维向量空间$V$上的表示可能有不变子空间,也就是说,存在$V$的一个子空间$V'$,使得李群中的任何一个成员作用在$v \in V'$上之后得到的结果都还是在$V'$中。当然,$V$以及$\{0\}$一定是不变子空间。
如果一个表示有不是这两个空间的不变子空间,那么这就是一个\textbf{可约表示},反之则为\textbf{不可约表示}。
可以证明,任何一个可约表示都可以写成一系列不可约表示的直和。因此对有限维表示而言,只需要讨论不可约表示就可以了,因为可约表示可以使用不可约表示组装出来。
现在讨论不可约有限维表示。
首先可以证明,任何李群的生成元至少有一个(当然也可以有很多个)可以相似变换为对角矩阵。
% TODO:是不是每一个生成元都可以?
这些被对角化的生成元的集合称为Cartan子代数,它是对应的李群的李代数的表示的子代数。
Cartan子代数中的诸算符共享一组可以张成整个$V$的本征矢量,对应的各生成元的本征值——也就是对角矩阵的对角元——可以用来标记这个不可约表示。
要找到一组Cartan子代数并不难:只需要从李群中找到一个交换子代数,然后尝试对角化这个交换子代数中的某一个成员就可以了。
% TODO:李代数在怎样的程度上决定了对应的算符的谱结构?
非奇异矩阵表示一定可以通过相似变换而变成幺正表示(就是所有矩阵都是幺正的表示)。
这也就是我们频繁地讨论幺正表示的原因。但有许多重要的群——例如洛伦兹群——都不是紧致的(或者说群对应的流形无界),因此它们实际上并没有有限维的幺正表示。就洛伦兹群而言,我们将会看到,其推动生成元的有限维表示不是厄米的,因此整个群也没有幺正的有限维表示。

容易验证,设$\hat{X}$是厄米算符,且
\begin{equation}
    \comm*{\hat{a}^\dagger}{\hat{X}} = c \hat{a}^\dagger,
    \label{eq:raising-operator}
\end{equation}
那么
\[
    \hat{a}^\dagger \ket{X} \propto \ket{X+c},
\]
相应的,
\[
    \hat{a} \ket{X} \propto \ket{X-c}.
\]
因此称$\hat{a}^\dagger$为$\hat{X}$的\textbf{升算符},$\hat{a}$为$\hat{X}$的\textbf{降算符}。
数学上可以证明,在李代数的有限维表示上可以定义内积
\begin{equation}
    \langle \hat{A}, \hat{B} \rangle = \trace \hat{A} \hat{B},
\end{equation}
通过合适的线性组合,能够写出一组正交归一化的生成元。
此时非Cartan子代数的生成元中的每一个都是Cartan子代数中的每一个成员的升降算符,
并且任意两个非Cartan子代数的生成元的对易子都可以使用Cartan子代数的成员线性表示。
% TODO:看起来Cartan子代数似乎构成它的不可约表示空间的一个CSCO
% Symmetry and the Standard Model, p108
因此对一个一般的、没有正交归一化的李代数的有限维表示,我们总是可以从李代数的成员构造出一个升算符。设$\hat{X}$为$g_i$,且
\[
    \hat{a}^\dagger = \lambda^j g_j,
\]
则\eqref{eq:raising-operator}等价于
\[
    \comm*{\lambda^j g_j}{g_i} = c \lambda^j g_j,
\]
代入\eqref{eq:structure-of-lie-algebra},上式又等价于
\begin{equation}
    \left( f^k_{ji} - c \delta_j^k \right) \lambda^j = 0,
    \label{eq:determine-ladder-operators}
\end{equation}
于是通过求解
\begin{equation}
    \det \left( f^k_{ji} - c \delta_j^k \right) = 0
    \label{eq:possible-c}
\end{equation}
就可以得到所有可能的$c$,然后将它们代入\eqref{eq:determine-ladder-operators}就能够得到所有能够被非Cartan子代数表示出来的升降算符。
最后,由于是有限维表示,通过以上手法得到的升降算符实际上就是全部可能的升降算符,因此从一个本征态出发,通过它们可以构造出所有的本征态。
有限维表示还意味着,设$\hat{a}^\dagger$是某个升算符,那么对充分大的$N$,$(\hat{a}^\dagger)^N = 0$,$\hat{a}^N=0$,因为本征态的个数有限。
这些条件可用于确定什么样的不可约表示是被允许的。
% TODO:数学证明,不过多半鸽了
这些操作的一个典型的例子见对旋转群的处理。

现在我们分析一种比较特殊的情况。以上我们都是在“李代数可以分解成一个Cartan子代数和非Cartan元素,后者构成前者的升降算符”的框架下分析问题,那么如果李代数中所有元素都对易,那此时它会有怎样的表示?
由于没有非Cartan元素,这样的一个李代数——从而它的李群——不会有有限维的不可约表示。
通常这样的李群对应着某种空间平移操作。

% TODO:连续谱的情况
以上讨论的不可约表示都是有限维的。无限维表示——这里指的是函数空间的表示——则需要一套不同的框架。设$\hat{q}$具有连续谱,且
\begin{equation}
    \comm*{\hat{q}}{\hat{p}} = \ii,
\end{equation}
则
\begin{equation}
    \exp \left( \ii \lambda \hat{p} \right) \ket{q} = \ket{q + \lambda}.
\end{equation}
也就是说$\exp (\ii \lambda \hat{p})$是让$\hat{q}$的本征矢对应的本征值上升$\lambda$的升算符。

由于空间坐标无非是一种向量,李群和李代数也可以被作用在坐标上。
作用在坐标上的有限维表示又诱导出了作用在函数上的无限维表示%
\footnote{在有限维表示中,上下标$\mu$标记向量的诸分量;在函数空间中,坐标$x^\mu$标记“向量”——也就是函数——的诸“分量”——也就是函数在这一点的值。
李群在有限维向量空间上的表示通常是某个矩阵群,它将不同分量混合在一起,即
\[
    \psi^\mu \longrightarrow R^\mu_\nu \psi^\nu.    
\]
李群在无限维向量空间上的表示通常是“改变坐标$x^\mu$”。
}%
。设$f=f(x)$,若李群在坐标上的表示为
\[
    x \longrightarrow x', \quad (x')^\mu = R_\nu^\mu (\Lambda) x^\nu,
\]
则它在关于坐标的函数——也就是“场”——组成的无限维向量空间上的表示就是
\[
    f \longrightarrow f', \quad f(x) = f'(x') = f'(R(\Lambda) x),
\]
或者等价的,
\begin{equation}
    (x \mapsto f(x)) \longrightarrow (x \mapsto f'(x) = f(R(\Lambda)^{-1} x)).
    \label{eq:infinite-dim-rep}
\end{equation}
换而言之,坐标变动“牵引”了从坐标到场值的映射。
考虑到$f$可能是某个多分量对象(比如矢量、矢量的张量积,或者接下来要看到的旋量)的分量,
李群在此多分量场上的作用还包括通常的有限维表示,也就是
\[
    \psi^a \longrightarrow M(\Lambda)^a_b \psi^b.
\]
需注意此处我们使用了另外一个表示$M^a_b$而不是$R^\mu_\nu$,因为不能够保证$\Lambda$在多分量场$\psi$上的作用和它在坐标向量上的作用来自同一个有限维表示。
由于大部分情况下我们都是从一个群在通常意义上的矢量的作用出发讨论其结构的,可以将$R(\Lambda) x$简记为$\Lambda x$,也就是群元$\Lambda$在$x$上的作用。
这样上式就可以简洁地写成
\begin{equation}
    \psi^a(x) \longrightarrow {\psi'}^a (x) = M^a_b (\Lambda) \psi^b (\Lambda^{-1} x).
    \label{eq:wigner-transform}
\end{equation}
这种同时考虑了多分量场在李群作用下各分量重新混合(这是一个有限维表示)和李群作用下坐标拖曳而改变场(这对坐标而言是另一个有限维表示,对场而言是一个无限维表示)的李群的表示就是\textbf{场表示}。
需要注意的是,不同的$\Lambda$作用到坐标上可能会得出同样的结果,而它们对应的$M$作用到场上却有不同的结果,正如$SU(2)$和$SO(3)$的关系告诉我们的那样。

\eqref{eq:wigner-transform}给出的是李群的场表示的一般形式,但此时我们还只有形式上的变换而没有显式的表达式。
我们来分析其李代数。取%
\footnote{虽然可以任意地调整群参数,从而让生成元前面的系数随意变动,但是通常对有限维表示和无限维表示我们总是采用同样的群参数。这就意味着,在有限维表示确定之后不能随意调节无限维表示的生成元前面的系数,不能随意加一个$\ii$或者改变正负号。这也就是我们在场表示中一并处理有限维表示和无限维表示的原因,因为此时两者的群参数自动地就是相同的。

下式中的$g$的定义可以是\[
    g = \frac{1}{\ii} \pdv{G}{g},
\]
但也可以是像我们定义旋转生成元时的那样,取
\[
    g = \ii \pdv{G}{g},
\]
只需要将$\epsilon$取为负值就可以了。无论$g$是怎么定义的,下式都是成立的。}%
\[
    \Lambda = I + \ii \epsilon g,
\]
其中$g$是一个生成元,我们就有
\[
    \begin{aligned}
        \psi^a \longrightarrow {\psi'}^a &= M^a_b (\Lambda) \psi^b (\Lambda^{-1} x) \\
        &= (I + \ii \epsilon M^a_b(g)) \psi^b (x - \ii \epsilon g x) \\
        &= (I + \ii \epsilon M^a_b(g)) (\psi^b - \ii \epsilon g x \cdot \grad{\psi^b}) \\
        &= \psi^b + \ii \epsilon M^a_b(g) \psi^b - \ii \epsilon g x \cdot \grad{\psi^b},
    \end{aligned}
\]
于是
\[
    {\psi'}^a = (I + \ii \epsilon  (M^a_b(g) - g x \cdot \grad)) \psi^a,
\]
于是场表示的生成元可以写成
\begin{equation}
    M_\text{field} = M_\text{fin} + M_\text{inf}, \quad M_\text{fin} = M^a_b(g), \quad M_\text{inf} = - (g x) \cdot \grad.
    \label{eq:fin-and-inf-rep}
\end{equation}
其中$M_\text{fin}$就是我们所熟悉的李群在有限维向量空间上的矩阵表示,而$M_\text{inf}$则是李群作用在坐标上,拖曳坐标而对场产生的影响。
显然,它们和$g$之间能够建立同态关系。$gx$和$\Lambda x$一样,都是“$g$在坐标空间上的有限维矩阵表示作用于$x$”的简写。
与通常物理中的记号不同,此处的梯度算符作用在所有坐标上,不仅仅是空间坐标,还包括时间坐标。

在以上讨论的基础上我们讨论态矢量。我们总是使用李群在希尔伯特空间上的幺正表示,因为需要保证变换前后的态矢量都是物理的,也就是说,都是正交归一化的。
我们刚才讨论了李群的场表示,这个场表示当然可以被作用在算符场上。但是注意到算符场是态空间上的算符,因此按照\eqref{eq:group-action-on-operators},李群的场表示自然地如下导出了李群在希尔伯特空间上的表示:
\begin{equation}
    \hat{U}(\Lambda) \hat{\psi}^b(\vb*{x}) \hat{U}^{-1}(\Lambda) = M^a_b (\Lambda) \psi^b (\Lambda^{-1} x).
    \label{eq:field-rep-and-state-rep-lie-group}
\end{equation}
由于对$\hat{\phi}$的变换等价于对其本征值做变换,这又等价于保持本征值不变而重新安排本征态,按照上式诱导出的在希尔伯特空间上的李群表示$\hat{U}$也是幺正的。

相应的,\eqref{eq:field-rep-and-state-rep-lie-group}也导致了对应的李代数在希尔伯特空间上的表示。对\eqref{eq:field-rep-and-state-rep-lie-group}取微元,得到
\[
    (1 + \ii \epsilon M_\text{state}) \hat{\psi}^b (1 - \ii \epsilon M_\text{state}) = \ii \epsilon M_\text{field} \hat{\psi},
\]
从而
\begin{equation}
    \comm*{M_\text{state}}{\psi} = M_\text{field} \psi.
    \label{eq:field-rep-and-state-rep-gen}
\end{equation}
实际上,时间演化方程\eqref{eq:quantum-evolution}就是一个例子:时间平移群在希尔伯特空间上的表示是哈密顿算符$\hat{H}$,在场——这里是任何一种物理量——上的表示是$\frac{1}{\ii} \dv{t}$,那么
\[
    \comm*{\hat{H}}{\hat{A}} = \frac{1}{\ii} \dv{t},
\]
这就是时间演化方程。

考虑一个简单的单粒子量子力学的例子:$\hat{x} + a$是将大小为$a$的平移作用在$\hat{x}$上的结果,而考虑被$\hat{x}$完全描述的一个希尔伯特空间,在其上有
\[
    \hat{x} + a = \int \dd{x} x \dyad{x} + a \int \dd{x} \dyad{x} 
    = \int \dd{x} (x + a) \dyad{x} = \int \dd{x'} x' \dyad{x'-a},
\]
因此作用在$\hat{x}$上的大小为$a$的平移就等价于作用在态空间基矢量上的大小为$-a$的平移。
更一般的,将某一个李群$Q(a)$作用在某一算符上就相当于将这一李群的群参数倒转过来得到新的李群$Q'$,
也就是定义$Q'(a) = Q(a)^{-1}$(由于是群,$Q'$和$Q$同构),然后将$Q'(a)$作用在态空间的基矢量上。
由于$Q'$和$Q$同构,两者的区别仅仅是重新规定了群参数,因此它们对应着同样的对称性。
% TODO:以上说法的推广
总之,我们既可以直接从某种李群的场表示出发,推导它允许的算符场有哪些,然后使用二次量子化的有关知识导出其对应的单粒子态,%
\footnote{关于何为“粒子”需要说明:一般把能够使用一个不很复杂的CSCO描述的量子系统称为粒子,例如可以使用$\hat{\vb*{x}}$或$\hat{\vb*{p}}$描述一个粒子。但按照这种定义,原子能级也可以算粒子了——实际上这并不是胡思乱想,在处理量子光学等领域的一些问题时确实可以将能级看成一种粒子,定义其产生湮灭算符,得到费米场,等等——因此,何为粒子更多的是一种约定的说法。实际上任何一个哈密顿量都可以对角化,写出能级之后将不同能级看成不同粒子,然后使用二次量子化的语言描述它。}%
也可以从李群在希尔伯特空间上的表示出发,直接得到单粒子态然后构造算符场。
两种方法是完全一致的。舒尔引理告诉我们,卡西米尔算符(和所有生成元都对易)在不可约表示中一定是恒等算符的常数倍。这个常数可以用来标记相应的不可约表示;事实上这一类常数往往会出现在相应的表示描写的场/粒子的运动方程中,因为运动方程中会出现卡西米尔算符的场表示。
相对而言,在推导运动方程的时候,使用场的观点更加方便,因为相对论情况下粒子数通常是不确定的,因此使用单粒子态难以写出哈密顿量。

概括以下我们至今得到的结果:李群和李代数的表示有下面几种,它们彼此之间有非常密切的关系。
首先,李群和李代数在有限维向量空间上的表示是矩阵,它们或者是可约表示,或者不可约,前者可以通过直和运算由后者组装出来。
不可约有限维表示的结构可以通过使用李代数中的非Cartan元素构造Cartan子代数的升降算符来确定。
通过将有限维表示作用在坐标上,我们得到了作用在关于坐标的函数组成的向量空间上的无限维表示。
将作用在多分量对象上的有限维表示和作用在坐标函数上的无限维表示结合起来,就得到了场表示。
李群在向量空间上的表示很自然地就诱导出了李群在作用在向量空间上的算符上的表示。

\section{动力学}

% 似乎拉格朗日动力学中含有虚部的场要看成两个场,而哈密顿动力学中含有虚部的场只需要看成一个场。
在进一步展开下面的叙述之前,我们先回顾现代物理的数学框架。总的来说,有两套可用的框架,
其一是拉格朗日动力学,路径积分方法是它的量子版本;其二是哈密顿动力学,正则量子化是它的量子版本。
尽管这两个框架在数学上是独立的,我们仍然可以找到它们之间非常深厚的联系。
本节首先从经典拉氏量出发,然后得到经典哈密顿量,然后再过渡到量子形式。

一个物理系统无非是这样的:我们有一组\textbf{动力学变量},其自由度就是\textbf{动力学自由度},能够表示其它任何动力学变量但是彼此不能相互表示的一组动力学变量称为一组\textbf{完备的}动力学变量,它们的总数当然就是动力学自由度;我们与此同时也有一个\textbf{系统状态},给定一个系统状态,动力学自由度可以是完全确定的也可以并不确定。
物理学的任务就是描述系统状态如何发生\textbf{演化},演化方式——比如说物理量满足的微分方程——就称为系统的\textbf{动力学}。
通常有一个参数称为\textbf{时间}(记作$t$)来标记演化的进度。
当然,实际上完全可以存在有多个时间维度的物理理论,但是本文暂时不考虑这么深入的问题。
一个物理量在不同时间点的取值放在一起就是它的\textbf{世界线}。
表面上看,似乎系统状态就应该是动力学自由度的取值,但我们将看到事情其实没有那么简单。

动力学自由度有\textbf{标签},标签的取值可以是离散的——如“第一号粒子,第二号粒子”等,或者离散而无限的,也可以是连续的。
后两者情况下,我们将不同格点/不同空间点处的某个动力学自由度放在一起,得到一个从点阵/流形到物理量的映射,这就是一个\textbf{场}。
理论中的彼此独立的场的数目称为\textbf{场自由度}。显然,有限个场自由度可以导致无限个动力学自由度。%
\footnote{有时候,希尔伯特空间的维数也会被称为自由度。实际上希尔伯特空间的维数的对数是自由度的很好的量度:它在两个系统取直积时具有可加性,并且不会因为我们将两个算符通过某种手段写成一个或者将一个算符拆成两个而发生变化。}%
如果系统中没有场自由度,且动力学自由度本身代表空间位置和/或动量等物理量,那么这就是一个关于\textbf{粒子}的理论。
时间轴直积上场的标签来自的空间得到的结果就是所谓的\textbf{底流形},它是物理现象发生的“舞台”。
我们设空间维数为$d$,则底流形就是$d+1$维时空,或者说$D$维时空。
常见的底流形如$3+1$维闵可夫斯基时空中或$0+1$维时空(即关于粒子的理论),而后者可以看成前者的一个退化情况,
所谓闵可夫斯基时空指的是度规可以化为
\begin{equation}
    \eta_{\mu\nu} = \diag (1, -1, -1, -1)
\end{equation}
的四维几何。通常使用$t, x, y, z$或者$x^0, x^1, x^2, x^3$来依次标记这4个坐标。
容易看出$x, y, z$或者说$x^1, x^2, x^3$就构成一个三维欧氏几何,它们是\textbf{空间维}。
$x^0$则是\textbf{时间维}。

很容易就可以想到,关于场的理论和关于粒子的理论是可以互相转化的,本文将时不时提到这一点。
例如,定义在格点上的场论也可以认为是关于大量(不怎么移动,动力学自由度主要是除了空间坐标以外的自由度或者说\textbf{内禀自由度}的)粒子的粒子理论,而由于我们总是可以对连续的场论取微元做离散化,那么任何一个连续场论实际上都可以看成一个粒子理论。
反过来,粒子可以认为是以场为产生湮灭算符而产生的(见\autoref{sec:second-quantization}),那么关于粒子的理论也可以转化为一个场论。

我们还将假设,所有场量在无穷远处的值都是零。
我们将要分析的对象是时空中的场,它是从底流形到某一线性空间的光滑映射,或者也可以说是空间中的场的世界线。

\subsection{拉格朗日动力学}

所谓\textbf{拉氏量密度}$\mathcal{L}$——在场论中简称为\textbf{拉氏量}——是这样一个量,它是场的局域%
\footnote{在物理中“局域”一词可以有很多不同的——虽然紧密相关的——意思。
在这里,“局域”指的是物理规律中没有显式的超距作用:不会出现$\phi(\vb*{x}) \phi(\vb*{y})$形式的项。
更强的“局域”指的是运动方程中没有显式的超距作用,费米子系统不具有这个意义上的局域性,因为不同点的费米子自由度反对易而可以不对易。
当被用于修饰连续对称性时,“局域”通常表示一个对称操作的作用范围限制在一个有限的空间区域内。
当用于修饰一个普通的场函数时(例如,修饰$\braket*{\vb*{x}}{\psi}$时),“局域”或者更常见的“定域”表示它在一个有限的区域外衰减得足够快。
}%
泛函,
这就是说,它可以写成$\phi, \partial_\mu \phi, \ldots$以及时空坐标的函数。
本文假定所有的拉氏量仅含有一阶导数,这是为了避免含有高阶导数的拉氏量产生“可以无穷下降的能量”等反直觉现象,并且简化计算。
幸运的是,已有的实验数据并不要求我们考虑更高阶的拉氏量。
我们还假定物理规律在时空上是均匀的,因此我们不认为拉氏量中显含时空坐标。%
\footnote{
    需要注意的是在系统中有相互作用且其中一部分的运动状态已知的情况下,另一部分的有效拉氏量中是有可能出现时空坐标的,
    例如粒子在势场中的运动就是一个典型例子,在那里由于产生势场的物理机制远远比粒子本身要强,因此势场可以看成是给定的,
    于是粒子具有的有效拉氏量就显含了空间坐标。}%
从而我们有
\begin{equation}
    \mathcal{L} = \mathcal{L}(\phi, \partial_\mu \phi).
    \label{eq:lagrangian}
\end{equation}
需要注意的是\eqref{eq:lagrangian}中的$\phi$可以代表任何一个“从时空坐标到数量”的映射,
它可能是一个标量场也可能是一个矢量场的分量,或者是别的什么东西。
\textbf{作用量}是拉氏量在整个闵可夫斯基时空上的积分。

现在我们将一个任意的无穷小变换作用在泛函$S$上,观察其无穷小变动。
需要注意的是无穷小变换同时作用在$\phi$的场值和坐标上,从而$\phi$完整的变化%
\footnote{在实际计算时往往更加容易求出$\var{\phi}$,因为一旦把$\phi'(x')$完全写出,只需要计算$\phi'(x')-\phi(x)$ 即可。}%
同时包含两部分:
\begin{equation}
    \var{\phi} = \bar{\var} \phi + \partial_\mu \phi \var{x}^\mu,
    \label{eq:variance-of-phi}
\end{equation}
其中第一项指的是场值本身的变化%
\footnote{这个变化又有可能来自两个方面。
其一是“场的平移”,也就是我们手动把场$\phi$加减特定值;
其二是“场的旋转”,当$\phi$实际上是某个更大的对象(如矢量)的某个分量时,基矢量的旋转会导致不同的分量混在一起。
通常我们使用一样的基矢量来书写场的分量和坐标的分量,因此除了坐标平移外,坐标变换也伴随着非零的$\bar{\var}{\phi}$。}%
,第二项指的是坐标变换的“拖曳”作用。
坐标的变化还会导致导数算符和积分测度发生变化。这两个几何效应的具体表达式为
\begin{equation}
    \begin{bigcase}
        \partial_{\mu'} = \partial_\mu - \partial_\mu \var{x^\nu} \partial_\nu, \\
        \dd[D]{x'} = (1 + \partial_\mu \var{x^\mu}) \dd[D]{x}.
    \end{bigcase}
\end{equation}
由于$\partial_\mu$算符随着坐标变换会发生变化,我们发现$\partial_\mu \phi$的变化量的形式和$\phi$不完全一致:
\begin{equation}
    \var{\partial_\mu \phi} = \partial_\mu \bar{\var}{\phi} + \partial_\mu \partial_\nu \phi \var{x^\nu}.
\end{equation}
这样一来我们可以计算出
\begin{equation}
    \var{S} = \int \dd[D]{x} \left(
        \left( \pdv{\mathcal{L}}{\phi} - \partial_\mu \pdv{\mathcal{L}}{\partial_\mu \phi} \right) \bar{\var}{\phi} + 
        \partial_\mu \left( \mathcal{L} \var{x^\mu} + \pdv{\mathcal{L}}{\partial_\mu \phi} \bar{\var}{\phi} \right)
    \right).
    \label{eq:variance-of-s}
\end{equation}
在推导\eqref{eq:variance-of-s}时我们没有使用任何关于$\var{\phi}$和$\var{x}$的假设,因此它给出的是最一般的$\var{S}$形式。

实际的场的动力学由保持时空坐标$x$不变且$\phi$在无穷远处固定为零(从而无穷远处$\bar{\var}{\phi}$为零)的情况下的泛函极值问题
\begin{equation}
    \var{S} = \var{\int \dd[D]x \mathcal{L}(\phi, \partial_\mu \phi)}
    \label{eq:min-action}
\end{equation}
给出。
显然这个泛函极值问题的解就是
\begin{equation}
    \pdv{\mathcal{L}}{\phi} - \partial_\mu \pdv{\mathcal{L}}{\partial_\mu \phi} = 0.
    \label{eq:el-eq}
\end{equation}
这就是欧拉-拉格朗日方程。
由于推导欧拉-拉格朗日方程时用到了$\var{\phi}$的任意性,这意味着$\phi$被假定是一个实的场。
如果某些场有虚部,那么在使用\eqref{eq:el-eq}以及相关结论的时候需要把它的实部和虚部分开,当成两个场来处理。
并且,容易证明,设复场$\phi$的实部和虚部分别是$\phi_1$和$\phi_2$,且
\[
    \pmqty{\psi_1 \\ \psi_2} = \pmqty{a & b \\ c & d} \pmqty{\phi_1 \\ \phi_2},
\]
其中$a,b,c,d$为复常数,则$\psi_1$和$\psi_2$的运动方程也可以从\eqref{eq:el-eq}得出。
常见的选择包括取
\[
    \psi_1 = \phi, \psi_2 = \phi^\dagger,
\]
或者如果$\phi$是多分量场,设有一系列复矩阵(不必都是复矩阵,有一个是复的就可以)$\gamma^\mu$,取
\[
    \psi_1 = \phi, \psi_2 = \gamma^\mu \phi_\mu.
\]

需要注意如果两个拉氏量的形式不同,这并不意味着它们描述了不同的物理过程。
实际上容易看出,两个拉氏量描述了相同的物理过程,
当且仅当,它们给出的作用量$S$只相差一个相对于$\dd[D]{x}$的零测集上的积分(这样的积分不影响泛函极值问题的求解,因为它“太小”),
这又等价于这两个拉氏量相差一个散度项,即存在一个$\Lambda^\mu$使得
\begin{equation}
\mathcal{L}' = \mathcal{L} + \partial_\mu \Lambda^\mu.
\end{equation}

当场量$\phi$是物理解的时候,将$\phi$代入到$S$中,然后再做一个无穷小变换,此时\eqref{eq:variance-of-s}中的第一项为零,
于是我们有
\[
    \var{S} = \int \dd[D]{x} \partial_\mu \left( \mathcal{L} \var{x}^\mu + \pdv{\mathcal{L}}{\partial_\mu \phi} \bar{\var}\phi \right).
\]
如果这个无穷小变换实际上不改变系统的动力学,也就是说系统在这个无穷小变化下是对称的,
那么$\var{S}$就应该能够写成一个表面积分,于是我们得到
\begin{equation}
    \partial_\mu \left(\pdv{\mathcal{L}}{\partial_\mu \phi} \bar{\var}\phi + \mathcal{L} \var{x^\mu} + \Lambda^\mu\right) = 0.
    \label{eq:noether}
\end{equation}
当然,如果无穷小变换更进一步不改变拉氏量,那么$\Lambda=0$。

如果无穷小变换是一个李群的李代数的表示,那么$\bar{\var}{\phi},\var{x^\mu}, \Lambda^\mu$都是完全确定的。可以使用小量近似将$\bar{\var}{\phi}$写成小量$ \ii \epsilon$乘以李代数的场表示\eqref{eq:fin-and-inf-rep},$\var{x^\mu}$写成小量$\ii \epsilon$乘以李代数的矢量表示,
于是我们在\eqref{eq:noether}中除去一个$\epsilon$,就得到了一个守恒流。
于是\eqref{eq:noether}的括号中的内容能够完全写成坐标的函数。
这就是\textbf{诺特定理}:系统的无穷小对称性诱导出一个守恒流。
时空中的守恒流
\begin{equation}
    \partial_\mu j^\mu = 0
\end{equation}
就意味着空间的一个输运方程
\begin{equation}
    \partial_t j^0 + \partial_a j^a = 0.
\end{equation}
从而,
\begin{equation}
    Q = \int \dd[3]x j^0
\end{equation}
就是一个\textbf{守恒荷}。如果其积分范围是一个有限的区域,那么它就是一个局域守恒量,也就是
\[
    \dv{t} Q = - \int \dd{\vb*{S}} \cdot \vb*{j},
\]
而如果其积分范围是全空间,那么它就是守恒的。

我们来检查一下常见的对称性导致的守恒量。%
\footnote{表面上看,下面的讨论在体系并不非常对称的情况下并无意义,而不非常对称的体系占了多数。
不对称性带来的后果是,我们不再有完美的守恒流方程,取而代之的是一个有源的输运方程
\[
    \partial_\mu j^\mu = \text{something},
\]
由于对称性分析无助于找到源的具体形式,使用对称性诱导出特定的物理量似乎并没有什么意义。
然而,我们相信,最基本的物理定律总应该是对称的,因此大部分体系的不对称性可以归结为我们人为地将它从环境中隔离出来进行研究,从而导致类似下面的方程:
\[
    \partial_\mu (j^\mu_\text{sys} + j^\mu_\text{env}), \quad \partial j^\mu_\text{sys} = - j^\mu_\text{env}
\]
第二个方程给出了我们想要的含源的输运方程。因此在分析基本的物理框架时我们可以不讨论“不对称”的情况,
而是导出了基本的方程之后再通过“隔离出一部分系统”来引入不对称性。
}%
假定拉氏量在变换下不变。下面处理的问题都只含有一个场,不过由拉氏量的叠加性,在拉氏量含有多个场的时候只需要把各部分加起来即可。
首先是最简单的平移。处理平移时假定场是标量场,这无损一般性,因为平移没有有限维表示,因此不会导致场分量发生混合。
平移变换作用于场上得到的结果是:
\[
    \begin{split}
        x^\mu \longrightarrow x^{\mu'} = x^\mu + a^\mu, \\
        \var{\phi} = \phi'(x') - \phi(x) = 0.
    \end{split}
\]
% TODO:群作用怎么取
按照\eqref{eq:variance-of-phi},可以计算出
\[
    \bar{\var}{\phi} = - \partial_\mu \phi \var{a^\mu},
\]
或者,由于场在坐标拖曳下的变动实际上就是平移变换的无限维表示,可以直接使用平移变换的无限维表示
\[
    P_\mu = - \ii \partial_\mu
\]
得到上式。
于是对应的守恒流为
\[
    0 = \partial_\mu \left( - \pdv{\mathcal{L}}{\partial_\mu \phi} \partial_\nu \phi \var{a^\nu} + \mathcal{L} \var{a^\mu} \right) 
    = \partial_\mu \left( - \pdv{\mathcal{L}}{\partial_\mu \phi} \partial_\nu \phi + \mathcal{L} \delta^\mu_\nu \right) \var{a^\nu},
\]
考虑到$\var{a^\mu}$的任意性,我们有
\begin{equation}
    T_\mu^\nu = \pdv{\mathcal{L}}{\partial_\nu \phi} \partial_\mu \phi - \mathcal{L} \delta^\nu_\mu, \quad \partial_\nu T_\mu^\nu = 0.
\end{equation}
我们称$T^\nu_\mu$为\textbf{能动张量}。它给出了一系列共$D$个守恒荷,其中一个是来自时间平移不变性的\textbf{能量}
\begin{equation}
    E = \int \dd[3]{x} T^0_0 = \int \dd[3]{x} \left( \pdv{\mathcal{L}}{\partial_0 \phi} \partial_0 \phi - \mathcal{L} \right) ,
    \label{eq:field-energy}
\end{equation}
另外$d$个是来自空间平移不变性的\textbf{动量}
\begin{equation}
    P_i = \int \dd[3]{x} T^0_i = \int \dd[3]{x} \pdv{\mathcal{L}}{\partial_0 \phi} \partial_i \phi .
    \label{eq:field-momentum}
\end{equation}
能动张量的纯空间部分是能量和动量的输运流,因此就是\textbf{应力张量}。%
\footnote{在非相对论连续介质力学中这些结果也是成立的,因为时间和空间平移同时出现在伽利略群和庞加莱群中。}
相应的,
\begin{equation}
    \mathcal{P}_\mu = \pdv{\mathcal{L}}{\partial_0 \phi} \partial_\mu \phi - g_\mu^0 \mathcal{L}
\end{equation}
为时空中的动量$(E, \vb*{p})$的密度。
在计算场的三维动量时要注意一点:由于闵可夫斯基度规为$(+, -, -, -)$,闵可夫斯基时空中空间部分的基矢量实际上是指向空间坐标减少的方向的。从而,
\[
    \begin{aligned}
        \vb*{P} &= \int \dd[3]{x} \pdv{\mathcal{L}}{\partial_0 \phi} \partial_i \phi \vb*{g}^i \\
        &= - \int \dd[3]{x} \pdv{\mathcal{L}}{\partial_0 \phi} \partial_i \phi \vb*{g}^i_{\text{3dim}},
    \end{aligned}
\]
也即
\begin{equation}
    \vb*{P} = - \int \dd[3]{\vb*{x}} \pi \grad{\phi}.
\end{equation}

上面的推导暗示能量是非常重要的物理量,因为它来自时间平移不变性,那么能量守恒应该是非常普遍的定律。的确如此——实际上在哈密顿动力学中能量是动力学演化的核心。

然后我们分析场的内禀对称性带来的守恒量。容易看出场的平移,也就是
\[
    \bar{\var}{\phi} = a, \; \var{x} = 0
\]
对应着守恒流
\[
    \partial_\mu \pdv{\mathcal{L}}{\partial_\mu \phi } a = 0,
\]
其守恒荷为
\begin{equation}
    \Pi = \int \dd[3]x \pdv{\mathcal{L}}{\partial_0 \phi}.
\end{equation}
这称为$\phi$的\textbf{共轭动量},相应的其密度
\begin{equation}
    \pi = \pdv{\mathcal{L}}{\partial_0 \phi}
    \label{eq:def-pi}
\end{equation}
就是\textbf{共轭动量密度}。
需注意此“动量”的名称只是类比而得,它未必和$P_i$有特别紧密的联系。

需要注意的是本节给出的诺特定理在量子情况下是有可能失效的。量子理论中作用量不再必须取极小值,而是被用作计算一个路径积分,在特定的变换下,可能出现作用量不变而路径积分的积分测度发生变化的情况,此时诺特定理失效。这种情况称为\textbf{量子反常}。

\subsection{哈密顿动力学}

\subsubsection{经典哈密顿动力学}
原本可以直接从拉氏量通过一个勒让德变换得到哈密顿动力学,但当底流形有多个坐标时我们需要选择合适的一个或几个坐标来充当“时间”,也就是哈密顿系统的参数。
共轭动量使我们有了一个很好的选择。本文取$t=x^0$为哈密顿系统的单参数。接下来我们要观察共轭动量的变化情况,从而凑出一个哈密顿系统。

容易看出$\Pi$的运动方程为%
\footnote{本节的结果也不仅仅适用于相对论性场论。任何能够良定义场的平移并且保证场平移不改变拉氏量的拉格朗日动力学场论都可以使用本节的方法构造对应的哈密顿表述,因为本节只用到了场的内禀平移不变性诱导出的结构。}
\[
    \dv{\Pi}{t} = \int \dd[3]x \partial_0 \pdv{\mathcal{L}}{\partial_0 \phi} = \int \dd[3]x \left( \pdv{\mathcal{L}}{\phi} - \partial_i \pdv{\mathcal{L}}{\partial_i \phi} \right).
\]
被积函数是$\int \dd[3]x \mathcal{L}$在将$x^0$当成常数后对$\phi$泛函求导的结果。于是定义%
\footnote{在$\phi$是多分量场的时候,我们把它看成列向量,记号$\partial \mathcal{L} / \partial \phi$定义为一个行向量,从而所有公式形式上仍然成立。例如,
\[
    \pdv{\vb*{a} \cdot \vb*{x}}{\vb*{x}} = \vb*{a}^\top.
\]
% TODO:需要使用度规吗?
}%
\begin{equation}
    H = \int \dd[3]x \mathcal{H} 
    = \int \dd[3]x \eval{\left( \pdv{\mathcal{L}}{\partial_0 \phi} \partial_0 \phi - \mathcal{L} \right)}_{\partial_0 \phi \to \pi} 
    = \int \dd[3]x \eval{\left( \pi \partial_0 \phi - \mathcal{L} \right)}_{\partial_0 \phi \to \pi}.
    \label{eq:lagrangian-to-hamitonian}
\end{equation}
我们通过将$\partial_0 \phi$用$\pi$表示使得任何对$H$的泛函求导都不会将$\partial_0 \phi$当成变量。
看出,$H$对$\phi$泛函求导就是$-\int \dd[3]x \mathcal{L}$对$\phi$泛函求导,于是我们有
\[
    \dv{\Pi}{t} = - \int \dd[3]x \fdv{H}{\phi}.
\]
另一方面由于$H$不显含任何$\pi$的导数,我们有
\[
    \begin{aligned}
        \fdv{H}{\pi} &= \pdv{\pi} \eval{\left( \pi \partial_0 \phi - \mathcal{L} \right)}_{\partial_0 \phi \to \pi} 
        = \partial_0 \phi + \pi \pdv{\partial_0 \phi}{\pi} - \pdv{\pi} \eval{\mathcal{L}}_{\partial_0 \to \pi} \\
        &= \partial_0 \phi + \pi \pdv{\partial_0 \phi}{\pi} - \pdv{\mathcal{L}}{\partial_0} \pdv{\partial_0}{\pi} = \partial_0 \phi.
    \end{aligned}
\]
于是就得到了3+1维场论的哈密顿表述:
\begin{equation}
    \dv{\pi}{t} = - \fdv{H}{\phi}, \quad \dv{\phi}{t} = \fdv{H}{\pi}.
    \label{eq:hamitonian-eq}
\end{equation}
其中$H$仅仅是$\phi, \partial_i \phi$和$\pi$的函数。
方程中的全导数也可以写成偏导数,我们把它写成全导数是因为我们通常只在一个固定的空间点观察场的变化,也就是说在\eqref{eq:hamitonian-eq}中我们只把时间看成变量而将空间坐标看成“标签”(见\autoref{note:spacial-label})。
由于我们讨论的基本上是场论问题,常常使用下面的记号:%
\footnote{在一些上下文中,场的时间全导数常常被定义为某个位置会随时间发生变化的场点处的场的导数,也就是
\[
    \dot{\phi} = \dv{t} \phi(\vb*{x}, t) = \pdv{\phi}{t} + \dot{\vb*{x}} \cdot \grad{\phi} = \partial_0 A = \partial^0 A.
\]
本文不涉及这样的问题,因此不使用这个记号。
}%
\[
    \dot{A} = \dv{A}{t} = \pdv{A(\vb*{x}, t)}{t}.
\]
在同一个场有多个分量的情况下,我们记各场为$\phi^i$,如果还是希望维持形式上的指标升降关系,$\pi$就可以写成$\pi_i$。

总之,使用拉氏量描述的3+1维经典场也能够使用一个哈密顿动力学描述,这个哈密顿动力学的演化参数为$x^0$也就是时间维,而使用空间维作为连续的“正则坐标”的标记。%
\footnote{也就是说,空间坐标$x^1, x^2, x^3$对应离散情况下的场量标签,
如$\phi^1(x, y, z)$指的是以$1, x, y, z$为标签的一个正则坐标,正如离散时的$q^{1}$代表以$1$为标签的一个正则坐标。
注意到这种哈密顿表述并没有以统一的方式对待时间和空间。\label{note:spacial-label}}%
任何物理量都是$\phi$和$\partial_\mu \phi$的函数,因此它们能够写成$\phi, \partial_i \phi$和$\pi$的函数,从而它们的演化都可以使用\eqref{eq:hamitonian-eq}确定,因为
\begin{equation}
    \dv{A}{t} = \pdv{A}{\phi} \dv{\phi}{t} + \pdv{A}{\partial_i \phi} \partial_i \dv{\phi}{t} + \pdv{A}{\pi} \dv{\pi}{t}.
    \label{eq:evolution-of-any-quantity}
\end{equation}

哈密顿动力学(无论是经典哈密顿动力学还是下一节讨论的正则量子化)中如果场是复的,仍然可以使用\eqref{eq:lagrangian-to-hamitonian}从拉氏量得到哈氏量,但此时不能够保证$\phi$、$\phi^\dagger$、$\pi$、$\pi^\dagger$彼此独立,例如如果选取了$\phi$和$\pi$作为独立变量那么$\phi$和$\phi^\dagger$就不再独立了。
总之,处理复场必须非常小心地注意哪些变量构成一组完备的动力学变量。

\subsubsection{正则量子化}\label{sec:canonical-quantization}

下面我们转而讨论量子情况下的哈密顿动力学。这种使用哈密顿动力学建立量子理论的方法称为\textbf{正则量子化}。%
\footnote{
    \textbf{量子化}这个术语指的是给定哈密顿量或拉格朗日量,写出对应的量子理论的过程;我们还会看到除了正则量子化以外还有路径积分量子化。很容易看出,由于量子理论才是对物理世界的精确描述,实际上能严格定义的只应该有“经典化”而不应该有“量子化”——给定一个经典理论,量子化方案可以不止一种。
    虽然如此,很多时候在经典的视角下可以更加直观地处理一些问题(例如,写出一个近似的相互作用能量),而路径积分量子化保证了这些经典的处理在量子情况下也是可行的,而路径积分量子化和正则量子化是等效的,因此“写出经典体系再量子化”在经典体系实际上有路径积分的对应时总是可以的。
    例如,可以用坐标-动量对描述的系统就能够先经典处理再量子化,而使用自旋矢量描述的系统这么做就可能出现问题。
}%
完整地描述一个量子系统的状态和演化情况需要一个三元组:
首先是一个希尔伯特空间,称为\textbf{态空间},其中的矢量称为\textbf{态矢量},它们表示了系统的状态,
并且我们认为只差了一个倍数的态矢量等价,从而我们可以仅使用单位长度的态矢量描述任何的系统;
其次是一组\textbf{可观察量},它们是希尔伯特空间上的厄米算符,这意味着它们可以被幺正对角化,并且本征值都是实数%
\footnote{后面会提到,如果一个理论在正则量子化时必须选择反对易子的量子化方案,那么实际上它描写的场算符的本征值是格拉斯曼数。
但是如果我们在正则量子化的框架下工作,就从来不关注这种理论对应的场算符的本征值到底是多少,因此没有必要特意讨论它们。
在路径积分量子化的框架中,由于需要讨论费米子的经典场,格拉斯曼数是比较重要的。}%
;
最后是一个\textbf{哈密顿量}或者说\textbf{哈密顿算符},它自身也是一个可观察量(在经典极限下就是经典哈密顿量),且它指示了系统的演化方式。
经典哈密顿理论中同样有对应的三元组,
只不过态空间并不是一个可以做线性叠加的向量空间,从而可观察量也只是从态到实数的映射而不是希尔伯特空间上的算符。
由于所谓的场量$\phi$需要使用算符$\hat{\phi}$代替,因此不再能够良好地定义$\mathcal{H}$对各场量的偏导数,
从而我们也不能良好地定义$\var{H}/\var{\phi}$,等等。
现在动力学方程由
\begin{equation}
    \dv{\hat{A}}{t} = \frac{1}{\ii \hbar} [\hat{A}, \hat{H}]
    \label{eq:quantum-evolution}
\end{equation}
确定。%
\footnote{我们不讨论其定义显含时间的算符,因为它们不会出现在基本的物理规律中。}%
此时有意义的物理量虽然是算符,但在正则量子化之下仍然能够写成场算符$\phi, \partial_i \phi$和$\pi$的函数,
因此一旦$\phi$和$\pi$的演化确定了,\eqref{eq:evolution-of-any-quantity}就以一种和经典情况完全一致的方式确定了所有物理量的演化。
换而言之,\eqref{eq:hamitonian-eq-quantum}完全等价于
\begin{equation}
    \dv{\hat{\phi}}{t} = \frac{1}{\ii \hbar} [\hat{\phi}, \hat{H}], 
    \quad \dv{\hat{\pi}}{t} = \frac{1}{\ii \hbar} [\hat{\pi}, \hat{H}].
    \label{eq:hamitonian-eq-quantum}
\end{equation}

运动方程\eqref{eq:quantum-evolution}意味着,哈密顿量就是与时间平移不变性对应的守恒量。我们称其本征值为\textbf{能量},相应的,其本征态就是\textbf{能量本征态}。

要确定系统的动力学,只需要讨论$[\hat{\phi}, \hat{H}]$和$[\hat{\pi}, \hat{H}]$就可以,
而要讨论这两者又只需要讨论所有有关的场之间的对易关系即可,因为我们总是可以把$H$写成这些场的多项式。
(下文中讨论量子化方案时有对这一点的形象说明)
因此,取对易子为李括号,一个理论中涉及的所有算符就构成了一个李代数,而基本的场之间的对易关系又完全确定了这个李代数的结构。

仅仅有一个抽象的李代数并不能获得完整的理论。
例如,单粒子体系中$\hat{\vb*{x}}$和$\hat{\vb*{p}}$之间的李代数和多粒子体系中每一个粒子的$\hat{\vb*{x}}$和$\hat{\vb*{p}}$之间的李代数具有完全一样的结构,但是显然单粒子体系不是多粒子体系。
例如单粒子体系中$\vb*{x}$的谱没有简并而多粒子体系中$\vb*{x}$的谱有简并。
要获得完整的理论,我们还需要讨论态空间的结构。
我们将不讨论完整的数学,而只是对物理上常用的一些操作做一些说明。

当我们选定一个希尔伯特空间并且将(抽象的李代数中的)算符作用于其上时,实际上是对这个李代数成员做了一个厄米表示。
进一步,当我们说一个系统的希尔伯特空间$H$能够被一组相互对易的算符$S$完全描述时,
我们实际上是说,算符集合$S$在$H$上的(厄米)表示组成了$H$上的一个完备相容算符集合,也即,$S$中各个算符在$H$上的表示共享的本征矢量构成$H$的一组基。
可以证明,如果$S_1,S_2$是$S$的一个划分,且$S_1$完全描述了$H_1$而$S_2$完全描述了$H_2$,那么就有$H$和$H_1 \otimes H_2$同构。
因此我们把$H$完全分解成了若干空间的直积,这些空间中的每一个都由完整描述系统需要的算符中的其中一个完全刻画。

一旦同时知道了各算符的对易关系(从而建立起它们的李代数),以及完整描述系统需要的完备相容算符集合,
我们就可以完整地推导出这个系统每一时刻的状态以及其演化方式了。
实际上,我们真正关注的是完备相容算符集合中各算符的谱结构。

根据上下文,我们可以容易地分辨作为抽象的李代数成员的算符,以及它们在各个希尔伯特空间上的表示,
因此为方便陈述,以下不再对这些略有不同的对象做详细的区分。
对算符而言这样做是合理的,因为从某个表示中得到的代数关系只要不涉及具体的表示的细节,就在抽象的李代数中也成立。
例如如果我们在某个表示中推导出$[\hat{x}, \hat{p}] = \ii \hbar \hat{I}$,那么在抽象的李代数中必定也有这个式子成立,
因为其中只牵扯到算符而没有牵扯到态矢量。
同样,可以比较容易地分辨各个希尔伯特空间中的态矢量,因此在不引起混淆的情况下我们也不刻意区分它们。

现在我们要做的是,分析$\phi$和$\pi$之间要具有什么样的代数关系%
\footnote{实际上只需要分析同一个时间$t$下$\phi(\vb*{x}, t)$和$\pi(\vb*{y}, t)$之间的关系就可以了,
因为\eqref{eq:hamitonian-eq-quantum}中从来不会出现不同时间的量之间的对易子。
另外请注意这套理论并不能原封不动地适用于时空度规和其它物理过程有关的情况(比如广义相对论),因为那里会需要讨论“不同时间处的量之间的关系”。},
才能够让\eqref{eq:hamitonian-eq-quantum}在$\hbar \to 0$(或者等价的,作用量$S$相对于$\hbar$很大)时退化到经典情况\eqref{eq:hamitonian-eq}。
选定这样一个代数关系就称为选取一种\textbf{量子化方案},因为一旦给定了这样的代数关系,我们就把\eqref{eq:hamitonian-eq}推广到了量子理论中。
我们将不试图穷举所有可以的量子化方案,而只是举两个行之有效的例子——也就是说,实验数据要求使用这样的量子化方案。%
\footnote{我们说“穷举所有情况”意味着,面对同一个经典哈密顿量密度$\mathcal{H}$,
有不止一种指定$\hat{\phi}$和$\hat{\pi}$的方式,
使得我们能够得到一个量子动力学\eqref{eq:hamitonian-eq-quantum},
并且在$\hbar \to 0$的极限情况下回退到经典动力学\eqref{eq:hamitonian-eq}。
这是可以预期的,因为“取$\hbar\to 0$的极限”这个操作显然不是一一对应的,或者说量子化方案可以不止一种。
}
% TODO:和二次量子化和拓扑的联系

第一个方案是指定对易子为
\begin{equation}
    [\hat{\phi}^i(\vb*{x}, t), \hat{\pi}_j(\vb*{y}, t)] = \ii \hbar \delta^i_j \delta^3(\vb*{x} - \vb*{y}), 
    \quad [\hat{\phi}^i(\vb*{x}, t), \hat{\phi}^j(\vb*{y}, t)] = [\hat{\pi}_i(\vb*{x}, t), \hat{\pi}_j(\vb*{y}, t)] = 0.
    \label{eq:symmetry-commutator}
\end{equation}
从中可以容易地看出
\begin{equation}
    [\partial_\mu \hat{\phi}^i(\vb*{x}, t), \hat{\phi}^j(\vb*{y}, t)] = 0, 
    \quad [\grad{\hat{\phi}^i(\vb*{x}, t)}, \hat{\pi}_j(\vb*{y}, t)] = \ii \hbar \delta^i_j \grad{\delta^3(\vb*{x} - \vb*{y})}.
    \label{eq:symmetry-partial-mu-commutator}
\end{equation}
只需要将导数看成非常接近的两个量之差,然后利用对易子的线性性即可导出上式。

现在我们来推导$\hbar\to 0$时\eqref{eq:hamitonian-eq-quantum}的极限。
为方便起见,推导过程中假定$\hat{\phi}$是标量场。这并不会有损一般性,因为推导过程中没有用到任何关于坐标变换导致场变化的知识,从而我们可以把同一个多分量场的不同分量看成不同的标量场。
使用\eqref{eq:symmetry-commutator},已知$\hat{\phi} \hat{\pi}$就能够写出$\hat{\pi} \hat{\phi}$。
将$\hat{\mathcal{H}}$写成关于$\hat{\phi}, \partial_i \hat{\phi}$和$\pi$的多项式,
我们可以使用\eqref{eq:symmetry-commutator}和\eqref{eq:symmetry-partial-mu-commutator}
将形如$\hat{\pi}\hat{\phi}$、$\hat{\pi}\partial_i \hat{\phi}$这样的式子改写为形如$\hat{\phi}\hat{\pi}$、$\hat{\partial_i \hat{\phi}}\hat{\pi}$这样的式子。这称为取\textbf{正规积序}:我们总是可以使用对易关系把一个算符多项式转化为一个与之恒等的多项式,后者中的每一项中的算符排列顺序都满足一定的要求。
因此不失一般性地,我们认为$\hat{\mathcal{H}}$的多项式表达式中的每一项都形如$\hat{\phi}^l (\partial_i \hat{\phi})^m \pi^n$。
我们有(因为时间$t$都一样,以下略去$t$变量)%
\footnote{注意下面的$\delta(\vb*{x} - \vb*{x}')$是系数,因此可以自由地移动;写在它们左边的算符不会作用在它们上面!}
\[
    \begin{aligned}
        &\quad\comm{\hat{\phi}(\vb*{x}')}{\int \dd[3]{x} \hat{\phi}^l(\vb*{x}) (\partial_i \hat{\phi})^m (\vb*{x}) \pi^n(\vb*{x})} \\
        &= \int \dd[3]{x} \comm{\hat{\phi}(\vb*{x}')}{\hat{\phi}^l (\vb*{x}) (\partial_i \hat{\phi})^m (\vb*{x}) \pi^n (\vb*{x})} \\
        &= \int \dd[3]{x} \left( \hat{\phi}^l (\vb*{x}) \comm{\hat{\phi}(\vb*{x}')}{(\partial_i \hat{\phi})^m (\vb*{x}) \pi^n (\vb*{x})} + [\hat{\phi}(\vb*{x}'), \hat{\phi}^l (\vb*{x})] \partial_i \hat{\phi}^m (\vb*{x}) \pi^n(\vb*{x}) \right) \\
        &= \int \dd[3]{x} \hat{\phi}^l (\vb*{x}) \left( (\partial_i \hat{\phi})^m (\vb*{x}) \comm{\hat{\phi} (\vb*{x}')}{\hat{\pi}^n (\vb*{x})} + \comm{\hat{\phi} (\vb*{x}')}{(\partial_i \hat{\phi})^m (\vb*{x})} \pi^n (\vb*{x}) \right) \\
        &= \int \dd[3]{x} \hat{\phi}^l (\vb*{x}) (\partial_i \hat{\phi})^m (\vb*{x}) \comm{\hat{\phi} (\vb*{x}')}{\hat{\pi}^n (\vb*{x})} \\
        &= \int \dd[3]{x} \hat{\phi}^l (\vb*{x}) (\partial_i \hat{\phi})^m (\vb*{x}) \ii \hbar \delta^3(\vb*{x'} - \vb*{x}) n \hat{\pi}^{n-1}(\vb*{x}) \\
        &= \ii \hbar n \hat{\phi}^l (\vb*{x}') (\partial_i \hat{\phi})^m (\vb*{x}') \hat{\pi}^{n-1}(\vb*{x}'), 
    \end{aligned}
\]
这正是求导公式。当$\hbar$接近零的时候$\phi$和$\pi$可以交换,于是$\mathcal{H}$可以写成普通的、字母顺序无关紧要的函数,此时
我们有
\[
    \begin{aligned}
        \dv{\hat{\phi}}{t} = \frac{1}{\ii \hbar} [\hat{\phi}(\vb*{x}), \hat{H}] &= \frac{1}{\ii \hbar} \sum_\text{terms} \comm{\hat{\phi}(\vb*{x}')}{\int \dd[3]{x} \hat{\phi}^l(\vb*{x}) (\partial_i \hat{\phi})^m (\vb*{x}) \pi^n(\vb*{x})} \\
        &= \frac{1}{\ii \hbar} \sum_\text{terms} \ii \hbar n \hat{\phi}^l (\vb*{x}') (\partial_i \hat{\phi})^m (\vb*{x}') \hat{\pi}^{n-1}(\vb*{x}') \\
        &= \pdv{\hat{\mathcal{H}}}{\hat{\pi}} = \fdv{\hat{H}}{\hat{\pi}},
    \end{aligned}
\]
这就意味着$\hbar\to 0$时关于$\hat{\phi}$的方程能够回退到经典版本。
同样也有
\[
    \begin{aligned}
        &\quad \comm{\hat{\pi}(\vb*{x}')}{\int \dd[3]{x} \hat{\phi}^l(\vb*{x}) (\partial_i \hat{\phi})^m (\vb*{x}) \pi^n(\vb*{x})} \\ 
        &= \int \dd[3]{x} \comm{\hat{\pi}(\vb*{x}')}{\hat{\phi}^l(\vb*{x}) (\partial_i \hat{\phi})^m (\vb*{x}) \pi^n(\vb*{x})} \\
        &= \int \dd[3]{x} \left( \hat{\phi}^l (\vb*{x}) (\partial_i \hat{\phi} )^m (\vb*{x}) \comm{\hat{\pi}(\vb*{x}')}{\hat{\pi}^n(\vb*{x})} + \hat{\phi}^l (\vb*{x}) \comm{\hat{\pi}(\vb*{x}')}{(\partial_i \hat{\phi})^m (\vb*{x})} \hat{\pi}^n (\vb*{x}) + \comm{\hat{\pi}(\vb*{x}')}{\hat{\phi}^l (\vb*{x})} (\partial_i \hat{\phi})^m (\vb*{x}) \hat{\pi}^n (\vb*{x}) \right) \\
        &= \int \dd[3]{x} \left( - \hat{\phi}^l (\vb*{x}) \ii \hbar \grad{\delta^3 (\vb*{x} - \vb*{x}')} m (\partial_i \hat{\phi})^{m-1} (\vb*{x}) \hat{\pi}^n (\vb*{x}) - \ii \hbar \delta^3(\vb*{x} - \vb*{x}') l \hat{\phi}^{l-1} (\vb*{x}) (\partial_i \hat{\phi})^m (\vb*{x}) \hat{\pi}^n (\vb*{x}) \right) \\
        &= \ii \hbar \partial_i \left(m (\partial_i \hat{\phi})^{m-1} (\vb*{x}') \hat{\pi}^n (\vb*{x}')\right) - \ii \hbar l \hat{\phi}^{l-1} (\vb*{x}') (\partial_i \hat{\phi})^m (\vb*{x}') \hat{\pi}^n (\vb*{x}').
    \end{aligned}
\]
当$\hbar \to 0$时
\[
    \begin{aligned}
        \dv{\hat{\pi}}{t} &= \frac{1}{\ii \hbar} [\hat{\pi}(\vb*{x}), H] \\
        &= \frac{1}{\ii \hbar} \sum_\text{terms} \left(\ii \hbar \partial_i \left(m (\partial_i \hat{\phi})^{m-1} (\vb*{x}') \hat{\pi}^n (\vb*{x}')\right) - \ii \hbar l \hat{\phi}^{l-1} (\vb*{x}') (\partial_i \hat{\phi})^m (\vb*{x}') \hat{\pi}^n (\vb*{x}')\right) \\
        &= - \sum_\text{terms} \left(l \hat{\phi}^{l-1} (\vb*{x}') (\partial_i \hat{\phi})^m (\vb*{x}') \hat{\pi}^n (\vb*{x}') - \partial_i \left(m (\partial_i \hat{\phi})^{m-1} (\vb*{x}') \hat{\pi}^n (\vb*{x}')\right) \right) \\
        &= - \left( \pdv{\hat{\mathcal{H}}}{\hat{\phi}} - \partial_i \pdv{\hat{\mathcal{H}}}{\partial_i \hat{\phi}} \right) = - \fdv{\hat{H}}{\hat{\phi}},
    \end{aligned}
\]
因此关于$\pi$的方程也回退到经典情况。
这表明\eqref{eq:symmetry-commutator}是一个可行的量子化方案。

第二个方案是指定反对易子——而不是对易子——为%
\footnote{彼此无关的场,无论它们自己服从\eqref{eq:symmetry-commutator}还是\eqref{eq:antisymmetry-commutator},相互之间总是对易的。在\eqref{eq:symmetry-commutator}中这是显然的,因为可以将无关的场看成某个多分量场的分量,然后因为它们是不同的分量,它们自然对易。但在\eqref{eq:antisymmetry-commutator}方案下需要额外增加一个规定:
\[
    \comm*{\hat{\phi}(\vb*{x}, t)}{\hat{\psi}(\vb*{y}, t)} = 0.
\]
}
\begin{equation}
    \{\hat{\phi}(\vb*{x}, t), \hat{\pi}(\vb*{y}, t)\} = \ii \hbar \delta^3(\vb*{x} - \vb*{y}), \quad \{\hat{\phi}(\vb*{x}, t), \hat{\phi}(\vb*{y}, t)\} = \{\hat{\pi}(\vb*{x}, t), \hat{\pi}(\vb*{y}, t)\} = 0.
    \label{eq:antisymmetry-commutator}
\end{equation}
同样,我们可以将哈密顿量写成若干个$\hat{\phi}^l (\partial_i \hat{\phi})^m \pi^n$形式的项的和。
需要注意的是\eqref{eq:antisymmetry-commutator}直接导出
\[
    \hat{\phi}(\vb*{x})^2 = 0, \quad \hat{\pi}(\vb*{x})^2 = 0, \quad (\partial_i \hat{\phi})^2(\vb*{x}) = 0,
\]
因此哈密顿量中$l, m, n \leq 1$。
这意味着这个量子化方案并不适用于所有的场,而是只适用于能够保证在任何情况下哈密顿量中的每一项都满足$l, m, n \leq 1$的场。
对于正常的实数/复数值场,这是一个不可能的事情。
事实上,设$\hat{\phi}$在$\vb*{x}$处的值为$\int \dd \phi(\vb*{x}) \dyad{\phi}$,
在$\vb*{y}$处的值为$\int \dd \phi(\vb*{y}) \dyad{\phi}$,则通过反对易关系能够得到
\[
    \phi(\vb*{x}) \phi(\vb*{y}) = - \phi(\vb*{y}) \phi(\vb*{x}).
\]
因此,反对易子意味着对应的场算符的本征值——也就是其经典极限——实际上并不是实数,甚至也不是复数,而是格拉斯曼数。
在复数域中满足反对易关系的场算符不能被对角化。
在路径积分量子化中,格拉斯曼数非常重要,因为路径积分量子化会分析经典场值的演化路径。
在正则量子化中只需要把这些格拉斯曼数看成算符(准确地说,是产生算符)就可以了——我们并不会用到它的微积分,因此也无需将它们看成数。

为了看出反对易方案的不同寻常,我们指出如下事实:一个通过\eqref{eq:antisymmetry-commutator}量子化的场不可能是厄米的。
我们有
\[
    \hat{\phi}(\vb*{x}, t) \hat{\pi} (\vb*{y}, t) + \hat{\pi} (\vb*{y}, t) \hat{\phi} (\vb*{x}, t) = \ii \hbar \delta(\vb*{x} - \vb*{y}),
\]
于是
\[
    \left(\hat{\phi}(\vb*{x}, t) \hat{\pi} (\vb*{y}, t) + \hat{\pi} (\vb*{y}, t) \hat{\phi} (\vb*{x}, t)\right)^\dagger = - \ii \hbar \delta(\vb*{x} - \vb*{y}),
\]
如果场是厄米的,那么就有
\[
    \hat{\pi} (\vb*{y}, t) \hat{\phi} (\vb*{x}, t) + \hat{\phi}(\vb*{x}, t) \hat{\pi} (\vb*{y}, t) = - \ii \hbar \delta(\vb*{x} - \vb*{y}).
\]
于是我们得到了一个矛盾。因此,使用\eqref{eq:antisymmetry-commutator}量子化的场应该分解成非零的厄米和反厄米部分,即
\begin{equation}
    \hat{\phi} = \hat{\phi}_1 + \ii \hat{\phi}_2,
\end{equation}
其中$\hat{\phi}_1$和$\hat{\phi}_2$分别是两个厄米算符。
对应的,描述它的拉氏量当中的场有实部和虚部,需要把它们——或者它们的线性组合——看成两个独立的场来列写\eqref{eq:el-eq}。

此外,$\mathcal{H}$中各项阶数的限制还意味着由此导出的运动方程在时间上只能是一阶的。
从而,$\pi$和$\phi$不是彼此独立的。这样,在哈密顿量只关于$\phi$和$\pi$时,我们总是可以适当地调节拉氏量和哈密顿量,或者对$\phi$和$\pi$做一些线性变换,使得$\pi$和$\phi$之间有线性关系。
这个关系显然不能是“乘以某个倍数”,否则将不能够区分这两个变量。
因此两者之间的关系涉及复共轭。通常取
\begin{equation}
    \pi = \ii \phi^\dagger.
\end{equation}
这也表明了取$\phi$为复场的重要性——否则将不能够区分$\phi$和$\pi$,从而难以建立哈密顿动力学。
% TODO:这段还是有问题

需要注意的是,无论是\eqref{eq:symmetry-commutator}还是\eqref{eq:antisymmetry-commutator},实际上都假定了$\phi^i$和$\pi_i$在时空变换下是协变的。
在场具有某些附加结构——例如,有某些外加约束以消除非物理的自由度——的时候,如果我们直接把独立的自由度拿出来写成$\phi^i$,就不能保证它们的协变性(虽然把原来的场恢复出来之后它仍是协变的),此时不能直接套用\eqref{eq:symmetry-commutator}或\eqref{eq:antisymmetry-commutator},而需要使用带约束的场论的有关知识。

此外,虽然本节通过表明指定对易子或者反对易子能够得到经典哈密顿动力学来论证量子化方案\eqref{eq:symmetry-commutator}和\eqref{eq:antisymmetry-commutator}的合理性,但是实际上这两个方案在本文展示的经典哈密顿动力学以外仍然适用。例如,如果哈氏量中出现了广义动量的导数,那么\eqref{eq:hamitonian-eq}需要做出修正,但是\eqref{eq:hamitonian-eq-quantum}仍然适用。换而言之,本节展示的量子动力学是物理上更加根本的理论,而经典哈密顿动力学只是它的某种“影子”或者特例。
很多量子系统很难找到任何经典对应,而即使有经典对应,也可以展现出完全不同的行为。

从正则量子化得到的算符运动方程就是经典的场运动方程算符化的结果,而后者又等价于通过最小作用量原理求出的运动方程。
这就产生了一个问题:路径积分量子化告诉我们,最小作用量原理只是路径积分的最速下降近似而已,
为什么在正则量子化中精确的运动方程却可以从最小作用量原理求出?
其原因在于,算符在演化过程中不同的本征态会混在一起(一个经典情况下不可能出现的现象),正是这一点构成了量子和经典的区别,
正则量子化中的本征态混合正好对应于路径积分量子化中非经典的路径。

\subsubsection{时间演化和绘景}\label{sec:time-evolution}

在\autoref{sec:canonical-quantization}中我们仅仅将态矢量当成一个可以让场算符作用上去的对象。
但实际上如果我们想要的话,也可以让态矢量动起来而对算符做对应的修改,使得算符的谱结构始终不变(本征矢量重数一一对应、彼此对应的本征矢的内积相同),并且本征值不变。
只要算符的谱结构不变、对应的各个本征值不变,算符就正确地描述了系统。
两个算符的谱结构一致、对应的本征值相同的充要条件是它们酉相似(相似矩阵可以随时间变化)。
需要注意的是两个描述了同一个系统的算符会给出不同的基矢量,所以切换绘景的时候还需要改变态矢量。
综上,绘景变换公式为
\begin{equation}
    \hat{A}' = \hat{Q} \hat{A} \hat{Q}^\dagger, \quad \ket{\psi'} = \hat{Q} \ket{\psi},
    \label{eq:picture-trans}
\end{equation}
其中$\hat{Q}$为一个幺正算符,它可以显含时间。
对易子在绘景变换之下会发生下面的改变:
\begin{equation}
    \comm{\hat{A}}{\hat{B}} \longrightarrow \hat{Q} \comm{A}{B} \hat{Q}^\dagger = \comm{\hat{A}'}{\hat{B}'}.
\end{equation}
最为平凡的绘景变换显然是保持算符不变而让态矢量乘上一个模长为$1$的复数。

在\autoref{sec:canonical-quantization}中我们已经讨论了态矢量固定不动时怎么确定系统的动力学。
这种让态矢量固定、算符变动的方案称为\textbf{海森堡绘景}。以下我们使用上标$H$代表海森堡绘景下的量。
我们要证明的第一件事是,不同时间点上的同一个可观察量的值彼此酉相似。
要看清楚这是为什么,我们将酉相似的方程
\begin{equation}
    \hat{A}^H (t) = \hat{U}^H(t, t_0) \hat{A}^H (t_0) (\hat{U}^H)^\dagger(t, t_0)
    \label{eq:quantum-evolution-hes-u-operator}
\end{equation}
做一个等价变换,看看它等价于什么。%
\eqref{eq:quantum-evolution-hes-u-operator}中的$U$在$t=t_0$时必定为恒等变换,因为此时$\hat{A}^H (t) = \hat{A}^H (t_0)$;同时容易看出$\hat{U}^H(\tau)$实际上构成一个李群。这样我们就能够写出其生成元,记之为$\hat{G}(t)$:
\[
    \hat{U}^H(t+\dd{t}, t) = \hat{I} + \frac{\ii}{\hbar} \hat{G}(t) \dd{t}.
\]
$\hat{U}^H$是幺正的等价于$\hat{G}$是厄米的。
于是就能够写出\eqref{eq:quantum-evolution-hes-u-operator}的无穷小等价形式:
\[
    \hat{A}^H (t_0) + \dd{\hat{A}^H}(t_0) = \left( \hat{I} + \frac{\ii}{\hbar} \hat{G}(t) \dd{t} \right) \hat{A}^H (t_0) \left( \hat{I} - \frac{\ii}{\hbar} \hat{G}(t) \right) = \hat{A}^H (t_0) + \frac{\dd{t}}{\ii \hbar} \comm{\hat{A}^H}{\hat{G}(t)}.
\]
我们发现这就是\eqref{eq:quantum-evolution},只需要把$\hat{G}(t)$换成$\hat{H}(t)$;并且正则量子化的时候已经要求$\hat{H}$是厄米的了,因此$\hat{G}$的确是厄米的,从而$\hat{U}^H$是幺正的。
于是我们得出结论:海森堡绘景中的算符演化实际上是在做幺正变换,或者等价地说,海森堡绘景中各算符的本征态随着时间演化在做幺正变换。%
\footnote{
    这和“态没有时间演化”不矛盾。这实际上是在说,我们在不同时间点,参考某个算符的本征态构造出来的态是不一样的,但是如果已经构造出来了这样的一个态,那么把它放置一段时间,它并不会变化。
}%
算符的变换式为\eqref{eq:quantum-evolution-hes-u-operator},相应的,本征态的变换式为
\begin{equation}
    \ket{a(t)} = \hat{U}^H(t, t_0) \ket{a(t_0)}.
\end{equation}
于是我们称$\hat{U}^H$为海森堡绘景下的时间演化算符。
$\hat{U}^H$可以写出显式表达式
\begin{equation}
    \hat{U}^H(t, t_0) = T \exp \left( \frac{\ii}{\hbar} \int_{t_0}^t \dd{t} \hat{H}^H (t) \right).
\end{equation}
注意\eqref{eq:quantum-evolution-hes-u-operator}保证了,一个可观察量在经过时间演化之后仍然是可观察量。

现在我们尝试使用\eqref{eq:picture-trans}来把时间演化完全转移到态矢量上面。
因此,我们希望在新的绘景中,$\hat{A}$始终不变。我们称这新的绘景为\textbf{薛定谔绘景}。
按照\eqref{eq:quantum-evolution-hes-u-operator},有
\[
    \hat{A}^H(t) = \hat{U}^H(t, t_0) \hat{A}^H (t_0) (\hat{U}^H)^\dagger(t, t_0) = \hat{U}^H(t, t_0) \hat{A}^S( \hat{U}^H)^\dagger(t, t_0),
\]
不失一般性地我们取$t=0$时的$\hat{A}^H$为$\hat{A}^S$,那么我们有
\[
    \hat{A}^H (t) = \hat{U}^H(t, 0) \hat{A}^S( \hat{U}^H)^\dagger(t, 0).
\]
将这个方程和\eqref{eq:picture-trans}对比可以看出
\[
    \hat{Q} = (\hat{U}^H)^\dagger(t, 0),
\]
于是得到薛定谔绘景下的态矢量演化公式
\[
    \ket{\psi^S(t)} = \hat{Q} \ket{\psi^H} = (\hat{U}^H)^\dagger (t, 0) \ket{\text{a constant}},
\]
考虑到$t=0$时$\hat{U}^H (t, 0)$就是恒等算符,上式又等价于
\[
    \ket{\psi^S(t)} = (\hat{U}^H)^\dagger (t, 0) \ket{\psi^S (0)},
\]
也即,薛定谔绘景下的时间演化算符和海森堡绘景下的时间演化算符互为逆。
这个方程还告诉我们,
\[
    \ket{\psi^H} = \ket{\psi^S(t_0)}.
\]
现在推导时间演化方程的微分形式。我们有
\[
    \begin{aligned}
        \ket{\psi^S (t + \dd{t})} &= \left( \hat{U}^H (t + \dd{t}, t) \hat{U}^H (t, 0)  \right)^\dagger \ket{\psi^S(0)} \\
        &= \left( (\hat{I} + \frac{\ii}{\hbar} \hat{H}(t) \dd{t})   \hat{U}^H (t, 0) \right)^\dagger \ket{\psi^S (0)} \\
        &= (\hat{U}^H)^\dagger (t, 0) \ket{\psi^S (0)} + \frac{\dd{t}}{\ii \hbar} (\hat{U}^H)^\dagger (t, 0) \hat{H}(t) \ket{\psi^S (0)} \\
        &= \ket{\psi^S (t)} + \frac{\dd{t}}{\ii \hbar} (\hat{U}^H)^\dagger (t, 0) \hat{H}(t) \hat{U}^H (t, 0) \ket{\psi^S (t)},
    \end{aligned}
\]
从而
\[
    \ii \hbar \dv{t} \ket{\psi^S (t)} = (\hat{U}^H)^\dagger (t, 0) \hat{H}(t) \hat{U}^H (t, 0) \ket{\psi^S (t)}.
\]
为了方便区分,我们将海森堡绘景中的$\hat{H}$记作$\hat{H}^H$,则它对应的薛定谔绘景中的算符为
\[
    \hat{H}^S = \hat{Q} \hat{H}^H \hat{Q}^\dagger = (\hat{U}^H)^\dagger (t, 0) \hat{H}^H(t) \hat{U}^H (t, 0), 
\]
这正是薛定谔绘景中态矢量的运动方程中出现的那个量,因此就获得了薛定谔绘景中的运动方程:
\begin{equation}
    \ii \hbar \dv{t} \ket{\psi^S(t)} = \hat{H}^S (t) \ket{\psi^S(t)}.
\end{equation}
% TODO:证明$\hat{H}^I_i$确实是$\hat{H}_i^H$在相互作用绘景下的
设$\hat{U}^S(t, t_0)$是薛定谔绘景下的时间演化算符,则容易证明$\hat{H}^S$是它的生成元,既然$\hat{H}^H$是厄米的,$\hat{H}^S$也是厄米的,从而$\hat{U}^S$是幺正的。%
\footnote{注意$\hat{H}^H$是$\hat{U}^H$的生成元而$\hat{H}^S$是$(\hat{U}^H)^\dagger$的生成元;由于$\hat{H}^H$可能含时,一般情况下
\[
    T \exp(\int \hat{H}^H (t) \dd{t})^\dagger \neq T \exp(- \int \hat{H}^H (t) \dd{t}),
\]
也就是说$\hat{H}^H$和$\hat{H}^S$之间没有简单的关系,而必须使用绘景变换公式联系两者。
}
因此薛定谔绘景中时间演化始终保持态矢量的幺正性。
时间演化算符的显式表达式为
\begin{equation}
    \hat{U}^S(t, t_0) = T \exp \left( - \frac{\ii}{\hbar} \int_{t_0}^t \dd{t} \hat{H}^S(t) \right),
\end{equation}
其中$T$为编时算符。

为了明显起见,我们将薛定谔绘景和海森堡绘景中哈密顿量相互换算的关系重复如下:
\begin{equation}
    \begin{aligned}
        \hat{H}^H(t) = T \exp \left( \frac{\ii}{\hbar} \int_{t_0}^t \dd{t} \hat{H}^H(t) \right) \hat{H}^S(t_0) \left(T \exp \left( \frac{\ii}{\hbar} \int_{t_0}^t \dd{t} \hat{H}^H(t) \right)\right)^\dagger, \\
        \hat{H}^S(t_0) = T \exp \left( - \frac{\ii}{\hbar} \int_{t_0}^t \dd{t} \hat{H}^S \right) \hat{H}^H(t) \left( T \exp \left( - \frac{\ii}{\hbar} \int_{t_0}^t \dd{t} \hat{H}^S \right)\right)^\dagger,
    \end{aligned}
\end{equation}
在$\hat{H}^H$在各个时间点的值彼此对易时,$\hat{U}^H$无非是$\hat{H}^H$的级数,因此它们对易,从而$\hat{H}^S$和$\hat{H}^H$相等。
这也等价于$\hat{H}^S$在各个时间点的值彼此对易。

事实上,虽然我们是从海森堡绘景出发建立我们的理论框架的,但\autoref{sec:back-to-classical}告诉我们,和经典力学中的系统状态直接对应的实际上就是态矢量,而不是算符,因此很多文献是从薛定谔绘景出发建立理论的。

% TODO:这里有些地方写得是有问题的。标准的相互作用绘景应该是薛定谔绘景的推论。但这个也奇怪得很:量子场论中的微扰论难道是使用薛定谔绘景的吗??
现在我们已经讨论了“让可观察量变动”和让基矢量变动“两种方案的不同了。我们还可以把哈密顿算符分解成一个比较简单的不含时部分和一个含时的部分,并要求这两者均为厄米算符,然后分别用两者让算符和态矢量都动起来。这样的方案称为\textbf{相互作用绘景}。
为方便起见,考虑从薛定谔绘景到相互作用绘景的变换。当然也可以从海森堡绘景出发推导相互作用绘景,但实际上这样会很不自然。对薛定谔绘景下的哈密顿量做分解
\begin{equation}
    \hat{H}^S = \hat{H}_0^S + \hat{H}_i^S,
\end{equation}
称前者为\textbf{自由哈密顿量}(通常我们要求它不显含时间),后者为\textbf{相互作用哈密顿量},并指定
\begin{equation}
    \ket{\psi^I(t)} = \hat{U}_0^\dagger(t,t_0) \ket{\psi^S(t)},
\end{equation}
其中
\begin{equation}
    \hat{U}_0 = T \exp \left( - \frac{\ii}{\hbar} \int_{t_0}^t \dd{t} \hat{H}_0^S(t) \right).
\end{equation}
于是可观察量的绘景变换为
\begin{equation}
    \hat{A}^I(t) = \hat{U}_0^\dagger(t,t_0) \hat{A}^S(t) \hat{U}_0(t,t_0).
    \label{eq:operator-from-schodinger-to-interaction}
\end{equation}
通过求导,分别可以计算出态矢量和可观察量的时间演化方程为
\begin{equation}
    \ii \hbar \dv{t} \ket{\psi^I(t)} = \hat{H}^I_i(t) \ket{\psi^I(t)},
    \label{eq:time-evolution-in-interation-picture}
\end{equation}
以及
\begin{equation}
    \dv{t} \hat{A}^I(t) = \frac{1}{\ii \hbar} \comm*{\hat{A}^I(t)}{\hat{H}_0^I}.
\end{equation}
其中$\hat{H}_0^I$和$\hat{H}_i^I$正是对$\hat{H}_0^S$和$\hat{H}_i^S$做绘景变换\eqref{eq:operator-from-schodinger-to-interaction}得到的结果。
这样我们就成功地让时间演化分别由态矢量和可观察量各自承担一部分。

如果我们在海森堡绘景中工作,要怎么样切换到相互作用绘景中呢?最一般的公式非常复杂。
但是,实际上,如果哈密顿量含时,通常直接在薛定谔绘景中工作;如果哈密顿量不含时,那么薛定谔绘景和海森堡绘景下的哈密顿量是一样的,那么只需要选择一个较简单的可观察量$\hat{H}_0$,指定它为$\hat{H}_0^S$,就可以切换到相互作用绘景。需注意整个过程并没有用到$\hat{H}_0^H$,一般来说,它和$\hat{H}_0^S$可能会有区别,但是我们从来不关注这个区别。

相互作用绘景在微扰量子场论计算中起到了非常重要的作用,因为通过对称性分析可以直接得到自由场的哈密顿量密度和演化方程,因此我们可以将相互作用项——也就是不同场之间的耦合——独立考虑,从而大大简化计算。
更加重要的是,此时相互作用绘景可以为我们提供有关量子场的态空间的结构的信息。如果假定态空间中有一个唯一的真空态——也就是所有场都是零的态——那么量子场的态空间就是多粒子态福克空间,在此基础上我们可以很自然地处理粒子创生和湮灭的过程。这就为我们展示了量子理论的另一面:波动看起来就像粒子一样。%
\footnote{需要注意的是,在处理相对论性量子场论的时候其实并不能完全放心地使用相互作用绘景。如果我们取$\hat{H}_i=0$,那么相互作用绘景就退化为了自由场的海森堡绘景;这样我们就看到了$\hat{H}_i$项的作用:它把带相互作用的场的态(也就是$\ket{\psi^I(t)}$)和自由场的态($\ket{\psi^I(0)}$,因为如果$\hat{H}_i=0$那么态就不会变化)使用一个幺正算符联系了起来,而且这个幺正算符是唯一的。然而Haag定理说,含相互作用的场有无数个不等价的幺正表示,因此我们并不能唯一地将带相互作用的场的态和自由场的态使用一个唯一的幺正算符联系起来。特别的,由于我们要求自由场和相互作用场的态空间都满足一定的物理条件(如有稳定的真空态,等等),自由场的态空间和相互作用场的满足这些条件的态空间一般来说并不幺正等价。这意味着类似于$\int \dd{t} \hat{H}^H_i$之类的表达式实际上并不收敛,于是相互作用绘景就失效了。但是有很多手段可以绕过这个定理的限制——例如因为我们从来只讨论一定能标下的物理现象而不把相对论性量子场论当成终极理论,实际上我们可以把空间格点化,这样量子场论就变成了有限自由度的量子力学,于是就可以使用相互作用绘景了。}

此外容易验证,各种形式的时间演化算符都满足以下公式:
\begin{equation}
    \hat{U}(t_3,t_2) \hat{U}(t_2,t_1) = \hat{U}(t_3,t_1),
\end{equation}
以及
\begin{equation}
    \hat{U}^\dagger (t_2, t_1) = \hat{U} (t_1, t_2).
\end{equation}

\subsubsection{测量}\label{sec:measure}

\textbf{测量}指的是这样一个过程:两个系统(分别称为\textbf{待测系统}和\textbf{仪器})发生相对剧烈而时间短促的相互作用,相互作用后待测系统的态发生很大改变,而仪器的态则体现了相互作用前待测系统的某些信息。
我们为了方便起见将主要分析粒子理论的测量,场论中的测量要复杂很多(可想而知),因为测量显然只能一次发生在一个空间点,因此测量导致的扰动也许需要以某个有限的速度传播出去。
采用相互作用绘景,设$\hat{q}$完全描述了仪器的态空间,$\hat{a}$是关于待测系统的某个算符,它和另一个算符$\hat{b}$共同描述了待测系统的态空间。(被测量的量$\hat{a}$未必能够完整描述待测系统。下文中需要将待测系统的态做展开,因此引入$\hat{b}$)
由于相互作用非常剧烈而时间短促,仪器和待测系统的相互作用哈密顿量可以写成
\begin{equation}
    H_\text{int} = - \gamma(t-t_0) \hat{a} \otimes \hat{p},
\end{equation}
% 为什么偏偏就是这个形式?为什么所有量都是一次项?
其中$\hat{p}$是$\hat{q}$对应的共轭动量,也就是说
\[
    \comm*{\hat{q}}{\hat{p}} = \ii \hbar,
\]
$\gamma$是一个函数,它是一个$t_0$附近的尖峰。
极限情况下,$\gamma(t) = g \delta(t)$,这称为\textbf{冯诺依曼测量}或者\textbf{标准量子测量},
我们在相互作用绘景下分析问题。系统初态为
\[
    \ket{i} = \ket{\psi_i} \ket{D} = \int \dd{q} \sum_{k, n} \braket{q}{D} \braket{a_k, b_n}{\psi_i} \ket{a_k} \ket{b_n} \ket{q},
\]
其中$\ket{\psi_i}$和$\ket{D}$分别为待测系统和仪器的初态,本征态$\ket{a}_k$,$\ket{b}_n$和$\ket{q}$是$t_0$时刻对应算符的本征态(下同)。%
\footnote{提醒:算符本征态反映的是算符的代数结构,它们的时间演化是由自由哈密顿量而不是相互作用哈密顿量指导的。}%
我们要求$\hat{q}$是连续谱,而$\hat{a}$和$\hat{b}$可以是离散谱也可以是连续谱。要求$\hat{q}$是连续谱的原因很快就可以看到。
系统的末态为
\[
    \begin{aligned}
        \ket{f} &= T \exp \left( - \frac{\ii}{\hbar} \int \dd{t} H_\text{int} \right) \ket{i} \\
        &= T \exp \left( \frac{\ii}{\hbar} g \int \dd{t} \delta(t-t_0) \hat{a}(t) \otimes \hat{p}(t) \right) \ket{i} \\
        &= \exp \left( \frac{\ii}{\hbar} g \hat{a}(t_0) \otimes \hat{p}(t_0) \right) \ket{i} \\
        &= \sum_{n=0}^\infty \frac{1}{n!} \left(\frac{\ii}{\hbar} g\right)^n \hat{a}(t_0)^n \hat{p}(t_0)^n \int \dd{q} \sum_{k, l} \braket{q}{D} \braket{a_k, b_l}{\psi_i} \ket{a_k} \ket{b_l} \ket{q} \\
        &= \int \dd{q} \sum_{k, l} \braket{q}{D} \braket{a_k, b_l}{\psi_i} \sum_{n=0}^\infty \frac{1}{n!} \left(\frac{\ii}{\hbar} g\right)^n \hat{a}(t_0)^n \ket{a_k} \ket{b_l} \hat{p}(t_0)^n \ket{q} \\
        &= \int \dd{q} \sum_{k, l} \braket{q}{D} \braket{a_k, b_l}{\psi_i} \sum_{n=0}^\infty \frac{1}{n!} \left(\frac{\ii}{\hbar} g\right)^n a_k^n \ket{a_k} \ket{b_l} \hat{p}(t_0)^n \ket{q} \\
        &= \int \dd{q} \sum_{k, l} \braket{q}{D} \braket{a_k, b_l}{\psi_i} \ket{a_k} \ket{b_l} \sum_{n=0}^\infty \frac{1}{n!} \left(\frac{\ii}{\hbar} g a_k \hat{p}(t_0) \right)^n \ket{q} \\
        &= \int \dd{q} \sum_{k, l} \braket{q}{D} \braket{a_k, b_l}{\psi_i} \ket{a_k} \ket{b_l} \exp \left( \frac{\ii}{\hbar} g a_k \hat{p}(t_0) \right) \ket{q} \\
        &= \int \dd{q} \sum_{k, l} \braket{q}{D} \braket{a_k, b_l}{\psi_i} \ket{a_k} \ket{b_l} \ket{q + g a_k} \\
        &= \int \dd{q} \sum_{k, l} \braket{q - g a_k}{D} \braket{a_k, b_l}{\psi_i} \ket{a_k} \ket{b_l} \ket{q} .
    \end{aligned}
\]
总之我们得到经典测量前后态的变化公式
\begin{equation}
    \ket{f} = \int \dd{q} \sum_{k, l} \braket{q - g a_k}{D} \braket{a_k, b_l}{\psi_i} \ket{a_k} \ket{b_l} \ket{q}.
    \label{eq:standard-measurement}
\end{equation}
需要注意的是由于我们采取的是相互作用绘景,算符$\hat{a}$和$\hat{b}$一直会发生变化。
然而,由于自由哈密顿量不显含时间,\eqref{eq:standard-measurement}中$\ket{a_k, b_l}$的时间演化和$\bra{a_k, b_l}$的时间演化抵消了,等等,从而$\ket{f}$在相互作用结束后没有时间演化——正如我们预期的那样,因为相互作用结束之后相互作用哈密顿量就是零。

\eqref{eq:standard-measurement}看起来仍然十分复杂。
然而,在很多情况下(具体是什么情况我们很快会看到)仪器的初始态非常接近$\hat{q}$的本征态,也就是说$\braket{q}{D}$只有在$q$和某一个$q_0$非常接近的时候才有较大的值,其余时候都接近零,因此实际上是一个$\delta$函数。
这样的情况称为\textbf{理想测量}。我们现在可以看到为什么要求$\hat{q}$具有连续谱了,因为要实施一次理想测量必须允许仪器有连续分布的状态。此时\eqref{eq:standard-measurement}近似为
\begin{equation}
    \ket{f} = \sum_{k, l} \braket{a_k, b_l}{\psi_i} \ket{a_k} \ket{b_l} \ket{q = q_0 + g a_k}.
    \label{eq:ideal-measurement}
\end{equation}
我们这样就得到了一个典型的纠缠态,其中每一个分量中,仪器和待测系统在测量之后都处于完全对应的状态。
总之,如果待测系统和仪器组成的系统和外界毫无相互作用,那么测量就是如下所示的过程:
\[
    \ket{i} = \left(\sum_{k, l} \braket{a_k, b_l}{\psi_i} \ket{a_k} \ket{b_l} \right) \ket{D} \longrightarrow \ket{f} = \sum_{k, l} \braket{a_k, b_l}{\psi_i} \ket{a_k} \ket{b_l} \ket{q = q_0 + g a_k},
\]
也就是待测系统将其信息复制到了仪器当中。
然而,假如仪器足够大,那么待测系统和仪器组成的系统和外界将会有大量的相互作用。
例如,仪器可能被放置在灯光下来方便我们读取其示数,这就意味着它要不停地受到四面八方的光子的轰击。
这就意味着\eqref{eq:ideal-measurement}会很快发生退相干,最后终结于$\hat{a} \otimes \hat{b} \otimes \hat{q}$的某个本征态上,因此最后仪器停留在某个$q=q_0 + g a_k$附近,且待测系统的态也转化为$\ket{a_k}$。
将待测系统和仪器组成的系统以及所有可能的环境变量放在一起就得到了一个系综;系综中,待测系统和仪器组成的系统在退相干之后停留在本征态$\ket{a_k} \ket{b_l} \ket{q = q_0 + g a_k}$的概率正是$\abs{\braket{a_k, b_l}{\psi_i}}^2$,
也就是说,在时刻$t$测量$\hat{a}$得到$a_k$(同时将待测系统的态转化为$\ket{a_k}$)的概率就是
\begin{equation}
    P_t(a_k) = \sum_l \abs{\braket{a_k, b_l(t)}{\psi_i}}^2,
    \label{eq:probablity-of-measurement}
\end{equation}
由\eqref{eq:probablity-of-measurement}出发容易证明,待测系统为$\ket{\psi_i}$态时做测量,测量值的期望为
\begin{equation}
    \expval{\hat{a}}(t) = \mel{\psi_i}{\hat{a}(t)}{\psi_i}.
\end{equation}

实际上,我们可以把四面八方的光子或者空气分子或者这一类的干扰看成是一个巨型仪器:它和待测系统的相互作用使待测系统和它的态按照\eqref{eq:ideal-measurement}纠缠在一起,而由于这是开放体系,退相干快速发生,这就意味着在充满干扰的环境中实际上很难真的展示出待测系统的量子特性:待测系统几乎总是出现在其偏好本征态附近,因为它没完没了地受到测量。
这也是理想测量很容易就能够实现的原因:真的会用来做测量的仪器总是被做得很大,因此它们自身可以看成不停地被空气、杂散光或者别的什么东西不断测量的系统,因此它们的态总是出现在其偏好本征态附近。

在$\hat{a}$本身是待测系统的一个CSCO,从而不需要$\hat{b}$的情况下,测量$\hat{a}$得到$a_k$的概率为
\begin{equation}
    P(a_k) = \abs{\braket{a_k}{\psi_i}}^2.
\end{equation}
这表明,假如我们有一个正交归一化基$\{\ket{a_k}\}_k$,就可以使用一组不同的实数$a_k$构造算符
\[
    \hat{a} = \sum_k a_k \dyad{a_k},
\]
使用这个算符对系统做测量,则测量结束之后系统位于态$\ket{a_k}$的概率就是
\begin{equation}
    P(\ket{a_k}) = \abs{\braket{a_k}{\psi_i}}^2.
\end{equation}
注意到这个表达式只和$\ket{a_k}$有关。因此,对态矢量为$\ket{\psi}$的系统做一次测量,发现系统测量后处于态$\ket{\phi}$的概率为
\begin{equation}
    P(\ket{\phi}) = \abs{\braket{\phi}{\psi}}^2,
\end{equation}
于是我们称$\braket{\phi}{\psi}$为\textbf{概率振幅}。

需注意以上讨论建立在几个关键假设上:其一,仪器和待测系统的相互作用非常强而短促;其二,仪器和环境有杂乱无章的相互作用。
这意味着合理地构造不怎么受外界干扰而又不会严重地扰动待测系统的仪器,我们就能够得到关于待测系统状态的不完整信息而与此同时不让待测系统的态塌缩到某个本征态上。
这称为\textbf{弱测量}。

在结束本节前我们需要指出量子力学中测量的步骤引出的一些术语混淆。
“可观察量”一词在前述所有部分中都用来指厄米算符,但是,相当合理的,它当然也可以用来描述这些厄米算符的期望值,或者它们的期望值的函数(因为我们总是可以涉及一个测量过程获得这些期望值和它们的函数)。
同样,“物理量”一词在前文中均表示算符,但是它们也可以表示普通的数。
实际上,在引入统计物理的概念之后,温度等统计量甚至会带来更多的混乱。
在可能引起混淆的地方,需要注意区分“算符”和普通的“变量”,“物理量”和它们的值,如算符的本征值,经典变量的可能取值,算符的期望值,等等。

\subsubsection{有效哈密顿量}

设有哈密顿量$\hat{H}$,如果某个物理量$\hat{A}$满足
\[
    \comm*{\hat{H}}{\hat{A}} = \comm*{\hat{H}'(A)}{\hat{A}},
\]
其中$\hat{H}'$是只和$\hat{A}$有关的物理量,那么在只关心$\hat{A}$时可以以$\hat{H}'$为有效哈密顿量。

其实也可以通过路径积分的观点看这个问题:与$\hat{A}$对易的那些自由度可以直接积掉而只留下一个因子。

有时,一个物理系统的哈密顿量涉及大量复杂的过程,而特定的初始条件意味着这个系统的态基本上只会出现在态空间的一小部分当中。
但这并不意味着态空间的其它部分就不会对系统的动力学造成影响。
例如,设想一个三能级系统,其中一个能级的能量远远高于另外两个能级,这意味着系统基本上不可能出现在这个能级上,但如果其余两个能级和这个高能量能级有耦合,那么这个高能量能级就可能成为另外两个能级相互转换的渠道。

设投影算符$\hat{P}$选择出了我们关注的那部分态空间,而且这部分态空间的定义不随时间变化而变化;设$\hat{Q}$是与之互补的投影算符,则
\[
    \hat{P} + \hat{Q} = 1, \quad \hat{P}^2 = \hat{P}, \quad \hat{Q}^2 = \hat{Q}.
\]
考虑薛定谔绘景,运动方程为
\[
    \hat{H} \ket{\psi} = \ii \hbar \dv{t} \ket{\psi},
\]
将投影算符作用于其上得到
\[
    \begin{aligned}
        \hat{P} \hat{H}(\hat{P}+\hat{Q}) \ket{\psi} = \ii \hbar \dv{t} \hat{P} \ket{\psi}, \\
        \hat{Q} \hat{H}(\hat{P}+\hat{Q}) \ket{\psi} = \ii \hbar \dv{t} \hat{Q} \ket{\psi}.
    \end{aligned}
\]
哈密顿量可以分成四部分,一部分完全位于$\hat{P}$筛选出来的空间中,一部分完全位于$\hat{Q}$筛选出来的空间中,另外两部分从其中一个空间跳跃到另一个空间,这四部分分别是
\[
    \hat{H}_{PP} = \hat{P} \hat{H} \hat{P}, \quad \hat{H}_{QQ} = \hat{Q} \hat{H} \hat{Q}, \quad \hat{H}_{PQ} = \hat{P} \hat{H} \hat{Q}, \quad \hat{H}_{QP} = \hat{Q} \hat{H} \hat{P}.
\]
使用投影算符的性质可以写出
\[
    \begin{aligned}
        \hat{H}_{PP} \hat{P} \ket{\psi} + \hat{H}_{PQ} \hat{Q} \ket{\psi} = \ii \hbar \dv{t} \hat{P} \ket{\psi}, \\
        \hat{H}_{QP} \hat{P} \ket{\psi} + \hat{H}_{QQ} \hat{Q} \ket{\psi} = \ii \hbar \dv{t} \hat{Q} \ket{\psi},
    \end{aligned}
\]
从后一个方程可以解出
\[
    \hat{Q} \ket{\psi} = \frac{1}{\ii \hbar \dv{t} - \hat{H}_{QQ}} \hat{H}_{QP} \hat{P} \ket{\psi},
\]
代入前一个方程就得到
\[
    \ii \hbar \dv{t} \hat{P} \ket{\psi} = \left(\hat{H}_{PP} + \hat{H}_{PQ} \frac{1}{\ii \hbar \dv{t} - \hat{H}_{QQ}} \hat{H}_{QP}\right) \hat{P} \ket{\psi}.
\]
因此我们发现,我们关注的那一部分态的时间演化由有效哈密顿量
\begin{equation}
    \hat{H}_\text{eff} = \hat{H}_{PP} + \hat{H}_{PQ} \frac{1}{\ii \hbar \dv{t} - \hat{H}_{QQ}} \hat{H}_{QP}
    \label{eq:effective-hamiltonian-original}
\end{equation}
指导,而且由$\hat{H}_{PP}, \hat{H}_{PQ}, \hat{H}_{QP}$的定义,该有效哈密顿量是$\hat{P}$筛选出的空间中的算符。
\eqref{eq:effective-hamiltonian-original}非常符合我们的直觉:时间演化可以仅仅涉及$\hat{H}_{PP}$,也可以以$\hat{H}_{QQ}$为中介。

一种特殊的情况是,态空间可以写成两个空间(记为$\mathcal{H}_1$和$\mathcal{H}_2$)的直积,系统的初始条件决定了大部分有意义的过程都发生在$\mathcal{H}_1$中,但由于耦合,不能简单地将$\mathcal{H}_2$排除掉。
这时可以构造算符$\hat{P}$使之筛选出只在$\mathcal{H}_1$中有显著活动的态,计算出有效哈密顿量;$\hat{P}$筛选出的态均形如$\ket{\psi}_1 \otimes \ket{0}_2$,由于有效哈密顿量仅涉及$\ket{\psi}_1$,不会出现两个空间之间的耦合,于是可以直接将$\mathcal{H}_2$去掉,使用$\mathcal{H}_1$和$\hat{H}_\text{eff}$来描述系统。

然而,\eqref{eq:effective-hamiltonian-original}显含一个时间求导算符的倒数,这意味着$\hat{H}_\text{eff}$实际上显含时间,而且还显含关于时间的算符,也即我们实际上是手动把关于$\mathcal{H}_2$的时间演化放进了有效哈密顿量当中,这是不便计算的。
为了让有效哈密顿量看起来像一个正常的哈密顿量,设我们考虑的过程的能量近似在$E_r$水平上,则对$\mathcal{H}_1$空间中的态,近似有
\[
    \ii \hbar \dv{t} \sim E_r,
\]
于是
\[
    \hat{H}_\text{eff} \sim \hat{H}_{PP} + \hat{H}_{PQ} \frac{1}{E_r - \hat{H}_{QQ}} \hat{H}_{QP}.
\]
对$\mathcal{H}_1$中$\hat{H}$的本征态而言,上式严格成立,我们得到自洽方程
\begin{equation}
    \left( \hat{H}_{PP} + \hat{H}_{PQ} \frac{1}{E - \hat{H}_{QQ}} \hat{H}_{QP} \right) \ket{\psi} = E \ket{\psi}.
\end{equation}
从这个方程求解出$E$,我们就得到了$\hat{H}_\text{eff}$在$\mathcal{H}_1$的一组基上的作用结果,于是也就完全确定下了$\hat{H}_\text{eff}$。
换而言之,完全精确求解的有效哈密顿量保留了原哈密顿量在我们关注的空间上的全部能谱。

然而,即使上述自洽方程也难以求解。为此通常使用微扰展开的方法。
设原哈密顿量中$\mathcal{H}_1$与$\mathcal{H}_2$没有耦合的部分为$\hat{H}_0$,其余部分为$\hat{H}'$,也即,以$\mathcal{H}_1$和$\mathcal{H}_2$为子空间将算符做分块,则$\hat{H}_0$包含对角部分,$\hat{H}'$包含非对角部分,则
\[
    \hat{H}_\text{eff} \sim \hat{H}_{0} + \hat{H}'_{PQ} \frac{1}{E_r - \hat{H}_{QQ}} \hat{H}'_{QP}.
\]
如果$\hat{H}'$让

在高能自由度和低能自由度的耦合并不明显时,高能自由度的存在与否对$\mathcal{H}_1$中的能量本征态只有不大的影响,这时可以以原哈密顿量中仅包含低能自由度的部分的本征值和本征态为起点,以低能自由度和高能自由度的耦合以及高能自由度的哈密顿量为微扰,求解出$\hat{H}$在$\mathcal{H}_1$中的本征态$\{\ket{n}\}$和本征值$\{E_n\}$。由于是本征态,它们和高能自由度没有耦合,于是低能自由度的运动完全由
\begin{equation}
    \hat{H}_\text{eff} = \sum_{\ket{n} \in \mathcal{H}_1} E_n \dyad{n}
\end{equation}
确定,我们也就得到了有效哈密顿量。

\begin{equation}
    \mel{m}{\hat{H}_\text{eff}}{n} = E_m \delta_{mn} + \mel{m}{\hat{H}'}{n} + \frac{1}{2} \sum_{\text{$l$ in $\mathcal{H}_2$}} \left( \frac{\mel{m}{\hat{H}'}{l} \mel{l}{\hat{H}'}{n}}{E_m - E_l} + \frac{\mel{m}{\hat{H}'}{l} \mel{l}{\hat{H}'}{n}}{E_n - E_l} \right) + \cdots.
\end{equation}

另一种做实际计算的方法是,考虑$\hat{H}$的本征态$\ket{\psi_n}$,其能量为$E_n$,则我们有
\[
    \hat{H} \ket{\psi_n} = E_n \ket{\psi_n}.
\]
有效哈密顿量只需要指导$\mathcal{H}_1$中的态的运行即可,因此它需要满足
\[
    \hat{H}_\text{eff} \ket{\psi_n} = E_n \ket{\psi_n}, \quad \text{for $\ket{\psi_n} \in \mathcal{H}_1$}.
\]
由于$\hat{H}_\text{eff}$是$\mathcal{H}_1$中的算符,以上方程的展开式为
\begin{equation}
    \sum_{\text{span}\{\ket{m}\} = \mathcal{H}_1} (\mel{l}{\hat{H}_\text{eff}}{m} - E_n \delta_{lm}) \braket{m}{\psi_n} = 0.
    \label{eq:effective-hamiltonian-eq-unfolded}
\end{equation}
这里我们使用一组任意的正交归一化基底$\{\ket{m}\}$,它们未必就是哈密顿量的本征态。相应的,$\hat{H}$满足
\begin{equation}
    \sum_m (\mel{l}{\hat{H}}{m} - E_n \delta_{lm}) \braket{m}{\psi_n} = 0, \quad \text{for $\ket{l} \in \mathcal{H}_1$}.
    \label{eq:hamiltonian-eq-unfolded}
\end{equation}
显然,\eqref{eq:hamiltonian-eq-unfolded}必须能够推导出\eqref{eq:effective-hamiltonian-eq-unfolded},线性代数上的结论告诉我们,记那些张成$\mathcal{H}_1$的基矢量的编号组成的集合为$W$,其余基矢量的编号组成的集合为$U$,则有
\[
    \mel{l}{\hat{H}_\text{eff}}{m} = \mel{l}{\hat{H}}{m} - \sum_{\alpha \in U} \frac{\mel{l}{H}{\alpha} \mel{\alpha}{H}{m}} {D_\alpha^W} + \sum_{\alpha \neq \beta \in U} \frac{\mel{l}{H}{\alpha} \mel{\alpha}{H}{\beta} \mel{\beta}{H}{m}}{D^W_{\alpha \beta}} + \cdots,
\]
其中,记$S$为一系列编号组成的集合,单脚标的$D$函数定义为
\[
    D_\alpha^S = H_{\alpha \alpha} - E_n - \sum_{\beta \in U, \beta \notin S} \frac{\mel{\alpha}{H}{\beta} \mel{\beta}{H}{\alpha}}{D_\alpha^S} + \sum_{\beta \neq \gamma \in U, \; \beta, \gamma \notin S} \frac{\mel{l}{H}{\alpha} \mel{\alpha}{H}{\beta} \mel{\beta}{H}{m}}{D_{\alpha \beta}^S} + \cdots,
\]
多脚标的$D$函数递归定义为
\[
    D_{\alpha \beta}^S = D_\alpha^S D_{\beta}^{S, \alpha}, \quad D_{\alpha, \beta, \gamma}^S = D_{\alpha}^S D_{\beta}^{S, \alpha} D_{\gamma}^{S, \alpha, \beta}, \ldots
\]
% TODO:和通常使用的微扰论有何关系???

TODO:扩大和缩小希尔伯特空间。后者当然就是积掉高能自由度,前者通常来自把算符“拆分”成一些东西的乘积。

\subsubsection{对称性和守恒量}\label{sec:hamiltonian-symmetry}

如果某个数学对象$A$在某个操作$O$下保持不变,我们就说$A$具有$O$-对称性或者$O$-不变性。
我们已经看到拉格朗日动力学中对称性和守恒量之间的关系(诺特定理),而在哈密顿动力学中这实际上更加明显。
设$U$是一个对称群,则系统的动力学不变等价于系统的哈密顿量在$U$作用下不变。
设$U$在态空间上的表示(而不是作用在算符上的表示)的生成元为$\{G_i\}$,共有$n$个,则系统的哈密顿量不变等价于
\[
    \comm*{\hat{H}}{G_i} = 0,
\]
于是我们就得到了$n$个守恒量。

如果对称群$U$中的每个操作都是“先做操作$U_1$,再做操作$U_2$,……”这样形成的,那么$G_i$的形式就会不同——如果是连续对称群,那么每一个$G_i$就是任取每个$U_m$的生成元中的一个加起来得到的,而对离散对称群,每一个$G_i$则是任取每个$U_m$的生成元中的一个乘起来得到的。

系统的动力学具有的对称性未必为它可能的状态具有,例如我们的世界遵从的动力学规律是平移不变的,但是它的状态未必是。
这种现象就是\textbf{自发对称性破缺}。
光观察哈密顿量的形式,未必能够看出来系统会发生自发对称性破缺。

然而,系统的状态的对称性未必就和动力学毫无关系。
设总的希尔伯特空间为$H$,由于守恒量、拓扑性质(也可以看成特殊的守恒量,拓扑不变量在拓扑变换下守恒)等,满足某个特定约束条件的态组成的空间为$H'$,则$H / H'$标记了不同的守恒量、拓扑性质等的参数。
如果出于某些原因,一个状态$\ket{\psi}$不能演化到$U \ket{\psi}$,那么希尔伯特空间就分裂成几个分支:每个分支都不具备$U$对称性,$U$的作用是在不同分支之间切换。
此时如果我们写下关于某个分支的有效哈密顿量,这个有效哈密顿量就很可能不具备$U$对称性。
实际上就是在这里,对称性自发破缺真的展现出了有趣的行为:我们从一个对称的理论得到了一个不对称的理论!

很容易看出上面的步骤其实可以反过来:我们可以在系统中引入冗余的、非物理的自由度,让加入这个自由度之后的态空间分裂成(互相不能演化过去的)几个分支。
这种情况下通常不说这是“对称性”,而是说这是“冗余的自由度”。这种对称性只是哈密顿量或者拉氏量的对称;态空间中的这种“对称”是应该被模掉的。
例如,在场论中,\textbf{规范对称性}就是这样一种冗余性,它是作用在空间局域上的一种变换,变换前后系统状态实际上没有发生任何物理的变化。
当然,实际上,规范变换作用在系统的态的标签上而不是系统的态上(系统的态根本没有发生变化),因此它并不适合被称为对称性。\textbf{规范结构}指的就是将不同的标签映射到同样的态上的结构。
很多东西都可以看成规范结构,如在周期性系统中,物理的自由度其实是一个“环”,但是我们当然可以用直线上的坐标做系统的自由度,从而就需要一个“取余”的规范结构。
又比如全同粒子系统的对称化/反对称化其实就是在选取规范。

\subsection{关于单位制的注记}

到现在为止我们的理论还带有一些常数。用以标记我们的理论多大程度上偏离了经典情况的$\hbar$是一个重要的常数,同时标记了时间和空间的换算关系的光速$c$是另外一个。
通过做变换
\[
    t \longrightarrow t' = ct,
\]
我们可以让光速$c$从所有的公式中消失。相应的,时间导数算符发生了
\[
    \partial_t \longrightarrow \partial_{t'} = \frac{1}{c} \partial_t
\]
的变换。
$\hbar$在计算对易子的时候出现。做变换
\[
    \pi \longrightarrow \pi' = \frac{\pi}{\hbar}
\]
也可以完全消去这个常数。由于$\pi$是通过$\mathcal{L}$对$\partial_0 \phi$求导计算出来的,这个变换实际上就是对拉氏量做了变换
\[
    \mathcal{L} \longrightarrow \mathcal{L}' = \frac{\mathcal{L}}{\hbar},
\]
而这当然不影响实际的物理。事实上它改变的是能量和动量的单位。

从本节开始,在本文的剩余部分我们将使用自然单位制,那就是说,取消时间和空间的单位差异,并且取$\hbar = 1$。
从自然单位制恢复到国际单位制就是把上面的变换反过来,也就是做变换
\[
    \begin{aligned}
        \mathcal{L}_\text{nat} &\longrightarrow \mathcal{L}_\text{int} = \hbar \mathcal{L}_\text{nat}, \\
        E_\text{nat} &\longrightarrow E_\text{int} = \hbar E_\text{nat} , \\
        \vb*{p}_\text{nat} &\longrightarrow \vb*{p}_\text{int} = \hbar \vb*{p}_\text{nat}, \\
        t_\text{nat} &\longrightarrow t_\text{int} = c t_\text{nat}.
    \end{aligned}
\]
与此同时保持各个公式的形式不变。

\subsection{单粒子情况}

在已经知道了3+1维场论的理论之后,单粒子情况实际上就是一个退化情况,因为它实际上是0+1维场论。
在单粒子情况下底流形就是时间轴,其上定义有各种物理量$\hat{A}(t)$。单粒子情况下几乎不需要使用反对易量子化方案\eqref{eq:antisymmetry-commutator},物理量和它的共轭动量之间的关系可以全部取
\begin{equation}
    \comm*{\hat{x}}{\hat{p}} = \ii.
\end{equation}
下面推导$\hat{x}, \hat{p}$和任意物理量的对易关系。
设能够将物理量$\hat{F}$展开为$\hat{x}, \hat{p}$的多项式$\hat{F} = F(\hat{x}, \hat{p})$。
对其中的每一项,都可以使用对易关系
\[
    \comm*{\hat{x}}{\hat{p}} = \ii
\]
把$\hat{x}$挪到最前面而把$\hat{p}$挪到后面,
因此展开式最后就可以写成若干个$a \hat{x}^m \hat{p}^n$形式的项之和。
现在分析其中的一项:
\[
    [\hat{x}, \hat{x}^m \hat{p}^n] = \hat{x}^m [\hat{x}, \hat{p}^n] + [\hat{x}, \hat{x}^m] \hat{p}^n = \hat{x}^m [\hat{x}, \hat{p}^n],
\]
而
\[
    [\hat{x}, \hat{p}^n] = [\hat{x}, \hat{p} \hat{p}^{n-1}] = 
    \hat{p} [\hat{x}, \hat{p}^{n-1}] + [\hat{x}, \hat{p}] \hat{p}^{n-1} = \hat{p} [\hat{x}, \hat{p}^{n-1}] + \ii \hat{p}^{n-1}
\]
于是递推得到
\[
    [\hat{x}, \hat{p}^n] = \ii n \hat{p}^{n-1},
\]
因此
\[
    [\hat{x}, \hat{x}^m \hat{p}^n] = \ii n \hat{x}^m \hat{p}^{n-1}.
\]
这样就可以写出
\begin{equation}
    [\hat{x}, \hat{F}(\hat{x}, \hat{p})] = \ii \pdv{p} \hat{F}(\hat{x}, \hat{p}),
\end{equation}
在作用偏微分符号之前需要先把$F$中的每一项都变形成$\hat{x}$在前$\hat{p}$在后的形式。
使用同样的方法还可以导出
\begin{equation}
    [\hat{p}, \hat{F}(\hat{x}, \hat{p})] = - \ii \pdv{x} \hat{F}(\hat{x}, \hat{p}),
\end{equation}
同样,作用偏微分符号之前需要先把$F$中的每一项都变形成$\hat{x}$在前$\hat{p}$在后的形式。

在海森堡绘景下
\[
    \dv{\hat{A}}{t} = \frac{1}{\ii} [\hat{A}, H] + \pdv{\hat{A}}{t},
\]
于是
\[
    \dv{\hat{x}}{t} = \frac{1}{\ii} [\hat{x}, H] = \pdv{p} \hat{H}(\hat{x}, \hat{p}), \quad
    \dv{\hat{p}}{t} = \frac{1}{\ii} [\hat{p}, H] = -\pdv{x} \hat{H}(\hat{x}, \hat{p})
\]
当$\hbar \to 0$时,上式仍然成立,而此时$\hat{x}$和$\hat{p}$已经是对易的了,因此它们退化为了可以直接使用实数表示的情况,我们也就过渡到了经典力学。

% TODO:如果两组同类型粒子彼此没有相互作用,那么计算其中一群时可以忽略另一群,举例:
\[
    \frac{1}{\sqrt{6}} \left( \ket{1}\ket{2}\ket{3} + \ket{1}\ket{3}\ket{2} + \ket{2}\ket{1}\ket{3} + \ket{2}\ket{3}\ket{1} + \ket{3}\ket{1}\ket{2} + \ket{3}\ket{2}\ket{1}
    \right)
\]
如果1、2有相互作用而和3没有相互作用(注意如果12和3都和另一个系统有相互作用,那么它们之间还是有相互作用;判断两个系统之间有没有相互作用要把别的自由度都积掉只剩下这两个系统),那么在只考虑3时可以把$\ket{3}$略去,这样就剩下
\[
    \ket{1}\ket{2} + \ket{2}\ket{1}
\]
在计算只涉及1和2的振幅时,不管使用含有123的态还是只含有12的态都会得到一样的结果。

实际上这就是密度矩阵的分解。

\subsection{能谱和动力学}

\subsubsection{束缚态、散射态、能级}

在哈密顿量不显含时间时,原则上总是可以将哈密顿量做对角化,从而得到
\begin{equation}
    \hat{H} \ket*{n} = E_n \ket*{n}.
\end{equation}
诸$\ket*{n}$有确定的能量,于是我们称$E_n$组成的序列为\textbf{能级}。容易看出诸$\ket*{n}$在薛定谔绘景下的时间演化为
\begin{equation}
    \ket*{n(t)} = \ee^{- \ii E_n t} \ket*{n(0)},
\end{equation}
可以看出这种时间演化实际上只是一个绘景变换而已,实际上什么也没有改变,因此我们称诸$\ket*{n}$为\textbf{定态}。
\textbf{好量子数}就是守恒量的值(如动量、能量等),我们可以寻找一组好量子数来标记一组定态。
在哈密顿量显含时间时,能级、定态等概念当然就失去了意义。

为了便于使用算符的方式分析能级(同时也为二次量子化做准备),我们引入\textbf{产生湮灭算符}的概念。
设算符$\hat{x}$组成希尔伯特空间$\mathcal{H}$上的CSCO,其本征态为$\ket{x_1}, \ket{x_2}, \ldots$。
由于需要且只需要给定基矢量的像就能够确定一个算符,必定存在这样一个算符$\hat{a}$,它能够将$\ket{x_1}$映射为$\ket{x_2}$的某个非零倍数,将$\ket{x_2}$映射为$\ket{x_3}$的某个非零倍数,等等。这个算符称为升算符;升算符的逆就是降算符。显然,只需要一个本征态和升降算符就能够完全把态空间确定下来。另:如果本征值有上界,那么升算符作用在最大的本征值对应的本征态后得到$0$;同理,如果本征值有下界,那么降算符作用在最小的本征值对应的本征态之后得到$0$。
将$\ket{x}$提升到$\ket{x+c}$(可能差一个常数)的算符$\hat{a}$满足
\[
    [\hat{x}, \hat{a}] = c \hat{a}.
\]
特别的,若$\hat{x}$是厄米算符,且$\hat{a}$让本征态提升了$c$,那么$\hat{a}^\dagger$就会让本征态下降$c$,也就是说升降算符互为共轭转置。

现在的问题是怎样构造出升降算符。当然,任何情况下升降算符都应该满足对易关系$[\hat{x}, \hat{a}] = c \hat{a}$,但是这是不是足够了?
实际上还是能够构造出反例的,但是这些反例都是基于具体的分析构造,而物理上应该仅仅关心有关的代数结构。

\subsubsection{准经典理论}\label{sec:back-to-classical}

在特定的条件下,一个量子理论可以退化为经典理论——各个量子数要足够大,以至于不连续性可以被忽略(所以散射态看起来更加像经典情形),但能量又不能太高(否则需要考虑完整考虑圈图修正,而如下所述,圈图修正在经典图景下不能被完整地考虑)。
此时可以使用经典的运动方程描述整个系统。
在做完二次量子化之后这里其实有一个微妙之处,因为二次量子化将场的图景和粒子的图景严格地联系了起来,但是经典近似下没有二次量子化,所以其实场的图景下的经典近似和粒子图景下的经典近似还不一样。
前者的运动方程是“二次量子化哈密顿量”(即用场写出的哈密顿量)导出的,后者的运动方程是“一次量子化哈密顿量”(用粒子的位置、动量之类写出的哈密顿量)导出的。
在计算费曼图时,前者只有一条真正的外线(因为在量子的观点下我们是在计算$\expval{\phi}$对外场的响应),外部输入通过外源加入,传播子就表示场变量的一个可能的值;后者可以有好多条外线,没有外源,外部输入就是粒子输入,传播子表示一个粒子的轨迹。
与量子的费曼图不同,这两种经典的费曼图中每个顶角都有输入线和输出线的区分,输入线代表上一阶修正而输出线代表下一阶修正,这会排除掉大部分圈图修正:例如,一个四条线的顶角如果有三条输入线和一条输出线,那么就没有圈图,因为输出线没法接到输入线上(输出线只能接到另一个代表更高阶的修正的顶角上面);但是诸如“一个电子先发出一个光子,再吸收这个光子”这种类型的圈图在经典图景下也需要考虑。
所以,的确——经典图景下其实也是会出现圈图发散的,“一个电子先发出一个光子,再吸收这个光子”这个过程——或者在经典图景下说,电子通过自己产生的电场和自己发生相互作用——原则上是需要考虑的。
没有特别令人满意的方法能够将这一圈图发散去除掉;但是即使局限在经典图景下,这些圈图也是有意义的,因为它们意味着电子的物理质量会被电磁相互作用修正,而的确有实验证据暗示这一点。
在经典理论中我们一般不去计算这种圈图修正,一方面是因为树图已经提供了足够好的近似,一方面是因为没有必要:直接在量子理论中做这件事可以方便很多。
不过,这两种经典理论给出的结果大体上应该差不多,因为虽然场的图景下一部分费曼图没有被计算,但是这部分费曼图中有更少的外源出现次数,由于经典近似下各个量子数都比较大,外源相对于$\hbar$是很大的,因此外源出现次数最多的图几乎占据了全部的权重。

“和热库紧密耦合”有时候也被拿来做准经典近似的条件,但是需要注意的是“和热库紧密耦合”只能够保证可以适用基于经典概率的统计模型,这个模型仍然可以涉及量子特性——不连续的物理量取值(从而低能时的自由度冻结)等等。

前面提到,$\hbar \to 0$时,海森堡绘景下的量子时间演化方程\eqref{eq:quantum-evolution}退化为经典的时间演化方程\eqref{eq:evolution-of-any-quantity}。
但需要注意的是,在$\hbar\to 0$时由\eqref{eq:quantum-evolution}退化得到的方程仍然是一个算符方程。
要获得通常的关于物理量的方程,还需要做一些操作。$\hbar\to 0$时得到的演化方程是
\[
    \dv{\hat{A}}{t} = \pdv{\hat{A}}{\hat{\phi}} \dv{\hat{\phi}}{t} + \pdv{\hat{A}}{\partial_i \hat{\phi}} \partial_i \dv{\hat{\phi}}{t} + \pdv{\hat{A}}{\hat{\pi}} \dv{\hat{\pi}}{t},
\]
这个方程仅在海森堡绘景下成立。记系统的态矢量为$\ket{\psi}$,我们就得到
\[
    \dv{t} \mel{\psi}{\hat{A}}{\psi} =  \mel{\psi}{\pdv{\hat{A}}{\hat{\phi}} \dv{\hat{\phi}}{t}}{\psi} + \mel{\psi}{\pdv{\hat{A}}{\partial_i \hat{\phi}} \partial_i \dv{\hat{\phi}}{t}}{\psi} + \mel{\psi}{\pdv{\hat{A}}{\hat{\pi}} \dv{\hat{\pi}}{t}}{\psi}.
\]
在$\hbar\to 0$时,所有算符都近似是对易的,从而它们全部可以在同一组基下对角化。设这一组基为$\{\ket{n}\}$,则
% TODO:似乎$\ket{\psi}$总是几乎是这组基中的一个,为什么?
\[
    \begin{aligned}
        \mel{\psi}{\hat{A}\hat{B}}{\psi} &= \sum_{m,n} \braket{\psi}{m} \mel{m}{\hat{A}\hat{B}}{n} \braket{n}{\psi} \\
        &= \sum_{n} \braket{\psi}{n} \mel{n}{\hat{A}\hat{B}}{n} \braket{n}{\psi} \\
        &= 
    \end{aligned}
\]
% TODO
% 总之核心思想是,算符在$\hbar\to 0$时也不是实数物理量,真正的实数物理量的表达式必定会牵扯到态矢量。这也就是场算符的傅里叶分量看起来似乎是固定的值一样的原因,因为场算符本身包含了所有可能的经典场的取值,在$\hbar\to 0$时经典场的取值是多少不是场算符决定的而是态矢量决定的。

正则对易关系
\[
    [q_i, p_j] = \delta_{ij}
\]
实际上是非常自然的,因为使用这个关系推导出来的方程和使用对应的拉氏量和E-L方程推导出来的运动方程是一样的。
任何两个物理量的对易子$[A,B]$最后都可以写成一系列形如$\gamma_1 \gamma_2 \cdots [\gamma, \gamma] \cdots$这样的项的
叠加,其中每一个$\gamma$都是一个基本算符(坐标、动量、自旋等等),如果我们已知$[p, q] = \ii \hbar \cdot \text{something}$
而运动方程为
\[
    \dv{A}{t} = \frac{1}{\ii \hbar} [A, H]
\]
那么在最后得到的运动方程中$\ii \hbar$就被消去了。
现在让$\hbar \to 0$,我们会发现运动方程的形式没有发生变化(因为它根本就和$\hbar$无关),但是此时所有的物理量都是对易的了。
重新定义
\[
    \{A, B\} = \frac{1}{\ii \hbar}[A, B],
\]
它$\hbar \to 0$时仍然收敛于有限值。然后使用对易关系可以推导出它就是所谓的泊松括号。

使用不随时间变化的态矢量$\ket{\psi}$表述系统。
可观察量算符$A$随着时间的演化为
\begin{equation}
    \dv{A}{t} = \frac{1}{\ii \hbar} [A, H] + \pdv{A}{t}
    \label{eq:canonical-time-evolution}
\end{equation}

形式上这个式子和经典力学中的式子差了一个系数$\ii \hbar$。表面上看这正是量子力学和经典力学不同的地方(引入了常数$\hbar$),但实际上并非如此,因为在量子力学中有
\begin{equation}
    [x_i, p_j] = \ii \hbar \delta_{ij}
\end{equation}
一来一去,系数$\ii \hbar$就约掉了,实际上,完全可以定义
\[
    [x_i, p_j] = \delta_{ij}
\]
而此时的演化方程就变成
\[
    \dv{A}{t} = [A, H] + \pdv{A}{t}
\]
形式上和经典情况完全一致。那么量子力学和经典力学到底相差在哪里?
最关键的差别实际上是,量子力学中的$x, p$等量都是算符,因此有可能
\[
    AB - BA \neq 0
\]
而经典情况下上式恒为零。并且,这个不对易性直接和$[\cdot, \cdot]$的定义有关:
\[
    [A, B] = AB - BA
\]
在经典力学中$AB-BA$也是一个反对称的运算,但是它恒为零,因此和系统的演化无关——经典力学中和系统演化有关的那种$[\cdot, \cdot]$完全由
\[
    [A, B] = \sum_i \left( \pdv{A}{q_i} \pdv{B}{p_i} - \pdv{A}{p_i} \pdv{B}{q_i} \right)
\]
定义,上式又等价于两个假设:乘法交换律,以及
\[
    [q_i, q_j] = 0, [p_i, p_j] = 0, [q_i, p_j] = \delta_{ij}
\];
而在量子力学中,我们假定$xp-px=\ii \hbar$,并且认为
\[
    [A, B] = AB - BA
\]

总之,在括号$[\cdot, \cdot]$的性质、坐标和动量之间的括号的取值上,经典力学和量子力学之间完全没有差异。两者的差异在于,经典力学假定所有物理量都是可交换的实数,此时我们可以推导出泊松括号的表达式;量子力学假定$[\cdot, \cdot]$就代表两个物理量(现在是算符了!)的交换子。

因此在经典力学中使用“对易”一词可能引起误解:它可能指“两个量的乘积是不是可以交换”,此时的回答一概是“是”;它也可能指“两个量的泊松括号是不是零”。这两种理解之间完全没有联系。而在量子力学中这两种理解实际上是等价的。

可观察量经过时间演化之后还应该是可观察量。但是这个怎么证明呢?

\[
    \left(\dv{A}{t}\right)^\dagger = - \frac{1}{\ii \hbar} [A, H]^\dagger + \left(\pdv{A}{t}\right)^\dagger = \frac{1}{\ii \hbar} [A^\dagger, H^\dagger] + \left(\pdv{A}{t}\right)^\dagger
\]
如果在某一时刻$A$是观察算符,则下一刻它仍然是观察算符的充要条件就是
\[
    \frac{1}{\ii \hbar} [A, H^\dagger] + \left(\pdv{A}{t}\right)^\dagger = \frac{1}{\ii \hbar} [A, H] + \pdv{A}{t}
\]
所以什么情况下确凿无疑的有$H$是厄米算符呢?

TODO:正则对易关系与运动方程。好像如果不使用正则对易关系,那么算符演化方程就和通过对应的拉氏量写出的运动方程不一致。

可以使用傅里叶变换把哈密顿量中的$\nabla \phi$之类的项弄掉。
然后得到的哈密顿量做对角化(大部分情况下已经对角化好了),就得到了一系列谐振子哈密顿量的叠加:
\[
    H = \int \dd x^3 a a^\dagger + \text{something}
\]

拉氏量的耦合对应着态空间的耦合?混合态、直积还有一系列神奇的东西。直和其实是增加了基矢量。
也就是说一个算符的各个不变子空间的直和构成全空间。
\[
    \delta(\vb*{r} - \vb*{r}_0) = \delta(x - x_0) \delta(y - y_0) \delta(z - z_0)
\]
所以三维态矢量其实是一维态矢量的直积。

把问题规范一下:现在我们已知系统的演化可以完全由一组算符$\hat{O}_1, \hat{O}_2, \ldots, \hat{O}_n$描述,也就是说能够写出哈密顿算符$\hat{H}$来描述它们的演化。此外,这些算符的对易关系全部给定,从而$[\hat{O}_i, \hat{H}]$也确定了。
现在的问题是,态矢量应该怎么取?或者说,对应的希尔伯特空间应该是怎样的结构?
实际上在量子场论中这似乎并不是一个问题,因为很少用到态矢量。这是因为只有算符是重要的,态空间实际上只是算符对应的李代数的幺正表示的表示空间而已。

设算符$\hat{O}_1, \hat{O}_2$分别是希尔伯特空间$H_1$、$H_2$的CSCO,且它们组成的集合是$H$的CSCO,那么$H = H_1 \otimes H_2$,并且$\ket{x_1, x_2} = \ket{x_1} \otimes \ket{x_2}$,其中$\ket{x_1} \in H_1, \ket{x_2} \in H_2$,$\ket{x_1, x_2} \in H$。
顺便抨击一下常见的量子力学教材:一上来就讲态矢量在概念上真的很不清楚!

我好像有点反应过来了。CSCO就是用来做这个的!
设$\hat{O}_1, \hat{O}_2, \ldots, \hat{O}_n$组成了一个空间$H$上的CSCO,且与$\hat{O}_1$对应的

作用在一个算符上的元算符如果不改变它所作用的那个算符的定义域,那么将这个元算符作用在另一个算符上就相当于将第三个算符和第二个算符复合。

一些问题:是不是任何一个幺正算符都对应着某个物理过程?(从初态到末态的映射)

粒子数表象。

\section{二次量子化}\label{sec:second-quantization}

在单粒子量子力学中我们发现要完整描述系统需要分析$\hat{\vb*{x}}$的各个本征态,
从而经典情况下描述$\vb*{x}$的轨迹方程在量子情况下不再适用,
而要改用关于\textbf{波函数}$\braket{\vb*{x}}{\phi} = \phi(\vb*{x})$的偏微分方程。
因此,量子理论中对粒子的描述涉及一个场。
本节将讨论相反的问题:量子理论中的场实际上能够自然地诱导出“粒子”的概念。
我们还将发现,单粒子量子力学中的波函数实际上真的相当于像电磁场这样的一个场。
将经典场量子化、并且从量子化的场当中发现粒子性的操作统称为\textbf{二次量子化}。
相应的,完全使用单粒子图像做的分析称为\textbf{一次量子化}。
有一些问题使用单粒子图像难以处理,因此二次量子化实际上是比一次量子化更加基本的理论,但很多时候我们先获得了一个一次量子化的理论,然后再构造一个能够推导出这个一次量子化的理论的场论以方便计算,从而似乎我们是在一次量子化理论上又做了一次量子化。
这并不是事实,虽然这种感觉正是“二次量子化”一词的来源。

在本节接下来的推导中(以及别的很多地方)我们都认为体系的动力学自由度就只有一个算符。
这个说法当然是不确切的,例如一个单粒子的状态就同时需要使用$\vb*{x}$和自旋来描述;但是能够将这两者直积起来,得到的算符的本征值是$\pmqty{ x^1 & x^2 & x^3 & S }$,于是它一个算符就成为了整个体系的CSCO。

\subsection{多粒子态空间}\label{sec:many-body-state}

本节我们将从两个方向分析含有多个粒子的系统,也就是所谓的\textbf{多体系统}。
首先我们将从一系列完全相同的单粒子希尔伯特空间构造多粒子福克空间,
然后我们将说明,这种多粒子福克空间实际上可以使用一对产生湮灭算符和一个唯一的真空态干脆利落地构造出来。

\subsubsection{$n$粒子希尔伯特空间}\label{sec:n-particle-space}

考虑一个有$n$个粒子构成的体系。我们使用$\hat{M}_i$表示完全描述了第$i$号粒子的单粒子算符,也就是说它无简并。
既然不同的$i$对应的$\hat{M}_i$作用于不同的希尔伯特空间,它们当然就是对易的,
从而$M_1, M_2, \ldots, M_n$组成了一个完备对易算符集,整个体系的哈密顿量$\hat{H}$是这个算符集的函数。
设每个粒子的希尔伯特空间为$H_i$,那么整个体系的希尔伯特空间就是
\[
    H = H_1 \otimes H_2 \otimes \cdots \otimes H_n,
\]
且每个$H_i$都彼此同构。空间$H$的本征态可以写成
\[
    \ket{\text{eigenstate}} = \prod_i \ket{\text{one of $\hat{M}_i$'s eigenstates}}.
\]

我们说这些粒子是全同粒子,当且仅当
\begin{enumerate}
    \item 诸$M_i$具有相同的谱结构;从而,诸态空间$H_i, \; i=1, 2, \ldots$幺正等价,于是我们可以不失一般性地认为诸粒子的态空间都是相同的,且$M_1, M_2, \ldots$实际上可以看成某个抽象的李代数的元素$\hat{M}$在一系列幺正等价的态空间$H_1, H_2, \ldots$上的表示;
    \item 交换任意两个粒子之后系统的动力学不变,或者说交换两个粒子之后系统的哈密顿量不变。%
    \footnote{如果系统中粒子数恒等,这就是说系统的哈密顿量中不同粒子具有同等的地位;不过我们在这里并不处理这样的$n$粒子哈密顿量,因为在很多情况下粒子数会发生变化,从而哈密顿量中不可能直接含有各个粒子的算符。我们在\autoref{sec:from-qft-to-many-body}中会看到,此时哈密顿量中应该显含某种场算符,而单粒子算符是可以使用场算符弄出来的。} % TODO:这一段没写清楚
\end{enumerate}
这两个要求缺一不可;前者要求诸粒子是同样的对象,后者要求交换粒子不改变系统的动力学。

下面我们来讨论交换粒子意味着什么。形如
\[
    \hat{P} (\ket{\psi_1}\ket{\psi_2}\cdots) = \ket{\psi_{k_1}}\ket{\psi_{k_2}}\cdots, \quad \{1, 2, \ldots\} = \{k_1, k_2, \ldots\}
\]
的算符$\hat{P}$称为\textbf{粒子交换算符}。显然,任何一个粒子交换算符都可以写成一系列交换两个粒子的算符的乘积。设算符$\hat{P}$交换了第$i$个粒子和第$j$个粒子,那么我们有
\[
    \hat{P} \left( \ket{\psi_1} \ket{\psi_2} \cdots \ket{\psi_i} \cdots \ket{\psi_j} \cdots \ket{\psi_n} \right) = \ket{\psi_1} \ket{\psi_2} \cdots \ket{\psi_j} \cdots \ket{\psi_i} \cdots \ket{\psi_n},
\]
既然$H$是各粒子态空间的直积而各个粒子的态空间实际上是完全一样的,算符$\hat{P}$实际上是从$H$到$H$的算符。
此外注意到$\hat{P}$并不改变态矢量的模,因此它还是一个幺正算符。
对一个全同粒子体系,交换两个粒子的信息不会改变系统的动力学,也就是说$\hat{H}$在$\hat{P}$的作用下不变,
因此$\hat{P}$和$\hat{H}$是对易的,这又意味着$\hat{P}$不会有时间演化。
于是可以迭代说明,任何一种粒子交换算符都没有时间演化。
这又意味着,设$\ket{\psi}$经过某个时间演化过程之后变化为$\ket{\psi'}$,那么$\hat{P} \ket{\psi}$经过同样的时间演化过程之后变化为$\hat{P} \ket{\psi'}$。
我们发现希尔伯特空间中出现了\autoref{sec:hamiltonian-symmetry}中提到的冗余自由度。% TODO

考虑到交换任意两个$M_i$和$M_j$不改变系统的动力学,任何有意义的关于此$n$体系统的可观察量(它们是诸$M$的函数)在交换$M_i$和$M_j$之后形式保持不变。这就带来了一个结果:设$\hat{P}$是一个粒子交换算符(交换了哪些粒子随意),且$\hat{\phi}$是一个关于整个$n$粒子系统的可观察量%
\footnote{正如这个符号暗示的那样,$\hat{\phi}$通常可以被理解为某种量子化的场。我们将在\autoref{sec:from-qft-to-many-body}中看到这样的例子。},
那么$\hat{P}$和$\hat{\phi}$对易,于是
\[
    \hat{\phi} \left(\hat{P} \ket{\phi}\right) = \hat{P} \hat{\phi} \ket{\phi} = \hat{P} \phi \ket{\phi} = \phi \hat{P} \ket{\phi},
\]
这意味着任何一个本征态的置换都是另一个本征态,且本征值相同。这意味着$H$上任何有意义的可观察量以及它们的组合都不能够成为一组CSCO:任何有意义的可观察量都一定有简并!这样的简并称为\textbf{交换简并}。
这样,如果我们只关注系统的$\phi$属性的值,那么使用$\ket{\text{system}}$做计算和使用归一化的态$\sum_i \hat{P}_i \ket{\text{system}}$做计算得到的结果完全一样。
% TODO:对这一句话的严格解释:做什么样的计算?
因此$H$实际上是过大的——我们应该讨论某种比它的维度更低的空间,但其上的算符的结构不应该有变化。
也就是说,描写实际的全同粒子体系只需要$H$的一个子代数%
\footnote{当然,不构造子代数原则上也是可以的,但是有意义的算符都不构成CSCO这件事会大大加大处理问题的难度。
需注意虽然我们是把$H$的一部分孤立出来讨论,但这和构造混合态时“只考虑系统的一部分”是完全不同的。
全同粒子体系的态不会是不满足对称或者反对称条件的态矢量,因此从$H$缩小到$H$的对称化或反对称化子代数不会损失任何信息。
反之,构造混合态时被忽视的那部分系统仍然携带了信息。因此构造全同粒子体系不会产生混合态。},
任何一个$H$中的态都可以对应到这个子代数中的一个态,并且$\hat{\phi}$构成这个子代数的一个CSCO
——由于$\hat{\phi}$在粒子交换下不变,这又意味着,将交换算符作用在这个子代数的某一基矢量之后态的改变应该只是乘上了一个复数因子。
而由于粒子交换算符是幺正的,这个复数因子的模长一定是1。
% TODO:在我们只关注某些算符提供的信息时对希尔伯特空间的简化
% 一个可能的思路:记$f$将$H$中的态$\ket{\psi}$映射到了$H$的一个子代数上,且$f$是一个满射,且等价的态被$f$作用之后仍然是等价的;由于子代数中我们关注的算符构成CSCO,$H$中等价的态被$f$作用之后的结果只差了一个系数。
% TODO:定义什么叫做态“等价”,或者说“表示同一个系统”

具体这个复数因子是多少,不同的基矢量对应的复数因子是不是都相同,都值得讨论;
我们将只讨论这因子取$\pm 1$的情况,其余情况可以类推。%
\footnote{这个因子具体能够取什么值和底流形拓扑等有关系,有些取值会导致一些矛盾或者非物理的结果。}%
总之,现在我们需要构造$H$的对称化和反对称化子代数。
最简单的方法是,使用$H$的诸基矢量构造一组对称化与反对称化基矢量,这样无需对$H$上的算符做任何修改。

\subsubsection{对称与反对称基矢量}

这一节我们需要分析具体的粒子交换,因此引入一些记号。
$n$元组有$n!$种排列方式;我们将这$n!$种排列方式从$1$到$n!$编上号,并设$\hat{P}_s$为第$s$种排列方式对应的粒子交换算符,$p_s$是第$s$种排列方式对应的交换数。

设$M^{(1)}, M^{(2)}, \ldots$是$\hat{M}$的本征值。这里我们把$\hat{M}$看成离散谱的,但是可以取其极限得到连续谱的情况。
这样$H$的基矢量就全部可以写成
\[
    \ket{M^{(k_1)}}_1 \ket{M^{(k_2)}}_2 \cdots \ket{M^{(k_n)}}_n = \ket{M^{(k_1)} M^{(k_2)} \cdots M^{(k_n)}}
\]
的形式。其中下标$1, 2, \ldots$指的是这个基矢量在空间$H_1, H_2, \ldots$中。
设$k_1, k_2, k_3, \ldots$中有$n_1$个$1$,$n_2$个$2$,等等,称这些$n_1, n_2, \ldots$为\textbf{占据数};不重复的$k_i$——从而不重复的$M^{(k_i)}$——总共有$m$个,并且
\begin{equation}
    \sum_{i=1}^m n_i = n.
\end{equation}

现在尝试构造对称化子代数和反对称化子代数的基矢量。设$\ket{\psi}$是$H_S$或$H_A$中的一个态。当然,它也是$H$中的态,从而
\[
    \ket{\psi} = \sum_{k_1, k_2, \ldots, k_n} c_{k_1 k_2 \cdots k_n} \ket{M^{(k_1)}}_1 \ket{M^{(k_2)}}_2 \cdots \ket{M^{(k_n)}}_n,
\]
且
\[
    \hat{P}_s \ket{\psi} = (\pm 1)^{p_s} \ket{\psi},
\]
若$\ket{\psi}$是对称化的,则取$+1$,若$\ket{\psi}$是反对称化的,取$-1$。
然后就有
\[
    \sum_s (\pm 1)^{p_s} \hat{P}_s \ket{\psi} = \sum_s (\pm 1)^{p_s} \hat{P}_s \sum_{k_1, k_2, \ldots, k_n} c_{k_1 k_2 \cdots k_n} \ket{M^{(k_1)}}_1 \ket{M^{(k_2)}}_2 \cdots \ket{M^{(k_n)}}_n,
\]
此方程的左边是
\[
    \sum_s (\pm 1)^{p_s} \hat{P}_s \ket{\psi} = \sum_s (\pm 1)^{p_s} (\pm 1)^{p_s} \ket{\psi} = \sum_s \ket{\psi} = n! \ket{\psi},
\]
右边是
\[
    \begin{aligned}
        &\quad \sum_s (\pm 1)^{p_s} \hat{P}_s \sum_{k_1, k_2, \ldots, k_n} c_{k_1 k_2 \cdots k_n} \ket{M^{(k_1)}}_1 \ket{M^{(k_2)}}_2 \cdots \ket{M^{(k_n)}}_n \\
        &= \sum_{k_1, k_2, \ldots, k_n} c_{k_1 k_2 \cdots k_n} \sum_s (\pm 1)^{p_s} \hat{P}_s \ket{M^{(k_1)}}_1 \ket{M^{(k_2)}}_2 \cdots \ket{M^{(k_n)}}_n,
    \end{aligned}
\]
因此$\ket{\psi}$可以完全被形如
\[
    \sum_s (\pm 1)^{p_s} \hat{P}_s \ket{M^{(k_1)}}_1 \ket{M^{(k_2)}}_2 \cdots \ket{M^{(k_n)}}_n
\]
的矢量线性表示。
如果我们能够归一化这些矢量,并且证明它们的正交性,那么它们归一化之后就是$H_S$和$H_A$的基矢量。

首先我们讨论对称化子代数$H_S$。我们取$H$的形如下式的对称化基矢量:
\[
    \ket{n; M^{(k_1)} M^{(k_2)} \cdots M^{(k_n)}}_S \propto \sum_s \hat{P}_s \ket{M^{(k_1)}}_1 \ket{M^{(k_2)}}_2 \cdots \ket{M^{(k_n)}}_n,
\]
我们要计算其归一化系数。首先
\[
    \begin{aligned}
        &\quad \left(\sum_s \hat{P}_s \ket{M^{(k_1)}}_1 \ket{M^{(k_2)}}_2 \cdots \ket{M^{(k_n)}}_n\right)^\dagger \sum_s \hat{P}_s \ket{M^{(k_1)}}_1 \ket{M^{(k_2)}}_2 \cdots \ket{M^{(k_n)}}_n \\
        &= \sum_{s, s'} \bra{M^{(k_1)}}_1 \bra{M^{(k_2)}}_2 \cdots \bra{M^{(k_n)}}_n \hat{P}_{s'}^\dagger \hat{P}_s \ket{M^{(k_1)}}_1 \ket{M^{(k_2)}}_2 \cdots \ket{M^{(k_n)}}_n, 
    \end{aligned}
\]
注意到由于$\hat{P}$的幺正性和可逆性,$\hat{P}_{s'}^\dagger \hat{P}_s$也是一个粒子交换算符,
并且固定$s$不动,不同的$s'$会让$\hat{P}_{s'}^\dagger \hat{P}_s$取不同的值;
从而,对每个$s$,$\hat{P}_{s'}^\dagger \hat{P}_s$都有$n!$个值,
也即固定$s$不动而让$s'$从$1$取到$n!$,$\hat{P}_{s'}^\dagger \hat{P}_s$的值不重复地取遍所有共计$n!$个粒子交换算符;
从而当$s$和$s'$都从$1$计数到$n!$时,$\hat{P}_{s'}^\dagger \hat{P}_s$的值取遍所有粒子交换算符,且每个重复$n!$次,从而
\[
    \begin{aligned}
        &\quad \sum_{s, s'} \bra{M^{(k_1)}}_1 \bra{M^{(k_2)}}_2 \cdots \bra{M^{(k_n)}}_n \hat{P}_{s'}^\dagger \hat{P}_s \ket{M^{(k_1)}}_1 \ket{M^{(k_2)}}_2 \cdots \ket{M^{(k_n)}}_n \\
        &= n! \sum_s \bra{M^{(k_1)}}_1 \bra{M^{(k_2)}}_2 \cdots \bra{M^{(k_n)}}_n \hat{P}_s \ket{M^{(k_1)}}_1 \ket{M^{(k_2)}}_2 \cdots \ket{M^{(k_n)}}_n.
    \end{aligned}
\]
我们知道$\hat{M}$的各个本征态是正交的,因此上式中的内积只有在$\hat{P}$作用在$\ket{M^{(k_1)}}_1 \ket{M^{(k_2)}}_2 \cdots \ket{M^{(k_n)}}_n$得到的结果和作用前一样时才能取$1$,否则均取零。这样的排列方式总共有$n_1!n_2!\cdots n_m!$个,也就是说上式右边的求和号中值为$1$的项共有$n_1!n_2!\cdots n_m!$个,其余均为零,所以我们得到
\[
    \begin{aligned}
        &\quad \left(\sum_s \hat{P}_s \ket{M^{(k_1)}}_1 \ket{M^{(k_2)}}_2 \cdots \ket{M^{(k_n)}}_n\right)^\dagger \sum_s \hat{P}_s \ket{M^{(k_1)}}_1 \ket{M^{(k_2)}}_2 \cdots \ket{M^{(k_n)}}_n \\
        &= n! \sum_s \bra{M^{(k_1)}}_1 \bra{M^{(k_2)}}_2 \cdots \bra{M^{(k_n)}}_n \hat{P}_s \ket{M^{(k_1)}}_1 \ket{M^{(k_2)}}_2 \cdots \ket{M^{(k_n)}}_n \\
        &= n! n_1 ! n_2! \cdots ,
    \end{aligned}
\]
于是得到归一化对称基矢量
\begin{equation}
    \ket{n; M^{(k_1)} M^{(k_2)} \cdots M^{(k_n)}}_S = \frac{1}{\sqrt{n! n_1 ! n_2! \cdots}} \sum_s \hat{P}_s \ket{M^{(k_1)}}_1 \ket{M^{(k_2)}}_2 \cdots \ket{M^{(k_n)}}_n.
    \label{eq:sym-basis}
\end{equation}

同样的,我们考虑形如
\[
    \ket{n; M^{(k_1)} M^{(k_2)} \cdots M^{(k_n)}}_A \propto \sum_s (-1)^{p_s} \hat{P}_s \ket{M^{(k_1)}}_1 \ket{M^{(k_2)}}_2 \cdots \ket{M^{(k_n)}}_n.
\]
的反对称化基矢量。
使用和对称化基矢量同样的方法可以归一化这个基矢量。
在动手之前,注意到如果$\ket{M^{(k_1)}}_1$,$\ket{M^{(k_2)}}_2$,..., $\ket{M^{(k_n)}}_n$中有重复的态,
那么必定会导致相应的$\ket{n; M^{(k_1)} M^{(k_2)} \cdots M^{(k_n)}}_A$为零。
这是因为设$\hat{P}$交换了两个重复的态,那么就有
\[
    \hat{P} \ket{n; M^{(k_1)} M^{(k_2)} \cdots M^{(k_n)}}_A = \ket{n; M^{(k_1)} M^{(k_2)} \cdots M^{(k_n)}}_A,
\]
而由于这是反对称化基矢量,我们又有
\[
    \hat{P} \ket{n; M^{(k_1)} M^{(k_2)} \cdots M^{(k_n)}}_A = - \ket{n; M^{(k_1)} M^{(k_2)} \cdots M^{(k_n)}}_A,
\]
于是相应的反对称化基矢量就是零。因此我们只需要讨论其中所有单粒子态都不重复的反对称化基矢量。
如果单粒子态不重复,那么求和号中的态彼此正交,于是
\[
    \begin{aligned}
        &\quad \left(\sum_s (-1)^{p_s} \hat{P}_s \ket{M^{(k_1)}}_1 \ket{M^{(k_2)}}_2 \cdots \ket{M^{(k_n)}}_n\right)^\dagger \sum_s (-1)^{p_s} \hat{P}_s \ket{M^{(k_1)}}_1 \ket{M^{(k_2)}}_2 \cdots \ket{M^{(k_n)}}_n \\
        &= \sum_s (-1)^{2p_s} \left(\hat{P}_s \ket{M^{(k_1)}}_1 \ket{M^{(k_2)}}_2 \cdots \ket{M^{(k_n)}}_n\right)^\dagger \hat{P}_s \ket{M^{(k_1)}}_1 \ket{M^{(k_2)}}_2 \cdots \ket{M^{(k_n)}}_n \\
        &= \sum_s 1 = n!,
    \end{aligned}
\]
从而
\begin{equation}
    \ket{n; M^{(k_1)} M^{(k_2)} \cdots M^{(k_n)}}_A 
    = \frac{1}{\sqrt{n!}} \sum_s (-1)^{p_s} \hat{P}_s \ket{M^{(k_1)}}_1 \ket{M^{(k_2)}}_2 \cdots \ket{M^{(k_n)}}_n.
    \label{eq:asym-basis}
\end{equation}
当然,由于此时$n_1 = n_2 = \cdots = 1$,\eqref{eq:asym-basis}也可以写成\eqref{eq:sym-basis}的形式。

然后我们来讨论\eqref{eq:sym-basis}和\eqref{eq:asym-basis}的正交归一性。
归一化性已经通过计算归一化系数完成了,我们接下来讨论正交性。
无论作用怎么样的$\hat{P}_s$,都不会改变一个态中的占据数$n_1, n_2, \ldots$,
因此两个\eqref{eq:sym-basis}中占据数不同的态的求和号中出现的所有态都不相同%
\footnote{需要注意的是占据数相同的态也有可能不同。在$H_S$中占据数相同的态完全相同,因为它们之间差了有限次粒子交换;
$H_A$中占据数相同的态如果差了奇数次粒子交换,那么它们就差了一个负号,差了偶数次粒子交换则相同。},
因此两个\eqref{eq:sym-basis}中占据数不同的态的内积是零;
同样的思路也说明\eqref{eq:asym-basis}中占据数不同的态的内积是零。
在已经做了归一化之后,我们确认,\eqref{eq:sym-basis}和\eqref{eq:asym-basis}都满足正交归一化条件。

至此我们发现,实际的全同粒子系统只需要使用$H$的对称化子代数$H_S$或者反对称化子代数$H_A$描述,
\eqref{eq:sym-basis}是$H_S$的基底,\eqref{eq:asym-basis}是$H_A$的基底。
我们还知道,$H_A$中的基矢量中不会有两个粒子处于同样的态上,
从这个结论容易推导出,$H_A$中任何一个态中都不会有两个粒子的单粒子态完全相同。

\subsubsection{福克空间与产生湮灭算符}
% TODO:将“粒子数算符”的名称改为“占据数算符”
现在对任何一个正整数$n$,我们都已经建立起了$n$粒子全同粒子系统的态空间——那就是说,对称化和反对称化希尔伯特空间。
而我们也可以指定$n=0$时的“全同粒子系统”的态空间是平凡的向量空间$\{0\}$,并记其中唯一一个矢量为$\ket{0}$。
于是我们记$n$粒子对称化或反对称化希尔伯特空间为$H_S^{(n)}$和$H_A^{(n)}$,且$H_S^{(0)}$和$H_A^{(0)}$就是$\{\ket{0}\}$。
本节我们转而考虑这样的问题:在这一系列空间之间有什么联系?

在维数不同、互不等价的线性空间之间操作是非常麻烦的,因此我们尝试把诸$H_S^{(n)}$和$H_A^{(n)}$放在一个更大的空间中讨论,
这个更大的空间包含且仅包含诸$H_S^{(n)}$和$H_A^{(n)}$中的矢量。
构造这种“更大的空间”的方法有很多。
我们需要的操作绝对不是直积,因为两个空间的直积中含有原来的两个空间没有的向量。
因此尝试使用直和操作。
的确,并没有特殊的证据要求我们一定要使用直和,
但是如果我们讨论的系统中不涉及粒子数变动,那么具体使用的是直和还是别的什么操作无关紧要,
因为此时诸$H_S^{(n)}$和$H_A^{(n)}$直和出来的空间并没有物理意义
——我们只会把其中的粒子数和我们讨论的系统的粒子数相同的部分拿来做动力学计算。
而在\autoref{sec:from-qft-to-many-body}中会看到,使用直和是正确的。
总之,定义\textbf{对称的福克空间}
\[
    F_S = H_S^{(0)} \oplus H_S^{(1)} \oplus H_S^{(2)} \oplus \cdots,
\]
以及\textbf{反对称的福克空间}
\[
    F_A = H_A^{(0)} \oplus H_A^{(1)} \oplus H_A^{(2)} \oplus \cdots.
\]
每个空间的基矢量为
\[
    \ket{0}, \; \ket{1;M^{(k_1)}}, \; \ket{2;M^{(k_1)} M^{(k_2)}}, \; \ldots,
\]
其中$k_1, k_2, \ldots$跑遍所有可能的本征值。

现在我们定义\textbf{产生湮灭算符}。%
\footnote{产生湮灭算符未必有动力学上的意义。例如,在粒子数固定的动力学中它们就只是一种新的观点。但是在粒子数会变的情况下它们很重要。}%
以下我们略去了下角标$A$和$S$,因为这些定义与对称还是反对称无关。
首先取\textbf{产生算符}$a^\dagger$,使之满足
\begin{equation}
    \hat{a}^\dagger (M^{(i)}) \ket{n; M^{(k_1)} M^{(k_2)} \cdots M^{(k_n)}} = \sqrt{n_i + 1} \ket{n+1; M^{(i)} M^{(k_1)} M^{(k_2)} \cdots M^{(k_n)}}.
    \label{eq:creation-operator}
\end{equation}
其中${n_i}$指的是$\ket{n; M^{(k_1)} M^{(k_2)}}$中$M^{(i)}$态的个数。
给定了相应的系数,上式就完全地确定了一个福克空间(无论对称还是反对称)上的算符。%
\footnote{至于为什么要取这个$\sqrt{n_i + 1}$的系数,我们将在定义粒子数算符的时候看到。}
产生算符将$n$粒子态转化为$n+1$粒子态。
现在要问:有没有一个算符能够将$n+1$粒子态转化为$n$粒子态?
实际上由共轭转置的定义(注意到\eqref{eq:creation-operator}只涉及基矢量),
$\hat{a}^\dagger$的共轭转置,也就是$\hat{a}$就是满足这个条件的算符,
因为由共轭转置的定义可以导出
\[
    \hat{a} (M^{(i)}) \ket{n+1; M^{(i)} M^{(k_1)} M^{(k_2)} \cdots M^{(k_n)}} = \sqrt{n_i + 1} \ket{n; M^{(k_1)} M^{(k_2)} \cdots M^{(k_n)}},
\]
其中$n_i$指的是$\ket{n;M^{(i)} M^{(k_1)} M^{(k_2)} \cdots M^{(k_n)}}$中$M^{(i)}$态的个数,
于是我们称$\hat{a}$为\textbf{湮灭算符}。
这里还有一个微妙的细节。注意到$\hat{a}^\dagger(M^{(i)})$作用在任何一个态上面都不可能产生一个不含有$M^{(i)}$态的态,因此不能仅仅依靠\eqref{eq:creation-operator}得到$\hat{a}(M^{(i)})$作用在一个不含$M^{(i)}$态的态上的结果。
于是我们额外规定:$\hat{a}(M^{(i)})$作用在一个不含$M^{(i)}$态的态上得到长度为零的向量%
\footnote{不是得到$\ket{0}$——$\ket{0}$是真实的物理态,而长度为零的向量不是。}%
,从而$\hat{a}$和$\hat{a}^\dagger$在整个福克空间上都定义好了。%
\footnote{刚才提到的这种情况实际上来自数学上的一个结论:在希尔伯特空间中,$(\hat{a}^\dagger)^\dagger$未必就是$\hat{a}$。当然,我们在这里的处理方法并没有什么逻辑漏洞——当我们要求\eqref{eq:creation-operator}成立时我们只是要求存在某个算符$\hat{a}$(而不是产生算符)使得此方程成立,而接下来要求$\hat{a}(M^{(i)})$作用在一个不含$M^{(i)}$态的态上得到长度为零的向量就唯一确定了$\hat{a}$。}
这样如果重新定义$n_i$为$\ket{n+1; M^{(i)} M^{(k_1)} M^{(k_2)} \cdots M^{(k_n)}}$中$M_i$态的数量,那么就可以使用一个式子完全刻画$\hat{a}(M^{(i)})$的行为:
\begin{equation}
    \hat{a} (M^{(i)}) \ket{n+1; M^{(i)} M^{(k_1)} M^{(k_2)} \cdots M^{(k_n)}} = \sqrt{n_i} \ket{n; M^{(k_1)} M^{(k_2)} \cdots M^{(k_n)}}.
    \label{eq:annihitation-operator}
\end{equation}
于是\eqref{eq:creation-operator}和\eqref{eq:annihitation-operator}就给出了产生湮灭算符的定义。
请注意这两个式子中的$n_i$是不同的;它们分别是两个式子中被算符作用前的态中$M^{(i)}$态的个数。

现在指出一个事实:任何一个\eqref{eq:sym-basis}和\eqref{eq:asym-basis}中的态都可以通过产生算符和真空态$\ket{0}$推导出来。原因很简单:重复使用\eqref{eq:creation-operator},我们有
\begin{equation}
    \ket*{n; \underbrace{M^{(1)} M^{(1)} \cdots}_{n_1} \underbrace{M^{(2)} M^{(2)} \cdots}_{n_2} \cdots} = \frac{1}{\sqrt{n_1! n_2! \cdots}} \left(\hat{a}^\dagger (M^{(1)})\right)^{n_1} \left(\hat{a}^\dagger (M^{(2)})\right)^{n_2} \cdots \ket{0},
    \label{eq:creation-basis}
\end{equation}
而通过排列算符,\eqref{eq:sym-basis}和\eqref{eq:asym-basis}中的每一个基矢量都可以写成上式左边的形式(可能差一个负号),
因此所有的基矢量都可以通过产生算符构造出来。我们称形如\eqref{eq:creation-basis}这样的态也即,像\eqref{eq:sym-basis}和\eqref{eq:asym-basis}这样的态,为\textbf{乘积态},因为它们可以写成一系列产生算符乘积作用在真空态上的形式。

定义产生湮灭算符之后可以定义
\begin{equation}
    \hat{n}(M^{(i)}) = \hat{a}^\dagger (M^{(i)}) \hat{a} (M^{(i)})
    \label{eq:number-operator}
\end{equation}
为\textbf{占据数算符}。它叫做占据数算符是因为,按照\eqref{eq:creation-operator}和\eqref{eq:annihitation-operator}可以验证,我们有
\begin{equation}
    \hat{n}(M^{(i)}) \ket{n; M^{(k_1)} M^{(k_2)} \cdots M^{(k_n)}} = n_i \ket{n; M^{(k_1)} M^{(k_2)} \cdots M^{(k_n)}},
    \label{eq:number-eigenstate}
\end{equation}
其中$n_i$指$\ket{n; M^{(k_1)} M^{(k_2)} \cdots M^{(k_n)}}$中$M^{(i)}$态的个数。
\eqref{eq:number-eigenstate}意味着\eqref{eq:sym-basis}和\eqref{eq:asym-basis}都是$\hat{n}(M^{(i)})$的本征态,本征值就是$M^{(i)}$态的数目。
这就是占据数算符一词的来源。
相应的
\begin{equation}
    \hat{N} = \sum_i \hat{n}(M^{(i)})
    \label{eq:total-number-operator}
\end{equation}
就是\textbf{总粒子数算符},它的本征态也是\eqref{eq:sym-basis}和\eqref{eq:asym-basis},本征值为对应的态的总粒子数。

\subsubsection{产生湮灭算符的代数结构}\label{sec:algebra-ca-op}

% TODO:然后问题就来了:诸粒子数算符能不能完整描述被场算符完整描述的态空间?
在前面几节中,我们通过$M$表象下的单粒子空间构造出了全同粒子体系的希尔伯特空间,
通过全同性得到了有物理意义的对称化和反对称化希尔伯特空间,然后将它们直和起来得到福克空间,最后在其上定义了产生湮灭算符。
本节我们将推导产生湮灭算符的几个代数性质,然后接着我们会发现,通过这几个性质可以反过来,
从产生湮灭算符和一个真空态$\ket{0}$能够得到对称和反对称福克空间,然后退化到单粒子态。

首先我们推导产生湮灭算符的对易关系。我们有
\[
    \begin{split}
        \ket{n+2;M^{(i)}M^{(j)}M^{(k_1)}M^{(k_2)} \cdots} = \hat{a}^\dagger (M^{(i)}) \hat{a}^\dagger (M^{(j)}) \ket{n; M^{(k_1)}M^{(k_2)} \cdots}, \\ 
        \quad \ket{n+2;M^{(i)}M^{(j)}M^{(k_1)}M^{(k_2)} \cdots} = \hat{a}^\dagger (M^{(i)}) \hat{a}^\dagger (M^{(j)}) \ket{n; M^{(k_1)}M^{(k_2)} \cdots},
    \end{split}
\]
从而对$F_S$和$F_A$分别有
\[
    \hat{a}^\dagger (M^{(i)}) \hat{a}^\dagger (M^{(j)}) \ket{n; M^{(k_1)}M^{(k_2)} \cdots} = \pm \hat{a}^\dagger (M^{(i)}) \hat{a}^\dagger (M^{(j)}) \ket{n; M^{(k_1)}M^{(k_2)} \cdots}
\]
$F_S$取正号,$F_A$取负号。考虑到被它们作用的矢量的任意性,得到
\[
    \hat{a}^\dagger (M^{(i)}) \hat{a}^\dagger = \pm \hat{a}^\dagger (M^{(i)}) \hat{a}^\dagger (M^{(j)}).
\]
于是对$F_S$有
\[
    [\hat{a}^\dagger (M^{(i)}), \hat{a}^\dagger (M^{(j)})] = 0,
\]
对$F_A$有
\[
    \acomm{\hat{a}^\dagger (M^{(i)})}{\hat{a}^\dagger (M^{(j)})} = 0.
\]
特别的,
\[
    \hat{a}^\dagger (M^{(i)}) \hat{a}^\dagger (M^{(i)}) = 0.
\]
对以上公式取共轭转置,可以推出,对$F_S$有
\[
    \comm{\hat{a}(M^{(i)})}{\hat{a}(M^{(j)})} = 0,
\]
对$F_A$有
\[
    \acomm{\hat{a}(M^{(i)})}{\hat{a}(M^{(j)})} = 0.
\]
然后再考虑一个产生算符和一个湮灭算符的对易关系。
使用和上面类似的方式,可以推导出
\[
    \hat{a}(M^{(i)}) \hat{a}^\dagger (M^{(j)}) \pm \hat{a}^\dagger (M^{(j)}) \hat{a} (M^{(i)}) = \delta_{ij} \hat{I}.
\]
综上,在$F_S$中有
\begin{equation}
    \comm{\hat{a}(M^{(i)})}{\hat{a}(M^{(j)})} = \comm{\hat{a}^\dagger(M^{(i)})}{\hat{a}^\dagger(M^{(j)})} = 0, \quad \comm{\hat{a} (M^{(i)})}{\hat{a}^\dagger (M^{(j)})} = \delta_{ij},
    \label{eq:commutator-of-ca-op}
\end{equation}
在$F_A$中有
\begin{equation}
    \acomm{\hat{a}(M^{(i)})}{\hat{a}(M^{(j)})} = \acomm{\hat{a}^\dagger(M^{(i)})}{\hat{a}^\dagger(M^{(j)})} = 0, \quad \acomm{\hat{a} (M^{(i)})}{\hat{a}^\dagger (M^{(j)})} = \delta_{ij}.
    \label{eq:anticommutator-of-ca-op}
\end{equation}
从\eqref{eq:anticommutator-of-ca-op}可以看出,在$n_i > 1$时
\[
    \left( \hat{a}^\dagger (M^{(i)}) \right)^{n_i} = 0,
\]
结合\eqref{eq:creation-basis},我们再次发现反对称情况下不可能有重复出现的单粒子态。

我们可以看到,\eqref{eq:commutator-of-ca-op}和\eqref{eq:anticommutator-of-ca-op}的导出和交换两个粒子,态矢量是不是会改变正负号有关系。
这两种方案实际上正好对应\autoref{sec:canonical-quantization}中的对易和反对易关系。
如果交换粒子导致其它的系数,那么就可能制造出除了对易关系和反对易关系以外的算符关系——实际上,的确有这样的情况。
实际上交换粒子导致的系数和底流形的拓扑有一定关系:例如,如果两个粒子交换后再交换回来并不让态矢量复原,那么就可以有除了\eqref{eq:commutator-of-ca-op}和\eqref{eq:anticommutator-of-ca-op}以外的关系。

得到了\eqref{eq:commutator-of-ca-op}和\eqref{eq:anticommutator-of-ca-op}之后,我们再来分析粒子数算符的性质。
首先可以计算出对易关系
\begin{equation}
    \begin{aligned}
        \comm{\hat{n}(M^{(i)})}{\hat{a}(M^{(i)})} = - \hat{a}(M^{(i)}), \quad \comm{\hat{n}(M^{(i)})}{\hat{a}^\dagger(M^{(i)})} = \hat{a}^\dagger(M^{(i)}), \\
        \comm{\hat{n}(M^{(i)})}{\hat{a}(M^{(j)})} = \comm{\hat{n}(M^{(i)})}{\hat{a}^\dagger (M^{(j)})} = 0
        \label{eq:commutator-n-a}
    \end{aligned}
\end{equation}
无论对$F_S$还是$F_A$均成立,这表明产生算符是与它关于同一个态的粒子数算符的升算符,湮灭算符是与它关于同一个态的粒子数算符的降算符。
将此对易关系线性叠加,得到
\begin{equation}
    \comm{\hat{n}}{\hat{a}^\dagger (M^{(i)})} =  \hat{a}^\dagger (M^{(i)}), \quad \comm{\hat{n}}{\hat{a} (M^{(i)})} =  -\hat{a} (M^{(i)}),
    \label{eq:commutator-ntotal-a}
\end{equation}
因此任何$M^{(i)}$标记的产生湮灭算符都是总粒子数算符的升降算符。
此外还有
\begin{equation}
    \comm{\hat{n}(M^{(i)})}{\hat{n}(M^{(j)})} = 0,
    \label{eq:commutator-n-and-n}
\end{equation}
无论$i$和$j$是不是相等、是$F_S$还是$F_A$。
另一方面,从基矢量\eqref{eq:sym-basis}和\eqref{eq:asym-basis}中任取一个,记它当中$M^{(1)}$态有$n_1$个,$M^{(2)}$态有$n_2$个,等等,则它和
\[
    \ket*{n; \underbrace{M^{(1)} M^{(1)} \cdots}_{n_1} \underbrace{M^{(2)} M^{(2)} \cdots}_{n_2} \cdots} 
\]
最多差一个负号,由于$n_1, n_2, \ldots$可以自由地变动而彼此无影响,诸粒子数算符$\hat{n}(M^{(1)}), \hat{n}(M^{(2)}), \ldots$足够完全描述诸态矢量了,从而也足够描述诸态空间了。
因此全体粒子数算符组成的集合是福克空间的一个CSCO。
既然我们使用的本征矢\eqref{eq:sym-basis}和\eqref{eq:asym-basis}是诸粒子数算符的共同本征矢(见\eqref{eq:number-eigenstate}),我们可以将使用诸粒子数算符为CSCO、使用\eqref{eq:sym-basis}或\eqref{eq:asym-basis}为基矢量的福克空间的表象称为\textbf{粒子数表象}或者\textbf{占据数表象}(因为我们讨论占据数$n_1$,$n_2$,等等)。
相应地,可以记基矢量为$\ket{n(M^{(1)}), n(M^{(2)}), \ldots}$。

现在我们选取一条反过来的思路。首先设我们找到了一族算符$\hat{a}(M^{(i)})$,$i=1, 2, \ldots$,并且这些算符满足\eqref{eq:commutator-of-ca-op}或\eqref{eq:anticommutator-of-ca-op}中的其中一个,并满足
\begin{equation}
    \hat{a}(M^{(i)}) \ket{0} = 0,
    \label{eq:annihitation-on-vaccum}
\end{equation}
以及一个唯一的真空态$\ket{0}$。
一旦确认了这一点,按照\eqref{eq:number-operator}和\eqref{eq:total-number-operator}定义粒子数算符,立刻可以推导出\eqref{eq:commutator-n-a},从而\eqref{eq:commutator-ntotal-a}和\eqref{eq:commutator-n-and-n}。
使用\eqref{eq:creation-basis}计算出一系列矢量,这些矢量满足\eqref{eq:creation-operator}和\eqref{eq:annihitation-operator},且容易证明这些矢量张成一个对称(如果使用\eqref{eq:commutator-of-ca-op})或者反对称(如果使用\eqref{eq:anticommutator-of-ca-op})福克空间,且粒子数算符组成的集合构成这个空间上的一个CSCO。
和前面从多粒子态上的产生算符导出湮灭算符时一样,单单依靠\eqref{eq:creation-basis}不能够确定湮灭算符作用在$\ket{0}$上的结果。但既然我们已经有了\eqref{eq:annihitation-on-vaccum},通过对易或反对易关系很容易就能够证明$\hat{a}(M^{(i)})$作用在一个不含$M^{(i)}$态的态上给出$0$。
最后,该福克空间上的单粒子态为
\begin{equation}
    \ket{1;M^{(i)}} = \hat{a}^\dagger (M^{(i)}) \ket{0}
\end{equation}
张成的希尔伯特空间,其上可以定义单粒子算符
\begin{equation}
    \hat{M} = \sum_i M^{(i)} \dyad{M^{(i)}}.
\end{equation}
这表明,一个唯一的真空态,和一组满足\eqref{eq:commutator-of-ca-op}或\eqref{eq:anticommutator-of-ca-op}中的其中一个的算符,就能够完全刻画一个福克空间。

但是很快我们就会注意到一件事:我们从单粒子态出发定义福克空间的时候是有一个内积结构的,而使用一套产生湮灭算符和一个真空态构造出来的福克空间上的内积结构应该怎么定义呢?
实际上并不需要为产生湮灭算符引入额外的结构来描述福克空间的内积。
要看出这是为什么,注意到两个态的内积无非形如
\[
    \begin{aligned}
        \braket{\psi_1}{\psi_2} &\propto \left( \hat{a}(M^{(i_1)})^\dagger \hat{a}(M^{(i_2)})^\dagger \cdots \ket{0} \right)^\dagger \left( \hat{a}(M^{(j_1)})^\dagger \hat{a}(M^{(j_2)})^\dagger \cdots \ket{0} \right) \\
        &= \mel{0}{\cdots \hat{a}(M^{(i_2)}) \hat{a}(M^{(i_1)}) \hat{a}^\dagger(M^{(j_1)}) \hat{a}^\dagger (M^{(j_2)}) \cdots }{0},
    \end{aligned}
\]
推导上式只使用了\eqref{eq:creation-basis},因此在通过产生湮灭算符构造出来的福克空间和使用单粒子态构造出来的福克空间中都成立。
我们总是可以使用对易关系\eqref{eq:commutator-of-ca-op}或者反对易关系\eqref{eq:anticommutator-of-ca-op},把形如$\hat{a}_i \hat{a}^\dagger_j$的表达式写成形如$\hat{a}^\dagger_j \hat{a}_i$的表达式和一个常数之和,不断重复这个步骤,整个
\[
    \cdots \hat{a}(M^{(i_2)}) \hat{a}(M^{(i_1)}) \hat{a}^\dagger(M^{(j_1)}) \hat{a}^\dagger (M^{(j_2)}) \cdots
\]
就能够写成几种\textbf{正规序列}——也就是所有产生算符都在所有湮灭算符左边的算符连乘积,还有若干常数的线性组合。线性组合的系数和线性组合中出现的常数完全由\eqref{eq:commutator-of-ca-op}或\eqref{eq:anticommutator-of-ca-op}确定。
因此,只要使用产生湮灭算符的代数关系求出正规序列的真空态期望,就能够使用产生湮灭算符的代数结构刻画福克空间的内积。
如果正规序列中有湮灭算符,那么其真空态期望就是零,因为
\[
    \mel{0}{\cdots \hat{a}(M^{(i)})}{0} = \left( \cdots \ket{0} \right)^\dagger \hat{a}(M^{(i)}) \ket{0} = 0.
\]
而如果正规序列中只有产生算符,那么由于
\[
    \mel{0}{\hat{a}^\dagger(M^{(i)}) \hat{a}^\dagger(M^{(j)}) \cdots}{0} = \mel{0}{\cdots \hat{a}(M^{(j)}) \hat{a}(M^{(i)})}{0}^*,
\]
其真空态期望照样是零。因此正规序列的真空态期望一定是零。
因此,福克空间的内积结构也被产生湮灭算符描述了。

最后我们讨论表象变换。以上所有的讨论使用的都是$M$表象。现在如果我们使用$G$表象的单粒子态构造出了另外一套福克空间,这两个空间之间有什么样的联系?
综合使用\eqref{eq:sym-basis}和\eqref{eq:asym-basis}以及\eqref{eq:creation-basis},还有我们熟悉的单粒子态的表象变换公式
\[
    \ket{M^{(i)}} = \sum_j \ket{G^{(j)}} \braket{G^{(j)}}{M^{(i)}},
\]
可以发现,设$\hat{a}(M^{(i)})$是用$M$表象建立起来的福克空间的湮灭算符,$\hat{a}(G^{(i)})$是用$G$表象建立起来的福克空间的湮灭算符,且两个空间同时使对称空间或者同时是反对称空间,那么
\begin{equation}
    \begin{bigcase}
        \hat{a}^\dagger (M^{(i)}) = \sum_j \braket{G^{(j)}}{M^{(i)}} \hat{a}^\dagger (G^{(j)}), \\
        \hat{a} (M^{(i)}) = \sum_j \braket{M^{(i)}}{G^{(j)}} \hat{a} (G^{(j)}).
    \end{bigcase}
    \label{eq:creation-and-annihitation-trans}
\end{equation}
反之,在已有一套$\hat{a}(M^{(i)})$满足\eqref{eq:creation-and-annihitation-trans}且有真空态,从而能够使用这套算符构造一个$M$表象的对称福克空间时,使用\eqref{eq:creation-and-annihitation-trans}可以得到另一套$\hat{a}(G^{(i)})$,使用它们可以构造出一个$G$表象的对称福克空间;反对称同理。

\subsubsection{二次量子化形式的算符}

在\autoref{sec:n-particle-space}中已经说明,多粒子系统中有意义的可观察量在粒子交换下不变。
本节讨论所有这样的有意义的可观察量的形式。
我们首先考虑能够在多粒子态基底\eqref{eq:sym-basis}或\eqref{eq:asym-basis}下被对角化的算符。设$\hat{A}$是一个这样的算符,则
\[
    \hat{A} = \sum_i \sum_{j_1, j_2, \ldots, j_i} \hat{A}^{(i)}_{j_1 j_2 \ldots j_i},
\]
上标$(i)$表示这是定义在$i$粒子希尔伯特空间中的算符,下标$j_1, j_2, \ldots$表示这是作用在第$j_1, j_2, \ldots$号粒子上的算符。
如果序列$j_1, j_2, \ldots$中有重复的序号,那么相应的$\hat{A}^{(i)}_{j_1 j_2 \ldots j_i}$实际上完全可以使用只涉及$i-1$粒子空间甚至粒子数目更少的空间上的算符来表示,因此不失一般性地,我们要求
\[
    \hat{A} = \sum_i \sum_{j_1 \neq j_2 \neq \ldots \neq j_i} \hat{A}^{(i)}_{j_1 j_2 \ldots j_i}.
\]
考虑到$\hat{A}$是有意义的可观察量,它和任何一个粒子交换算符都是对易的。
这又意味着,所有的$\hat{A}^{(i)}_{j_1 j_2 \ldots j_i}$都是厄米算符,且它们和粒子交换算符都是对易的。
因此如果两个序列$j_1, j_2, \ldots, j_i$ \\
和$j'_1, j'_2, \ldots, j'_i$中出现的数完全相同,它们对应的$\hat{A}^{(i)}_{j_1 j_2 \ldots j_i}$和$\hat{A}^{(i)}_{j'_1 j'_2 \ldots j'_i}$就是完全相同的算符。
因此通过调整每一项所含的算符前面的系数我们可以写出
\begin{equation}
    \hat{A} = \sum_i \sum_{j_1 < j_2 < \ldots < j_i} \hat{A}^{(i)}_{j_1 j_2 \ldots j_i},
    \label{eq:fock-observable-original}
\end{equation}
求和号中包含的所有项都是厄米的,且它们各不相同。如果处理严格递增的指标比较麻烦,可以等价地写出
\begin{equation}
    \hat{A} = \sum_i \frac{1}{i!} \sum_{j_1 \neq j_2 \neq \ldots \neq j_i} \hat{A}^{(i)}_{j_1 j_2 \ldots j_i}.
    \label{eq:fock-observable}
\end{equation}
\eqref{eq:fock-observable}和\eqref{eq:fock-observable-original}中的$\hat{A}^{(i)}_{j_1 j_2 \ldots j_i}$完全相同;\eqref{eq:fock-observable}中的阶乘系数是为了消除下标重复计数,因为含有$i$个各不相同的数的序列有$i!$种排列方式,每种排列方式对应的$\hat{A}^{(i)}_{j_1 j_2 \ldots j_i}$都完全一样,且都对应着\eqref{eq:fock-observable-original}中要求$j_1 < j_2 < \ldots$的那个$\hat{A}^{(i)}_{j_1 j_2 \ldots j_i}$。
再次,由于$\hat{A}$是有意义的可观察量,\eqref{eq:fock-observable}中的$\hat{A}^{(i)}_{j_1 j_2 \ldots j_i}$和$\hat{A}^{(i)}_{j'_1 j'_2 \ldots j'_i}$只相差了一个粒子交换变换(请注意这是作用在算符上的变换而不是作用在态上的),即使它们涉及编号不同的粒子。
反之,如果任何两个$\hat{A}^{(i)}_{j_1 j_2 \ldots j_i}$和$\hat{A}^{(i)}_{j'_1 j'_2 \ldots j'_i}$只相差一个粒子交换变换,且对彼此互为重排的两个序列
\[
    j_1, j_2, \ldots, j_i, \quad j'_1, j'_2, \ldots, j'_i
\]
它们对应的$\hat{A}^{(i)}$完全相同,
那么通过\eqref{eq:fock-observable}或等价的\eqref{eq:fock-observable-original}给出的$\hat{A}$一定是福克空间上的可观察量。由于\eqref{eq:fock-observable}中第二个求和号涉及的任何一个$j_1, j_2, \ldots$这样的序列都不含有重复的元素,诸$\hat{A}^{(i)}_{j_1 j_2 \ldots j_i}$实际上可以通过将一个作用在$i$粒子对称化或反对称化希尔伯特空间上的算符作用在$j_1, j_2, \ldots$号粒子上得到,也就是说,\eqref{eq:fock-observable}中对$i$的求和号中的每一项都是使用同一个$i$粒子算符作用在不同的粒子上得出的。
因此我们有最后的结论:算符$\hat{A}$是福克空间上可使用\eqref{eq:sym-basis}或\eqref{eq:asym-basis}对角化的可观察量,当且仅当,可以找到一系列定义在$i$粒子希尔伯特空间中的可观察量$\hat{A}^{(i)}$,使得\eqref{eq:fock-observable}或等价的\eqref{eq:fock-observable-original}成立,其中$\hat{A}^{(i)}_{j_1 j_2 \ldots j_i}$指的是$\hat{A}^{(i)}$在$j_1, j_2, \ldots, j_i$号粒子上的作用。

需要注意的是,现在我们有两种“$i$粒子算符”:一种是定义在$i$粒子希尔伯特空间下的算符,一种是定义在福克空间中的,可以从前者根据\eqref{eq:fock-observable}导出的算符。
前者在粒子交换变换下未必不变,但后者在粒子交换变换下肯定不变;前者是一次量子化的,后者是二次量子化的。
为了区分,同时强调二次量子化方法可以处理含有任意多的粒子的情况,我们称后者为\textbf{$i$体算符}。

现在再尝试使用产生湮灭算符写出\eqref{eq:fock-observable}中的每一项。
对粒子的编号完全是任意的(当然也可以是任意的,既然交换粒子之后什么都没有变),因此我们将随时采用最方便的安排方式。
设$\hat{A}^{(i)}$是一个定义在$i$粒子希尔伯特空间上的算符,且它与粒子交换算符对易。
% 下面的说法仍然有很大的直觉考量因素。能否从数学上严格处理?
% 在$n < i$时,
% \[
%     \hat{A}^{(i)}_{j_1 j_2 \ldots j_i} \ket{n;M^{(k_1)} M^{(k_2)} \cdots M^{(k_n)}} = 0,
% \]
% 而在$n \geq i$时,
% \[
%     \begin{aligned}
%         &\quad \hat{A}^{(i)}_{j_1 j_2 \ldots j_i} \ket{n; M^{(k_1)} M^{(k_2)} \cdots M^{(k_n)}} \\
%         &= \begin{bigcase}
%             \sum_{l_1, l_2, \ldots, l_i} \ket{n; M^{(k'_1) M^{(k'_2)}} \cdots M^{(k'_n)}} \mel{M^{(l_1)} M^{(l_2)} \cdots M^{(l_i)}}{\hat{A}^{(i)}}{M^{(k_{j_1})} M^{(k_{j_2})} \cdots M^{(k_{j_i})}}, \\ \text{with no repetition in $M^{(k_1)}, M^{(k_2)}, \ldots$}, \\
%             0, \quad \text{with repetition.}
%         \end{bigcase},
%     \end{aligned}
% \]
% 其中
% \[
%     k'_a = \begin{cases}
%         k_a, \quad a \neq j_1, j_2, \ldots, j_i, \\
%         l_b, \quad a = j_b, \; b = 1, 2, \ldots, i.
%     \end{cases}
% \]
% TODO:实际上以上两个式子就是“\hat{A}^{(i)}作用在n粒子态上”的定义,为什么可以这么定义?
那么,$\hat{A}^{(i)}_{j_1 j_2 \ldots j_i}$就是
\[
    \begin{aligned}
        &\hat{A}^{(i)}_{j_1 j_2 \ldots j_i} 
        = \sum_{k_1, k_2, \ldots, k_i} \sum_{l_1, l_2, \ldots, l_i} \\
        &\ket{M^{(k_1)}}_{j_1} \ket{M^{(k_2)}}_{j_2} \cdots \ket{M^{(k_i)}}_{j_i} 
        \bra{M^{(l_1)}}_{j_1} \bra{M^{(l_2)}}_{j_2} \cdots \bra{M^{(l_i)}}_{j_i} \\ 
        &\mel{M^{(k_1)} M^{(k_2)} \cdots M^{(k_i)}}{\hat{A}^{(i)}}{M^{(l_1)} M^{(l_2)} \cdots M^{(l_i)}}.
    \end{aligned}
\]
我们把$\hat{A}^{(i)}_{j_1 j_2 \ldots j_i}$作用在一个任意粒子数的基矢量上,其结果是将上式中的右矢和多粒子态基矢量做张量缩并的产物。
可以看出,
\[
    \begin{aligned}
        &\quad \sum_{j_1 \neq j_2 \neq \ldots \neq j_i} \ket{M^{(k_1)}}_{j_1} \ket{M^{(k_2)}}_{j_2} \cdots \ket{M^{(k_i)}}_{j_i} 
        \bra{M^{(l_1)}}_{j_1} \bra{M^{(l_2)}}_{j_2} \cdots \bra{M^{(l_i)}}_{j_i} \\
        &= \hat{a}^\dagger (M^{(k_1)}) \hat{a}^\dagger (M^{(k_2)}) \cdots \hat{a}^\dagger (M^{(k_i)}) \hat{a} (M^{(l_i)}) \cdots \hat{a} (M^{(l_2)}) \hat{a} (M^{(l_1)})  
    \end{aligned}
\]
结合以上各式,我们得出:
\begin{equation}
    \begin{aligned}
        \hat{A} = \sum_i \frac{1}{i!} &\sum_{k_1, k_2, \ldots, k_i} \sum_{l_1, l_2, \ldots, l_i} \\
        &\mel{M^{(k_1)} M^{(k_2)} \cdots M^{(k_i)}}{\hat{A}^{(i)}}{M^{(l_1)} M^{(l_2)} \cdots M^{(l_i)}} \\
        &\hat{a}^\dagger (M^{(k_1)}) \hat{a}^\dagger (M^{(k_2)}) \cdots \hat{a}^\dagger (M^{(k_i)}) \hat{a} (M^{(l_i)}) \cdots \hat{a} (M^{(l_2)}) \hat{a} (M^{(l_1)}).
        \label{eq:general-form-of-fock-observable}
    \end{aligned}
\end{equation}
\eqref{eq:general-form-of-fock-observable}给出了福克空间上的可使用多粒子态基矢量对角化的可观察量的一般形式。
特别的,一个$n$体(二次量子化)可观察量一定具有形式
\begin{equation}
    \begin{aligned}
        \hat{A} = &\frac{1}{n!} \sum_{k_1, k_2, \ldots, k_n} \sum_{l_1, l_2, \ldots, l_n} A_{k_1 k_2 \cdots k_n l_1 l_2 \cdots l_n} \\
        &\hat{a}^\dagger (M^{(k_1)}) \hat{a}^\dagger (M^{(k_2)}) \cdots \hat{a}^\dagger (M^{(k_n)}) \hat{a} (M^{(l_n)}) \cdots \hat{a} (M^{(l_2)}) \hat{a} (M^{(l_1)}),
    \end{aligned}
    \label{eq:n-particles-observable}
\end{equation}
其中系数$A_{k_1 k_2 \cdots k_n l_1 l_2 \cdots l_n}$在交换$k$和$l$以后应该和原来的值相差一个共轭变换,因为它是一个定义在$n$粒子希尔伯特空间上的厄米算符的分量矩阵。
占据数算符是一种特殊的单体算符(它是二次量子化算符,不是一次量子化算符)。

同样本节的论述可以双向地使用:给定一系列$\hat{A}^{(1)}, \hat{A}^{(2)}, \ldots$,总是可以使用\eqref{eq:general-form-of-fock-observable}构造出福克空间上的可观察量;反之,假定我们从一套产生湮灭算符和一个真空态构造出了一个福克空间,并且确认$\hat{A}$是这个福克空间上可用多粒子态基矢量对角化的可观察量,那么它一定可以展开成\eqref{eq:general-form-of-fock-observable}的形式。

以上讨论了可以使用\eqref{eq:sym-basis}或\eqref{eq:asym-basis}对角化的二次量子化可观察量。
还有一类二次量子化可观察量不能使用多粒子态基矢量对角化。在$n$粒子空间中不可能定义这样的可观察量,但在福克空间中可以,因此这一类可观察量不能和$n$个单粒子态空间的张量积上的算符建立类似于\eqref{eq:fock-observable}这样的联系。
一个不能使用多粒子态矢量对角化的可观察量的例子:
\[
    \hat{a}(M^{(i)}) + \hat{a}^\dagger(M^{(i)}).
\]
这一类二次量子化可观察量写成产生湮灭算符的多项式后,同一项内产生算符的数目和湮灭算符不一样。
无论如何,对任何一个算符,它在乘积态基矢量下的矩阵元和把它写成产生湮灭算符的多项式的各项系数之间是一一对应的:
\begin{equation}
    \hat{A} = \sum_{(i_1, i_2, \ldots), (j_1, j_2, \ldots)} \cdots \hat{a}(M^{(i_2)}) \hat{a}(M^{(i_1)}) \mel{M^{(i_1)} M^{(i_2)} \cdots}{\hat{A}}{M^{(j_1)} M^{(j_2)} \cdots} \hat{a}^\dagger(M^{(i_1)}) \hat{a}^\dagger(M^{(i_2)}) \cdots.
\end{equation}

\subsubsection{关于连续谱的注记}

以上讨论都建立在单粒子态可以使用离散谱描述的基础上。连续谱的情况基本上是一样的,只不过要将“对所有基的求和”改成积分,并且将$\delta_{ij}$改成$\delta(x - x')$。
需要注意的是可能会使用不同的积分测度,一般的,若有
\[
    \sum_i \longrightarrow \int \dd[n]{x} f(x),
\]
则有
\[
    \delta_{ij} \longrightarrow \frac{1}{f(x)} \delta^n(x - x').
\]
另外,由于连续谱情况下,福克空间中的多粒子态中两个粒子具有完全一样的状态的概率是零(点在连续谱中是零测的),所以可以忽略这种情况,而有
\begin{equation}
    \ket{\vb*{x}_1, \vb*{x}_2, \ldots, \vb*{x}_n} = \hat{a}^\dagger(\vb*{x}_1) \hat{a}^\dagger(\vb*{x}_2) \cdots \hat{a}^\dagger (\vb*{x}_n ) \ket{0}.
\end{equation}

在连续谱的情况中还有另一个问题需要注意。一般的,一个单粒子算符可以展开为
\[
    \mel{x}{\hat{A}}{\psi} = \mathcal{A} \braket{x}{\psi},
\]
如果算符$\hat{A}$是对角化的,那么$\mathcal{A}$就是一个数,否则它是一个作用在一个经典场上的算符。傅里叶变换可以证明即使$\hat{A}$不能对角化,$\hat{A}$的二次量子化形式也是
\[
    \hat{A}_\text{sq} = \int \dd{x} \hat{a}^\dagger (x) \mathcal{A} \hat{a}(x), 
\]
$\mathcal{A}$作用在湮灭算符$\hat{a}(x)$上。

\subsection{多体系统的动力学}\label{sec:many-body-dynamics}

% TODO:自由场的格林函数、格林算符的矩阵元等实际上就是一次量子化的格林函数、格林算符的矩阵元

\subsubsection{二次量子化哈密顿量}\label{sec:second-quant-hamiltonian}

% TODO:首先我们有二次量子化的哈密顿量,然后不同的守恒荷的值将相空间分成了不同的部分,每个部分内部的运动可以使用一个与二次量子化哈密顿量不同的哈密顿量描述。
% 需要做的则是讨论这两者的关系。
% 例如,考虑一个薛定谔场和静电场耦合的系统,无论系统中有多少粒子,关于场算符的哈密顿量也就是所谓的“二次量子化哈密顿量”都是完全一样的;另一方面,我们可以写出$n$粒子的这种系统的哈密顿量,这完全就是一个粒子数确定的单粒子量子力学的哈密顿量。前者是二次量子化哈密顿量,后者是一次量子化哈密顿量;对各守恒荷给定的态,一次量子化哈密顿量和二次量子化哈密顿量作用于其上得到完全一样的结果。
% 可以考虑将“多粒子态算符”分为一次量子化算符和二次量子化算符

\autoref{sec:many-body-state}节描述了描述全同粒子系统态的基本框架,现在我们来让这些态动起来。
由于产生湮灭算符足够描写福克空间,我们可以使用产生湮灭算符构造哈氏量(它本身应该是一个可观察量!);哈氏量中不应该显式出现一次量子化算符。
从而,只含有一种粒子的哈氏量的基本形式为
\begin{equation}
    \begin{split}
        \hat{H} = \sum_i c_i \hat{a}^\dagger (M^{(i)}) \hat{a} (M^{(i)}) + \underbrace{\sum_{m, n} S_{mn} \hat{a}^\dagger (M^{(m)}) \hat{a} (M^{(n)}) }_\text{one particle terms} + \\
        \underbrace{\frac{1}{2} \sum_{m, n, p, q} D_{mnpq} \hat{a}^\dagger (M^{(m)}) \hat{a}^\dagger (M^{(n)}) \hat{a} (M^{(q)}) \hat{a} (M^{(p)}) }_\text{two particles terms} + \cdots \\
        + \text{creation and annihilation terms},
    \end{split}
    \label{eq:one-kind-many-body-hamitonian}
\end{equation}
其中单粒子项、二粒子项等项均取\eqref{eq:general-form-of-fock-observable}的形式,产生湮灭项指的是不能使用多粒子基矢量\eqref{eq:sym-basis}或\eqref{eq:asym-basis}对角化的项,也就是不能和某个粒子数固定的希尔伯特空间上的可观察量建立类似于\eqref{eq:fock-observable}这样的对应的那些项,或者说每一项内产生算符的数目和湮灭算符不一样的项。我们称它们为产生湮灭项的原因是,系统粒子数守恒的充要条件是
\begin{equation}
    0 = \comm*{\hat{N}}{\hat{H}} = \sum_i \comm*{\hat{a}^\dagger (M^{(i)}) \hat{a} (M^{(i)})}{\hat{H}},
\end{equation}
而如果没有所谓的产生湮灭项,那么哈密顿量可以写成一系列形如
\[
    \hat{a}^\dagger (M^{(k_1)}) \hat{a}^\dagger (M^{(k_2)}) \cdots \hat{a}^\dagger (M^{(k_i)}) \hat{a} (M^{(l_i)}) \cdots \hat{a} (M^{(l_2)}) \hat{a} (M^{(l_1)})
\]
的算符的线性组合。无论是费米子还是玻色子,都有
\[
    \comm*{\hat{a}^\dagger (M^{(k_1)}) \hat{a}^\dagger (M^{(k_2)}) \cdots \hat{a}^\dagger (M^{(k_i)}) \hat{a} (M^{(l_i)}) \cdots \hat{a} (M^{(l_2)}) \hat{a} (M^{(l_1)})}{\hat{N}} = 0,
\]
因此在没有产生湮灭项时粒子数一定是守恒的。
没有产生湮灭项等价于粒子数守恒;事实上,这还等价于系统具有\textbf{$U(1)$对称性},即具有和复数辐角旋转相同的对称性。这是因为,对系统做$U(1)$变换就是做变换
\[
    \hat{a}^\dagger(M^{(k)}) \longrightarrow \ee^{\ii \alpha} \hat{a}^\dagger(M^{(k)}), \quad \hat{a}(M^{(k)}) \longrightarrow \ee^{-\ii \alpha} \hat{a}(M^{(k)}),
\]
哈密顿量中没有产生湮灭项意味着哈密顿量中的每一项都含有相同数目的产生算符和湮灭算符,所以诸$\ee^{\ii \alpha}$因子和$\ee^{-\ii \alpha}$因子相互抵消了,于是,哈密顿量在$U(1)$变换下保持不变。
我们可以看到粒子数正是$U(1)$对称性对应的守恒荷。

当然,哈氏量中还可以出现导数,但是通过对相应的算符做傅里叶变换总是可以把所有导数都变成系数,从而得到\eqref{eq:one-kind-many-body-hamitonian}型的哈密顿量。

在哈密顿量仅仅含有\eqref{eq:one-kind-many-body-hamitonian}中第一项的时候,也即,总是可以找到一组表象而把哈密顿量写成
\begin{equation}
    \hat{H} = \sum_i \epsilon_i \hat{a}^\dagger_i \hat{a}_i,
    \label{eq:free-field-hamiltonian}
\end{equation}
的时候,我们有
\[
    \dv{t} \hat{a}(M^{(i)}) = - \ii c_i \hat{a} (M^{(i)}),
\]
这不论是在$F_S$还是$F_A$上都是成立的。因此就有
\[
    \hat{a} (M^{(i)}) = \ee^{- \ii t} \hat{a}_0 (M^{(i)}).
\]
因此这样的哈密顿量描述了一个没有相互作用的体系。这样的体系实际上对应着一个自由场。
在这样的体系中,也就只有在这样的体系中,场做简谐运动是被允许的。当然,自由场中粒子数是守恒的。实际上,这也是仅有的我们能够解析计算出二次量子化体系能级的情况——有相互作用不方便处理能级。
当且仅当系统中的粒子之间没有相互作用时我们可以把二次量子化哈密顿量写成\eqref{eq:free-field-hamiltonian}的形式;找到一组表象将二次量子化哈密顿量写成\eqref{eq:free-field-hamiltonian}的形式实际上就是要求解单粒子的能谱,然后可以讨论(单粒子哈密顿量的)某个能级上有几个粒子,等等。
在这里,“能级”一词有两种意思,一个是场的能级,即所有
\[
    E = \sum_i n_i \epsilon_i, \quad n_i = 0, 1, 2, \ldots,
\]
一个是粒子的能级,即
\[
    \epsilon_1, \epsilon_2, \ldots.
\]

含有多种粒子的哈密顿量的形式则为
\begin{equation}
    \hat{H} = \sum \hat{H}_\text{single} + \hat{H}_\text{interaction},
\end{equation}
其中每一个$\hat{H}_\text{single}$均取\eqref{eq:one-kind-many-body-hamitonian}的形式,而相互作用部分则取关于不同的粒子的\eqref{eq:general-form-of-fock-observable}的乘积。
显然,要出现粒子数的变化,相互作用部分同样应该出现产生算符的数目和湮灭算符不一样的项,如
\[
    \hat{a} \hat{b} + \hat{b}^\dagger \hat{a}^\dagger
\]

需要注意的是,由于产生湮灭算符可以服从\eqref{eq:commutator-of-ca-op}也可以服从\eqref{eq:anticommutator-of-ca-op},哈密顿量的形式相同绝对不意味着系统的行为相同。
例如,同样是自由哈密顿量,玻色子可以取同样的量子态而费米子不行,为什么哈密顿量相同而竟然有不同的物理现象?
因为玻色子的产生湮灭算符和费米子的产生湮灭算符的对易关系不一样。

\subsubsection{元激发}

在有相互作用时,粒子之间会发生相互作用,会出现粒子的产生和湮灭,因此并无稳定存在、可以追踪轨迹的单个粒子。%
\footnote{
    有必要指出的是,有相互作用的场论中仍然可以有稳定的、不会发生衰变的多粒子态,这样的多粒子态也是带相互作用的哈密顿量的本征态。
    我们的世界就是一个这样的例子:电子和光子之间存在耦合,即存在相互作用,但是当然存在“可以一直向前飞行”的单个电子,即存在单电子态,它是带有相互作用的QED哈密顿量的本征态,有定义良好的能量,等等。
    但是需要注意的是,此时的多粒子态并非场算符作用在修正后真空上得到的。
    相互作用对系统能谱的修正体现在两方面:一方面,真空态从$\ket{0}$变成了一个一般难以直接算出的$\ket{\Omega}$;另一方面,带有相互作用的哈密顿量的多粒子本征态$\ket{M^{(1)} M^{(2)} \cdots}$不再是$\hat{a}^\dagger(M^{(1)}) \hat{a}^\dagger(M^{(2)}) \cdots \ket{0}$,也不是$\hat{a}^\dagger(M^{(1)}) \hat{a}^\dagger(M^{(2)}) \cdots \ket{\Omega}$,因为只有哈密顿量形如$\epsilon \hat{a}^\dagger \hat{a}$时才能够用$\hat{a}^\dagger$的连乘积构造出所有能量本征态,现在既然相互作用破坏了哈密顿量的这个形式,带有相互作用的哈密顿量的多粒子本征态也就无法用场算符写出了。
    相互作用的存在还意味着多粒子态的能量不再是构成它的所有粒子的单粒子态的能量的叠加。
    这意味着我们不再能够将多粒子态严格地看成某个谐振子模型的本征态,从而,永远不可能通过适当地重新定义场算符,将哈密顿量写成自由哈密顿量。

    以带有相互作用的哈密顿量的多粒子本征态为基底计算$S$矩阵矩阵元某种意义上相当于在做重整化,将可能的虚过程——跃迁到高粒子数状态再跃迁回来——通过参数跑动来给出。
    因此,虽然
}%
有时我们会发现,对已有的产生湮灭算符做一个线性组合可以得到另一种产生湮灭算符,它们能够大大简化哈密顿量的形式,这种凭空构造出来的产生湮灭算符描述的多粒子态称为\textbf{准粒子}或者\textbf{元激发},它们是粒子的运动模式。
类似的,准粒子之间也会有相互作用,会有准粒子的产生湮灭,等等。
通常的习惯是,将玻色型的准粒子称为元激发,而将费米型的准粒子称为准粒子。

准粒子可以分成\textbf{个别激发}和\textbf{集体激发},前者的准粒子产生湮灭算符是单个实际粒子的产生湮灭算符做线性变换之后的结果,后者的准粒子产生湮灭算符则是多个实际粒子的产生湮灭算符的线性组合。
个别激发的典型例子包括\textbf{空穴},即做变换
\[
    \hat{b}^\dagger(M^{(i)}) = \hat{a}(M^{(i)}),
\]
以实际粒子的不存在为准粒子。
另一个个别激发的例子是\textbf{裹粒子},即某个实际粒子(相应地称为\textbf{裸粒子})射入已有的一个体系,从而该实际粒子一边发生运动,体系内原有的粒子跟着它运动而形成的模式。
集体激发的例子包括系统的CSCO的一部分可以看成一个场(无论坐标是连续的还是离散的),从而可以通过一个线性变换得到这个场对应的产生湮灭算符;如果系统的一部分能量本征态可以使用一套产生算符从一个“基态”(未必是定义系统时取的真空态)构造出来,那么我们也获得了一种集体激发。

还有一些激发只涉及少数几个粒子。例如如果一种粒子总是成对出现或者消失,那么可以认为$\hat{a}^\dagger_1 \hat{a}^\dagger_2$是一种准粒子。
可以递归地证明,这种复合粒子的自旋(就是组成它的各个粒子的自旋之和)如果是半整数,则它是费米子,否则是玻色子。
由于轨道角动量的$j$一定是整数(空间旋转群是角动量代数的矢量表示,不是旋量表示,因此$j$不可能是半整数),我们也可以讨论复合粒子的角动量代数,$j$为偶数时为玻色子,为奇数时是费米子。

对准粒子而言的真空态和实际粒子的真空态通常是不一样的。实际上,虽然我们使用了准粒子和实际粒子的区分,它们仅有的区别是真空态不相同,我们并不能够保证现在所认为的实际粒子——比如,电子、夸克等——不是某些更为基本的粒子形成的准粒子。

虽然准粒子可以认为是场的特殊模式从而可以定义其产生湮灭算符,但很多时候它们的产生湮灭算符并没有特别简单的对易关系。
因此,在推导描述准粒子的低能有效理论时未必会用其产生湮灭算符为基本的自由度。
实际上,也许我们今天认为的“基本粒子”也不是基本的自由度。这个思路导致了弦论和其它代替标准的“粒子是场的激发”的理论的理论框架。

常用的获得准粒子或者说元激发的方法是,设场构型满足某些特殊的条件(如矢量场无源或者无旋等),然后在此基础上对场构型做特殊的变换,得到准粒子产生湮灭算符。
设在不同的约束条件下得到了不同的准粒子,我们可以将原场$\phi$分解成满足不同约束条件的一系列场$\phi_i$,如果原来的场动力学关于$\phi$是线性的,那么不同准粒子之间就没有相互作用;否则就有相互作用。

% TODO:设有$n$个动力学自由度的情况下出现了基态简并,则一定存在多余的动力学自由度能够区分这些基态。在二次量子化的框架中,这就是说简并的出现意味着系统中有一些隐藏的激发,不同的简并态具有不同的激发;例如,对称性自发破缺之后一定会多出来一种激发,在破缺的对称性是连续的局域对称性的情况下这就是Goldstone模,如果破缺的对称性是全局的那这可能是拓扑序,等等

以上讨论了哈密顿量。关于二次量子化系统的绘景还需要说一句:在哈密顿量不显含时间时,海森堡绘景中的产生湮灭算符产生出的二次量子化基矢量以薛定谔绘景的方式演化。

\subsubsection{从量子场论到多粒子体系}\label{sec:from-qft-to-many-body}

以上我们讨论了怎样从单粒子态得到全同多粒子态,以及全同多粒子态可以使用一个真空态和一组产生湮灭算符产生。
现在讨论一个反向的问题:我们需要的这种产生湮灭算符从何处来?
马上可以看到,我们需要的这种产生湮灭算符实际上就是算符场的一个线性组合。
我们并不在意场算符$\hat{\phi}$是矢量、旋量还是就是标量,因为不管怎么样,$\hat{\phi}^\dagger \hat{\phi}$肯定是标量,因此粒子数算符也是标量,从而不会产生任何矛盾。
% TODO 但是下面的讨论明显把场算符当成了标量

以下我们讨论自由场。这无损一般性,因为相互作用绘景意味着我们能够将带相互作用的问题的态空间和自由场的态空间使用某个幺正变换(也就是相互作用绘景的时间演化算符)相互联系,而只要从自由场场算符构造出一套产生湮灭算符,与场的真空态——也就是所有场都为零的一个唯一的态——我们就完全刻画了自由场的态空间。
\autoref{sec:canonical-quantization}中提到了两种量子化方案:其一是对易方案\eqref{eq:symmetry-commutator},其二是反对易方案\eqref{eq:antisymmetry-commutator}。
对前者,在某个固定的时间点$t$我们设
%
\footnote{具体系数是多少无关紧要,系数只是为了让产生湮灭算符的对易子给出正确的系数。}
\[
    \hat{\phi}^i(\vb*{x}, t) = \frac{1}{\sqrt{2}} \left( \hat{a}_i^\dagger (\vb*{x}, t) + \hat{b}_i(\vb*{x}, t) \right), \quad 
    \hat{\pi}_i(\vb*{x}, t) = \frac{\ii}{\sqrt{2}} \left( \hat{a}_i^\dagger (\vb*{x}, t) - \hat{b}_i (\vb*{x}, t) \right)
\]
考虑到$\hat{\phi}$和$\hat{\pi}$的厄米性,有
\[
    \hat{a}_i^\dagger = \hat{b}^i,
\]
于是
\begin{equation}
    \hat{\phi}^i(\vb*{x}, t) = \frac{1}{\sqrt{2}} \left( \hat{a}_i^\dagger (\vb*{x}, t) + \hat{a}_i(\vb*{x}, t) \right), \quad 
    \hat{\pi}_i(\vb*{x}, t) = \frac{\ii}{\sqrt{2}} \left( \hat{a}_i^\dagger (\vb*{x}, t) - \hat{a}_i (\vb*{x}, t) \right).
\end{equation}
切换到自然单位制下,取$\hbar=1$,则通过\eqref{eq:symmetry-commutator}计算得到
\begin{equation}
    \comm*{\hat{a}_i(\vb*{x}, t)}{\hat{a}_j^\dagger (\vb*{y}, t)} = \delta_{ij} \delta (\vb*{x} - \vb*{y}), \quad \comm*{\hat{a}_i(\vb*{x}, t)}{\hat{a}_j(\vb*{y}, t)} = \comm*{\hat{a}_i^\dagger (\vb*{x}, t)}{\hat{a}_j^\dagger (\vb*{y}, t)} = 0.
    \label{eq:ca-op-from-field}
\end{equation}
因此对一个固定的时间$t$,$\hat{a}_i^\dagger(\vb*{x}, t)$和$\hat{a}_i (\vb*{x}, t)$构成了一组产生湮灭算符,由它们产生的单粒子态可以完全被$\vb*{x}$和$i$标记。
从而,由它们产生的单粒子态可以分解成两个空间$\{\ket{\vb*{x}}\}_{\vb*{x}}$和$\{\ket{i}\}_i$的直积;
单粒子态的CSCO为位置算符$\hat{\vb*{x}}$(当然也可以是与之能够相互导出的另一些算符,比如动量),还有与$\{\ket{i}\}_i$有关的算符。
我们称前者为\textbf{空间自由度},而称后者为\textbf{内禀自由度},因为后者来自于场算符的内部结构(有几个分量、在$\Lambda$作用下如何变换,等等)。%
\footnote{如果场算符的取值有某些限制(例如,电磁波一定是横波,等等),那么不能够保证$i$的可能取值的数目一定是场算符的分量数。}
总之,量子化方案\eqref{eq:symmetry-commutator}给出的场算符导致一个对称的福克空间$F_S$。我们称这样的福克空间描述的粒子为\textbf{玻色子}。
如果场满足特殊的时空对称性——比如洛伦兹对称性——那么内禀自由度实际上由这种时空对称性中不改变单粒子态的$\vb*{x}$(或者也可以是不改变$\vb*{p}$)的群(称为\textbf{小群})确定,这个群的李代数在单粒子态上的表示的Cartan子代数就是内禀自由度空间的CSCO。
当然,也可以通过讨论场的内禀自由度来分析单粒子态的内禀自由度。

% TODO:实际上,内禀自由度还可以用来表示复合粒子。我们将两个场$\psi, \phi$写成一个列向量$\Psi = \pmqty{\psi & \phi}^\top$,它是洛伦兹群的一个可约表示;$\Psi(x)$可以分解成产生算符和湮灭算符之和,它就描述了由$\psi$场对应的粒子和$\phi$场对应的粒子复合而成的粒子。如果哈密顿量可以近似地用$\Psi$表示且不涉及其内部自由度(这又涉及到重整化的概念:$\Psi$的内部自由度仅和较高能的过程耦合,在低能下可以被略去),那么就不必考虑复合粒子的内部结构。

对于量子化方案\eqref{eq:antisymmetry-commutator},我们不能够使用\eqref{eq:ca-op-from-field},因为这不能导出正确的对易关系。
实际上由于使用这种量子化方案的场一定是复的,相当于两个厄米场,我们需要引入两对产生湮灭算符。

我们看到,费米子和玻色子的概念的导出实际上是两种量子化方案\eqref{eq:symmetry-commutator}和\eqref{eq:antisymmetry-commutator}的必然推论。
值得注意的是,在讨论场的动力学或者粒子的动力学时,我们都是首先获得了一组能够完全描述态空间的可观察量,然后用它们标记不同的态;而在二次量子化中,我们是首先通过带标记$\vb*{x}$或$i$等的产生湮灭算符得到了一组多粒子态,并自然地使用这些量标记多粒子态,然后再讨论关于单粒子、二粒子……的可观察量。

之前我们提到过,场表示自然诱导出时空对称性在这个算符场作用的希尔伯特空间上的表示。
场算符诱导出的产生湮灭算符刻画了一个由单粒子态空间、二粒子态空间……直和而成的福克空间,这表明,算符场作用的希尔伯特空间实际上是一个可约表示。
它取某个粒子数的子空间则是一个不可约表示。由于$n$粒子态空间就是$n$个单粒子态空间的直积的对称化或者反对称化,我们只需要讨论单粒子态即可。于是现在我们把注意力转移到单粒子态上。
完全刻画空间自由度只需要一个位置算符就可以了,或者当然也可以使用与之满足正则对易关系的动量算符。

% TODO:场表示的标签,比如$P_\mu P^\mu$给出的$m^2$,无论是在场表示还是其$n$粒子态表示中,都是一样的。天晓得这是不是真的。。。

% TODO:哈密顿量的全体本征态就是单粒子态、二粒子态、三粒子态……

粒子是场的激发,也即,“系统中的粒子有某某某某”和“系统的态矢量是真空态作用了某某场的产生算符”是等价的说法。
这个说法的经典版本意味着,场值的某种模式(如平面波)实际上就对应着一群粒子(如平面波对应一群动量确定的粒子)。在量子理论中事情稍微复杂一些,因为物理量和系统状态分别对应算符和态矢量;场的状态由一个态矢量描述,因此场算符本身的作用是诱导出产生湮灭算符,而系统中有几个粒子、是什么样的粒子的信息则储存在态矢量中。

经典极限:能量较高,从而粒子性不明显,但又不能太高,否则圈图修正等会明显起来

\subsubsection{回退到一次量子化}

以上我们论证了,有粒子生灭的系统需要使用福克空间描述,并且对应一个场论。
那么如果粒子生灭比较微弱,以至于在一段时间内可以认为粒子数不变,我们其实还是可以使用粒子数固定的一次量子化方法讨论问题。
态空间不必多言,比较重要的是动力学。当然,二次量子化哈密顿量在$n$粒子希尔伯特空间下的投影就是我们需要的一次量子化哈密顿量。
实际上,如果我们知道了二次量子化哈密顿量遵循的限制(或者说对应的场论遵循的限制,比如对称性),那么也可以使用一次量子化哈密顿量反推二次量子化哈密顿量。

另一个值得说明的事实是,我们有
\[
    \ket{\vb*{x}} = \hat{\phi}(\vb*{x}, t) \ket{0}
\]
然后我们会发现关于$\hat{\phi}$的海森堡绘景方程实际上就是关于$\ket{\vb*{x}}$的薛定谔绘景方程。

% TODO:二次量子化场满足的方程和单粒子量子力学的波动方程之间的关系

\subsubsection{关于费米子系统的注记}

在一开始使用对称化和反对称化条件来约束希尔伯特空间时,我们可能会觉得玻色子系统和费米子系统的希尔伯特空间看起来非常不同,但粒子数表象告诉我们,它们的差别并没有那么大——仅有的差别在于玻色子系统的希尔伯特空间上,每个模式上可以有可数个粒子,而费米子系统的希尔伯特空间上每个模式上或是没有粒子或是只有一个粒子。
(请注意纯粹的粒子数表象中无所谓态矢量对称或是反对称,因为根本没有算符可以“交换”两个粒子!)
如果玻色子系统的哈密顿量具有某些特殊的性质,使得我们可以忽略占据数大于$2$的态(如所谓“硬球玻色子气体”),那么玻色子系统和费米子系统的希尔伯特空间就是完全一样的。
实际上,如果我们在每个粒子模式上放置一个$1/2$自旋自由度,所得的希尔伯特空间和费米子系统、硬球玻色子气体的希尔伯特空间都是一样的。%
\footnote{
    当然,如果我们坚持采用“多粒子态对称化/反对称化”的方法,那么的确,费米子系统和玻色子系统的希尔伯特空间看起来很不同,但应该注意到,实际上粒子数表象才是分析多体系统时真正自然的表象。
}%

这就产生了一个很大的问题:费米子系统到底特殊在哪里?以上论述实际上暗示我们,哈密顿量的形式可以区分费米子系统和玻色子系统。
但是,\autoref{sec:second-quant-hamiltonian}中给出的多体系统的哈密顿量的形式对费米子系统和对玻色子系统的形式都是一样的,似乎并没有区分这两者的可能。
这里的微妙之处在于\autoref{sec:second-quant-hamiltonian}中的哈密顿量都是使用产生湮灭算符表述的,而费米子的产生湮灭算符实际上是一个细看起来非常奇特的构造:它本质上不是局域的,因为局域性要求不同空间点的算符对易而不是反对易。
之所以费米子系统仍然能够满足局域性条件是因为哈密顿量的形式凑得刚刚好,使得一点的产生湮灭算符和另一点的哈密顿量密度能够对易。
在粒子数表象下费米子系统和玻色子系统的希尔伯特空间看起来没有特别大的区别,但是如果设法仅仅用局域的算符给出哈密顿量,那么费米子系统的哈密顿量的形式将会变得非常特殊,并且可能有非局域性。
通常使用的产生湮灭算符表述的哈密顿量(“二次量子化哈密顿量”)将这件事隐藏了起来。
但实际上,即使是产生湮灭算符表述的哈密顿量也给出了足够的暗示:我们使用费米子产生湮灭算符的反对易关系时实际上并不需要知道关于希尔伯特空间的任何信息,我们只是在抽象的算符代数中处理问题,让费米子和玻色子不同的是算符代数而不是希尔伯特空间。

实际上,整个二次量子化理论的出发点都是“我们有良定义的单粒子态,需要将它们组合为合理的多粒子态”,然后我们发现多体系统的量子力学实际上就是这个多体系统的产生湮灭算符的量子力学,这就得到了量子场论。
但是,从多粒子态引出量子场论的构造实际上并不依赖“单粒子态如何组成多粒子态”,因此实际上我们可以一开始就丢弃多粒子态的概念!
设体系的时间演化由以下路径积分表述给出:
\[
    \mel{f}{\hat{U}}{i} = \sum_{\text{paths from $i$ to $f$}} \ee^{\frac{\ii}{\hbar} S[\text{path}]},
\]
如果$S$中有一些关于演化路径的拓扑不变量,将路径分类为不同的拓扑等价类,则
\[
    \mel{f}{\hat{U}}{i} = \sum_{\text{topo class}} \ee^{\frac{\ii}{\hbar} S_\text{top}[\text{path}]} \sum_{\text{path}} \ee^{\frac{\ii}{\hbar} S_\text{dyn}[\text{path}]},
\]
由于$\ee^{\frac{\ii}{\hbar} S_\text{top}[\text{path}]}$具有可乘性,它实际上是关于演化路径的拓扑性质的某个群的表示。
在初末态给定时这就是空间的braiding group。
将$S_\text{top}$设为零,就给出玻色子的情况;在演化路径中有$n$次粒子交换时让$\ee^{\ii S_\text{top} / \hbar}$取$(-1)^n$,就给出费米子的情况。
对三维空间,其braiding group为$\mathbb{Z}_2$,因此以上两种表示已经是全部可能了,即只有玻色子和费米子;而对二维空间还有任意子。

\subsubsection{经典极限}

需要注意的是经典理论中的场的叠加在量子情况中对应着乘积态;这是合理的,因为经典理论中场经过叠加之后还是场,叠加前后都是场算符的本征态;量子理论中对应不同场量的态叠加得到的显然不是场算符的本征态。
“经典”体现在假定时间演化前后系统状态始终是场本征态。

\section{微扰论}

本节讨论\textbf{微扰论},即在系统的哈密顿量中引入小的扰动之后观察系统行为的技术。这样做是非常必要的——很多时候,哈密顿量可以写成
\begin{equation}
    \hat{H} = \hat{H}_0 + \hat{H}',
\end{equation}
其中$\hat{H}_0$是一个很容易对角化的算符,其谱为
\begin{equation}
    \hat{H}_0 \ket*{n^{(0)}} = E^{(0)}_n \ket*{n^{(0)}},
\end{equation}
而$\hat{H}'$是某种相互作用。
很多时候,即使扰动很大,我们也使用微扰论,这会导致微扰展开仅仅具有渐进级数的意义,但是足够提取出很多有用的信息。

\subsection{不含时微扰论}

\textbf{不含时微扰论}指的是在$\hat{H}'$不含时时,我们对$\ket*{n^{(0)}}$和$E^{(0)}_n$做微扰修正的过程。
这个思路在$\hat{H}'$含时时不适用,因为此时根本没有定态。

\subsubsection{无简并的情况}

由于能量本征值和本征态都需要做修正,不含时微扰论可以非常繁琐。我们先考虑没有任何能量简并的情况,此时能量本身足以标记能量本征态。
设
\begin{equation}
    \hat{H} = \hat{H}_0 + \lambda \hat{H}',
\end{equation}
其中$\lambda$是一个小参数。将能量本征态和能量本征值对$\lambda$做展开,即
\begin{equation}
    \ket{n} = \sum_{k=0}^\infty \lambda^n \ket*{n^{(k)}}, \quad E_n = \sum_{k=0}^\infty \lambda^n E_n^{(k)},
\end{equation}
代入本征方程
\[
    \hat{H} \ket{n} = E_n \ket{n},
\]
令方程左右两边$\lambda$次数相同的项相等,就得到
\begin{equation}
    \hat{H}_0 \ket*{n^{(0)}} = E^{(0)}_n \ket*{n^{(0)}},
\end{equation}
以及
\begin{equation}
    \hat{H}_0 \ket*{n^{(k)}} + \hat{H}' \ket*{n^{(k-1)}} = \sum_{k'=0}^k E_n^{k'} \ket*{n^{(k-k')}}, \quad k \geq 1.
\end{equation}
此外,我们有

如果一阶简并微扰论能够完全解除简并,那么做到这一阶就够了,后面的都可以通过非简并微扰论计算;如果一阶简并微扰论没有解除任何微扰,那么可以考虑二阶简并微扰论。

大部分实际上会用到的不含时微扰论都可以统一在近简并微扰论的框架下,即挑选出希尔伯特空间的一个子空间,观察相互作用哈密顿量在其上的扰动。

\subsection{含时微扰论,Dyson级数和跃迁}

\subsubsection{Dyson级数}

现在我们转而讨论\textbf{含时微扰论}。在含时微扰论中我们不再讨论加入$\hat{H}_\text{int}$之后能级怎么修正,因为此时哈密顿量可以随时间变化。
我们转而讨论在相互作用下系统从一个态开始演化,经过一段时间之后做一次观察,落到另一个态的概率,也即,要计算
\[
    \braket*{n^{(0)}(t_2)}{m^{(0)}(t_1)} = \mel{n^{(0)}(t_2)}{\hat{U}(t_2, t_1)}{m^{(0)}(t_1)}.
\]
实际上这就完全描述了系统的行为,因为显然这给出了时间演化算符的全部矩阵元。

设系统的自由哈密顿量为$\hat{H}_0$,它的一组基矢量为$\{\ket{m}\}$,本征值记为$\{E_m\}$(由于不需要计算任何加入微扰之后的本征态,我们略去上标$^{(0)}$;暂不考虑它们的时间演化),相互作用哈密顿量为$\hat{H}'$。
态随时间的演化为
\[
    \ket{\psi(t)} = \sum_n a_n(t) \ket{n} \ee^{- \ii E_n t / \hbar},
\]
显然$t=0$时除了$a_m=1$以外其它$a$均为零。考虑诸$\{a_k\}$的时间演化,我们有
\begin{equation}
    \hat{c}_n(t) = \sum_n c_n(t) \ee^{- \ii E_n t} \ket*{n},
\end{equation}
自相迭代就得到
\begin{equation}
    \begin{aligned}
        c_n(t) &= c_n(0) + \sum_m \int_0^t \dd{t} (- \ii \mel{n}{\hat{H}'(t_1)}{m}) \ee^{- \ii (E_m - E_n) t} c_m(0) \\
        &- \sum_{m, k} \int_0^t \dd{t_1} \int_0^{t_1} \dd{t_2} \mel{n}{\hat{H}'(t_1)}{m} \ee^{- \ii (E_m - E_n) t} \mel{m}{\hat{H}'(t_2)}{k} \ee^{- \ii (E_k - E_m) t} c_k(0) \\
        &+ \cdots,
    \end{aligned}
\end{equation}
这就是\textbf{Dyson级数}。应当指出$c_n(t)$等实际上是相互作用绘景下的态矢量分量,而等号中的矩阵元则是薛定谔绘景下的,$\ee$因子正是绘景变换。我们当然还可以进一步扩充Dyson级数的适用范围,得到
\begin{equation}
    \begin{aligned}
        c_n(t) &= c_n(t') + \sum_m \int_{t'}^t \dd{t_1} (- \ii \mel{n}{\hat{H}'(t_1)}{m}) \ee^{- \ii (E_m - E_n) t_1} c_m(t') \\
        &- \sum_{m, k} \int_{t'}^t \dd{t_1} \int_{t'}^t \dd{t_2} \mel{n}{\hat{H}'(t_1)}{m} \ee^{- \ii (E_m - E_n) t_1} \mel{m}{\hat{H}'(t_2)}{k} \ee^{- \ii (E_k - E_m) t_2} c_k(t') \\
        &+ \cdots.
    \end{aligned}
\end{equation}

没有什么理由认为微扰展开的级数一定收敛。实际上,在比较复杂的问题中——比如说大多数量子场论问题中——它都是发散的。
这是正确的,因为很容易想到量子理论中有一些非微扰效应,它们本身是非常良定义的,但是却不能被一个有限阶的微扰展开捕捉到,典型的例子是量子隧穿,它导致一个指数衰减$A \ee^{- x}$,如果要求全空间成立,那么它不能被任何一个多项式恰当地拟合。
我们看到了一个看起来非常怪诞但是完全正确的现象:有的时候,计算有限阶微扰会得到非常大的结果,而一口气把所有阶都算上却反而会得到正确的结果!这就是所谓的\textbf{非微扰效应}。
虽然有非微扰效应存在,如果将微扰论看成完全形式的从一个量子场论到一个关于形式级数等比较“看得见”的对象的范畴的函子,它仍然是非常有用的。
例如,如果我们能够设法将整个微扰无穷级数中的一部分无穷级数求和(即\textbf{重求和}),那么还是可以捕捉到非微扰效应。

\subsubsection{周期性策动}

再考虑一个略有不同的情况——外界作用周期性地施加在
% TODO

% TODO:什么时候可以将算符用其期望替代,因为这会改变对易关系。。

\subsubsection{散射}

\textbf{散射}指的是一种$\hat{H}'$实际上不含时的过程,通常我们讨论从$t=-\infty$到$t=\infty$的过程,这意味着我们忽略所有“中间态”,假定物理过程发生得非常快,从而可以简单地将散射看成“一个态经过一个变换变成另一个态”的一个变换。

% TODO:T矩阵,S矩阵,似乎前者是关于“跃迁率”的,即每秒事件发生的可能性,后者是关于无限长时间的散射
散射的微扰计算可以是相互作用绘景下的(粒子物理中的常见方法),也可以通过近似求解总哈密顿量$\hat{H}$的能谱(多见于单粒子量子力学,在量子场论中几乎不可行),即所谓求解\textbf{散射定态}。

似乎$U(1)$场论的格林函数的单体部分就是直接使用薛定谔方程算出来的格林函数。

\begin{equation}
    \hat{T} = \hat{H}' + \hat{H}' \frac{1}{E_i - \hat{H}_0} \hat{T}
\end{equation}
是能量为$E_i$的散射过程的$T$矩阵。实际上这个$T$矩阵就是$S$矩阵扣除恒等过程得到的$T$矩阵,只不过差一个常数,在这里可以证明
\begin{equation}
    S_{fi} = 1 - 2\pi \delta(E_f - E_i) \ii T_{fi},
\end{equation}
而量子场论中常见的定义是
\begin{equation}
    S = 1 + \ii T.
\end{equation}
定义
\begin{equation}
    \hat{T} \ket*{i} = \hat{H}' \ket*{\psi^+_i}
\end{equation}
即可得到
\begin{equation}
    \ket*{\psi^+_i} = \ket*{i} + \frac{1}{E_i - \hat{H}_0} \hat{H'} \ket*{\psi^+_i},
\end{equation}
这就是“散射导致入射波增加了一个散射波”的方程,即\textbf{李普曼-施温格方程}。散射振幅实际上就藏在这个方程中,因为
\begin{equation}
    \begin{aligned}
        \ket*{\psi^+_i} &= \ket*{i} + \frac{1}{E_i - \hat{H}_0} \hat{H}' \ket*{\psi_i} \\
        &= \ket*{i} + \sum_n \frac{1}{E_i - \hat{H}_0} \mel*{n}{\hat{H}'}{\psi^+_i} \\
        &= \ket*{i} + \sum_n \frac{1}{E_i - \hat{H}_0} \ket*{n} \mel{n}{\hat{T}}{i}. 
    \end{aligned}
\end{equation}
容易验证$\ket*{\psi_i^+}$在$E_i$取$\hat{H}$的本征值的时候实际上就是散射定态,而由于散射态中能谱连续分布,这总是可行的。

由于计算以$\ket*{i}$为初态的散射过程时需要用到$\ket*{\psi}^+_i$,我们可以说,$\ket*{\psi}^+_i$是对应于$\ket*{i}$的散射定态。

任何一个幺正过程都对应一个散射过程。

似乎可以认为以入射态$\ket{i}$为初态的系统经过散射和充分长的时间之后可以近似认为收敛到(适当归一化后的)$\ket{\psi}_i^+$上

\subsubsection{费米黄金法则}

假定相互作用哈密顿量不显含时间,不失一般性地设系统初态为$\ket{m}$,使用Dyson级数并只计算到一阶,有
\[
    \ii \hbar a^{(1)}_k(t) = \sum_n \int \dd{t'} \mel{k}{\hat{H}'}{n} \ee^{\ii \omega_{kn} t'} a_n^{(0)},
\]
其中
\[
    \omega_{mn} = \frac{E_m - E_n}{\hbar}.
\]
$a_n^{(0)}$只在$n=m$时有非零值,且时间演化是从$t'=0$演化到$t'=t$,于是
\[
    \begin{aligned}
        \ii \hbar a^{(1)}_k(t) &= \sum_n \int \dd{t'} \mel{k}{\hat{H}'}{n} \ee^{\ii \omega_{kn} t'} a_n^{(0)} \\
        &= \mel{k}{\hat{H}'}{m} \frac{\sin \omega_{km} t / 2}{\omega_{km} / 2} \ee^{\ii \omega_{km} t / 2}. 
    \end{aligned}
\]
注意到$m \neq k$时$a_k = a_k^{(1)}$,而$a_k(t)$的模长平方正是$t'=0$时系统状态为$\ket{m}$而$t'=t$时系统经过观测状态为$\ket{k}$的概率,此概率就是所谓的\textbf{跃迁概率},于是跃迁概率的表达式就是
\begin{equation}
    P_k(t) = \frac{4 \abs*{\mel{k}{\hat{H}'}{m}}^2}{\hbar^2} \frac{\sin^2 \omega_{km} t / 2}{\omega_{km}^2}.
\end{equation}
现在假如系统实际上是一个开放系统,且诸$\ket{m}$构成一组偏好基,则可以使用一个经典马尔可夫过程来描述系统的演化,而使用经典的态(各个态出现的概率就是$\ket{m}$的振幅的模长平方)描述系统的状态。
系统每个时刻都有一定概率跃迁,也有一定概率不跃迁而等待下一时刻,此时跃迁速率为% TODO:从量子转向经典地一般描述,还有类似于“经典电磁场和量子粒子耦合”这样有经典有量子的系统
\begin{equation}
    \Gamma_k(t) = \dv{P_k}{t} = \frac{2 \abs*{\mel{k}{\hat{H}'}{m}^2}}{\hbar^2} \frac{\sin \omega_{km} t}{\omega_{km}}.
\end{equation}
如果我们要计算系统跃迁到一系列态上的概率,那么总跃迁速率为
\[
    \Gamma(t) = \sum_k \Gamma_k(t) = \sum_{E_k} \frac{2 \abs*{\mel{k}{\hat{H}'}{m}^2}}{\hbar^2} \frac{\sin \omega_{km} t}{\omega_{km}},
\]
让$\omega$连续取值,并引入态密度来表示哪些态是允许的,有
\begin{equation}
    \begin{aligned}
        \Gamma(t) &= \int \dd{E} \rho(E) \frac{2 \abs*{\mel{k}{\hat{H}'}{m}^2}}{\hbar^2} \frac{\sin \omega t}{\omega} \\
        &= \int \dd{\omega} \rho(E) \frac{2 \abs*{\mel{k}{\hat{H}'}{m}^2}}{\hbar} \frac{\sin \omega t}{\omega}.
    \end{aligned}
\end{equation}
其中$E = \hbar \omega + E_m$。如果能够保证能量守恒,并且可以持续很长时间都有跃迁,则$t$很大,于是$\sin \omega t / \omega$是高度振荡的函数,则可以把态密度$\rho(E)$提到积分号外面,而积分号变成跃迁矩阵元的平均乘以$\sin \omega t / \omega$的积分值,最后计算得到
\begin{equation}
    \Gamma(t) = \frac{2\pi}{\hbar} \rho(E) \expval{\abs*{\mel{k}{\hat{H}'}{m}^2}}_k.
\end{equation}
这样,如果我们有数目巨大的一系列完全相同的系统,总数为$N$,它们之间相互影响很小,那么在$\dd{t}$时间内,发生跃迁的系统的数目几乎确定为$N \Gamma(t) \dd{t}$。
因此可以列出某个状态的系统的数目服从的微分方程。

实际上以上推导可以推广到更高阶近似,我们有
\[
    \Gamma = \frac{2\pi}{\hbar} \rho(E) \abs*{T}^2.
\]

\subsubsection{振幅学}

我们已经讨论了如何使用微扰论计算散射振幅;实际上,有一些办法利用量子演化过程的幺正性直接将高阶修正转化为一阶近似,这种方法即\textbf{幺正性方法}。

本节采用记号
\[
    \hat{S} = 1 + \ii \hat{T}.
\]
由$\hat{S}$的幺正性,有
\begin{equation}
    2 \Im \mel*{f}{\hat{T}}{i} = \sum_k \mel{f}{\hat{T}}{k} \mel*{k}{\hat{T}}{i}. 
\end{equation}

总之就是高阶圈图可以写成低阶圈图的函数,低阶圈图可以写成树图的形式。
这当然不让人奇怪,因为相互作用顶角原则上给出了关于系统的动力学的全部信息,而树图可以给出全部关于相互作用顶角的信息。

\subsubsection{绝热演化}

量子绝热定理:
Berry相因子为
\begin{equation}
    \gamma_n = \int_0^t \dd{t'} \braket{\phi_n(t')}{\pdv{t'} \phi_n(t')}
\end{equation}

\subsection{费曼图}

一些物理量和它们对应的图:

$\mel{0}{\phi_1 \phi_2 \phi_3}{0}$对应没有相互作用顶角的连线

$\mel{0}{\phi_1 \phi_2 \ee^{-\ii H t} \phi_3}{0}$对应所有入射、出射粒子线正确的费曼图

(海森堡绘景下的)$\mel{\Omega}{\phi_1 \phi_2 \phi_3}{\Omega}$对应没有真空气泡图的费曼图(可以不完全连通,只要图的所有部分都至少连到一个入射/出射线上)

配分函数对应真空气泡图,可以连通也可以不连通;自由能对应连通的真空气泡图

以有相互作用时的单粒子态为基底计算矩阵元时为避免难以处理的发散,应当将单粒子自能修正单独重求和,即只计算amputated的图,然后乘上自能修正,通常包括场重整化导致的因子和参数跑动。
计算S矩阵时通常以经过自能修正的单粒子态为基底,从而不能简单地将编时关联函数中的时间差推到无穷大。(有相互作用时相互作用编时关联函数相当于$S$矩阵在一个相当奇怪的基底下的矩阵元,这些基底不是无相互作用多粒子态$\hat{a}^\dagger_1 \hat{a}^\dagger_2 \cdots \ket{0}$,也不是做了完整的自能修正的、带有相互作用的哈密顿量的多粒子本征态,而是$\hat{a}^\dagger_1 \hat{a}^\dagger_2 \cdots \ket{\Omega}$)

\subsection{多粒子系统}

前面所说的所有微扰论都是非常一般的,没有涉及任何系统的具体性质。很容易看出,可以使用图形规则(所谓\textbf{费曼图})来表示这些级数:系统先根据$\hat{H}_0$“传播”一会,然后根据$\hat{H}_\text{int}$发生一次相互作用,然后再根据$\hat{H}_0$“传播”一会……相互作用的次数就是微扰的阶数,把所有阶的项加起来就得到精确的结果。
如果系统实际上只是单粒子,那么这个图像就非常直观:一个粒子不受干扰地自由运动一会,然后发生一次散射,然后再自由运动一会……但是如果系统是多粒子系统——或者,正如我们在二次量子化一节中看到的那样,是一个以场为自由度的系统——那么“不受干扰自由运动”的实际上是整个体系而不是单个粒子。
对二次量子化系统而言,无相互作用情况下可以将每个粒子看成独立的,因此我们希望有一种能够以单粒子为基础的微扰论。这就是本节要展示的。

在多粒子系统中我们想要计算什么物理量?最容易想到的显然是所谓的\textbf{格林函数}。如前所述,只需要计算出时间演化算符的矩阵元,就可以得到系统的一切行为,而在二次量子化中多粒子态构成了系统的一组基,那么只需要计算出在一个时间点向系统中放入一些粒子,在另一个时间点做以多粒子态为基的观测而得到的概率振幅就可以了,而考虑到多粒子态可以使用产生湮灭算符构造,这就是说要计算
\[
    \mel{\Omega}{\hat{a}_{n_1} \hat{a}_{n_2} \cdots \hat{U}(t_2, t_1) \cdots \hat{a}^\dagger_{m_2} \hat{a}^\dagger_{m_1}}{\Omega}
\]
其中所有的产生湮灭算符都定义在同一个时刻。这些量被称为格林函数是因为它们实际上可以看成某个外加场策动下的响应函数(不同的初态可以用不同的外加场和系统中的场的耦合方式策动,如单粒子初态可以使用一个形如$\hat{a}^\dagger J$的相互作用哈密顿量策动,二粒子初态可以用形如$\hat{a}^\dagger \hat{a}^\dagger J$的相互作用哈密顿量策动),而在自由场的情况下场的运动方程是线性的,那么这样的响应函数就是场方程的格林函数。

有必要指出的是,只有在自由场的情况下如上定义的格林函数才是时间演化算符在哈密顿量的(多粒子)本征态下的矩阵元。
在有相互作用时,$\hat{a}^\dagger_{m_2} \hat{a}^\dagger_{m_1} \ket{\Omega}$并不是哈密顿量的本征态,虽然它可以看成某种多粒子态,但它并非稳定的、不发生衰变的、实验上能够制备出来的那种多粒子态。

在这里,我们使用$\ket*{\Omega}$来表示真空,而不是$\ket*{0}$,是因为一般来说一个加入了相互作用的系统的真空态和它的自由场论的真空态会有一些不同。

无论怎么展开,都会涉及到哈密顿量的自乘,有自乘就有费曼图。这就是费曼图可以用来计算有效哈密顿量的原因。

费曼图中,费米子线出现交叉意味着差一个负号。

费曼图与真空:真空图(无外线的图)给出了真空零点能;费曼图中非外线的粒子称为虚粒子(虚粒子的出现有时称为真空极化),树图直接对应着经典图景,而圈图实际上是外线和真空图连接,所以圈图可以看成是“和真空发生相互作用”。
handwaving的图像:圈图的出现是因为真空零点能。(当然这个说法是不非常严格的)
真空零点能肯定会出现否则系统就没有动力学(?)

由于

% TODO: 费曼图
费曼关联函数,又称两点格林函数
\begin{equation}
    G(x, y) = \mel{\Omega}{T \hat{\phi} (x) \hat{\phi}^\dagger(y) }{\Omega},
\end{equation}

四点格林函数
\begin{equation}
    G^{(4)} (x_1, x_2, x_3, x_4) = \mel{\Omega}{T \hat{\phi} (x_1) \hat{\phi}(x_2) \hat{\phi}^\dagger(x_3) \hat{\phi}^\dagger (x_4) }{\Omega}
\end{equation}

自由场
\begin{equation}
    \Delta (x, y) = \mel{0}{T \hat{\phi} (x) \hat{\phi}^\dagger(y) }{0}
\end{equation}
自由场的传播子确实是场方程的格林函数。
% TODO:为什么会有这个巧合?

% TODO
费曼图中的元素可以分成几种:
\begin{itemize}
    \item 节点,即几条柄的交点,实际上就是相互作用哈密顿量中的一项
    \item 柄,表示一种场,连接到同一个节点的柄关于同一个时空坐标,连接到不同的节点的柄关于不同的时空坐标
    \item 外腿,有一段不连接任何东西的线,表示产生湮灭算符(就是$\mel{0}{\hat{a}_{\vb*{p'}}\hat{S}\hat{a}^\dagger_{\vb*{p}}}{0}$中左右那两个),可以看成一种特殊的柄
    \item 线,连接两个柄或者一个柄一个外腿或者两个外腿。连接两个柄的时候它表示传播子$\mel{0}{T\hat{\phi}(x)\hat{\psi}(y)}{0}$,连接一个外腿一个柄的时候表示$\mel{0}{T\hat{\phi}(x)\hat{a}^\dagger_{\vb*{p}}}{0}$,连接两个外腿的时候表示$\mel{0}{\hat{a}_{\vb*{p}}\hat{a}^\dagger_{\vb*{p'}}}{0}$。只有同类的柄才能用同样的线条连接。
    \item 源,表示一个给定的外场的贡献。
\end{itemize}
节点出现几个,就说明这是几阶微扰。

% 费曼图中来自同一节点的线实际上代表一个多粒子态,可以使用一个把这些线覆盖住的圈表示一个态。

绘制费曼图的方法:先画出外腿和节点,彼此间不相连,然后尝试使用线把它们连起来;同一张图能够使用多少种方法连接出来,它在最后的展开式中就会出现多少次,费曼图越对称它重复出现的次数就越多。

从费曼图写出$\hat{S}$算符的矩阵元的方法称为\textbf{费曼规则}。
获得费曼规则的方法:使用不同的线条表示不同的场,然后计算每条线可能表示的因子,一张图中每一条线表示的因子全部乘起来然后对所有涉及的时空坐标(每个节点涉及一个时空坐标)求积分,最后乘上对称性因子和$\alpha^n/n!$($\alpha$指的是相互作用哈密顿量前面的系数,也即是“关联常数”)之类的量就得到了这一阶微扰。$n!$除以费曼图重复的次数就是对称性因子。

由于传播子实际上是自由场的格林函数,它通常有
\[
    \int \frac{\dd[3]{\vb*{p}}}{(2\pi)^3} \ee^{\ii p \cdot x} f(p)
\]
的形式,$f(p)$为动量空间中的传播子。
将对时空坐标的积分应用到这些指数函数上就得到了动量空间中的费曼图:可以使用一个值不确定的动量标记每条线,并且对这些未确定的动量积分:
\[
    \int \frac{\dd[4]p}{(2\pi)^4}
\]
因此费曼图中的线可以被认为是行进中的粒子。
容易看出所有可能的费曼图就是粒子所有可能的碰撞方式。有一些线并不来自任何一条外腿,但是由于相互作用的特征(例如在$\phi^4$理论中任何一个节点都连着四条线)必须有,它们就是所谓的虚粒子。它们的动量是不定的,也就是说它们不必遵循动量守恒。

由于费曼图中的节点实际上是相互作用哈密顿量的一项,由费曼规则,我们可以发现相互作用哈密顿量中的每一项都可以被赋予“允许某个过程发生”的意义,例如$\lambda \hat{a}_1^\dagger \hat{a}_2$可以理解为一个类型2的粒子转化为类型1的粒子的过程,设$\hat{\phi}$是实场,那么产生和湮灭的方向是随意的,例如$\phi^4$可以表示一个粒子转化为三个粒子,两个粒子转化为两个粒子,三个粒子转化为一个粒子。% TODO:说明$\hat{\phi}(\vb*{x})$虽然不是产生湮灭算符,在计算格林函数和费曼图时却可以当成产生湮灭算符用

费曼图也可以用于计算有效哈密顿量,因为从有效哈密顿量计算出的低能空间的散射振幅必须和从原哈密顿量计算出的低能空间的散射振幅一致,因此初末态均为低能过程但中间涉及高能过程的图也必须被考虑进从有效哈密顿量计算出的低能空间的散射振幅中,也即,我们把每一张中间涉及高能过程的部分遮盖掉,使用一个常数代替它,这就是重整化群作用下参数跑动的原因。

自能修正实际上也是这样的过程:我们把所有的相互作用都使用一个节点(正规自能)代替,做无穷求和,得到
\[
    G = G^0 + G^0 \Sigma^* G^0 + G^0 \Sigma^* G^0 \Sigma^* G^0 + \cdots,
\]
然后使用等比数列推导出Dyson方程
\[
    G = G^0 + G^0 \Sigma^* G.
\]
% 换而言之,我们把所有相互作用打包成了$V$,写出有效哈密顿量$H = H_0 + \Sigma$
也可以将相互作用引入的东西收集到一个自能中:
\[
    G = G^0 + G^0 \Sigma G^0,
\]
然后容易看出
\[
    \Sigma = \Sigma^* + G^0 \Sigma^* G^0 + \cdots,
\]
从而
\[
    \Sigma = \Sigma^* + \Sigma^* G^0 \Sigma.
\]
自能和正规自能有很明显的费曼图意义:前者就是单粒子格林函数的费曼图中所有可能的基本的相互作用图形之和,后者就是将前者中的传播子切断后得到的“构件”之和。
由于$G^0$就是$G$的零阶近似,我们称以上计算过程为\textbf{自能修正}。

可以对单粒子线做修正,当然也可以对中间粒子做修正,即所谓\textbf{真空极化}。

还有所谓顶角函数,就是二

正规图形可以画成做了完整的自能修正、顶角修正、真空极化修正等修正的一阶图形,即可以画成\textbf{骨架图形}之和。

如果一个二次量子化哈密顿量中只含有单体算符,那么无论入射粒子有几个,其任意阶的费曼图中都不可能出现粒子间相互作用(正如我们所预期的),因此这种问题可以直接化归为单粒子问题。

% 总之一切都可以通过自由场格林函数算出来,而自由场格林函数又遵循Wick定理
% 另外Peskin上面说上述方法实际上只是严格使用$\ket{\Omega}$推出来的方法的一个简化版(见111页)
% 使用$\ket{\Omega}$的格林函数是真空态格林函数除以一个归一化因子,但是在计算散射截面时所有的归一化因子都抵消了

由于傅里叶变换同时也是坐标表象和动量表象之间的变换,对关于坐标的场算符给出的格林函数做傅里叶变换,就得到了对关于动量的场算符给出的格林函数。

关于费曼图需要说一句:如果相互作用哈密顿量形如$\int \dd[d]{\vb*{x}_1} \dd[d]{\vb*{x}_2} \hat{\psi}^\dagger(\vb*{x}_1) \hat{\psi}^\dagger(\vb*{x}_2) \hat{\psi}(\vb*{x}_2) \hat{\psi}(\vb*{x}_1)$,那么虽然这是四体相互作用,它对应的顶角还是两个$\psi$粒子连接一条虚线。
这其实也好理解,因为这种类型的“超距”作用总是可以表示成积掉了某种中间粒子的非超距相互作用,虚线就是那个中间粒子。
但此时“一阶微扰”为带有两个顶角的图,“二阶微扰”为带有四个顶角的图,等等;顶角数目同时也是虚线数目的两倍。
在根据费曼图书写关联函数时,需要将两个配对的顶角和虚线整体认读,因为毕竟中间粒子还是已经被积掉了。

此时可以看到忽略交换对称性得到的经典能量和交换-关联能实际上就是费曼图中不同的两个图形而已;后者在没有超距作用时不存在。

真的完全按照费曼规则算图是非常麻烦的,所幸有不少简化手段。

表面上看,微扰散射理论似乎给出了求解一个体系的一般方法——实际上,甚至可以在微扰散射理论的基础上建立量子力学!但很多时候相互作用实际上根本没法做微扰,级数展开都是发散的。

\section{路径积分量子化}

通过生成泛函求导计算出来的是编时格林函数。这可以按照定义验证,但实际上是非常合理的:交换求导次序,生成泛函的导数不变或加一个负号,而交换算符顺序,不变或只是加一个负号的只有编时格林函数。
乍一看这是非常奇怪的,因为生成泛函求导显然对时间和空间是平等的,然而编时格林函数仅仅有时间上的排序算符而没有任何空间上的特殊处理。
然而,这是因为编时格林函数是在正则量子化表述中定义的,而正则量子化中时间和空间本身并不平等(具体来说,同一个时间点,不同空间点的算符或是对易或是反对易,但是不同时间点就未必),所以需要某种机制将时间特别拎出来处理,从而恢复时间和空间的平等性,也是非常合理的。

% TODO:哈密顿量中的随机变量

下面是欧拉-拉格朗日方程的量子版本,称为Schwinger-Dyson方程:
\begin{equation}
    \expval{\left( \fdv{S}{\varphi(x)} \right) \varphi(x_1) \varphi(x_2) \cdots \varphi(x_n)} = \sum_{i=1}^n \expval{\varphi(x_1) \cdots (\ii \delta(x - x_i)) \cdots \varphi(x_n)}.
\end{equation}
下面是诺特定理的量子版本,称为Ward-Takahashi方程:
\begin{equation}
    0 = \partial_\mu \expval{j^\mu(x) \varphi(x_1) \cdots \varphi(x_n)} + \ii \sum_{i=1}^n \expval{\phi(x_1) \cdots \varphi(x) \delta(x - x_i) \cdots \phi(x_n)}.
\end{equation}

\section{有效理论}

\subsection{重整化群}

实际上,更好的做法是直接根据对称性写出拉氏量,因为如果一个拉氏量中缺少一些对称性允许的项,那么做一下重整化之后这些项都会跑出来。
很多时候我们假定我们的世界位于这个理论的红外不动点,此时只需要保留拉氏量中符合对称性、同时在重整化下是相关的的项就可以。
需要注意的是这并不总是正确的,例如,广义相对论不可重整,但是它确实非常精确,理由也很简单:我们的世界中有大量物质,实际上并非引力理论的红外不动点。

\subsubsection{积掉自由度}

\subsubsection{尺度变换}

\subsubsection{截断和发散}

\subsection{如何忽略涨落}

\subsubsection{平均场近似}

鞍点近似就是平均场,高斯展开就是RPA

平均场近似和RPA实际上就是对费曼图做部分求和。

\subsubsection{配分函数展开}

在前两节中我们模模糊糊地感受到,我们做的近似“忽略”了一些“涨落”,例如用算符的期望值代替算符本身,等等。
什么是涨落?从路径积分的角度,涨落就是偏离经典的、作用量极值的路径的那些路径。
因此,本节中我们将从路径积分配分函数的展开计算出发来审视这些近似。

\end{document}