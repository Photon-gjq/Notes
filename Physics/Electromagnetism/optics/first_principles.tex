\section{光与物质的耦合}

\subsection{粒子,以及它和光场的耦合}

\subsubsection{非相对论性粒子的哈密顿量}

考虑与电磁场发生相互作用的粒子,我们通常将这些粒子称为物质而将电磁场称为光场或是辐射,虽然严格说起来辐射也算是一种物质。
我们假定粒子做低速运动,从而不需要使用相对论性的理论描述粒子。
粒子轨道部分的哈密顿量是以下保证局部$U(1)$规范对称性的极小耦合:
\begin{equation}
    {H}_\text{orbit} = \frac{1}{2m} ({\vb*{p}} - q \vb*{A})^2 + q \phi,
    \label{eq:minimal-coupling}
\end{equation}
自旋-磁场相互作用还会引入以下哈密顿量:
\begin{equation}
    {H}_\text{spin} = - \frac{q}{m} {\vb*{S}} \cdot \vb*{B} = - \vb*{\mu} \cdot \vb*{B},
\end{equation}
而场的哈密顿量是
\begin{equation}
    {H}_\text{field} = \frac{\epsilon_0}{2} \int \dd[3]{\vb*{r}} (\vb*{E}^2 + c^2 \vb*{B}^2),
\end{equation}
则体系的总哈密顿量
\begin{equation}
    {H} = \sum_i \left( \frac{1}{2m_i} ({\vb*{p}_i} - q_i \vb*{A})^2 + q_i \varphi - \frac{q_i}{m_i} \vb*{S}_i \cdot \vb*{B} \right) + {H}_\text{field} + {H}_\text{int} + {H}_\text{ext},
\end{equation}
其中${H}_\text{int}$和${H}_\text{ext}$分别表示粒子间相互作用和外加势场。
粒子部分——包括轨道和自旋——的拉氏量也可以写成
\begin{equation}
    L = \sum_i \left( \frac{1}{2} m_i \vb*{v}_i^2 - q_i \varphi + q_i \vb*{v}_i \cdot \vb*{A} + \vb*{\mu}_i \cdot \vb*{B} \right).
\end{equation}
具体什么是粒子-粒子相互作用其实有一定人为因素,比如说凝聚态场论中默认电子之间的相互作用是库伦相互作用,但是库伦相互作用其实也是交换光子导致的,实际上是近场辐射的一个无时间延迟近似。
同样,“外加势场”也有人为因素。
不过,由于本文将要讨论光学,实际上可以以一种比较前后一致的方式确定哪些电磁场模式被粒子-粒子等效相互作用替代,哪些被纳入考虑。
我们总是可以将电磁波模式分解成无源有旋的和无旋有源的。通过简单的QED计算可以发现,全体电磁波模式造成的粒子间散射几乎压倒性地来自一个电子发射、一个电子接受的$\varphi$模式,切换到电场中基本上就是库伦场,这是有源无旋的;另一方面,介质中的电磁波宏观地看都满足横波条件$\div{\vb*{E}}=0$。%
\footnote{
    我们称它为横波条件是因为在无穷大空间中这等价于$\vb*{k} \cdot \vb*{E} = 0$,但是这\emph{并不}意味着任何能够称为波矢的$\vb*{k}$都满足$\vb*{k} \cdot \vb*{E} = 0$。
    波导就是一个典型的例子。
}%
因此我们可以只在$\varphi$和$\vb*{A}$中保留满足横波条件、看起来就像真空中电磁波的电磁波模式,这些模式本身就不易被积掉;剩下的不满足横波条件,同时的确很容易积掉的模式——如库伦场——就归入粒子-粒子等效相互作用。
至于外加势场,它或者就是库伦场,或者是外加磁场,后者同样是一个容易被积掉,并且和电磁波非常不相似的模式。

总之,在量子光学问题中,\eqref{eq:couple-ham}原则上给出了所有值得关注的信息,并且我们可以取横波条件$\div{\vb*{E}}=0$。
这可以让我们施加一个比一般的情况更加严格的规范。根据$\div{\vb*{E}}=0$我们有
\[
    \laplacian{\varphi} + \pdv{t} \div{\vb*{A}} = 0,
\]
此时我们没有加入任何限制。我们总是可以取$\varphi=0$,此时
\[
    \pdv{t} \div{\vb*{A}} = 0,
\]
即$\div{\vb*{A}}$是不随时间变化的。那么,总是可以找到一个不随着时间变化的标量场$\chi$,使得
\[
    \div{(\vb*{A} + \grad{\chi})} = 0,
\]
因为这个条件等价于调和方程
\[
    \laplacian{\chi} = - \div{\vb*{A}}.
\]
因此,我们可以做规范变换
\[
    \varphi' = \varphi - \pdv{\chi}{t} = \varphi, \quad \vb*{A}' = \vb*{A} + \grad{\chi},
\]
变换后就有$\varphi=0$和$\div{\vb*{A}}=0$同时成立。
因此,在量子光学中,我们可以同时施加以下两个规范:
\begin{equation}
    \varphi = 0, \quad \div{\vb*{A}} = 0,
\end{equation}
而不用担心产生冲突。这就是\concept{辐射规范}。辐射规范下$\div{\vb*{A}}=0$这一条件保证了$\vb*{p}$和$\vb*{A}$是可交换的。

\eqref{eq:minimal-coupling}中的$\vb*{p}$是正则动量,而不是机械动量。
然而,这反倒有好处:我们要讨论的是“向一个物理系统入射光会得到怎样的出射光”,根本不需要去测量机械动量。
这种情况下我们完全没有必要关注$\vb*{p}$是正则动量这回事:完全可以打开括号$(\vb*{p} - q \vb*{A})^2$,然后求解束缚态问题
\begin{equation}
    H = \sum_i \frac{\vb*{p}_i^2}{2m_i} + H_\text{ext} + H_\text{int},
    \label{eq:levels-ham}
\end{equation}
具体求解时可以直接援引将$\vb*{p}$当成机械动量而得到的现成的解,得到能谱之后引入电子-电磁波耦合项
\begin{equation}
    H_\text{couple} = q \varphi - \frac{q}{2m} (\vb*{p} \cdot \vb*{A} + \vb*{A} \cdot \vb*{p}) + \frac{q^2}{2m} \vb*{A}^2,
    \label{eq:couple-ham}
\end{equation}
计算物质和光场的耦合。($\vb*{A}^2$项中含有粒子的位置,因此也是耦合项)

\subsubsection{束缚态系统,微扰论和多极矩展开}

将\eqref{eq:couple-ham}当成微扰做微扰论的适用条件是$H_\text{couple}$相对于\eqref{eq:levels-ham}来说很小。
如果微扰论适用,那么显然$q \vb*{A} \ll \vb*{p}$,从而$\vb*{A}^2$项相较于$\vb*{p} \cdot \vb*{A}$项总是非常小的。%
\footnote{
    一个可以抬杠的地方是$\vb*{p}$很小时,似乎$\vb*{p} \cdot \vb*{A}$项远小于$\vb*{A}^2$项。
    然而,由能量守恒,$\vb*{A}^2$项相比于动能加上势能的\eqref{eq:levels-ham}总是很小的。
    如果我们只要求$\vb*{A}^2$级别的精度,那么在$\vb*{p}$大时显然$\vb*{p} \cdot \vb*{A}$项比$\vb*{A}^2$项重要,而$\vb*{p}$小时$\vb*{A}^2$项小于我们的精度要求。
    无论如何,$\vb*{A}^2$项都不如$\vb*{p} \cdot \vb*{A}$重要——后者重要时前者不重要,后者不重要时前者也没有重要到哪儿去。
}%
$q \vb*{A} \ll \vb*{p}$的条件实际上是不那么平凡的。
对散射态系统,机械动量估计为
\[
    m v \sim m \omega x,
\]
而
\[
    q E = m \ddot{x} \sim m \omega^2 x,
\]
最后有
\[
    E \sim - \pdv{A}{t} \sim \omega A,
\]
于是我们会发现$mv$和$eA$实际上是同个量级的。反之,对束缚态系统,$\vb*{p}$的最大值或者说振幅可以估计为
\[
    m \omega^2 x \sim q \grad{V_\text{ext}},
\]
而
\[
    mv \sim m \omega x,
\]
于是$p \gg eA$,等价于$mv \ll eA$,就等价于
\[
    mv \sim \frac{q}{\omega} \grad{V_\text{ext}} \gg q A,
\]
即等价于
\begin{equation}
    \grad{V_\text{ext}} \gg \omega A \sim E_\text{light},
\end{equation}
即束缚电场远大于光场。这应该是能够保证的,否则就不是束缚态了,此时介质就被打穿为等离子体了,并且,这种情况下,将光场撤去,介质也未必会恢复为原状,即出现了光学损伤。

在知道了能将\eqref{eq:couple-ham}当成微扰的系统中的带电粒子高度定域之后,我们立刻想到,由于这些带电粒子的位置高度有界,可以做多极矩展开。
实际上我们看到,多极矩展开合法、带电粒子位置高度定域(这意味着带电粒子)、$e \vb*{A} \ll \vb*{p}$这几个条件是等价的。
应该说\eqref{eq:couple-ham}是很不直观的,因为它是关于$\vb*{A}$的而不是$\vb*{E}$和$\vb*{B}$的,做完多极矩展开之后我们就可以讨论“某个过程在电偶极矩跃迁下可以发生,另一个过程需要电四极矩跃迁,从而很弱”,等等。
以下我们用$0$作为带点粒子位置的“原点”,$\vb*{r}$不会偏离$0$太远。

下面我们尝试使用多种近似手段。我们很快会发现,这些方法都指向同一个事实:对介质中的量子光学,基本上只有电偶极子相互作用是重要的。

直接丢弃$\vb*{A}^2$项。此时如果采取辐射规范,根据$\vb*{p}$和$\vb*{A}$的可交换性,我们就有
\begin{equation}
    H_\text{couple} = - \frac{q}{m} \vb*{A} \cdot \vb*{p}.
    \label{eq:velocity-gauge}
\end{equation}
这称为\concept{速度规范}下的哈密顿量。如果我们进一步,假定$\vb*{A}$在电子运动的区域内没有明显的空间变化,则在一个规范变换之下我们可以得到
\begin{equation}
    H_\text{couple} = - q \vb*{r} \cdot \vb*{E} = - \vb*{d} \cdot \vb*{E}.
    \label{eq:electric-dipole}
\end{equation}
或者,由于$\vb*{A}$在电子运动的区域内没有明显的空间变化,我们根据\eqref{eq:velocity-gauge}可以写出(这里我们假装$\vb*{p}$就是机械动量,但是因为$\vb*{p}$的实际物理意义在做了近似\eqref{eq:velocity-gauge}不再影响哈密顿量的形式,这是可以的)
\[
    \begin{aligned}
        S &= \int \dd{t} \left( \frac{1}{2} m \vb*{v}^2 + \frac{q}{m} \vb*{A} \cdot (m \vb*{v}) \right) \\
        &= \int \dd{t} \left( \frac{1}{2} m \vb*{v}^2 - q \dv{\vb*{A}}{t} \cdot \vb*{r} \right) \\
        &= \int \dd{t} \left( \frac{1}{2} m \vb*{v}^2 + q \vb*{r} \cdot \vb*{E} \right),
    \end{aligned} 
\]
第二个等号用到了分部积分法。再做勒让德变换,就得到\eqref{eq:electric-dipole}。
实际上,我们会注意到以上构造拉氏量以后用分布积分法的方法只用到了一个条件,就是$\vb*{A}$的空间变化不大(从而它对时间的全导数就是它对时间的偏导数,就是电场的相反数),因此只需要“$\vb*{A}$的空间变化不大”就足够推导出\eqref{eq:electric-dipole}。
我们称\eqref{eq:electric-dipole}为\concept{长度规范}下的哈密顿量。

我们现在考虑$\vb*{A}^2$项能够丢弃,但是$\vb*{A}$尚有比较大的空间变化的情况;当然,这是为了将磁场和轨道自由度做耦合。
乍一看,我们可以使用磁标势方法来得到磁场,但是这是行不通的:我们在处理的并非静磁学问题,位移电流项是到处都在的,从而如果要用磁标势方法,磁壳必须取在我们讨论的电子的周围,从而让磁标势毫无用处。
我们会发现取
\begin{equation}
    \vb*{A} = \frac{1}{2} \vb*{B} \times \vb*{r} - \int_0^t \dd{t'} \grad(\vb*{r} \cdot \vb*{E}(\vb*{r}, t'))
    \label{eq:a-containing-e-and-b}
\end{equation}
能够提供足够好的近似。直接计算就会发现
\[
    \curl{\vb*{A}} = \vb*{B},
\]
而
\[
    \begin{aligned}
        \pdv{\vb*{A}}{t} &= \frac{1}{2} \pdv{\vb*{B}}{t} \times \vb*{r} - \grad{(\vb*{r} \cdot \vb*{E})} = \frac{1}{2} \vb*{r} \times (\curl{\vb*{E}}) - \grad{(\vb*{r} \cdot \vb*{E})} \\
        &= \frac{1}{2} \left( \grad{(\vb*{r} \cdot \vb*{E})} - (\vb*{r} \cdot \grad) \vb*{E} - (\vb*{E} \cdot \grad) \vb*{r} - \vb*{E} \times (\curl{\vb*{r}}) \right) - \grad{(\vb*{r} \cdot \vb*{E})} \\
        &= \frac{1}{2} \left( \grad{(\vb*{r} \cdot \vb*{E})} - (\vb*{r} \cdot \grad) \vb*{E} - \vb*{E} \right)- \grad{(\vb*{r} \cdot \vb*{E})} .
    \end{aligned}
\]
如果假定$\vb*{E}$和$\vb*{B}$在空间上没有什么变化,那么就有
\[
    \pdv{\vb*{A}}{t} = \frac{1}{2} (\vb*{E} - \vb*{E}) - \vb*{E} = - \vb*{E}.
\]
因此,在电场和磁场在我们关心的区域基本均匀的情况下,\eqref{eq:a-containing-e-and-b}近似是辐射规范下的矢势。
现在我们再做一个规范变换:
\[
    \vb*{A} \longrightarrow \vb*{A} + \grad{\chi}, \quad \varphi \longrightarrow \varphi - \pdv{\chi}{t}, \quad \chi = \int_0^t \dd{t'} \vb*{r} \cdot \vb*{E}(\vb*{r}, t'),
\]
就有
\[
    \begin{aligned}
        H_\text{couple} &= q \varphi - \frac{q}{m} \vb*{A} \cdot \vb*{p} \\
        &= - q \vb*{r} \cdot \vb*{E} - \frac{q}{m} \frac{1}{2} (\vb*{B} \times \vb*{r}) \cdot \vb*{p} \\
        &= - \vb*{d} \cdot \vb*{E} - \frac{q}{2m} \vb*{B} \cdot (\vb*{r} \times \vb*{p}),
    \end{aligned}
\]
从而
\begin{equation}
    H_\text{couple} = - \vb*{d} \cdot \vb*{E} - \frac{q}{2m} \vb*{L} \cdot \vb*{B}.
\end{equation}
这里多出来了一项,即磁场和轨道角动量的耦合。
% TODO:但是此时电四极矩也开始变得重要了

现在我们有了三种相互作用通道,有电偶极跃迁
\begin{equation}
    H_1 = - \vb*{d} \cdot \vb*{E},
\end{equation}
有自旋取向作用
\begin{equation}
    H_2 = - \frac{q}{m} \vb*{S} \cdot \vb*{B},
\end{equation}
还有轨道角动量取向作用
\begin{equation}
    H_3 = - \frac{q}{2m} \vb*{L} \cdot \vb*{B}.
\end{equation}
实际上,磁场对自旋的取向作用${H}_2$是很弱的。设电磁波波长的尺度为$\lambda$,则
\[
    \vb*{B} = \curl{\vb*{A}} \sim \frac{A}{\lambda},
\]
电子的活动范围的尺度和原子半径$a_0$同阶,由不确定性关系,
\[
    p a_0 \sim \hbar.
\]
于是
\[
    \frac{H_2}{H_1} \sim \frac{\hbar \frac{A}{\lambda}}{\frac{\hbar}{a_0} A} = \frac{a_0}{\lambda}.
\]
波长通常在几百纳米级别,而原子半径在纳米级别以下,从而${H}_1$远大于${H}_2$。

\subsubsection{散射态系统和等离子体}

\subsection{光场}

电磁场足够强以至于难以看到单光子效应,而又足够弱以至于能量不至于强到需要考虑量子电动力学的圈图修正,这样就可以使用经典电动力学描述整个系统。
为了讨论电磁波的量子涨落(在分析诸如腔内辐射场,或是非线性光学中的DFG过程时非常重要),即使没有圈图效应,我们也要做光场的量子化。

\subsubsection{真空}

我们首先考虑真空中的光场的量子化,此时我们无非是在重复QED中的运算。
QED中矢量场展开为
\begin{equation}
    A_\mu(\vb*{x}, t) = (\frac{\varphi}{c}, - \vb*{A}) = \int \frac{\dd[3]{\vb*{k}}}{(2\pi)^3} \sqrt{\frac{\hbar}{2\omega_{\vb*{k}} \epsilon_0}} \sum_{\sigma=1}^2 \left( a_{\vb*{k} \sigma} \epsilon_\mu^\sigma(\vb*{k}) \ee^{\ii \vb*{k} \cdot \vb*{x} - \ii \omega_{\vb*{k}} t} + a_{\vb*{k} \sigma}^\dagger \epsilon_\mu^\sigma(\vb*{k})^* \ee^{- \ii \vb*{k} \cdot \vb*{x} + \ii \omega_{\vb*{k}} t} \right),
\end{equation}
其中电磁场模式为平面波。
取费曼规范,做一些分部积分并去掉表面项,得到
\begin{equation}
    \mathcal{L} = - \frac{1}{2 \mu_0} \partial_\mu A_\nu \partial^\mu A^\nu,
\end{equation}
从而正则动量为
\begin{equation}
    \pi^\mu = \pdv{\mathcal{L}}{\partial_0 A_\mu} = - \partial^0 A^\mu,
\end{equation}
可以据此写出正则量子化条件,即时间相同时,$A^\mu$同$A^\nu$对易,而
\begin{equation}
    [A^\mu(\vb*{x}, t), \pi^\mu(\vb*{y}, t)] = \ii \eta^{\mu \nu} \delta^{(3)}(\vb*{x} - \vb*{y}).
\end{equation}
哈密顿量为
\[
    \begin{aligned}
        H &= \int \dd[3]{\vb*{r}} (\pi^\mu \partial_0 A_\mu - \mathcal{L}) \\
        &= \int \dd[3]{\vb*{r}} \left( - \frac{1}{c^2} (\partial_t A^\mu)^2 + \frac{1}{2} \partial_\mu A_\nu \partial^\mu A^\nu \right),
    \end{aligned}
\]
这里要注意$x^0 = c t$。代入$A_\mu$的展开式计算得到
\begin{equation}
    H = \sum_{\sigma=1}^2 \int \frac{\dd[2]{\vb*{k}}}{(2\pi)^3} \hbar \omega_{\vb*{k}} \left( a^\dagger_{\vb*{k} \sigma} a_{\vb*{k} \sigma} + \frac{1}{2} \right), \quad \omega_{\vb*{k}} = c \abs*{\vb*{k}}.
\end{equation}
这就得到了量子化的能量。
在量子化过程中我们已经通过限制$\sigma$而施加了规范,不过这个规范并不是辐射规范,而是洛伦兹规范。横波条件通过$\epsilon$矢量和波矢垂直而保证。
不过,既然我们只关心电偶极辐射而有关的相互作用哈密顿量可以完全写成$\vb*{E}$,这也不重要。

从四维矢量计算电场,得到
\begin{equation}
    \vb*{E}(\vb*{r}, t) = \int \frac{\dd[3]{\vb*{k}}}{(2\pi)^3} \sqrt{\frac{\hbar}{2\omega_{\vb*{k}} \epsilon_0}} \sum_{\sigma=1}^2 \left( (- \ii \vb*{k} \epsilon_0^\sigma(\vb*{k}) + \ii \omega_{\vb*{k}} \vb*{\epsilon}^\sigma(\vb*{k})) a_{\vb*{k} \sigma} \ee^{\ii \vb*{k} \cdot \vb*{r} - \ii \omega_{\vb*{k}} t} + \text{h.c.} \right),
\end{equation}
以及
\begin{equation}
    \vb*{B}(\vb*{r}, t) = \int \frac{\dd[3]{\vb*{k}}}{(2\pi)^3} \sqrt{\frac{\hbar}{2\omega_{\vb*{k}} \epsilon_0}} \sum_{\sigma=1}^2 \left( \ii \vb*{k} \times \vb*{\epsilon}_\sigma a_{\vb*{k} \sigma} \ee^{\ii \vb*{k} \cdot \vb*{r} - \ii \omega_{\vb*{k}} t } + \text{h.c.} \right).
\end{equation}
其中
\begin{equation}
    \epsilon^\mu_\sigma = (\epsilon_\sigma^0, \vb*{\epsilon}_\sigma), \quad \frac{\omega_{\vb*{k}}}{c} \epsilon_0^\sigma - \vb*{k} \cdot \vb*{\epsilon}_\sigma = 0, \quad \abs*{\epsilon_\sigma^0}^2 - \abs*{\vb*{\epsilon}_\sigma}^2 = 1.
\end{equation}
通过以上公式,能够验证以下哈密顿量形式:
\begin{equation}
    H = \int \dd[3]{\vb*{r}} \left( \frac{\epsilon_0}{2} \vb*{E}^2 + \frac{1}{2\mu_0} \vb*{B}^2 \right).
    \label{eq:e-and-b-hamiltonian}
\end{equation}

在本文讨论的光学问题中,我们可以使用一种对具体计算更加友好的形式,即采用辐射规范。
在辐射规范之下,我们有
\begin{equation}
    \begin{aligned}
        \mathcal{L} &= - \frac{1}{4 \mu_0} (\partial_\mu A_\nu - \partial_\nu A_\mu) (\partial^\mu A^\nu - \partial^\nu A^\mu) \\
        &= \frac{1}{2 \mu_0} \frac{1}{c^2} (\partial_t \vb*{A})^2 - \frac{1}{4\mu_0} (\partial_i A_j - \partial_j A_i) (\partial^i A^j - \partial^j A^i) \\
        &= \frac{\epsilon_0}{2} ((\dot{\vb*{A}})^2 - c^2 (\curl{\vb*{A}})^2).
    \end{aligned}
\end{equation}
以这个拉氏量为出发点做正则量子化。做展开
\begin{equation}
    \vb*{A}(\vb*{r}, t) = \int \frac{\dd[3]{\vb*{k}}}{(2\pi)^3} \sqrt{\frac{\hbar}{2 \omega_{\vb*{k}} \epsilon_0}} \sum_{\sigma=1}^2 (a_{\vb*{k} \sigma} \vu*{e}^\sigma \ee^{\ii \vb*{k} \cdot \vb*{r} - \ii \omega_{\vb*{k}} t} + a^\dagger_{\vb*{k} \sigma} (\vu*{e}^\sigma)^* \ee^{- \ii \vb*{k} \cdot \vb*{r} + \ii \omega_{\vb*{k}} t}),
\end{equation}
从而电场和磁场分别为
\begin{equation}
    \vb*{E}(\vb*{r}, t) = \ii \int \frac{\dd[3]{\vb*{k}}}{(2\pi)^3} \sqrt{\frac{\hbar \omega_{\vb*{k}}}{2 \epsilon_0}} \sum_{\sigma=1}^2 (a_{\vb*{k} \sigma} \vu*{e}^\sigma \ee^{\ii \vb*{k} \cdot \vb*{r} - \ii \omega_{\vb*{k}} t} - a^\dagger_{\vb*{k} \sigma} (\vu*{e}^\sigma)^* \ee^{- \ii \vb*{k} \cdot \vb*{r} + \ii \omega_{\vb*{k}} t})
\end{equation}
和
\begin{equation}
    \vb*{B}(\vb*{r}, t) = \ii \int \frac{\dd[3]{\vb*{k}}}{(2\pi)^3} \sqrt{\frac{\hbar}{2 \omega_{\vb*{k}} \epsilon_0}} \sum_{\sigma=1}^2 (a_{\vb*{k} \sigma} \vb*{k} \times \vu*{e}_\sigma \ee^{\ii \vb*{k} \cdot \vb*{r} - \ii \omega_{\vb*{k}} t} - a^\dagger_{\vb*{k} \sigma} \vb*{k} \times \vu*{e}_\sigma^* \ee^{- \ii \vb*{k} \cdot \vb*{r} + \ii \omega_{\vb*{k}} t}).
\end{equation}
正则动量为
\begin{equation}
    \vb*{\pi} = \epsilon_0 \dot{\vb*{A}},
\end{equation}
施加正则对易关系,会得到正确的
\begin{equation}
    \comm*{a_{\vb*{k} \sigma}}{a_{\vb*{k}' \sigma'}} = (2\pi)^3 \delta(\vb*{k} - \vb*{k}') \delta_{\sigma \sigma'},
\end{equation}
而哈密顿量为
\begin{equation}
    \begin{aligned}
        H &= \int \dd[3]{\vb*{r}} \left(\vb*{\pi} \cdot \pdv{\vb*{A}}{t} - \mathcal{L} \right) = \int \dd[3]{\vb*{r}} \left( \frac{\epsilon_0}{2} \left(\pdv{\vb*{A}}{t}\right)^2 + \frac{\epsilon_0}{2} c^2 (\curl{\vb*{A}})^2 \right) \\
        &= \int \dd[3]{\vb*{r}} \left( \frac{\epsilon_0}{2} \vb*{E}^2 + \frac{1}{2\mu_0} \vb*{B}^2 \right) \\
        &= \sum_{\sigma=1}^2 \int \frac{\dd[3]{\vb*{k}}}{(2\pi)^3} \hbar \omega_{\vb*{k}} \left(a^\dagger_{\vb*{k} \sigma} a_{\vb*{k} \sigma} + \frac{1}{2} \right).
    \end{aligned}
\end{equation}
因此,辐射规范给出的结果和完整的QED计算是完全一致的。
在辐射规范中我们还可以证明一个在横场条件成立时也成立,并且在一般的QED中很难计算的公式:
\begin{equation}
    \comm*{E^i(\vb*{r}, t)}{B^j(\vb*{r}', t)} = - \frac{\ii \hbar}{\epsilon_0} \pdv{x^k} \delta(\vb*{r} - \vb*{r}')
\end{equation}
其中$i, j, k$是$x, y, z$的轮换排列;其它情况下对易子为零。
还能够发现电场和自己的对易子始终为零,磁场亦然。因此电场的三个分量可以同时确定地被测量,磁场亦然。
但是不能同时准确测出$\vb*{E}$和$\vb*{B}$。
由于$(\vb*{E}, \vb*{B})$,$\vb*{A}$和$a_{\vb*{k} \sigma}$直接的关系是线性的,$a$的产生湮灭算符对易关系、$\vb*{A}$和$\vb*{\pi}$的正则对易关系以及$\vb*{E}$和$\vb*{B}$的对易关系是彼此等价的。

\subsubsection{长波光子和介质中的麦克斯韦方程}

现在我们考虑介质作用。直接将QED和介质耦合起来,虽然的确是正确的,在实际计算时却会产生一些理论上的问题。
例如,我们知道,介质通常出于热态,因此,一个光子和介质发生相互作用之后就处于混合态了,似乎不能写出一个场论来描述介质中光子;从介质吸收光子到发射光子会有时间延迟;介质微观上是非常不均匀的,从而平面波进入介质后波阵面将面目全非。
第一个问题的严格处理显然需要用到非平衡态场论,但我们可以认为我们将介质积掉了,从而用介质中的电磁场-电磁场关联函数代替了介质作用。
这样,介质的线性效应体现为电磁场的作用量的二次型部分出现一个修正,非线性效应体现为电磁场的自相互作用,非幺正的部分体现为以上修正中的虚部。
后两个问题可以采用和经典电动力学类似的方法解决,即我们只处理“经过空间平均”的电磁场,这相当于做了一个动量截断,只讨论波长足够长的那部分电磁波模式,则介质中发生的过程相比于我们讨论的过程来说是非常快、且空间细节不甚清楚的,从而,介质导致的电磁场关联函数的修正可以认为没有时间上的延迟效应或是空间上的非局域效应。
在电磁场的波长和晶格常数接近时,这么做就失效了,此时必须使用完整的第一性原理做计算。

在以上条件——只需要考虑波长远大于介质的微观不均匀性的空间尺度的光子——成立时,形式上,我们可以直接将介质中的麦克斯韦方程做正则量子化。要看出这是为什么,首先考虑线性部分,描述光场的宏观的线性介质中的麦克斯韦方程是
\[
    \begin{aligned}
        &\div{\vb*{D}} = 0, \quad \curl{\vb*{E}} = - \pdv{\vb*{B}}{t}, \\
        &\div{\vb*{B}} = 0, \quad \curl{\vb*{H}} = \pdv{\vb*{D}}{t} + \vb*{j},
    \end{aligned}
\]
这里我们保留了传导电流,这是为了提示系统哈密顿量中外加激励项$\vb*{j} \cdot \vb*{A}$的存在。取规范$\varphi=0$,并切换到频域,我们会发现以上方程等价于辐射规范加上
\begin{equation}
    \curl{(\mu^{-1} \cdot \curl{\vb*{A}})} - \omega^2 \epsilon \cdot \vb*{A} = \vb*{j}.
    \label{eq:photon-in-material}
\end{equation}
如果介质修正后的电磁场关联函数实际上就是上式的格林函数,我们就可以直接将线性介质中的麦克斯韦方程中的电场和磁场提升为算符,完成正则量子化。

对称性告诉我们,在长波光子条件成立时,破缺空间平移对称性和空间各向同性,但保留局域性,则\eqref{eq:photon-in-material}是最一般的方程。
可以在整个方程左边再乘上一个张量,但是我们随即可以将这个张量吸收到$\vb*{j}$的定义中;$\curl{\vb*{A}}$的形式不能改变,因为无论如何,从$\vb*{A}$出发能够得到的局域的规范不变矢量除了$\pdv*{\vb*{A}}{t}$——在频域下就正比于$\vb*{A}$——以外就只有它了。
因此,的确,对波长远大于介质微观不均匀性(晶格常数等)的光子(大部分能够称为“光学”的问题都是这样的,因为晶格常数差不多几百皮米,已经对应X射线的波长了),至少线性介质中的麦克斯韦方程可以被理解为海森堡绘景下的方程。
非线性项可以如法炮制。因此,在这里,我们的思路和高能物理类似,先获得一个自由理论,再加入相互作用;另一种思路是使用量子版本的极化矢量。

从哈密顿量的角度出发可以更加容易地看出为什么线性麦克斯韦方程\eqref{eq:photon-in-material}可以直接量子化。
破缺空间平移对称性和空间各向同性之后,\eqref{eq:e-and-b-hamiltonian}能够有的修正方式是非常有限的:如果保持哈密顿量为二次型,我们只能够让$\vb*{E}^2$项和$\vb*{B}^2$项变得各向异性,即让它变成
\begin{equation}
    H = \int \dd[3]{\vb*{r}} \left( \frac{1}{2} \vb*{E} \cdot \vb*{\epsilon} \cdot \vb*{E} + \frac{1}{2} \vb*{B} \cdot (\vb*{\mu}^{-1}) \cdot \vb*{B} \right) = \frac{1}{2} \int \dd[3]{\vb*{r}} (\vb*{D} \cdot \vb*{E} + \vb*{B} \cdot \vb*{H}).
    \label{eq:material-hamiltonian}
\end{equation}
它和\eqref{eq:photon-in-material}是等价的。哈密顿量被修正在物理上对应着积掉介质,如果只考虑长波光子,那么这个过程应该给出在时间上和空间上都是局域的等效光子相互作用。
原则上可以产生$\vb*{E}$和$\vb*{B}$的任意次方项,只保留两项就得到\eqref{eq:material-hamiltonian},保留更多项就得到非线性光学效应。

下面我们讨论和\eqref{eq:material-hamiltonian}匹配的对易关系,以及它对角化之后将给出什么样的能谱。
应当指出,此时真空中的那些对易关系——$\vb*{A}$和$\epsilon_0 \dot{\vb*{A}}$之间的对易关系,$\vb*{E}$和$\vb*{B}$之间的对易关系——可能不能够直接适用。
这是因为正则量子化中,积掉自由度会导致哈密顿量的本征态的意义发生变化,从而算符的意义会发生变化。在高能物理中这导致场强重整化,在本文讨论的量子光学中则还会让对易关系发生变化。
同样这也会让横波条件的形式发生变化——介质中横波条件是$\div{\vb*{\epsilon} \cdot \vb*{E}} = 0$。

我们需要直接从\eqref{eq:material-hamiltonian}计算正则动量。同样取辐射规范,以$\vb*{A}$为基本自由度,则\eqref{eq:material-hamiltonian}就是
\begin{equation}
    H = \int \dd[3]{\vb*{r}} \left( \frac{1}{2} \dot{\vb*{A}} \cdot \vb*{\epsilon} \cdot \dot{\vb*{A}} + \frac{1}{2} (\curl{\vb*{A}}) \cdot (\vb*{\mu}^{-1}) \cdot (\curl{\vb*{A}}) \right),
\end{equation}
于是正则动量为
\begin{equation}
    \vb*{\pi} = \vb*{\epsilon} \cdot \dot{\vb*{\vb*{A}}}.
\end{equation}
我们现在需要展开$\vb*{A}$。此时空间平移不变性不能保持,我们不能使用动量来标记电场的振动模式,
我们将\eqref{eq:photon-in-material}右边的$\vb*{j}$取为零——我们此处在对线性介质做正则量子化,暂时不考虑电流——那就得到了一个广义本征值问题。
这就意味着,我们可以求解出一整套本征函数,它们由下式
\begin{equation}
    \curl{(\mu^{-1} \cdot \curl{\vb*{u}_n})} - \omega_n^2 \epsilon \cdot \vb*{u}_n = 0
\end{equation}
确定,其中$\omega_n$对应着能够在系统中稳定传播的电磁波模式的频率,且有正交归一关系
\begin{equation}
    \int_V \dd[3]{\vb*{r}} \vb*{u}^*_m \cdot \vb*{\epsilon} \cdot \vb*{u}_n = \delta_{mn}, 
\end{equation}
请注意由于$\vb*{A}$的厄米性,$\vb*{u}_m^*$一般是另一个$\vb*{u}_n$。
正交归一关系又意味着
\begin{equation}
    \omega_n^2 \delta_{mn} = \int \dd[3]{\vb*{r}} (\curl{\vb*{u}_m^*}) \cdot (\vb*{\mu}^{-1}) \cdot (\curl{\vb*{u}_n}) + \int \dd{\vb*{S}} \cdot (\vb*{u}_m^* \times ((\vb*{\mu}^{-1}) \cdot (\curl{\vb*{u}_n}))),
\end{equation}
在自由空间中等式右边第二项可以略去,在一个反射性能尚可的反射腔体(如果我们只讨论有限空间中的问题,那么基本上这个问题需要放在一个腔体中,否则无法忽视外界影响)中可以把第二项当成微扰。
本节仅仅给出最为简单的理论,暂时不考虑第二项。
用这组基底$\{\vb*{u}_n\}$做展开
\begin{equation}
    \vb*{A}(\vb*{r}, t) = \sum_n \ii \sqrt{\frac{\hbar}{2\omega_{\vb*{k}}}} \vb*{u}_n(\vb*{r}) a_n \ee^{- \ii \omega_n t} + \text{h.c.},
\end{equation}
得到
\begin{equation}
    - \vb*{E} = \dot{\vb*{A}} = \sum_n \sqrt{\frac{\hbar \omega_n}{2}} \vb*{u}_n(\vb*{r}) a_n \ee^{-\ii \omega_n t} + \text{h.c.},
\end{equation}
以及
\begin{equation}
    \vb*{B} = \curl{\vb*{A}} = \sum_n \ii \sqrt{\frac{\hbar}{2\omega_n}} \curl{\vb*{u}_n(\vb*{r})} a_n \ee^{-\ii \omega_n t} + \text{h.c.}.
\end{equation}
施加正则对易关系
\begin{equation}
    \comm*{A^i(\vb*{r}, t)}{\pi^j(\vb*{r}', t)} = \ii \hbar \delta(\vb*{r} - \vb*{r}') \delta^{ij},
\end{equation}
我们发现我们能够得到我们想要的产生湮灭算符对易关系
\begin{equation}
    \comm*{a_n}{a_m^\dagger} = \delta_{mn}.
\end{equation}
然后,可以计算出哈密顿量为
\begin{equation}
    H = \sum_n \hbar \omega_n \left( a^\dagger_n a_n + \frac{1}{2} \right).
\end{equation}
这个哈密顿量的形式和真空中完全一样,不同的地方在于$\omega_{\vb*{k}}$被$\omega_n$取代,色散关系可能变得非常不一样。

既然$\epsilon$和$\mu$的概念对长波光子在量子情况下仍然适用,反射、折射等概念对长波光子仍然有意义,且和经典情况非常类似。
特别的,光场可能被约束在一个四面都是反射镜的腔体中,此时的光场被所谓的\concept{cavity QED}或者简写为\concept{cQED}描述。

总之,有两种方法处理介质:一种是显式地根据光场和介质的电偶极子耦合做微扰计算;但是,也可以首先计算介质中的电磁场关联函数,然后根据\eqref{eq:photon-in-material}得到$\epsilon$和$\mu$,代入算符版本的介质中麦克斯韦方程。
算符版本的介质中麦克斯韦方程适用于波长长于介质微观结构尺度(如晶格常数)的光子,因此适用范围是很大的。
一个介质系统中的量子化光场的自由哈密顿量就是普通的谐振子哈密顿量。
使用本质上是经典的方程\eqref{eq:photon-in-material},得到一系列振动模式,其频率即为这个介质系统中的量子化光场中的模式的频率,振动模式的场强分布就是\eqref{eq:photon-in-material}给出的本征模式。

实际上从这里我们可以看出,经典的麦克斯韦方程本身已经是一个具有一定量子特性的理论了——“单光子波函数”(虽然没有良定义的单光子量子力学,但是我们不妨这么指代$\mel*{0}{A^i(\vb*{r})}{\psi}$)服从的方程就是麦克斯韦方程。
也可以从另一个角度看这件事:在麦克斯韦方程两边乘上$\hbar$,由于$E \sim \hbar \omega$,得到的理论看上去就是一个量子理论。
将光场量子化引入的新物理只有两件事:光束由分立的光子构成,以及存在光子数的量子涨落,但是,在没有非线性光学效应的情况下,光子数目守恒,第一件事完全可以通过手动引入“光子”的概念并指派其波函数为经典电磁场来做到。
光的量子性只有在下面的地方才会变得重要:
\begin{itemize}
    \item 光子生灭明显时,即处理非线性光学时,因为此时会有一些原本完全没有光子分布的模式上出现了光子。经典处理只能在有种子光的时候处理光子的产生——并不奇怪,因为光子从零到一产生的过程涉及一个极为弱的场强,弱到经典场论不再使用。
    \item 纠缠重要时。经典电动力学面对“光子增多了”的描述方法是更大的场强,而没有直积的希尔伯特空间这样的概念,从而无法捕捉到纠缠。
\end{itemize}
可以看到这些光的量子性变得明显的情况都涉及多光子态。我们其实可以在这里看到一个相当有趣的情况:当光场非常弱时,经典电动力学就失效了,然而经典电动力学的形式却又很像是在处理“单光子波函数”。
当然,这两者并没有矛盾,本质上是因为经典电动力学无法正确处理“多光子形成的直积态”:“单光子波函数”不涉及直积态,它给出的所有物理就是一个麦克斯韦方程,正好和经典电动力学一致;光子数足够多时态的形式变得不重要,同样可以被经典电动力学处理。
经典电动力学无法正确处理“多光子形成的直积态”,因为光场没有量子化,没有Fock空间;但是这并不是说经典电动力学就缺乏(相比于经典质点动力学的)一切量子性,如坐标和动量的不确定性等。

在本文中“电磁场”可能代表量子化的场算符,也可能代表经典电磁场,也可能代表“单光子波函数”。在不考虑非线性效应时这三者的时间演化是相同的。

\section{线性介质与经典麦克斯韦方程}

\subsection{有介质情况下的麦克斯韦方程}

真空中的麦克斯韦方程组为我们熟知的形式:
\begin{equation}
    \begin{bigcase}
        \div{\vb*{E}} &= \frac{\rho}{\epsilon_0} \\
        \curl{\vb*{E}} &= - \pdv{\vb*{B}}{t} \\
        \div{\vb*{B}} &= 0 \\
        \curl{\vb*{B}} &= \mu_0 \vb*{j} + \mu_0 \epsilon_0 \pdv{\vb*{E}}{t}
    \end{bigcase}
    \label{eq:original-maxwell}
\end{equation}
介质的存在事实上在微观层面不会改变\eqref{eq:original-maxwell}的形式。
介质起作用的方式是,其内部已经有一个电荷分布,当外加电场的时候电荷重新排列、发生运动,在此过程中产生额外的电流、电场、磁场。
于是假定电荷和电流可以做以下分解:
\[
    \begin{bigcase}
        &\vb*{j} = \vb*{j}_\text{f} + \vb*{j}_\text{r}, \quad \rho = \rho_\text{f} + \rho_\text{r}, \\
        &\pdv{\rho_\text{f}}{t} + \div{\vb*{j}_\text{f}} = 0, \\
        &\pdv{\rho_\text{r}}{t} + \div{\vb*{j}_\text{r}} = 0
    \end{bigcase}
\]
其中$\vb*{j}_\text{f}$是所谓的自由电流,而$\vb*{j}_\text{r}$是介质的响应。但是这种二分法实际上很大程度上是任意的。
例如,金属能导电,因为其内部含有大量几乎是自由的电子——那么,外加电场产生的金属中的电流就应该是自由电流了;
但是分析金属的光学属性的时候,这些由于外加电场产生的电流又无疑是介质的响应。
因此$\vb*{j}_\text{f}$和$\vb*{j}_\text{r}$只是辅助量,没有特殊的物理含义。

为了能够将$\vb*{j}_\text{f}$和$\vb*{j}_\text{r}$整合进两个形式上和电场和磁感应强度很像的辅助量,
从而在形式上让\eqref{eq:original-maxwell}变成一个只和自由电荷和自由电流有关的方程组,我们进一步做下面的分解:
\[
    \vb*{j}_\text{r} = \vb*{j}_\text{s} + \vb*{j}_\text{c}
\]
且$\vb*{j}_\text{c}$是一个有旋无源场。光有这个条件不足以在给定$\vb*{j}_\text{r}$时唯一地确定下$\vb*{j}_\text{s}$和$\vb*{j}_\text{c}$,
因此还可以引入一个假设而不至于让$\vb*{j}_\text{s}$和$\vb*{j}_\text{c}$无解。
为了让\eqref{eq:original-maxwell}中第一式的右边只剩下自由电荷,假定
\[
    \rho_\text{r} = - \div{\vb*{P}}
\]
这个假设\concept{没有}缩小$\vb*{j}_\text{s}$和$\vb*{j}_\text{c}$的选择范围,因为任意给定性质足够良好的$\rho_\text{r}$,相对应的$\vb*{P}$总是存在的(而且显然不唯一)。
同时由于$\vb*{j}_\text{c}$是一个有旋无源场,可以再引进一个辅助量$\vb*{M}$使
\[
    \vb*{j}_\text{c} = \curl{\vb*{M}}
\]
此时$\rho_\text{r}$的输运方程成为
\[
    \pdv{\rho_\text{r}}{t} + \div{\vb*{j}_\text{s}} = 0
\]
因为$\curl{\vb*{j}_\text{c}}$的散度为零。这个式子又可以写成
\[
    \div{\left(\vb*{j}_\text{s}-\pdv{\vb*{P}}{t}\right)} = 0
\]
受到这个式子的启发,我们\concept{假设}(不是推出,因为光有上式不能定解,而先前我们只对$\vb*{j}_\text{c}$做过假设而没有对$\vb*{j}_\text{s}$做过假设,因此后者的取值仍然是任意的)有
\[
    \vb*{j}_\text{s} = \pdv{\vb*{P}}{t}
\]
这个假设不会让$\vb*{j}_\text{s}$和$\vb*{j}_\text{c}$无解。

将以上引入的所有物理量代入\eqref{eq:original-maxwell},得到
\[
    \begin{bigcase}
        \epsilon_0 \div{\vb*{E}} &= \rho_\text{f} - \div{\vb*{P}}, \\
        \curl{\vb*{E}} &= - \pdv{\vb*{B}}{t}, \\
        \div{\vb*{B}} &= 0, \\
        \curl{\frac{\vb*{B}}{\mu_0}} &= \vb*{j}_\text{f} + \curl{\vb*{M}} + \pdv{\vb*{P}}{t} + \epsilon_0 \pdv{\vb*{E}}{t}
    \end{bigcase}
\]
引入辅助量
\[
    \vb*{D} = \epsilon_0 \vb*{E} + \vb*{P}, \quad \vb*{H} = \frac{\vb*{B}}{\mu_0} - \vb*{M}
\]
就得到了
\begin{equation}
    \begin{bigcase}
        \div{\vb*{D}} &= \rho_\text{f}, \\
        \curl{\vb*{E}} &= - \pdv{\vb*{B}}{t}, \\
        \div{\vb*{B}} &= 0, \\
        \curl{\vb*{H}} &= \vb*{j}_\text{f} + \pdv{\vb*{D}}{t}
    \end{bigcase}
    \label{eq:maxwell-material}
\end{equation}

方程组\eqref{eq:maxwell-material}除去了\eqref{eq:original-maxwell}中由于介质产生的电荷密度和电流密度,形式上更加简洁,
但是即使在自由电荷密度和电流密度已经给定的情况下,只靠\eqref{eq:maxwell-material}本身也没有办法定解,因为未知数太多了。
考虑到从$\vb*{E}, \vb*{B}$到$\vb*{D}, \vb*{H}$的变换是线性的,
这就意味着\eqref{eq:original-maxwell}在自由电荷密度和电流密度已经给定的情况下其实也不能定解。
这是理所当然的。

下面的问题是,在自由电荷密度和电流密度已经给定的情况下,增加什么方程能够让\eqref{eq:maxwell-material}定解?
当然,只要知道了从$\vb*{E}, \vb*{B}$到$\vb*{D}, \vb*{H}$的变换的具体计算式(而不是显含$\vb*{j}_\text{r}$的定义式)
就能够定解。
更进一步,在什么都不知道,只有初始条件和边界条件的情况下,怎样能够让\eqref{eq:maxwell-material}定解?
只需要增补$\vb*{j}_\text{f}$和$\vb*{E}$的显式关系,以及输运方程
\begin{equation}
    \pdv{\rho_\text{f}}{t} + \div{\vb*{j}_\text{f}} = 0
    \label{eq:transportation}
\end{equation}
就能够定解。

因此要求解出介质中的电磁场变化情况,首先需要\concept{物理方程}\eqref{eq:maxwell-material},
然后是\concept{本构关系}也就是$\vb*{D}$,$\vb*{H}$,$\vb*{j}_\text{f}$关于其他量的表达式,最后是\concept{几何关系}\eqref{eq:transportation},
再加上适当的\concept{边界条件}和\concept{初始条件},就能够定解。

关于本构关系实际上有一个问题,就是从$\vb*{E}$,$\vb*{B}$,$\vb*{j}_\text{f}$到$\vb*{D}$和$\vb*{H}$是不是真的有一个函数关系。
如果相同的$\vb*{E}$,$\vb*{B}$,$\vb*{j}_\text{f}$实际上对应着不同的系统状态,那就糟糕了。
但是在经典电动力学中确实只需要$\vb*{E}$,$\vb*{B}$是仅有的场,
而如果对$\vb*{j}_\text{s}$和$\vb*{j}_\text{c}$加上足够的限制,总是可以使用$\vb*{j}_\text{f}$确定下整个$\vb*{j}$的分布,从而$\rho$的分布,
因此$\vb*{E}$,$\vb*{B}$,$\vb*{j}_\text{f}$能够完全确定系统状态,从而本构关系总是可以写出来的。

\subsection{线性介质假设到光学方程}

\subsubsection{单色波解}

现在我们认为不存在自由电流和自由电荷,从而\eqref{eq:maxwell-material}成为
\begin{equation}
    \begin{bigcase}
        \div{\vb*{D}} &= 0, \\
        \curl{\vb*{E}} &= - \pdv{\vb*{B}}{t}, \\
        \div{\vb*{B}} &= 0, \\
        \curl{\vb*{H}} &= \pdv{\vb*{D}}{t}
    \end{bigcase}
    \label{eq:no-free-charge}
\end{equation}
并且假定所有本构关系都是线性的(这里由于自由电荷被认为是零,不再需要考虑关于$\vb*{j}_\text{f}$的本构关系),
且$\vb*{D}$只和$\vb*{E}$有关,$\vb*{H}$只和$\vb*{B}$有关。
这样就可以写出一般的本构关系
\[
    \begin{aligned}
        \vb*{D}(\vb*{r}, t) &= \int K_E(\vb*{r} - \vb*{r}', t - t') \vb*{E}(\vb*{r}', t') \dd^3 \vb*{r}' \dd t', \\
        \vb*{H}(\vb*{r}, t) &= \int K_B(\vb*{r} - \vb*{r}', t - t') \vb*{B}(\vb*{r}', t') \dd^3 \vb*{r}' \dd t'
    \end{aligned}
\]
其中的$K_E$和$K_B$都是张量。我们再要求本构关系是局部的,则应有%
\footnote{在本节中$K_E$和$K_B$代表不同的常(张)量,其值在不同地方可能不同}
\[
    \begin{aligned}
        \vb*{D}(\vb*{r}, t) &= \int K_E(\vb*{r}, t-t') \vb*{E}(\vb*{r}, t') \dd t', \\
        \vb*{H}(\vb*{r}, t) &= \int K_B(\vb*{r}, t-t') \vb*{B}(\vb*{r}, t') \dd t'
    \end{aligned}
\]
这是一个时间上的卷积运算。由傅里叶变换,我们只需要讨论平面波的本构关系就可以了,因为其它所有形式的场都能够写成平面波的叠加,既然\eqref{eq:no-free-charge}是线性齐次的(这就是假定没有自由电荷的好处!)。于是,不失一般性的,假设所有的场都取
\[
    \vb*{A}(\vb*{r}) \ee^{-\ii \omega t}
\]
的形式,其中$\vb*{A}(\vb*{r})$可以有虚部,但是一定能够写成某个实矢量乘以$\ee^{\ii \phi}$的形式%
\footnote{在这个条件下,可能的$\vb*{A}$并不能覆盖整个$\complexes^3$,但附加这个条件并不会失去一般性,因为允许$\vb*{A}$有虚部是为了表示$\vb*{A}_0 \cos (\omega t + \phi)$这样的函数,而这一点在限制$\vb*{A}(\vb*{r})$取某个实矢量乘以$\ee^{\ii \phi}$的形式时足以满足。这是三维下的实函数的傅里叶变换的一个特点。}
,则只需要形如
\begin{equation}
    \vb*{D}(\vb*{r}) = K_E(\vb*{r}, \omega) \vb*{E}(\vb*{r}), \quad \vb*{H}(\vb*{r}) = K_B(\vb*{r}, \omega) \vb*{B}(\vb*{r})
    \label{eq:linear-constitutive}
\end{equation}
的本构关系就可以了。\eqref{eq:linear-constitutive}中的所有变量可以取它们完整的表达式,也可以取它们的振幅。注意这些量可以是复数(用以表示相位)。
此时的麦克斯韦方程组成为
\begin{equation}
    \begin{bigcase}
        \div{\vb*{D}} &= 0, \\
        \curl{\vb*{E}} &= \ii \omega \vb*{B}, \\
        \div{\vb*{B}} &= 0, \\
        \curl{\vb*{H}} &= - \ii \omega \vb*{D}
    \end{bigcase}
    \label{eq:sin-wave-eqs}
\end{equation}

现在转而讨论有自由电流和自由电荷的情况。我们认为自由电流正比于电场,即服从\concept{欧姆定律}。
同样不失一般性地设所有的场都可以写成一个复振幅乘以$\ee^{- \ii \omega t}$的形式。
使用导出$\vb*{D}$和$\vb*{E}$的关系、$\vb*{H}$和$\vb*{B}$的关系同样的方法,有
\begin{equation}
    \vb*{j}_\text{f}(\vb*{r}) = \sigma(\vb*{r}, \omega) \vb*{E}(\vb*{r})
    \label{eq:ohm-law-sin-wave}
\end{equation}
那么从\eqref{eq:maxwell-material}和\eqref{eq:transportation}可以得到
\[
    \begin{bigcase}
        \div{\vb*{D}} &= \rho_\text{f}, \\
        \curl{\vb*{E}} &= \ii \omega \vb*{B}, \\
        \div{\vb*{B}} &= 0, \\
        \curl{\vb*{H}} &= \vb*{j}_\text{f} - \ii \omega \vb*{D}, \\
        - \ii \omega \rho_\text{f} + \div{\vb*{j}_\text{f}} &= 0,
    \end{bigcase}
\]
我们希望能够消去$\rho_\text{f}$和$\vb*{j}_\text{f}$,为此反复使用输运方程得到
\[
    \begin{bigcase}
        \div{\left( \vb*{D} - \frac{\vb*{j}_\text{f}}{\ii \omega} \right)   } &= 0, \\
        \curl{\vb*{E}} &= \ii \omega \vb*{B}, \\
        \curl{\vb*{H}} &=  - \ii \omega \left( \vb*{D} - \frac{\vb*{j}_\text{f}}{\ii \omega} \right)
    \end{bigcase}
\]
代入本构关系\eqref{eq:ohm-law-sin-wave}得到
\[
    \begin{bigcase}
        \div{\left( \vb*{D} - \frac{\sigma \vb*{E}}{\ii \omega} \right)   } &= 0, \\
        \curl{\vb*{E}} &= \ii \omega \vb*{B}, \\
        \curl{\vb*{H}} &=  - \ii \omega \left( \vb*{D} - \frac{\sigma \vb*{E}}{\ii \omega} \right)
    \end{bigcase}
\]
注意到$\vb*{D}$和$\vb*{E}$、$\vb*{H}$和$\vb*{B}$之间的关系(见\eqref{eq:linear-constitutive}),做下面的替换
\[
    \vb*{D} - \frac{\sigma \vb*{E}}{\ii \omega} \longrightarrow \vb*{D}, \quad K_E - \frac{\sigma}{\ii \omega} \longrightarrow K_E
\]
得到的方程和\eqref{eq:sin-wave-eqs}形式完全一致,本构关系的形式也还是\eqref{eq:linear-constitutive}。
唯一的不同是,$K_E$和$K_B$都是复数,并且
\begin{equation}
    \Im K_E = \frac{\sigma}{\omega}
\end{equation}

\subsubsection{各向同性介质中的亥姆霍兹方程}

我们特别感兴趣的是\eqref{eq:linear-constitutive}中$K_E$和$K_B$都是标量(不一定是实数)的情况,
也就是说,\eqref{eq:linear-constitutive}中给出的响应是\concept{各向同性}的。
此时有
\begin{equation}
    \vb*{D}(\vb*{r}) = \epsilon(\vb*{r}, \omega) \vb*{E}(\vb*{r}), \quad \vb*{H}(\vb*{r}) = \frac{1}{\mu(\vb*{r}, \omega)} \vb*{B}(\vb*{r})
    \label{eq:scalar-constitutive}
\end{equation}
第二个公式的比例系数特意被放到了分母中,以达到和\eqref{eq:original-maxwell}相同的形式。
将\eqref{eq:scalar-constitutive}代入\eqref{eq:sin-wave-eqs}中,得到
\begin{equation}
    \begin{bigcase}
        \div{(\epsilon \vb*{E})} &= 0, \\
        \curl{\vb*{E}} &= \ii \omega \vb*{B}, \\
        \div{\vb*{B}} &= 0, \\
        \curl{\left(\frac{\vb*{B}}{\mu}\right)} &= - \ii \omega \epsilon \vb*{E}
    \end{bigcase}
    \label{eq:scalar-cons-maxwell-e-and-b}
\end{equation}
上式中第三式可以通过第二式推导而来;将第二式表示出的$\vb*{B}$代入第四式可以消去$\vb*{B}$。于是将\eqref{eq:scalar-cons-maxwell-e-and-b}简化为
\begin{equation}
    \begin{bigcase}
        \div{(\epsilon \vb*{E})} &= 0, \\
        \curl{ \left(\frac{1}{\mu} \curl{\vb*{E}} \right) } &= \omega^2 \epsilon \vb*{E}, \\
        \curl{\vb*{E}} &= \ii \omega \vb*{B}
    \end{bigcase}
    \label{eq:e-in-material}
\end{equation}
这个方程关于电场的部分等价于
\[
    \begin{bigcase}
        \div{(\epsilon \vb*{E})} &= 0, \\
        \laplacian \vb*{E} + \mu \epsilon \omega^2 \vb*{E} &= - \frac{1}{\epsilon} \grad{(\vb*{E} \cdot \grad{\epsilon})} - \frac{1}{\mu} \grad{\mu} \cross (\curl{\vb*{E}})
    \end{bigcase}
\]
以上方程是基于$\vb*{E}$的,也可以写出基于$\vb*{H}$的类似的方程,遵从同样的步骤,可以得到
\begin{equation}
    \begin{bigcase}
        \div{\mu \vb*{H}} &= 0, \\
        \curl{\left( \frac{1}{\epsilon} \curl{\vb*{H}} \right)} &=  \mu \omega^2 \vb*{H}, \\
        \curl{\vb*{E}} &= \ii \omega \mu \vb*{H}.
    \end{bigcase}
    \label{eq:h-in-material}
\end{equation}
\eqref{eq:e-in-material}或\eqref{eq:h-in-material}称为\concept{主方程}。

特别的,在$\epsilon, \mu$在空间中处处相等或者变化得比较缓慢时,有
\begin{equation}
    \laplacian \vb*{E} + \frac{\omega^2}{c^2} \vb*{E} = 0, \quad c^2 = \frac{1}{\epsilon \mu}
    \label{eq:halmholtz-eq}
\end{equation}
\eqref{eq:halmholtz-eq}就是所谓的\concept{亥姆霍兹方程},求解出它就求解出了整个\eqref{eq:e-in-material}。
需要注意的是并不是所有\eqref{eq:halmholtz-eq}的解都是\eqref{eq:e-in-material}的解,因为\eqref{eq:halmholtz-eq}没有包含\eqref{eq:e-in-material}的第一式。当然这个信息可以在求解\eqref{eq:halmholtz-eq}的时候使用一些边界条件加上去。
这里占用了$c$表示介质中光速,相对应地,设真空中光速为$c_0$,
其值为$1/\sqrt{\epsilon_0 \mu_0}$,因为简单地令$\epsilon=\epsilon_0$,$\mu = \mu_0$,
\eqref{eq:scalar-cons-maxwell-e-and-b}就退化到了真空的情况。

需要注意的是由于$\mu$和$\epsilon$可能有虚部(对应着吸收等情况),$c^2$不一定是实数。这就产生了一个问题:开根号在复平面上是多值的,那么$c$应该取哪一个值呢?

当$c$在某区域中基本上可以看成常的实数时,它就对应着平面波传播的速度。\eqref{eq:halmholtz-eq}有下面的平面波解
\begin{equation}
    \vb*{E} = \vb*{E}_0 \ee^{\ii(\vb*{k} \cdot \vb*{r} - \omega t)}, \quad k = \frac{\omega}{c}
    \label{eq:plane-wave}
\end{equation}
其中$\vb*{k}$是实矢量。
由空间中的傅里叶变换,\eqref{eq:halmholtz-eq}的所有解都可以写成不同$\vb*{k}$的平面波的叠加。

以上完成了关于介质中的控制方程的探讨。下面考虑界面上的衔接条件。最自然的想法,由\eqref{eq:scalar-cons-maxwell-e-and-b},可以写出自然边界条件
\[
    \begin{bigcase}
        \vb*{n} \cdot (\epsilon_i \vb*{E}_1 - \epsilon_t \vb*{E}_2) &= 0, \\
        \vb*{n} \times (\vb*{E}_1 - \vb*{E}_2) &= 0, \\
        \vb*{n} \cdot (\vb*{B}_1 - \vb*{B}_2) &= 0, \\
        \vb*{n} \times \left( \frac{\vb*{B}_1}{\mu_1} - \frac{\vb*{B}_2}{\mu_2} \right) &= 0
    \end{bigcase}
\]
但是正如\eqref{eq:scalar-cons-maxwell-e-and-b}实际上有冗余一样,上式也有方程是多余的。实际上,由\eqref{eq:e-in-material}可以得出的关于电场的相互独立的边界条件为
\begin{equation}
    \left\{\quad
        \begin{aligned}
            \epsilon_1 \vb*{n} \cdot \vb*{E}_1 = \epsilon_2 \vb*{n} \cdot \vb*{E}_2, \\
            \vb*{n} \times \vb*{E}_1 = \vb*{n} \times \vb*{E}_2, \\
            \vb*{n} \times \left( \frac{1}{\mu_1} \curl{\vb*{E}_1} \right) = \vb*{n} \times \left(\frac{1}{\mu_2} \curl{\vb*{E}_2} \right)
        \end{aligned}
    \right.
    \label{eq:e-bound-condition}
\end{equation}
只需要这些边界条件结合\eqref{eq:e-in-material}就能够定解。

\subsubsection{电各向异性光学介质}

现在考虑一种稍为推广的情况。很多介质——比如晶体——都具有空间上的各向异性,这是因为从不同的方向施加电场可以导致不同强度的极化。
在几乎所有常见的情况中,各向异性仅限于$\vb*{D}$和$\vb*{E}$的关系中,于是\eqref{eq:scalar-constitutive}修正为
\begin{equation}
    \vb*{D}(\vb*{r}) = \vb*{\epsilon}(\vb*{r}, \omega) \vb*{E}(\vb*{r}), \quad \vb*{H}(\vb*{r}) = \frac{1}{\mu(\vb*{r}, \omega)} \vb*{B}(\vb*{r})
    \label{eq:e-tensor-constitutive}
\end{equation}
相应的,\eqref{eq:e-in-material}修改为
\begin{equation}
    \begin{bigcase}
        \div{(\vb*{\epsilon} \cdot \vb*{E})} &= 0, \\
        \curl{ \left(\frac{1}{\mu} \curl{\vb*{E}} \right) } &= \omega^2 \vb*{\epsilon} \cdot \vb*{E}, \\
        \curl{\vb*{E}} &= \ii \omega \vb*{B}
    \end{bigcase}
    \label{eq:e-in-tensor-material}
\end{equation}

我们要研究$\mu$和$\vb*{\epsilon}$变化不大时\eqref{eq:e-in-tensor-material}的平面波解。
此时关于电场的全部方程为
\[
    \begin{aligned}
        \div{\vb*{\epsilon} \cdot \vb*{E}} = 0, \\
        \grad{(\div{\vb*{E}})} - \laplacian{\vb*{E}} = \omega^2 \mu \vb*{\epsilon} \cdot \vb*{E}
    \end{aligned}
\]
不过不难注意到第一个方程可以从第二个方程推导出来,因此方程
\begin{equation}
    \grad{(\div{\vb*{E}})} - \laplacian{\vb*{E}} = \omega^2 \mu \vb*{\epsilon} \cdot \vb*{E}
    \label{eq:anisotropy}
\end{equation}
如果$\vb*{E}$是一个波矢为$\vb*{k}$,频率为$\omega$的单色波,那么这两个方程就推出
\[
    \begin{aligned}
        - (\vb*{k} \cdot \vb*{E}) \vb*{k} + k^2 \vb*{E} = \omega^2 \mu \vb*{\epsilon} \cdot \vb*{E}, \\
    (\vb*{k} \cdot \vb*{\epsilon}) \cdot \vb*{E} = 0
    \end{aligned}
\]
但是这两个方程不是相互独立的——在第一个方程两边点乘$\vb*{k}$就可以推出第二个。
顺带提一句:第二个方程表明,在介质是电各向异性的情况下,波矢$\vb*{k}$并不垂直于电场,而是垂直于$\vb*{D}$。

那么电各向异性光学介质中的平面波就完全地被下式描写:
\[
    k^2 \vb*{E} - (\vb*{k} \cdot \vb*{E}) \vb*{k} - \omega^2 \mu \vb*{\epsilon} \cdot \vb*{E} = 0
\]
也就是说
\begin{equation}
    \left( k^2 \vb*{\delta} - \vb*{k} \vb*{k} - \omega^2 \mu \vb*{\epsilon} \right) \cdot \vb*{E} = 0
    \label{eq:plain-wave-in-anistrophy}
\end{equation}
其中$\vb*{\delta}$为单位张量。

\eqref{eq:plain-wave-in-anistrophy}将电场的方向和波矢的方向联系了起来。要观察波矢自己要满足什么条件,只需要求解
\begin{equation}
    \det \left( k^2 \vb*{\delta} - \vb*{k} \vb*{k} - \omega^2 \mu \vb*{\epsilon} \right) = 0
    \label{eq:k-det}
\end{equation}
即可。

直接处理矢量形式的方程会比较棘手。但是注意到由于能量守恒的原因,$\vb*{D}$不可能和$\vb*{E}$反向,也就是
\[
    \vb*{D} \cdot \vb*{E} = \vb*{E} \cdot \vb*{\epsilon} \cdot \vb*{E} > 0
\]
那么我们可以确定$\vb*{\epsilon}$是正定的(TODO:有虚部时怎么办?)另一方面,晶体的空间对称性意味着$\vb*{\epsilon}$一定可以对角化,因此我们总是可以找到一个坐标系(未必是正交坐标系),在其中$\vb*{\epsilon}$被对角化,且对角元均为正数。
这个坐标系就是所谓的\concept{主轴系}。

\subsection{介质中的能流}

为了讨论能量,这一节我们再次引入自由电荷的概念,用于讨论场中有实物粒子的情况。(如果实物粒子不带电荷,那么它就不参与电磁相互作用,此时不方便从牛顿力学出发讨论能量)
我们只追踪自由电荷的能量,而不追踪所有电荷的能量。
于是,使用\eqref{eq:maxwell-material}以及洛伦兹力公式
\[
    \vb*{f} = q \vb*{E} + q \vb*{v} \times \vb*{B}
\]
可以得到
\[
    \vb*{f} \cdot \vb*{v} = - \div{(\vb*{E} \times \vb*{H})} - \vb*{H} \cdot \pdv{\vb*{B}}{t} - \vb*{E} \cdot \pdv{\vb*{D}}{t}
\]
则可以选取最简单的电磁能量密度和电磁能流为
\begin{equation}
    \begin{bigcase}
        \vb*{S} &= \vb*{E} \times \vb*{H}, \\
        \pdv{w}{t} &= \vb*{E} \cdot \pdv{\vb*{D}}{t} + \vb*{H} \cdot \pdv{\vb*{B}}{t}
    \end{bigcase}
    \label{eq:energy-in-material}
\end{equation}
其中的$\vb*{S}$就是\concept{坡印廷矢量}。公式\eqref{eq:energy-in-material}的导出没有使用线性介质假设。

在已知$\vb*{D}$和$\vb*{E}$的关系线性、$\vb*{H}$和$\vb*{B}$的关系也是线性的情况下,有
\begin{equation}
    w = \frac{1}{2} (\vb*{E} \cdot \vb*{D} + \vb*{H} \cdot \vb*{B})
    \label{eq:energy-density-linear-material}
\end{equation}

需要注意的是\eqref{eq:energy-in-material}和\eqref{eq:energy-density-linear-material}都不是线性的,
因此在这些公式中不能让各个场有虚部。

在稳态时各个场做正弦变化,这也是本文关注的主要场景。首先讨论介质中只有一个单色波的情况。不失一般性地认为$\vb*{E}$的相位零点在$t=0$处,则有
\[
    \vb*{S} = \vb*{E}_0 \times \vb*{H}_0 \cos(\omega t) \cos(\omega t + \phi)
\]
其中$\phi$是两者的相位差,$\vb*{E}_0, \vb*{H}_0$分别是振幅,这三者均只有实部。我们更加关心一段时间内的平均能流,则有
\[
    \expval*{\vb*{S}} = \frac{1}{2} \vb*{E}_0 \times \vb*{H}_0 \cos \phi
\]
现在考虑含有虚部的$\vb*{E}, \vb*{H}$表达式
\[
    \vb*{E} = \vb*{E}_0 \ee^{\ii \omega t}, \vb*{H} = \vb*{H}_0 \ee^{\ii \phi} \ee^{\ii \omega t}
\]
我们会发现一个有趣的结果:
\[
    \Re \vb*{E}^* \times \vb*{H} = \expval*{\vb*{S}}
\]
直观地看,可以将$\vb*{E}^* \times \vb*{H}$的实部看成“有功分量”,虚部看作“无功分量”。
这样可以定义\concept{实数版本的辐照度}
\begin{equation}
    \vb*{I} = \expval*{\vb*{S}} = \frac{1}{2} \vb*{E}_0 \times \vb*{H}_0 \cos \phi
    \label{eq:radiation-real}
\end{equation}
也可以定义\concept{复数版本的辐照度}
\begin{equation}
    \vb*{I} = \frac{1}{2} \Re \vb*{E}^* \times \vb*{H}
    \label{eq:radiation-complex}
\end{equation}
\eqref{eq:radiation-complex}的实部就是\eqref{eq:radiation-real}。

接着再讨论平面波的能量密度。由于
% \[
%     \vb*{k} \cdot \vb*{B} = 0, \quad \vb*{D} = - \frac{\vb*{k}}{\omega} \cross \vb*{H},
% \]
% 我们有
% \[
%     D^2 = \frac{k^2}{\mu^2 \omega^2} B^2
% \]
% 则
% \[
%     \frac{1}{2} \vb*{H} \cdot \vb*{B} = \frac{\mu \omega^2}{2 k^2} D^2, 
% \]
% \[
%     \frac{1}{2} \vb*{E} \cdot \vb*{D} = \frac{1}{2} \vb*{E} \cdot \vb*{\epsilon} \cdot \vb*{E}
% \]
% 于是
% \begin{equation}
%     w = w_E + w_B = \frac{1}{2} \vb*{E} \cdot \vb*{\epsilon} \cdot \vb*{E} + \frac{\mu \omega^2}{2 k^2} D^2
%     \label{eq:energy-density}
% \end{equation}
\[
    w = w_E + w_B,
\]
分别化简两项,有
\[
    w_E = \frac{1}{2} \vb*{E} \cdot \vb*{D} 
        = \frac{1}{2} \vb*{E} \cdot \left( - \frac{\vb*{k}}{\omega} \times \vb*{H} \right) 
        = \frac{1}{2} \frac{\vb*{k}}{\omega} \cdot (\vb*{E} \times \vb*{H}),
\]
且
\[
    w_B = \frac{1}{2} \vb*{H} \cdot \vb*{B} 
        = \frac{1}{2} \vb*{H} \cdot \left( \frac{\vb*{k}}{\omega} \cdot \vb*{E} \right) 
        = \frac{1}{2} \frac{\vb*{k}}{\omega} \cdot (\vb*{E} \times \vb*{H}),
\]
因此电场能和磁场能各占总能量的一半,且
\begin{equation}
    w = \frac{\vb*{k}}{\omega} \cdot (\vb*{E} \times \vb*{H}) = \frac{k S}{\omega} \cos \alpha
    \label{eq:energy-density-and-s}
\end{equation}
其中$\alpha$是$\vb*{k}$和$\vb*{S}$的夹角。
相应的,能量传递的速度为
\begin{equation}
    \vb*{v} = \frac{\vb*{S}}{w} = \frac{\omega}{k}  \cos \alpha \vu*{S}
\end{equation}
