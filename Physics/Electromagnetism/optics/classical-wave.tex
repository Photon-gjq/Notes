\section{光强,成像和相干性}

\subsection{几何光学中的光强传输}

实际上,我们也可以将\eqref{eq:optical-distance}当成$L$的\emph{定义},此时无论几何光学是不是适用,都能够定义光线等。在此基础上,\eqref{eq:expanded-wave-eq}化为
\begin{equation}
    \laplacian \vb*{A} + 2 \frac{\ii \omega}{c_0} \grad{\vb*{A}} \cdot \grad{L} + \frac{\ii \omega}{c_0} \vb*{A} = 0.
\end{equation}
如果我们做\concept{慢变振幅近似},则上式可以写成
\begin{equation}
    \div{(\vb*{A}^2 \grad{L})} = 0.
\end{equation}
因此实际上$\vb*{A}^2 \grad{L}$可以看成一个静态的“光强传输的流”,即光线实际上给出了光强传输的流管。
物理地看,由于是各向同性介质,$\vb*{S}$和$\vb*{k}$平行,这意味着$\vu*{k}$的方向连缀而成的“光线”实际上就是$\vb*{S}$的流量线,因此若干光线包围出的“管路”实际上就是能量流动的流管。

当两束光相交时,光线可以交叉,因此“流管”的概念实际上并没有良好定义——实际上,这正是衍射、干涉等现象的起源。% TODO

因此,在慢变振幅近似成立时,使用几何光学确定光线,并以光线为振幅传输的“流线”就足够给出可靠的结果了。
我们通常称\emph{此时几何光学适用}。理论上我们对任何体系都可以通过光线方程计算光线,但是最终我们关系的是空间中各点的亮度分布,因此如果光线的概念无助于计算亮度分布,则几何光学没有什么意义,这就好像对任何一个量子理论我们都可以计算其经典版本,但是有时候计算经典版本并不能提供什么信息。

假定光学系统在每个瞬时都可以认为是处在某个稳态上,从而,虽然我们在分析动态问题,沿用亥姆霍兹方程的解足以给出精确的结果。
然而,假定光学系统中的光源会以一种随机的方式发射电磁波,从而空间中某一点的光强实际上是一系列具有不同概率权重的亥姆霍兹方程的解在这一点给出的光强的期望值。
写成公式,设某一点的电场包含从各个方向传来的电场(注意此时我们已经切换到了亥姆霍兹方程下,所谓“电场传播”实际上是电场在空间上的联系,虽然它和时域下波包的传播是直接相关的)之和
\begin{equation}
    \vb*{E}(\vb*{r}) = \sum_i \vb*{E}_i \ee^{\ii k_i \abs*{\vb*{r} - \vb*{r}_i}},
\end{equation}
展开电场平方的期望值,有
\begin{equation}
    \begin{aligned}
        \expval*{\vb*{E}(\vb*{r})^2} &= \sum_P P(\vb*{E}_1, \vb*{E}_2, \cdots, \vb*{E}_n) \sum_{i, j} \vb*{E}_i \ee^{\ii k_i \abs*{\vb*{r} - \vb*{r}_i}} \vb*{E}_j^* \ee^{- \ii k_j \abs*{\vb*{r} - \vb*{r}_j}}  \\
        &= \sum_{i, j} \expval*{\vb*{E}_i \vb*{E}_j^*} \ee^{\ii (k_i \abs*{\vb*{r} - \vb*{r}_i} - k_j \abs*{\vb*{r} - \vb*{r}_j})}.
    \end{aligned}
\end{equation}
在组成$\vb*{E}$的各个组分极度非相干的情况下,$\expval*{\vb*{E}_i \vb*{E}_i^*}$项占据压倒性优势,从而
\begin{equation}
    \expval*{\vb*{E}^2} = \sum_i \expval*{\vb*{E}_i^2}.
\end{equation}
这就是说,对高度非相干的情况,不同来源的光强可以直接相加,而无需考虑衍射等问题。
这实际上说明对高度非相干的光几何光学通常都是适用的。但这并不是说以高度非相干的光为光源就产生不了干涉和衍射,例如我们将一束非相干光分束,得到的两束光中,一束光的一个分量和另一束光中的一个分量相干,从而仍然可能产生干涉。
必须每两个$\vb*{E}_i$和$\vb*{E}_j$之间都不相干才能够保证几何光学总是成立。

由于几何光学成立时光线实际上就是光功率的流线,如果在一个闭合表面上光功率通量正比于光线条数,那么做光线追踪后会发现任何一个面上的光通量也都正比于这个面上的光线条数。

何时衍射不明显:波长非常短的时候肯定不明显;相干性差的时候也是;这和“量子效应什么时候不明显”是类似的: