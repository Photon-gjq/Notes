\documentclass[UTF8, a4paper]{ctexart}

\usepackage{geometry}
\usepackage{titling}
\usepackage{titlesec}
\usepackage{paralist}
\usepackage{footnote}
\usepackage{enumerate}
\usepackage{amsmath, amssymb, amsthm}
\usepackage{cite}
\usepackage{graphicx}
\usepackage{subfigure}
\usepackage{physics}
\usepackage[colorlinks, linkcolor=black, anchorcolor=black, citecolor=black]{hyperref}

\geometry{left=3.28cm,right=3.28cm,top=2.54cm,bottom=2.54cm}
\titlespacing{\paragraph}{0pt}{1pt}{10pt}[20pt]
\setlength{\droptitle}{-5em}
\preauthor{\vspace{-10pt}\begin{center}}
\postauthor{\par\end{center}}

\newcommand*{\ee}{\mathrm{e}}
\newcommand*{\ii}{\mathrm{i}}
\newcommand*{\st}{\quad \text{s.t.} \quad}
\newcommand*{\const}{\mathrm{const}}
\newcommand*{\natnums}{\mathbb{N}}
\newcommand*{\reals}{\mathbb{R}}
\newcommand*{\complexes}{\mathbb{C}}
\DeclareMathOperator{\timeorder}{T}
\newcommand*{\ogroup}[1]{\mathrm{O}(#1)}
\newcommand*{\sogroup}[1]{\mathrm{SO}(#1)}
\DeclareMathOperator{\legpoly}{P}

\newcommand{\concept}[1]{\underline{#1}}
\renewcommand{\emph}{\textbf}

\title{量子光学}
\author{吴晋渊}

\begin{document}

\maketitle

\section{第一性原理}

\subsection{粒子,以及它和光场的耦合}

\subsubsection{非相对论性粒子的哈密顿量}

考虑与电磁场发生相互作用的粒子,我们通常将这些粒子称为物质而将电磁场称为光场或是辐射,虽然严格说起来辐射也算是一种物质。
我们假定粒子做低速运动,从而不需要使用相对论性的理论描述粒子。
粒子轨道部分的哈密顿量是以下保证局部$U(1)$规范对称性的极小耦合:
\begin{equation}
    {H}_\text{orbit} = \frac{1}{2m} ({\vb*{p}} - q \vb*{A})^2 + q \phi,
    \label{eq:minimal-coupling}
\end{equation}
自旋-磁场相互作用还会引入以下哈密顿量:
\begin{equation}
    {H}_\text{spin} = - \frac{q}{m} {\vb*{S}} \cdot \vb*{B} = - \vb*{\mu} \cdot \vb*{B},
\end{equation}
而场的哈密顿量是
\begin{equation}
    {H}_\text{field} = \frac{\epsilon_0}{2} \int \dd[3]{\vb*{r}} (\vb*{E}^2 + c^2 \vb*{B}^2),
\end{equation}
则体系的总哈密顿量
\begin{equation}
    {H} = \sum_i \left( \frac{1}{2m_i} ({\vb*{p}_i} - q_i \vb*{A})^2 + q_i \varphi - \frac{q_i}{m_i} \vb*{S}_i \cdot \vb*{B} \right) + {H}_\text{field} + {H}_\text{int} + {H}_\text{ext},
\end{equation}
其中${H}_\text{int}$和${H}_\text{ext}$分别表示粒子间相互作用和外加势场。
粒子部分——包括轨道和自旋——的拉氏量也可以写成
\begin{equation}
    L = \sum_i \left( \frac{1}{2} m_i \vb*{v}_i^2 - q_i \varphi + q_i \vb*{v}_i \cdot \vb*{A} + \vb*{\mu}_i \cdot \vb*{B} \right).
\end{equation}
具体什么是粒子-粒子相互作用其实有一定人为因素,比如说凝聚态场论中默认电子之间的相互作用是库伦相互作用,但是库伦相互作用其实也是交换光子导致的,实际上是近场辐射的一个无时间延迟近似。
同样,“外加势场”也有人为因素。

\eqref{eq:minimal-coupling}中的$\vb*{p}$是正则动量,而不是机械动量。
然而,这反倒有好处:我们要讨论的是“向一个物理系统入射光会得到怎样的出射光”,根本不需要去测量机械动量。
这种情况下我们完全没有必要关注$\vb*{p}$是正则动量这回事:完全可以打开括号$(\vb*{p} - q \vb*{A})^2$,然后求解束缚态问题
\begin{equation}
    H = \sum_i \frac{\vb*{p}_i^2}{2m_i} + H_\text{ext} + H_\text{int},
    \label{eq:levels-ham}
\end{equation}
具体求解时可以直接把$\vb*{p}$当成机械动量而套用一些现成的解,得到能谱之后引入电子-电磁波耦合项
\begin{equation}
    H_\text{couple} = q \varphi - \frac{q}{2m} (\vb*{p} \cdot \vb*{A} + \vb*{A} \cdot \vb*{p}) + \frac{q^2}{2m} \vb*{A}^2,
    \label{eq:couple-ham}
\end{equation}
计算物质和光场的耦合。($\vb*{A}^2$项中含有粒子的位置,因此也是耦合项)

\eqref{eq:couple-ham}原则上就给出了所有值得关注的信息。
在一些情况下我们还可以进一步简化问题。
我们在讨论的到底还是一个光学问题。光是从远处发射过来的,从而在我们讨论的空间区域内,光场可以认为没有源??
可以自由地取$\varphi=0$而于此同时$\div{\vb*{A}}=0$。

\subsubsection{束缚态系统,微扰论和多极矩展开}

将\eqref{eq:couple-ham}当成微扰做微扰论的适用条件是$H_\text{couple}$相对于\eqref{eq:levels-ham}来说很小。
如果微扰论适用,那么显然$q \vb*{A} \ll \vb*{p}$,从而$\vb*{A}^2$项相较于$\vb*{p} \cdot \vb*{A}$项总是非常小的。%
\footnote{
    一个可以抬杠的地方是$\vb*{p}$很小时,似乎$\vb*{p} \cdot \vb*{A}$项远小于$\vb*{A}^2$项。
    然而,由能量守恒,$\vb*{A}^2$项相比于动能加上势能的\eqref{eq:levels-ham}总是很小的。
    如果我们只要求$\vb*{A}^2$级别的精度,那么在$\vb*{p}$大时显然$\vb*{p} \cdot \vb*{A}$项比$\vb*{A}^2$项重要,而$\vb*{p}$小时$\vb*{A}^2$项小于我们的精度要求,因此也可以直接略去。
    无论如何,$\vb*{A}^2$项都不如$\vb*{p} \cdot \vb*{A}$重要——后者重要时前者不重要,后者不重要时前者也没有重要到哪儿去。
}%
$q \vb*{A} \ll \vb*{p}$的条件实际上是不那么平凡的。
对散射态系统,机械动量估计为
\[
    m v \sim m \omega x,
\]
而
\[
    q E = m \ddot{x} \sim m \omega^2 x,
\]
最后有
\[
    E \sim - \pdv{A}{t} \sim \omega A,
\]
于是我们会发现$mv$和$eA$实际上是同个量级的。反之,对束缚态系统,$\vb*{p}$的最大值或者说振幅可以估计为
\[
    m \omega^2 x \sim q \grad{V_\text{ext}},
\]
而
\[
    mv \sim m \omega x,
\]
于是$p \gg eA$,等价于$mv \ll eA$,就等价于
\[
    mv \sim \frac{q}{\omega} \grad{V_\text{ext}} \gg q A,
\]
即等价于
\begin{equation}
    \grad{V_\text{ext}} \gg \omega A \sim E_\text{light},
\end{equation}
即束缚电场远大于光场。这应该是能够保证的,否则就不是束缚态了,此时介质就被打穿为等离子体了,并且,这种情况下,将光场撤去,介质也未必会恢复为原状,即出现了光学损伤。

在知道了能将\eqref{eq:couple-ham}当成微扰的系统中的带电粒子处于束缚态之后,我们立刻想到,由于这些带电粒子的位置高度有界,可以做多极矩展开。
实际上我们看到,多极矩展开合法、带电粒子位置定域、带电粒子处于束缚态、$e \vb*{A} \ll \vb*{p}$、$\vb*{A}^2$可以略去这几个条件是等价的。
应该说\eqref{eq:couple-ham}是很不直观的,因为它是关于$\vb*{A}$的而不是$\vb*{E}$和$\vb*{B}$的,做完多极矩展开之后我们就可以讨论“某个过程在电偶极矩跃迁下可以发生,另一个过程需要电四极矩跃迁,从而很弱”,等等。
以下我们用$0$作为带点粒子位置的“原点”,$\vb*{r}$不会偏离$0$太远。

我们将从两条进路出发得到多极矩展开形式的哈密顿量。其中之一是一开始并不丢掉$\vb*{A}^2$项,但是利用带点粒子高度定域这个特点。
我们知道一般来说不能找到一个规范变换让$\vb*{A}$完全消失,因为这样一个变换要求
\[
    \vb*{A} \longrightarrow \vb*{A}' = \vb*{A} - \grad{\chi} = 0, \quad \varphi \longrightarrow \varphi' = \varphi + \pdv{\chi}{t},
\]
但是积分
\begin{equation}
    \chi(\vb*{r}, t) = \int_0^{\vb*{r}} \dd{\vb*{r}'} \cdot \vb*{A}(\vb*{r}', t)
    \label{eq:chi-transform}
\end{equation}
一般来说是多值的,并且$\curl{(\grad{\chi})}$不必然为零。然而,带电粒子高度定域的情况下,我们仍然可以取\eqref{eq:chi-transform}的一个单值分支,即将磁壳选取在远离带电粒子运动区域的地方。
这样,\eqref{eq:minimal-coupling}就被变换为
\[
    H_\text{orbit} = \frac{\vb*{p}^2}{2m} + q \pdv{\chi}{t} + q \varphi.
\]
在此过程中电子波函数也需要做规范变换,但是这当然并不重要,因为反正我们需要的只是电子能谱。


\begin{equation}
    \varphi = 0, \quad \vb*{A}(\vb*{r}, t) = - \int_0^t \dd{t'} \vb*{E}(\vb*{r}, t') + \frac{1}{2} \int_0^{\vb*{r}} \vb*{B}(\vb*{r}', t) \times \dd{\vb*{r}'}
\end{equation}
服从条件$\varphi=0, \div{\vb*{A}}=0$

\subsubsection{散射态系统和等离子体}

\subsubsection{电偶极跃迁的首要地位}

能够导出电偶极跃迁的方式:

\subsection{光场}

电磁场足够强以至于难以看到单光子效应,而又足够弱以至于能量不至于强到需要考虑量子电动力学的圈图修正,这样就可以使用经典电动力学描述整个系统。

\section{经典弱场极限}

\subsection{弱场极限的哈密顿量}

考虑单粒子和光场的相互作用的哈密顿量。假定场较弱,则可以略去$\vb*{A}$的高阶项,从而得到坐标表象下带电粒子和电磁场发生相互作用的哈密顿量:
\[
    {H} = - \frac{\hbar^2 \laplacian}{2m} + \frac{\ii \hbar q}{m} \vb*{A} \cdot \grad + \frac{\ii \hbar q}{2 m} \div\vb*{A} + q \phi - \frac{q}{m} {\vb*{S}} \cdot \vb*{B}.
\]
第一项就是粒子动能;可以通过适当选取规范,让第三、四项消失,于是我们就得到弱场下带电粒子和经典光场的相互作用哈密顿量:
\begin{equation}
    {H}_\text{light} = \frac{\ii \hbar q}{m c} \vb*{A} \cdot \grad - \frac{q}{m} {\vb*{S}} \cdot \vb*{B} = \underbrace{- \frac{q}{m} {\vb*{p}} \cdot \vb*{A}}_{{H}_\text{1}} \underbrace{- \frac{q}{m} {\vb*{S}} \cdot \vb*{B}}_{{H}_2}.
\end{equation}
实际上,磁场对自旋的取向作用${H}_2$是很弱的。设电磁波波长的尺度为$\lambda$,则
\[
    \vb*{B} = \curl{\vb*{A}} \sim \frac{A}{\lambda},
\]
电子的活动范围的尺度和原子半径$a_0$同阶,由不确定性关系,
\[
    p a_0 \sim \hbar.
\]
于是
\[
    \frac{H_2}{H_1} \sim \frac{\hbar \frac{A}{\lambda}}{\frac{\hbar}{a_0} A} = \frac{a_0}{\lambda}.
\]
波长通常在几百纳米级别,而原子半径在纳米级别以下,从而${H}_1$远大于${H}_2$。

下面给出了一种更加接近静电势多级展开的方法。首先取适当的规范让矢势消失,于是就得到
\[
    {H} = - \frac{\hbar^2 \laplacian}{2m} + q \phi - \frac{q}{m} {\vb*{S}} \cdot \vb*{B},
\]
粒子的运动高度定域,于是可以将坐标系原点选取在粒子活动区域的“中心”,做多极展开
\[
    \phi(\vb*{r}) = \phi(0) + \vb*{r} \cdot \grad{\phi} + \frac{1}{2} \vb*{r} \vb*{r} : \grad{\grad{\phi}} + \cdots,
\]
第一项是一个无关紧要的能量零点,第二项是电偶极辐射,等等。
如通常所做的那样定义偶极矩
\begin{equation}
    {\vb*{d}} = q {\vb*{r}},
\end{equation}
这样就有
\begin{equation}
    {H} = - \frac{\hbar^2 \laplacian}{2m} - {\vb*{d}} \cdot \vb*{E} + \cdots - \frac{q}{m} {\vb*{S}} \cdot \vb*{B}.
\end{equation}

\subsection{偶极辐射}


\end{document}