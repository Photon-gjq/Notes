\documentclass[UTF8, a4paper]{ctexart}

\usepackage{geometry}
\usepackage{titling}
\usepackage{titlesec}
\usepackage{paralist}
\usepackage{footnote}
\usepackage{enumerate}
\usepackage{amsmath, amssymb, amsthm}
\usepackage{cite}
\usepackage{graphicx}
\usepackage{subfigure}
\usepackage{physics}
\usepackage[colorlinks, linkcolor=black, anchorcolor=black, citecolor=black]{hyperref}

\geometry{left=3.28cm,right=3.28cm,top=2.54cm,bottom=2.54cm}
\titlespacing{\paragraph}{0pt}{1pt}{10pt}[20pt]
\setlength{\droptitle}{-5em}
\preauthor{\vspace{-10pt}\begin{center}}
\postauthor{\par\end{center}}

\newcommand*{\ee}{\mathrm{e}}
\newcommand*{\ii}{\mathrm{i}}
\newcommand*{\st}{\quad \text{s.t.} \quad}
\newcommand*{\const}{\mathrm{const}}
\newcommand*{\natnums}{\mathbb{N}}
\newcommand*{\reals}{\mathbb{R}}
\newcommand*{\complexes}{\mathbb{C}}
\DeclareMathOperator{\timeorder}{T}
\newcommand*{\ogroup}[1]{\mathrm{O}(#1)}
\newcommand*{\sogroup}[1]{\mathrm{SO}(#1)}
\DeclareMathOperator{\legpoly}{P}

\newcommand{\concept}[1]{\underline{\textbf{#1}}}
\renewcommand{\emph}{\textbf}

\title{非线性光学}
\author{吴何友}

\begin{document}

\maketitle

\section{激光}

当一束光照射在介质上时,分子、原子中的电子会离开其原有位置,并受到一个回复力。
如果光照较弱,电子总是可以看成简谐振子,从而电子自己发出的电磁辐射的频率和入射光的频率是完全一样的,电磁辐射的幅度正比于入射光的幅度,于是可以写出外场和总场之间的线性关系,从而得到所谓的\concept{线性光学}。
但是,如果光照非常强,非线性效应就会出现。强激光光源的出现让对这一领域的实验研究成为可能。
本节首先介绍强光如何能够产生。

如果我们能够有一个长期保持粒子数反转的系统(显然需要持续的能量输入),那么向这个系统入射一束光将会产生更强的出射光,因为会有受激发射,且受激发射出的光和入射光是非常相干的。
因此我们向粒子数反转的系统注入的能量可以用于增强入射光,并且如果入射光相干性非常好,那么出射光也保持非常好的相干性,并且比入射光更亮,这就是\concept{激光}。

最简单的方案——使用一个二能级系统,直接通过一次入射来得到激光——是不现实的,因为此时大量的能量会消耗在保持粒子数反转上,而为了 TODO
两个可能的改进:将粒子数反转的系统放在一个四壁强反射的腔体内,从而光束可以来回走,不断被增强,并且使用一个三能级系统,其中间能级是一个亚稳态,从而很容易制造粒子数反转。
可以在腔体的一个地方“开洞”——比如说让反射率稍微低一些——让一些光子泄露出来,激光就从这里被导出。

多光子过程:可以避免光学屏障,以及排除荧光本底

\section{非线性谐振子}

\begin{equation}
    m \dv[2]{x}{t} + m \gamma \dv{x}{t} + m \omega_0^2 x = q E,
\end{equation}
而

\end{document}