\documentclass[UTF8, a4paper]{ctexart}

\usepackage{geometry}
\usepackage{titling}
\usepackage{titlesec}
\usepackage{paralist}
\usepackage{footnote}
\usepackage{enumerate}
\usepackage{amsmath, amssymb, amsthm}
\usepackage{cite}
\usepackage{graphicx}
\usepackage{subfigure}
\usepackage{physics}
\usepackage{slashed}
\usepackage[colorlinks, linkcolor=black, anchorcolor=black, citecolor=black]{hyperref}
\usepackage{prettyref}

\geometry{left=3.28cm,right=3.28cm,top=2.54cm,bottom=2.54cm}
\titlespacing{\paragraph}{0pt}{1pt}{10pt}[20pt]
\setlength{\droptitle}{-5em}
\preauthor{\vspace{-10pt}\begin{center}}
\postauthor{\par\end{center}}

\newcommand*{\ee}{\mathrm{e}}
\newcommand*{\ii}{\mathrm{i}}
\newcommand*{\st}{\quad \text{s.t.} \quad}
\newcommand*{\const}{\mathrm{const}}
\newcommand*{\natnums}{\mathbb{N}}
\newcommand*{\reals}{\mathbb{R}}
\newcommand*{\complexes}{\mathbb{C}}
\DeclareMathOperator{\timeorder}{T}
\newcommand*{\ogroup}[1]{\mathrm{O}(#1)}
\newcommand*{\sogroup}[1]{\mathrm{SO}(#1)}
\DeclareMathOperator{\legpoly}{P}
\DeclareMathOperator{\diag}{diag}

\renewcommand{\emph}[1]{\textbf{#1}}
\newcommand*{\concept}[1]{\underline{\textbf{#1}}}

\newrefformat{sec}{第\ref{#1}节}
\newrefformat{note}{注\ref{#1}}
\newrefformat{fig}{图\ref{#1}}
\renewcommand{\autoref}{\prettyref}

\newenvironment{bigcase}{\left\{\quad\begin{aligned}}{\end{aligned}\right.}

\title{经典电动力学常见问题}
\author{吴何友}

\begin{document}

\maketitle

以下如无特殊说明,源点坐标用$\vb*{r}'$表示,场点坐标用$\vb*{r}$表示;从源点指向场点的矢量记作$\vb*{R} = \vb*{r} - \vb*{r}'$。规定$\vb*{r}_{12} = \vb*{r}_1 - \vb*{r}_2$,为从2指向1的矢量。

\section{麦克斯韦方程及其推论}

\subsection{不同单位制下的麦克斯韦方程}

自然单位制下,麦克斯韦方程为
\begin{equation}
    \left\{
        \begin{aligned}
            \div{\vb*{E}} &= \rho, \\
            \curl{\vb*{E}} &= - \pdv{\vb*{B}}{t}, \\
            \div{\vb*{B}} &= 0, \\
            \curl{\vb*{B}} &= \pdv{\vb*{E}}{t} + \vb*{j}.
        \end{aligned}
    \right.
    \label{eq:maxwell-eq}
\end{equation}
在\concept{国际单位制}下,我们有
\begin{equation}
    \left\{
        \begin{aligned}
            \div{\vb*{E}} &= \frac{\rho}{\epsilon_0}, \\
            \curl{\vb*{E}} &= - \pdv{\vb*{B}}{t}, \\
            \div{\vb*{B}} &= 0, \\
            \curl{\vb*{B}} &= \mu_0 \epsilon_0 \pdv{\vb*{E}}{t} + \mu_0 \vb*{j}.
        \end{aligned}
    \right.
    \label{eq:maxwell-eq-si}
\end{equation}
以下如无特殊说明,均使用国际单位制。

从以上方程可以推导出电荷输运方程
\begin{equation}
    \pdv{\rho}{t} + \div{\vb*{j}} = 0,
    \label{eq:charge-transportation}
\end{equation}
以及场满足的波动方程
\begin{equation}
    \frac{1}{c^2} \pdv[2]{\vb*{E}}{t} - \laplacian{\vb*{E}} = - \frac{1}{\epsilon_0} \grad{\rho} - \mu_0 \pdv{\vb*{j}}{t} , \quad \frac{1}{c^2} \pdv[2]{\vb*{B}}{t} - \laplacian{\vb*{B}} = \mu_0 \curl{\vb*{j}},
    \label{eq:wave-eq-general}
\end{equation}
其中
\begin{equation}
    \frac{1}{c^2} = \epsilon_0 \mu_0
\end{equation}
为\concept{光速},实际上就是电磁波传播的速度。
麦克斯韦方程本身没有给出电荷的动力学,但是它确实给出了一个守恒荷。

本文讨论的都是经典电动力学,在其中,电场和磁场既是物理量又标记了系统的状态,因此有时候它们看起来像量子理论中的算符,有时候看起来像某种波函数,这都是合理的。

\subsection{常见规范}

库伦规范、洛伦兹规范、辐射规范

库伦规范一般用于分析静磁学问题。

\subsection{能量和动量}

电磁场中的粒子的运动方程为
\begin{equation}
    \dv{\vb*{p}}{t} = q \vb*{E} + q \vb*{v} \times \vb*{B},
\end{equation}

一个区域内部所有粒子的能量由于电磁场而发生的变化为
\begin{equation}
    \dv{E}{t} = \sum q \vb*{v} \cdot \vb*{E} = \int \dd[3]{\vb*{r}} \vb*{j} \cdot \vb*{E},
\end{equation}
而
\[
    \begin{aligned}
        \vb*{j} \cdot \vb*{E} &= \left( \frac{1}{\mu_0} \curl{\vb*{B}} - \epsilon_0 \pdv{\vb*{E}}{t} \right) \cdot \vb*{E} \\
        &= \frac{1}{\mu_0} \div{(\vb*{B} \times \vb*{E})} - \frac{\epsilon_0}{2} \pdv{\vb*{E}^2}{t} + \frac{1}{\mu_0} (\curl{\vb*{E}}) \cdot \vb*{B} \\
        &= \frac{1}{\mu_0} \div{(\vb*{B} \times \vb*{E})} - \frac{\epsilon_0}{2} \pdv{\vb*{E}^2}{t} - \frac{1}{\mu_0} \pdv{\vb*{B}}{t} \cdot \vb*{B} \\
        &= \frac{1}{\mu_0} \div{(\vb*{B} \times \vb*{E})} - \pdv{t} \left( \frac{\epsilon_0}{2} \vb*{E}^2 + \frac{1}{2 \mu_0} \vb*{B}^2 \right),
    \end{aligned}
\]
因此
\[
    \dv{E}{t} = - \int \dd[3]{\vb*{r}} \left( \frac{\epsilon_0}{2} \vb*{E}^2 + \frac{1}{2 \mu_0} \vb*{B}^2 \right) - \frac{1}{\mu_0} \int \dd{\vb*{S}} \cdot (\vb*{E} \times \vb*{B}).
\]
设电磁场能量密度为$u$,能流密度为$\vb*{S}$,则
\[
    \dv{E}{t} + \int \dd[3]{\vb*{r}} u = - \int \dd{\vb*{S}} \cdot \vb*{S},
\]
于是
\begin{equation}
    \int \dd[3]{\vb*{r}} u + \int \dd{\vb*{S}} \cdot \vb*{S} = \int \dd[3]{\vb*{r}} \left( \frac{1}{2} \epsilon_0 \vb*{E}^2 + \frac{1}{2\mu_0} \vb*{B}^2 \right) + \int \dd{\vb*{S}} \cdot \frac{1}{\mu_0} \vb*{E} \times \vb*{B},
    \label{eq:energy-flow-original}
\end{equation}
从而可以取
\begin{equation}
    u = \frac{1}{2} \epsilon_0 \vb*{E}^2 + \frac{1}{2\mu_0} \vb*{B}^2, \quad \vb*{S} = \frac{1}{\mu_0} \vb*{E} \times \vb*{B}.
    \label{eq:energy-flow}
\end{equation}
当然,实际上从\eqref{eq:energy-flow-original}不能唯一地确定能量密度和能流,因为在变换
\[
    \rho' = \rho + \div{\vb*{c}}, \quad \vb*{j}' = \vb*{j} - \pdv{\vb*{c}}{t}
\]
下输运方程保持成立。这也是可以预期的,因为可以看出\eqref{eq:energy-flow-original}是麦克斯韦方程能够给出的关于能量的全部结论,而通常从运动学方程出发并不能唯一地定义能量。
使用\eqref{eq:energy-flow}定义的$\vb*{S}$称为\concept{坡印廷矢量}。

使用类似的方法,设$\vb*{P}$为一个区域内的粒子总能量,则
\[
    \dv{\vb*{P}}{t} = \sum (q \vb*{E} + q \vb*{v} \times \vb*{B}) = \int \dd[3]{\vb*{r}} \left( \rho \vb*{E} + \vb*{j} \times \vb*{B} \right),
\]
代入$\rho$和$\vb*{j}$,得到
\[
    \begin{aligned}
        \dv{\vb*{P}}{t} &= \int \dd[3]{\vb*{r}} \left(\epsilon_0 (\div{\vb*{E}}) \vb*{E} + \left( \frac{1}{\mu_0} \curl{\vb*{B}} - \epsilon_0 \pdv{\vb*{E}}{t} \right) \times \vb*{B} \right) \\
        &= \int \dd[3]{\vb*{r}} \epsilon_0 (\div{\vb*{E}}) \vb*{E} + \int \dd[3]{\vb*{r}} \frac{1}{\mu_0} (\curl{\vb*{B}}) \times \vb*{B} - \epsilon_0 \int \dd[3]{\vb*{r}} \pdv{t} (\vb*{E} \times \vb*{B}) + \epsilon_0 \int \dd[3]{\vb*{r}} \vb*{E} \times \pdv{\vb*{B}}{t} \\
        &= \epsilon_0 \int \dd[3]{\vb*{r}} ((\div{\vb*{E}}) \vb*{E} + (\curl{\vb*{E}}) \times \vb*{E}) + \frac{1}{\mu_0} \int \dd[3]{\vb*{r}} (\curl{\vb*{B}}) \times \vb*{B} - \epsilon_0 \int \dd[3]{\vb*{r}} \pdv{t} (\vb*{E} \times \vb*{B}),
    \end{aligned}
\]
而注意到
\[
    \begin{aligned}
        (\div{\vb*{E}}) \vb*{E} + (\curl{\vb*{E}}) \times \vb*{E}) &= \div{(\vb*{E} \vb*{E})} - (\vb*{E} \cdot \grad) \vb*{E} - \vb*{E} \times (\curl{\vb*{E}}) \\
        &= \div{(\vb*{E} \vb*{E})} - \frac{1}{2} \grad{\vb*{E}^2} \\
        &= \div{(\vb*{E} \vb*{E})} - \frac{1}{2} \div{(\vb*{E}^2 \vb*{I})},
    \end{aligned}
\]
且类似的可以得到
\[
    \begin{aligned}
        \underbrace{(\div{\vb*{B}}) \vb*{B}}_{=0} + (\curl{\vb*{B}}) \times \vb*{B}) &= \div{(\vb*{B} \vb*{B})} - (\vb*{B} \cdot \grad) \vb*{B} - \vb*{B} \times (\curl{\vb*{B}}) \\
        &= \div{(\vb*{B} \vb*{B})} - \frac{1}{2} \grad{\vb*{B}^2} \\
        &= \div{(\vb*{B} \vb*{B})} - \frac{1}{2} \div{(\vb*{B}^2 \vb*{I})},
    \end{aligned}
\]
于是就有
\[
    \begin{aligned}
        \dv{\vb*{P}}{t} &= - \epsilon_0 \int \dd[3]{\vb*{r}} \pdv{t} (\vb*{E} \times \vb*{B}) - \int \dd{\vb*{S}} \cdot \left( \frac{1}{2} \left( \epsilon_0 \vb*{E}^2 + \frac{1}{\mu_0} \vb*{B}^2 \right) \vb*{I} - \epsilon_0 \vb*{E} \vb*{E} - \frac{1}{\mu_0} \vb*{B} \vb*{B} \right) \\
        &= - \int \dd[3]{\vb*{r}} \vb*{g} - \int \dd{S_i} T_{ij},
    \end{aligned}
\]
其中$\vb*{g}$是动量密度而$T_{ij}$是动量流密度(一个二阶张量)。同样,只是知道上式不能够唯一确定动量密度和动量流密度,但是以下的选择是最简单的:
\begin{equation}
    T_{ij} = u \delta_{ij} - \epsilon_0 E_i E_j - \frac{1}{\mu_0} B_i B_j, \quad \vb*{g} = \frac{1}{c^2} \vb*{S}.
\end{equation}
张量$T_{ij}$称为\concept{麦克斯韦张量},它可以看成电磁场提供的应力。
电磁场对实物粒子的动量转移速率,也即力,就是
\begin{equation}
    \vb*{F} = - \int \dd[3]{\vb*{r}} \vb*{g} - \int \dd{S_i} T_{ij}.
\end{equation}

\subsection{非相对论性粒子}

直接通过$U(1)$规范对称性可以得到
\begin{equation}
    \vb*{p} = m \vb*{v} + q \vb*{A},
\end{equation}
其中$\vb*{p}$称为\concept{正则动量}。
这个式子成立的条件是粒子本身低速运动,从而粒子产生的磁场可以忽略,且外场不变。



\section{真空中的解}

\subsection{稳恒问题}

\subsubsection{静电学}

首先考虑\concept{静电}的情况。所谓“静电”指的是系统中没有任何电流的情况,此时我们有$\div{\vb*{j}}=0$,从而电荷密度分布没有变化。
注意我们没有使用“电荷密度没有变化”作为定义,因为恒稳电路也具有这样的性质,但是与此同时的确有电荷流动。
在静电的条件下我们有
\[
    \curl{\vb*{E}} = - \pdv{\vb*{B}}{t}, \quad \curl{\vb*{B}} = \mu_0 \epsilon_0 \pdv{\vb*{E}}{t},
\]
然后我们会发现磁场$\vb*{B}$满足一个波动方程。看起来这非常奇怪,因为在根本没有电流的时候磁场怎么会存在呢?
这实际上是来自边界条件的不清晰。在处理电磁波时我们并不要求无穷远处场强衰减至零,因为我们实际上是认为电磁波由非常远的地方的一个源产生的,没完没了地传播到其它地方,从而处理电磁波时“无穷远处”有源是完全可以的。
在静电学中我们要求电荷约束在一个有限的范围内,从而无穷远处场强快速衰减,那么磁场满足的波动方程如果要有非平凡解,只能取类似球面波的形式,但是这样一来$\div{\vb*{B}}$会不为零,和磁场无源的条件违背。
总之,在静电学情况下$\vb*{B}=0$。于是我们就有静电学方程
\begin{equation}
    \div{\vb*{E}} = \frac{\rho}{\epsilon_0}, \quad \curl{\vb*{E}} = 0,
    \label{eq:static-e-field}
\end{equation}
或者
\begin{equation}
    \vb*{E} = - \grad{\varphi}, \quad \laplacian{\varphi} = - \frac{\rho_0}{\epsilon_0}.
    \label{eq:static-phi-field}
\end{equation}
这是拉普拉斯方程,有现成的解法,即
\begin{equation}
    \varphi(\vb*{r}) = \int \dd[3]{\vb*{r}'} \frac{1}{4\pi \epsilon_0} \frac{\rho(\vb*{r}')}{\abs*{\vb*{r} - \vb*{r}'}}.
    \label{eq:from-q-to-phi}
\end{equation}

静电学中能量可以看成是电荷携带的而不是电场携带的。这是因为一个区域内的能量为
\[
    E = \int \dd[3]{\vb*{r}} \frac{1}{2} \epsilon_0 \vb*{E}^2 = \frac{1}{2} \epsilon_0 \int \dd[3]{\vb*{r}} (\grad{\varphi})^2,
\]
做分部积分,并使用无穷远处场强为零这一条件,得到
\begin{equation}
    E = - \frac{\epsilon_0}{2} \int \dd[3]{\vb*{r}} \varphi \laplacian{\varphi} = \int \dd[3]{\vb*{r}} \rho \varphi.
\end{equation}
因此在静电学的情况下能量可以认为是定域在电荷周围的。

\subsubsection{静磁学}

和静电学类似,我们可以考虑恒定电流的情况,即虽然有电流但是没有任何电荷变化,电流强度也不变的情况,则由输运方程有$\div{\vb*{j}}=0$。
我们可以直接引用\eqref{eq:wave-eq-general},得到
\[
    \frac{1}{c^2} \pdv[2]{\vb*{E}}{t} - \laplacian{\vb*{E}} = - \frac{1}{\epsilon_0} \grad{\rho} , \quad \frac{1}{c^2} \pdv[2]{\vb*{B}}{t} - \laplacian{\vb*{B}} = \mu_0 \curl{\vb*{j}},
\]
由于$\rho$和$\vb*{j}$都不随时间变化,如果我们像在静电学中一样,要求无穷远处场强衰减足够快,那么以上两式可以直接化为静态的拉普拉斯方程
\[
    \laplacian{\vb*{E}} = \frac{1}{\epsilon_0} \grad{\rho}, \quad \laplacian{\vb*{B}} = - \mu_0 \curl{\vb*{j}}.
\]
由于$\vb*{B}$不会变化,我们直接得到\eqref{eq:static-e-field},于是就可以求解出电场。
至于$\vb*{B}$,引入磁矢势,就得到
\[
    \curl{\laplacian{\vb*{B}}} = - \mu_0 \curl{\vb*{j}},
\]
那么只需要取库伦规范$\div{A}=0$就可以有
\[
    \div{\laplacian{\vb*{B}}} = - \mu_0 \div{\vb*{j}}
\],
于是就得到
\begin{equation}
    \vb*{B} = \curl{\vb*{A}}, \quad \laplacian{\vb*{A}} = - \mu_0 \vb*{j}.
\end{equation}
这个方程的形式和\eqref{eq:static-phi-field}非常相似,也是拉普拉斯方程,从而直接可以写出
\begin{equation}
    \vb*{A}(\vb*{r}) = \int \dd[3]{\vb*{r}'} \frac{\mu_0}{4\pi} \frac{\vb*{j}(\vb*{r}')}{\abs*{\vb*{r} - \vb*{r}'}}.
    \label{eq:from-j-to-a}
\end{equation}

总之,在电荷分布不变、电流分布不变的情况下,电场可以用$\rho$表示出来,并且是无旋场;磁场可以用$\vb*{j}$表示出来。这分别称为静电学和静磁学。

\subsection{运动电荷和辐射}

\subsubsection{李纳-维谢尔势}

本节求解空间中有单个运动电荷时的电势和矢势分布情况,所得结果称为\concept{李纳-维谢尔势}。
在取洛伦兹规范之后,我们需要求解
\begin{equation}
    \begin{bigcase}
        \laplacian{\varphi} - \frac{1}{c^2} \pdv[2]{\varphi}{t} &= - \frac{\rho(\vb*{r})}{\epsilon_0}, \\
        \laplacian{\vb*{A}} - \frac{1}{c^2} \pdv[2]{\vb*{A}}{t} &= - \mu_0 \vb*{j}(\vb*{r}).
    \end{bigcase}
\end{equation}
使用格林函数法,有(暂时先不引入无穷小虚部)
\[
    \begin{aligned}
        \varphi(\vb*{r}, t) &= \int \frac{\dd{\omega}}{2\pi} \int \frac{\dd[3]{\vb*{k}}}{(2\pi)^3} \frac{\ee^{\ii (\vb*{k} \cdot \vb*{r} - \omega t)}}{- \vb*{k}^2 + \omega^2/c^2} \left( - \frac{\rho(\vb*{k}, \omega)}{\epsilon_0} \right) \\
        &= \frac{1}{\epsilon_0} \int \frac{\dd{\omega}}{2\pi} \int \frac{\dd[3]{\vb*{k}}}{(2\pi)^3} \frac{\ee^{\ii (\vb*{k} \cdot \vb*{r} - \omega t)}}{\vb*{k}^2 - \omega^2/c^2} \int \dd[3]{\vb*{r}'} \int \dd{t'} \ee^{\ii (\omega t' - \vb*{k} \cdot \vb*{r}')} \rho(\vb*{r}', t') \\
        &= \frac{1}{\epsilon_0} \int \dd[3]{\vb*{r}'} \int \dd{t'} \rho(\vb*{r}', t') \int \frac{\dd{\omega}}{2\pi} \int \frac{\dd[3]{\vb*{k}}}{(2\pi)^3} \frac{\ee^{\ii (\vb*{k} \cdot (\vb*{r} - \vb*{r}') - \omega (t - t'))}}{\vb*{k}^2 - \omega^2/c^2}.
    \end{aligned}
\]
首先计算$\vb*{k}$部分的积分,有
\[
    \begin{aligned}
        \int \dd[3]{\vb*{k}} \frac{\ee^{\ii \vb*{k} \cdot \vb*{R}}}{\vb*{k}^2 - \omega^2/c^2} &= \int k^2 \sin \theta \dd{k} \dd{\theta} \dd{\varphi} \frac{\ee^{\ii k R \cos \theta}}{k^2 - \omega^2 / c^2} \\
        &= 2\pi \int_0^\infty \frac{k^2 \dd{k}}{k^2 - \omega^2 / c^2} \frac{1}{\ii k R} (\ee^{\ii k R} - \ee^{-\ii k R}) \\
        &= \frac{\pi}{\ii R} \int_{-\infty}^\infty \frac{k \dd{k}}{k^2 - \omega^2 / c^2} (\ee^{\ii k R} - \ee^{-\ii k R}) \\
        &= \frac{\pi}{2 \ii R} \int_{-\infty}^\infty \dd{k} \left( \frac{1}{k + \omega / c} + \frac{1}{k - \omega / c} \right) (\ee^{\ii k R} - \ee^{-\ii k R}).
    \end{aligned}
\]
此时必须在分母上加入无穷小虚部。按照关于$\omega$的零点必须在下半平面以保证因果性的原则,我们将$\omega$替换为$\omega + \ii 0^+$,并使用留数定理(注意$\ee^{\ii k R}$项应取上半平面极点而$\ee^{- \ii k R}$项应取下半平面极点)就得到
\[
    \int \dd[3]{\vb*{k}} \frac{\ee^{\ii \vb*{k} \cdot \vb*{R}}}{\vb*{k}^2 - \omega^2/c^2} = \frac{2 \pi^2}{R} \ee^{\ii \omega R / c}, 
\]
于是
\begin{equation}
    \begin{aligned}
        \varphi(\vb*{r}, t) &= \int \dd[3]{\vb*{r}'} \int \dd{t'} \int \frac{\dd{\omega}}{2\pi} \ee^{-\ii \omega (t-t')} \rho(\vb*{r}', t') \frac{1}{4\pi \epsilon_0} \frac{\ee^{\ii \omega R / c}}{R} \\
        &= \int \dd[3]{\vb*{r}'} \int \frac{\dd{\omega}}{2\pi} \ee^{- \ii \omega t} \rho(\vb*{r}', \omega) \frac{1}{4\pi \epsilon_0} \frac{\ee^{\ii \omega R / c}}{R}.
    \end{aligned}
\end{equation}
这个结果展示了一个出射波:从$\rho(\vb*{r}', t')$出发的向外传播的球面波,而不是向内聚集的波。
现在我们再做关于$\omega$的积分,会直接得到一个$\delta$函数:
\[
    \begin{aligned}
        \varphi(\vb*{r}, t) &= \int \dd[3]{\vb*{r}'} \int \dd{t'} \int \frac{\dd{\omega}}{2\pi} \ee^{-\ii \omega (t-t')} \rho(\vb*{r}', t') \frac{1}{4\pi \epsilon_0} \frac{\ee^{\ii \omega R / c}}{R} \\
        &= \int \dd[3]{\vb*{r}'} \int \dd{t'} \delta(R/c + t' - t) \rho(\vb*{r}', t') \frac{1}{4\pi \epsilon_0 R} \\
        &= \int \dd[3]{\vb*{r}'} \frac{1}{4\pi \epsilon_0} \frac{\rho(\vb*{r}', t - R / c)}{R}.
    \end{aligned}
\]
同样的操作也可以对$\vb*{A}$和$\vb*{j}$做,最终得到
\begin{equation}
    \begin{bigcase}
        \varphi(\vb*{r}, t) &= \int \dd[3]{\vb*{r}'} \frac{1}{4\pi \epsilon_0} \frac{\rho(\vb*{r}', t - R / c)}{R}, \\
        \vb*{A}(\vb*{r}, t) &= \int \dd[3]{\vb*{r}'} \frac{\mu_0}{4\pi} \frac{\vb*{j}(\vb*{r}', t - R / c)}{R}.
    \end{bigcase}
    \label{eq:general-solution-wave}
\end{equation}

当空间中只有一个电荷时,有
\[
    \rho(\vb*{r}, t) = q \delta(\vb*{r} - \vb*{r}_0(t)), \quad \vb*{j}(\vb*{r}, t) = q \dot{\vb*{r}}_0(t) \delta(\vb*{r} - \vb*{r}_0(t)),
\]
其中$\vb*{r}_0 = \vb*{r}_0(t)$是该电荷的运动轨迹。代入\eqref{eq:general-solution-wave},有
\[
    \varphi(\vb*{r}, t) = \int \dd[3]{\vb*{r}'} \frac{1}{4\pi \epsilon_0} \frac{q \delta(\vb*{r}' - \vb*{r}_0(t - R / c))}{R},
\]
因此只有满足
\begin{equation}
    \vb*{r}' = \vb*{r}_0(t - R/c)
    \label{eq:retarded-position-original}
\end{equation}
的部分才有贡献。但是要注意,$\vb*{r}'$同时也出现在$R$中,因此积分时不能仅仅将$\vb*{r}'$替换为$\vb*{r}_0(t-R/c)$,还需要做一个积分测度的变换。
我们有
\[
    \grad_{\vb*{r}'} {(\vb*{r}' - \vb*{r}_0(t - R/c))} = \vb*{I} - \frac{\vb*{R}}{cR} \dot{\vb*{r}_0}(t-R/c) ,
\]
于是
\[
    \det(\grad_{\vb*{r}'} {(\vb*{r}' - \vb*{r}_0(t - R/c))}) = 1 - \frac{\vb*{R}}{cR} \cdot \dot{\vb*{r}}_0(t-R/c),
\]
从而
\[
    \begin{aligned}
        \varphi(\vb*{r}, t) &= \int \dd[3]{\vb*{r}'} \frac{1}{4\pi \epsilon_0} \frac{q \delta(\vb*{r}' - \vb*{r}_0(t - R / c))}{R} \\
        &= \frac{1}{4\pi \epsilon_0} \eval{\frac{1}{\det(\grad_{\vb*{r}'} {(\vb*{r}' - \vb*{r}_0(t - R/c))})} \frac{q}{R}}_{\vb*{r}' = \vb*{r}_0(t - R/c)} \\
        &= \frac{1}{4\pi \epsilon_0} \eval{\frac{q}{R - \frac{\vb*{R} \cdot \dot{\vb*{r}}_0(t-R/c)}{c}}}_{\vb*{r}' = \vb*{r}_0(t - R/c)}.
    \end{aligned}
\]
用$\vb*{v}$表示粒子的运动速度,就有
\begin{equation}
    \varphi(\vb*{r}, t) = \frac{1}{4\pi \epsilon_0} \frac{q}{R' - \frac{\vb*{R}' \cdot \vb*{v}'}{c}},
    \label{eq:retarded-phi}
\end{equation}
类似的
\begin{equation}
    \vb*{A}(\vb*{r}, t) = \frac{\mu_0}{4\pi} \frac{q \vb*{v}'}{R' - \frac{\vb*{R}' \cdot \vb*{v}'}{c}},
    \label{eq:retarded-a}
\end{equation}
其中$R'$和$\vb*{v}'$均为$t'$时刻的$R$和$\vb*{v}$而$t'$由
\begin{equation}
    R(t') = \abs*{\vb*{r} - \vb*{r}_0(t')} = c(t-t')
    \label{eq:retarded-time}
\end{equation}
确定。这个方程看起来非常合理,我们将$\rho$有速度地出现在某个地方当成一个事件,它传递到$\vb*{r}$必然存在时间延迟,事件传播的速度就是光速,在$t$时刻,$\vb*{r}$点看到的$\vb*{r}_0$处的情况是$t'$时刻的,两者之差为
\[
    t - t' = \frac{\abs*{\vb*{r} - \vb*{r}_0(t')}}{c},
\]
就得到\eqref{eq:retarded-time}。

前面$\delta$函数的积分改变了积分测度,让它比通常的要大一些。这看起来似乎有些奇怪,因为狭义相对论中似乎应该有尺缩效应,积分测度应该缩小。
这里的关键在于运动电荷对某一点的电场的贡献涉及的空间积分应该体现的是“在这一点看到的运动物体的长度”(在静止参考系看到的物体两端传来的信号可能来自不同时刻)而不是“试图在静止参考系中测量得到的运动物体的长度”(测量时物体两端到达观察点的用时是一样的)。

\subsubsection{运动电荷的电场和磁场}



\subsection{电磁波}

\section{介质和边界条件}

\subsection{介质作用}

\eqref{eq:maxwell-eq-si}常常被称为“真空中的麦克斯韦方程”,但是实际上它当然是什么地方都适用的。
介质起作用的方式是,其内部已经有一个电荷分布,当外加电场的时候电荷重新排列、发生运动,在此过程中产生额外的电流、电场、磁场。
于是假定电荷和电流可以做以下分解:
\[
    \left\{
        \begin{aligned}
            &\vb*{j} = \vb*{j}_\text{f} + \vb*{j}_\text{r}, \quad \rho = \rho_\text{f} + \rho_\text{r}, \\
            &\pdv{\rho_\text{f}}{t} + \div{\vb*{j}_\text{f}} = 0, \\
            &\pdv{\rho_\text{r}}{t} + \div{\vb*{j}_\text{r}} = 0
        \end{aligned}
    \right.
\]
其中$\vb*{j}_\text{f}$是所谓的自由电流或者说传导电流,而$\vb*{j}_\text{r}$是介质的响应。但是这种二分法实际上很大程度上是任意的。
例如,金属能导电,因为其内部含有大量几乎是自由的电子——那么,外加电场产生的金属中的电流就应该是自由电流了;
但是分析金属的光学属性的时候,这些由于外加电场产生的电流又无疑是介质的响应。
因此$\vb*{j}_\text{f}$和$\vb*{j}_\text{r}$只是辅助量,没有特殊的物理含义。
为了能够将$\vb*{j}_\text{f}$和$\vb*{j}_\text{r}$整合进两个形式上和电场和磁感应强度很像的辅助量,
从而在形式上让\eqref{eq:maxwell-eq-si}变成一个只和自由电荷和自由电流有关的方程组,我们进一步做下面的分解:
\[
    \vb*{j}_\text{r} = \vb*{j}_\text{s} + \vb*{j}_\text{c}
\]
且$\vb*{j}_\text{c}$是一个有旋无源场。光有这个条件不足以在给定$\vb*{j}_\text{r}$时唯一地确定下$\vb*{j}_\text{s}$和$\vb*{j}_\text{c}$,
因此还可以引入一个假设而不至于让$\vb*{j}_\text{s}$和$\vb*{j}_\text{c}$无解。
为了让\eqref{eq:maxwell-eq-si}中第一式的右边只剩下自由电荷,做以下假定,引入\concept{极化强度}$\vb*{P}$:
\[
    \rho_\textbf{r} = - \div{\vb*{P}}
\]
这个假设\emph{没有}缩小$\vb*{j}_\text{s}$和$\vb*{j}_\text{c}$的选择范围,因为任意给定性质足够良好的$\rho_\text{r}$,相对应的$\vb*{P}$总是存在的(而且显然不唯一)。
同时由于$\vb*{j}_\text{c}$是一个有旋无源场,可以再引进一个辅助量$\vb*{M}$(称为\concept{磁化强度})使
\[
    \vb*{j}_\text{c} = \curl{\vb*{M}}
\]
此时$\rho_\text{r}$的输运方程成为
\[
    \pdv{\rho_\text{r}}{t} + \div{\vb*{j}_\text{s}} = 0
\]
因为$\curl{\vb*{j}_\text{c}}$的散度为零。这个式子又可以写成
\[
    \div{\left(\vb*{j}_\text{s}-\pdv{\vb*{P}}{t}\right)} = 0
\]
受到这个式子的启发,我们\emph{假设}(不是推出,因为光有上式不能定解,而先前我们只对$\vb*{j}_\text{c}$做过假设而没有对$\vb*{j}_\text{s}$做过假设,因此后者的取值仍然是任意的)有
\[
    \vb*{j}_\text{s} = \pdv{\vb*{P}}{t}
\]
这个假设不会让$\vb*{j}_\text{s}$和$\vb*{j}_\text{c}$无解。

将以上引入的所有物理量代入\eqref{eq:maxwell-eq-si},得到
\[
    \begin{bigcase}
        \epsilon_0 \div{\vb*{E}} &= \rho_\text{f} - \div{\vb*{P}}, \\
        \curl{\vb*{E}} &= - \pdv{\vb*{B}}{t}, \\
        \div{\vb*{B}} &= 0, \\
        \curl{\frac{\vb*{B}}{\mu_0}} &= \vb*{j}_\text{f} + \curl{\vb*{M}} + \pdv{\vb*{P}}{t} + \epsilon_0 \pdv{\vb*{E}}{t}
    \end{bigcase}
\]
引入辅助量$\vb*{D}$和$\vb*{H}$(分别称为\concept{电位移矢量}和\concept{磁场强度})
\begin{equation}
    \vb*{D} = \epsilon_0 \vb*{E} + \vb*{P}, \quad \vb*{H} = \frac{\vb*{B}}{\mu_0} - \vb*{M},
\end{equation}
就得到了
\begin{equation}
    \begin{bigcase}
        \div{\vb*{D}} &= \rho_\text{f}, \\
        \curl{\vb*{E}} &= - \pdv{\vb*{B}}{t}, \\
        \div{\vb*{B}} &= 0, \\
        \curl{\vb*{H}} &= \vb*{j}_\text{f} + \pdv{\vb*{D}}{t}
    \end{bigcase}
    \label{eq:maxwell-material}
\end{equation}
这就是\concept{介质中的麦克斯韦方程}。

方程组\eqref{eq:maxwell-material}除去了\eqref{eq:maxwell-eq-si}中由于介质产生的电荷密度和电流密度,形式上更加简洁,
但是即使在自由电荷密度和电流密度已经给定的情况下,只靠\eqref{eq:maxwell-material}本身也没有办法定解,因为未知数太多了。
考虑到从$\vb*{E}, \vb*{B}$到$\vb*{D}, \vb*{H}$的变换是线性的,这就意味着\eqref{eq:maxwell-eq-si}在自由电荷密度和电流密度已经给定的情况下其实也不能定解。
这是理所当然的,因为到现在为止我们没有真的描述介质如何响应自由电荷。

下面的问题是,在自由电荷密度和电流密度已经给定的情况下,增加什么方程能够让\eqref{eq:maxwell-material}定解?
当然,只要知道了从$\vb*{E}, \vb*{B}$到$\vb*{D}, \vb*{H}$的变换的具体计算式(而不是显含$\vb*{j}_\text{r}$的定义式)
就能够定解。
更进一步,在什么都不知道,只有初始条件和边界条件的情况下,怎样能够让\eqref{eq:maxwell-material}定解?
只需要增补$\vb*{j}_\text{f}$和$\vb*{E}$的显式关系,以及输运方程
\begin{equation}
    \pdv{\rho_\text{f}}{t} + \div{\vb*{j}_\text{f}} = 0
    \label{eq:transportation}
\end{equation}
就能够定解。

因此要求解出介质中的电磁场变化情况,首先需要\emph{物理方程}\eqref{eq:maxwell-material},
然后是\emph{本构关系}也就是$\vb*{D}$,$\vb*{H}$,$\vb*{j}_\text{f}$关于其他量的表达式,最后是\emph{几何关系}\eqref{eq:transportation},
再加上适当的边界条件和初始条件,就能够定解。

关于本构关系实际上有一个问题,就是从$\vb*{E}$,$\vb*{B}$,$\vb*{j}_\text{f}$到$\vb*{D}$和$\vb*{H}$是不是真的有一个函数关系。
如果相同的$\vb*{E}$,$\vb*{B}$,$\vb*{j}_\text{f}$实际上对应着不同的系统状态,那就糟糕了。
但是在经典电动力学中$\vb*{E}$,$\vb*{B}$是仅有的场,而如果对$\vb*{j}_\text{s}$和$\vb*{j}_\text{c}$加上足够的限制,总是可以使用$\vb*{j}_\text{f}$确定下整个$\vb*{j}$的分布,从而$\rho$的分布,因此$\vb*{E}$,$\vb*{B}$,$\vb*{j}_\text{f}$能够完全确定系统状态,从而本构关系总是可以写出来的。

\subsection{线性介质}

如果$\vb*{D}$和$\vb*{E}$、$\vb*{H}$和$\vb*{B}$之间的关系是线性的,那么这样的介质就是\concept{线性介质}。
最一般的关系是
\begin{equation}
    \vb*{D}(\vb*{r}, t) = \int \dd[3]{\vb*{r}} \dd{t} \vb*{\epsilon}(\vb*{r} - \vb*{r}', t - t') \vb*{E}(\vb*{r}', t'), \quad
    \vb*{B}(\vb*{r}, t) = \int \dd[3]{\vb*{r}} \dd{t} \vb*{\mu}(\vb*{r} - \vb*{r}', t - t') \vb*{H}(\vb*{r}', t').
\end{equation}
当然,总是可以做傅里叶变换将上式切换到频域空间。

真空是一种典型的线性介质。

大部分介质都是各向同性的,且没有记忆,于是
\begin{equation}
    \vb*{D} = \epsilon \vb*{E}, \quad \vb*{B} = \mu \vb*{B}.
\end{equation}
$\epsilon$和$\mu$这两个参数仍然可以在空间中发生变化。

$\vb*{j}_\text{f}$和$\vb*{E}$之间的关系如果是线性的,我们说介质中有\concept{欧姆定律}。

对满足欧姆定律的均匀系统,我们有
\[
    \frac{1}{\mu} \curl{\vb*{B}} = \sigma \vb*{E} + \epsilon \pdv{\vb*{E}}{t},
\]
在上式两边作用散度算符,就得到
\begin{equation}
    \pdv{\rho}{t} + \frac{\sigma}{\epsilon} \rho = 0.
\end{equation}
因此导体内部电荷快速衰减。因此,只有在不均匀的地方——如边界上——才能够积累电荷。

\subsection{介质中的电磁场能量和动量}

\subsection{边界条件}

当两个介质被贴在一起时,实际上出现了三层介质结构:两层原来的介质,一层过渡层。
通常我们并不在乎过渡层内部的细节,或者说将过渡层做粗粒化,那么两层介质两边就可能有一些物理量不连续。
根据麦克斯韦方程可以写出
\begin{equation}
    \vb*{n} \cdot 
\end{equation}

在边界条件中含有导数算符时可能会产生一些歧义,例如,如果要将$\vb*{n}$移动到边界条件
\[
    \vb*{n} \cdot (\curl{\vb*{A}_1}) = \vb*{n} \cdot (\curl{\vb*{A}_2})
\]
中的导数算符后面,那么导数算符会作用在(在曲面上有空间变化的)$\vb*{n}$上吗?
好像还真的要作用上去……

\section{静电学}

\subsection{静电系统的基本方程}

本节讨论仅含有导体和线性电介质的静电系统。此时我们有

\begin{equation}
    \varphi|_\text{surface} = \const,
\end{equation}
\begin{equation}
    \pdv{\varphi}{\vb*{n}} = - \frac{\sigma}{\epsilon}
\end{equation}

真空中或者均匀线性电介质中不能有电势极大值或者极小值,因为在这样的区域内$\varphi$是调和函数,而调和函数在它调和的区域内部不能有极大值、极小值。
物理上这很好理解,如极大值出现意味着从这一点向它周围的各个方向都有电场,因此这一点上应该有电荷,矛盾。

在计算静电系统中导体的受力时,不能简单地将无导体的空间内的电场外推到导体表面,然后使用$\vb*{f}=\sigma\vb*{E}$,因为导体表面电场是不连续的。
更加物理地看,这是因为导体表面实际上是非常复杂的一个系统:电场在微观层面快速衰减,表面上的电荷之间有相互作用力,数量级估计可以发现这些电荷之间的相互作用力和电荷受到的电场力是同阶的,因此简单的$\vb*{f}=\sigma\vb*{E}$会漏掉一部分作用力。
最为可靠的方法是使用麦克斯韦张量来计算,因为动量守恒是总是成立的,则在静电情况下动量流是连续的,所以直接计算导体外的麦克斯韦张量然后外推到导体表面即可。%
\footnote{
    这里还有一个可能的疑难:麦克斯韦张量计算的是电场对自由电荷的作用力,但是首先导体上的电荷并不是自由的,其次我们要计算的也是导体受到的作用力。
    但是,电磁场本身对导体并没有任何作用,而由受力平衡,导体对电荷施加的作用力应该和电场对电荷施加的作用力平衡,于是电场对电荷的作用力就传递给了导体。
}%

\subsection{唯一性定理}

唯一性定理成立的条件是$\vb*{D}$和$\vb*{E}$之间的关系应该是一一对应的。
反之,在两者之间的关系实际上不一一对应的时候,唯一性定理就被破坏了。例如,如果$\vb*{D}-\vb*{E}$关系实际上构成了一条电滞回线,那就没有唯一性定理。

% TODO: 有限大小的体系内电荷总量应该为零:这是高斯定理的推论

\[
    \int \dd[3]{\vb*{r}} (\varphi_2 \laplacian{\varphi_1} - \varphi_1 \laplacian{\varphi_2}) = \int \dd{\vb*{S}} (\varphi_2 \grad{\varphi_1} - \varphi_1 \grad{\varphi_2}),
\]
右边是
\begin{equation}
    \int \dd[3]{\vb*{r}} \varphi_1 \rho_2 = \int \dd[3]{\vb*{r}} \varphi_2 \rho_1.
\end{equation}
在导体系统中电荷仅仅分布在导体表面上,而且同一个导体表面电势处处相同,于是
\begin{equation}
    \sum_i \varphi_i^{(1)} q_i^{(2)} = \sum_i \varphi_i^{(2)} q_i^{(1)}.
\end{equation}

\subsection{电多极子}

设空间中的电荷密度为$\rho$,可能还要算上面密度,电势为
\[
    \varphi(\vb*{r}) = \frac{1}{4\pi \epsilon_0} \int \dd[3]{\vb*{r}'} \frac{1}{\abs*{\vb*{r} - \vb*{r}'}} \rho(\vb*{r}'),
\]
对$1/\abs*{\vb*{r}-\vb*{r}'}$做多级展开:
\[
    \frac{1}{\abs*{\vb*{r}-\vb*{r}'}} = \frac{1}{\abs*{\vb*{r}}} - \vb*{r}' \cdot \grad{\frac{1}{\abs*{\vb*{r}}}} + \frac{1}{2} \vb*{r}' \vb*{r}' : \grad{\grad{\frac{1}{\abs*{\vb*{r}}}}} + \cdots,
\]
就得到一个$\varphi$的展开式,即所谓\concept{多级展开},其中
\begin{equation}
    \varphi^{(0)}(\vb*{r}) = \frac{1}{4\pi \epsilon_0} \frac{1}{\abs*{\vb*{r}}} \underbrace{\int \dd[3]{\vb*{r}'} \rho(\vb*{r}')}_{Q}
\end{equation}
就是将整个体系当成一个点电荷计算得到的电势,
\begin{equation}
    \begin{aligned}
        \varphi^{(1)}(\vb*{r}) &= - \frac{1}{4\pi \epsilon_0} \grad{\frac{1}{\abs*{\vb*{r}}}} \cdot \int \dd[3]{\vb*{r}'} \rho(\vb*{r}') \vb*{r}' \\
        &= \frac{1}{4\pi \epsilon_0} \frac{\vb*{r}}{\abs*{\vb*{r}}^3} \cdot \underbrace{\int \dd[3]{\vb*{r}'} \rho(\vb*{r}') \vb*{r}'}_{\vb*{p}}
    \end{aligned}
\end{equation}
是电偶极电势,电四极矩是
\begin{equation}
    \varphi^{(2)}(\vb*{r}) = \frac{1}{4 \pi \epsilon_0} \frac{1}{6} \grad{\grad{\frac{1}{\abs*{\vb*{r}}}}} : \underbrace{3 \int \dd[3]{\vb*{r}'} \rho(\vb*{r}') \vb*{r}' \vb*{r}' }_{\vb*{D}}.
\end{equation}
在电荷分布相对于坐标系原点空间反演对称时,电偶极矩是零,而当电荷分布相对于坐标系原点空间反演反对称时,电四极矩是零。

容易看出,电四极矩$D_{ij}$是对称的,因此有6个独立分量。实际上这些独立分量并不都是有用的,注意到$\vb*{r} \neq 0$时
\[
    \laplacian{\frac{1}{\abs*{\vb*{r}}}} = 0,
\]
我们发现
\[
    \grad{\grad{\frac{1}{\abs*{\vb*{r}}}}} : \vb*{I} = 0,
\]
即我们可以任意地在$\vb*{D}$中加上单位张量的倍数,而不改变电势分布。因此我们可以手动加入一个约束:定义\concept{约化电四极矩}
\begin{equation}
    \tilde{\vb*{D}} = \vb*{D} - \frac{1}{3} \trace(\vb*{D}) \vb*{I} = \int \dd[3]{\vb*{r}'} (3 \vb*{r}' \vb*{r}' - \abs*{\vb*{r}'}^2 \vb*{I}) \rho(\vb*{r}') ,
\end{equation}
将$\vb*{D}$的迹消除掉,然后用$\tilde{\vb*{D}}$代替$\vb*{D}$同样可以得到正确的电四极矩;$\tilde{\vb*{D}}$独立的分量有5个,因为读多了一个无迹的条件。

电四极矩造成的电势衰减得比电偶极矩造成的电势快,电偶极矩造成的电势的衰减又比点电荷快。随着场点越来越接近源点,越来越复杂的电多极矩结构开始展现出来。

\section{静磁学}

静磁学指的是存在电荷流动,但是各个物理量的分布都恒稳的情况。要保持电流存在必须有一个外部的驱动力,这意味着此时的电磁场%
\footnote{
    当然,这个外部驱动力通常归根到底也是电磁力;但是我们将与它有关的那部分场自由度积掉了。
}%
不再是一个孤立体系。这可能让一些使用能量做的推导不再成立。%
\footnote{
    一种可能的诘难是,维持静电场的稳定也需要外部力(恩肖定理),为什么我们从来将静电场当成孤立系统看待?
    原因是,单纯从理论上说,要维持静电场稳定我们只需要将各个导体、电荷的动力学“关掉”即可(如认为点电荷受力不运动),等价的,维持静电场稳定的外力并不做功。
    另一方面,我们不能对电流做同样的事情:我们必须引入电流和电场之间的本构关系,从而自然地产生一个能量耗散项。
    静磁学理论中不可能不考虑这个能量耗散项。
}%

\begin{equation}
    \vb*{A}(\vb*{r}) = \frac{\mu_0}{4\pi} \frac{I \vb*{S} \times \vb*{r}}{\abs*{\vb*{r}}^3} = \frac{\mu_0}{4\pi} \frac{\vb*{m} \times \vb*{r}}{\abs*{\vb*{r}}^3}.
\end{equation}

\begin{equation}
    \vb*{B}(\vb*{r}) = - \frac{\mu_0}{4\pi} \left( \frac{\vb*{m} - 3 (\vb*{m} \cdot \vb*{e}_r) \vb*{e}_r}{r^3} \right).
\end{equation}

由于在边界上$\vb*{B}$有限大,应有
\begin{equation}
    \vb*{n} \times (\vb*{A}_2 - \vb*{A}_1) = 0.
\end{equation}
对库伦规范,

讨论静磁学系统的能量时需要把电源考虑进去,因为系统构型的小的变化会带来一个感生电动势,从而改变一些分布?

\section{似稳场和电路}

\subsection{基本方程}

\subsubsection{似稳条件}

本节开始我们讨论随时间发生变化的系统。当然,完整地解麦克斯韦方程是最精确的,但是很多情况下我们发现这类系统并没有特别明显的电磁辐射。
只要电场生磁场、磁场生电场,就可以有电磁辐射,因此电磁辐射不明显的系统中要么基本上没有电场产生的磁场,要么没有磁场产生的电场。 % TODO:,要么两者都有但是可以在感生电场和感生磁场之间建立直接的关系从而简化
这就是\concept{似稳场}或者\concept{准静态场}。体系中的电流被束缚在一些体积相对于电磁波波长不大的导体中的情况,也\concept{电路},经常可以用似稳场处理。
本节将主要讨论没有电场产生的磁场的情况,即忽略了位移电流的情况。
如果特殊需求,假定系统中的各个本构关系都是线性的。
此时麦克斯韦方程为
\begin{equation}
    \begin{bigcase}
        \div{\vb*{D}} &= \rho, \\
        \curl{\vb*{E}} &= - \pdv{\vb*{B}}{t}, \\
        \div{\vb*{B}} &= 0, \\
        \curl{\vb*{H}} &= \vb*{j}.
    \end{bigcase}
\end{equation}

何时能够使用似稳场近似?对良导体,位移电流肯定要充分小,即
\[
    \pdv*{\vb*{D}}{t} \ll \sigma \vb*{E},
\]
即
\begin{equation}
    \omega \ll \omega_\sigma = \frac{\sigma}{\epsilon}.
    \label{eq:quasi-stable-field-cond-1}
\end{equation}
这个条件实际上是过于宽松的,对很多金属,在$\omega$增大时,$\sigma$会有较大的虚部,因为金属中的电子在外场作用下不断“折返跑”,不会有宏观上的定向移动。
换而言之,此时的电流更像是束缚电流而不是传导电流,$\vb*{j} = \sigma \vb*{E}$已经没有意义了。

对绝缘体,即在导电区域以外,显然只有$\omega=0$即完全静态的情况下才有\eqref{eq:quasi-stable-field-cond-1}成立。
然而,这仅仅意味着我们没有全空间的似稳场近似,并不意味着在系统的空间尺度较小时没有似稳场近似。
在绝缘体条件下,以$\epsilon$和$\mu$代替真空中的李纳-维谢尔势中的$\epsilon_0$和$\mu_0$,于是
\[
    \begin{aligned}
        \vb*{B}(\vb*{r}, t) &\approx \frac{\mu}{4\pi} \int \dd[3]{\vb*{r}'} \frac{\vb*{j}(\vb*{r}', t - R / c) \times \vb*{R}}{R^3} \\
        &= \frac{\mu}{4\pi} \int \dd[3]{\vb*{r}'} \frac{\vb*{j}(\vb*{r}', t) \ee^{- \ii \omega (t - R / c)} \times \vb*{R}}{R^3},
    \end{aligned}
\]
如果某一点的电流变化要瞬间传递到系统的各处,应有
\begin{equation}
    R \ll \frac{c}{\omega}.
\end{equation}
这当然是非常合理的:扰动基本上以光速传递,因此如果系统足够小,那么系统内一点的扰动总是可以快速传遍整个系统。

\subsubsection{场的扩散方程}

% TODO:似乎涉及电荷的重新分布等问题的情况不能用似稳场近似,因为在似稳场情况下电流散度为零

在系统中各处的本构关系都是空间均匀的情况下,经过大约为$\epsilon/\sigma$量级的时间,电荷密度为零(请注意这个结论和是否有外加场、外加场是否变化无关),因此大部分时候我们只需要求解
\[
    \begin{bigcase}
        \div{\vb*{D}} &= 0, \\
        \curl{\vb*{E}} &= - \mu \pdv{\vb*{H}}{t}, \\
        \div{\vb*{H}} &= 0, \\
        \curl{\vb*{H}} &= \sigma \vb*{E},
    \end{bigcase}
\]
从而
\begin{equation}
    \laplacian{\vb*{E}} = \mu \sigma \pdv{\vb*{E}}{t}, \quad \laplacian{\vb*{H}} = \mu \sigma \pdv{\vb*{H}}{t}.
\end{equation}
这就是\concept{场的扩散方程}。可以发现,对良好的导体,场的扩散反而是非常慢的,这是正确的,因为静电场中导体内部不应该有电场,因此在似稳场下导体内部的电场应该很弱,正好说明场的扩散很差。

\section{电磁波的传播}

本节讨论电磁波在各种环境中的传播;电磁波的产生,即\concept{辐射}机制,留待 % TODO

\subsection{基本方程}

绝缘体中没有传导电流或是电荷,于是
% TODO:带导电性,从而$\sigma$可以灵活调节
\begin{equation}
    \begin{bigcase}
        \div{\vb*{D}} &= 0, \\
        \curl{\vb*{E}} &= - \pdv{\vb*{B}}{t}, \\
        \div{\vb*{B}} &= 0, \\
        \curl{\vb*{H}} &= \pdv{\vb*{D}}{t},
    \end{bigcase}
\end{equation}
仿照真空中电磁波波动方程的推导,我们有
\begin{equation}
    \laplacian{\vb*{E}} = \mu \epsilon \pdv[2]{\vb*{E}}{t}, \quad \laplacian{\vb*{B}} = \mu \epsilon \pdv[2]{\vb*{B}}{t}.
\end{equation}

\section{辐射}

上一节我们讨论了电磁波的传播,本节则讨论电磁波是怎么产生的。\eqref{eq:retarded-phi}和\eqref{eq:retarded-a}给出了

\end{document}
