\chapter{用量子场论计算可观察量}

\section{散射}

高能物理实验中涉及的物理过程的时间尺度通常远远小于我们能够观察的时间尺度,并且一般很难研究束缚态(部分是因为束缚态问题基本上是凝聚态研究的范畴而此时除了库仑定律以外也不需要太多物理,部分是因为微扰计算束缚态问题非常困难)。
因此,大部分情况下我们都只需要考虑初末态均在无穷远处的散射问题即可。因此,需要计算的主要就是$S$矩阵——实际上是动量表象下的$S$矩阵,以下如无特殊说明,$S$矩阵指的就是动量表象下的$S$矩阵。
然而,实验中能够制备的含有一定量的动量确定的粒子的态是完整的哈密顿量的本征态(而不是自由理论的本征态),此时的多粒子态的能量不是其单粒子能量的简单相加。
原则上我们同样可以微扰计算完整的哈密顿量的本征态,但是这将是非常费力并且困难的。

然而,在实际的散射问题中,无论是入射态还是出射态的结构都相对简单,因为实际的物理粒子的波函数的分布不可能是遍布全空间的平面波,而只能是动量大体上确定的波包,而将时间推到$\infty$或是$-\infty$时这些波包会相隔得足够远,从而它们之间没有相互作用,整个量子态的能量也就是各个粒子的能量之和。(或者等价地说,入射和出射时相互作用可能认为是“关闭”的,而只有粒子相隔得足够近才被打开)
因此,入射态和出射态实际上仍然能够写成
\[
    \ket{p_1, p_2, \cdots, p_n} = \sqrt{2\omega_{\vb*{p}_1}} \cdots \sqrt{2\omega_{\vb*{p}_n}} a^\dagger_{\vb*{p}_1} a^\dagger_{\vb*{p}_2} \cdots a^\dagger_{\vb*{p}_n} \ket{\Omega}
\]
的形式,但是此时的$a$算符——可以记为$a_\text{in}$或$a_\text{out}$——已经不再是自由场(“裸的”场)的$a$算符了,因为它创建的单粒子态是实际的、加入了相互作用的哈密顿量的本征态,即它创建的单粒子态是物理粒子,而不是没有相互作用时的裸粒子;同样$\omega$也不再是裸的单粒子能量。%
\footnote{
    我们对入射态和出射态的要求是非常严格的:一方面,出现在这些态中的粒子一定是能够稳定存在的粒子,寿命有限的准粒子(对应完整的哈密顿量的展宽的能级,可以看成是近似的本征态,但是因为各种扰动而不是严格的本征态)不能出现在入射态和出射态中;我们还进一步要求这些粒子之间的距离足够远,以至于可以近似看成自由的;最后我们还要求这些态被正确地归一化了,从而引入场强重整化因子。

    凝聚态理论考虑的系统状态比这多得多:寿命有限的准粒子时常需要被讨论,并且通常不会一次重整化就把所有相互作用都去除而只留下带自能修正的粒子。
    例如,费米液体理论中的“电子”实际上是已经经过相互作用修正的能带电子,但是即使在系统基态中这些准粒子的相互碰撞仍然需要纳入考虑。
}%
在海森堡绘景下,$S$矩阵的矩阵元基本上就具有
\[
    \braket*{p_1, p_2}{k_1, k_2} = \sqrt{2\omega_{\vb*{p}_1}} \sqrt{2\omega_{\vb*{p}_2}} \sqrt{2\omega_{\vb*{k}_1}} \sqrt{2\omega_{\vb*{k}_2}} \mel*{\Omega}{a_{\vb*{p}_2} a_{\vb*{p}_1} a^\dagger_{\vb*{k}_1} a^\dagger_{\vb*{k}_2}}{\Omega}
\]
这样的形式,这意味着只要能够找到$a_\text{in}$,$a_\text{out}$和裸场$a$之间的关系,即可确定关联函数和$S$矩阵的关系。
为了看出$a_\text{in}$,$a_\text{out}$和裸场$a$之间的关系,我们可以尝试为这种近乎独立的$a_\text{in}$和$a_\text{out}$所描述的“自由粒子”写下一个有效理论,显然这个有效理论是完整的、带有相互作用的理论不断重整化的结果,相互作用会修正“自由理论”的参数。
由于只考虑单粒子,相互作用带来的修正实际上就是所谓的自能修正:能够调整的参数包括质量项和$\partial_\mu \phi \partial^\mu \phi$项,对后者的调节等价于对场本身的变换。(此外,由于只考虑单粒子,且四维动量$p$的安排非常接近在壳,粒子无法衰变,因此自能修正没有虚部)
因此我们得出结论:$a_\text{in}$,$a_\text{out}$和裸场$a$之间应该差一个常数,这个常数是场强重整化引入的因子。
这就确定了关联函数和$S$矩阵之间的关系:在统一到动量空间中之后,两者首先由于相互作用修正的原因会差一个场强重整化因子;动量空间中的关联函数的解析性质和自由场的关联函数非常相似,只不过前者的极点给出的“质量”会出现跑动;相比之下,$S$矩阵要光滑很多,因为没有极点。
可以从关联函数计算$S$矩阵,但是不能反过来,因为$S$矩阵仅仅包含初态位于$-\infty$而末态位于$\infty$的过程。
因此应当有:
\[
    \prod_{i} \frac{\text{renormalization factors}}{\omega - \omega_{\vb*{k}} + \ii 0^+} \times \mel*{p}{S}{k} \propto \text{Fourier transformation of} \mel*{\Omega}{\phi(x_1) \cdots \phi(x_n)}{\Omega}.
\]

虽然我们做出了以上形式的论证,完整地做Wilson重整化群计算而得到场强重整化因子当然还是非常繁琐的,因此本节尝试以一种更加简单的方法建立微扰计算$S$矩阵的方法。
本节将始终在以下对高能物理来说非常一般的假设下工作:系统具有完整的洛伦兹对称性;入射和出射态中各个粒子相隔足够远;只有接近关联函数极点时的动量-能量安排才是值得分析的,因为其它时候实验现象也不明显。

本节将首先介绍一些可以使用$S$矩阵计算的物理量,以展示计算$S$矩阵的必要性,然后给出微扰计算$S$矩阵的方法。

\subsection{可观察物理量}

\subsubsection{散射截面}

\begin{equation}
    \mel*{p_1, p_2, \ldots, p_m}{\ii T}{k_1, k_2, \ldots, k_n} = \ii (2\pi)^4 \delta^4(\sum_i p_i - \sum_j k_j) \mathcal{M}(k_1, k_2, \ldots, k_n \to p_1, p_2, \ldots, p_m).
\end{equation}

\subsubsection{跃迁率}

\subsubsection{非相对论极限}

我们知道$S$矩阵可以用所谓李普曼-施温格方程写出。非相对论性单粒子量子力学中通常采取这样的归一化方案:
\[
    \hat{T} = \hat{H}' + \hat{H}' \frac{1}{E - \hat{H}_0} \hat{H}' + \cdots, \quad \hat{S} = 1 - 2 \pi \ii \delta(E_f - E_i) \hat{T},
\]
% TODO:怎么计算等效势能

\subsection{S矩阵和关联函数的关系}

\subsubsection{标量场编时格林函数的解析结构和LSZ约化公式}

考虑一个$n$点关联函数$\mel{\Omega}{T \phi(x_1) \cdots \phi(x_n)}{\Omega}$。
我们对$x_1$做傅里叶变换,这是因为实际上用于计算$S$矩阵矩阵元的本征态都是动量本征态,因此我们首先应该在$n$点关联函数中引入动量的概念。
我们没有直接使用用动量标记的产生湮灭算符,因为我们还希望用频率代替时间,而“频率”并不是量子数,因此不是任何一套产生湮灭算符的标签的一部分,如果使用用动量标记的产生湮灭算符就还需要对时间做一次傅里叶变换,表达式就看起来不协变了。
反之,对四维坐标$x_1$做完傅里叶变换之后,我们就可以用四维动量$k_1$代替$x_1$作为标记了,洛伦兹协变性就很明显。
这样得到的动量空间关联函数不保证入射和出射的四维动量是在壳的。(另一方面,$S$矩阵肯定是在壳的)

我们有
\begin{equation}
    \begin{aligned}
        &\quad \int \dd[4]{x_1} \ee^{\ii k_1 \cdot x_1} \mel{\Omega}{T \phi(x_1) \phi(x_2) \cdots \phi(x_n)}{\Omega} \\
        &= \left( \int_{-\infty}^{T_-} + \int_{T_-}^{T_+} + \int_{T_+}^\infty \right) \dd{t_1} \int \dd[3]{\vb*{x}_1} \ee^{\ii \omega_1 t_1 - \ii \vb*{k}_1 \cdot \vb*{x}_1} \mel{\Omega}{T \phi(x_1) \phi(x_2) \cdots \phi(x_n)}{\Omega} \\
        &= \int_{-\infty}^{T_-} \dd{t_1} \ee^{\ii \omega_1 t_1} \int \dd[3]{\vb*{x}_1} \ee^{- \ii \vb*{k}_1 \cdot \vb*{x}_1} \mel{\Omega}{T [\phi(x_2) \cdots \phi(x_n)] \phi(x_1)}{\Omega} \\
        &+ \int_{T_+}^{\infty} \dd{t_1} \ee^{\ii \omega_1 t_1} \int \dd[3]{\vb*{x}_1} \ee^{- \ii \vb*{k}_1 \cdot \vb*{x}_1} \mel{\Omega}{\phi(x_1) T [\phi(x_2) \cdots \phi(x_n)]}{\Omega} \\
        &+ \int_{T_-}^{T_+} \dd{t_1} \ee^{\ii \omega_1 t_1} \int \dd[3]{\vb*{x}_1} \ee^{- \ii \vb*{k}_1 \cdot \vb*{x}_1} \mel{\Omega}{T [\phi(x_1) \phi(x_2) \cdots \phi(x_n)]}{\Omega}.
    \end{aligned}
    \label{eq:correlation-after-fourier}
\end{equation}
最后一个等号只需要令$T^+$充分大,$T_-$充分小即可成立。
做完傅里叶变换之后的结果肯定具有奇异性:在相互作用下稳定的单粒子态有确定的动量和能量,因此在$\omega_1$和$\vb*{k}_1$之间满足正确的色散关系时,上式发散。
(多粒子态由于能量给定时其中某个动量可以连续变化,不会产生奇异性,但是会产生有限高度的峰)

高能物理实验中实际会测量的区域通常就在这种会导致奇异性的$\omega$和$\vb*{k}$的取值附近,因为这里效应最明显。
在\eqref{eq:correlation-after-fourier}的极点附近,$T_-$到$T_+$区段的积分基本上没有什么作用,因为其奇异性弱于$T_+$到$\infty$的积分和$-\infty$到$T_-$的积分。
因此之后我们不再考虑$T_-$到$T_+$区段的积分。

现在使用常用的技巧:在算符连乘积序列中插入一个完备关系,从而将$n$点关联函数转化为$n-1$点关联函数。
即使在加入相互作用之后,动量仍然是好量子数,因此可以有完备关系
\[
    1 = \sum_{\Lambda} \int \frac{\dd[3]{\vb*{p}}}{(2\pi)^3} \dyad*{\Lambda_{\vb*{p}}}, 
\]
其中$\Lambda$标记了冗余的量子数。对\eqref{eq:correlation-after-fourier}最后一步的第二项,插入完备关系得到
\[
    \begin{aligned}
        &\quad \int_{T_+}^{\infty} \dd{t_1} \ee^{\ii \omega_1 t_1} \int \dd[3]{\vb*{x}_1} \ee^{- \ii \vb*{k}_1 \cdot \vb*{x}_1} \mel{\Omega}{\phi(x_1) T [\phi(x_2) \cdots \phi(x_n)]}{\Omega} \\
        &= \int_{T_+}^{\infty} \dd{t_1} \ee^{\ii \omega_1 t_1} \int \dd[3]{\vb*{x}_1} \ee^{- \ii \vb*{k}_1 \cdot \vb*{x}_1} \mel*{\Omega}{\phi(x_1) \sum_{\Lambda} \int \frac{\dd[3]{\vb*{p}}}{(2\pi)^3} \dyad*{\Lambda_{\vb*{p}}} T [\phi(x_2) \cdots \phi(x_n)]}{\Omega} \\
        &= \sum_{\Lambda} \int \frac{\dd[3]{\vb*{p}}}{(2\pi)^3} \int_{T_+}^{\infty} \dd{t_1} \ee^{\ii \omega_1 t_1} \int \dd[3]{\vb*{x}_1} \ee^{- \ii \vb*{k}_1 \cdot \vb*{x}_1} \mel*{\Omega}{\phi(x_1)}{\Lambda_{\vb*{p}}} \mel*{\Lambda_{\vb*{p}}}{T [\phi(x_2) \cdots \phi(x_n)]}{\Omega}.
    \end{aligned}
\]
因子$\mel*{\Omega}{\phi(x_1)}{\Lambda_{\vb*{p}}}$可以进一步展开。
首先,因子$\mel*{\Omega}{\phi(0)}{\Lambda_{\vb*{k}_1}}$实际上只在$\ket{\Lambda_{\vb*{k}_1}}$是单粒子态时才能够有非零值(因为它可以写成一个经过场强重整化的产生算符作用在真空态上,而入射态和出射态可以近似看成自由理论,从而被夹在$\mel*{\Omega}{\cdot}{\Omega}$中的产生算符和湮灭算符数目必须一致)。
其次,我们注意到,实际上可以将$\phi(x_1)$写成$\ee^{\ii P \cdot x} \phi(0) \ee^{- \ii P \cdot x}$,其中$P$是全空间的动量算符,或者说平移群的生成元。在有相互作用的情况下,这样作用的结果并不方便计算,因为其中包含一个时间演化算符,但是我们总是可以形式地写出这样的式子。
于是
\[
    \mel*{\Omega}{\phi(x)}{\Lambda_{\vb*{p}}} = \mel*{\Omega}{\phi(x)}{\Lambda_{\vb*{p}}} \ee^{- \ii p \cdot x} |_{p^0 = E_{\vb*{p}}}. 
\]
于是就有
\[
    \begin{aligned}
        &\quad \int_{T_+}^{\infty} \dd{t_1} \ee^{\ii \omega_1 t_1} \int \dd[3]{\vb*{x}_1} \ee^{- \ii \vb*{k}_1 \cdot \vb*{x}_1} \mel{\Omega}{\phi(x_1) T [\phi(x_2) \cdots \phi(x_n)]}{\Omega} \\
        &= \sum_{\Lambda} \int \frac{\dd[3]{\vb*{p}}}{(2\pi)^3} \int_{T_+}^{\infty} \dd{t_1} \ee^{\ii (\omega_1 - E_{\vb*{p}}) t_1} \int \dd[3]{\vb*{x}_1} \ee^{- \ii (\vb*{k}_1 - \vb*{p}) \cdot \vb*{x}_1} \mel*{\Omega}{\phi(0)}{\Lambda_{\vb*{p}}} \mel*{\Lambda_{\vb*{p}}}{T [\phi(x_2) \cdots \phi(x_n)]}{\Omega} \\
        &= \sum_{\Lambda} \int \frac{\dd[3]{\vb*{p}}}{(2\pi)^3} \left( - \frac{1}{\ii (\omega_1 - E_{\vb*{p}} + \ii 0^+)} \ee^{\ii (\omega_1 - E_{\vb*{p}}) T_+} \right) (2\pi)^3 \delta(\vb*{k}_1 - \vb*{p}) \\
        & \quad \times \mel*{\Omega}{\phi(0)}{\Lambda_{\vb*{p}}} \mel*{\Lambda_{\vb*{p}}}{T [\phi(x_2) \cdots \phi(x_n)]}{\Omega} \\
        &= \sum_{\Lambda} \frac{\ii}{\omega_1 - E_{\vb*{k}_1}(\Lambda) + \ii 0^+} \ee^{\ii (\omega_1 - E_{\vb*{p}}) T_+} \mel*{\Omega}{\phi(0)}{\Lambda_{\vb*{k}_1}} \mel*{\Lambda_{\vb*{k}_1}}{T [\phi(x_2) \cdots \phi(x_n)]}{\Omega}.
    \end{aligned}
\]
上式中(以及之后),$E_{\vb*{k}}$代表的都是$\ket{\Lambda_{\vb*{k}_1}}$的能量,不是裸粒子的能量,已经加入了自能修正。
在这里我们使用标准的在对时间的积分中引入无穷小虚部,让$\omega_1$的奇点出现在下半平面的方法;实际上,如果不这样,积分也无法收敛。
我们现在稍微改变一下$\ket{\Lambda_{\vb*{p}}}$的归一化方式。目前使用的归一化方案是不协变的,对应于$\ket{\vb*{p}}$,现在我们转而使用
\[
    \ket{\lambda_{\vb*{p}}} = \sqrt{2 E_{\vb*{p}}} \ket{\Lambda_{\vb*{p}}},
\]
于是就有
\[
    \begin{aligned}
        &\quad \int_{T_+}^{\infty} \dd{t_1} \ee^{\ii \omega_1 t_1} \int \dd[3]{\vb*{x}_1} \ee^{- \ii \vb*{k}_1 \cdot \vb*{x}_1} \mel{\Omega}{\phi(x_1) T [\phi(x_2) \cdots \phi(x_n)]}{\Omega} \\
        &= \sum_{\lambda} \frac{1}{2E_{\vb*{k}_1}(\Lambda)} \frac{\ii}{\omega_1 - E_{\vb*{k}_1}(\Lambda) + \ii 0^+} \ee^{\ii (\omega_1 - E_{\vb*{p}}) T_+} \mel*{\Omega}{\phi(0)}{\lambda_{\vb*{k}_1}} \mel*{\lambda_{\vb*{k}_1}}{T [\phi(x_2) \cdots \phi(x_n)]}{\Omega}.
    \end{aligned}
\]
在接近\eqref{eq:correlation-after-fourier}的极点时,就有
\[
    \begin{aligned}
        &\quad \int_{T_+}^{\infty} \dd{t_1} \ee^{\ii \omega_1 t_1} \int \dd[3]{\vb*{x}_1} \ee^{- \ii \vb*{k}_1 \cdot \vb*{x}_1} \mel{\Omega}{\phi(x_1) T [\phi(x_2) \cdots \phi(x_n)]}{\Omega} \\
        &= \sum_{\lambda} \frac{\ii}{(\omega_1)^2 - (E_{\vb*{k}_1}(\lambda))^2 + \ii 0^+} \mel*{\Omega}{\phi(0)}{\lambda_{\vb*{k}_1}} \mel*{\lambda_{\vb*{k}_1}}{T [\phi(x_2) \cdots \phi(x_n)]}{\Omega}.
    \end{aligned}
\]
类似地可以得到
\[
    \begin{aligned}
        &\quad \int_{-\infty}^{T_-} \dd{t_1} \ee^{\ii \omega_1 t_1} \int \dd[3]{\vb*{x}_1} \ee^{- \ii \vb*{k}_1 \cdot \vb*{x}_1} \mel{\Omega}{T [\phi(x_2) \cdots \phi(x_n)] \phi(x_1)}{\Omega} \\
        &= - \sum_{\lambda} \frac{1}{2E_{\vb*{k}_1}(\lambda)} \frac{\ii}{\omega_1 + E_{\vb*{k}_1}(\lambda) + \ii 0^+} \ee^{\ii (\omega_1 + E_{\vb*{p}}) T_-} \mel*{\Omega}{T [\phi(x_2) \cdots \phi(x_n)]}{\lambda_{\vb*{k}_1}} \mel*{\lambda_{\vb*{k}_1}}{\phi(0)}{\Omega},
    \end{aligned}
\]
其中由于$\phi(x_1)$的位置发生了变化,一些量的正负号和左右矢的顺序发生了变化。不过,这一项并不产生极点。
因此最终我们得到
\[
    \begin{aligned}
        &\quad \int_{T_+}^{\infty} \dd{t_1} \ee^{\ii \omega_1 t_1} \int \dd[3]{\vb*{x}_1} \ee^{- \ii \vb*{k}_1 \cdot \vb*{x}_1} \mel{\Omega}{\phi(x_1) T [\phi(x_2) \cdots \phi(x_n)]}{\Omega} \\
        &\stackrel{\omega_1 \to E_{\vb*{k}_1}(\Lambda)}{\sim} \sum_{\lambda} \frac{\ii}{(\omega_1)^2 - (E_{\vb*{k}_1}(\lambda))^2 + \ii 0^+} \mel*{\Omega}{\phi(0)}{\lambda_{\vb*{k}_1}} \mel*{\lambda_{\vb*{k}_1}}{T [\phi(x_2) \cdots \phi(x_n)]}{\Omega}.
    \end{aligned}
\]
此外,设$U$是一个让三维动量减小$\vb*{k}_1$的洛伦兹变换,则
\[
    \begin{aligned}
        \mel*{\Omega}{\phi(0)}{\lambda_{\vb*{k}_1}} &= \mel*{\Omega}{\phi(0)U^{-1} U}{\lambda_{\vb*{k}_1}} \\
        &= \mel*{\Omega}{U \phi(0)U^{-1} U}{\lambda_{\vb*{k}_1}} \\
        &= \mel*{\Omega}{\phi(0)}{\lambda_0},
    \end{aligned}
\]
第二个等号是因为真空态在洛伦兹变换下显然是不变的,第三个等号用到了$U \phi(0) U^{-1} = \phi(0)$这一事实(将$\phi(0)$展开为傅里叶级数就能看出为什么)。
于是就得到
\[
    \begin{aligned}
        &\quad \int \dd[4]{x_1} \ee^{\ii k_1 \cdot x_1} \mel{\Omega}{\phi(x_1) T [\phi(x_2) \cdots \phi(x_n)]}{\Omega} \\
        &\stackrel{\omega_1 \to E_{\vb*{k}_1}(\lambda)}{\sim} \sum_{\lambda} \frac{\ii}{(\omega_1)^2 - (E_{\vb*{k}_1}(\lambda))^2 + \ii 0^+} \mel*{\Omega}{\phi(0)}{\lambda_{0}} \mel*{\lambda_{\vb*{k}_1}}{T [\phi(x_2) \cdots \phi(x_n)]}{\Omega}.
    \end{aligned}
\]
上面的结果的意义非常明显了:$x_1$变量做了傅里叶变换的关联函数会有一系列极点,这些极点的具体位置由在带相互作用的场论下的本征态的能谱(而不是自由粒子的能谱)决定;并且,会多出来一个因子$\mel*{\Omega}{\phi(0)}{\lambda_{0}}$。
由于$\ket{\lambda_{\vb*{k}_1}}$是经过相互作用修正的单粒子态,实际上只需要一个动量参数就足够标记它,于是我们去掉对$\lambda$的求和(因为显然只有一个可能的$\lambda$),并且用$\omega$代替$E$(再次提醒:这是已经经过相互作用修正的单粒子能量),就得到
\begin{equation}
    \begin{aligned}
        &\quad \int \dd[4]{x_1} \ee^{\ii k_1 \cdot x_1} \mel{\Omega}{\phi(x_1) T [\phi(x_2) \cdots \phi(x_n)]}{\Omega} \\
        &\stackrel{\omega_1 \to \omega_{\vb*{k}_1}}{\sim} \frac{\ii}{\omega_1^2 - \omega_{\vb*{k}_1}^2 + \ii 0^+} \mel*{\Omega}{\phi(0)}{p={0}} \mel*{k_1}{T [\phi(x_2) \cdots \phi(x_n)]}{\Omega}.
    \end{aligned}
    \label{eq:scalar-correlation-pole-single}
\end{equation}
这里我们已经用$\ket{k_1}$代替了$\ket{\lambda_{\vb*{k}_1}}$,前者就表示无穷远处的单粒子态。

现在设想我们对$\mel*{\Omega}{T \phi(x_1) \phi(x_n)}{\Omega}$中的每一个位置变量都做傅里叶变换。
由\eqref{eq:scalar-correlation-pole-single},会发现每个$k_i$实际上都有极点。
我们实际上可以在一个公式内把所有这些极点都反映出来。考虑对\eqref{eq:scalar-correlation-pole-single}中的$x_2$做傅里叶变换,由于$x_2$仅仅包含在最后一个因子中,只需要计算
\[
    \begin{aligned}
        &\quad \int \dd[4]{x_2} \ee^{\ii k_2 \cdot x_2} \mel*{k_1}{T [\phi(x_2) \cdots \phi(x_n)]}{\Omega} \\
        &= \left( \int_{-\infty}^{T_-} + \int_{T_-}^{T_+} + \int_{T_+}^\infty \right) \dd{t_2} \ee^{\ii \omega_2 t_2} \int \dd[3]{\vb*{x}_2} \ee^{-\ii \vb*{k}_2 \cdot \vb*{x}_2} \mel*{k_1}{T [\phi(x_2) \cdots \phi(x_n)]}{\Omega}.
    \end{aligned}
\]
仿照我们先前做的操作,只需要计算$T_+$到$\infty$的积分即可得到最为奇异的部分,于是可以将$\phi(x_2)$提出编时算符的作用域内,放在最左边,然后仿照前面的操作,在$\phi(x_2)$和编时算符序列之间插入完备性关系,得到
\[
    \begin{aligned}
        &\quad \int_{T_+}^\infty \dd{t_2} \ee^{\ii \omega_2 t_2} \int \dd[3]{\vb*{x}_2} \ee^{-\ii \vb*{k}_2 \cdot \vb*{x}_2} \mel*{k_1}{\phi(x_2) T [\phi(x_3) \cdots \phi(x_n)]}{\Omega} \\
        &= \int \frac{\dd[3]{\vb*{p}_1}}{(2\pi)^3} \frac{1}{2 \omega_{\vb*{p}_1}} \int \frac{\dd[3]{\vb*{p}_2}}{(2\pi)^3} \frac{1}{2 \omega_{\vb*{p}_2}} \int_{T_+}^\infty \dd{t_2} \ee^{\ii \omega_2 t_2} \int \dd[3]{\vb*{x}_2} \ee^{-\ii \vb*{k}_2 \cdot \vb*{x}_2} \\ 
        &\quad \quad \times \mel*{k_1}{\phi(x_2)}{p_1, p_2} \mel*{p_1, p_2}{T [\phi(x_3) \cdots \phi(x_n)]}{\Omega} \\
        &= \int \frac{\dd[3]{\vb*{p}}}{(2\pi)^3} \frac{1}{2 \omega_{\vb*{p}}} \int_{T_+}^\infty \dd{t_2} \ee^{\ii \omega_2 t_2} \int \dd[3]{\vb*{x}_2} \ee^{-\ii \vb*{k}_2 \cdot \vb*{x}_2}  \mel*{\Omega}{\phi(x_2)}{p} \mel*{k_1, p}{T [\phi(x_3) \cdots \phi(x_n)]}{\Omega} \\
    \end{aligned}
\]
这一次只有$\mel*{k_1}{\phi(x_2)}{p_1, p_2}$有非零值(注意$\ket{p_1, p_2}$是入射/反射态,可以用经过场重整化的产生算符作用两次产生),于是我们引入了两个动量积分;第二个等号还是因为$\ket{p_1, p_2}$可以拆分,从而其中一个必须和$k_1$相等。
对因子$\mel*{\Omega}{\phi(x_2)}{p}$施加如前所述的插入四维平移算符的操作,得到
\[
    \mel*{\Omega}{\phi(x_2)}{p} = \mel*{\Omega}{\phi(0)}{p=0} \ee^{-\ii p \cdot x_2}|_{p^0=E_{\vb*{p}}},
\]
然后积掉$t_2$和$\vb*{x}_2$,就得到
\begin{equation}
    \begin{aligned}
        &\quad \int \dd[4]{x_1} \ee^{\ii k_1 \cdot x_1} \int \dd[4]{x_2} \ee^{\ii k_2 \cdot x_2} \mel{\Omega}{\phi(x_1) T [\phi(x_2) \cdots \phi(x_n)]}{\Omega} \\
        &\stackrel{\omega_1 \to \omega_{\vb*{k}_1}, \omega_2 \to \omega_{\vb*{k}_2}}{\sim} \frac{\ii \mel*{\Omega}{\phi(0)}{p={0}}}{\omega_1^2 - \omega_{\vb*{k}_1}^2 + \ii 0^+} \frac{\ii \mel*{\Omega}{\phi(0)}{p={0}}}{\omega_2^2 - \omega_{\vb*{k}_2}^2 + \ii 0^+} \mel*{k_1, k_2}{T [\phi(x_2) \cdots \phi(x_n)]}{\Omega}.
    \end{aligned}
    \label{eq:scalar-correlation-pole-double}
\end{equation}
如此重复——实际上,我们还可以反过来做以上步骤,此时为了让极点出现,傅里叶变换的$\ee$指数要加上一个减号,在本节处理的标量场中这无所谓,但是对有粒子-准粒子区别的场,代表入射粒子的场算符和代表出射粒子的场算符差一个负号,从而一些要正着做傅里叶变换一些要反着做傅里叶变换——就得到
\begin{equation}
    \begin{aligned}
        &\quad \prod_{i=1}^m \int \dd[4]{x_i} \ee^{\ii p_i \cdot x_i} \prod_{j=1}^n \int \dd[4]{y_j} \ee^{- \ii k_j \cdot y_j} \mel{\Omega}{T [\phi(x_1) \cdots \phi(x_n) \phi(y_1) \cdots \phi(y_n)]}{\Omega} \\
        &\stackrel{p_i^0 \to \omega_{\vb*{p}_i}, \; k_j^0 \to \omega_{\vb*{k}_j}}{\sim} \prod_{i=1}^m \frac{\ii \sqrt{Z}}{\omega_i^2 - \omega_{\vb*{p}_i}^2 + \ii 0^+} \prod_{j=1}^n \frac{\ii \sqrt{Z}}{\omega_j^2 - \omega_{\vb*{k}_j}^2 + \ii 0^+} \braket*{p_1, p_2, \ldots, p_m}{k_1, k_2, \ldots, k_n},
    \end{aligned}
    \label{eq:lsz-reduction-scalar}
\end{equation}
其中
\begin{equation}
    Z = \abs{\mel*{\Omega}{\phi(0)}{p=0}}^2.
    \label{eq:z-factor-def}
\end{equation}
以上所有的推导都是在海森堡绘景下完成的,因此上式右边的因子就是$S$矩阵。
于是我们就得到了联系$S$矩阵和关联函数的公式,其形式和之前的分析完全一样,$Z$正是场强重整化因子。%
\eqref{eq:lsz-reduction-scalar}就是标量场的\concept{LSZ约化公式},其形式和之前我们预期的完全一样。
实际上,LSZ约化公式说明重整化后的格林函数是重整化前的$1 / Z^{(m+n)/2}$倍。

\subsubsection{$S$矩阵的微扰计算}

在得到了\eqref{eq:lsz-reduction-scalar}之后就可以微扰计算$S$矩阵了,因为可以微扰计算关联函数。
实际上,计算关联函数比计算$S$矩阵更加困难。这本质上是因为$S$矩阵丢弃了初末态均为有限时间的信息($S$矩阵仅仅保留了动量空间关联函数那些四维动量均在壳的那部分初末态),并且在费曼图的语言下有非常显然的解释。
下面我们来分析关联函数的费曼图,并给出直接微扰计算$S$矩阵的方法。

计算关联函数的任何一张费曼图都具有这样的形式:一个amputated diagram居于中间,外线和它之间连接有自能修正图。
在动量空间下,一张自能修正图可以等价地看成
\[
    \int \dd[4]{x} \ee^{\ii p \cdot x} \int \dd[4]{y} \ee^{- \ii k \cdot y} \mel{\Omega}{\phi(x) \phi(y)}{\Omega},
\]
按照\eqref{eq:z-factor-def},$\sqrt{Z}$对应$\mel{\Omega}{\phi(x_1) a^\dagger_\text{in}}{\Omega}$,这个关联函数中的两个场中,一个做了场强重整化而另一个没有。
既然$\mel{\Omega}{a_\text{out} a^\dagger_\text{in}}{\Omega}$就是$1$,$\mel{\Omega}{\phi(x_1) \phi(x_2)}{\Omega}$的场强重整化因子为$Z$。

关联函数$\mel{\Omega}{\phi(x_1) \phi(x_2)}{\Omega}$可以直接微扰计算,从而在理论已经给定的情况下,我们可以把$Z$到底是什么写出来。
设$- \ii M^2(p^2)$是单粒子不可约图(对应自能修正),那么就有
\[
    \begin{aligned}
        &\quad \int \dd[4]{x} \ee^{\ii p \cdot x} \int \dd[4]{y} \ee^{- \ii k \cdot y} \mel{\Omega}{\phi(x_1) \phi(x_2)}{\Omega} \\
        &= \frac{\ii}{p^2 - m_0^2} + \frac{\ii}{p^2 - m_0^2} (- \ii M^2(p^2)) \frac{\ii}{p^2 - m_0^2} + \cdots \\
        &= \frac{\ii}{p^2 - m_0^2 - M^2(p^2)},
    \end{aligned}
\]
其中$m_0$为粒子裸质量。自能$M^2$关于$p^2$的最低阶项当然是$p^2$自己,即
\[
    M^2 = c_0 + c_1 p^2 + \cdots.
\]
在低阶近似下$c_1=1$,高阶下则会有可见的修正,实际上这就对应着对$(\partial_\mu \phi)^2$的修正。
因此,在极点附近,我们有
\begin{equation}
    \int \dd[4]{x} \ee^{\ii p \cdot x} \int \dd[4]{y} \ee^{- \ii k \cdot y} \mel{\Omega}{\phi(x_1) \phi(x_2)}{\Omega} \stackrel{p^0 \to E_{\vb*{p}}}{\sim} \frac{\ii Z}{p^2 - m^2} + \text{regular}. 
\end{equation}
因此这给出了计算$Z$和有效质量$m$的方法:微扰计算自能修正,然后按照上式化简,观察极点位置就得到了$m$,在极点附近比较关联函数和$\ii / (p^2 - m^2)$就得到了$Z$。
将上式代入\eqref{eq:lsz-reduction-scalar},就发现
\begin{equation}
    \braket*{p_1, p_2, \ldots, p_m}{k_1, k_2, \ldots, k_n} \stackrel{p_i^0 \to \omega_{\vb*{p}_i}, \; k_j^0 \to \omega_{\vb*{k}_j}}{\sim} (\sqrt{Z})^{m+n} \times \text{amputated diagrams}.
    \label{eq:amputated-diagram-z-factor}
\end{equation}
因此,计算自能修正并得到$Z$之后,只需要计算amputated diagrams就能够微扰计算得到$S$矩阵。

我们还可以更加简化一些。注意到,所有只含有单粒子自能修正的图——即有$n$个入射粒子,$n$个出射粒子,总共有$n$个连通子图,不同粒子对应不同连通子图的图——在amputate之后就是简单地将入射端和出射端连接起来,而这些图正好对应$S=1 + \ii T$中的$1$,因此如果只计算$\mathcal{M}$,无需计算这些图。

如果只计算树图,那么显然$Z=1$,因为此时没有任何自能修正,从而无论是物理质量还是场强都没有做重整化。
因此,计算树图时只将amputated tree diagram求和即可。
圈图的计算一般要用到重整化,此时为了方便看出场强重整化因子,并不会使用\eqref{eq:amputated-diagram-z-factor}。
但是,在圈图计算中,场强重整化的那些$\sqrt{Z}$通常是作为抵消项被引入了,在施加重整化条件之后,它们和圈图计算中的发散抵消了,从而其实也无需显式计算$Z$。
总之我们实际上不会直接使用LSZ约化公式计算散射振幅:如果只计算树图,那么$Z=1$;而如果计算圈图,那么$Z$是作为抵消项被考虑进去的,圈图计算的最终结果是对树图中的传播子的物理参数(基本上是质量)和顶角函数做了一定修正,因此$Z$同样不会出现在最终的计算结果中。

% TODO:路径积分量子化中的LSZ

\section{正规化和重整化}

如果朴素地做圈图计算,通常会得到紫外发散。%
\footnote{
    相对论性量子场论中的红外发散一般是因为少考虑了一些过程,如由于光子无能隙,可以任意地产生和消灭,从而一个过程的概率分散在很多个有入射和出射“软光子”(能量很低并无可观测效应的光子)。
    当求和所有这些过程后,发散一般就消失了,并不具有特别的意义。

    凝聚态场论中的情况正好相反:紫外发散是没有关系的,因为凝聚态系统中有最小的特征长度,如果出现了发散,可以引入一个物理可观测的紫外截断消除这个发散(如BCS理论中的$\omega_\text{D}$)。
    红外发散反而不那么好处理。
}%
这并不特别令人意外,因为没有什么保证了我们的理论在任意能标下都一定成立,从而,将动量积分的上限推到无穷大大约是不合适的。
要想计算出有意义的结果通常要求我们知道更高能标处的物理的细节;然而,如果我们的理论实际上具有低能标下的一个不动点,那么更高能标处的物理实际上无关紧要,只要我们只关心低能的现象。

在这种情况下,可以做下面的操作来消除发散:
\begin{enumerate}
    \item 在圈图计算中寻找一个能够标记重整化群流的参数,可能是和Wilson重整化群流一致的动量积分上限$\Lambda$,也可以将维数延拓到实数中,从而用维数偏离$4$的程度$\epsilon$做这个参数。此时积分的发散部分可以被分离出来,这就是\concept{正规化}。
    \item 令理论中的各个参数(所谓“裸”参数)跑动起来,包括场强,即引入\concept{抵消项},其大小暂时未知。
    \item 计算若干个可以实际观察到的物理量(通常具有和裸参数类似的物理意义,从而它们可以称为\concept{物理参数},如自由场的关联函数的极点给出裸质量,而有相互作用的关联函数的极点给出物理质量),将它们写成裸参数、参数跑动(即抵消项中的参数)和重整化群流参数的形式。
    \item 在低能不动点处,物理参数应当和重整化群流参数无关,因此可以反过来将参数跑动写成重整化群流参数和物理参数的函数,从而求解出所有抵消项。
    用于确定抵消项的条件即为\concept{重整化条件}。
    如果理论可重整,此时所有其它可以实际观察到的物理量中的发散都会相互抵消,从而我们成功地将一些可以实际观察到的物理量写成了另一些可以实际观察到的物理量(即物理参数)的函数,即给出了实验预言。
\end{enumerate}
在最终的计算结果中裸参数都没有出现;这是正确的,因为实际上并没有什么能够真的“观察”到裸参数——无法确定实际测到的物理参数有多少来自裸参数,多少来自相互作用修正。
实际上,为了和发散抵消,裸参数一般都是反向发散的。

在实际的计算中,还有以下技巧:
\begin{itemize}
    \item 由于圈图数目非常多,通常我们会以需要计算的散射振幅的树图当成骨架,将圈图当成对骨架图中各个成分的修正。
    这样的好处是,做完全部修正的骨架图的各个成分通常足够给出物理参数了,如做完自能修正的传播子可以给出物理质量,做完顶角修正的amputated vertex diagram可以给出有效相互作用强度。
    \item 由于实际有意义的关联函数、散射振幅等均已经做过场强重整化,可以用场强重整化之后的场来做拉氏量中的基本自由度;这样与场强重整化有关的抵消项会自动出现在拉氏量中,并且做完重整化之后直接计算amputated diagram即可得到散射振幅,没有必要再显式计算$Z$。
    \item 抵消项可以被显式给出。如果不希望引入太多顶角,也可以首先形式地写出含有未知的裸参数的散射振幅的形式,然后用未知的裸参数去拼凑出物理参数。
    如果显式地使用抵消项,由于我们在重整化不动点附近工作,其实可以将抵消项设置为“裸参数偏离物理参数的多少”,而直接将物理参数放进拉氏量中。
\end{itemize}

表面上,任何一个理论都可以做这样的操作——对称性允许的拉氏量中的项是无限多的,我们可以引入任意多的抵消项来消除发散。
但是,如果需要引入无数多的抵消项,那么重整化操作就是无法完成的。

为了估计发散的程度,我们可以引入一些指标。\concept{原始发散图}指的是只要切断一根内线(即不计算这根内线的积分),就能够收敛的图。发散的图是用原始发散图组装起来的。
如果一个理论中的原始发散图的个数有限,

由于紫外发散来自动量积分有太多重,一张图$\Gamma$中的动量的幂次——所谓\concept{表观发散度}——为
\begin{equation}
    D(\Gamma) = \sum_i n_i d_i + 2 I_\text{B} + 3 I_\text{F} - 4(\sum_i n_i - 1),
\end{equation}
其中$n_i$指的是某一类型的顶角的个数,$d_i$是类型$i$的顶角中的动量幂次,$I_\text{B}$和$I_\text{F}$分别表示玻色子和费米子线的个数,因为费米子传播子的分母中只有一个$k$,做完四维动量积分之后动量幂次为3,而玻色子传播子的分母中有两个$k$,做完四维动量积分之后动量幂次为2。
最后一项是因为顶角会引入一个动量守恒条件;我们故意减去了$1$,因为费曼图最终的计算结果也肯定满足动量守恒条件,即有一个$\delta(\sum \vb*{k})$并没有被积分掉,而是留到了计算结果中。
现在我们进一步设$i$类型顶角中有$b_i$个玻色子线,$f_i$个费米子线,并设有$E_\text{F}$条费米子外线,$E_\text{B}$条玻色子外线,则
\[
    E_\text{F} + 2 I_\text{F} = \sum_i n_i f_i, \quad E_\text{B} + 2 I_\text{B} = \sum_i n_i b_i,
\]
因为一条内线连接两个顶角。这样就有
\begin{equation}
    D(\Gamma) = \sum_i n_i \left( d_i + b_i + \frac{3}{2} f_i - 4  \right) + 4 - E_\text{B} - \frac{3}{2} E_\text{F}.
\end{equation}

表观发散度实际上就是在做量纲分析,而且做的是朴素的工程量纲分析,因此是不尽然可靠的。
大体上说,如果表观发散度大于零,那么这张图发散,如果小于零,那么这张图收敛,如果等于零,那么这张图应该对数发散,但是这只是一个非常粗略,可能不准确的估计。
不过,\concept{Weinberg power counting theorem}保证了,当且仅当一张图及其子图的表观发散度都是小于零,它收敛。

为了保证尽可能多的图的表观发散度小于等于零,我们会要求
\begin{equation}
    d_i + b_i + \frac{3}{2} f_i - 4 \leq 0,
\end{equation}
因此比较安全的安排是,一个顶角最多有四条玻色子线、两条费米子线,否则有可能产生无穷多种发散的图,理论可能不能重整化。

\subsection{维数正规化}

注:经常用Wick转动来化简此处的积分,但是由于$p^0 > 0$时极点在下半平面而$p^0 < 0$时极点在上半平面,为了避免撞上奇点,Wick转动应该将积分路径顺时针旋转\SI{90}{\degree},从而可以设$l = \ii l^\text{E}$。
TODO:这和我们后面做的将整个理论做Wick转动时用的记号似乎不一样?

将内线动量从$4$维扩充为$n$维度,外线动量保持不变。

让维数变化时,没有必要让$\gamma$矩阵的维数发生变化,因此对$\gamma$矩阵的乘积的迹计算无需做任何调整,即自旋指标不需要做任何调整,而对坐标指标($\mu$这种)的迹计算(如$\gamma^\mu \gamma_\mu$)则需要调整。

\subsection{骨架图的修正}

重整化条件:
\begin{itemize}
    \item 在修正后的单粒子格林函数的极点处,有质量粒子的自能修正对$p^2$(玻色子)或者$\slashed{p}$(费米子)的一阶导数为零;无质量粒子的自能修正为零。
    这是为了确保没有场强重整化。
    动量远离单粒子格林函数极点时它们当然可以不是零;本应如此,否则圈图修正无法体现。
    \item 在修正后的单粒子格林函数的极点处,极点给出的质量(通过$p^2=m^2$解出)就是我们设定的物理质量;在显式引入自能修正时,有质量粒子的自能修正为零。
    这是为了确保质量的修正为零。
    \item 顶角函数和物理相互作用强度相同,这是为了确保顶角修正为零。
    具体什么是“物理相互作用强度”取决于探测方式,如量子电动力学中通常是使用静电学方法测定电磁相互作用的强度,于是我们要求顶角函数在光子动量为零时和静电学方法测得的电磁相互作用强度(其实就是元电荷)相同。
    其它时候顶角函数可以有偏离,以展现高阶过程的修正。
\end{itemize}

