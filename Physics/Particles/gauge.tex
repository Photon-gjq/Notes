\documentclass[hyperref, UTF8, a4paper]{ctexart}

\usepackage{geometry}
\usepackage{titling}
\usepackage{titlesec}
\usepackage{paralist}
\usepackage{footnote}
\usepackage{enumerate}
\usepackage{amsmath, amssymb, amsthm}
\usepackage{simplewick}
\usepackage{cite}
\usepackage{graphicx}
\usepackage{subfigure}
\usepackage{physics}
\usepackage{centernot}
\usepackage{tikz}
\usepackage[colorlinks, linkcolor=black, anchorcolor=black, citecolor=black]{hyperref}
\usepackage{prettyref}

\geometry{left=3.18cm,right=3.18cm,top=2.54cm,bottom=2.54cm}
\titlespacing{\paragraph}{0pt}{1pt}{10pt}[20pt]
\setlength{\droptitle}{-5em}
\preauthor{\vspace{-10pt}\begin{center}}
\postauthor{\par\end{center}}

\DeclareMathOperator{\timeorder}{T}
\DeclareMathOperator{\diag}{diag}
\newcommand*{\ii}{\mathrm{i}}
\newcommand*{\ee}{\mathrm{e}}
\newcommand*{\const}{\mathrm{const}}
\newcommand*{\comment}{\paragraph{注记}}
\newcommand{\fsl}[1]{{\centernot{#1}}}
\newcommand*{\reals}{\mathbb{R}}
\newcommand*{\complexes}{\mathbb{C}}

\newrefformat{sec}{第\ref{#1}节}
\newrefformat{note}{注\ref{#1}}
\renewcommand{\autoref}{\prettyref}

\newenvironment{bigcase}{\left\{\quad\begin{aligned}}{\end{aligned}\right.}

\newcommand{\concept}[1]{\underline{\textbf{#1}}}
\renewcommand{\emph}{\textbf}

\allowdisplaybreaks[4]

\title{规范场论}
\author{吴何友}

\begin{document}

\maketitle

\section{规范对称性和规范场}

物理学中的对称性通常包括时空对称性(即将物理事件的时空坐标做一个变换,一般来说,是洛伦兹变换)和内部对称性(即某个参数空间中的变换,通常是各点上场的变换)。
\concept{规范对称性}指的则是变换参数依赖场和物理量的局域时空坐标的对称性,即与定域的变换相关的对称性。
电动力学的Maxwell理论就具有规范对称性。

对二维系统,共形变换是局域变换群,共性不变的场论也是一种规范理论,称为共形场论。
高维共性不变性是整体的,没有局域不变性,从而也不是规范理论。

\subsection{规范场论的拉氏量}

\begin{equation}
    \mathcal{L}
\end{equation}

\section{规范场的量子化}

规范场在正则量子化之下是非常难以处理的。规范冗余性的存在会让最为

\section{Abel规范理论示例:$U(1)$规范对称性导致的电动力学理论}

\begin{equation}
    \mathcal{L} = - \frac{1}{4} F_{\mu \nu} F^{\mu \nu} - \bar{\psi} 
\end{equation}

\end{document}