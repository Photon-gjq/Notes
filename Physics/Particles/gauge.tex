\documentclass[hyperref, UTF8, a4paper]{ctexart}

\usepackage{geometry}
\usepackage{titling}
\usepackage{titlesec}
\usepackage{paralist}
\usepackage{footnote}
\usepackage{enumerate}
\usepackage{amsmath, amssymb, amsthm}
\usepackage{simplewick}
\usepackage{cite}
\usepackage{graphicx}
\usepackage{subfigure}
\usepackage{physics}
\usepackage{tikz-feynhand}
\usepackage{centernot}
\usepackage{slashed}
\usepackage{tikz}
\usepackage[colorlinks, linkcolor=black, anchorcolor=black, citecolor=black]{hyperref}
\usepackage{prettyref}

\geometry{left=3.18cm,right=3.18cm,top=2.54cm,bottom=2.54cm}
\titlespacing{\paragraph}{0pt}{1pt}{10pt}[20pt]
\setlength{\droptitle}{-5em}
\preauthor{\vspace{-10pt}\begin{center}}
\postauthor{\par\end{center}}

\DeclareMathOperator{\timeorder}{T}
\DeclareMathOperator{\diag}{diag}
\newcommand*{\ii}{\mathrm{i}}
\newcommand*{\ee}{\mathrm{e}}
\newcommand*{\const}{\mathrm{const}}
\newcommand*{\comment}{\paragraph{注记}}
\newcommand{\fsl}[1]{{\centernot{#1}}}
\newcommand*{\reals}{\mathbb{R}}
\newcommand*{\complexes}{\mathbb{C}}
\newcommand*{\fd}[1]{{\mathcal{D} #1}}

\newrefformat{sec}{第\ref{#1}节}
\newrefformat{note}{注\ref{#1}}
\renewcommand{\autoref}{\prettyref}

\newenvironment{bigcase}{\left\{\quad\begin{aligned}}{\end{aligned}\right.}

\newcommand{\concept}[1]{\underline{\textbf{#1}}}
\renewcommand{\emph}{\textbf}

\tikzfeynhandset{
    every boldfermion@@/.style={
    /tikz/draw=none,
    /tikz/decoration={name=none},
    /tikz/postaction={
            /tikz/draw,
            /tikz/double,
            /tikz/line width = \feynhandlinesize,
            /tikzfeynhand/with arrow=0.5,
        },
    },
    every boldfermion/.style={/tikzfeynhand/every boldfermion@@/.append style={#1}},
    boldfermion/.style={
    /tikzfeynhand/every boldfermion@@,
    }
}

\allowdisplaybreaks[4]

\title{规范场论}
\author{吴晋渊}

\begin{document}

\maketitle

物理学中的对称性通常包括时空对称性(即将物理事件的时空坐标做一个变换,一般来说,是洛伦兹变换)和内部对称性(即某个参数空间中的变换,通常是各点上场的变换)。
\concept{规范对称性}指的则是变换参数依赖场和物理量的局域时空坐标的对称性,即与定域的变换相关的对称性。
实际上,规范对称性的要求足够确定系统中各个场的相互作用方式!这一事实——即所谓\concept{规范原理}——是量子场论历史上所谓“改变人心的转换”,它被系统应用之前,各个场的相互作用基本上只能唯象确定,它被系统应用之后,只需要写出规范群(即局域对称性的对称群)就能够确定相互作用。

本文将首先介绍电动力学,分析其性质,然后通过考虑其自然推广而得到杨-米尔斯理论。

我们将经常用到李代数。可以采用下面的约定: % TODO
\begin{equation}
    [T^a, T^b] = \ii f^{abc} T^c.
\end{equation}
\begin{equation}
    \trace{(T^a T^b)} = \frac{1}{2} \delta^{ab}, \quad \trace{T^a} = 0.
\end{equation}

\section{电动力学}

\subsection{规范场和狄拉克旋量场的最小耦合}

在相对论性量子场论中——也即,在以闵可夫斯基时空为底流形的量子场论中,我们尝试将一个自旋$1/2$的狄拉克旋量场和一个无质量矢量场耦合起来。
旋量场的拉氏量为
\begin{equation}
    \mathcal{L}_\text{spin} = - m \bar{\psi} \psi + \ii \bar{\psi} \gamma_\mu \partial^\mu \psi,
    \label{eq:spin-lagrangian}
\end{equation}
而矢量场的拉氏量为
\begin{equation}
    \mathcal{L}_\text{vec} = - \frac{1}{2} (\partial^\mu A^\nu \partial_\mu A_\nu - \partial^\mu A^\nu \partial_\nu A_\mu).
    \label{eq:vec-lagrangian}
\end{equation}
很容易看出\eqref{eq:spin-lagrangian}具有全局$U(1)$对称性:它在变换
\[
    \psi \longrightarrow \psi' = \psi \ee^{\ii \alpha}
\]
下保持不变。同样,\eqref{eq:vec-lagrangian}具有场的全局平移不变性(自由无质量矢量场的规范对称性),它在变换
\[
    A^\mu \longrightarrow A'^\mu = A^\mu + a^\mu
\]
下保持不变。这两个对称性都是全局的:如果$\alpha$或$a^\mu$依赖于坐标,由于导数的链式法则,会多出来一些项。
具体来说,我们有
\begin{equation}
    \mathcal{L}_\text{spin} \longrightarrow \mathcal{L}_\text{spin}' = \mathcal{L}_\text{spin} - \bar{\psi} \gamma_\mu \psi \partial^\mu \alpha.
    \label{eq:psi-change}
\end{equation}
对于矢量场,在$a^\mu$的形式任意的情况下,$\mathcal{L}_\text{vec}$的变换无规律可循,但是如果我们用某个标量的梯度$\partial^\mu a$代替$a^\mu$,那么有
\[
    A^\mu \longrightarrow A'^\mu = A^\mu + \partial^\mu a, \quad
    \mathcal{L}_\text{vec} \longrightarrow \mathcal{L}_\text{vec}' = \mathcal{L}_\text{vec}.
\]
也就是说,矢量场的场的平移对称性实际上可以稍加推广而仍然成立。

$\psi$的变换的相位因子是一个标量;$A^\mu$的场的平移量也是一个标量的梯度。很容易想到的尝试是,我们是否可以将两个场耦合起来,并要求整个系统在局域变换($e$是常数而$a(\vb*{x})$依赖于坐标)
\begin{equation}
    \psi \longrightarrow \psi' = \psi \ee^{\ii e a}, \quad A^\mu \longrightarrow A'^\mu = A^\mu + \partial^\mu a
    \label{eq:gauge-transformation}
\end{equation}
下保持不变?%
\footnote{早期的物理学家会认为,一个变换应该是物理上可行的,因此它不应该是全局的,而\emph{只能}是局域的(例如我们可以让$a$在很小的范围内才不为零)。
但是实际上这种观点是错误的——使得我们想要从头写下一个拉氏量的原因实际上是我们希望从一个非常简洁的源头推导出麦克斯韦方程,但是后面会看到,在经典情况下描述了一切电磁现象的麦克斯韦方程本身在$U(1)$规范变换下不变,这意味着\eqref{eq:gauge-transformation}展示的对称性实际上是一种冗余,即系统中存在非物理、可以略去的自由度。
这些自由度不参与和实际观测值有关的任何相互作用,作用在它们上面的变换完全没有必要是局域的。
我们要求系统的动力学在局域$U(1)$变换下不变,归根到底还是满足实验观测结论的需要。
}%
自由矢量场部分肯定是不变的,那么就要适当设计相互作用项的形式,把$\psi$做局域$U(1)$变换之后拉氏量多出来的一项吸收掉。
当然,如果相互作用项是$- e A^\mu \bar{\psi} \gamma_\mu \psi$,那就正好,因为
\[
    - e A^\mu \bar{\psi} \gamma_\mu \psi - \bar{\psi} \psi \gamma_\mu \partial^\mu (e a) = - e A'^\mu \bar{\psi'} \gamma_\mu \psi'.
\]
于是我们得出结论:拉氏量
\begin{equation}
    \mathcal{L} = 
    \underbrace{- m \bar{\psi} \psi + \ii \bar{\psi} \gamma_\mu \partial^\mu \psi }_{\mathcal{L}_\text{spin}}
    \underbrace{- \frac{1}{2} (\partial^\mu A^\nu \partial_\mu A_\nu - \partial^\mu A^\nu \partial_\nu A_\mu)}_{\mathcal{L}_\text{vec}}
    \underbrace{- e A^\mu \bar{\psi} \gamma_\mu \psi}_\text{interaction}
    \label{eq:qed-lagrangian}
\end{equation}
具有局域$U(1)$不变性。推导出\eqref{eq:qed-lagrangian}的方法就是\concept{最小耦合}。%
\footnote{需要注意的是最小耦合实际上并不是唯一的能够让理论满足局域$U(1)$对称性的方案。理论中有哪些规范场、相互作用的形式如何,归根到底都需要实验上的提示。
例如,电磁理论中最小耦合适用是因为它能够导出麦克斯韦方程,而麦克斯韦方程是已经验证了的在经典情况下正确的电磁定律。}%

于是,我们得出结论:一个自旋$1/2$狄拉克旋量场和一个无质量矢量场耦合,并要求理论具有\emph{局域}$U(1)$对称性,那么就会得到
我们将会看到,这个理论实际上就是电动力学。

还可以引入一些记号来简化\eqref{eq:qed-lagrangian}。首先引入反对称张量\concept{电磁张量}
\begin{equation}
    F^{\mu \nu} = \partial^\mu A^\nu - \partial^\nu A^\mu,
\end{equation}
则自由矢量场拉氏量为
\[
    \mathcal{L}_\text{vec} = - \frac{1}{4} F_{\mu \nu} F^{\mu \nu}.
\]
另一方面,相互作用项和含有旋量场的导数的项形式非常接近,因此可以定义\concept{协变导数}
\begin{equation}
    \ii D^\mu = \ii \partial^\mu - e A^\mu,
\end{equation}
最后将\eqref{eq:qed-lagrangian}写成
\begin{equation}
    \begin{aligned}
        \mathcal{L} &= \bar{\psi} (\ii \gamma^\mu D_\mu - m) \psi - \frac{1}{4} F_{\mu \nu} F^{\mu \nu} \\
        &= \bar{\psi} (\ii \slashed{D} - m) \psi - \frac{1}{4} F_{\mu \nu} F^{\mu \nu}. 
    \end{aligned}
    \label{eq:short-qed-lagrangian}
\end{equation}
这里我们用斜杠记号表示$\gamma_\mu A^\mu$。

\subsection{运动方程和守恒量}\label{sec:four-eqs}

从\eqref{eq:qed-lagrangian}马上可以使用欧拉-拉格朗日方程写出运动方程。对$\psi$我们有
\[
    \ii \partial_\mu \bar{\psi} \gamma^\mu + e \bar{\psi} \gamma_\mu A^\mu + m \bar{\psi} = 0,
\]
对其取共轭,或者对$\bar{\psi}$应用欧拉-拉格朗日方程,就得到
\begin{equation}
    \ii \gamma^\mu \partial_\mu \psi - m \psi = e \gamma_\mu A^\mu \psi.
    \label{eq:movement-eq-1}
\end{equation}
对$A^\mu$应用欧拉-拉格朗日方程,则有
\begin{equation}
    \partial_\mu F^{\mu \nu} = \partial_\mu (\partial^\mu A^\nu - \partial^\nu A^\mu) = e \bar{\psi} \gamma^\nu \psi.
    \label{eq:movement-eq-2}
\end{equation}
以上两个方程给出了\eqref{eq:qed-lagrangian}的运动方程。(如前所述,$\psi$和$\bar{\psi}$虽然是独立的场,但它们的运动方程并不独立,因为运动方程是一阶的)

现在我们分析局域$U(1)$对称性带来的守恒量。在局域$U(1)$变换下,我们有
\[
    \var{\psi} = \ii e \psi \var{a}, \quad \var{A^\mu} = \partial^\mu \var{a},
\]
则守恒流为
\[
    \begin{aligned}
        J^\mu \var{a} &= - e \bar{\psi} \gamma^\mu \psi \var{a} + (-\partial^\mu A^\nu + \partial^\nu A^\mu) \partial_\nu \var{a} \\
        &= - e \bar{\psi} \gamma^\mu \psi \var{a} - \partial_\nu (\partial^\nu A^\mu - \partial^\mu A^\nu) \var{a},
    \end{aligned}
\]
第二个等号实际上是忽略了一个边界项后得到的结果。%
\footnote{考虑到$\var{a}$在每一点都可以独立地变化,$\int A\var{a} = \int B \var{a}$意味着$A=B$。}%
无论如何,这个守恒流的第二项是平凡的,因为它就是电磁张量的一个指标求散度之后的结果,它的散度当然是零。
那么,我们就有以下守恒荷:
\begin{equation}
    J^\mu = e \bar{\psi} \gamma^\mu \psi , \quad \partial_\mu J^\mu = 0.
    \label{eq:four-current}
\end{equation}
回过头看,实际上这是\emph{全局$U(1)$对称性}的守恒荷——全局$U(1)$对称性中$a$在时空上是均匀的,那么$\partial_\mu \var{a}$就是零,正好让含有$A$的那个平凡的项消失。
实际上从\eqref{eq:movement-eq-2}中我们也可以得到这个守恒流。由于电磁张量是反对称的,我们有:
\[
    0 = \partial_\mu \partial_\nu F^{\mu \nu} = \partial_\mu (e \bar{\psi} \gamma^\mu \psi).
\]
这就导出了\eqref{eq:four-current}。
使用\eqref{eq:four-current}可以将\eqref{eq:movement-eq-2}写成
\begin{equation}
    \partial_\mu F^{\mu \nu} = J^\nu.
    \label{eq:four-maxwell}
\end{equation}

我们最后评论一下以上守恒量和运动方程的意义。\eqref{eq:four-maxwell}实际上就是麦克斯韦方程的一部分。构成麦克斯韦方程另外一部分的是以下恒等式
\begin{equation}
    \partial_\mu F_{\nu \rho} + \partial_\nu F_{\rho \mu} + \partial_\rho F_{\mu \nu} = 0,
    \label{eq:bianchi-identity}
\end{equation}
它是$F_{\mu \nu}$定义为$A^\mu$的梯度的反对称化导致的结果。%
$J^\mu$给出了麦克斯韦方程中的$\rho$和$\vb*{j}$,即它正是\concept{电荷}的守恒流,电荷是全局$U(1)$对称性对应的守恒荷。
\eqref{eq:four-current}就是$e$乘以粒子数密度,因此$e$就是$\psi$场激发的粒子携带的电荷量。对电子,它是$e = -\abs*{e}$,$\abs*{e}$是元电荷。

\subsection{规范}\label{sec:gauge-def}

电动力学在局域$U(1)$变换下的对称性实际上是一个\emph{规范对称性},也就是说,做任意的局域$U(1)$变换,不会有任何可以观察到的变化,也就是说\eqref{eq:qed-lagrangian}中实际上有多余的自由度。
我们需要对$A$和$\psi$施加适当的约束,以确保满足这个约束的$A$和$\psi$取值可以覆盖所有物理上可能产生的状态,同时不含有任何冗余的自由度,也即要\emph{选取一个规范}。
既然局域$U(1)$对称性是规范对称性,任何反映系统实际状态的物理量都应该在$U(1)$规范变换下不变。

形式最漂亮的应该是\concept{洛伦兹规范},也就是
\begin{equation}
    \partial_\mu A^\mu = 0,
\end{equation}
在这个规范下\eqref{eq:four-maxwell}转化为
\begin{equation}
    \Box^2 A = J, \quad \Box^2 = \partial_\mu \partial^\mu.
\end{equation}
我们得到了一个四维波动方程,当然,这就是\concept{电磁波}。
一个很自然的问题是,洛伦兹规范是否不失一般性?是否存在一组$A^\mu$不能够通过一个规范变换变换为一组满足洛伦兹规范的$A'^\mu$?
实际上洛伦兹规范确实是不失一般性的,因为波动方程的性质很良好,对一个给定的标量场$C$,总是可以找到一个标量场$a$,使得
\[
    \partial_\mu \partial^\mu a = C,
\]
这样不论原本$A^\mu$取什么值,只需要解出一个$a^\mu$使得
\[
    \partial_\mu \partial^\mu a = \partial_\mu A^\mu,
\]
然后做规范变换
\[
    A'^\mu = A^\mu - \partial^\mu a, \quad \psi' = \psi \ee^{-\ii e a},
\]
得到的$A'^\mu$就是服从洛伦兹规范的——并且表示和$A^\mu$完全一样的物理状态。
因此洛伦兹规范确实是不失一般性的。

容易看出$F^{\mu \nu}$是一个规范不变量。实际上,在选定了规范之后,可以从它恢复出$A^\mu$。
不失一般性地选择洛伦兹规范,则我们有
\[
    \partial_\mu F^{\mu \nu} = \partial_\mu \partial^\mu A^\nu - \partial^\nu \partial_\mu A^\mu = \partial_\mu \partial^\mu A^\nu,
\]
由于波动方程的良好性质,我们就从上式反解出$A^\mu$了。
如果是别的规范,就按照它转换到洛伦兹规范的方式,从洛伦兹规范转换到原有规范即可。
总之,原则上任何规范不变量都可以通过$F^{\mu \nu}$求导、积分等得到。

\section{量子电动力学}

本节我们讨论量子化之后的电动力学,即\concept{量子电动力学},或者简称\concept{QED}。

\subsection{可重整性}

\subsubsection{Ward-Takahashi恒等式}

考虑仅仅对$\psi$和$\bar{\psi}$做局域$U(1)$变换而不对矢量场做变换的一个情况。
此时根据\eqref{eq:psi-change},有
\begin{equation}
    \var{\psi} = \ii e \alpha \psi, \quad \var{\bar{\psi}} = - \ii e \alpha \bar{\psi}, \quad \var{\mathcal{L}} = - e \bar{\psi} \gamma^\mu \psi \partial_\mu \alpha.
    \label{eq:psi-change-only}
\end{equation}
这不是一个经典意义下的对称操作,因为作用量会发生变化,但是我们知道量子情况下只要保持路径积分测度不变,有时仍然能够得到类似于诺特定理的结果。
此外如果没有适当的规范对称性,也无法写出如此漂亮的“拉氏量的变化就是$\partial_\mu \alpha$乘以电流”的变换。
因此,应当记住\eqref{eq:psi-change-only}实际上是规范对称性的产物。
Ward-Takahashi恒等式是量子版本的诺特定理,而这里我们要导出它关于某些散射振幅的特定形式。
对关联函数$\mel{\Omega}{T \psi(x_1) \bar{\psi}(x_2)}{\Omega}$做以上变换,会发现
\[
    \begin{aligned}
        0 &= \int \fd{\psi} \fd{\bar{\psi}} \fd{A} \ee^{\ii S} \big( - \ii (\int \dd[4]{x} \partial_\mu \alpha(x)) (j^\mu(x) \psi(x_1) \bar{\psi}(x_2) ) \\
        & \quad \quad \quad + (\ii e \alpha(x_1) \psi(x_1)) \bar{\psi}(x_2) + \psi(x_1)(- \ii e \alpha(x_2) \bar{\psi}(x_2)) \big).
    \end{aligned}
\]
其中$j^\mu$就是$e \bar{\psi} \gamma^\mu \psi$。对第一项做分部积分,并且将后两项写成$\delta$函数的形式,就得到
\begin{equation}
    \partial_\mu \mel{\Omega}{T j^\mu(x) \psi(x_1) \bar{\psi}(x_2)}{\Omega} = - e \delta(x - x_1) \mel{\Omega}{T \psi(x_1) \bar{\psi}(x_2)}{\Omega} + e \delta(x - x_2) \mel{\Omega}{T \psi(x_1) \bar{\psi}(x_2)}{\Omega}.
\end{equation}
这就是量子版本的电荷守恒方程。

现在我们做傅里叶变换
\[
    \int \dd[4]{x} \ee^{-\ii k \cdot x} \int \dd[4]{x_1} \ee^{\ii q \cdot x_1} \int \dd[4]{x_2} \ee^{- \ii p \cdot x_2} ,
\]
等式右边没有太多可说,等式左边的$\partial_\mu j^\mu(x)$变成了
\[
    \int \dd[4]{x} \ee^{-\ii k \cdot x} \partial_\mu j^\mu(x) = \int \frac{\dd[4]{k_1}}{(2\pi)^4} \ii k_\mu e \bar{\psi}(k_1 + k) \gamma^\mu \psi(k_1).
\]
值得注意的是,组成$j^\mu(x)$的两个费米子算符现在都具有不确定的动量,即它们对应内线或者说传播子。
此外应当注意$-\ii e \gamma^\mu$正是光子-电子相互作用顶角,因此上式在去掉$k_\mu$之后乘上因子$\epsilon_\mu$就得到子图“一个光子入射,打出一对电子和正电子”。
这样我们就有
\[
    \begin{aligned}
        &\quad \int \dd[4]{x} \ee^{-\ii k \cdot x} \int \dd[4]{x_1} \ee^{\ii q \cdot x_1} \int \dd[4]{x_2} \ee^{- \ii p \cdot x_2} \partial_\mu \mel{\Omega}{T j^\mu(x) \psi(x_1) \bar{\psi}(x_2)}{\Omega} \\
        &= - k_\mu \begin{gathered}
            \begin{tikzpicture}
                \begin{feynhand}
                    \vertex (a) at (-1.5, 0) {$\mu$};
                    \vertex [grayblob] (b) at (0, 0) {};
                    \vertex (c) at (0, 1.5);
                    \vertex (d) at (0, -1.5);
                    \propag [photon, mom={$k$}] (a) to (b); 
                    \propag [fermion] (b) to [edge label={$q$}] (c);
                    \propag [fermion] (d) to [edge label={$p$}] (b);
                \end{feynhand}
            \end{tikzpicture}
        \end{gathered}.
    \end{aligned}
\]
这里有一个微妙的地方:费曼图实际上就是微扰计算某种“格林矩阵”或是“跃迁矩阵”的矩阵元的图形,其外线就好像矩阵元的指标;通常,外线要么代表$S$矩阵的入射态和出射态(此时外线没有传播子),要么代表关联函数中出现的场(此时外线有传播子,并且通常会有自能修正)。
然而,上式中的图形却混合了这两种外线:电子线是关联函数中的外线,有传播子,但光子线是(给定偏振,从而不和$\epsilon_\mu$点乘的)$S$矩阵的外线。
很容易验证,上式左边没有光子传播子,那么右边当然也没有。
实际上,下图
\[
    \begin{tikzpicture}
        \begin{feynhand}
            \vertex [crossdot] (a) at (-3.5, 0) {};
            \vertex (e) at (-2, 0);
            \vertex [grayblob] (b) at (0, 0) {};
            \vertex (c) at (0, 1.5);
            \vertex (d) at (0, -1.5);
            \propag [photon, mom={$k, \mu$}] (a) to (e); 
            \propag [fermion] (b) to [in=60, out=150, looseness=1.5] (e);
            \propag [fermion] (e) to [in=210, out=300, looseness=1.5] (b);
            \propag [fermion] (b) to [edge label={$q$}] (c);
            \propag [fermion] (d) to [edge label={$p$}] (b);
        \end{feynhand}
    \end{tikzpicture}
\]
是更加合理的画法,光子线与$\otimes$相连代表它没有传播子这一事实,虽然一些教科书(如Peskin并没有采取这种画法)。
所以最后我们得到
\begin{equation}
    - k_\mu \times \begin{gathered}
        \begin{tikzpicture}
            \begin{feynhand}
                \vertex [crossdot] (a) at (-3.5, 0) {};
                \vertex (e) at (-2, 0);
                \vertex [grayblob] (b) at (0, 0) {};
                \vertex (c) at (0, 1.5);
                \vertex (d) at (0, -1.5);
                \propag [photon, mom={$k, \mu$}] (a) to (e); 
                \propag [fermion] (b) to [in=60, out=150, looseness=1.5] (e);
                \propag [fermion] (e) to [in=210, out=300, looseness=1.5] (b);
                \propag [fermion] (b) to [edge label={$q$}] (c);
                \propag [fermion] (d) to [edge label={$p$}] (b);
            \end{feynhand}
        \end{tikzpicture}
    \end{gathered} 
    = - e \begin{gathered}
        \begin{tikzpicture}
            \begin{feynhand}
                \vertex [grayblob] (b) at (0, 0) {};
                \vertex (c) at (0, 1.5);
                \vertex (d) at (0, -1.5);
                \propag [fermion] (b) to [edge label={$q-k$}] (c);
                \propag [fermion] (d) to [edge label={$p$}] (b);
            \end{feynhand}
        \end{tikzpicture}
    \end{gathered}
    \quad + \quad e \begin{gathered}
        \begin{tikzpicture}
            \begin{feynhand}
                \vertex [grayblob] (b) at (0, 0) {};
                \vertex (c) at (0, 1.5);
                \vertex (d) at (0, -1.5);
                \propag [fermion] (b) to [edge label={$q$}] (c);
                \propag [fermion] (d) to [edge label={$p+k$}] (b);
            \end{feynhand}
        \end{tikzpicture}
    \end{gathered}\ .
    \label{eq:ward-takahashi-qed}
\end{equation}
这就是量子电动力学中通常所说的\concept{Ward-Takahashi恒等式}。
实际上,由一般的Ward-Takahashi恒等式(即“量子诺特定理”)可以看出,在变换\eqref{eq:psi-change-only}下,我们在上式两边的关联函数中添加更多的场(即加入更多入射和出射电子线),上式也是满足的。
注意以上公式中从空间做傅里叶变换得到的动量空间的关联函数并不是在壳的,且\eqref{eq:ward-takahashi-qed}中的$e$实际上是裸参数$e_0$。

\eqref{eq:ward-takahashi-qed}看起来非常不自然,但应当注意到它的左边包括两个电子传播子而右边有一个电子传播子,从而在左边留下因子$Z_2$;此外,它显然含有顶角函数,即含有$Z_1$。
光子没有传播子,但是仍然有自能修正(因为项链图还是存在的;在算$S$矩阵矩阵元时无需考虑项链图,因为只需要算amputated的图,在算关联函数时项链图自动被纳入了光子传播子的自能修正;这里没有光子传播子,但是项链图还在)。
但是,在$k \to 0$时光子不再有自能修正,因为电子有质量,一个动量几乎为零的光子无法激发出正负电子对,因此可以忽略因子$Z_3$。
因此,等式\eqref{eq:ward-takahashi-qed}意味着在量子电动力学中$Z_1$和$Z_2$之间有某种约束关系。
在$k \to 0$时,由重整化条件,顶角函数退化为“物理电荷”乘以$\gamma^\mu$。
因此\eqref{eq:ward-takahashi-qed}左边在$k \to 0$时可以写成
\[
    \begin{aligned}
        &\quad - k_\mu \times \begin{gathered}
            \begin{tikzpicture}
                \begin{feynhand}
                    \vertex [crossdot] (a) at (-3.5, 0) {};
                    \vertex (e) at (-2, 0);
                    \vertex [grayblob] (b) at (0, 0) {};
                    \vertex (c) at (0, 1.5);
                    \vertex (d) at (0, -1.5);
                    \propag [photon, mom={$k, \mu$}] (a) to (e); 
                    \propag [fermion] (b) to [in=60, out=150, looseness=1.5] (e);
                    \propag [fermion] (e) to [in=210, out=300, looseness=1.5] (b);
                    % TODO:粗线表示自能修正
                    \propag [fermion] (b) to [edge label={$q$}] (c);
                    \propag [fermion] (d) to [edge label={$p$}] (b);
                \end{feynhand}
            \end{tikzpicture}
        \end{gathered} 
        = - k_\mu \times \begin{gathered}
            \begin{tikzpicture}
                \begin{feynhand}
                    \vertex [crossdot] (a) at (-1.5, 0) {};
                    \vertex [grayblob] (b) at (0, 0) {$-\ii e \Gamma$};
                    \vertex (c) at (0, 1.5);
                    \vertex (d) at (0, -1.5);
                    \propag [photon, mom={$k, \mu$}] (a) to (b); 
                    \propag [boldfermion] (b) to [edge label={$q$}] (c);
                    \propag [boldfermion] (d) to [edge label={$p$}] (b);
                \end{feynhand}
            \end{tikzpicture}
        \end{gathered} \\
        & = 
        - k_\mu \frac{\ii Z_2}{\slashed{p} - m + \ii 0^+} \frac{\ii Z_2}{\slashed{q} - m + \ii 0^+} (- \ii e \gamma^\mu) (2\pi)^4 \delta^4(q-k-p), \quad \text{as $k \to 0$} .
    \end{aligned}
\]
$\sqrt{Z_3}$因子就是$1$,$e$是重整化之后的电荷。
另一方面,\eqref{eq:ward-takahashi-qed}的右边则是(请注意推导\eqref{eq:ward-takahashi-qed}时用的电荷就是裸电荷,这里记作$e_0$)
\[
    - e_0 \begin{gathered}
        \begin{tikzpicture}
            \begin{feynhand}
                \vertex [grayblob] (b) at (0, 0) {};
                \vertex (c) at (0, 1.5);
                \vertex (d) at (0, -1.5);
                \propag [fermion] (b) to [edge label={$q-k$}] (c);
                \propag [fermion] (d) to [edge label={$p$}] (b);
            \end{feynhand}
        \end{tikzpicture}
    \end{gathered}
    \quad + \quad e_0 \begin{gathered}
        \begin{tikzpicture}
            \begin{feynhand}
                \vertex [grayblob] (b) at (0, 0) {};
                \vertex (c) at (0, 1.5);
                \vertex (d) at (0, -1.5);
                \propag [fermion] (b) to [edge label={$q$}] (c);
                \propag [fermion] (d) to [edge label={$p+k$}] (b);
            \end{feynhand}
        \end{tikzpicture}
    \end{gathered} 
    = - e_0 (2\pi)^4 \delta(q-k-p) \left( \frac{\ii Z_2}{\slashed{p} - m + \ii 0^+} - \frac{\ii Z_2}{\slashed{p} + \slashed{k} - m + \ii 0^+} \right).
\]
考虑到电荷的重整化为$e_0 Z_2 Z_3^{1/2} = e Z_1$而$Z_3$在$k \to 0$时为$1$,有
\[
    - k_\mu \frac{\ii Z_2}{\slashed{p} - m + \ii 0^+} \frac{\ii Z_2}{\slashed{p} + \slashed{k} - m + \ii 0^+} (- \ii e_0 Z_2 Z_1^{-1} \gamma^\mu) = - e_0 \left( \frac{\ii Z_2}{\slashed{p} - m + \ii 0^+} - \frac{\ii Z_2}{\slashed{p} + \slashed{k} - m + \ii 0^+} \right),
\]
从而最终得到
\begin{equation}
    Z_1 = Z_2.
\end{equation}
这个结论是严格成立的;换句话说,计算无穷阶微扰之后,重整化因子$Z_1$和$Z_2$一定是一样的。

\eqref{eq:ward-takahashi-qed}还有另一个用处。\eqref{eq:ward-takahashi-qed}中的电子外线全部是关联函数中的外线,带有传播子,四维动量可以不在壳。
然而,如果我们考虑四维动量在壳的那些情况,那么\eqref{eq:ward-takahashi-qed}右边的两个关联函数中均有四维动量是离壳的(例如容易验证,如果$q$和$k$在壳那么$q-k$肯定是离壳的),因此这些关联函数对$S$矩阵没有贡献。
因此,对那些所有入射和出射外线——无论是电子还是光子——都是$S$矩阵型外线的单光子图——其实就是单光子过程的$\mathcal{M}$——我们有
\begin{equation}
    k_\mu \mathcal{M}^\mu = 0,
\end{equation}
其中$\mathcal{M} = \epsilon_\mu \mathcal{M}^\mu$。这称为\concept{Ward恒等式}。这是Ward-Takahashi恒等式的在壳情况。

总之,规范对称性、所之而来的电荷守恒、Ward恒等式基本上具有同样的来源。

\section{杨-米尔斯理论}

\subsection{杨-米尔斯理论的拉氏量}

\subsubsection{规范场的引入和协变导数}

电动力学的Maxwell理论是一个比较简单的规范理论,其中旋量场可以做任意的局域$U(1)$变换
\[
    \Omega(x) = \ee^{\ii a(x)},
\]
为了让拉氏量在此变换下保持不变,一个额外的矢量场被耦合到旋量场上,当旋量场做$U(1)$变换时矢量场的场值发生一个平移,从而拉氏量在局域$U(1)$变换下不变。
$U(1)$群的特性决定了系统中存在矢量场这一事实,并且决定了相互作用的形式。
以一种系统性的方式决定一个理论中应该有什么场,以及相互作用应该取什么形式,显然是非常有吸引力的。
因此,非常自然地,我们希望用一个更加复杂的李群做规范变换,并且开发一套看着一个李群就能够写下一个具有规范对称性的理论的方式。
对$U(1)$规范理论的推广看起来似乎有非常多可能的选择,但所幸我们已经有一个成熟的、和微分几何紧密相关的、已经在描述强相互作用和弱相互作用的方面大获成功的理论框架:\concept{杨-米尔斯理论}。

下面我们将用一种启发式的方法去导出杨-米尔斯理论,主要是通过模仿电动力学中的概念。
我们只讨论紧致的李群$G$,此时其一定具有幺正表示,这正是我们想要的。
对标量场,看起来唯一能够作用在其上的操作就是乘以一个复因子。
因此如果我们想要让$G$有一个$n$维幺正矩阵表示就需要引入$n$个标量场。($n$和李代数维数没有必然关系)此时,自由理论形如
\[
    \mathcal{L} = \frac{1}{2} \partial_\mu \phi^\dagger \partial^\mu \phi - \frac{1}{2} m^2 \phi^2,
\]
其中$\phi$是$n$个标量场排成的一个列矢量。我们用$t^a$标记$G$的$n$维幺正表示的李代数成员。如果我们想让拉氏量规范不变,即在变换
\[
    \phi(x) \to \Omega(x) \phi(x), \quad \phi^\dagger(x) \to \phi^\dagger(x) \Omega^\dagger(x), \quad \Omega(x) = \ee^{- \ii g \theta_a(x) t^a},
\]
下不变,那么无需调整质量项,因为显然
\[
    \phi^\dagger \Omega^\dagger \Omega \phi = \phi^2.
\]
含有导数的项则需要修正,具体来说,我们需要找到某种协变导数,使得
\begin{equation}
    D_\mu (\Omega(x) \phi(x)) = \Omega(x) D_\mu \phi(x).
    \label{eq:covariant-derivative}
\end{equation}
$\partial_\mu$肯定不满足这个条件,因为
\[
    \partial_\mu (\Omega(x) \phi(x)) = \Omega(x) \partial_\mu \phi + (\partial_\mu \Omega) \phi.
\]
对狄拉克旋量,表面上看如果有$n$个旋量场,$G$可以有$4n$维表示,但是这实际上是行不通的,因为旋量场的拉氏量为
\[
    \mathcal{L} = \bar{\psi} (\ii \gamma^\mu \partial_\mu - m) \psi,
\]
如果$\Omega$是$4n$维的,那么可以让$\Omega$作用到旋量内部的各个分量上。
然而,此时$\bar{\psi}$的变换方式为
\[
    \bar{\psi} \to \psi^\dagger \Omega^\dagger \gamma^0,
\]
没有什么能够保证$\gamma$和$\Omega^\dagger$一定对易,但是$\gamma$和$\Omega^\dagger$最好是对易的,否则简单地接受\eqref{eq:covariant-derivative}并不能让拉氏量在规范变换下不变。
一种比较方便的做法是在将$\Omega$作用于旋量场上的时候将旋量(以及各个$\gamma$矩阵)看成一个整体,不允许对其分量单独进行操作,于是$\Omega$应该是$n$维的,并且因为$\gamma^0$此时相当于一个标量,它和$\Omega$肯定是对易的。
同样,所有$n$个旋量场的质量必须一样,否则$m$将成为一个任意的对角矩阵,而未必和$\Omega$对易,那么$\bar{\psi} m \psi$就不是不变的了。
接受这个做法之后,在旋量场的情况下同样只需要设法找到某种协变导数使得\eqref{eq:covariant-derivative}成立即可。

现在我们分析协变导数的具体形式。在杨-米尔斯理论中,我们模仿电动力学,直接引入矢量场$A_\mu$并要求
\begin{equation}
    D_\mu = \partial_\mu - \ii g A_\mu,
\end{equation}
并以此为依据决定$A_\mu$在规范变换下如何变动,即让$A_\mu$“吸收掉”多余的$(\partial_\mu \Omega) \phi$项。
这种协变导数的形式和微分几何中的协变导数非常一致,$A$就是一个联络(我们将在\autoref{sec:transition}中看到它确确实实就是几何上那种“平移时矢量分量跟着转”的联络),即所谓\concept{规范联络}或者说\concept{规范场}。
显然应有
\[
    \partial_\mu - \ii g A_\mu' = \Omega(x) (\partial_\mu - \ii g A_\mu) \Omega^{-1}(x),
\]
容易看出$A_\mu$的变换规则应为
\begin{equation}
    A_\mu(x) \to \Omega(x) A_\mu(x) \Omega^{-1}(x) + \frac{\ii}{g} \Omega (\partial_\mu \Omega^{-1}(x)). 
\end{equation}
请注意$\Omega(x)$是$n$维矩阵,因此为了避免得到平庸的结果,实际上我们也需要让$A_\mu$变成一个$n$维矩阵,也就是除了时空指标$\mu=0, 1, 1, 3$以外还需要让$A$带两个从$1$跑到$n$的矩阵指标。
在杨-米尔斯理论中我们实际上会将每个时空点、每个时空分量上都是$n$维矢量的$A$场限制为李代数$\{t^a\}$的成员,因为$A_\mu$的变换规则的无穷小版本为
\[
    \var{A_\mu} = \ii g \theta_a \comm*{A_\mu}{t^a} - t^a \partial_\mu \theta_a,
\]
因此如果我们要求$A_\mu(x)$是李代数$\{t^a\}$的成员那么变换之后它还是李代数的成员。
这样,$A$可以用三个标签标记,一个是时空点$x$,一个是矢量指标$\mu$,还有一个是规范指标$a$,写成
\begin{equation}
    A_\mu(x) = A_\mu^a(x) t^a.
\end{equation}
规范场和旋量场(或标量场)现在都带上了一个规范指标$a$,其
于是就有
\begin{equation}
    \begin{aligned}
        \var{A_\mu} &= \ii g \theta_a \comm*{A_\mu^b t^b}{t^a} - t^a \partial_\mu \theta_a \\
        &= - t^a D_\mu \theta_a,
    \end{aligned}
\end{equation}
其中
\begin{equation}
    D_\mu \theta^a = \partial_\mu \theta^a + g f^{abc} A^b_\mu \theta^c.
\end{equation}
这里需要解释一下记号$D_\mu$。我们之前定义的$D_\mu$作用在$G$的一个$n$维表示上;然而,规范场$A_\mu^a$是将$A_\mu$以$t^a$为基底展开得到的分量。
作用在$A_\mu$上的变换并不是$n$维表示空间上的线性变换,而是李代数的伴随表示的表示空间上的线性变换。
此处的协变导数$D_\mu$在后者中而不在前者中,它和作用在$\psi$或是$\phi$上的$D_\mu$的具体形式是不同的(虽然都代表“协变的平移”)。
此外$D_\mu \theta_a$实际上是$(D_\mu \theta)_a$,即它会将$\theta_a$的各个分量混合起来,正如微分几何中的情况那样,即我们有$D_\mu \theta_a = D_{\mu \ ac} \theta_c$。

然后我们就会发现,如果李群是非阿贝尔的,那么$A_\mu$的无穷小变换不仅仅是场值做一个平移,还需要加上一个对易子。
电动力学仅讨论$U(1)$变换,属于阿贝尔规范理论,杨-米尔斯理论则是非阿贝尔规范场论。
今后我们称$A_\mu$为\concept{规范场},而称$\psi$或是$\phi$为\concept{物质场},因为和电动力学中的图像类似,似乎“物质场通过携带动量的规范玻色子发生相互作用”。
当然,规范玻色子——如光子——其实也是一种物质,所以这个说法有不准确之处。

\subsubsection{场强张量}

规范场可以有它自己的动能项。在杨-米尔斯理论中,这个动能项大体上仍然应该和电动力学一致,即大体上仍有
\[
    \mathcal{L}_A = - \frac{1}{4} F_{\mu \nu} F^{\mu \nu}
\]
成立。在电动力学中我们有
\[
    \comm*{D_\mu}{D_\nu} = \ii e F_{\mu \nu},
\]
而在杨-米尔斯理论中,$\comm*{D_\mu}{D_\nu}$在规范变换下为
\[
    \comm*{D_\mu}{D_\nu} \to \Omega \comm*{D_\mu}{D_\nu} \Omega^{-1},
\]
因此可以定义
\begin{equation}
    \comm*{D_\mu}{D_\nu} = - \ii g F_{\mu \nu} = - \ii g (\partial_\mu A_\nu - \partial_\nu A_\mu - \ii g \comm*{A_\mu}{A_\nu}),
\end{equation}
作为电动力学中的场强张量的推广。由于$A$实际上是$n$维矩阵,$F_{\mu \nu} F^{\mu \nu}$也是$n$维矩阵,因此我们还需要加上一个求迹操作就能够得到规范不变而同时洛伦兹不变的拉氏量:
\begin{equation}
    \mathcal{L}_A = - \frac{1}{2} \trace(F_{\mu \nu} F^{\mu \nu}) = - \frac{1}{4} F_{\mu \nu}^a F^{a \ \mu \nu},
\end{equation}
如果$G$是$U(1)$,那么上式就自动退化为了电动力学。

可以看到在这种思路下面规范场本身是不能有质量的,因为质量项$m^2 A_\mu A^\mu$无论如何没法变得规范不变。
但是,通过希格斯机制,实际上可以给规范场引入一个等效的质量。本节暂时不讨论这些内容。

因此我们现在就得到了杨-米尔斯理论的拉氏量:如果规范场和旋量场耦合,那么拉氏量就是
\begin{equation}
    \mathcal{L} = \bar{\psi} (\ii \slashed{D} - m) \psi - \frac{1}{4} F_{\mu \nu}^a F^{a \ \mu \nu}.
    \label{eq:yang-mills-lagrangian}
\end{equation}
$F_{\mu \nu} F^{\mu \nu}$项前面的系数本来可以有变化,但是我们完全可以将其吸收到$A$中,然后用调整$g$来保持协变导数不变。
这个拉氏量看起来和电动力学基本上一样,但是因为$F$中的非线性部分,其经典行为实际上就非常有趣。

前面已经说明过,所有旋量场的质量都是一样的,并且从\eqref{eq:yang-mills-lagrangian}也可以看出,规范玻色子没有质量。
因此表面上,杨-米尔斯理论是非常局限的。
但实际上并不是这样:通过希格斯机制是可以引入质量的。

\subsubsection{关于李代数的限制}

% TODO:对李代数的限制

\subsection{对称性和守恒量}

\subsubsection{平移}\label{sec:transition}

规范场的平移操作值得特别讨论。简单地令
\begin{equation}
    x \longrightarrow x + \var{x}, \quad \psi'(x') = \psi(x), \quad A'(x') = A(x)
    \label{eq:naive-transition}
\end{equation}
的确可以保持拉氏量不变,也的确能够据此计算出一个诺特守恒流,但是这样的诺特守恒流不是规范不变的。
这就是说,应该从中“删去”一些虽然平移不变,但是并非规范不变的东西,才能够得到定义良好的能量-动量张量。
我们知道使用\eqref{eq:naive-transition}计算能动张量实际上同时做了一个从拉氏量到哈密顿量的切换,其中我们取
\[
    \pi^\mu = \pdv{\mathcal{L}}{\partial \partial_\mu \phi}
\]
为正则动量,所以其实\eqref{eq:naive-transition}得不到有意义的结果是非常正确的——规范场论本身含有冗余自由度,不能指望此时朴素的勒让德变换仍然成立。
还可以从另一个角度出发看这个问题:变换\eqref{eq:naive-transition}和规范变换不对易,其生成元自然和规范变换也不对易。

注意到,协变导数$D_\mu$是和规范变换对易的,并且它的确代表某种平移,因此我们尝试以它做无穷小变换,或者等价地说,先做一个纯粹的平移变换,再做一个规范变换,即取
\begin{equation}
    \begin{aligned}
        x^\mu &\longrightarrow x^\mu + \var{x^\mu}, \\
        \phi(x) &\longrightarrow \phi'(x') = \exp(\ii g \var{x^\mu} A_\mu) \phi(x), \\
        A_\mu(x) &\longrightarrow A_{\mu}'(x') = \exp(\ii g \var{x^\nu} A_\nu) A_\mu(x) \exp(- \ii g \var{x^\rho} A_\rho) \\
        &+ \frac{\ii}{g} \exp(\ii g \var{x^\rho} A_\rho) \partial_\mu \exp(- \ii g \var{x^\nu} A_\nu),
    \end{aligned}
\end{equation}
计算得到
\begin{equation}
    \begin{aligned}
        \var{\phi} &= - \var x^\mu D_\mu \phi, \\
        \var{A_\mu} &= - \var{x^\nu} F_{\nu \mu},
    \end{aligned} 
\end{equation}
于是守恒流就是

\subsection{Wick转动}

在做完Wick转动

\subsection{微分几何的观点}

\section{杨-米尔斯理论的量子化}

\subsection{Faddeev–Popov量子化}

\subsubsection{规范固定和鬼场}

规范对称性会导致正则量子化变得比较困难,因为需要做复杂的规范选取来消除多余的自由度,而在路径积分量子化中则可以通过Faddeev–Popov量子化比较容易地解决。
我们已经在自由无质量矢量场的量子化中使用过了Faddeev–Popov量子化,这回我们如法炮制。
和自由无质量矢量场的情况不同,此时不仅需要引入规范固定项,还需要引入鬼场。

我们通过在$\int \fd{A_\mu}$之后插入
\[
    1 = \int \fd{\alpha} \delta(G(A^\alpha)) \det(\fdv{G(A^\alpha)}{\alpha})
\]
来设法将对只相差一个规范变换的场重复计数导致的因子提取出来,其中$\alpha$标记规范变换的参数,它带有一个规范指标;规范固定为$G(A)=0$,$G$定义为洛伦兹协变的
\begin{equation}
    G(A^\alpha) = \partial^\mu (A^\alpha)_\mu - \omega(x),
\end{equation}
其中$\omega(x)$是任意的标量场。我们让$\alpha$取无穷小量,则有
\[
    G(A^\alpha) = \partial^\mu A_\mu + \frac{1}{g} \partial^\mu D_\mu \alpha^a - \omega(x),
\]
于是
\[
    \begin{aligned}
        Z &= \int \fd{A} \fd{\psi} \ee^{\ii S[A, \psi]} \\
        &= \int \fd{A} \fd{\psi} \int \fd{\alpha} \delta(G(A^\alpha)) \det(\fdv{G(A^\alpha)}{\alpha}) \ee^{\ii S[A, \psi]} \\
        &= \frac{1}{g} \det(\partial^\mu D_\mu) \int \fd{A} \int \fd{\psi} \int \fd{\alpha} \delta(G(A^\alpha)) \ee^{\ii S[A, \psi]} \\
        &= \frac{1}{g} \det(\partial^\mu D_\mu) \int \fd{\psi} \int \fd{\alpha} \int \fd{A^\alpha} \delta(G(A^\alpha)) \ee^{\ii S[A^\alpha, \psi]} \\
        &= \frac{1}{g} \det(\partial^\mu D_\mu) \int \fd{\psi} \int \fd{\alpha} \int \fd{A} \delta(G(A)) \ee^{\ii S[A, \psi]},
    \end{aligned}
\]
倒数第二个等号是因为$\fd{A}$和$\fd{A^\alpha}$相同,倒数第一个等号是我们重新标记了场。
既然$\omega(x)$可以任意取值,我们不妨重新定义配分函数,去掉无用的因子$g$,并对所有的$\omega(x)$求和,得到
\[
    \begin{aligned}
        Z &= \det(\partial^\mu D_\mu) \int \fd{\omega} \ee^{-\ii \int \dd[4]{x} \frac{\omega^2}{2 \xi}} \int \fd{A} \int \fd{\psi} \int \fd{\alpha} \delta(\partial^\mu A_\mu - \omega(x)) \ee^{\ii S[A, \psi]} \\
        &= \det(\partial^\mu D_\mu) \int \fd{\alpha} \int \fd{A} \int \fd{\psi} \exp(-\ii \int \dd[4]{x} \frac{(\partial^\mu A_\mu^a)^2}{2 \xi}) \ee^{\ii S[A, \psi]}.
    \end{aligned}
\]
无用的$\int \fd{\alpha}$因子可以略去。与自由场的情况不同,此时因子$\det(\partial^\mu D_\mu)$中仍然含有场变量,不能直接丢弃。
为此,可以引入一个\concept{鬼场}$c$,它是一个复标量场,但是是格拉斯曼数,这样就能够满足
\[
    \int \fd{c} \int \fd{\bar{c}} \exp(\ii \int \dd[4]{x} \bar{c} (- \partial^\mu D_\mu) c) = \det(\partial^\mu D_\mu),
\]
如果$c$是普通标量场,那么行列式会出现在分母上。显然$c$并没有什么物理意义,在计算协变的物理量时也没有外线。

现在我们就完成了规范场论\eqref{eq:yang-mills-lagrangian}的量子化:只需要用等效的拉氏量
\begin{equation}
    \mathcal{L} = \bar{\psi} (\ii \slashed{D} - m) \psi - \frac{1}{4} F_{\mu \nu}^a F^{a \ \mu \nu} \underbrace{- \frac{(\partial^\mu A_\mu^a)^2}{2 \xi}}_{\text{gauge fixing}} + \underbrace{\bar{c} (- \partial^\mu D_\mu) c}_{\text{ghost}}
\end{equation}
做路径积分即可。注意规范场和旋量场都有$n$个,因此协变导数$D_\mu$是一个$n$维矩阵,鬼场$c$也是$n$维的,多出来一个规范指标标记这$n$个维度。

\subsubsection{传播子和顶角}

关于费米子交换加负号这件事:规范理论中相互作用项中费米场均呈现为$\bar{\psi} \psi$形式,若干个相互作用项的乘积形如$\bar{\psi} \psi \cdots \bar{\psi} \psi$。
容易验证,这样一个算符序列和外部的算符缩并,缩并线的交叉次数一定是偶数,从而不会因为费米子算符的交换而产生负号。
因此,负号只应该存在于这样一个算符序列内部的缩并,即出现在费米子线形成一个圈的时候。
而容易验证,此时负号仅仅存在于$\expval*{\bar{\psi} \psi}$一个因子中。
因此,费米子交换加负号这件事在规范场论的费曼图中体现为:但凡费米子线形成了一个圈,加负号,否则什么都不做。
在单条费米子线形成一个圈时可以直接把这个圈算出来,此时负号已经体现在这个圈的值当中了,因此无需做任何额外处理。

从规范场论演生出的低能有效理论——如库伦相互作用——虽然不再有规范玻色子传播子,但是顶角的形式仍然保持不变,因此费米子交换加负号这一特点同样可以通过“但凡费米子线形成了圈,加负号”完全描述。

\subsection{BRST对称性}

引入鬼场之后,有效作用量就不再是规范不变的了,因为规范冗余性已经消除,规范不变性被规范固定项和鬼场去掉了。
不过,BRST四人发现,之前定义的规范变换如果补充上一个鬼场的变换,能够有一个整体的对称性。
这个对称性显然不是通常意义上的整体对称性,因为鬼场等都是非物理的,从而,毫不意外的,这个对称性对应的守恒荷是一个格拉斯曼数,其平方为零。
这种变换——\concept{BRST变换}——可以很好地描述规范场的拓扑性质。
通过BRST变换还可以得到Ward-Takahashi恒等式和Slavnov-Taylor恒等式。

首先,我们引入一个玻色辅助场$B$——实际上是一系列玻色辅助场$B^a$,带有一个规范指标,其总数和李代数的维数一致,和$n$没有特别直接的关系——让规范固定项消失,得到
\begin{equation}
    \mathcal{L} = \bar{\psi} (\ii \slashed{D} - m) \psi - \frac{1}{4} F_{\mu \nu}^a F^{a \ \mu \nu} + \bar{c}^a (- \partial^\mu D_\mu^{ac}) c^c + \frac{\xi}{2} (B^a)^2 + B^a  \partial^\mu A_\mu^a. 
    \label{eq:gauge-fixed-with-b}
\end{equation}
现在考虑如下无穷小变换:
\begin{equation}
    \begin{aligned}
        \var{\psi} &= \ii g \epsilon c^a t^a \psi, \\
        \var{A^a_\mu} &= \epsilon D_\mu^{ac} c^c, \\
        \var{c^a} &= - \frac{1}{2} g \epsilon f^{abc} c^b c^c, \\
        \var{\bar{c}^a} &= \epsilon B^a, \\
        \var{B^a} &= 0,
    \end{aligned}
    \label{eq:brst}
\end{equation}
整个拉氏量\eqref{eq:gauge-fixed-with-b}在变换\eqref{eq:brst}之下完全就是不变的。
首先,$A_\mu$和$\psi$的变换就是以$\epsilon c^c$为位移的规范变换,因此\eqref{eq:gauge-fixed-with-b}的头两项不变。
第四项也不变,因为$B$根本就没有发生任何变化。很容易注意到
\[
    \var{B^a \partial^\mu A_\mu^a} = \epsilon B^a \partial^\mu D_\mu^{ac} c^c = - (\var{\bar{c}^a}) (- \partial^\mu D_\mu^{ac}) c^c,
\]
因此只需要检验
\[
    \var{(D_\mu^{ac} c^c)} = 0
\]
即可。将上式展开,
\[
    \begin{aligned}
        \var{(D_\mu^{ac} c^c)} &= D^{ac}_\mu (- \frac{1}{2} g \epsilon f^{cbd} c^b c^d) + g f^{abc} (\var{A_\mu^b}) c^c \\
        &= - \frac{1}{2} g \epsilon \partial_\mu (f^{aed} c^e c^d) - \frac{1}{2} g^2 \epsilon f^{abc} f^{ced} A_\mu^b c^e c^d + \epsilon g f^{abc} (\partial_\mu c^b + g f^{bed} A_\mu^e c^d) c^c ,
    \end{aligned}
\]
展开第一项并使用鬼场的反交换性可以得到
\[
    - \frac{1}{2} g \epsilon \partial_\mu (f^{aed} c^e c^d) = g \epsilon f^{ade} c^e \partial_\mu c^d,
\]
于是
\[
    \var{(D_\mu^{ac} c^c)} = - \frac{1}{2} g^2 \epsilon f^{abc} f^{ced} A_\mu^b c^e c^d + \epsilon g^2 f^{abc} f^{bed} A_\mu^e c^d c^c.
\]
我们将上式右边第二项写得更加对称一些(又一次用到了鬼场的反对称性):
\[
    \begin{aligned}
        \var{(D_\mu^{ac} c^c)} &= - \frac{1}{2} g^2 \epsilon f^{abc} f^{ced} A_\mu^b c^e c^d + \epsilon g^2 f^{abc} f^{bed} A_\mu^e c^d c^c \\
        &= - \frac{1}{2} g^2 \epsilon f^{abc} f^{ced} (A_\mu^b c^e c^d + A^e_\mu c^d c^b + A_\mu^d c^b c^e),
    \end{aligned}
\]
重新排列指标,得到
\[
    \var{(D_\mu^{ac} c^c)} = - \frac{1}{2} g^2 \epsilon (f^{abc} f^{ced} + f^{adc} f^{cbe} + f^{aec} f^{cdb}) A_\mu^b c^e c^d.
\]
由雅可比恒等式发现上式为零。这就表明\eqref{eq:gauge-fixed-with-b}在变换\eqref{eq:brst}之下不变。这称为\concept{BRST}对称性。

设BRST变换的无穷小生成元为$Q$,则由于\eqref{eq:brst}中的每一条变换都正比于鬼场而鬼场是反对易的,我们立刻发现$Q^2=0$。
如果我们对\eqref{eq:gauge-fixed-with-b}做正则量子化,那么它一定有一个幂零的守恒荷$Q$。
一个幂零算符会给出希尔伯特空间的如下分割:
\begin{itemize}
    \item 将被$Q$作用后不为零的那些态组成的子空间记作$\mathcal{H}_1$;
    \item 将$Q$作用在$\mathcal{H}_1$上得到的子空间记作$\mathcal{H}_2$;
    \item 将除此以外的部分记作$\mathcal{H}_0$。
\end{itemize}
由于$Q$和$H$对易,任何一个本征态被作用了$Q$之后得到的还是本征态。$\mathcal{H}_1$中的本征态被作用了$Q$之后就得到$\mathcal{H}_2$中的本征态。
$\mathcal{H}_0$中的本征态被作用了$Q$之后得到的本征态还是在$\mathcal{H}_0$中。
也即,$\mathcal{H}_0$中有一个二重简并,$\mathcal{H}_1$和$\mathcal{H}_2$放在一起构成二重简并。

为了获得定义良好的单粒子态,我们暂时令$g=0$,此时会发现,在$Q$的作用下反鬼场$\bar{c}$变为规范玻色子(注意根据\eqref{eq:gauge-fixed-with-b},$B^a$和$A^a_\mu$满足的算符方程是线性的,它们实际上对应同一种激发),
% TODO

还有一个微妙的细节需要说明:BRST变换保持路径积分的积分测度不变,即这里没有量子反常。
这件事的证明如下。首先,不难验证
\[
    \pdv{B^{a'}}{B^a} = \pdv{\bar{c}^{a'}}{\bar{c}^{a}} = \pdv{c^{a'}}{c^a} = 0,
\]
因此只需要计算$\psi$和$A$两组变量的雅可比行列式即可。我们有
\[
    \pdv{\psi'}{\psi} = \ii g \epsilon c^a t^a, \quad \pdv{\bar{\psi}'}{\bar{\psi}} = - \ii g \epsilon c^a t^a,
\]
以及
\[
    \pdv{A^{a'}_\mu}{A^b_\mu} = g f^{abc} c^c.
\]
路径积分测度的变换因子为
\[
    \frac{\det A}{\det \psi \det \bar{\psi}} = \frac{\prod_\mu (1 + g f^{abc} c^c)}{(\det (1 - \ii g \epsilon c^a t^a)) (\det (1 + \ii g \epsilon c^a t^a))} = 1 + \mathcal{O}(\epsilon^2),
\]
因此BRST变换下路径积分测度的确是几乎不变的。

\subsection{微扰计算}

微扰计算使用费曼图,费曼图中我们要对内线上的动量做四重积分。

\section{正规化和重整化}

本节将花大量篇幅分析微扰计算时,是否可以通过引入有限个抵消项消除各阶费曼图中的所有发散。
乍一看,基于费曼图的分析是不必要的,因为重整化群流具有紫外不动点这件事足够保证一个量子场论的可重整性。
但是应当注意,重整化群流的计算本身常用微扰论,一阶或是二阶微扰论给出的重整化群流是不是可靠是不好说的;
此外,一个理论可重整化和它的每个费曼图都能够消除所有发散并不是一回事——可能存在这样的情况,使得需要重求和一些图才能够消除所有发散。
因此,基于费曼图和抵消项来分析理论的可重整性还是必要的。

\subsection{发散的来源}

\subsubsection{用表观发散度做估计}

从\eqref{eq:yang-mills-lagrangian}可以看出,各个物理量的量纲满足
\[
    [\dd[4]{x}] + [F^2] = 0, \quad [F] = [\partial A] = 1 + [A],
\]
于是
\[
    [A] = 1 = [\partial].
\]
另一方面,根据协变导数的定义,$\partial$和$g A$的量纲一致,因此只有一种可能:$g$根本就没有量纲。

量纲分析对表观发散度计算很重要。我们知道一张图的表观发散度是
\begin{equation}
    D(\Gamma) = \sum_i n_i \left( d_i + b_i + \frac{3}{2} f_i - 4  \right) + 4 - E_\text{B} - \frac{3}{2} E_\text{F},
\end{equation}
其中$d_i$,$b_i$和$f_i$分别表示$i$类型顶角的动量幂次,玻色子线数目(鬼场也算玻色子,因为其动能项形式和玻色子一致;量纲分析和对易还是反对易无关)和费米子线数目。
玻色场的量纲是$1$(前面刚刚证明过),费米场的量纲是$3/2$,而如前所述耦合常数无量纲,因此对来自协变导数的规范场-旋量场耦合项,有
\[
    [\dd[4]{x}] + d_i + b_i + \frac{3}{2} f_i = 0,
\]
即
\[
    d_i + b_i + \frac{3}{2} f_i - 4 = 0.
\]
其它的顶角来自$F_{\mu \nu} F^{\mu \nu}$项,其中的三玻色子顶角的耦合常数$\sim g$而四玻色子顶角的耦合常数$\sim g^2$,因此同样耦合常数量纲为零,同样有上式成立。
这就意味着,在杨-米尔斯理论中,我们有
\begin{equation}
    D(\Gamma) = 4 - E_\text{B} - \frac{3}{2} E_\text{F}.
\end{equation}
考虑让$D(\Gamma) \geq 0$的情况,会发现以下几种图按照表观发散度的估计会发散:
\begin{itemize}
    \item 四条玻色子外线;
    \item 三条玻色子外线;
    \item 两条玻色子外线;
    \item 一条玻色子外线,两条费米子外线;
    \item 两条费米子外线。
\end{itemize}
还有一些图,如一条费米子外线和两条玻色子外线的图,如果存在,也会发散,但是杨-米尔斯理论的顶角形式意味着这样的图是不可能出现的。

\subsection{重整化群}

杨-米尔斯理论的重整化群真的是一个群?

到目前为止,只有非阿贝尔的杨-米尔斯理论在四维闵可夫斯基时空中具有渐近自由。QED不存在渐近自由。

\section{量子色动力学}

本节给出非阿贝尔杨-米尔斯理论的一个例子,著名的描述了强相互作用的量子理论,\concept{量子色动力学}或者简称\concept{QCD}。

QCD是通过$SU(2)$规范对称性得到的理论。我们取$SU(2)$的维数最小的幺正表示,此时李代数的表示的基底就是泡利矩阵。
这是一个二维表示,从而我们需要两个放在一起的旋量场做表示空间,记作
\begin{equation}
    \psi = \pmqty{\psi_1 \\ \psi_2},
\end{equation}
称为\concept{二重态}。

\section{共形场论}

对二维系统,共形变换是局域变换群,共性不变的场论也是一种规范理论,称为共形场论。
高维共性不变性是整体的,没有局域不变性,从而也不是规范理论。

\end{document}