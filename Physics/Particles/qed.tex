\documentclass[hyperref, UTF8, a4paper]{ctexart}

\usepackage{geometry}
\usepackage{titling}
\usepackage{titlesec}
\usepackage{paralist}
\usepackage{footnote}
\usepackage{enumerate}
\usepackage{amsmath, amssymb, amsthm}
\usepackage{simplewick}
\usepackage{cite}
\usepackage{graphicx}
\usepackage{subfigure}
\usepackage{physics}
\usepackage{mathtools}
\usepackage{slashed}
\usepackage{centernot}
\usepackage{tikz}
\usepackage{tikz-feynhand}
\usepackage[colorlinks, linkcolor=black, anchorcolor=black, citecolor=black]{hyperref}
\usepackage{prettyref}

\geometry{left=3.18cm,right=3.18cm,top=2.54cm,bottom=2.54cm}
\titlespacing{\paragraph}{0pt}{1pt}{10pt}[20pt]
\setlength{\droptitle}{-5em}
\preauthor{\vspace{-10pt}\begin{center}}
\postauthor{\par\end{center}}

\DeclareMathOperator{\timeorder}{T}
\DeclareMathOperator{\diag}{diag}
\newcommand*{\ii}{\mathrm{i}}
\newcommand*{\ee}{\mathrm{e}}
\newcommand*{\const}{\mathrm{const}}
\newcommand*{\comment}{\paragraph{注记}}
\newcommand{\fsl}[1]{{\centernot{#1}}}
\newcommand*{\reals}{\mathbb{R}}
\newcommand*{\complexes}{\mathbb{C}}

\newrefformat{sec}{第\ref{#1}节}
\newrefformat{note}{注\ref{#1}}
\renewcommand{\autoref}{\prettyref}

\newenvironment{bigcase}{\left\{\quad\begin{aligned}}{\end{aligned}\right.}

\newcommand{\concept}[1]{\underline{\textbf{#1}}}
\renewcommand{\emph}{\textbf}

\newcommand*{\bigO}[1]{\mathcal{O}{#1}}

\allowdisplaybreaks[4]

\title{量子电动力学的具体计算}
\author{吴晋渊}

\begin{document}

\maketitle

\section{非相对论极限}

\subsection{电子,光子和电场}

光子无法做非相对论近似,因为无论如何,麦克斯韦方程都应该成立,而这个方程就是洛伦兹协变的。
需要做非相对论近似的只有电子。在非相对论近似下,一切有质量的场都退化为薛定谔场,电子也不例外。
因此在QED的非相对论极限下,基本的粒子包括电子和光子,光子无任何变化,电子场则是动能为$\vb*{k}^2 / 2m$,不再满足相对论协变性,由动量和自旋标记的场。

\subsubsection{电子的薛定谔-泡利方程}

我们现在推导非相对论性电子场遵循的方程,然后得到非相对论极限下的电子哈密顿量。对称性告诉我们这个方程和狄拉克方程和电磁场耦合后取非相对论性近似的结果肯定是一样的。
但是,应该注意,前者中的电磁场都是算符场,而后者中的电磁场是一个经典外场,即后者忽略了电磁场的量子涨落。
电磁场的量子涨落会产生可观测的效应(即所谓\concept{辐射修正})。实际上,这是量子场论的理论框架适用于电动力学的实验证据:一个相对论性的关于电子和光的理论未必要采取量子场论的形式,将电磁场量子化之后,圈图修正就会预言一些经典的电磁场不会产生的现象。
如果这些现象实际上没有观测到,那么量子场论的理论框架就是没有用的或者说错误的,但是实际上我们观测到了这些现象,那么量子场论很可能就是对的。

QED中,电子的狄拉克方程为
\[
    (\ii \gamma^\mu \partial_\mu - e \gamma^\mu A_\mu - m) \psi = 0.
\]
使用狄拉克表象以区分电子和正电子,设旋量场为
\[
    \psi = \pmqty{\phi \\ \chi},
\]
我们将要分析$\phi$遵循的运动方程。为此,根据
\[
    \gamma^0 = \pmqty{1 & 0 \\ 0 & -1}, \quad \gamma^i = \pmqty{0 & \sigma^i \\ - \sigma^i & 0},
\]
写出狄拉克方程的分量表达式:
\[
    \begin{aligned}
        (\ii \partial_t - e \varphi - m) \phi + (e (\vb*{\sigma} \cdot \vb*{A}) + \ii (\vb*{\sigma} \cdot \grad)) \chi &= 0, \\
        (\ii \partial_t - e \varphi + m) \chi + (e (\vb*{\sigma} \cdot \vb*{A}) + \ii (\vb*{\sigma} \cdot \grad)) \phi &= 0,
    \end{aligned}
\]
这里我们需要指出一个容易弄混的地方:正确的分量是
\[
    \partial_\mu = (\partial_t, \grad), \quad A_\mu = (\varphi, - \vb*{A}),
\]
而贸贸然地很容易认为前者是$(\partial_t, -\grad)$而后者是$(\varphi, \vb*{A})$。
我们消去$\chi$,就得到
\begin{equation}
    (\ii \partial_t - e \varphi - m) \phi - (e (\vb*{\sigma} \cdot \vb*{A}) + \ii (\vb*{\sigma} \cdot \grad)) \frac{1}{\ii \partial_t - e \varphi + m} (e (\vb*{\sigma} \cdot \vb*{A}) + \ii (\vb*{\sigma} \cdot \grad)) = 0.
    \label{eq:electron-only}
\end{equation}

到目前为止我们没有做任何近似。现在我们做最强的非相对论近似。我们按照通常的从克莱因-高登方程获得薛定谔方程的方法,设
\[
    \phi = \psi \ee^{-\ii m t}.
\]
这里我们重复使用了符号$\psi$,用它表示一个旋量场,以及从这个旋量场得到的一个薛定谔场;通常这不会导致混淆,因为前者只出现在高能物理中而后者只出现在凝聚态物理中。
我们发现
\[
    (\ii \partial_t - e \varphi - m) \phi = \ee^{-\ii m t} \partial_t \psi - e \varphi \psi \ee^{-\ii m t},
\]
而
\[
    (\ii \partial_t - e \varphi + m) \phi = \ee^{-\ii m t} \partial_t \psi - e \varphi \psi \ee^{-\ii m t} + 2 m \ee^{-\ii m t} \psi.
\]
在非相对论极限下粒子的动能相较于其静能(请注意这里都是自然单位制,$c=1$,$m$就是静能)是非常低的,于是
\[
    \partial_t \psi \ll 2m \psi,
\]
因此我们有
\[
    \ii \partial_t \psi - e \varphi \psi - \vb*{\sigma} \cdot (e\vb*{A} + \ii \grad) \frac{1}{2m - e\varphi} \vb*{\sigma} \cdot (e\vb*{A} + \ii \grad) \psi = 0.
\]
进一步,认为电场不很强(否则电子会轻易被加速到很高的速度,非相对论极限不正确),从而
\[
    e \varphi \ll m,
\]
那么就得到
\begin{equation}
    \ii \partial_t \psi - e \varphi \psi - \frac{(\vb*{\sigma} \cdot (e\vb*{A} + \ii \grad))^2}{2m} \psi = 0.
\end{equation}
做了如此的近似之后,我们相当于忽略了所有涉及正电子的辐射修正。
这个方程可以化成更加清晰的一个形式。我们知道
\[
    (\vb*{\sigma} \cdot \vb*{a}) (\vb*{\sigma} \cdot \vb*{b}) = \vb*{a} \cdot \vb*{b} + \ii \vb*{\sigma} \cdot (\vb*{a} \times \vb*{b}),
\]
而由于$(e\vb*{A} - \ii \grad)$是算符,它自乘并不是零,而是
\[
    (e\vb*{A} + \ii \grad) \times (e\vb*{A} + \ii \grad) \psi = \ii e \vb*{A} \times (\grad{\psi}) + \ii e \curl{(\vb*{A} \psi)} = \ii e  (\curl{\vb*{A}}) \psi = \ii e \vb*{B} \psi,
\]
于是我们就有
\begin{equation}
    \ii \partial_t \psi = \frac{(e \vb*{A} + \ii \grad)^2}{2m} \psi + e \varphi \psi - \frac{e \vb*{\sigma} \cdot \vb*{B}}{2m} \psi.
\end{equation}
场论哈密顿量为
\begin{equation}
    \begin{aligned}
        H &= \int \dd[3]{\vb*{x}} \psi^\dagger \left( \frac{(- \ii \grad - e \vb*{A})^2}{2m} + e \varphi \psi - \frac{e \vb*{\sigma} \cdot \vb*{B}}{2m} \right) \psi \\
        &= \int \dd[3]{\vb*{x}} \psi^\dagger \left( - \frac{(\grad - \ii e \vb*{A})^2}{2m} + e \varphi \psi - \frac{e \vb*{\sigma} \cdot \vb*{B}}{2m} \right) \psi ,
    \end{aligned}
    \label{eq:minimal-coupling}
\end{equation}
其中
\begin{equation}
    \int \dd[3]{\vb*{x}} \psi^\dagger \frac{(- \ii \grad - e \vb*{A})^2}{2m} \psi = \sum_\sigma \int \frac{\dd[3]{\vb*{p}}}{(2\pi)^3} a^\dagger_{\vb*{p} \sigma} \frac{(\vb*{p} - e \vb*{A})^2}{2m} a_{\vb*{p} \sigma}.
\end{equation}
$\psi$是电子的湮灭算符,其运动方程就是薛定谔绘景下单电子波函数的运动方程。
于是,坐标表象下单电子的波函数(当然,此时已经隐含地认为电磁场是经典场,否则会有等效电子-电子相互作用,单电子图像不再适用)满足的方程就是
\begin{equation}
    \ii \partial_t \psi = \frac{(e \vb*{A} + \ii \grad)^2}{2m} \psi - \frac{e \vb*{\sigma} \cdot \vb*{B}}{2m} \psi + e \varphi \psi = \frac{(\vb*{p} - e \vb*{A})^2}{2m} \psi - \frac{e \vb*{\sigma} \cdot \vb*{B}}{2m} \psi + e \varphi \psi.
\end{equation}
这个方程的形式和自由电子薛定谔方程很像,但是哈密顿量为
\begin{equation}
    H = \frac{(\vb*{p} - e \vb*{A})^2}{2m} - \frac{e \vb*{\sigma} \cdot \vb*{B}}{2m} + e \varphi = \frac{(\vb*{p} - e \vb*{A})^2}{2m} - \vb*{\mu} \cdot \vb*{B} + e \varphi.
    \label{eq:pauli-eq}
\end{equation}
我们称此方程为\concept{泡利方程}或是\concept{薛定谔-泡利方程}。
这里的$\vb*{p}$的意义实际上尚不明确,它的定义是“在坐标表象下是$-\ii \grad$的那个算符”。
大体上它是“动量”,但是实际上它不是$m \vb*{v}$的那个机械动量。
如果我们计算\eqref{eq:pauli-eq}作为经典哈密顿量给出的运动方程,会得到
\begin{equation}
    \dv{\vb*{r}}{t} = \frac{\vb*{p} - e \vb*{A}}{m}, \quad \dv{\vb*{p}}{t} = - \grad_{\vb*{r}}{H},
\end{equation}
展开计算,并注意到
\[
    \dv{\vb*{A}}{t} = \pdv{\vb*{A}}{t} + \vb*{v} \cdot \grad{\vb*{A}},
\]
最终可以计算得到
\begin{equation}
    \begin{aligned}
        m \dv{\vb*{v}}{t} &= \dv{(\vb*{p} - e \vb*{A})}{t} = - e \grad{\varphi} - e \pdv{\vb*{A}}{t} + e \vb*{v} \times (\curl{\vb*{A}}) + \grad(\vb*{\mu} \cdot \vb*{B}) \\
        &= e \vb*{E} + e \vb*{v} \times \vb*{E} + \grad(\vb*{\mu} \cdot \vb*{B}).
    \end{aligned}
    \label{eq:non-rel-electron-motion}
\end{equation}
因此这里的$\vb*{p}$不是机械动量。

无论电磁场是经典场还是考虑量子涨落,\eqref{eq:minimal-coupling}都是成立的。
这些式子可以看成对\eqref{eq:pauli-eq}中的电子做二次量子化得到的;这些式子也意味着,如果我们要将相对论性的、量子化的电磁场和非相对论性的电子耦合起来,只需要量子化自由电磁场即可,因为电磁场和非相对论性的电子耦合的哈密顿量总是\eqref{eq:minimal-coupling},无论电磁场是经典的还是量子的。
\eqref{eq:minimal-coupling}也可以看成对非相对论性自由电子做$U(1)$最小耦合得到的结果。

\eqref{eq:minimal-coupling}中,$\vb*{A}$出现二阶项这件事值得分析,因为在QED中不存在这个情况:电磁场和电子旋量场之间的耦合中电磁场是一阶的。
回顾以上求解过程,这个$\vb*{A}^2$项实际上来自积掉$\chi$场的过程:我们只希望保留电子模式$\phi$而不想保留正电子模式$\chi$,于是就有\eqref{eq:electron-only},实际上就是积掉了$\chi$场。
分析$\ii \bar{\psi} \gamma^\mu D_\mu \psi$项的形式可以知道,与$\chi$场相关的过程包括:$\chi$模式吸收一个$\varphi$光子;$\chi$模式和$\phi$模式互换,这个顶角的耦合常数含有动量的一次方;$\chi$模式和$\phi$模式互换,吸收一个$\vb*{A}$光子。
因此,积掉$\chi$场引入了这样一个等效过程:$\phi$模式或是吸收一个$\vb*{A}$光子或是不吸收,变成$\chi$模式,然后吸收若干个$\varphi$光子,最后或是吸收一个$\vb*{A}$光子或是不吸收而重新变成$\phi$模式。
这个过程涉及$\vb*{A}$的零次方、一次方和平方,以及任意阶次的$\varphi$。
然而,$\varphi$光子实际上是在给$\chi$场提供自能修正:
\[
    \frac{1}{\ii \partial_t + m} + \frac{1}{\ii \partial_t + m} e \varphi \frac{1}{\ii \partial_t + m} + \cdots = \frac{1}{\ii \partial_t + m - e \varphi},
\]
而在非相对论极限下$e \varphi$相比于$m$肯定是很小的,从而可以忽略。
另一方面,$\vb*{A}$的大小和$\vb*{p}$的大小却不好比较:两者都不能太大。
因此在\eqref{eq:minimal-coupling}中出现了非线性的$\vb*{A}$耦合,但是没有出现非线性的$\varphi$耦合。

应当指出这个非线性的$\vb*{A}$耦合并不意味着以机械动量标记的电子与光子的耦合中,电磁场就是非线性的,因为\eqref{eq:pauli-eq}中的$\vb*{p}$是正则动量而不是机械动量。
\eqref{eq:pauli-eq}中的$\vb*{p}$不是实际可以直接观察到的机械动量,并且存在$\vb*{A}^2$项,这两者彼此抵消,正好导致了关于机械动量,并且电磁场耦合项完全线性的\eqref{eq:non-rel-electron-motion}。
回顾\eqref{eq:non-rel-electron-motion}的推导,会发现其中$\vb*{A}^2$项在运动方程中产生一些类似于$\vb*{A} \times (\curl{\vb*{A}})$的项,而$\vb*{p} \cdot \vb*{A}$项产生了一些类似于$\vb*{p} \times (\curl{\vb*{A}})$的项,由于在\eqref{eq:pauli-eq}中$\vb*{p}$是正则动量,这两项彼此抵消,正好产生了$\vb*{v} \times (\curl{\vb*{A}})$这样的项。
可以使用与\eqref{eq:pauli-eq}等价的拉氏量
\begin{equation}
    L = \frac{1}{2} m \vb*{v}^2 - e \varphi + e \vb*{v} \cdot \vb*{A} + \vb*{\mu} \cdot \vb*{B}
\end{equation}
来更加清楚地看到这一点。我们不可能只使用机械动量就得到完整的非相对论性粒子在电磁场中的运动,因为这相当于做了一个规范变换让$\vb*{A}=0$,而这是不可能的——能够产生磁场的那部分$\vb*{A}$无法转化为$\varphi$。

\subsubsection{相对论修正}

% TODO:自旋轨道耦合之类?
相对论修正指的是当电子的能量仍然不是特别高(从而没有必要使用完整的狄拉克方程)但是已经比较高时,相对论效应造成的修正。
它和量子涨落造成的修正是基本上无关的,因为即使不考虑量子涨落,狄拉克方程仍然会导致偏离\eqref{eq:pauli-eq}的结果。

\subsubsection{腔体中的光场}

\subsection{树图阶有效相互作用}

原则上\eqref{eq:minimal-coupling}和量子化后的电磁场给出了非相对论性电子和电磁辐射的全部理论,包括了全部电子和光的相互作用。
然而,实际上我们还可以做进一步的近似,将一些过程用等效相互作用替代。
做完这种近似之后需要特别小心,以避免将同样的过程重复计数。

一种在凝聚态物理中特别常见的做法是,将那些不满足横波条件的光子积掉,因为凝聚态物理中的光子基本上能够直接对应为能够长距离传播的经典电磁波,它能够长距离地“逃出”介质。
能够通过大片真空的电磁波模式只能是满足横波条件的电磁波——不满足横波条件的电磁波模式只能出现在电荷附近。
因此,不满足横波条件的光子应当被积掉,因为它们代表“介质内部发生的、无法直接观察的事情”,而满足横波条件的光子应该被保留,介质的光学性质就是在描述介质和它们的相互作用。
这样不会有重复计数。

还有一些时候,不满足横波条件的光子实际上是我们讨论的那部分系统和一个非常“重”的系统(它基本上不会受到我们讨论的那部分系统的影响,可以看成一个背景)交换的,即是外场的一部分。

积掉光子的过程实际上就是计算所有外线都是电子的散射振幅的过程。
我们下面要做的计算都是在高能场论中完成的;\eqref{eq:pauli-eq}只是关于电子旋量场的一部分分量的,从而,我们下面获得的等效相互作用在加入\eqref{eq:pauli-eq}时可能需要适当地作用投影算符。
凝聚态物理能标下基本上只需要计算树图,本节也不计算圈图。

\subsubsection{库伦相互作用}

在非相对论情况下,光子的能量不足以激发出电子-正电子对,真空极化并不重要。
因此,光子虽然是无能隙的,仍然可以积掉光子而得到电子-电子等效相互作用。
我们将其它场看成背景场,因为QED中没有光子-光子相互作用顶角,这个等效相互作用背后的实际的QED过程只包含树图。

我们首先直接在QED中计算电子-电子等效相互作用,即使用旋量场$\psi$做计算,这相当于将所有的“电子发出一个光子,这个光子被另一个电子吸收”的子图用一个电子-电子等效相互作用替代,而不去碰“电子发出一个光子,这个光子被一个不是电子的东西吸收”,或是“一个不知道哪儿来的光子被电子吸收”的过程(这些过程可能又被做了别的近似,比如说如果系统中充满了“不知道哪儿来的光子”,那就可以把电磁场当成经典场,等等)。%
\footnote{
    这里说的“电子”指的都是我们的理论考虑的\emph{那部分}电子。
    “无穷远处射来一束光,被电子吸收”这个过程显然在我们的理论中不是电子-电子相互作用,但是把视角调远一些,很可能这个“无穷远处射来的光”本身是另一些电子辐射产生的,例如它可能来自原子能级跃迁产生的辐射,但是能级跃迁产生的辐射说到底还是电子产生的。
    然而,也许在我们讨论的问题的能标下,原子可以看成一个整体,那么,这个过程在我们的低能有效理论中就不是电子-电子相互作用。
}%
这么做了之后,不计入电子-电子等效相互作用的那些过程仍然是电子与电磁场耦合的过程,从而仍然可以得到\eqref{eq:minimal-coupling}。
于是最终我们就得到一个由\eqref{eq:minimal-coupling},一个电子-电子相互作用,和电磁场自身的哈密顿量三部分拼在一起的模型,其中\eqref{eq:minimal-coupling}中的$\varphi$和$\vb*{A}$由电子激发出来的部分不能作用在电子上(从而避免了重复计数)。
这实际上就是凝聚态场论涉及电子和光子的部分(涉及声子的部分还未加入,因为此时尚无晶格)。

我们也可以换一个等价的思路,首先直接做非相对论极限得到\eqref{eq:minimal-coupling},其中的$\vb*{A}$和$\varphi$由电子激发出来的部分可以作用在电子上。
然后我们假定电子运动速度非常慢,从而将所有的“电子发出一个光子,这个光子被另一个电子吸收”的子图用一个电子-电子等效相互作用替代,而不去碰“电子发出一个光子,这个光子被一个不是电子的东西吸收”,或是“一个不知道哪儿来的光子被电子吸收”的过程。

这两种思路会得到一样的理论,在其中一方面光子场仍然是实际存在的场而没有被真的积掉,一方面纯粹的电子-光子-电子相互作用(即光子只是内线)却又不用考虑,因为它们已经被等效的库仑相互作用考虑进去了。
这看起来很不自然,但在很多情况下——例如,在凝聚态物理,包括量子光学中——光子可以分成截然不同的两部分,一部分是物质之间交换的、基本上导致一个库伦相互作用的光子,一部分是遥远处的光源产生的“入射光”;前者是有源无旋的,后者是无源有旋的。
这是两种光子模式,彼此没有任何相互作用(能标足够低,不会有QED的四光子等效相互作用),并且前者从来不会出现在外线中。
这种情况下使用等效的电子-电子库伦相互作用就是非常合理的:我们马上要做的计算表明电子之间由光子传递的相互作用的头阶项是$\varphi$或者说$A^0$传递的库仑相互作用,这已经将有源无旋的光子产生的相互作用完全囊括了;然后我们就可以假定,在我们的模型中显式出现的光子都是无源有旋的、和我们通常对“电磁波”的想象一致的那部分光子。
当然我们也总是可以从头做计算,不引入等效的电子-电子库伦相互作用,计算每个电子受外场驱动而产生的辐射(包括近场和远场)。
等效电子-电子库伦相互作用实际上就是这里的近场辐射的无推迟时间近似而已。

给出等效电子-电子相互作用的图有两张:
\[
    \begin{tikzpicture}
        \begin{feynhand}
            \vertex (a) at (-1.5, 0.8);
            \vertex (b) at (-1.5, -0.8);
            \vertex (c) at (-1, 0);
            \vertex (d) at (0, 0);
            \vertex (e) at (0.5, 0.8);
            \vertex (f) at (0.5, -0.8);

            \propag[anti fermion] (a) to (c);
            \propag[fermion] (b) to (c);
            \propag[photon] (c) to (d);
            \propag[fermion] (d) to (e);
            \propag[anti fermion] (d) to (f);
        \end{feynhand}
    \end{tikzpicture}, \quad \quad 
    \begin{tikzpicture}
        \begin{feynhand}
            \vertex (a) at (-1.5, 0.8);
            \vertex (b) at (-1.5, -0.8);
            \vertex (c) at (-1, 0);
            \vertex (d) at (0, 0);
            \vertex (e) at (0.5, 0.8);
            \vertex (f) at (0.5, -0.8);

            \propag[anti fermion] (e) to (c);
            \propag[fermion] (b) to (c);
            \propag[photon] (c) to (d);
            \propag[fermion] (d) to (a);
            \propag[anti fermion] (d) to (f);
        \end{feynhand}
    \end{tikzpicture}
\]
如果参与散射的两个粒子是可以分辨的(即除了动量和自旋以外还有别的标签可以区分它们),那么第二张图和第一张图不在一个相互作用通道中。
低能过程由于动量低,相应的特征尺度很大,即粒子不会离得很近,这种情况下粒子的“位置”近似起到了区分两个粒子的标签的作用。这意味着第二张图可以忽略。

于是我们计算第一张图,它给出
\[
    \begin{gathered}
        \begin{tikzpicture}
            \begin{feynhand}
                \vertex (a) at (-1.5, 0.8);
                \vertex (b) at (-1.5, -0.8);
                \vertex (c) at (-1, 0);
                \vertex (d) at (0, 0);
                \vertex (e) at (0.5, 0.8);
                \vertex (f) at (0.5, -0.8);
    
                \propag[fermion] (c) to [edge label={$p'$}] (a);
                \propag[fermion] (b) to [edge label={$p$}] (c);
                \propag[photon] (c) to (d);
                \propag[fermion] (d) to [edge label={$k'$}] (e);
                \propag[fermion] (f) to [edge label={$k$}] (d);
            \end{feynhand}
        \end{tikzpicture}
    \end{gathered} = (-\ii e)^2 \bar{u}(p') \gamma^\mu u(p) \frac{-\ii \eta_{\mu \nu}}{(p' - p)^2 + \ii 0^+} \bar{u}(k') \gamma^\nu u(k).
\]
我们考虑$p, p', k, k'$都几乎是零的情况,并且只对$\mu=\nu=0, 3$的情况求和——其实我们会看到,$\mu = \nu = 1, 2$两种情况并不会有贡献。
此时计算会发现
\[
    \bar{u}(p') \gamma^0 u(p) = u^\dagger(p') u(p) \approx m \pmqty{\xi^\dagger & \xi^\dagger} \pmqty{\xi \\ \xi} = 2m \sigma^0,
\]
而
\[
    \bar{u}(p') \gamma^i u(p) = u^\dagger(p') \pmqty{\dmat{- \sigma^i, \sigma^i}} u(p) \approx m \pmqty{\xi^\dagger & \xi^\dagger} \pmqty{\dmat{- \sigma^i, \sigma^i}} \pmqty{\xi \\ \xi} = 0.
\]
于是我们就得到
\[
    \begin{aligned}
        \begin{gathered}
            \begin{tikzpicture}
                \begin{feynhand}
                    \vertex (a) at (-1.5, 0.8);
                    \vertex (b) at (-1.5, -0.8);
                    \vertex (c) at (-1, 0);
                    \vertex (d) at (0, 0);
                    \vertex (e) at (0.5, 0.8);
                    \vertex (f) at (0.5, -0.8);
        
                    \propag[fermion] (c) to [edge label={$p'$}] (a);
                    \propag[fermion] (b) to [edge label={$p$}] (c);
                    \propag[photon] (c) to (d);
                    \propag[fermion] (d) to [edge label={$k'$}] (e);
                    \propag[fermion] (f) to [edge label={$k$}] (d);
                \end{feynhand}
            \end{tikzpicture}
        \end{gathered} &= (-\ii e)^2 \bar{u}(p') \gamma^\mu u(p) \frac{-\ii \eta_{\mu \nu}}{(p' - p)^2 + \ii 0^+} \bar{u}(k') \gamma^\nu u(k) \\
        &= \ii e^2 2 m (\sigma^0)_p \frac{1}{(p' - p)^2 + \ii 0^+} 2m (\sigma^0)_k \\
        &= - \frac{\ii e^2 (2m)^2 \sigma^0}{\abs*{\vb*{p}' - \vb*{p}}^2 - \ii 0^+}.
    \end{aligned}
\]
这里由于我们取了非相对论极限,有
\[
    p^\mu = (m, \vb*{p}),
\]
于是
\[
    (p' - p)^2 = - \abs*{\vb*{p} - \vb*{p}'}^2.
\]
上式给出的是$S$矩阵的矩阵元,考虑到外线全部是电子,它是$\{\ket*{p}\}$表象下的等效相互作用哈密顿量矩阵元;下标$p$和$k$用于区分作用在不同单粒子态上的矩阵。
矩阵$\sigma^0$给出了自旋的变化情况,可以看到以上相互作用通道不挑选入射自旋,也不改变入射自旋。
我们要做非相对论近似,所以要转换到非洛伦兹协变的、在单粒子量子力学中使用的$\{\ket*{\vb*{p}}\}$表象下,由于有四条外线,要除以因子$(\sqrt{2m})^{4}$。
于是非相对论极限下,我们获得相互作用顶角
\begin{equation}
    \begin{gathered}
        \begin{tikzpicture}
            \begin{feynhand}
                \vertex (a) at (-1.5, 0.8);
                \vertex (b) at (-1.5, -0.8);
                \vertex (c) at (-1, 0);
                \vertex (d) at (0, 0);
                \vertex (e) at (0.5, 0.8);
                \vertex (f) at (0.5, -0.8);
    
                \propag[fermion] (c) to [edge label={$p', \alpha$}] (a);
                \propag[fermion] (b) to [edge label={$p, \alpha$}] (c);
                \propag[photon] (c) to (d);
                \propag[fermion] (d) to [edge label={$k', \beta$}] (e);
                \propag[fermion] (f) to [edge label={$k, \beta$}] (d);
            \end{feynhand}
        \end{tikzpicture}
    \end{gathered} = -\ii \frac{e^2}{\abs*{\vb*{p} - \vb*{p}'}^2} (2\pi)^4 \delta^4(k' + p' - p - k).
\end{equation}
这个相互作用顶角的形式实际上正是动量空间中的库伦定律。
为了更加清晰地看出库伦定律,我们将上式切换回实空间,做傅里叶变换
\[
    \begin{aligned}
        \int \frac{\dd[4]{p'}}{(2\pi)^4} \ee^{\ii p' \cdot x_1} \int \frac{\dd[4]{k'}}{(2\pi)^4} \ee^{\ii k' \cdot x_2} \int \frac{\dd[4]{p}}{(2\pi)^4} \ee^{- \ii p \cdot x_3} \int \frac{\dd[4]{p}}{(2\pi)^4} \ee^{- \ii p \cdot x_4},
    \end{aligned}
\]
计算发现
\begin{equation}
    \begin{aligned}
        \begin{gathered}
            \begin{tikzpicture}
                \begin{feynhand}
                    \vertex (a) at (-1.7, 1) {$x_1, \alpha$};
                    \vertex (b) at (-1.7, -1) {$x_3, \alpha$};
                    \vertex (c) at (-1, 0);
                    \vertex (d) at (0, 0);
                    \vertex (e) at (0.7, 1) {$x_2, \beta$};
                    \vertex (f) at (0.7, -1) {$x_4, \beta$};
        
                    \propag[fermion] (c) to (a);
                    \propag[fermion] (b) to (c);
                    \propag[photon] (c) to (d);
                    \propag[fermion] (d) to (e);
                    \propag[fermion] (f) to (d);
                \end{feynhand}
            \end{tikzpicture}
        \end{gathered} &= -\ii e^2 \delta(t_4 - t_1) \delta^4(x_1 - x_3) \delta^4(x_2 - x_4) \int \frac{\dd[3]{\vb*{q}}}{(2\pi)^3} \frac{\ee^{-\ii \vb*{q} \cdot (\vb*{x}_4 - \vb*{x}_1)}}{\abs*{\vb*{q}}^2 - \ii 0^+} \\
        &= -\ii \delta(t_4 - t_1) \delta^4(x_1 - x_3) \delta^4(x_2 - x_4) \frac{e^2}{4\pi \abs*{\vb*{x}_4 - \vb*{x}_1}}.
    \end{aligned}
    \label{eq:coulomb-interaction}
\end{equation}
因此我们的确得到了库伦相互作用。在计算时有一个细节:计算\eqref{eq:coulomb-interaction}的第一个等号右边的积分时,我们有
\[
    \begin{aligned}
        \int \frac{\dd[3]{\vb*{q}}}{(2\pi)^3} \frac{\ee^{-\ii \vb*{q} \cdot (\vb*{x}_4 - \vb*{x}_1)}}{\abs*{\vb*{q}}^2 - \ii 0^+} &= \frac{2\pi}{(2\pi)^3} \int_0^{\pi} \sin \theta \dd{\theta} \int_0^\infty q^2 \dd{q} \frac{\ee^{- \ii q \abs*{\vb*{x}_4 - \vb*{x}_1} \cos \theta}}{\abs*{\vb*{q}}^2 - \ii 0^+} \\
        &= \frac{1}{4\pi^2} \int_0^\infty \frac{q^2}{q^2 - \ii 0^+} \dd{q} \frac{\ee^{- \ii q \abs*{\vb*{x}_4 - \vb*{x}_1}} - \ee^{\ii q \abs*{\vb*{x}_4 - \vb*{x}_1}}}{- \ii q \abs*{\vb*{x}_4 - \vb*{x}_1}} \\
        &= \frac{1}{4\pi^2 \ii} \int_{-\infty}^\infty \frac{q^2}{q^2 - \ii 0^+} \dd{q} \frac{\ee^{\ii q \abs*{\vb*{x}_4 - \vb*{x}_1}}}{q \abs*{\vb*{x}_4 - \vb*{x}_1}}.
    \end{aligned}
\]
如果我们将因子$q^2/(q^2 - \ii 0^+)$直接当成$1$,上式就没有确定的值了,因为极点直接出现在了积分路径上。
不过,我们有
\[
    \frac{q^2}{q^2 - \ii \epsilon} = \frac{1}{1 - \frac{\ii \epsilon}{q^2}} \to 0 \ \ \text{as $q \to 0$},
\]
因此我们应该取积分主值,即取
\[
    \int \frac{\dd[3]{\vb*{q}}}{(2\pi)^3} \frac{\ee^{-\ii \vb*{q} \cdot (\vb*{x}_4 - \vb*{x}_1)}}{\abs*{\vb*{q}}^2 - \ii 0^+} = \frac{1}{4\pi^2 \ii} \text{P} \int_{-\infty}^\infty \dd{q} \frac{\ee^{\ii q \abs*{\vb*{x}_4 - \vb*{x}_1}}}{q \abs*{\vb*{x}_4 - \vb*{x}_1}} = \frac{1}{4\pi^2 \ii} \frac{\pi \ii}{\abs*{\vb*{x}_4 - \vb*{x}_1}} = \frac{1}{4\pi \abs*{\vb*{x}_4 - \vb*{x}_1}}. 
\]

单粒子量子力学中的散射理论相当于梯形图近似。

从上面的计算可以看出,光子传播子和电子-电子等效相互作用势能是等效的;之后考虑真空极化时,我们可以认为库伦势能被修正了——这正是“真空极化”的说法的来源,因为好像真空自己作为一种介质而导致了对纯库伦相互作用的修正一样。

\subsubsection{自旋磁矩}\label{sec:spin-magnetic-moment}

在\eqref{eq:pauli-eq}和\eqref{eq:minimal-coupling}中我们看到,由于算符不对易这一特点,哈密顿量中除了经典的机械动能项(正则动量$\vb*{p}$减去电磁动量$e \vb*{A}$得到机械动量)以外还多出来一项。
电子自旋算符为
\[
    \vb*{S} = \frac{\vb*{\sigma}}{2},
\]
于是这一项就是
\begin{equation}
    H_\text{spin} = - \vb*{\mu} \cdot \vb*{B}, 
\end{equation}
其中
\begin{equation}
    \vb*{\mu} = \frac{e \vb*{\sigma}}{2m} = \frac{e}{m} \vb*{S} = - \frac{\abs*{e}}{m} \vb*{S} = - g \frac{\abs*{e}}{2m} \vb*{S}.
\end{equation}
这意味着电子即使在静止时也有磁矩,这个磁矩是来自其自旋而不是轨道运动的。$g=2$是自旋磁矩的朗德$g$因子。

% TODO:电子磁矩之间的相互作用
上面的推导实际上是将$\vb*{B}$当成了一个完全确定的外场;然而,电磁场存在量子涨落。
电子磁矩是实验可测的。粗略的实验确实指出$g=2$,更加精确的实验(基本上远远超过了凝聚态场论的能标)则显示$g$其实比$2$稍微大一些,并且计算QED圈图修正之后能够得到和实验测量结果非常接近的值,这是一个证明QED可靠性的有力论据。

\section{低能一圈图辐射修正}

并非所有辐射修正都需要完整地做重整化。可以设想,对那些入射动量真的能够保证接近于零的过程(如那些非相对论近似下就能出现的过程),应该有办法不做完整的重整化就得到有限的结果。

\subsection{顶角函数和反常磁矩}\label{sec:abnormal-magnetic}

QED会导致的一个能够观测的效应是\concept{反常磁矩},即自旋磁矩实际上不严格是$2$。
我们此处先给出计算自旋磁矩的大致步骤。我们是要计算如下的图:
\[
    \begin{tikzpicture}
        \begin{feynhand}
            \vertex (a) at (-1.2, 0);
            \vertex[grayblob] (b) at (0, 0) {};
            \vertex (c) at (1.2, 0);
            \vertex[crossdot] (d) at (0, 1.2) {};
            
            \propag[fermion] (a) to (b);
            \propag[fermion] (b) to (c);
            \propag[photon] (d) to (b);
            \end{feynhand}
    \end{tikzpicture},
\]
其中光子线代表一个比较强的、大小基本上确定的外场;这不代表不会有量子涨落。
在上图中所有的量子涨落可以归结入顶角函数中;电子线由于amputating,无需自能修正,而光子线代表的外场也可以默认已经是amputated的。
因此,严格计算电子自旋磁矩就归结为计算顶角函数。

在本节中,我们将计算一圈顶角函数修正,从中得到反常磁矩,并做重整化。

\subsubsection{对称性分析和形状因子}\label{sec:vertex-function-symmetry}

$S$矩阵矩阵元
\[
    \ii \mathcal{M}^\mu = \begin{gathered}
        \begin{tikzpicture}
            \begin{feynhand}
                \vertex (a) at (-1.2, 0);
                \vertex[grayblob] (b) at (0, 0) {};
                \vertex (c) at (1.2, 0);
                \vertex[crossdot] (d) at (0, 1.2) {};
                
                \propag[fermion] (a) to[edge label={$q_1$}] (b);
                \propag[fermion] (b) to[edge label={$q_2$}] (c);
                \propag[photon] (d) to[edge label={$p, \mu$}] (b);
                \end{feynhand}
        \end{tikzpicture}
    \end{gathered}
\]
是一个矢量,并且费曼图的形式显示它具有$\bar{u} \times \cdots \times u$的形式,而系统中出现的矢量指标只有$\gamma^\mu$以及三个动量,于是我们设
\[
    \ii \mathcal{M}^\mu = \begin{gathered}
        \begin{tikzpicture}
            \begin{feynhand}
                \vertex (a) at (-1.2, 0);
                \vertex[grayblob] (b) at (0, 0) {};
                \vertex (c) at (1.2, 0);
                \vertex[crossdot] (d) at (0, 1.2) {};
                
                \propag[fermion] (a) to[edge label={$q_1$}] (b);
                \propag[fermion] (b) to[edge label={$q_2$}] (c);
                \propag[photon] (d) to[edge label={$p, \mu$}] (b);
                \end{feynhand}
        \end{tikzpicture}
    \end{gathered}
    = \bar{u}(q_2) (f_1 \gamma^\mu + f_2 p^\mu + f_3 q_1^\mu + f_4 q_2^\mu) u(q_1),
\]
其中四个$f_i$因子是可以依赖于动量的复标量,此处复标量包括多分量对象$\slashed{a}$。
这些因子必须是标量或是$\slashed{a}$,因为$\bar{u}$和$u$能够组合成的双线性形式的数目是有限的,以上四种情况穷尽了所有的$\bar{u}$和$u$能够组合成的双线性矢量)。
动量守恒意味着$f_2, f_3, f_4$不完全独立;不失一般性地我们不妨让$f_2=0$。
$f_i$中仅有的多分量对象就是$\slashed{q}_1$和$\slashed{q}_2$,这两者都作用在$u(q_1)$或是$\bar{u}(q_2)$上,通过狄拉克方程
\[
    \slashed{q}_1 u(q_1) = m u(q_1), \quad \bar{u}(q_2) \slashed{q}_2 = m \bar{u}(q_2)
\]
可以消去它们。因此$f_1, f_3, f_4$中只显含$q_1^2, q_2^2, q_1 \cdot q_2$和$m$。
前两者根据自由粒子的色散关系都是$m^2$,因此最终$f_1, f_3, f_4$中只显含$m$和$q_1 \cdot q_2$,或者等价地说$m$和$p^2$;而量纲分析又告诉我们在这里真正重要的是$p^2/m^2$。
使用Ward恒等式,我们有
\[
    \begin{aligned}
        0 &= \ii p_\mu \mathcal{M}^\mu \\
        &= \bar{u}(q_2) (f_1 p_\mu \gamma^\mu + f_3 p_\mu q_1^\mu + f_4 p_\mu q_2^\mu) u(q_1) \\
        &= f_1 \bar{u}(q_2) \slashed{p} u(q_1) + f_3 \bar{u}(q_2) p \cdot q_1 {u}(q_1) + f_4 \bar{u}(q_2) p \cdot q_2 u(q_1),
    \end{aligned}
\]
而
\[
    \begin{aligned}
        \bar{u}(q_2) \slashed{p} u(q_1) &= \bar{u}(q_2) \slashed{q}_2 u(q_1) - \bar{u}(q_2) \slashed{q}_1 u(q_1) \\
        &= m \bar{u}(q_2) u(q_1) - m \bar{u}(q_2) u(q_1) = 0,
    \end{aligned}
\]
且
\[
    \begin{aligned}
        p \cdot q_1 &= (q_2 - q_1) \cdot q_1 = q_2 \cdot q_1 - m^2, \\
        p \cdot q_2 &= (q_2 - q_1) \cdot q_2 = m^2 - q_1 \cdot q_2,
    \end{aligned}
\]
于是就能够证明$f_3=f_4$。因此独立的$f_i$只有两个。这样我们就有
\[
    \begin{aligned}
        \ii \mathcal{M}^\mu &= \bar{u}(q_2) (f_1 \gamma^\mu + f_3 q_1^\mu + f_3 q_2^\mu) u(q_1) \\
        &= (f_1 + 2m f_3) \bar{u}(q_2) \gamma^\mu u(q_1) + \ii f_3 \bar{u}(q_2) \sigma^{\mu \nu} (q_1^\nu - q_2^\nu) u(q_1), 
    \end{aligned}
\]
第二个等号用到了Gordon恒等式
\[
    \bar{u}(q_2) (q_1^\mu + q_2^\mu) u(q_1) = 2m \bar{u}(q_2) \gamma^\mu u(q_1) + \ii \bar{u}(q_2) \sigma^{\mu \nu} (q_1^\nu - q_2^\nu) u(q_1).
\]
这样我们可以形式地写出
\begin{equation}
    \ii \mathcal{M}^\mu = -\ii e \bar{u}(q_2) \left( F_1\left(\frac{p^2}{m^2}\right) \gamma^\mu + \frac{\ii \sigma^{\mu \nu}}{2m} p_\nu F_2\left(\frac{p^2}{m^2}\right) \right) u(q_1),
    \label{eq:form-factor-vertex}
\end{equation}
其中$F_1$和$F_2$被称为\concept{形状因子}。通常将$F_2$限制为\emph{圈图贡献的修正}。本节暂时将树图也纳入它们之中以便说明,最后再恢复正确的记号。

这个散射振幅提供了完整的外场和电子发生单次相互作用的方式。

\subsubsection{一圈顶角函数}

对自旋磁矩涉及的相互作用,我们考虑外场的$\mu=1, 2, 3$分量,因为$\vb*{B}$和电势无关。
\eqref{eq:form-factor-vertex}的第一项和自旋无关,因此我们将要考虑第二项。
此外,在非相对论极限下,$q_1^0=q_2^0=m$,因此在这个极限下$p_0=0$。
因此我们只需要计算
\[
    \ii (\mathcal{M}^\text{spin})^i = - \ii e \bar{u}(q_2) \frac{\ii \sigma^{ij}}{2m} p_j F_2\left(\frac{p^2}{m^2}\right).
\]
由于$\vb*{B} = \curl{\vb*{A}}$,在动量空间中$\vb*{B}=\ii \vb*{p} \times \vb*{A}$,即
\[
    B^i = \ii \epsilon^{ijk} p^j A^k,
\]
因此我们有
\[
    \begin{aligned}
        \ii (\mathcal{M}^\text{spin})^i A_i &= - \ii e \bar{u}(q_2) \frac{\ii \sigma^{ij}}{2m} p_j F_2\left(\frac{p^2}{m^2}\right) u(q_1) A_i \\
        &= - \ii e \bar{u}(q_2) \frac{\ii \sigma^{ij}}{2m} p^j A^i u(q_1) F_2\left(\frac{p^2}{m^2}\right) \\
        &= -\ii e \bar{u}(q_2) \frac{\ii}{2m}  \epsilon^{ijk} \pmqty{\dmat{\sigma^k, \sigma^k}} p_j A_i u(q_1) F_2\left(\frac{p^2}{m^2}\right) \\
        &= - \frac{\ii e}{m} \bar{u}(q_2) \epsilon^{ijk} (\ii p^j) A^i S^k u(q_1) F_2\left(\frac{p^2}{m^2}\right) \\
        &= - \frac{\ii e}{m} ((- \ii \vb*{p} \times \vb*{A}) \cdot \vb*{S}) \bar{u}(q_2) u(q_1) F_2\left(\frac{p^2}{m^2}\right) \\
        &= - \frac{\ii e}{m} (- \vb*{B} \times \vb*{S}) \bar{u}(q_2) u(q_1) F_2\left(\frac{p^2}{m^2}\right).
    \end{aligned}
\]
然后,在非相对论极限下$u(q_1)$和$\bar{u}(q_2)$可以写成$(\xi \ \ \xi)_{\vb*{p}=0}$及其转置,如同在库伦相互作用中那样,我们有
\[
    \bar{u}(q_2) u(q_1) = 2m \sigma^0,
\]
于是
\[
    \ii (\mathcal{M}^\text{spin})^i A_i = - \frac{\ii e}{m} (- \vb*{B} \times \vb*{S}) 2m \sigma^0 F_2(0).
\]
这里我们已经将$F_2$的括号中的$p$取为零,因为是非相对论极限。
最后与推导库伦相互作用时类似,切换到$\ket*{\vb*{p}}$表象下,由于有两条电子外线,除以因子$(\sqrt{2m})^2$,就得到非相对论量子力学中的等效相互作用顶角
\begin{equation}
    \begin{gathered}
        \begin{tikzpicture}
            \begin{feynhand}
                \vertex (a) at (-1.2, 0);
                \vertex[grayblob] (b) at (0, 0) {};
                \vertex (c) at (1.2, 0);
                \vertex[crossdot] (d) at (0, 1.2) {};
                
                \propag[fermion] (a) to[edge label={$q_1$}] (b);
                \propag[fermion] (b) to[edge label={$q_2$}] (c);
                \propag[photon] (d) to[edge label={$p, \mu$}] (b);
                \end{feynhand}
        \end{tikzpicture}
    \end{gathered} = - \ii \left( - F_2(0) \frac{e}{m} \vb*{S} \cdot \vb*{B} \right),
\end{equation}
即,电子的自旋和外磁场的相互作用哈密顿量为
\begin{equation}
    H_\text{spin} = - \vb*{\mu} \cdot \vb*{B}, \quad \vb*{\mu} = \frac{e}{m} F_2(0) \vb*{S}.
\end{equation}
$F_2(0)$中树图阶的贡献我们已经在\autoref{sec:spin-magnetic-moment}中看到了,就是$1$,所以现在我们恢复惯常的记号,让$F_2$只包括圈图部分的贡献,就得到
\begin{equation}
    H_\text{spin} = - \vb*{\mu} \cdot \vb*{B}, \quad \vb*{\mu} = \frac{e}{m} (1 + F_2(0)) \vb*{S},
\end{equation}
或者用朗德$g$因子表示,为
\begin{equation}
    g = 2 + 2 F_2(0),
\end{equation}
其中$F_2(0)$给出顶角函数圈图修正带来的修正。

下面我们要做的是计算$F_2(0)$。本节只计算一圈图,仅考虑顶角修正,有
\begin{equation}
    \begin{aligned}
        \ii \mathcal{M}^{(2)} &= \int \frac{\dd[4]{k}}{(2\pi)^4} \bar{u}(q_2) (-\ii e \gamma^\nu) \frac{\ii (\slashed{p} + \slashed{k} + m)}{(p + k)^2 - m^2 + \ii 0^+} (- \ii e \gamma^\mu) \frac{\ii (\slashed{k} + m)}{k^2 - m^2 + \ii 0^+} \\
        &\quad \quad \times (-\ii e \gamma^\sigma) u(q_1) \frac{-\ii \eta_{\sigma \nu}}{(k - q_1)^2 + \ii 0^+} \\
        &= - e^3 \bar{u}(q_2) \int \frac{\dd[4]{k}}{(2\pi)^4} \frac{\gamma^\nu (\slashed{p} + \slashed{k} + m) \gamma^\mu (\slashed{k} + m) \gamma_\nu}{(k^2 - m^2 + \ii 0^+) ((p + k)^2 - m^2 + \ii 0^+) ((k - q_1)^2 + \ii 0^+)} u(q_1).
    \end{aligned}
\end{equation}
为了计算上式我们做费曼参数化,使用公式
\[
    \frac{1}{ABC} = 2 \int_0^1 \dd{x} \dd{y} \dd{z} \delta(x + y + z - 1) \frac{1}{(x A + y B + z C)^3},
\]
就得到
\begin{equation}
    \ii \mathcal{M}^{(2)} = - 2 e^3 \int_0^1 \dd{x} \dd{y} \dd{z} \delta(x + y + z - 1) \int \frac{\dd[4]{k}}{(2\pi)^4} \frac{N^\mu}{D},
    \label{eq:vertex-one-loop-integral}
\end{equation}
其中
\begin{equation}
    N^\mu = \bar{u}(q_2) \gamma^\nu (\slashed{p} + \slashed{k} + m) \gamma^\mu (\slashed{k} + m) \gamma_\nu u(q_1),
\end{equation}
而
\begin{equation}
    D = (x (k^2 - m^2 + \ii 0^+) + y ((p + k)^2 - m^2 + \ii 0^+) + z ((k - q_1)^2 + \ii 0^+))^3.
\end{equation}
上式中,$q_1$和$q_2$是在壳的,而$p$是离壳的。我们有动量守恒关系
\begin{equation}
    q_2 = q_1 + p.
\end{equation}

我们首先化简$D$。将其展开,配方,并且利用在壳关系$q_1^2=m^2$,就得到
\[
    \begin{aligned}
        D &= (k^2 + 2y k \cdot p - 2z k \cdot q_1 + yp^2 + zq_1^2 - (x + y)m^2)^3 \\
        &= ((k + yp - z q_1)^2 - y^2 p^2 - z^2 q_1^2 + 2 yz p \cdot q_1 + y p^2 + z q_1^2 - (x + y) m^2)^3 \\
        &= ((k + yp - z q_1)^2 - y^2 p^2 + 2 yz p \cdot q_1 + y p^2 - z^2 m^2 + z m^2 - (x + y) m^2)^3.
    \end{aligned}
\]
利用$x + y + z = 1$得到
\[
    - z^2 m^2 + z m^2 - (x + y) m^2 = - (1-z)^2 m^2.
\]
由于
\[
    p \cdot q_1 = \frac{1}{2} (q_1 + p)^2 - \frac{1}{2} q_1^2 - \frac{1}{2} p^2 = \frac{1}{2} (m^2 - m^2) - \frac{1}{2} p^2 = - \frac{1}{2} p^2,
\]
我们有
\[
    - y^2 p^2 + 2 yz p \cdot q_1 + y p^2 = (-y^2 + y - yz) p^2 = y(1 - y - z) p^2 = xy p^2.
\]
于是就得到
\begin{equation}
    \begin{aligned}
        D &= ((k + yp - z q_1)^2 + xy p^2 - (1-z)^2 m^2)^3 \\
        &= (k'^2 - \Delta)^3,
    \end{aligned}
\end{equation}
其中
\begin{equation}
    k' = k + yp - z q_1, \quad \Delta = - xy p^2 + (1-z)^2 m^2.
\end{equation}
我们下面在对$k$做积分时可以将积分变量换成$k'$以简化计算。

然后我们分析$N^\mu$。展开$\slashed{p}$的定义,并且使用$\gamma$矩阵的代数关系,可以得到
\begin{equation}
    \begin{aligned}
        N^\mu &= \bar{u}(q_2) \gamma^\nu (p_\rho \gamma^\rho + k_\rho \gamma^\rho + m) \gamma^\mu (k_\sigma \gamma^\sigma + m) \gamma_\nu u(q_1) \\
        &= \bar{u}(q_2) (4 \eta^{\rho \mu} m (p_\rho + k_\rho) + (p_\rho + k_\rho) k_\sigma (- 2 \gamma^\sigma \gamma^\mu \gamma^\rho) + 4 \eta^{\sigma \mu} m k_\sigma + m^2 (-2 \gamma^\mu)) u(q_1) \\
        &= \bar{u}(q_2) (4 m (p^\mu + 2 k^\mu) - 2 \slashed{k} \gamma^\mu (\slashed{p} + \slashed{k}) - 2 m^2 \gamma^\mu) u(q_1).
    \end{aligned}
\end{equation}

到了这一步,使用原本的$k$就很难计算下去了。我们现在将$k'$当成$k$,即在所有表达式中,做替换
\begin{equation}
    k \longrightarrow k + z q_1 - y p,
    \label{sec:vertex-one-loop-k-replacement}
\end{equation}
就得到
\begin{equation}
    D = (k^2 - \Delta)^3,
    \label{eq:final-d}
\end{equation}
而
\begin{equation}
    - \frac{1}{2} N^\mu = \bar{u}(q_2) (- 2 m (p^\mu + 2 k^\mu + 2 z q_1^\mu - 2 y p^\mu) + (\slashed{k} + z \slashed{q}_1 - y \slashed{p}) \gamma^\mu (\slashed{p} + \slashed{k} + z \slashed{q}_1 - y \slashed{p}) + m^2 \gamma^\mu) u(q_1).
\end{equation}
然后我们需要利用几个关系——Gordon恒等式,狄拉克方程的动量空间形式和$x+y+z=1$——来化简上式。
此外,应注意到$D$对$k$是偶函数,从而$N^\mu$相对$k$是奇函数的部分全部可以略去,既然$k$的取值范围遍布全空间。
在上式第一项中,$k^\mu$项可以直接略去。在上式第二项中我们有
\begin{equation}
    \begin{aligned}
        &\quad \bar{u}(q_2) (\slashed{k} + z \slashed{q}_1 - y \slashed{p}) \gamma^\mu (\slashed{p} + \slashed{k} + z \slashed{q}_1 - y \slashed{p}) u(q_1) \\
        &= \bar{u}(q_2) (\slashed{k} + z \slashed{q}_2 - (y + z) \slashed{p}) \gamma^\mu (\slashed{k} + z \slashed{q}_1 + (1 - y) \slashed{p}) u(q_1) \\
        &= \bar{u}(q_2) (\slashed{k} + z m - (y + z) \slashed{p}) \gamma^\mu (\slashed{k} + z m + (1 - y) \slashed{p}) u(q_1).
    \end{aligned}
    \label{eq:second-term-n-mu}
\end{equation}
以上计算中用到了
\[
    \bar{u}(q_2) \slashed{q}_2 = m \bar{u}(q_2), \quad \slashed{q}_1 u(q_1) = m u(q_1).
\]
\eqref{eq:second-term-n-mu}中又有一些$k$的奇函数项,可以直接略去。不能略去的项包括$\slashed{k} \gamma^\mu \slashed{k}$项,$\slashed{p} \gamma^\mu$项及其转置,以及$\slashed{p} \gamma^\mu \slashed{p}$项。
在计算这些项时,我们有这样一个有用的公式:在所有量都是洛伦兹协变的时候,有
\begin{equation}
    \int \dd[4]{k} k_\mu k_\nu \times (\cdots) = \int \dd[4]{k} \frac{1}{4} \eta_{\mu \nu} k^2 \times (\cdots),
    \label{eq:k-mu-nu-to-square}
\end{equation}
其推导是,首先注意到$\mu \neq \nu$时积分为零,而上式左边给出一个张量,满足这个条件的张量只有度规张量的某个倍数;系数$1/4$是通过在等式两边同时乘上$\eta^{\mu \nu}$并缩并而计算出来的。
于是
\[
    \begin{aligned}
        \slashed{k} \gamma^\mu \slashed{k} &= k_\rho k_\sigma \gamma^\rho \gamma^\mu \gamma^\sigma = \frac{1}{4} \eta_{\rho \sigma} k^2 \gamma^\rho \gamma^\mu \gamma^\sigma \\
        &= \frac{1}{4} k^2 \gamma^\sigma \gamma^\mu \gamma_\sigma = - \frac{1}{2} k^2 \gamma^\mu,
    \end{aligned}
\]
而
\[
    \gamma^\mu \slashed{p} = p_\sigma \frac{\acomm*{\gamma^\sigma}{\gamma^\mu} - \comm*{\gamma^\sigma}{\gamma^\mu}}{2} = p_\sigma \frac{2 \eta^{\sigma \mu} - 2 \ii \sigma^{\mu \sigma}}{2} = p^\mu - \ii \sigma^{\mu \sigma} p_\sigma,
\]
同理
\[
    \slashed{p} \gamma^\mu = p^\mu + \ii \sigma^{\mu \sigma} p_\sigma.
\]
对$\slashed{p} \gamma^\mu \slashed{p}$项则有
\[
    \begin{aligned}
        \slashed{p} \gamma^\mu \slashed{p} &= p_\rho p_\sigma \left( 2 \gamma^\rho \eta^{\mu \sigma} - \frac{\acomm*{\gamma^\rho}{\gamma^\sigma} + \comm*{\gamma^\rho}{\gamma^\sigma}}{2} \gamma^\mu \right),
    \end{aligned}
\]
注意到$p_\rho p_\sigma \comm*{\gamma^\rho}{\gamma^\sigma}$是零,因为对易子反对称,于是
\[
    \slashed{p} \gamma^\mu \slashed{p} = 2 \slashed{p} p^\mu - p^2 \gamma^\mu,
\]
而这个式子中的第一项又应当略去,因为
\[
    \bar{u}(q_2) \slashed{p} u(q_1) = \bar{u}(q_2) (\slashed{q}_2 - \slashed{q}_1) u(q_1) = 0.
\]
因此
\[
    \slashed{p} \gamma^\mu \slashed{p} = - p^2 \gamma^\mu.
\]
综合以上各式,现在我们已经把$N^\mu$写成不带有$\slashed{p}$的形式了:
\begin{equation}
    \begin{aligned}
        - \frac{1}{2} N^\mu &= \bar{u}(q_2) (
            \gamma^\mu (- \frac{1}{2} k^2 + m^2 + z^2 m^2 + (y+z)(1-y) p^2) \\
            &\quad \quad - 2m ((1-2y) p^\mu + 2z q_1^\mu) \\
            &\quad \quad + zm((1-y) (p^\mu - \ii \sigma^{\mu \sigma} p_\sigma) - (y+z) (p^\mu + \ii \sigma^{\mu \sigma} p_\sigma))
        ) u(q_1).
    \end{aligned}
    \label{eq:no-slashed-p-n-mu}
\end{equation}
最后,将括号中的第二项和第三项放在一起,得到
\[
    \begin{aligned}
        &\quad - 2m ((1-2y) p^\mu + 2z q_1^\mu) + zm((1-y) (p^\mu - \ii \sigma^{\mu \sigma} p_\sigma) - (y+z) (p^\mu + \ii \sigma^{\mu \sigma} p_\sigma)) \\
        &= m((4y - 2 + z - 2yz - z^2) p^\mu - z(1+z) \ii \sigma^{\mu \nu} p_\nu - 4z q_1^\mu) \\
        &= m((4y - 2 + 3z - 2yz - z^2) p^\mu - z(1+z) \ii \sigma^{\mu \nu} p_\nu - 2z (q_1^\mu + q_2^\mu)) \\
        &= m(- (z-2)(z+2y-1) p^\mu - z(1+z) \ii \sigma^{\mu \nu} p_\nu - 2z (q_1^\mu + q_2^\mu)) ,
    \end{aligned}
\]
使用Gordon恒等式,将$q_1 + q_2$用$2m \gamma^\mu + \ii \sigma^{\mu \nu}(q_1^\nu - q_2^\nu)$代替,得到
\[
    \begin{aligned}
        &\quad - 2m ((1-2y) p^\mu + 2z q_1^\mu) + zm((1-y) (p^\mu - \ii \sigma^{\mu \sigma} p_\sigma) - (y+z) (p^\mu + \ii \sigma^{\mu \sigma} p_\sigma)) \\
        &= m(- (z-2)(z+2y-1) p^\mu - z(1+z) \ii \sigma^{\mu \nu} p_\nu - 2z (2m \gamma^\mu - \ii \sigma^{\mu \nu} p_\nu)) \\
        &= m((z-2)(x-y) p^\mu + z(1 - z) \ii \sigma^{\mu \nu} p_\nu - 4mz \gamma^\mu ) ,
    \end{aligned}
\]
然后我们再将上式和\eqref{eq:no-slashed-p-n-mu}右边括号中的第一项放在一起,得到
\begin{equation}
    \begin{aligned}
        - \frac{1}{2} N^\mu &= \bar{u}(q_2) (\gamma^\mu (- \frac{1}{2} k^2 + (z^2 - 4z + 1) m^2 + (1-x)(1-y) p^2) \\
        &\quad \quad + \ii m z (1-z) \sigma^{\mu \nu} p_\nu + m (z-2) (x-y) p^\mu) u(q_1).
    \end{aligned}
    \label{eq:final-n-mu}
\end{equation}

最后,我们将\eqref{eq:final-d}和\eqref{eq:final-n-mu}放在一起,计算对$x, y, z, p$的积分。
$N^\mu$中的三项会让\eqref{eq:vertex-one-loop-integral}中出现三项,而按照\eqref{eq:form-factor-vertex},只应该有两项。
实际上我们会注意到$\Delta$中交换$x$和$y$不变,而\eqref{eq:final-n-mu}中,$p^\mu$这一项中$x$和$y$交换会多一个负号,因此$p^\mu$项对\eqref{eq:vertex-one-loop-integral}其实是没有贡献的。
实际上,这个事实对应着Ward恒等式,因为\eqref{eq:final-n-mu}中的另外两项中,
\[
    \bar{u}(q_2) p^\mu \gamma^\mu u(q_1) = \bar{u}(q_2) (\slashed{q}_2 - \slashed{q}_1) u(q_1) = \bar{u}(q_2) (m - m)u(q_1) = 0, \quad p_\mu \sigma^{\mu \nu} p_\nu = 0,
\]
因此
\[
    p_\mu N^\mu = 0, \quad p_\mu \mathcal{M}^\mu = 0.
\]
这就是Ward恒等式。
与\eqref{eq:form-factor-vertex}对比,我们能够得到
\begin{equation}
    \begin{aligned}
        F_1\left(\frac{p^2}{m^2}\right) &= 4 \ii e^2 \int_0^1 \dd{x} \dd{y} \dd{z} \delta(x + y + z - 1) \\
        &\quad \quad \times \int \frac{\dd[4]{k}}{(2\pi)^4} \frac{- \frac{1}{2} k^2 + (z^2 - 4z + 1) m^2 + (1-x) (1-y) p^2}{(k^2 - \Delta)^3},
    \end{aligned}
    \label{eq:f1-in-4d}
\end{equation}
而
\begin{equation}
    F_2\left(\frac{p^2}{m^2}\right) = 8 \ii e^2 m^2 \int_0^1 \dd{x} \dd{y} \dd{z} \delta(x + y + z - 1) \int \frac{\dd[4]{k}}{(2\pi)^4} \frac{z(1-z)}{(k^2 - \Delta)^3}.
\end{equation}

要计算反常磁矩只需要计算$F_2$。由于
\[
    \int \frac{\dd[4]{k}}{(2\pi)^4} \frac{1}{(k^2 - \Delta)^3} = - \frac{\ii}{32 \pi^2 \Delta},
\]
我们有
\[
    \begin{aligned}
        F_2\left(\frac{p^2}{m^2}\right) &= \frac{e^2 m^2}{4\pi^2} \int_0^1 \dd{x} \dd{y} \dd{z} \delta(x + y + z - 1) \frac{z(1-z)}{(1-z)^2 m^2 - xy p^2} \\
        &= \frac{\alpha}{\pi} m^2 \int_0^1 \dd{x} \dd{y} \dd{z} \delta(x + y + z - 1) \frac{z(1-z)}{(1-z)^2 m^2 - xy p^2}.
    \end{aligned}
\]
最后,我们可以计算出反常磁矩的朗德$g$因子了:
\[
    \begin{aligned}
        F_2(0) &= \frac{\alpha}{\pi} m^2 \int_0^1 \dd{x} \dd{y} \dd{z} \delta(x + y + z - 1) \frac{z(1-z)}{(1-z)^2 m^2} \\
        &= \frac{\alpha}{\pi} \int_0^1 \dd{x} \dd{y} \dd{z} \frac{z}{1-z} \\
        &= \frac{\alpha}{2\pi},
    \end{aligned}
\]
于是
\begin{equation}
    g = 2 + 2 F_2(0) = 2 + \frac{\alpha}{\pi}.
\end{equation}
这给出了电子的磁矩的正比于$\alpha$的第一阶修正。

我们的运气是很好的,在计算过程中没有遇到发散,因为发散都在$F_1$里面了。

\section{一圈图的正规化和重整化}

本节更加系统地计算一圈图修正。我们现在要做的事情是:
\begin{enumerate}
    \item 画出电子传播子、光子传播子和顶角函数的一圈图修正;
    \item 根据费曼规则写出这些图(其中所有“外线”均不在壳)的表达式,然后依次做:
    \begin{enumerate}
        \item $\gamma$矩阵缩并;
        \item 将$1/(k^2 - m^2)$连乘做费曼参数化或者说费曼折叠,引入一系列从$0$到$1$的积分;
        \item 做出动量积分,可能需要做适当的动量平移以便于套用一些公式;
        \item 完成费曼参数化中引入的积分。
    \end{enumerate}
    在此过程中用维数正规化的$d = 4 - \epsilon$将积分中的紫外发散部分提取出来。
    如果有红外发散,需要给光子加一个非常小的质量,以避免软光子散射。
    \item 根据前述的紫外发散部分,适当引入对应于电子传播子、光子传播子和顶角函数的抵消项。
    \item 计算电子传播子、光子传播子和顶角函数的一圈图修正和抵消项的贡献的和,消去所有的发散,得到修正后的有限的电子传播子、光子传播子和顶角函数。
    \item 用这三个量组合得到需要的散射振幅。
\end{enumerate}

\subsection{三种紫外发散的一圈图的维数正规化}

\subsubsection{真空极化}

极化指的是介质中光子的传播偏离经典电动力学中真空的情况,在QED中真空极化实际上就是光子自能,是光子传播子的圈图修正。
对称性分析表明光子自能的表达式一定是某个关于光子动能$p$的标量乘以某个正比于$p$的二次项的二阶张量。
关于$p$的标量当然一定是$p^2$的函数,而用$p$能够构造出来的二阶张量无非是$p^\mu p^\nu$和$p^2 \eta^{\mu \nu}$,因此光子自能的一般形式是
\begin{equation}
    \Pi^{\mu \nu}(p) = \Delta_1(p^2, m^2) p^2 \eta^{\mu \nu} + \Delta_2(p^2, m^2) p^\mu p^\nu.
\end{equation}
然而,这第二项实际上没有太大作用,因为QED是规范场论,我们可以通过调整$\xi$来消掉这一项,而不产生其它任何影响。
换而言之,在之后计算光子自能时,但凡出现$p^\mu p^\nu$项的,全部可以忽略——它们除了让$\xi$需要调整以下以外什么也不导致。

光子的正规自能中的一圈图是
\begin{equation}
    \begin{aligned}
        \ii \Pi^{\mu \nu} &= \begin{gathered}
            \begin{tikzpicture}
                \begin{feynhand}
                    \vertex (a) at (-1.5, 0);
                    \vertex (b) at (-0.5, 0);
                    \vertex (c) at (0.5, 0);
                    \vertex (d) at (1.5, 0);
                    \propag[photon, mom={$p$}] (a) to (b);
                    \propag[fermion] (b) to[half right, looseness=1.5, edge label'={$k$}] (c);
                    \propag[fermion] (c) to[half right, looseness=1.5, edge label'={$k-p$}] (b);
                    \propag[photon, mom={$p$}] (c) to (d);
                \end{feynhand}
            \end{tikzpicture}
        \end{gathered} \\
        &= - \int \frac{\dd[4]{k}}{(2\pi)^4} \trace (-\ii e \gamma^\mu ) \frac{\ii (\slashed{k} - \slashed{p} + m)}{(k - p)^2 - m^2 + \ii 0^+} (-\ii e \gamma^\nu) \frac{\ii (\slashed{k} + m)}{k^2 - m^2 + \ii 0^+} \\
        &= - e^2 \int \frac{\dd[4]{k}}{(2\pi)^4} \frac{\trace (\gamma^\mu (\slashed{k} - \slashed{p} + m) \gamma^\nu (\slashed{k} + m))}{((k - p)^2 - m^2 + \ii 0^+) (k^2 - m^2 + \ii 0^+)}.
    \end{aligned}
\end{equation}
注意这里我们把光子自能图设成$\ii \Pi$而不是$- \ii \Sigma$,因为光子关联函数的形式相比电子关联函数多出来了一个负号。(但是一些书,如Schwartz,对电子用的也是$\ii \Sigma$)
图中所有的动量都不必在壳。

如果直接动手计算,马上就会发现这里有发散。我们此处做维数正规化,即计算
\begin{equation}
    \Pi^{\mu \nu} = \ii e^2 \int \frac{\dd[d]{k}}{(2\pi)^d} \frac{\trace (\gamma^\mu (\slashed{k} - \slashed{p} + m) \gamma^\nu (\slashed{k} + m))}{((k - p)^2 - m^2 + \ii 0^+) (k^2 - m^2 + \ii 0^+)}.
\end{equation}
首先做费曼参数化,有
\[
    \begin{aligned}
        \frac{1}{((k - p)^2 - m^2 + \ii 0^+) (k^2 - m^2 + \ii 0^+)} &= \int_0^1 \dd{x} \frac{1}{(x (k^2 - m^2) + (1-x) ((k - p)^2 - m^2))^2} \\
        &= \int_0^1 \dd{x} \frac{1}{(k^2 + (1-x) p^2 - 2(1-x) k \cdot p - m^2)^2}.
    \end{aligned}
\]
然后计算分子,根据$\gamma$矩阵的迹公式(这些公式在维数正规化下是不变的),有
\[
    \trace (\gamma^\mu (\slashed{k} - \slashed{p} + m) \gamma^\nu (\slashed{k} + m)) = 4 (2 k^\mu k^\nu - p^\mu k^\nu - k^\mu p^\nu + \eta^{\mu \nu} (m^2 - k^2 + k \cdot p)).
\]
上式中的$p^\mu p^\nu$项显然可以略去,它们只会让$\xi$做一个调整;由对称性,$p^\mu k^\nu$项和$k^\mu p^\nu$项在对$k$积分后显然都会给出正比于$p^\mu p^\nu$的项,从而也可以略去。
最后,为了保持量纲——或者说为了显式地引入$\epsilon$标记的重整化群流的标度变换部分——我们注意到在$d$维空间中$e$的量纲是$2-d/2$,因此如果我们固定$e$为四维空间中的电荷(从而其取值和$\epsilon$无关),则在$d$维空间中应该用$\mu^{2-d/2} e$代替$e$,以手动引入$\epsilon$变动导致的跑动。
因此我们得到
\[
    \Pi^{\mu \nu} = 4 \ii \mu^{4-d} e^2 \int_0^1 \dd{x} \int \frac{\dd[d]{k}}{(2\pi)^d} \frac{2k^\mu k^\nu + \eta^{\mu \nu} (m^2 + k \cdot p - k^2)}{(k^2 + (1-x) p^2 - 2(1-x) k \cdot p - m^2)^2}.
\]
容易观察出,将分母配方的方法是做积分变量平移
\[
    k \longrightarrow k + (1-x) p,
\]
就能够让分母对$k$的依赖只含有$k^2$。此时并再一次忽略所有$p^\mu p^\nu$项,以及,由于此时分母对$k$是偶函数,可以忽略所有的$k \cdot p$项,就有
\begin{equation}
    \Pi^{\mu \nu} = 4 \ii \mu^{4-d} e^2 \int_0^1 \dd{x} \int \frac{\dd[d]{k}}{(2\pi)^d} \frac{2k^\mu k^\nu - \eta^{\mu \nu} (k^2 - x(1-x)p^2 - m^2)}{(k^2 + x (1-x) p^2 - m^2)^2}.
\end{equation}
使用\eqref{eq:k-mu-nu-to-square},注意此时由于维数发生变化,应该使用其$d$维版本,即
\begin{equation}
    \int \dd[4]{k} k_\mu k_\nu \times (\cdots) = \int \dd[4]{k} \frac{1}{d} \eta_{\mu \nu} k^2 \times (\cdots)
\end{equation}
就得到
\[
    \Pi^{\mu \nu} = 4 \ii \mu^{4-d} e^2 \int_0^1 \dd{x} \int \frac{\dd[d]{k}}{(2\pi)^d} \frac{(\frac{2}{d} - 1) \eta^{\mu \nu} k^2 + \eta^{\mu \nu} (x(1-x) p^2 + m^2)}{(k^2 - \Delta)^2},
\]
其中
\[
    \Delta = m^2 - x(1-x) p^2.
\]
使用公式
\[
    \int \frac{\dd[d]{k}}{(2\pi)^d} \frac{k^2}{(k^2 - \Delta)^2} = - \frac{\ii}{(4\pi)^{d/2}} \frac{d}{2} \Gamma\left(1 - \frac{d}{2}\right) \left(\frac{1}{\Delta}\right)^{1-\frac{d}{2}},
\]
以及
\[
    \int \frac{\dd[d]{k}}{(2\pi)^d} \frac{1}{(k^2 - \Delta)^2} = \frac{\ii}{(4\pi)^{d/2}} \Gamma\left(2 - \frac{d}{2} \right) \left(\frac{1}{\Delta}\right)^{2-\frac{d}{2}},
\]
并且在$\Pi^{\mu \nu}$的第一项中做替换
\[
    \left(\frac{1}{\Delta}\right)^{1-\frac{d}{2}} = \left(\frac{1}{\Delta}\right)^{2-\frac{d}{2}} (m^2 - x(1-x) p^2), \quad \Gamma\left(1 - \frac{d}{2}\right) = \frac{\Gamma\left(2 - \frac{d}{2}\right)}{1 - \frac{d}{2}},
\]
计算得到
\[
    \Pi^{\mu \nu} = - 8 p^2 \eta^{\mu \nu} \frac{e^2}{(4\pi)^{d/2}} \mu^{4-d} \Gamma\left(2 - \frac{d}{2}\right) \int_0^1 \dd{x} x(1-x) \left(\frac{1}{m^2 - x(1-x) p^2}\right)^{2 - \frac{d}{2}}.
\]
现在引入$d = 4 - \epsilon$展开,我们得到
\begin{equation}
    \begin{aligned}
        \Pi^{\mu \nu} &= - \frac{8 p^2 \eta^{\mu \nu} e^2}{(4\pi)^2} \int_0^1 \dd{x} x(1-x) \left( 1 + \frac{\epsilon}{2} \ln(\frac{4 \pi \mu^2}{m^2 - x(1-x) p^2}) + \bigO(\epsilon^2) \right) \\
        &\quad \quad \times \left( \frac{2}{\epsilon} - \gamma_\text{E} + \bigO(\epsilon) \right) \\
        &= - \frac{p^2 \eta^{\mu \nu} e^2}{2 \pi^2} \int_0^1 \dd{x} x(1 - x) \left( \frac{2}{\epsilon} + \ln(\frac{\tilde{\mu}^2}{m^2 - x(1-x) p^2}) \right),
    \end{aligned}
    \label{eq:regularized-photon}
\end{equation}
其中我们定义
\begin{equation}
    \tilde{\mu}^2 = 4 \pi \ee^{-\gamma_\text{E}} \mu^2.
\end{equation}
\eqref{eq:regularized-photon}给出正规化之后的光子自能。可以看到的确存在发散,不过发散已经被提取出来了,它是$\epsilon$的一个一阶极点,即
\begin{equation}
    \Pi^{\mu \nu} = - \frac{p^2 e^2 \eta^{\mu \nu}}{6 \pi^2 \epsilon} + \text{finite}.
    \label{eq:photon-one-loop-divergence}
\end{equation}

\subsubsection{电子自能}

电子和光子的相互作用存在相互作用这一事实意味着可以有“电子吸收它自己产生的光子”这样的过程,从而电子会带上“电磁质量”。
这是在经典电动力学中也已经知道的一个现象,但是在经典电动力学中不足以处理自能导致的发散。

我们现在在QED中处理电子自能。电子的正规自能中的一圈图是
\begin{equation}
    \begin{aligned}
        - \ii \Sigma(\slashed{p}) &= \begin{gathered}
            \begin{tikzpicture}
                \begin{feynhand}
                    \vertex (a) at (-1.5, 0);
                    \vertex (b) at (-0.5, 0);
                    \vertex (c) at (0.5, 0);
                    \vertex (d) at (1.5, 0);
                    \propag[fermion] (a) to[edge label={$p$}] (b);
                    \propag[fermion] (b) to[edge label'={$k$}] (c);
                    \propag[fermion] (c) to[edge label={$p$}] (d);
                    \propag[photon, mom={$p - k$}] (b) to[half left, looseness=1.5] (c); 
                \end{feynhand}
            \end{tikzpicture}
        \end{gathered} \\
        &= \int \frac{\dd[4]{k}}{(2\pi)^4} (-\ii e \gamma^\mu) \frac{\ii}{\slashed{k} - m + \ii 0^+} (-\ii e \gamma^\nu) \frac{- \ii \eta_{\mu \nu}}{(p-k)^2 + \ii 0^+} \\
        &= - e^2 \int \frac{\dd[4]{k}}{(2\pi)^4} \gamma^\mu \frac{\slashed{k} + m}{k^2 - m^2 + \ii 0^+} \gamma_\mu \frac{1}{(p-k)^2 + \ii 0^+}.
    \end{aligned}
\end{equation}
图中所有的动量都不必在壳。

仿照先前做光子自能计算时的步骤,首先做维数正规化,并引入因子$\mu$,得到
\begin{equation}
    \Sigma(\slashed{p}) = - \ii \mu^{4-d}  e^2 \int \frac{\dd[d]{k}}{(2\pi)^d} \gamma^\mu \frac{\slashed{k} + m}{k^2 - m^2 + \ii 0^+} \gamma_\mu \frac{1}{(p-k)^2 + \ii 0^+}.
\end{equation}
首先处理分子,根据$\gamma$矩阵的代数公式得到
\[
    \gamma^\mu (\slashed{k} + m) \gamma_\mu = md - (d-2) \slashed{k}.
\]
对分母,做费曼参数化,得到
\[
    \begin{aligned}
        \frac{1}{(k^2 - m^2)(p - k)^2} &= \int_0^1 \dd{x} \frac{1}{(x (p-k)^2 + (1-x) (k^2 - m^2))^2} \\
        &= \int_0^1 \dd{x} \frac{1}{(k^2 - 2 x p \cdot k + x p^2 + (x-1) m^2)^2}.
    \end{aligned}
\]
容易观察出正确的配方方式是做变量替换
\[
    k \longrightarrow k + xp,
\]
此时分母变成$(k^2 - \Delta)^2$,其中
\[
    \Delta = (1 - x)(m^2 - p^2 x),
\]
而分子成为
\[
    md - (d-2)(\slashed{k} + x \slashed{p}).
\]
此时分母是$k$的偶函数,从而分子中是$k$的奇函数的项积分之后都是零,可以略去,因此最终得到
\begin{equation}
    \begin{aligned}
        \Sigma(\slashed{p}) &= - \ii \mu^{4-d} e^2 \int_0^1 \dd{x} \int \frac{\dd[d]{k}}{(2\pi)^d} \frac{md - x (d-2) \slashed{p}}{(k^2 - \Delta)^2} \\
        &= - \ii \mu^{4-d} e^2 \int_0^1 \dd{x} (md - x (d-2) \slashed{p}) \frac{\ii}{(4\pi)^{d/2}} \Gamma\left(2 - \frac{d}{2} \right) \left(\frac{1}{\Delta}\right)^{2-\frac{d}{2}}. 
    \end{aligned}
\end{equation}
做$d = 4 - \epsilon$展开,得到
\begin{equation}
    \begin{aligned}
        \Sigma(\slashed{p}) &= \frac{\ii e^2 \mu^\epsilon}{(4\pi)^2} \int_0^1 \dd{x} \Gamma\left(\frac{\epsilon}{2}\right) (4\pi)^{\epsilon/2} \frac{(4-\epsilon) m - (2-\epsilon) x \slashed{p}}{((1-x) (m^2 - x p^2))^{\epsilon/2}} \\
        &= \frac{\alpha}{2\pi} \int_0^1 \dd{x} \left( \left( \frac{2}{\epsilon} + \ln(\frac{\tilde{\mu}^2}{(1-x) (m^2 - xp^2)}) \right) (2 m - x \slashed{p}) + (x \slashed{p} - m) \right),
    \end{aligned}
\end{equation}
提取发散部分,为
\begin{equation}
    \Sigma(\slashed{p}) = \frac{\alpha (4m - \slashed{p})}{2 \pi \epsilon} + \text{finite}.
    \label{eq:electron-one-loop-divergence}
\end{equation}

在计算电子自能时我们都采用了最为谨慎的做法,即写下费曼图对应的振幅后立刻将维数延拓到$d=4-\epsilon$,然后做后面的计算。
在计算光子自能时我们首先在$d=4$下消去了所有$\gamma$矩阵,然后再将维数延拓到$d = 4 - \epsilon$。
为什么计算电子自能时不能这么做?答案是,计算光子自能时我们在做的是对一系列$\gamma$矩阵的乘积求迹,这在任何维数下都给出一样的结果;换句话说,计算光子自能时我们是在对自旋和手性指标做缩并,这和空间维数无关。
在计算电子自能时,我们是在缩并空间指标,例如我们会需要计算$\gamma^\mu \gamma^\nu \gamma_\nu$这样的式子,而空间指标能够跑遍$1$到$d$,因此和$d$是有关的,因此需要先做维数延拓做$\gamma$矩阵缩并和动量积分。

不过,如果我们只是想计算出发散部分,那么无论先做$\gamma$矩阵缩并还是先做维数延拓都是一样的,因为$\gamma$矩阵缩并不导致任何发散。
换而言之,\eqref{eq:electron-one-loop-divergence}和\eqref{eq:photon-one-loop-divergence}这样的式子可以通过先做维数延拓再计算$\gamma$矩阵缩并的方法计算出来,也可以通过先计算$\gamma$矩阵缩并再做维数延拓的方法计算出来。
使用后一种方法能够得到正确的发散项,但是不能得到正确的有限部分;使用这种方法认真计算\eqref{eq:electron-one-loop-divergence}和\eqref{eq:photon-one-loop-divergence}中的有限部分就错了,只计算发散部分却是对的。

\subsubsection{顶角函数}

我们现在重做\autoref{sec:abnormal-magnetic}中的计算。按理说,此时两个电子也应该是离壳的,因为这是在修正顶角函数,本身不能保证在壳。
但是我们马上会注意到QED中amputated one loop diagram中外线不可能是光子线,因此,在本节讨论的一圈图修正中,实际上只需要计算电子在壳的顶角函数的修正即可。

关于形状因子之类的讨论仍然是适用的。我们将一圈图导致的修正复述如下:
\[
    - \ii e \Gamma^\mu(p) = - e^3 \int \frac{\dd[4]{k}}{(2\pi)^4} \frac{\bar{u}(q_2) \gamma^\nu (\slashed{p} + \slashed{k} + m) \gamma^\mu (\slashed{k} + m) \gamma_\nu u(q_1)}{(k^2 - m^2 + \ii 0^+) ((p + k)^2 - m^2 + \ii 0^+) ((k - q_1)^2 + \ii 0^+)} ,
\]
这里没有引入任何外线和$u(q)$矩阵。一开始就引入维数延拓(用于消去紫外发散)和光子质量(用于消去红外发散),得到
\begin{equation}
    \Gamma^\mu(p) = - \ii \mu^{4-d} e^2 \int \frac{\dd[d]{k}}{(2\pi)^d} \frac{\bar{u}(q_2) \gamma^\nu (\slashed{p} + \slashed{k} + m) \gamma^\mu (\slashed{k} + m) \gamma_\nu u(q_1)}{(k^2 - m^2 + \ii 0^+) ((p + k)^2 - m^2 + \ii 0^+) ((k - q_1)^2 - m_\gamma^2 + \ii 0^+)} ,
\end{equation}
并且对称性分析意味着
\begin{equation}
    \Gamma^\mu(p) = F_1\left(\frac{p^2}{m^2}\right) \gamma^\mu + \frac{\ii \sigma^{\mu \nu}}{2m} p_\nu F_2\left(\frac{p^2}{m^2}\right)
\end{equation}
使用$\gamma$矩阵的代数关系可以将分子化为
\[
    \begin{aligned}
        &\quad \gamma^\nu (\slashed{p} + \slashed{k} + m) \gamma^\mu (\slashed{k} + m) \gamma_\nu \\
        &= \epsilon (\slashed{p} + \slashed{k} - m) \gamma^\mu \slashed{k} - 2 \slashed{k} \gamma^\mu (\slashed{p} + \slashed{k}) - m^2 (2 - \epsilon) \gamma^\mu + 4m (p^\mu + 2 k^\mu).
    \end{aligned}
\]
再对分母做费曼折叠,得到
\begin{equation}
    \Gamma^\mu(p) = - 2 \ii \mu^{4-d} e^2 \int_0^1 \dd{x} \dd{y} \dd{z} \delta(x + y + z - 1) \int \frac{\dd[d]{k}}{(2\pi)^d} \frac{N^\mu}{D},
\end{equation}
其中
\begin{equation}
    D = (x (k^2 - m^2 + \ii 0^+) + y ((p + k)^2 - m^2 + \ii 0^+) + z ((k - q_1)^2 + \ii 0^+))^3
\end{equation}
而
\begin{equation}
    N^\mu = \epsilon (\slashed{p} + \slashed{k} - m) \gamma^\mu \slashed{k} - 2 \slashed{k} \gamma^\mu (\slashed{p} + \slashed{k}) - m^2 (2 - \epsilon) \gamma^\mu + 4m (p^\mu + 2 k^\mu).
\end{equation}

之后的步骤和\autoref{sec:abnormal-magnetic}中是完全一样的。
一些观察可以大大简化计算:首先注意到$F_2$根本就不发散,我们实际上只需要计算$F_1$然后把$F_2$加上去即可。
然后我们会发现,$d=4-\epsilon$导致的额外的项可以归结为:$m^2 \gamma^\mu$项和$\slashed{k} \gamma^\mu \slashed{k}$项均需要乘上$(2-\epsilon) / 2$;$\slashed{k} \gamma^\mu \slashed{p}$项也需要乘上$(2 - \epsilon) / 2$(实际上是需要加上$\epsilon \slashed{p} \gamma^\mu \slashed{k}$,但是交换$\slashed{p}$和$\slashed{k}$只会多出来一个$\sigma^{\mu \nu}$项,从而归入$F_2$项中);多出来一个$- \epsilon m \gamma^\mu \slashed{k}$。
如前所述,$p^\mu$最终会消失,这是Ward恒等式的一个推论。
% TODO

实际上,如果只是希望得到发散部分,那么直接对\eqref{eq:f1-in-4d}的动量积分做维数延拓即可。其发散部分由$k^2$项贡献,为
\[
    \begin{aligned}
        F_1 &= 4 \ii e^2 \int_0^1 \dd{x} \dd{y} \dd{z} \int \frac{\dd[d]{k}}{(2\pi)^d} \frac{-\frac{1}{2} k^2}{(k^2 - \Delta)^3} + \text{finite} \\
        &= - 2 \ii e^2 \int_0^1 \dd{x} \dd{y} \dd{z} \delta(x + y + z - 1) \frac{\ii}{(4\pi)^{2 - \epsilon / 2}} (2 - \epsilon / 2) \frac{\Gamma\left(\frac{\epsilon}{2}\right)}{2} \left(\frac{1}{\Delta}\right)^{\epsilon} + \text{finite} \\
        &= 2 e^2 \int_0^1 \dd{x} \dd{y} \dd{z} \delta(x + y + z - 1) \frac{1}{(4\pi)^2} \frac{2}{\epsilon} + \text{finite} \\
        &= \frac{e^2}{8\pi^2 \epsilon} + \text{finite}, 
    \end{aligned}
\]
即
\begin{equation}
    \Gamma^\mu(p^2) = \gamma^\mu \left( 1 + \frac{e^2}{8\pi^2 \epsilon} \right) + \text{finite} = \gamma^\mu + \gamma^\mu \frac{\alpha}{2\pi \epsilon} + \text{finite}.
    \label{eq:vertex-one-loop-divergence}
\end{equation}

\subsection{重整化方案}

\subsubsection{抵消项}

到目前为止,我们发现树图中需要修正的组件——电子传播子,光子传播子,相互作用顶角——的圈图修正均存在发散,并且已经通过维数正规化提取出了这些发散。
然后就需要通过引入抵消项来消除这些发散。

本节采用一种比较简单的做法。我们要计算散射振幅,而LSZ约化公式为
\begin{equation}
    \braket*{p_1, p_2, \ldots, p_m}{k_1, k_2, \ldots, k_n} \stackrel{p_i^0 \to \omega_{\vb*{p}_i}, \; k_j^0 \to \omega_{\vb*{k}_j}}{\sim} (\sqrt{Z})^{m+n} \times \text{amputated diagrams}.
    \label{eq:lsz-formula-copied}
\end{equation}
其中
\begin{equation}
    Z = \abs{\mel*{\Omega}{\phi(0)}{p=0}}^2.
\end{equation}
\eqref{eq:lsz-formula-copied}右边的amputated diagram是使用裸的场和(会发散的)裸的参数计算出来的,左边则是关于(有限的)物理参数的散射振幅,它就是重整化后的场的关联函数扣除了入射、出射线的传播子得到的结果。
右边的$Z$因子将右边的场强重整化前的场和左边的场强重整化后的场联系起来。
将\eqref{eq:lsz-formula-copied}右边的$Z$因子和amputated diagram合并起来,就相当于在计算外线是重整化后的场,但是费曼规则还是裸的的amputated diagram。

下面我们将拉氏量中的所有裸的东西都加上下标0,即将拉氏量写成
\begin{equation}
    \mathcal{L} = - \frac{1}{4} (F_0^{\mu \nu})^2 + \bar{\psi}_0 (\ii \gamma_\mu \partial^\mu - m_0) \psi_0  - e_0 A^\mu_0 \bar{\psi}_0 \gamma_\mu \psi_0 \eqqcolon \mathcal{L}_\text{bare}.
\end{equation}
我们用下标R表示重整化之后的量,为了和\eqref{eq:lsz-formula-copied}保持一致,我们设
\begin{equation}
    \psi_0 = \sqrt{Z_2} \psi_\text{R}, \quad A^\mu_0 = \sqrt{Z_3} A^\mu_\text{R}, \quad m_0 = Z_m m_\text{R}, \quad e_0 = Z_e e_\text{R}, 
\end{equation}
这样裸拉氏量就是
\begin{equation}
    \begin{aligned}
        \mathcal{L}_\text{bare} &= - \frac{1}{4} Z_3 (F_\text{R}^{\mu \nu})^2 + Z_2 \bar{\psi}_\text{R} (\ii \gamma_\mu \partial^\mu - Z_m m_\text{R}) \psi_\text{R}  - \sqrt{Z_3} Z_e Z_2 e_\text{R} A^\mu_\text{R} \bar{\psi}_\text{R} \gamma_\mu \psi_\text{R} \\
        &= - \frac{1}{4} Z_3 (F_\text{R}^{\mu \nu})^2 + Z_2 \bar{\psi}_\text{R} (\ii \gamma_\mu \partial^\mu - Z_m m_\text{R}) \psi_\text{R}  - Z_1 e_\text{R} A^\mu_\text{R} \bar{\psi}_\text{R} \gamma_\mu \psi_\text{R},
    \end{aligned}
    \label{eq:renormalized-qed-lagrangian}
\end{equation}
其中我们定义
\begin{equation}
    Z_1 = \sqrt{Z_3} Z_e Z_2 .
\end{equation}
我们把\eqref{eq:renormalized-qed-lagrangian}分离成物理部分$\mathcal{L}_\text{phys}$和抵消项$\var{\mathcal{L}}$,就有
\begin{equation}
    \begin{aligned}
        \mathcal{L} &= \underbrace{- \frac{1}{4} (F_\text{R}^{\mu \nu})^2 + \bar{\psi}_\text{R} (\ii \gamma_\mu \partial^\mu - m_\text{R}) \psi_\text{R}  - e_\text{R} A^\mu_\text{R} \bar{\psi}_\text{R} \gamma_\mu \psi_\text{R}}_{\mathcal{L}_\text{phys}} \\
        & - \underbrace{\frac{1}{4} \var{Z_3} (F_\text{R}^{\mu \nu})^2 + \var{Z_2} \bar{\psi}_\text{R} \ii \gamma_\mu \partial^\mu - (\var{Z}_2 + \var{Z}_m) m_\text{R} \bar{\psi}_\text{R} \psi_\text{R}  - \var{Z_1} e_\text{R} A^\mu_\text{R} \bar{\psi}_\text{R} \gamma_\mu \psi_\text{R}}_{\var{\mathcal{L}}},
    \end{aligned}
    \label{eq:separated-qed-lagrangian}
\end{equation}
其中
\begin{equation}
    Z_1 = 1 + \var{Z_1}, \quad Z_2 = 1 + \var{Z_2}, \quad Z_3 = 1 + \var{Z_3}.
\end{equation}
严格来说按照\eqref{eq:separated-qed-lagrangian},应该有
\begin{equation}
    \var{Z_m} = Z_2 (Z_m - 1),
\end{equation}
但是在本节讨论的一圈图的重整化中$\var{Z_i}$和$\var{Z_m}$都被用于消去\eqref{eq:photon-one-loop-divergence},\eqref{eq:electron-one-loop-divergence}和\eqref{eq:vertex-one-loop-divergence}这三个发散,而它们都正比于$e^2$,从而可以认为
\begin{equation}
    \var{Z_m} = Z_m - 1.
\end{equation}

我们计算\eqref{eq:photon-one-loop-divergence},\eqref{eq:electron-one-loop-divergence}和\eqref{eq:vertex-one-loop-divergence}这三个发散时可以认为是在使用裸的拉氏量$\mathcal{L}_\text{bare}$,也可以认为是在使用物理的拉氏量$\mathcal{L}_\text{phys}$,既然这两个拉氏量的形式完全相同。
本文采取的方法是,认为\eqref{eq:photon-one-loop-divergence},\eqref{eq:electron-one-loop-divergence}和\eqref{eq:vertex-one-loop-divergence}这三个发散是在$\mathcal{L}_\text{phys}$下计算的。
在\eqref{eq:separated-qed-lagrangian}中,$\mathcal{L}_\text{phys}$是关于物理参数的,但是会产生发散,而$\var{\mathcal{L}}$对应着裸的拉氏量中无法观测到的发散部分,正好能够消去用$\mathcal{L}_\text{phys}$计算的图的发散。

切换到动量空间下,抵消项给出了三个可以用来抵消发散的相互作用顶角:
\begin{equation}
    \begin{gathered}
        \begin{tikzpicture}
            \begin{feynhand}
                \vertex (a) at (-1, 0);
                \vertex[crossdot] (b) at (0, 0) {};
                \vertex (c) at (1, 0);
                \propag[fermion] (a) to (b);
                \propag[fermion] (b) to (c);
            \end{feynhand}
        \end{tikzpicture}
    \end{gathered} = \ii (\slashed{p} \var{Z_2} - (\var{Z_2} + \var{Z_m}) m_\text{R}),
\end{equation}
\begin{equation}
    \begin{gathered}
        \begin{tikzpicture}
            \begin{feynhand}
                \vertex (a) at (-1, 0);
                \vertex[crossdot] (b) at (0, 0) {};
                \vertex (c) at (1, 0);
                \propag[photon] (a) to (b);
                \propag[photon] (b) to (c);
            \end{feynhand}
        \end{tikzpicture}
    \end{gathered} = - \ii \var{Z_3} (p^2 \eta^{\mu \nu} - p^\mu p^\nu),
\end{equation}
或者拉氏量做了Faddeev–Popov量子化,那么费曼规范$\frac{1}{2} (\partial_\mu A_\mu)^2$也会有一个$Z_3$因子,从而会消掉上式中第二项,给出
\begin{equation}
    \begin{gathered}
        \begin{tikzpicture}
            \begin{feynhand}
                \vertex (a) at (-1, 0);
                \vertex[crossdot] (b) at (0, 0) {};
                \vertex (c) at (1, 0);
                \propag[photon] (a) to (b);
                \propag[photon] (b) to (c);
            \end{feynhand}
        \end{tikzpicture}
    \end{gathered} = - \ii \var{Z_3} p^2 \eta^{\mu \nu} .
\end{equation}
还有一个顶角的抵消项
\begin{equation}
    \begin{gathered}
        \begin{tikzpicture}
            \begin{feynhand}
                \vertex (a) at (-0.86, 0);
                \vertex[crossdot] (b) at (0, 0.5) {};
                \vertex (c) at (0.86, 0);
                \vertex (d) at (0, 1.5);
                \propag[fermion] (a) to (b);
                \propag[fermion] (b) to (c);
                \propag[photon] (b) to (d);
            \end{feynhand}
        \end{tikzpicture}
    \end{gathered} = - \ii \var{Z_1} e_\text{R} \gamma^\mu .
\end{equation}

不言而喻,物理参数必须要能够通过一个实验过程提取出来,即它必须要和某个散射振幅建立联系,即我们需要指定“什么样的过程给出物理参数”。
物理参数的具体定义、减除方案、抵消项这三者知道了一个就知道了剩下两个。
指定这三者中的其中一个就称为指定了一个\concept{重整化方案}。

我们现在有四个抵消项。但是,后面我们将看到,由于Ward恒等式,实际上只有三个——而且也只需要三个。

\subsubsection{最小减除和修正的最小减除}

所谓\concept{最小减除}或者说\concept{MS}指的是通过适当调整抵消项,让\eqref{eq:photon-one-loop-divergence},\eqref{eq:electron-one-loop-divergence}和\eqref{eq:vertex-one-loop-divergence}这三个发散中的$1/\epsilon$项全部被减除。
此时考虑了抵消项的电子自能是
\[
    - \ii \Sigma = \begin{gathered}
        \begin{tikzpicture}
            \begin{feynhand}
                \vertex (a) at (-1.5, 0);
                \vertex (b) at (-0.5, 0);
                \vertex (c) at (0.5, 0);
                \vertex (d) at (1.5, 0);
                \propag[fermion] (a) to[edge label={$p$}] (b);
                \propag[fermion] (b) to[edge label'={$k$}] (c);
                \propag[fermion] (c) to[edge label={$p$}] (d);
                \propag[photon, mom={$p - k$}] (b) to[half left, looseness=1.5] (c); 
            \end{feynhand}
        \end{tikzpicture}
    \end{gathered} + \begin{gathered}
        \begin{tikzpicture}
            \begin{feynhand}
                \vertex (a) at (-1, 0);
                \vertex[crossdot] (b) at (0, 0) {};
                \vertex (c) at (1, 0);
                \propag[fermion] (a) to[edge label={$p$}] (b);
                \propag[fermion] (b) to[edge label={$p$}] (c);
            \end{feynhand}
        \end{tikzpicture}
    \end{gathered} ,
\]
即
\[
    \Sigma = \frac{\alpha (4m_\text{R} - \slashed{p})}{2 \pi \epsilon} - (\slashed{p} \var{Z_2} - (\var{Z_2} + \var{Z_m}) m_\text{R}) + \text{finite},
\]
即
\begin{equation}
    \var{Z_2} = - \frac{\alpha}{2 \pi \epsilon}, \quad \var{Z_m} = - \frac{3\alpha}{2 \pi \epsilon}.
\end{equation}
光子自能的做法类似,可以得到
\begin{equation}
    \var{Z_3} = - \frac{2\alpha}{3\pi \epsilon}.
\end{equation}
而顶角函数的最小减除则要求
\begin{equation}
    \var{Z_1} = - \frac{\alpha}{2\pi \epsilon}.
\end{equation}
因此MS方案要求的剪除项可以归结为:
\begin{equation}
    \var{Z_1} = \var{Z_2} = - \frac{\alpha}{2\pi \epsilon} = - \frac{e_\text{R}^2}{8 \pi^2 \epsilon}, \quad \var{Z_3} = - \frac{e_\text{R}^2}{6 \pi^2 \epsilon}, \quad \var{Z_m} = - \frac{3 e_\text{R}^2}{8 \pi^2 \epsilon}.
\end{equation}

\concept{修正的最小减除}或者说

\subsubsection{在壳重整化}

\concept{在壳重整化}通过

\subsection{重整化群}



\end{document}