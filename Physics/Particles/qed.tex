\documentclass[hyperref, UTF8, a4paper]{ctexart}

\usepackage{geometry}
\usepackage{titling}
\usepackage{titlesec}
\usepackage{paralist}
\usepackage{footnote}
\usepackage{enumerate}
\usepackage{amsmath, amssymb, amsthm}
\usepackage{simplewick}
\usepackage{cite}
\usepackage{graphicx}
\usepackage{subfigure}
\usepackage{physics}
\usepackage{centernot}
\usepackage{tikz}
\usepackage{tikz-feynhand}
\usepackage[colorlinks, linkcolor=black, anchorcolor=black, citecolor=black]{hyperref}
\usepackage{prettyref}

\geometry{left=3.18cm,right=3.18cm,top=2.54cm,bottom=2.54cm}
\titlespacing{\paragraph}{0pt}{1pt}{10pt}[20pt]
\setlength{\droptitle}{-5em}
\preauthor{\vspace{-10pt}\begin{center}}
\postauthor{\par\end{center}}

\DeclareMathOperator{\timeorder}{T}
\DeclareMathOperator{\diag}{diag}
\newcommand*{\ii}{\mathrm{i}}
\newcommand*{\ee}{\mathrm{e}}
\newcommand*{\const}{\mathrm{const}}
\newcommand*{\comment}{\paragraph{注记}}
\newcommand{\fsl}[1]{{\centernot{#1}}}
\newcommand*{\reals}{\mathbb{R}}
\newcommand*{\complexes}{\mathbb{C}}

\newrefformat{sec}{第\ref{#1}节}
\newrefformat{note}{注\ref{#1}}
\renewcommand{\autoref}{\prettyref}

\newenvironment{bigcase}{\left\{\quad\begin{aligned}}{\end{aligned}\right.}

\newcommand{\concept}[1]{\underline{\textbf{#1}}}
\renewcommand{\emph}{\textbf}

\allowdisplaybreaks[4]

\title{量子电动力学的具体计算}
\author{吴晋渊}

\begin{document}

\maketitle

\section{非相对论极限}

\subsection{电子,光子和电场}

光子无法做非相对论近似,因为无论如何,麦克斯韦方程都应该成立,而这个方程就是洛伦兹协变的。
需要做非相对论近似的只有电子。在非相对论近似下,一切有质量的场都退化为薛定谔场,电子也不例外。
因此在QED的非相对论极限下,基本的粒子包括电子和光子,光子无任何变化,电子场则是动能为$\vb*{k}^2 / 2m$,不再满足相对论协变性,由动量和自旋标记的场。

\subsubsection{电子的薛定谔-泡利方程}

我们现在推导非相对论性电子场遵循的方程,然后得到非相对论极限下的电子哈密顿量。对称性告诉我们这个方程和狄拉克方程和电磁场耦合后取非相对论性近似的结果肯定是一样的。
但是,应该注意,前者中的电磁场都是算符场,而后者中的电磁场是一个经典外场,即后者忽略了电磁场的量子涨落。
电磁场的量子涨落会产生可观测的效应(即所谓\concept{辐射修正})。实际上,这是量子场论的理论框架适用于电动力学的实验证据:一个相对论性的关于电子和光的理论未必要采取量子场论的形式,将电磁场量子化之后,圈图修正就会预言一些经典的电磁场不会产生的现象。
如果这些现象实际上没有观测到,那么量子场论的理论框架就是没有用的或者说错误的,但是实际上我们观测到了这些现象,那么量子场论很可能就是对的。

QED中,电子的狄拉克方程为
\[
    (\ii \gamma^\mu \partial_\mu - e \gamma^\mu A_\mu - m) \psi = 0.
\]
使用狄拉克表象以区分电子和正电子,设旋量场为
\[
    \psi = \pmqty{\phi \\ \chi},
\]
我们将要分析$\phi$遵循的运动方程。为此,根据
\[
    \gamma^0 = \pmqty{1 & 0 \\ 0 & -1}, \quad \gamma^i = \pmqty{0 & \sigma^i \\ - \sigma^i & 0},
\]
写出狄拉克方程的分量表达式:
\[
    \begin{aligned}
        (\ii \partial_t - e \varphi - m) \phi + (e (\vb*{\sigma} \cdot \vb*{A}) + \ii (\vb*{\sigma} \cdot \grad)) \chi &= 0, \\
        (\ii \partial_t - e \varphi + m) \chi + (e (\vb*{\sigma} \cdot \vb*{A}) + \ii (\vb*{\sigma} \cdot \grad)) \phi &= 0,
    \end{aligned}
\]
这里我们需要指出一个容易弄混的地方:正确的分量是
\[
    \partial_\mu = (\partial_t, \grad), \quad A_\mu = (\varphi, - \vb*{A}),
\]
而贸贸然地很容易认为前者是$(\partial_t, -\grad)$而后者是$(\varphi, \vb*{A})$。
我们消去$\chi$,就得到
\begin{equation}
    (\ii \partial_t - e \varphi - m) \phi - (e (\vb*{\sigma} \cdot \vb*{A}) + \ii (\vb*{\sigma} \cdot \grad)) \frac{1}{\ii \partial_t - e \varphi + m} (e (\vb*{\sigma} \cdot \vb*{A}) + \ii (\vb*{\sigma} \cdot \grad)) = 0.
    \label{eq:electron-only}
\end{equation}

到目前为止我们没有做任何近似。现在我们做最强的非相对论近似。我们按照通常的从克莱因-高登方程获得薛定谔方程的方法,设
\[
    \phi = \psi \ee^{-\ii m t},
\]
我们发现
\[
    (\ii \partial_t - e \varphi - m) \phi = \ee^{-\ii m t} \partial_t \psi - e \varphi \psi \ee^{-\ii m t},
\]
而
\[
    (\ii \partial_t - e \varphi + m) \phi = \ee^{-\ii m t} \partial_t \psi - e \varphi \psi \ee^{-\ii m t} + 2 m \ee^{-\ii m t} \psi.
\]
在非相对论极限下粒子的动能相较于其静能(请注意这里都是自然单位制,$c=1$,$m$就是静能)是非常低的,于是
\[
    \partial_t \psi \ll 2m \psi,
\]
因此我们有
\[
    \ii \partial_t \psi - e \varphi \psi - \vb*{\sigma} \cdot (e\vb*{A} + \ii \grad) \frac{1}{2m - e\varphi} \vb*{\sigma} \cdot (e\vb*{A} + \ii \grad) \psi = 0.
\]
进一步,认为电场不很强(否则电子会轻易被加速到很高的速度,非相对论极限不正确),从而
\[
    e \varphi \ll m,
\]
那么就得到
\begin{equation}
    \ii \partial_t \psi - e \varphi \psi - \frac{(\vb*{\sigma} \cdot (e\vb*{A} + \ii \grad))^2}{2m} \psi = 0.
\end{equation}
这个方程可以化成更加清晰的一个形式。我们知道
\[
    (\vb*{\sigma} \cdot \vb*{a}) (\vb*{\sigma} \cdot \vb*{b}) = \vb*{a} \cdot \vb*{b} + \ii \vb*{\sigma} \cdot (\vb*{a} \times \vb*{b}),
\]
而由于$(e\vb*{A} - \ii \grad)$是算符,它自乘并不是零,而是
\[
    (e\vb*{A} + \ii \grad) \times (e\vb*{A} + \ii \grad) \psi = \ii e \vb*{A} \times (\grad{\psi}) + \ii e \curl{(\vb*{A} \psi)} = \ii e  (\curl{\vb*{A}}) \psi = \ii e \vb*{B} \psi,
\]
于是我们就有
\begin{equation}
    \ii \partial_t \psi = \frac{(e \vb*{A} + \ii \grad)^2}{2m} \psi + e \varphi \psi - \frac{e \vb*{\sigma} \cdot \vb*{B}}{2m} \psi.
\end{equation}
场论哈密顿量为
\begin{equation}
    \begin{aligned}
        H &= \int \dd[3]{\vb*{x}} \psi^\dagger \left( \frac{(- \ii \grad - e \vb*{A})^2}{2m} + e \varphi \psi - \frac{e \vb*{\sigma} \cdot \vb*{B}}{2m} \right) \psi \\
        &= \int \dd[3]{\vb*{x}} \psi^\dagger \left( - \frac{(\grad - \ii e \vb*{A})^2}{2m} + e \varphi \psi - \frac{e \vb*{\sigma} \cdot \vb*{B}}{2m} \right) \psi ,
    \end{aligned}
    \label{eq:minimal-coupling}
\end{equation}
其中
\begin{equation}
    \int \dd[3]{\vb*{x}} \psi^\dagger \frac{(- \ii \grad - e \vb*{A})^2}{2m} \psi = \sum_\sigma \int \frac{\dd[3]{\vb*{p}}}{(2\pi)^3} a^\dagger_{\vb*{p} \sigma} \frac{(\vb*{p} - e \vb*{A})^2}{2m} a_{\vb*{p} \sigma}.
\end{equation}
$\psi$是电子的湮灭算符,其运动方程就是薛定谔绘景下单电子波函数的运动方程。
于是,坐标表象下单电子的波函数(当然,此时已经隐含地认为电磁场是经典场,否则会有等效电子-电子相互作用,单电子图像不再适用)满足的方程就是
\begin{equation}
    \ii \partial_t \psi = \frac{(e \vb*{A} + \ii \grad)^2}{2m} \psi - \frac{e \vb*{\sigma} \cdot \vb*{B}}{2m} \psi + e \varphi \psi = \frac{(\vb*{p} - e \vb*{A})^2}{2m} \psi - \frac{e \vb*{\sigma} \cdot \vb*{B}}{2m} \psi + e \varphi \psi.
\end{equation}
这个方程的形式和自由电子薛定谔方程很像,但是哈密顿量为
\begin{equation}
    H = \frac{(\vb*{p} - e \vb*{A})^2}{2m} - \frac{e \vb*{\sigma} \cdot \vb*{B}}{2m} + e \varphi.
    \label{eq:pauli-eq}
\end{equation}
我们称此方程为\concept{泡利方程}或是\concept{薛定谔-泡利方程}。

无论电磁场是经典场还是考虑量子涨落,\eqref{eq:minimal-coupling}都是成立的。
这些式子可以看成对\eqref{eq:pauli-eq}中的电子做二次量子化得到的;这些式子也意味着,如果我们要将相对论性的、量子化的电磁场和非相对论性的电子耦合起来,只需要量子化自由电磁场即可,因为电磁场和非相对论性的电子耦合的哈密顿量总是\eqref{eq:minimal-coupling},无论电磁场是经典的还是量子的。
\eqref{eq:minimal-coupling}也可以看成对非相对论性自由电子做$U(1)$最小耦合得到的结果。

\eqref{eq:minimal-coupling}中,$\vb*{A}$和电子场之间的耦合的非线性性值得分析,因为在QED中不存在这个情况:电磁场和电子旋量场之间的耦合就是线性的。
回顾以上求解过程,这个$\vb*{A}^2$项实际上来自积掉$\chi$场的过程:我们只希望保留电子模式$\phi$而不想保留正电子模式$\chi$,于是就有\eqref{eq:electron-only},实际上就是积掉了$\chi$场。
分析$\ii \bar{\psi} \gamma^\mu D_\mu \psi$项的形式可以知道,与$\chi$场相关的过程包括:$\chi$模式吸收一个$\varphi$光子;$\chi$模式和$\phi$模式互换,这个顶角的耦合常数含有动量的一次方;$\chi$模式和$\phi$模式互换,吸收一个$\vb*{A}$光子。
因此,积掉$\chi$场引入了这样一个等效过程:$\phi$模式或是吸收一个$\vb*{A}$光子或是不吸收,变成$\chi$模式,然后吸收若干个$\varphi$光子,最后或是吸收一个$\vb*{A}$光子或是不吸收而重新变成$\phi$模式。
这个过程涉及$\vb*{A}$的零次方、一次方和平方,以及任意阶次的$\varphi$。
然而,$\varphi$光子实际上是在给$\chi$场提供自能修正:
\[
    \frac{1}{\ii \partial_t + m} + \frac{1}{\ii \partial_t + m} e \varphi \frac{1}{\ii \partial_t + m} + \cdots = \frac{1}{\ii \partial_t + m - e \varphi},
\]
而在非相对论极限下$e \varphi$相比于$m$肯定是很小的,从而可以忽略。
另一方面,$\vb*{A}$的大小和$\vb*{p}$的大小却不好比较:两者都不能太大。
因此在\eqref{eq:minimal-coupling}中出现了非线性的$\vb*{A}$耦合,但是没有出现非线性的$\varphi$耦合。

\subsubsection{相对论修正}

% TODO:自旋轨道耦合之类?
相对论修正指的是当电子的能量仍然不是特别高(从而没有必要使用完整的狄拉克方程)但是已经比较高时,相对论效应造成的修正。
它和量子涨落造成的修正是基本上无关的,因为即使不考虑量子涨落,狄拉克方程仍然会导致偏离\eqref{eq:pauli-eq}的结果。

\subsection{树图阶的有效相互作用}

原则上\eqref{eq:minimal-coupling}和量子化后的电磁场给出了非相对论性电子和电磁辐射的全部理论,包括了全部电子和光的相互作用。
然而,实际上我们还可以做进一步的近似,将一些过程用等效相互作用替代。
做完这种近似之后需要特别小心,以避免将同样的过程重复计数。

\subsubsection{库伦相互作用}

在非相对论情况下,光子的能量不足以激发出电子-正电子对,真空极化并不重要。
因此,光子虽然是无能隙的,仍然可以积掉光子而得到电子-电子等效相互作用。
我们将其它场看成背景场,因为QED中没有光子-光子相互作用顶角,这个等效相互作用背后的实际的QED过程只包含树图。

我们首先直接在QED中计算电子-电子等效相互作用,即使用旋量场$\psi$做计算,这相当于将所有的“电子发出一个光子,这个光子被另一个电子吸收”的子图用一个电子-电子等效相互作用替代,而不去碰“电子发出一个光子,这个光子被一个不是电子的东西吸收”,或是“一个不知道哪儿来的光子被电子吸收”的过程(这些过程可能又被做了别的近似,比如说如果系统中充满了“不知道哪儿来的光子”,那就可以把电磁场当成经典场,等等)。%
\footnote{
    这里说的“电子”指的都是我们的理论考虑的\emph{那部分}电子。
    “无穷远处射来一束光,被电子吸收”这个过程显然在我们的理论中不是电子-电子相互作用,但是把视角调远一些,很可能这个“无穷远处射来的光”本身是另一些电子辐射产生的,例如它可能来自原子能级跃迁产生的辐射,但是能级跃迁产生的辐射说到底还是电子产生的。
    然而,也许在我们讨论的问题的能标下,原子可以看成一个整体,那么,这个过程在我们的低能有效理论中就不是电子-电子相互作用。
}%
这么做了之后,不计入电子-电子等效相互作用的那些过程仍然是电子与电磁场耦合的过程,从而仍然可以得到\eqref{eq:minimal-coupling}。
于是最终我们就得到一个由\eqref{eq:minimal-coupling},一个电子-电子相互作用,和电磁场自身的哈密顿量三部分拼在一起的模型,其中\eqref{eq:minimal-coupling}中的$\varphi$和$\vb*{A}$由电子激发出来的部分不能作用在电子上(从而避免了重复计数)。
这实际上就是凝聚态场论涉及电子和光子的部分(涉及声子的部分还未加入,因为此时尚无晶格)。
当然,实际上也可以先得到\eqref{eq:minimal-coupling},其中的$\vb*{A}$和$\varphi$由电子激发出来的部分可以作用在电子上,然后将所有的“电子发出一个光子,这个光子被另一个电子吸收”的子图用一个电子-电子等效相互作用替代,而不去碰“电子发出一个光子,这个光子被一个不是电子的东西吸收”,或是“一个不知道哪儿来的光子被电子吸收”的过程。

给出等效电子-电子相互作用的图有两张:
\[
    \begin{tikzpicture}
        \begin{feynhand}
            \vertex (a) at (-1.5, 0.8);
            \vertex (b) at (-1.5, -0.8);
            \vertex (c) at (-1, 0);
            \vertex (d) at (0, 0);
            \vertex (e) at (0.5, 0.8);
            \vertex (f) at (0.5, -0.8);

            \propag[anti fermion] (a) to (c);
            \propag[fermion] (b) to (c);
            \propag[photon] (c) to (d);
            \propag[fermion] (d) to (e);
            \propag[anti fermion] (d) to (f);
        \end{feynhand}
    \end{tikzpicture}, \quad \quad 
    \begin{tikzpicture}
        \begin{feynhand}
            \vertex (a) at (-1.5, 0.8);
            \vertex (b) at (-1.5, -0.8);
            \vertex (c) at (-1, 0);
            \vertex (d) at (0, 0);
            \vertex (e) at (0.5, 0.8);
            \vertex (f) at (0.5, -0.8);

            \propag[anti fermion] (e) to (c);
            \propag[fermion] (b) to (c);
            \propag[photon] (c) to (d);
            \propag[fermion] (d) to (a);
            \propag[anti fermion] (d) to (f);
        \end{feynhand}
    \end{tikzpicture}
\]
如果参与散射的两个粒子是可以分辨的(即除了动量和自旋以外还有别的标签可以区分它们),那么第二张图和第一张图不在一个相互作用通道中。
低能过程由于动量低,相应的特征尺度很大,即粒子不会离得很近,这种情况下粒子的“位置”近似起到了区分两个粒子的标签的作用。这意味着第二张图可以忽略。

于是我们计算第一张图,它给出
\[
    \begin{gathered}
        \begin{tikzpicture}
            \begin{feynhand}
                \vertex (a) at (-1.5, 0.8);
                \vertex (b) at (-1.5, -0.8);
                \vertex (c) at (-1, 0);
                \vertex (d) at (0, 0);
                \vertex (e) at (0.5, 0.8);
                \vertex (f) at (0.5, -0.8);
    
                \propag[fermion] (c) to [edge label={$p'$}] (a);
                \propag[fermion] (b) to [edge label={$p$}] (c);
                \propag[photon] (c) to (d);
                \propag[fermion] (d) to [edge label={$k'$}] (e);
                \propag[fermion] (f) to [edge label={$k$}] (d);
            \end{feynhand}
        \end{tikzpicture}
    \end{gathered} = (-\ii e)^2 \bar{u}(p') \gamma^\mu u(p) \frac{-\ii \eta_{\mu \nu}}{(p' - p)^2 + \ii 0^+} \bar{u}(k') \gamma^\nu u(k).
\]
我们考虑$p, p', k, k'$都几乎是零的情况,并且只对$\mu=\nu=0, 3$的情况求和——其实我们会看到,$\mu = \nu = 1, 2$两种情况并不会有贡献。
此时计算会发现
\[
    \bar{u}(p') \gamma^0 u(p) = u^\dagger(p') u(p) \approx m \pmqty{\xi^\dagger & \xi^\dagger} \pmqty{\xi \\ \xi} = 2m \sigma^0,
\]
而
\[
    \bar{u}(p') \gamma^i u(p) = u^\dagger(p') \pmqty{\dmat{- \sigma^i, \sigma^i}} u(p) \approx m \pmqty{\xi^\dagger & \xi^\dagger} \pmqty{\dmat{- \sigma^i, \sigma^i}} \pmqty{\xi \\ \xi} = 0.
\]
于是我们就得到
\[
    \begin{aligned}
        \begin{gathered}
            \begin{tikzpicture}
                \begin{feynhand}
                    \vertex (a) at (-1.5, 0.8);
                    \vertex (b) at (-1.5, -0.8);
                    \vertex (c) at (-1, 0);
                    \vertex (d) at (0, 0);
                    \vertex (e) at (0.5, 0.8);
                    \vertex (f) at (0.5, -0.8);
        
                    \propag[fermion] (c) to [edge label={$p'$}] (a);
                    \propag[fermion] (b) to [edge label={$p$}] (c);
                    \propag[photon] (c) to (d);
                    \propag[fermion] (d) to [edge label={$k'$}] (e);
                    \propag[fermion] (f) to [edge label={$k$}] (d);
                \end{feynhand}
            \end{tikzpicture}
        \end{gathered} &= (-\ii e)^2 \bar{u}(p') \gamma^\mu u(p) \frac{-\ii \eta_{\mu \nu}}{(p' - p)^2 + \ii 0^+} \bar{u}(k') \gamma^\nu u(k) \\
        &= \ii e^2 2 m (\sigma^0)_p \frac{1}{(p' - p)^2 + \ii 0^+} 2m (\sigma^0)_k \\
        &= - \frac{\ii e^2 (2m)^2 \sigma^0}{\abs*{\vb*{p}' - \vb*{p}}^2 - \ii 0^+}.
    \end{aligned}
\]
这里由于我们取了非相对论极限,有
\[
    p^\mu = (m, \vb*{p}),
\]
于是
\[
    (p' - p)^2 = - \abs*{\vb*{p} - \vb*{p}'}^2.
\]
上式是$\{\ket*{p}\}$表象下的相互作用哈密顿量矩阵元;下标$p$和$k$用于区分作用在不同单粒子态上的矩阵。
矩阵$\sigma^0$给出了自旋的变化情况,可以看到以上相互作用通道不挑选入射自旋,也不改变入射自旋。
我们要做非相对论近似,所以要转换到$\{\ket*{\vb*{p}}\}$表象下,由于有四条外线,要除以因子$(\sqrt{2m})^{4}$。
于是非相对论极限下,我们获得相互作用顶角
\begin{equation}
    \begin{gathered}
        \begin{tikzpicture}
            \begin{feynhand}
                \vertex (a) at (-1.5, 0.8);
                \vertex (b) at (-1.5, -0.8);
                \vertex (c) at (-1, 0);
                \vertex (d) at (0, 0);
                \vertex (e) at (0.5, 0.8);
                \vertex (f) at (0.5, -0.8);
    
                \propag[fermion] (c) to [edge label={$p', \alpha$}] (a);
                \propag[fermion] (b) to [edge label={$p, \alpha$}] (c);
                \propag[photon] (c) to (d);
                \propag[fermion] (d) to [edge label={$k', \beta$}] (e);
                \propag[fermion] (f) to [edge label={$k, \beta$}] (d);
            \end{feynhand}
        \end{tikzpicture}
    \end{gathered} = -\ii \frac{e^2}{\abs*{\vb*{p} - \vb*{p}'}^2} (2\pi)^4 \delta^4(k' + p' - p - k).
\end{equation}
这个相互作用顶角的形式实际上正是动量空间中的库伦定律。
为了更加清晰地看出库伦定律,我们将上式切换回实空间,做傅里叶变换
\[
    \begin{aligned}
        \int \frac{\dd[4]{p'}}{(2\pi)^4} \ee^{\ii p' \cdot x_1} \int \frac{\dd[4]{k'}}{(2\pi)^4} \ee^{\ii k' \cdot x_2} \int \frac{\dd[4]{p}}{(2\pi)^4} \ee^{- \ii p \cdot x_3} \int \frac{\dd[4]{p}}{(2\pi)^4} \ee^{- \ii p \cdot x_4},
    \end{aligned}
\]
计算发现
\begin{equation}
    \begin{aligned}
        \begin{gathered}
            \begin{tikzpicture}
                \begin{feynhand}
                    \vertex (a) at (-1.7, 1) {$x_1, \alpha$};
                    \vertex (b) at (-1.7, -1) {$x_3, \alpha$};
                    \vertex (c) at (-1, 0);
                    \vertex (d) at (0, 0);
                    \vertex (e) at (0.7, 1) {$x_2, \beta$};
                    \vertex (f) at (0.7, -1) {$x_4, \beta$};
        
                    \propag[fermion] (c) to (a);
                    \propag[fermion] (b) to (c);
                    \propag[photon] (c) to (d);
                    \propag[fermion] (d) to (e);
                    \propag[fermion] (f) to (d);
                \end{feynhand}
            \end{tikzpicture}
        \end{gathered} &= -\ii e^2 \delta(t_4 - t_1) \delta^4(x_1 - x_3) \delta^4(x_2 - x_4) \int \frac{\dd[3]{\vb*{q}}}{(2\pi)^3} \frac{\ee^{-\ii \vb*{q} \cdot (\vb*{x}_4 - \vb*{x}_1)}}{\abs*{\vb*{q}}^2 - \ii 0^+} \\
        &= -\ii \delta(t_4 - t_1) \delta^4(x_1 - x_3) \delta^4(x_2 - x_4) \frac{e^2}{4\pi \abs*{\vb*{x}_4 - \vb*{x}_1}}.
    \end{aligned}
    \label{eq:coulomb-interaction}
\end{equation}
因此我们的确得到了库伦相互作用。在计算时有一个细节:计算\eqref{eq:coulomb-interaction}的第一个等号右边的积分时,我们有
\[
    \begin{aligned}
        \int \frac{\dd[3]{\vb*{q}}}{(2\pi)^3} \frac{\ee^{-\ii \vb*{q} \cdot (\vb*{x}_4 - \vb*{x}_1)}}{\abs*{\vb*{q}}^2 - \ii 0^+} &= \frac{2\pi}{(2\pi)^3} \int_0^{\pi} \sin \theta \dd{\theta} \int_0^\infty q^2 \dd{q} \frac{\ee^{- \ii q \abs*{\vb*{x}_4 - \vb*{x}_1} \cos \theta}}{\abs*{\vb*{q}}^2 - \ii 0^+} \\
        &= \frac{1}{4\pi^2} \int_0^\infty \frac{q^2}{q^2 - \ii 0^+} \dd{q} \frac{\ee^{- \ii q \abs*{\vb*{x}_4 - \vb*{x}_1}} - \ee^{\ii q \abs*{\vb*{x}_4 - \vb*{x}_1}}}{- \ii q \abs*{\vb*{x}_4 - \vb*{x}_1}} \\
        &= \frac{1}{4\pi^2 \ii} \int_{-\infty}^\infty \frac{q^2}{q^2 - \ii 0^+} \dd{q} \frac{\ee^{\ii q \abs*{\vb*{x}_4 - \vb*{x}_1}}}{q \abs*{\vb*{x}_4 - \vb*{x}_1}}.
    \end{aligned}
\]
如果我们将因子$q^2/(q^2 - \ii 0^+)$直接当成$1$,上式就没有确定的值了,因为极点直接出现在了积分路径上。
不过,我们有
\[
    \frac{q^2}{q^2 - \ii \epsilon} = \frac{1}{1 - \frac{\ii \epsilon}{q^2}} \to 0 \text{as $q \to 0$},
\]
因此我们应该取积分主值,即取
\[
    \int \frac{\dd[3]{\vb*{q}}}{(2\pi)^3} \frac{\ee^{-\ii \vb*{q} \cdot (\vb*{x}_4 - \vb*{x}_1)}}{\abs*{\vb*{q}}^2 - \ii 0^+} = \frac{1}{4\pi^2 \ii} \text{P} \int_{-\infty}^\infty \dd{q} \frac{\ee^{\ii q \abs*{\vb*{x}_4 - \vb*{x}_1}}}{q \abs*{\vb*{x}_4 - \vb*{x}_1}} = \frac{1}{4\pi^2 \ii} \frac{\pi \ii}{\abs*{\vb*{x}_4 - \vb*{x}_1}} = \frac{1}{4\pi \abs*{\vb*{x}_4 - \vb*{x}_1}}. 
\]

单粒子量子力学中的散射理论相当于梯形图近似。

\subsubsection{自旋磁矩}

在\eqref{eq:pauli-eq}和\eqref{eq:minimal-coupling}中我们看到,由于算符不对易这一特点,哈密顿量中除了经典的机械动能项(正则动量$\vb*{p}$减去电磁动量$e \vb*{A}$得到机械动量)以外还多出来一项。
电子自旋算符为
\[
    \vb*{S} = \frac{\vb*{\sigma}}{2},
\]
于是这一项就是
\begin{equation}
    H_\text{spin} = - \vb*{\mu} \cdot \vb*{B}, 
\end{equation}
其中
\begin{equation}
    \vb*{\mu} = \frac{e \vb*{\sigma}}{2m} = \frac{e}{m} \vb*{S} = - \frac{\abs*{e}}{m} \vb*{S} = - g \frac{\abs*{e}}{2m} \vb*{S}.
\end{equation}
这意味着电子即使在静止时也有磁矩,这个磁矩是来自其自旋而不是轨道运动的。$g=2$是自旋磁矩的朗德$g$因子。

电子磁矩是实验可测的。粗略的实验确实指出$g=2$,更加精确的实验则显示$g$其实比$2$稍微大一些。
计算QED圈图修正之后能够得到和实验测量结果非常接近的值,这是一个证明QED可靠性的论据。

\section{散射实验}

积掉光子还意味着电子自己需要做自能修正,让质量什么的发生变化,即由于电子和光子的相互作用,电子带上了“电磁质量”。
这是在经典电动力学中也已经知道的一个现象,但是在经典电动力学中不足以处理自能导致的发散。

\section{辐射修正}

\end{document}