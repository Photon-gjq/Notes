\documentclass[hyperref, UTF8, a4paper]{ctexart}

\usepackage{geometry}
\usepackage{titling}
\usepackage{titlesec}
\usepackage{paralist}
\usepackage{footnote}
\usepackage{enumerate}
\usepackage{amsmath, amssymb, amsthm}
\usepackage{bbm}
\usepackage{cite}
\usepackage{graphicx}
\usepackage{subfigure}
\usepackage{physics}
\usepackage{tikz}
\usepackage{tikz-feynhand}
\usepackage[colorlinks, linkcolor=black, anchorcolor=black, citecolor=black]{hyperref}
\usepackage{prettyref}

\geometry{left=3.18cm,right=3.18cm,top=2.54cm,bottom=2.54cm}
\titlespacing{\paragraph}{0pt}{1pt}{10pt}[20pt]
\setlength{\droptitle}{-5em}
\preauthor{\vspace{-10pt}\begin{center}}
\postauthor{\par\end{center}}

\DeclareMathOperator{\timeorder}{T}
\DeclareMathOperator{\diag}{diag}
\DeclareMathOperator{\legpoly}{P}
\DeclareMathOperator{\primevalue}{P}
\DeclareMathOperator{\sgn}{sgn}
\newcommand*{\ii}{\mathrm{i}}
\newcommand*{\ee}{\mathrm{e}}
\newcommand*{\const}{\mathrm{const}}
\newcommand*{\comment}{\paragraph{注记}}
\newcommand*{\suchthat}{\quad \text{s.t.} \quad}
\newcommand*{\argmin}{\arg\min}
\newcommand*{\argmax}{\arg\max}
\newcommand*{\normalorder}[1]{: #1 :}
\newcommand*{\pair}[1]{\langle #1 \rangle}
\newcommand*{\fd}[1]{\mathcal{D} #1}

\newrefformat{sec}{第\ref{#1}节}
\newrefformat{note}{注\ref{#1}}
\newrefformat{fig}{图\ref{#1}}
\renewcommand{\autoref}{\prettyref}

\usetikzlibrary{arrows,shapes,positioning}
\usetikzlibrary{arrows.meta}
\usetikzlibrary{decorations.markings}
\tikzstyle arrowstyle=[scale=1]
\tikzstyle directed=[postaction={decorate,decoration={markings,
    mark=at position .5 with {\arrow[arrowstyle]{stealth}}}}]
\tikzstyle ray=[directed, thick]
\tikzstyle dot=[anchor=base,fill,circle,inner sep=1pt]

\renewcommand{\emph}[1]{\textbf{#1}}
\newcommand*{\concept}[1]{\underline{\textbf{#1}}}

\title{流体力学}
\author{吴晋渊}

\begin{document}

\maketitle

The Effective Field Theory Approach to Fluid Dynamics

严格来说流体力学需要在一个非平衡态统计场论中做。但是如果这个formalism里面能够用拉氏量的概念,下面的所有讨论都是正确的。
实际上,原文作者非常狡猾地没有提我们如何量子化流体力学。

大概套路:基本自由度可以是位移场,即comoving coordinates相对于实际的坐标和时间的分布。
先根据不可压缩流体的性质,将“自由场拉氏量”写成一个保度规坐标变换下的不变量。
然后可以求出能量动量张量
\begin{equation}
    T_{\mu \nu} = (\rho + p) u_\mu u_\nu + p \eta_{\mu \nu}
\end{equation}
正好是正确的能动张量。

计算偏离基态的小的振动,可以得到声波和涡旋,后者是退化的,这意味着流体中不存在横波。

那么耗散来自哪儿呢?我们对位移场做一个小的偏移(由于位移场的物理意义,这等价于坐标变换),位移场部分的作用量不变。此外,可以确信,整个理论的作用量——同时包括位移场部分的作用量、不考虑的微观自由度的作用量和耦合项——也是不变的。
理论中那些微观自由度也是连通着流体一起动的,我们据此确定,耦合项随着坐标变换发生的变分和仅仅关于微观自由度的作用量随着坐标变换发生的变分正好抵消。
这意味着耦合项相对于位移场的一阶展开的形式就是微观自由度自己的能动张量乘上位移场的导数。

后面使用文小刚书上那种“积掉热浴”的方法就会发现位移场的格林函数存在有限大的一个虚部,因此存在耗散,并且通过计算格林函数的运动方程可以看出这个耗散就是$-k v$。

这就得到了不可压缩流体的纳维-斯托克斯方程。

但是这种场论语言并没有什么特殊的用处。耗散项是$-kv$这件事本身可以通过与前述分析等价的,基于微分方程形式的对称性分析得到。
微扰论提供不了太多信息因为常见的流体问题都涉及很大的位移场,实际上是非微扰的。

其实即使能够做微扰论,场论语言也没太大意义,因为基于微分方程的计算实际上就能够画费曼图。

\end{document}