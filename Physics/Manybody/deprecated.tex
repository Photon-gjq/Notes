请注意\eqref{eq:4-electron-interaction-by-phonon}实际上是做了近似的,因为声子的传播速度有限,因此把声子积掉之后得到的任何相互作用必然是一个推迟相互作用,而\eqref{eq:4-electron-interaction-by-phonon}给出的相互作用是瞬时的。%
\footnote{
    实际上我们甚至可以大致确定推迟相互作用的形式。$\omega_{\vb*{q}}$是声子能量,它不太可能和推迟有关,因为推迟应当是“某个关于电子的能量量纲的量乘以电子场的时间”,即$\exp(\ii \epsilon_{\vb*{k}} \tau)$。
    电子能量的绝对大小无关紧要,同样不会直接决定推迟,那么唯一能够决定推迟时间的时间尺度的只能是
    \[
        \omega = \epsilon_{\vb*{k}} - \epsilon_{\vb*{k} + \vb*{q}}.
    \]
    这样,设推迟相互作用的作用量为
    \[
        S \sim D \bar{c}(\vb*{k}+\vb*{q}, \tau) \bar{c}(\vb*{k}'-\vb*{q}, \tau') c(\vb*{k}, \tau) c(\vb*{k}', \tau'),
    \]
    则
    \[
        D \sim \sum \abs{M_{\vb*{q}}}^2 \frac{\omega_{\vb*{q}}}{\omega^2 - \omega_{\vb*{q}}^2} \ee^{\omega(\tau - \tau')}.
    \]
    这是为了保证$\tau=\tau'$时回退到\eqref{eq:4-electron-interaction-by-phonon}。这样,对\eqref{eq:4-electron-interaction-by-phonon}中的相互作用强度以$\omega$为变量做傅里叶变换就能够大致知道推迟相互作用的形式。

    当然,实际上以上推导仅仅表明推迟相互作用在推迟不大时可以回退到\eqref{eq:4-electron-interaction-by-phonon},并不能够严格说明\eqref{eq:4-electron-interaction-by-phonon}一定来自一个推迟相互作用。我们归根到底还是通过声子的性质来确定相互作用确实是推迟的这一事实的,\eqref{eq:4-electron-interaction-by-phonon}中有$\omega$依赖只是一个暗示。
}%