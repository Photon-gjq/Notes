\documentclass[hyperref, a4paper]{article}

\usepackage{geometry}
\usepackage{titling}
\usepackage{titlesec}
\usepackage{paralist}
\usepackage{footnote}
\usepackage{enumerate}
\usepackage{amsmath, amssymb, amsthm}
\usepackage{bbm}
\usepackage{cite}
\usepackage{graphicx}
\usepackage{subfigure}
\usepackage{physics}
\usepackage{tikz}
\usepackage{autobreak}
\usepackage[ruled, vlined, linesnumbered, noend]{algorithm2e}
\usepackage[colorlinks, linkcolor=black, anchorcolor=black, citecolor=black]{hyperref}
\usepackage{prettyref}

% Page style
\geometry{left=3.18cm,right=3.18cm,top=2.54cm,bottom=2.54cm}
\titlespacing{\paragraph}{0pt}{1pt}{10pt}[20pt]
\setlength{\droptitle}{-5em}
\preauthor{\vspace{-10pt}\begin{center}}
\postauthor{\par\end{center}}

% Math operators
\DeclareMathOperator{\timeorder}{T}
\DeclareMathOperator{\diag}{diag}
\DeclareMathOperator{\legpoly}{P}
\DeclareMathOperator{\primevalue}{P}
\DeclareMathOperator{\sgn}{sgn}
\newcommand*{\ii}{\mathrm{i}}
\newcommand*{\ee}{\mathrm{e}}
\newcommand*{\const}{\mathrm{const}}
\newcommand*{\suchthat}{\quad \text{s.t.} \quad}
\newcommand*{\argmin}{\arg\min}
\newcommand*{\argmax}{\arg\max}
\newcommand*{\normalorder}[1]{: #1 :}
\newcommand*{\pair}[1]{\langle #1 \rangle}
\newcommand*{\fd}[1]{\mathcal{D} #1}
\DeclareMathOperator{\bigO}{\mathcal{O}}

% TikZ setting
\usetikzlibrary{arrows,shapes,positioning}
\usetikzlibrary{arrows.meta}
\usetikzlibrary{decorations.markings}
\tikzstyle arrowstyle=[scale=1]
\tikzstyle directed=[postaction={decorate,decoration={markings,
    mark=at position .5 with {\arrow[arrowstyle]{stealth}}}}]
\tikzstyle ray=[directed, thick]
\tikzstyle dot=[anchor=base,fill,circle,inner sep=1pt]

% Python-style code
\SetKwIF{If}{ElseIf}{Else}{if}{:}{elif:}{else:}{}
\SetKwFor{For}{for}{:}{}
\SetKwFor{While}{while}{:}{}
\SetArgSty{textnormal}

\newcommand*{\concept}[1]{\underline{\textbf{#1}}}
\newcommand*{\Ztwo}{$\mathbb{Z}_2$\ }

\title{Advanced Topics in Spin Systems}
\author{wujinq}

\begin{document}

\maketitle

\section{Gauge field theories on a lattice}

\subsection{\Ztwo lattice gauge theory}

\subsubsection{The Hilbert space}

\Ztwo lattice gauge theory is the simplest lattice gauge theory with a gauge field, but it is essential for understanding some quantum liquid states because it appears to be an effective theory of several topologically-ordered states.

First we describe the Hilbert space.
Consider a two-dimensional square lattice labeled by $\vb*{i}=(i_x, i_y)$, and put a link variable $S_{\vb*{i} \vb*{j}} = s_{\vb*{j} \vb*{i}}$ which can take the two values $\pm 1$.
If nothing else is given, this is just the Hilbert space of Ising model and $s_{\vb*{i} \vb*{j}}$s may be interpreted as spins, acting as a one-to-one labeling.
Now we impose a local \Ztwo gauge structure: we requires that two configurations $\{s_{\vb*{i} \vb*{j}}\}$ and $\{s'_{\vb*{i} \vb*{j}}\}$ label the same state if 
\begin{equation}
    s'_{\vb*{i} \vb*{j}} = W_{\vb*{i}} s_{\vb*{i} \vb*{j}} W_{\vb*{j}}^{-1},
\end{equation}
where $W_{\vb*{i}}$ can take values $\pm 1$. 

Suppose the total number of sites of the lattice is $N_\text{site}$, then there are $2 N_\text{site}$ nearest-neighbor links, so the number of possible $\{s_{\vb*{i} \vb*{j}}\}$s is $2^{2 N_\text{site}}$.
Since $\{W_{\vb*{i}}\}$ is defined on sites, there are $2^{N_\text{site}}$ elements in the gauge group.
However, since both of the two gauge transformations
\[
    W_{\vb*{i}} = 1, \quad W_{\vb*{i}} = -1
\]
do nothing to physical states, there is an \emph{invariant gauge group} with two elements in the gauge group.
The invariant gauge group is obviously a normal subgroup.
Therefore, there are actually only $2^{N_\text{site}}/2$ elements in the representation of the gauge group on the space of all $\{s_{\vb*{i} \vb*{j}}\}$s.
So finally we get $2^{2N_\text{site}} / (2^{N_\text{site}} / 2) = 2 \times 2^{N_\text{site}}$ different gauge-equivalent classes, corresponding to $2 \times 2^{N_\text{site}}$ physical states.

Now it is time to find a proper way to label these states.
For a loop on the lattice, we define
\begin{equation}
    U(C) = s_{\vb*{i} \vb*{j}} s_{\vb*{j} \vb*{k}} \cdots s_{\vb*{l} \vb*{i}},
\end{equation}
where $\vb*{i}, \vb*{j}, \ldots, \vb*{l}$ are sites on the loop, and call $U(C)$ the \concept{\Ztwo flux} through the loop. 
It is easy to prove $U(C)$ is \Ztwo invariant, so it is a part of the whole label of a physical state.
Now consider the \Ztwo flux through a plaquette:
\begin{equation}
    F_{\vb*{i}} = s_{\vb*{i}, \vb*{i}+\hat{\vb*{x}}} s_{\vb*{i}+\hat{\vb*{x}}, \vb*{i}+\hat{\vb*{x}}+\hat{\vb*{y}}} s_{\vb*{i}+\hat{\vb*{x}}+\hat{\vb*{y}}, \vb*{i}+\hat{\vb*{y}}} s_{\vb*{i}+\hat{\vb*{y}}, \vb*{i}}.
\end{equation}
There 

\section{Topological order}

\section{Spin liquid}

\subsection{Parton construction}

In the early age right after the establishment of quantum mechanics, people were confused about Heisenberg

ferromagnetic 

\section{String-net condensation}

\end{document}