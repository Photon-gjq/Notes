\begin{back}{从玻尔兹曼方程到纳维-斯托克斯方程}{n-s}
    实际上,从玻尔兹曼方程出发可以推导出纳维-斯托克斯方程。
    将\eqref{eq:boltzmann-eq-with-force}写成简约形式
    \begin{equation}
        \pdv{f}{t} + \vb*{v} \cdot \pdv{f}{\vb*{r}} + \frac{\vb*{F}}{m} \cdot \pdv{f}{\vb*{v}} = \eval{\pdv{f}{t}}_\text{col},
    \end{equation}
    设单粒子质量为$m$,在上式两边乘上$m$并对$\vb*{v}$积分。
    第一项显然就是$\pdv*{\expval*{\rho}}{t}$。第二项是
    \[
        \int \dd[3]{\vb*{v}} m \vb*{v} \cdot \pdv{f}{\vb*{r}} f = \pdv{\vb*{r}} \expval{\rho \vb*{v}}_{\vb*{v}},
    \]
    这里期望值只是将$\vb*{v}$平均了,保留了$\vb*{r}$。
    第三项是典型的对散度积分的项,由于无穷远处粒子分布为零而为零。
    方程右边,由于
    忽略$\rho$和$\vb*{v}$的涨落,我们有
    \begin{equation}
        \pdv{\rho}{t} + \div{(\rho \vb*{v})} = 0,
    \end{equation}
    即粒子数守恒方程。
    类似的可以

    需要注意纳维-斯托克斯方程类似于一个“不动点”,它可以通过诸如“液体不可压缩”等条件和对称性分析直接推导出来,因此对玻尔兹曼方程不适用的系统——比如说稠密系统——纳维-斯托克斯方程可能仍然成立。
\end{back}