\begin{back}{从玻尔兹曼方程到纳维-斯托克斯方程}{n-s}
    完全经典的体系似乎无法“积掉”变量,但是正如\cite{reall2021effective}所示,在满足一定的条件之后,对完全经典的系统也可以定义低能有效理论,并且通过向系统中引入小幅的随机扰动能够扩大可以定义低能有效理论的经典体系的范围。
    按照这种思路,玻尔兹曼方程应该也可以有低能有效理论。

    实际上,从玻尔兹曼方程出发可以推导出纳维-斯托克斯方程。
    纳维-斯托克斯方程实际上是一个重整化群不动点,它可以通过对称性分析和一些别的直观的条件直接推导出来\cite{Visscher_1985}。
    在工程学科的资料中它通常是通过宏观的牛顿三定律推导出来的,但是我们看到,其中的各种参数是可以微观地计算的。
    此外,对玻尔兹曼方程不适用的系统——比如说稠密系统——纳维-斯托克斯方程可能仍然成立。
    另一方面,玻尔兹曼方程的低能有效理论也不止纳维-斯托克斯方程一个。
    通过将碰撞项的强度的量级记作$1/\epsilon$,根据$\epsilon$做一种称为Chapman–Enskog级数的微扰展开\cite{chapman_mathematical_1990},我们能够从玻尔兹曼方程得到越来越精确的一系列近似解,纳维-斯托克斯方程仅仅是其中的一个低阶项而已。
    在Chapman-Enskog级数之外同样存在玻尔兹曼方程的解,如$\ee^{- 1 / \epsilon}$形式的解,出现在冲击波中。    
    与玻尔兹曼方程的宏观极限有关的问题如今是偏微分方程领域的话题。
    我们现在仅仅讨论如何从玻尔兹曼方程得到纳维-斯托克斯方程,不讨论能够从玻尔兹曼方程得到的纳维-斯托克斯方程以外的流体力学方程。

    我们需要首先指定在玻尔兹曼方程的低能有效理论中被保留的变量。一个非常合理的粗粒化定义是
    \begin{equation}
        \expval{Q} = \frac{1}{\rho} \int \dd[3]{\vb*{v}} m Q(\vb*{r}, \vb*{v}) f(\vb*{r}, \vb*{v}, t),
        \label{eq:corse-grain-in-boltzman}
    \end{equation}
    其中$\rho$定义为
    \begin{equation}
        \rho = \int \dd[3]{\vb*{v}} m f(\vb*{r}, \vb*{v}, t).
    \end{equation}
    这里期望值只是将$\vb*{v}$平均了,保留了$\vb*{r}$。
    粒子数、动量、能量是任何情况下都能够保证有的守恒量,玻尔兹曼方程的宏观理论的基本自由度中应该是有它们的。
    于是我们尝试列写它们做了\eqref{eq:corse-grain-in-boltzman}粗粒化之后的运动方程,如果能够得到封闭的方程组,那么就得到了一个玻尔兹曼方程的低能有效理论。
    这里的具体计算可以在著名的朗道物理学教程的第十卷\cite{lifsic_physical_2008}的第一章中找到。

    先证明一个有用的结论。我们可以通过加入适当的$\delta$函数因子,将\eqref{eq:boltzmann-eq-with-force}右边的碰撞积分写成
    \begin{equation}
        C[f] = \int \dd[3]{\vb*{v}_2} \int \dd[3]{\vb*{v}'} \int \dd[3]{\vb*{v}_2'} w(\vb*{p}, \vb*{p}_2, \vb*{p}', \vb*{p}_2') (f(\vb*{r}, \vb*{v}', t) f(\vb*{r}, \vb*{v}'_{2}, t) - f(\vb*{r}, \vb*{v}, t) f(\vb*{r}, \vb*{v}_{2}, t)),
    \end{equation}
    其中$w$是$\sigma \abs{\vb*{v} - \vb*{v}_2}$乘以适当的保证能量守恒和动量守恒的$\delta$函数因子。
    可以验证,由于时间反演不变性,交换$(\vb*{p}, \vb*{p}_2)$并且同时交换$(\vb*{p}', \vb*{p}_2')$之后$w$不变。
    将\eqref{eq:boltzmann-eq-with-force}右边的碰撞积分中关于入射速度$\vb*{v}$的一个任意的函数为$\varphi(\vb*{p})$,简记为$\varphi$,则
    \[
        \int \dd[3]{\vb*{v}} \varphi(\vb*{p}) C[f] = \int \dd[3]{\vb*{v}} \int \dd[3]{\vb*{v}_2} \int \dd[3]{\vb*{v}'} \int \dd[3]{\vb*{v}_2'} \varphi(\vb*{p}) w(\vb*{p}, \vb*{p}_2, \vb*{p}', \vb*{p}_2') (f' f_2' - f f_2).
    \]
    上式的积分测度现在是高度对称的,不妨记作$\dd{\Gamma}$。
    在上式的第二项中交换$(\vb*{p}, \vb*{p}_2)$并且同时交换$(\vb*{p}', \vb*{p}_2')$之后,得到
    \[
        \int \dd[3]{\vb*{v}} \varphi(\vb*{p}) C[f] = \int \dd{\Gamma} (\varphi - \varphi') w f' f_2'.
    \]
    上式在交换$(\vb*{p}, \vb*{p}_2)$同时交换$(\vb*{p}', \vb*{p}_2')$的变换下不变,则
    \[
        \begin{aligned}
            \int \dd[3]{\vb*{v}} \varphi(\vb*{p}) C[f] &= \int \dd{\Gamma} ((\varphi - \varphi') w f' f_2' + (\varphi_2 - \varphi_2') w f'_2 f') / 2 \\
            &= \int \dd{\Gamma} (\varphi + \varphi_2 - \varphi' - \varphi'_2) w f'_2 f' / 2.
        \end{aligned}
    \]
    因此,如果$\varphi(\vb*{p})$对$\vb*{p}$求和是某种守恒量,那么相应的就有
    \begin{equation}
        \int \dd[3]{\vb*{v}} \varphi(\vb*{p}) C[f] = 0.
        \label{eq:collision-conservation}
    \end{equation}

    将\eqref{eq:boltzmann-eq-with-force}写成简约形式
    \begin{equation}
        \pdv{f}{t} + \vb*{v} \cdot \pdv{f}{\vb*{r}} + \frac{\vb*{F}}{m} \cdot \pdv{f}{\vb*{v}} = C[f],
        \label{eq:boltmann-eq-simple-form}
    \end{equation}
    设单粒子质量为$m$,在上式两边乘上$m$并对$\vb*{v}$积分。
    第一项显然就是$\pdv*{\rho}{t}$。第二项根据\eqref{eq:corse-grain-in-boltzman},是
    \[
        \int \dd[3]{\vb*{v}} m \vb*{v} \cdot \pdv{f}{\vb*{r}} = \pdv{\vb*{r}} (\rho \expval{\vb*{v}}),
    \]
    第三项是典型的对散度积分的项,由于无穷远处粒子分布为零而为零。
    方程右边根据\eqref{eq:collision-conservation}为零。
    于是我们有
    \begin{equation}
        \pdv{\rho}{t} + \div{(\rho \vb*{u})} = 0,
        \label{eq:ns-particle}
    \end{equation}
    即粒子数守恒方程;这里我们定义
    \begin{equation}
        \vb*{u} = \expval{\vb*{v}}, \quad \vb*{w} = \vb*{v} - \vb*{u}.
    \end{equation}
    
    类似的,我们在玻尔兹曼方程\eqref{eq:boltmann-eq-simple-form}两边乘上$m v_i$,并对$\vb*{v}$积分,由于动量守恒,根据\eqref{eq:collision-conservation},方程右边为零,方程左边则是
    \[
        \pdv{t} \int \dd[3]{\vb*{v}} m v_i f + \int \dd[3]{\vb*{v}} m v_i v_j \pdv{f}{r_j} + \int \dd[3]{\vb*{v}} v_i F_j \pdv{f}{v_j} = 0.
    \]
    第一项是$\pdv*{(\rho u_i)}{t}$,第二项是
    \[
        \pdv{r_j} \int \dd[3]{\vb*{v}} mf v_j v_i = \pdv{r_j} (\rho \expval{v_j v_i}) = \pdv{r_j} (\rho u_j u_i + \rho \expval{w_j w_i}).
    \]
    第三项通过分部积分法可以化简为
    \[
        \int \dd[3]{\vb*{v}} v_i F_j \pdv{f}{v_j} = - \int \dd[3]{\vb*{v}} \pdv*{v_i}{v_j} F_j f = - \int \dd[3]{\vb*{v}} m f \frac{F_j}{m} = - \rho f_j,
    \]
    其中我们定义
    \begin{equation}
        \vb*{f} = \expval{\frac{\vb*{F}}{m}}.
    \end{equation}
    这样就有
    \begin{equation}
        \pdv{(\rho u_i)}{t} + \pdv{(\rho u_j u_i)}{r_j} + \pdv{\rho \expval*{w_j w_i}}{r_j} = \rho f_i.
        \label{eq:ns-momentum-in-components}
    \end{equation}
    定义
    \begin{equation}
        \pi_{ij} = - \rho \expval*{w_i w_j},
    \end{equation}
    \eqref{eq:ns-momentum-in-components}就成为
    \begin{equation}
        \pdv{(\rho \vb*{u})}{t} + \div{(\rho \vb*{u} \vb*{u})} = \div{\vb*{\pi}} + \rho \vb*{f}.
        \label{eq:ns-momentum}
    \end{equation}
    我们由于没有计算$\vb*{\pi}$是什么,似乎得到了\eqref{eq:ns-momentum}也没有什么用,但是无论如何,$\vb*{\pi}$一定是$\vb*{r}, \vb*{v}, \rho, e$的函数,

    同样,在\eqref{eq:boltmann-eq-simple-form}两边乘上$m \vb*{v}^2 / 2$,根据\eqref{eq:collision-conservation},方程右边为零,方程左边则是
    \begin{equation}
        \pdv{e}{t} % TODO
        \label{eq:ns-energy}
    \end{equation}

    以上含有$w$的量或是待求解变量,或是可以通过状态方程写成待求解变量的函数,从而实际上方程\eqref{eq:ns-particle},\eqref{eq:ns-momentum}和\eqref{eq:ns-energy}联立,并给定$\vb*{\pi}$和$\vb*{u}, \rho, \vb*{r}, e$的关系(当然,这就是流体的\concept{状态方程},它的导出涉及流体的微观细节)就给出了封闭求解$\rho, \vb*{u}, e$需要的全部方程。
    这里的过程和常规的场论计算中计算被积掉的变量的关联函数而得到低能有效理论中的系数的步骤是完全一样的。
    将\eqref{eq:ns-particle}代入\eqref{eq:ns-momentum},可以将\eqref{eq:ns-momentum}转化为
    \begin{equation}
        \rho \pdv{\vb*{u}}{t} + \rho \vb*{u} \cdot \grad{\vb*{u}} = \div{\vb*{\pi}} + \rho \vb*{f},
    \end{equation}
    上式实际上就是牛顿定律作用于流体微团上得到的纳维-斯托克斯方程,将它和\eqref{eq:ns-particle}以及状态方程联立求解同样能够封闭求解$\rho, \vb*{u}, e$。
    类似的能量方程也可以被修改来包括流体的动能,% TODO

    在更加工程的资料中,可能基本的变量不是$\rho, \vb*{u}, e$,而是更加容易测量的$p, \vb*{u}, T$,即我们将状态方程反过来用,用$p, T$表示$\rho, e$,这当然也是可以的,不过并没有揭示问题的物理实质,并且,在一些比较特殊的系统中可能并没有良定义的压强和温度,此时我们必须回到$\rho, \vb*{u}, e$的组合上来。
\end{back}