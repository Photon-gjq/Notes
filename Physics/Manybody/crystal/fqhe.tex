\chapter{分数量子霍尔效应}

\section{$1/m$型分数量子霍尔效应}

Laughlin论证——以及有关的各种基于单电子图像的计算——只能够得到整数量子霍尔效应。
显然,唯一的可能是,分数量子霍尔效应是电子之间的相互作用产生的,并且只能发生在强关联系统中。
本节讨论
\[
    \nu = \frac{1}{m}, \quad m = 1, 3, 5, \ldots
\]
型的分数量子霍尔效应,这是分数量子霍尔效应中比较简单的一种情况。

\subsection{Laughlin波函数,Laughlin液体和其中的任意子}

直接从库伦排斥出发做场论计算分析强关联效应显然是不现实的。在\autoref{chap:interaction-transition}中我们通常是通过将重要的相互作用通道挑选出来,构造特定的模型研究强关联效应,但分数量子霍尔效应的研究实际上走了一条相当不同的道路。
Laughlin通过其天才的创造,一步到位地给出了能产生$1/m$分数霍尔效应的系统的基态波函数:
\begin{equation}
    \Psi_m(z_1, \ldots, z_{N}) = \prod_{i < j} (z_i - z_j)^{m} \ee^{- \sum_{{i}} \abs*{z_{{i}}}^2 / 4 l_0^2}, \quad m = 1, 3, 5, \ldots.
    \label{eq:laughlin-wavefunction-m}
\end{equation}
容易验证以上波函数满足交换反对称性;当$z_{\vb*{i}}$趋于$z_j$时波函数趋于零,这是多电子波函数的必然要求。
如果$m$是偶数,那么上式就不再是一个好的电子多体波函数了,因为交换反对称性没有。
这个波函数是怎么猜出来的还是有一些线索的:它实际上是推广了整数量子霍尔效应的多体波函数。
% TODO

Laughlin波函数虽然是猜测出来的,不过将它和数值计算出来的波函数相对比,两者通常吻合得比较好。
因此,作为讨论分数量子霍尔效应的起点它是很够格的。相应的,基态为Laughlin波函数的分数量子霍尔效应系统可以称为\concept{Laughlin液体},它的名字称为\emph{液体}是因为其中相互作用很重要,但是没有任何对称性自发破缺。
例如,可以验证电子数密度是均一的。我们下面就来分析Laughlin波函数的一些性质。

\subsubsection{与无动能的等离子体系统的对应}

Laughlin波函数的一个相当值得注意的地方是它能够和一个无动能(从而,无量子涨落)的等离子体系统对应。
考虑多粒子联合概率分布
\begin{equation}
    p(z_1, z_2, \ldots, z_N) = \abs*{\Psi(z_1, \ldots, z_N)}^2,
\end{equation}
则有
\begin{equation}
    p(z_1, \ldots, z_N) = \ee^{- \beta U(\{z_i\})},
\end{equation}
其中
\begin{equation}
    \beta = \frac{2}{m}, \quad U(z_1, \ldots, z_N) = - m^2 \sum_{i < j} \ln \abs*{z_i - z_j} + \frac{m}{4 l_0^2} \sum_i \abs*{z_i}^2.
\end{equation}
因此零温Laughlin波函数和一个逆温为$m$,能量为$U$的无动能等离子体系统对应。%
\footnote{
    这不是一个很奇怪的操作——我们在DFT中使用分数占据时实际上就是在这么处理。
}%

$U$的第一项的意义是显然的:二维平面上不同粒子之间的库伦排斥。其第二项

\subsubsection{Laughlin液体中的准空穴和准粒子}

由于我们是先有了Laughlin波函数基态波函数而并不知道怎么样的哈密顿量(或者说怎么样的晶格)才能够产生Laughlin波函数,实际上无法确切地知道Laughlin液体中的准粒子是什么样的。
我们只能够冒险\emph{猜测}一些看起来还像回事的准粒子和准空穴\emph{定义}确实是实际上的分数量子霍尔效应系统中的准粒子。
因此,本节我们将“生造”出一些看起来像是粒子和空穴的构造,并且分析它们的性质。

在位置$\xi$处有一个准空穴的波函数定义为
\begin{equation}
    \braket*{z_1, \ldots, z_n}{\Psi^\text{h}(\xi)} = \sqrt{C(\xi)} \prod_i (\xi - z_i) \Psi_m.
\end{equation}
这里$C(\xi)$是归一化因子;我们相信这是正确的准空穴波函数,是因为在$\xi$处电荷密度为零,因此看起来这里确实存在一个空穴。

从$\ket{\Psi}$中移走一个电子得到的波函数是什么样的?由于其它电子都没有任何变化,移走位置在$\xi$处的电子,得到的波函数应该是
\begin{equation}
    \braket*{z_1, \ldots, z_{N-1}}{\Psi_m \text{ removed one electron at $\xi$}} \propto \sum_{i=1}^{N-1} (\xi - z_i)^m \braket*{z_1, \ldots, z_{N-1}}{\Psi_m \text{ with $N-1$ electrons}},
\end{equation}
上式实际上就是将\eqref{eq:laughlin-wavefunction-m}中的一个电子坐标替换成$\xi$得到的结果。
然后我们就发现了这里有一些不同寻常的地方:移走一个电子相当于创生了$m$个准空穴——换句话说,一个准空穴等于$1/m$个电子空穴。

这真是匪夷所思。朗道费米液体中的准粒子就是经过修饰的电子;我们在\autoref{chap:low-and-super}里面讨论的有序相中的最低能的准粒子都是两个或多个电子凝聚而成的;在相对“正常”一些的强关联系统——如Luttinger液体中,也顶多存在自旋-电荷分离,怎么到了这里一个电荷还能够被拆分成几片?



\begin{back}{任意子和任意子统计}{anyons}
    费米子和玻色子的概念可以通过自由场的量子化直接得到。
    不过我们也不妨稍稍推广一下“点粒子”的定义:如果一个格点系统(连续空间可以先离散化)满足如下条件:
    \begin{itemize}
        \item 希尔伯特空间的某一组基矢量可以通过在每个格点上放置一个自然数$n_{\vb*{i}}$(所谓\concept{粒子数})标记。
        \item 哈密顿量中有若干项\concept{跃迁项}$t_{\vb*{i} \vb*{j}}$,其中$\vb*{i}$和$\vb*{j}$都是格点。跃迁项需要满足如下条件:
        \begin{itemize}
            \item 如果状态$\ket{\psi}$中$\vb*{i}$点没有粒子而$\vb*{j}$点有一个粒子,那么$t_{\vb*{i} \vb*{j}}$算符作用在$\ket{\psi}$上,会产生一个$\vb*{i}$点有一个粒子而$\vb*{j}$点没有粒子的态,乘上一个相位因子;该相位因子未必只和$\vb*{i}$和$\vb*{j}$有关,而是可以依赖于全局的信息。
            \item 如果$\vb*{j}$点根本没有粒子,那么$t_{\vb*{i} \vb*{j}}$作用到系统状态上得到零,或者说$t_{\vb*{i} \vb*{j}}$消灭$\vb*{j}$点没有粒子的态。
        \end{itemize}
        \item 如果$\vb*{i}, \vb*{j}, \vb*{k}, \vb*{l}$彼此不相同,则
        \begin{equation}
            \comm*{t_{\vb*{i} \vb*{j}}}{t_{\vb*{k} \vb*{l}}} = 0.
        \end{equation}
        这是为了保证哈密顿量的局域性。
    \end{itemize}
    那么就可以说这个系统是一个粒子系统。这几条假设捕捉了“粒子”的直觉概念的大部分。例如,\concept{弦算符}
    \[
        t_{\vb*{i}_1 \vb*{i}_2} t_{\vb*{i}_2 \vb*{i}_3} \cdots t_{\vb*{i}_{n-1} \vb*{i}_n} 
    \]
    将一个有一个粒子在$\vb*{i}_n$点,没有粒子在$\vb*{i}_1$点的状态转化为一个有一个粒子在$\vb*{i}_1$点而没有粒子在$\vb*{i}_n$点的状态乘上一个可能是全局性的复因子。

    这里要注意一个地方:实际上一种粒子的统计是由系统哈密顿量确定的,而不是由希尔伯特空间。
    硬球玻色子同样有不相容性,而费米子大抵也可以通过粗粒化而近似地认为“多个费米子处于同一状态”。
    但是,产生湮灭算符的对易和反对易性是确定的,而系统哈密顿量由费米子或玻色子场算符组成,因此是\emph{哈密顿量}决定了粒子的统计,或者更加准确地说是哈密顿量中的跃迁项决定了粒子的统计。
    这马上就会产生一个问题,就是如果哈密顿量可以拆分出两种不同的跃迁项怎么办——但是这实际上是没有任何问题的,例如我们都知道产生超导的是一个电子系统(费米子系统),但是其低能激发是库伯对(玻色子系统)。
    
    这些假设留下了很多变数。
    我们没有给出“粒子如何被产生”:实际上,对一些系统可能无法良定义何为“粒子产生”,也没有能够在局部激发出粒子的场算符。(这实际上是一件好事,因为这样一来,我们只需要通过某种方式确定系统的能量本征态可以使用某些“粒子”标记,就可以使用和普通的量子场论中非常类似的方法研究系统的行为,至于这些粒子是演生激发,拓扑缺陷还是别的什么很多时候不重要)
    我们也没有讨论粒子除了坐标以外还能有什么标签。
    在有多种粒子时,不同粒子的跃迁算符之间是什么关系也没有明确规定——没有什么规定它们一定要对易。

    我们现在来分类可能的跃迁项,或者说分析粒子能够有怎样的统计。
    首先注意跃迁项不改变系统中的总粒子数,从而可以单独分析含有$n$个粒子的系统中粒子的统计。
    含有$n$个粒子的系统中所有粒子位置确定时,系统构型的取值范围为$\otimes_{n} \mathbb{R}^d = \mathbb{R}^{nd}$。
    为了简化分析,我们假定每个空间点上同种类型的粒子只能有一个;如果有多个粒子,就讨论一个对偶的模型,在其中每个空间点被扩充为一个“停车场”,停车场内部可以有多个粒子,但是它们占据的空间点还是不一样的。
    这样,系统构型的取值范围为$\mathbb{R}^{nd}$去掉两个同种粒子占据同一个空间点的情况后剩下来的$\mathbb{Y}^{nd}$。
    由于前述定义中粒子没有被编号,即$\mathbb{R}^{nd}$中的$i$号位置坐标为$\vb*{r}_i$,$j$号位置坐标为$\vb*{r}_j$和$i$号位置坐标为$\vb*{r}_j$,$j$号位置坐标为$\vb*{r}_i$完全一样,只要粒子$i$和$j$是同种粒子,最终的系统构型取值范围为$X = \mathbb{Y}^{nd} / \mathcal{P}$,其中$\mathcal{P}$表示任意一个同种粒子的坐标重新排列。
    于是系统的希尔伯特空间的一组基矢量的标签可以取为$X$,希尔伯特空间为$\mathbb{C}^{X}$。

    现在考虑一系列圈状的弦算符的乘积。经典的$X$中的构型在这些弦算符代表的粒子位移操作下保持不变,因此,量子的$\mathbb{C}^X$应当携带这种“绕了一圈又绕回来了”的操作——或者说“交换”(注意这和$\mathcal{P}$这种直接交换$\vb*{r}_i$和$\vb*{r}_j$不一样,交换必须要在$X$空间中真的绕上一圈)——的一个表示。或者更清楚的说,$\mathbb{C}^X$应该携带$X$的基本群的一个表示。
    这个表示可以是一维的,此时两个任意子交换只会让态矢量多出来一个复因子,那么我们称涉及其中的任意子是阿贝尔的;这个表示也可以是高维的,此时涉及其中的任意子是非阿贝尔的。

    $d \geq 3$时能够证明$\mathbb{Y}^{nd}$单连通,此时$X$的基本群同构于$\mathcal{P}$,是不同种类的粒子的置换群的直积。
    置换群的一维表示要么是平凡的,要么偶置换取$1$而奇置换取$-1$,于是$d \geq 3$时阿贝尔任意子只有玻色子和费米子,或者有时候说没有阿贝尔任意子,虽然原则上可能有非阿贝尔任意子。% TODO:怎么没有?

    $d=2$时$X$的基本群同构于辫子群,于是可以有非平凡的、但是也不是$\pm 1$的阿贝尔任意子。
    这就是为什么我们时常在研究二维系统时看到任意子。
    我们前面都是假定粒子的坐标的取值范围为$\mathbb{R}$。但是实际上,粒子也可以在一个拓扑非平庸的空间中运行。
    因此任意子统计实际上能够“看到”空间的拓扑特性,因为它的可能选择由$X$的基本群决定。
    直觉上这是非常合理的,因为弦算符不是局域的算符,因此它当然能够探测空间的拓扑性质。

    除了交换相位未必是$\pm 1$以外,还应该注意到如果一个弦算符是闭环,它\emph{未必}等价于恒等操作。
    这在\autoref{back:anyon-tensor-category}中很重要。
\end{back}

\subsection{Chern-Simons理论}

\begin{back}{任意子和量子场论}{anyon-field-theory}
    任意子存在场论表述,具体方法是通过构造特定的规范场,通过A-B效应的方式引入额外的相位。这个规范场本身没有非平庸的动力学,然后通过相位$\int \dd{\vb*{r}} \cdot \vb*{A}$,一个粒子可以“看到”被它的路径环绕的其它粒子,从而将这样一个规范场和某种费米子或是玻色子耦合,就得到了任意子。
    
    % TODO:cyon,对cyon,flux看得见charge但是对anyon通常是看不见的
\end{back}



\begin{back}{任意子,张量范畴和量子信息}{anyon-tensor-category}
    二维系统中的任意子可以使用modular category描述。关于本节的各种定义的严格表述、交换图等、范畴论除了量子力学以外的其它例子等可以看\cite{beer2018categories},关于任意子的部分\cite{new_structures}中A categorical view of computing with anyons一章也提供了容易读懂的介绍。
    
    以彼此能够通过某些时间演化过程相互转化的量子系统为对象,以它们之间的时间演化过程为态射,可以组成一个范畴,具体来说是一个compact dagger monoidal category,其中monoidal代表某种范畴论意义下的直积存在,compact是确保投影算符$\dyad{\psi}$可以良定义的,dagger的意思就是共轭转置的代数结构。
    需注意正如费米子和玻色子多体波函数所示的那样,该compact dagger monoidal category中的张量积——将两个粒子放在一起这个操作——\emph{未必是}线性代数中的张量积。

    通过数自由度我们承认系统$A \otimes B$和$B \otimes A$之间应该具有同构,即总是存在同构$b_{AB} :A \otimes B \to B \otimes A$,但是一般来说,我们只能确信任意子系统是braided compact dagger monoidal category,而未必是symmetric compact dagger monoidal category。
    两者的区别在于,后者要求$b_{AB} \otimes b_{BA} = \mathrm{id}$,前者不需要;使用图形表示,就是后者无需区分braiding操作$b_{AB}$的图形表示中,$A$的轨迹和$B$的轨迹的上下叠放顺序,而前者需要区分。
    三维及以上的粒子由于是置换群的表示,因此由symmetric的范畴描述,没有非平庸的braiding结构;它们要么是费米子要么是玻色子。

    描述任意子的范畴还必须将任意子的有自相交的路径和没有自相交的路径区分开——或者说,需要将闭弦算符并非恒等算符这件事考虑在内。
    有自相交的路径上多了一个“自己绕一圈”的操作,这个操作必须服从特定的相容性条件,直观地说即 % TODO:截图
    满足以上条件的范畴称为ribbon category,因为以上结构图形化地展示,就是一个丝带被扭曲的过程。

    我们还要求任意两个对象之间的态射构成复数域上的向量空间,态射的张量积是双线性的等等。
    在compact dagger monoidal category中实际上已经能够定义一些态射的叠加等操作,不过我们要求它们构成\emph{复数域}上的向量空间,从而确保任意子范畴是一个量子力学理论而不是别的什么框架下的理论。

    由于种类确定的任意子除了坐标以外找不出更多的自由度,单任意子态是simple objects:它们的自态射一定是只能是$k \  \mathrm{id}, k \in \mathbb{C}$的形式,且两个不同的simple object之间没有态射。
    我们要求任意子范畴是semisimple的。这里semisimple的定义中的“直和”的物理意义略有些不清楚,不过可以做如下理解:设$\phi_1$和$\phi_2$是两种任意子,则$\phi_1 \otimes \phi_2$对应一个二维的希尔伯特空间,其中的波函数是两个单任意子态的线性叠加。
    另一方面,$\phi_1 \otimes \phi_1$同样对应一个二维的希尔伯特空间,其中的波函数中任意子数目确定只有一个,但是$\phi_1$有“两种存在方式”:可能有一个额外的标签能够区分$\phi_1 \otimes \phi_1$的两个基矢量。
    这样理解的话,任意子范畴是semisimple的意思实际上就是两个任意子放在一起,远看仍然是\emph{一个}任意子,只不过这个任意子的类型是不确定的。
    这意味着我们可以有所谓的\concept{融合规则}:对任意两种任意子$i$和$j$,存在一系列自然数$\{N_{ij}^k\}$满足
    \begin{equation}
        i \otimes j = \sum_k N_{ij}^k k, \quad N_{ij}^k = \dim \mathrm{Hom}(i \otimes j, k).
    \end{equation}

    我们还需要引入迹的定义,这是比较直截了当的。

    最后一个需要的元素是$S$矩阵——不是散射振幅的$S$矩阵而是CFT中的。 % TODO,不过估计也不会仔细看

    满足以上条件的范畴称为\concept{modular tensor category (MTC)}。任意子体系中的各种状态和它们之间的变换原则上都可以纳入这样一个范畴中。
    有有限个同构等价类的MTC也可以称为fusion category。
\end{back}

\section{层级分数量子霍尔效应}