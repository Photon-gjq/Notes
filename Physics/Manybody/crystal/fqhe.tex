\chapter{分数量子霍尔效应}

\section{$1/m$型分数量子霍尔效应}

既然Laughlin论证只能够得到整数量子霍尔效应,我们要问,分数量子霍尔效应是怎么产生的。
显然,唯一的可能是,电子之间的相互作用产生了分数阶能级。
本节讨论
\[
    \nu = \frac{1}{m}, \quad m = 1, 3, 5, \ldots
\]
型的分数量子霍尔效应,这是最简单的情况。

Laughlin通过其天才的创造,一步到位地给出了能产生分数霍尔效应的波函数:
\begin{equation}
    \Phi(z_1, \ldots, z_N) = \prod_{i < j} (z_{\vb*{i}} - z_j)^{m} \ee^{- \sum_{\vb*{i}} \abs*{z_{\vb*{i}}}^2 / 4 l_0^2}, \quad m = 1, 3, 5, \ldots.
\end{equation}
容易验证以上波函数满足交换反对称性;当$z_{\vb*{i}}$趋于$z_j$时波函数趋于零,这是多电子波函数的必然要求。

\begin{back}{任意子激发}{anyons}
    费米子和玻色子的概念可以通过自由场的量子化直接得到。
    不过我们也不妨稍稍推广一下“点粒子”的定义:如果一个格点系统(连续空间可以先离散化)满足如下条件:
    \begin{itemize}
        \item 希尔伯特空间的某一组基矢量可以通过在每个格点上放置一个自然数$n_{\vb*{i}}$(所谓\concept{粒子数})标记。
        \item 哈密顿量中有若干项\concept{跃迁项}$t_{\vb*{i} \vb*{j}}$,其中$\vb*{i}$和$\vb*{j}$都是格点。跃迁项需要满足如下条件:
        \begin{itemize}
            \item 如果状态$\ket{\psi}$中$\vb*{i}$点没有粒子而$\vb*{j}$点有一个粒子,那么$t_{\vb*{i} \vb*{j}}$算符作用在$\ket{\psi}$上,会产生一个$\vb*{i}$点有一个粒子而$\vb*{j}$点没有粒子的态,乘上一个相位因子;该相位因子未必只和$\vb*{i}$和$\vb*{j}$有关,而是可以依赖于全局的信息。
            \item 如果$\vb*{j}$点根本没有粒子,那么$t_{\vb*{i} \vb*{j}}$作用到系统状态上得到零,或者说$t_{\vb*{i} \vb*{j}}$消灭$\vb*{j}$点没有粒子的态。
        \end{itemize}
        \item 如果$\vb*{i}, \vb*{j}, \vb*{k}, \vb*{l}$彼此不相同,则
        \begin{equation}
            \comm*{t_{\vb*{i} \vb*{j}}}{t_{\vb*{k} \vb*{l}}} = 0.
        \end{equation}
        这是为了保证哈密顿量的局域性。
    \end{itemize}
    那么就可以说这个系统是一个粒子系统。这几条假设捕捉了“粒子”的直觉概念的大部分。例如,\concept{弦算符}
    \[
        t_{\vb*{i}_1 \vb*{i}_2} t_{\vb*{i}_2 \vb*{i}_3} \cdots t_{\vb*{i}_{n-1} \vb*{i}_n} 
    \]
    将一个有一个粒子在$\vb*{i}_n$点,没有粒子在$\vb*{i}_1$点的状态转化为一个有一个粒子在$\vb*{i}_1$点而没有粒子在$\vb*{i}_n$点的状态乘上一个可能是全局性的复因子。
    
    这些假设留下了很多变数。
    我们没有给出“粒子如何被产生”:实际上,对一些系统可能无法良定义何为“粒子产生”,也没有能够在局部激发出粒子的场算符。
    我们也没有讨论粒子除了坐标以外还能有什么标签。
    在有多种粒子时,不同粒子的跃迁算符之间是什么关系也没有明确规定——没有什么规定它们一定要对易。

    这里要注意一个地方:实际上一种粒子的统计是由系统哈密顿量确定的,而不是由希尔伯特空间。
    硬球玻色子同样有不相容性,而费米子大抵也可以通过粗粒化而近似地认为“多个费米子处于同一状态”。
    但是,产生湮灭算符的对易和反对易性是确定的,而系统哈密顿量由费米子或玻色子场算符组成,因此是哈密顿量决定了粒子的统计。
\end{back}

