\chapter{分数量子霍尔效应}

\section{$1/m$型分数量子霍尔效应}

既然Laughlin论证只能够得到整数量子霍尔效应,我们要问,分数量子霍尔效应是怎么产生的。
显然,唯一的可能是,电子之间的相互作用产生了分数阶能级。
本节讨论
\[
    \nu = \frac{1}{m}, \quad m = 1, 3, 5, \ldots
\]
型的分数量子霍尔效应,这是最简单的情况。

Laughlin通过其天才的创造,一步到位地给出了能产生分数霍尔效应的波函数:
\begin{equation}
    \Phi(z_1, \ldots, z_N) = \prod_{i < j} (z_{\vb*{i}} - z_j)^{m} \ee^{- \sum_{\vb*{i}} \abs*{z_{\vb*{i}}}^2 / 4 l_0^2}, \quad m = 1, 3, 5, \ldots.
\end{equation}
容易验证以上波函数满足交换反对称性;当$z_{\vb*{i}}$趋于$z_j$时波函数趋于零,这是多电子波函数的必然要求。

\begin{back}{任意子和任意子统计}{anyons}
    费米子和玻色子的概念可以通过自由场的量子化直接得到。
    不过我们也不妨稍稍推广一下“点粒子”的定义:如果一个格点系统(连续空间可以先离散化)满足如下条件:
    \begin{itemize}
        \item 希尔伯特空间的某一组基矢量可以通过在每个格点上放置一个自然数$n_{\vb*{i}}$(所谓\concept{粒子数})标记。
        \item 哈密顿量中有若干项\concept{跃迁项}$t_{\vb*{i} \vb*{j}}$,其中$\vb*{i}$和$\vb*{j}$都是格点。跃迁项需要满足如下条件:
        \begin{itemize}
            \item 如果状态$\ket{\psi}$中$\vb*{i}$点没有粒子而$\vb*{j}$点有一个粒子,那么$t_{\vb*{i} \vb*{j}}$算符作用在$\ket{\psi}$上,会产生一个$\vb*{i}$点有一个粒子而$\vb*{j}$点没有粒子的态,乘上一个相位因子;该相位因子未必只和$\vb*{i}$和$\vb*{j}$有关,而是可以依赖于全局的信息。
            \item 如果$\vb*{j}$点根本没有粒子,那么$t_{\vb*{i} \vb*{j}}$作用到系统状态上得到零,或者说$t_{\vb*{i} \vb*{j}}$消灭$\vb*{j}$点没有粒子的态。
        \end{itemize}
        \item 如果$\vb*{i}, \vb*{j}, \vb*{k}, \vb*{l}$彼此不相同,则
        \begin{equation}
            \comm*{t_{\vb*{i} \vb*{j}}}{t_{\vb*{k} \vb*{l}}} = 0.
        \end{equation}
        这是为了保证哈密顿量的局域性。
    \end{itemize}
    那么就可以说这个系统是一个粒子系统。这几条假设捕捉了“粒子”的直觉概念的大部分。例如,\concept{弦算符}
    \[
        t_{\vb*{i}_1 \vb*{i}_2} t_{\vb*{i}_2 \vb*{i}_3} \cdots t_{\vb*{i}_{n-1} \vb*{i}_n} 
    \]
    将一个有一个粒子在$\vb*{i}_n$点,没有粒子在$\vb*{i}_1$点的状态转化为一个有一个粒子在$\vb*{i}_1$点而没有粒子在$\vb*{i}_n$点的状态乘上一个可能是全局性的复因子。

    这里要注意一个地方:实际上一种粒子的统计是由系统哈密顿量确定的,而不是由希尔伯特空间。
    硬球玻色子同样有不相容性,而费米子大抵也可以通过粗粒化而近似地认为“多个费米子处于同一状态”。
    但是,产生湮灭算符的对易和反对易性是确定的,而系统哈密顿量由费米子或玻色子场算符组成,因此是\emph{哈密顿量}决定了粒子的统计,或者更加准确地说是哈密顿量中的跃迁项决定了粒子的统计。
    这马上就会产生一个问题,就是如果哈密顿量可以拆分出两种不同的跃迁项怎么办——但是这实际上是没有任何问题的,例如我们都知道产生超导的是一个电子系统(费米子系统),但是其低能激发是库伯对(玻色子系统)。
    
    这些假设留下了很多变数。
    我们没有给出“粒子如何被产生”:实际上,对一些系统可能无法良定义何为“粒子产生”,也没有能够在局部激发出粒子的场算符。(这实际上是一件好事,因为这样一来,我们只需要通过某种方式确定系统的能量本征态可以使用某些“粒子”标记,就可以使用和普通的量子场论中非常类似的方法研究系统的行为,至于这些粒子是演生激发,拓扑缺陷还是别的什么很多时候不重要)
    我们也没有讨论粒子除了坐标以外还能有什么标签。
    在有多种粒子时,不同粒子的跃迁算符之间是什么关系也没有明确规定——没有什么规定它们一定要对易。

    我们现在来分类可能的跃迁项,或者说分析粒子能够有怎样的统计。
    首先注意跃迁项不改变系统中的总粒子数,从而可以单独分析含有$n$个粒子的系统中粒子的统计。
    含有$n$个粒子的系统中所有粒子位置确定时,系统构型的取值范围为$\otimes_{n} \mathbb{R}^d = \mathbb{R}^{nd}$。
    为了简化分析,我们假定每个空间点上同种类型的粒子只能有一个;如果有多个粒子,就讨论一个对偶的模型,在其中每个空间点被扩充为一个“停车场”,停车场内部可以有多个粒子,但是它们占据的空间点还是不一样的。
    这样,系统构型的取值范围为$\mathbb{R}^{nd}$去掉两个同种粒子占据同一个空间点的情况后剩下来的$\mathbb{Y}^{nd}$。
    由于前述定义中粒子没有被编号,即$\mathbb{R}^{nd}$中的$i$号位置坐标为$\vb*{r}_i$,$j$号位置坐标为$\vb*{r}_j$和$i$号位置坐标为$\vb*{r}_j$,$j$号位置坐标为$\vb*{r}_i$完全一样,只要粒子$i$和$j$是同种粒子,最终的系统构型取值范围为$X = \mathbb{Y}^{nd} / \mathcal{P}$,其中$\mathcal{P}$表示任意一个同种粒子的坐标重新排列。
    于是系统的希尔伯特空间的一组基矢量的标签可以取为$X$,希尔伯特空间为$\mathbb{C}^{X}$。

    现在考虑一系列圈状的弦算符的乘积。经典的$X$中的构型在这些弦算符代表的粒子位移操作下保持不变,因此,量子的$\mathbb{C}^X$应当携带这种“绕了一圈又绕回来了”的操作——或者说“交换”(注意这和$\mathcal{P}$这种直接交换$\vb{r}_i$和$\vb{r}_j$不一样,交换必须要在$X$空间中真的绕上一圈)——的一个表示。或者更清楚的说,$\mathbb{C}^X$应该携带$X$的基本群的一个表示。
    这个表示可以是一维的,此时两个任意子交换只会让态矢量多出来一个复因子,那么我们称涉及其中的任意子是阿贝尔的;这个表示也可以是高维的,此时涉及其中的任意子是非阿贝尔的。

    $d \geq 3$时能够证明$\mathbb{Y}^{nd}$单连通,此时$X$的基本群同构于$\mathcal{P}$,是不同种类的粒子的置换群的直积。
    置换群的一维表示要么是平凡的,要么偶置换取$1$而奇置换取$-1$,于是$d \geq 3$时阿贝尔任意子只有玻色子和费米子,或者有时候说没有阿贝尔任意子,虽然原则上可能有非阿贝尔任意子。

    $d=2$时$X$的基本群同构于辫子群,于是可以有非平凡的、但是也不是$\pm 1$的阿贝尔任意子。
    这就是为什么我们时常在研究二维系统时看到任意子。
    我们前面都是假定粒子的坐标的取值范围为$\mathbb{R}$。但是实际上,粒子也可以在一个拓扑非平庸的空间中运行。
    因此任意子统计实际上能够“看到”空间的拓扑特性,因为它的可能选择由$X$的基本群决定。
    直觉上这是非常合理的,因为弦算符不是局域的算符,因此它当然能够探测空间的拓扑性质。
\end{back}

