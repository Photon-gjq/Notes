\chapter{局域电子和自旋自由度}

\section{自旋自由度}

\subsection{自旋的路径积分}

自旋是可以使用路径积分描述的。这里的路径积分和通常的场论的路径积分不同,因为此时的基本自由度是自旋,没有形如$\comm*{x}{p} = \ii$这样的对易关系。
虽然如此,我们仍然可以有路径积分,因为我们有“对自旋构型的积分”——$SU(2)$上的Haar测度——并且实际上也有“自旋相干态”。
首先,我们知道$SU(2)$群的任何一个有限维不可约表示均可以使用如下的欧拉角表示出来:
\begin{equation}
    g(\phi, \theta, \psi) = \ee^{- \ii \phi S_3} \ee^{- \ii \theta S_2} \ee^{- \ii \psi S_3}, \quad \phi, \psi \in [0, 2\pi], \ \theta \in [0, \pi].
\end{equation}
设$\ket*{\uparrow}$为$S_3$的最高权本征态,即本征值最大的本征态,我们就有
\begin{equation}
    \ee^{- \ii \psi S_3} \ket*{\uparrow} = \ee^{- \ii \psi S} \ket*{\uparrow},
\end{equation}
其中$S$是$S_3$的最大本征值,对自旋$1/2$表示它是$1/2$,对自旋$1$表示它是$1$,等等。
我们注意到
\[
    \begin{aligned}
        \ii \mel*{\uparrow}{S_2}{\uparrow} &= \mel*{\uparrow}{\comm*{S_3}{S_1}}{\uparrow} \\
        &= S \mel*{\uparrow}{S_1}{\uparrow} - S \mel*{\uparrow}{S_1}{\uparrow} = 0,
    \end{aligned}
\]
同理
\[
    \mel*{\uparrow}{S_1}{\uparrow} = 0.
\]
定义
\begin{equation}
    \ket*{g} = g \ket*{\uparrow}, \quad g \in SU(2),
\end{equation}
称它为\concept{自旋相干态}。这个说法的依据在于,根据Haar测度的定义,有
\[
    h \int \dd{g} \dyad{g} = \int \dd{g} \dyad*{hg}{g} = \int \dd{g} \dyad*{g}{h^{-1} g} = \int \dd{g} \dyad{g} h, 
\]
根据不可约表示的Schur引理,我们有
\begin{equation}
    \int \dd{g} \dyad{g} = \const \times \mathrm{id}, 
\end{equation}
得到了定义路径积分需要的完备性关系,而$\ket*{g}$的地位和基于$\vb*{x}, \vb*{p}$的路径积分中的相干态类似。
使用“将时间分片并插入完备性关系”的方法,就有
\begin{equation}
    Z = \int \fd{g} \exp(\int_0^\beta \dd{\tau} (\braket*{\partial_\tau g}{g} - \mel*{g}{H}{g})).
    \label{eq:spin-partition-original}
\end{equation}
在这里我们可以看到,自旋的路径积分只涉及一类算符($\vb*{S}$的各个分量)而不是两类($\vb*{x}$和$\vb*{p}$),从而相干态路径积分看起来会简单一些;但是自旋的路径积分中$\vb*{S}$是在一个球上取值而不是在平直的坐标空间和动量空间中取值,并且彼此不对易,因此下面当我们把这些内积展开时又会有比坐标-动量路径积分更复杂的东西。

用欧拉角把$\ket*{g}$写出来就是
\[
    \ket*{g} = \ee^{- \ii \psi S} \ee^{- \ii \phi S_3} \ee^{- \ii \theta S_2} \ket*{\uparrow}.
\]
我们首先处理\eqref{eq:spin-partition-original}的第一项,我们有
\[
    \int_0^\beta \dd{\tau} \braket*{\partial_\tau g}{g} = \int_0^\beta \dd{\tau} (\ii S \partial_\tau \psi + \mel*{\uparrow}{\partial_\tau (\ee^{\ii \theta S_2} \ee^{\ii \phi S_3}) \ee^{- \ii \phi S_3} \ee^{- \ii \theta S_2}}{\uparrow} ),
\]
其中的第一项是零,因为$\psi$在$\tau = 0$和$\tau = \beta$处是相同的。
第二项是
\[
    \begin{aligned}
        \int_0^\beta \dd{\tau} \mel*{\uparrow}{\partial_\tau (\ee^{\ii \theta S_2} \ee^{\ii \phi S_3}) \ee^{- \ii \phi S_3} \ee^{- \ii \theta S_2}}{\uparrow} &= \int_0^\beta \dd{\tau} \ii \partial_\tau \theta \mel*{\uparrow}{S_2 \ee^{\ii \theta S_2} \ee^{\ii \phi S_3} \ee^{- \ii \phi S_3} \ee^{- \ii \theta S_2}}{\uparrow} \\
        &\quad + \int_0^\beta \dd{\tau} \mel*{\uparrow}{\ee^{\ii \theta S_2} \ii \partial_\tau \phi S_3 \ee^{\ii \phi S_3} \ee^{- \ii \phi S_3} \ee^{- \ii \theta S_2}}{\uparrow},
    \end{aligned}
\]
这里的第一项还是零,因为$\mel*{\uparrow}{S_2}{\uparrow}$是零。
第二项是
\[
    \begin{aligned}
        \int_0^\beta \dd{\tau} \mel*{\uparrow}{\ee^{\ii \theta S_2} \ii \partial_\tau \phi S_3 \ee^{\ii \phi S_3} \ee^{- \ii \phi S_3} \ee^{- \ii \theta S_2}}{\uparrow} &= \int_0^\beta \dd{\tau} \ii \partial_\tau \phi \mel*{\uparrow}{\ee^{\ii \theta S_2} S_3 \ee^{- \ii \theta S_2}}{\uparrow} \\
        &= \ii S \int_0^\beta \dd{\tau} \partial_\tau \phi \cos \theta,
    \end{aligned}
\]
这里我们用到了
\[
    \begin{aligned}
        \mel*{\uparrow}{\ee^{\ii \theta S_2} S_3 \ee^{- \ii \theta S_2}}{\uparrow} &= \mel*{\uparrow}{\ee^{\ii \theta [S_2, \ ]} S_3}{\uparrow} \\
        &= \mel*{\uparrow}{1 + (\ii \theta) \ii S_1 + \frac{(\ii \theta)^2 + \cdots}{2} S_3 }{\uparrow} \\
        &= \mel*{\uparrow}{1 - \frac{\theta^2}{2} S + \cdots}{\uparrow} = S \cos \theta, 
    \end{aligned}
\]
因此我们有
\begin{equation}
    \int_0^\beta \dd{\tau} \braket*{\partial_\tau g}{g} = \ii S \int_0^\beta \dd{\tau} \partial_\tau \phi \cos \theta.
\end{equation}

然后我们计算$\mel*{g}{H}{g}$,它基本上是$\vb*{S}$的线性函数加上某个常数项,因为自旋是矢量,而$H$是标量,因此$\vb*{S}$出现在$H$中的方式不是和某个另外的矢量点乘就是和自己点乘,或者是以上两者的多项式。
在一个不可约表示中$\vb*{S}^2 = S(S+1)$,是常数,因此$H$是$\vb*{B} \cdot \vb*{S}$的函数,其中$\vb*{B}$是某个矢量。
\[
    \begin{aligned}
        \mel*{g}{S_i}{g} &= \mel*{\uparrow}{\ee^{\ii \theta S_2} \ee^{\ii \phi S_3} S_i \ee^{- \ii \phi S_3} \ee^{- \ii \theta S_2}}{\uparrow}
    \end{aligned}
\]
这里我们用到了更加一般的
% TODO:BS^2这种怎么办?

\begin{equation}
    S[\theta, \phi] = S \int_0^\beta \dd{\tau} (B \cos \theta + \ii (1 - \cos \theta) \partial_\tau \phi).
\end{equation}
最终,$\psi$没有出现在路径积分中。

