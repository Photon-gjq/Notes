\chapter{铁磁性和反铁磁性的自旋模型}\label{chap:magnetic}

自旋是电子的内禀性质。然而,在一些情况下,电子几乎总是定域在某些格点附近而不发生移动,此时系统中不存在电子位置的变化,主要的自由度是各个格点上的自旋。
这样的模型即为所谓\concept{自旋模型}。自旋模型是非常常见的,如Hubbard模型在$U$非常大时,以动能项为微扰就能够得到一个海森堡模型。
对称性说明自旋-自旋相互作用通常可以取$\vb*{S}_{\vb*{i}} \cdot \vb*{S}_{\vb*{j}}$的形式,或者也许是各向异性的$\vb*{S}_{\vb*{i}} \cdot \vb*{T} \cdot \vb*{S}_{\vb*{j}}$。

自旋模型是解释铁磁性和反铁磁性的重要模型。

\section{海森堡模型}

\subsection{海森堡模型的哈密顿量}

\subsubsection{局域电子相互作用给出的海森堡模型}

考虑一种晶体中的局域化电子,在一个晶胞上只有一个,一方面未配对,一方面不能远距离移动。
在没有外加电磁场激励时它稳定地呆在自己的轨道上,不发生跃迁,因此作为一个低能有效理论,我们暂时只需要考虑它的位置。
切换到Wannier表象下,由于这种电子是高度局域的,它自己的轨道波函数几乎就是Wannier波函数,跃迁能力很差,能带很窄。
极端情况下动能项可以忽略,只需要考虑相互作用项。
我们正在讨论一个单带模型,且电子跃迁能力差,则哈密顿量形式为\eqref{eq:tight-binding-single-band-interaction},且无序项不存在。
由于假定每个晶胞上只有一个电子,化学势项、on-site repulsion项和密度-密度相互作用均可以直接略去,因为他们基本上是常数。
因此唯一剩下的就是自旋-自旋相互作用,因此哈密顿量为
\begin{equation}
    H = - J \sum_{\pair{\vb*{i}, \vb*{j}}} \vb*{S}_{\vb*{i}} \cdot \vb*{S}_{\vb*{j}},
    \label{eq:heisenberg-nearest}
\end{equation}
这里的$\vb*{S}_i$和$\vb*{S}_j$仍然是电子产生湮灭算符相乘得到的;然而,由于电子基本上没有跃迁,我们可以认为每个电子的$\vb*{i}$其实也是不怎么会变化的,因此我们可以\emph{彻底}忽略电子的轨道自由度,只保留自旋自由度。
这样就得到了自旋$1/2$的\concept{海森堡模型},如前所述,它是描述晶体中单占据、高度局域、跃迁能力差、绝缘的电子的模型。

回顾\eqref{eq:tight-binding-single-band-interaction}的导出,我们发现导致海森堡模型中的自旋-自旋耦合的主要是电子之间的交换相互作用。
我们后面会看到铁磁序和反铁磁序在海森堡模型中能够观察到,因此,这些磁性序的相当一部分是纯粹的量子效应的产物。

\subsubsection{更加一般的海森堡模型}

\eqref{eq:heisenberg-nearest}在两个方面可以推广:首先,可以有非最近邻的自旋-自旋相互作用;其次,实际上有相当一类体系不是自旋$1/2$的系统,但是仍然能够使用\eqref{eq:heisenberg-nearest}形式的哈密顿量描述。
\begin{equation}
    H = 
\end{equation}

\subsubsection{一般自旋的海森堡模型}



\subsection{磁性离子的铁磁序和自旋波}

\begin{back}{自旋自由度的一些性质}{spin-degree-of-freedom}
    用$S^j_{\vb*{i}}$表示格点$\vb*{i}$上的自旋,$j=1, 2, 3$或是$x, y, z$。通常习惯用$S^z$标记一个状态。
    我们有
    \begin{equation}
        \comm*{S^{j}_{\vb*{i}}}{{S}^{j'}_{\vb*{i}'}} = \ii \epsilon_{j j' j''} S_{\vb{i}}^{j''} \delta_{\vb*{i} \vb*{i}'},
    \end{equation}
    根据此关系可以对易所谓自旋升降算符,定义
    \begin{equation}
        {S}_{\vb*{i}}^+ = \frac{1}{\sqrt{2}} ({S}^x_{\vb*{i}} + \ii {S}^y_{\vb*{i}}) , \quad {S}_{\vb*{i}}^- = \frac{1}{\sqrt{2}} ({S}^x_{\vb*{i}} - \ii {S}^y_{\vb*{i}}) ,
    \end{equation}
    则有
    \begin{equation}
        \comm*{S^z_{\vb*{i}}}{S_{\vb*{i}}^+} = S^+_{\vb*{i}}, \quad \comm*{S^z_{\vb*{i}}}{S_{\vb*{i}}^-} = - S^-_{\vb*{i}},
    \end{equation}
    从而
    \begin{equation}
        \begin{aligned}
            S_{\vb*{i}}^+ \ket{\cdots, m, \cdots} &= \sqrt{\frac{(s + m + 1) (s - m)}{2}} \ket{m + 1}, \\
            S_{\vb*{i}}^- \ket{\cdots, m, \cdots} &= \sqrt{\frac{(s + m) (s - m + 1)}{2}} \ket{m - 1}.
        \end{aligned}
    \end{equation}
    特别的,对自旋$1/2$的情况,以$\ket{\uparrow}$和$\ket{\downarrow}$为基底,有
    \begin{equation}
        \vb*{S}_{\vb*{i}} = \frac{1}{2} \vb*{\sigma}_{\vb*{i}}, \quad {S}_{\vb*{i}}^+ = \frac{1}{\sqrt{2}} \pmqty{0 & 1 \\ 0 & 0}, \quad S_{\vb*{i}}^- = \frac{1}{\sqrt{2}} \pmqty{0 & 0 \\ 1 & 0}.
    \end{equation}
\end{back}

在$J > 0$时相邻自旋的相互作用会让自旋倾向于形成铁磁态。我们将海森堡哈密顿量用自旋升降算符写出,为
\begin{equation}
    H = 
\end{equation}

我们在这里只讨论铁磁序确确实实已经形成了的情况,不失一般性设其方向为$z$方向。

\subsection{反铁磁序}

海森堡模型中同时有$S^z$和$S^x$,$S^y$算符,因此存在很明显的量子涨落。
一些人猜测,这种量子涨落甚至可能在特定的晶格上完全破坏铁磁序!有关的情况见\autoref{chap:spin-liquid}。

\subsubsection{Hubbard模型和海森堡模型}

Goodenough规则:如果电子跃迁是在两个半满轨道之间,那么就是反铁磁序,如果电子跃迁是从一个半满轨道到一个空轨道,或是从一个全满轨道到一个半满轨道,那么就是铁磁序。

\section{各向异性海森堡模型}

% TODO:XXZ模型等

\section{伊辛模型}

\begin{equation}
    H = - J \sum_{\pair{\vb*{i}, \vb*{j}}} S_{\vb*{i}}^z S_{\vb*{j}}^z ,
\end{equation}
在自旋$1/2$的情况下,很多时候我们会重新定义$J$而使用泡利矩阵给出哈密顿量:
\begin{equation}
    H = - J \sum_{\pair{\vb*{i}, \vb*{j}}} \sigma^z_{\vb*{i}} \sigma^z_{\vb*{j}}.
\end{equation}

横场

\section{t-J模型}

\begin{equation}
    H = - t \sum_{\vb*{i}, \sigma} (c_{\vb*{i}})
\end{equation}

\section{XXZ模型}

\chapter{自旋玻璃}
