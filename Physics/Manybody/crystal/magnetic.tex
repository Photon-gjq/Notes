\chapter{铁磁性和反铁磁性的自旋模型}\label{chap:magnetic}

自旋是电子的内禀性质。然而,在一些情况下,电子几乎总是定域在某些格点附近而不发生移动,此时系统中不存在电子位置的变化,主要的自由度是各个格点上的自旋。
这样的模型即为所谓\concept{自旋模型}。自旋模型是非常常见的,如Hubbard模型在$U$非常大时,以动能项为微扰就能够得到一个海森堡模型。
对称性说明自旋-自旋相互作用通常可以取$\vb*{S}_{\vb*{i}} \cdot \vb*{S}_{\vb*{j}}$的形式,或者也许是各向异性的$\vb*{S}_{\vb*{i}} \cdot \vb*{T} \cdot \vb*{S}_{\vb*{j}}$。

自旋模型是解释铁磁性和反铁磁性的重要模型。

\section{海森堡模型}

\subsection{海森堡模型的哈密顿量}

\subsubsection{局域电子相互作用给出的海森堡模型}

考虑一种晶体中的局域化电子,在一个晶胞上只有一个,一方面未配对,一方面不能远距离移动。
在没有外加电磁场激励时它稳定地呆在自己的轨道上,不发生跃迁,因此作为一个低能有效理论,我们暂时只需要考虑它的位置。
切换到Wannier表象下,由于这种电子是高度局域的,它自己的轨道波函数几乎就是Wannier波函数,跃迁能力很差,能带很窄。
极端情况下动能项可以忽略,只需要考虑相互作用项。
我们正在讨论一个单带模型,且电子跃迁能力差,则哈密顿量形式为\eqref{eq:tight-binding-single-band-interaction},且无序项不存在。
由于假定每个晶胞上只有一个电子,化学势项、on-site repulsion项和密度-密度相互作用均可以直接略去,因为他们基本上是常数。
因此唯一剩下的就是自旋-自旋相互作用,因此哈密顿量为
\begin{equation}
    H = - J \sum_{\pair{\vb*{i}, \vb*{j}}} \vb*{S}_{\vb*{i}} \cdot \vb*{S}_{\vb*{j}},
    \label{eq:heisenberg-nearest}
\end{equation}
这里的$\vb*{S}_i$和$\vb*{S}_j$仍然是电子产生湮灭算符相乘得到的;然而,由于电子基本上没有跃迁,我们可以认为每个电子的$\vb*{i}$其实也是不怎么会变化的,因此我们可以\emph{彻底}忽略电子的轨道自由度,只保留自旋自由度。
这样就得到了自旋$1/2$的\concept{海森堡模型},如前所述,它是描述晶体中单占据、高度局域、跃迁能力差、绝缘的电子的模型。

回顾\eqref{eq:tight-binding-single-band-interaction}的导出,我们发现导致海森堡模型中的自旋-自旋耦合的主要是电子之间的交换相互作用。
我们后面会看到铁磁序和反铁磁序在海森堡模型中能够观察到,因此,这些磁性序的相当一部分是纯粹的量子效应的产物。

\subsubsection{一般自旋的海森堡模型}

\eqref{eq:heisenberg-nearest}在两个方面可以推广:首先,可以有非最近邻的自旋-自旋相互作用;其次,实际上有相当一类体系不是自旋$1/2$的系统,但是仍然能够使用\eqref{eq:heisenberg-nearest}形式的哈密顿量描述。
前者的原因是显然的:交换相互作用(见\eqref{eq:hartree-fock-scf-with-spin})不局限在最近邻电子。
对后者,我们考虑某种磁性离子中有若干电子,这些电子任取两个都有交换相互作用。忽略电子的轨道跃迁,则系统状态可以使用系统中各个电子的自旋标记,通过自旋角动量的合成,等于是说可以使用各个原子的自旋来标记系统状态。
因此,关于磁性离子的哈密顿量只包含一系列这些离子的自旋算符。
同一个原子上的电子彼此之间的交换相互作用无非导致哈密顿量中出现一个$\vb*{S}_{\vb*{i}}^n$的多项式,而原子1和原子2上的电子彼此的交换相互作用形如
\[
    - J_{12} \sum_{i, j} \vb*{S}_{1i} \cdot \vb*{S}_{2j} = - J_{12} \vb*{S}_1 \cdot \vb*{S}_2,
\]
这里$i$和$j$标记原子1和原子2上的不同电子。
海森堡模型主要是关于磁性序形成的,因此我们忽略$\vb*{S}_{\vb*{i}}^n$项。
对简单晶格,我们现在写出最一般的海森堡模型:
\begin{equation}
    H = - \sum_{\vb*{i}, \vb*{j}} J_{\vb*{i} \vb*{j}} \vb*{S}_{\vb*{i}} \cdot \vb*{S}_{\vb*{j}},
\end{equation}
其中$J_{\vb*{i} \vb*{j}} = J_{\vb*{j} \vb*{i}}$。
对复式晶格,不同子格上的$J$还可以不一样。

海森堡模型仍然有进一步扩充的空间。例如,没有什么保证哈密顿量只含有自旋的二次方;此外自旋相互作用也可以不是各向同性的。这些我们将在其它模型中引入。

\subsection{简单晶格中磁性原子的铁磁序和自旋波}

\subsubsection{铁磁序基态}

\begin{back}{自旋自由度的一些性质}{spin-degree-of-freedom}
    用$S^j_{\vb*{i}}$表示格点$\vb*{i}$上的自旋,$j=1, 2, 3$或是$x, y, z$。通常习惯用$S^z$标记一个状态。
    我们有
    \begin{equation}
        \comm*{S^{j}_{\vb*{i}}}{{S}^{j'}_{\vb*{i}'}} = \ii \epsilon_{j j' j''} S_{\vb{i}}^{j''} \delta_{\vb*{i} \vb*{i}'},
    \end{equation}
    根据此关系可以对易所谓自旋升降算符,定义
    \begin{equation}
        {S}_{\vb*{i}}^+ = \frac{1}{\sqrt{2}} ({S}^x_{\vb*{i}} + \ii {S}^y_{\vb*{i}}) , \quad {S}_{\vb*{i}}^- = \frac{1}{\sqrt{2}} ({S}^x_{\vb*{i}} - \ii {S}^y_{\vb*{i}}) ,
    \end{equation}
    则有
    \begin{equation}
        \comm*{S^z_{\vb*{i}}}{S_{\vb*{i}}^+} = S^+_{\vb*{i}}, \quad \comm*{S^z_{\vb*{i}}}{S_{\vb*{i}}^-} = - S^-_{\vb*{i}},
    \end{equation}
    从而
    \begin{equation}
        \begin{aligned}
            S_{\vb*{i}}^+ \ket{\cdots, m, \cdots} &= \sqrt{\frac{(s + m + 1) (s - m)}{2}} \ket{m + 1}, \\
            S_{\vb*{i}}^- \ket{\cdots, m, \cdots} &= \sqrt{\frac{(s + m) (s - m + 1)}{2}} \ket{m - 1}.
        \end{aligned}
    \end{equation}
    特别的,对自旋$1/2$的情况,以$\ket{\uparrow}$和$\ket{\downarrow}$为基底,有
    \begin{equation}
        \vb*{S}_{\vb*{i}} = \frac{1}{2} \vb*{\sigma}_{\vb*{i}}, \quad {S}_{\vb*{i}}^+ = \frac{1}{\sqrt{2}} \pmqty{0 & 1 \\ 0 & 0}, \quad S_{\vb*{i}}^- = \frac{1}{\sqrt{2}} \pmqty{0 & 0 \\ 1 & 0}.
    \end{equation}
\end{back}

在$J > 0$时相邻自旋的相互作用会让自旋倾向于形成铁磁态。不失一般性设其方向为$z$方向。我们将海森堡哈密顿量用自旋升降算符写出,为
\begin{equation}
    H = - J \sum_{\pair{\vb*{i}, \vb*{j}}} (S_{\vb*{i}}^z S_{\vb*{j}}^z + S^+_{\vb*{i}} S^-_{\vb*{j}} + S^-_{\vb*{i}} S^+_{\vb*{j}}).
    \label{eq:heisenberg-model-up-down}
\end{equation}
铁磁态为
\begin{equation}
    \ket{\text{FM}} = \ket{S_1^z = S, S_2^z = S, \ldots, S_N^z = S},
\end{equation}
因为铁磁态中所有的$S_{\vb*{i}}^z$都是最大的,任何一个$S^-$算符作用于其上给出的都是0,因此$H$作用在$\ket{FM}$上实际上就是一串$S^z$算符作用在$\ket{\text{FM}}$上,因此$\ket{\text{FM}}$是能量本征态,其能量为
\begin{equation}
    E_{\text{FM}} = - J \sum_{\pair{\vb*{i}, \vb*{j}}} S^2 = - J \frac{N S^2 n_\text{bond}}{2},
\end{equation}
其中$n_\text{bond}$是一个格点的最近邻格点数目。
由于$J > 0$,这是一个非常小的值。
其它用诸$S_{\vb*{i}}^z$标记的态不是能量本征态,并且它们的能量期望值要大于$E_\text{FM}$(因为会对能量期望值有贡献的只有$S^z_{\vb*{i}} S^z_{\vb*{j}}$一项)。
因此,$\ket{\text{FM}}$实际上是\emph{基态}。
当然由于\eqref{eq:heisenberg-nearest}具有自旋旋转不变性,自旋统一指向其它方向的态也是基态。
这是为数不多能够严格求解出基态的凝聚态物理模型,虽然三维情况下整个海森堡模型的能谱是一个到现在还没有求出的问题。

\subsubsection{自旋波}

\begin{back}{Holstein-Primakoff变换}{holstein-primakoff}
    \concept{Holstein-Primakoff变换}是一种将自旋自由度转化为玻色子自由度的局域变换。
    设某个自旋自由度的模长为$s$,$S^z$为$m$,定义
    \begin{equation}
        n = s - m,
    \end{equation}
    则有
    \begin{equation}
        S^+ \ket{n} = \sqrt{\frac{n (2s - n + 1)}{2}} \ket{n-1}, \quad S^- \ket{n} = \sqrt{\frac{(n+1)(2s-n)}{2}} \ket{n+1}.
    \end{equation}
    现在定义
    \begin{equation}
        S^+ = \sqrt{\frac{2s - a^\dagger a}{2}} a, \quad S^- = a^\dagger \sqrt{\frac{2s - a^\dagger a}{2}}, \quad S^z = s - a^\dagger a,
    \end{equation}
    则能够验证,对易关系
    \[
        \comm*{S^z}{S^+} = S^+, \quad \comm*{S^z}{S^-} = - S^-
    \]
    等价于
    \begin{equation}
        [a, a^\dagger] = 1, \quad [a, a] = [a^\dagger, a^\dagger] = 0.
    \end{equation}
    因此,可以认为
    \begin{equation}
        n = a^\dagger a,
    \end{equation}
    即自旋偏离$n$的多少等价于某种玻色子的数目,但是要注意该玻色子的数量有上限。

    不同自旋自由度的自旋算符彼此对易,因此相应的$a$也对易,即$a$确实是玻色子算符。
\end{back}

我们现在要分析铁磁态附近的激发。由于是磁性原子,我们假定每个原子的$S^2$相比于激发态的自旋偏移足够大,从而做Holstein-Primakoff变换之后可以不考虑玻色子数目有上限这件事。
对海森堡模型\eqref{eq:heisenberg-model-up-down}做Holstein-Primakoff变换,得到
\begin{align}
    \begin{autobreak}
        H = - J \sum_{\pair{\vb*{i}, \vb*{j}}} \Bigl( (S - a^\dagger_{\vb*{i}} a_{\vb*{i}}) (S - a^\dagger_{\vb*{j}} a_{\vb*{j}}) 
        + \frac{1}{2} \sqrt{2S - a^\dagger_{\vb*{i}} a_{\vb*{i}}} a_{\vb*{i}} a_{\vb*{j}}^\dagger \sqrt{2S - a^\dagger_{\vb*{j}} a_{\vb*{j}}} 
        + \frac{1}{2} a^\dagger_{\vb*{i}} \sqrt{2s - a^\dagger_{\vb*{i}} a_{\vb*{i}}} \sqrt{2s - a^\dagger_{\vb*{j}} a_{\vb*{j}}} a_{\vb*{j}} \Bigl).
    \end{autobreak}
\end{align}
由于$a$玻色子对应于某格点上自旋偏离基态的多少,我们称它为\concept{自旋波}。

由于只考虑低能激发态,对$a^\dagger_{\vb*{i}} a_{\vb*{i}}$做小量展开,有
\begin{align}
    \begin{autobreak}
        H = - \frac{1}{2} J S^2 N n_\text{bond} 
        + J S n_\text{bond} \sum_{\vb*{i}} a^\dagger_{\vb*{i}} a_{\vb*{i}}
        - J S \sum_{\pair{\vb*{i}, \vb*{j}}} (a^\dagger_{\vb*{i}} a_{\vb*{j}} + a^\dagger_{\vb*{j}} a_{\vb*{i}}) 
        - J \sum_{\pair{\vb*{i}, \vb*{j}}} \left(a^\dagger_{\vb*{i}} a_{\vb*{i}} a^\dagger_{\vb*{j}} a_{\vb*{j}} - \frac{1}{4} ( a^\dagger_{\vb*{i}} a_{\vb*{i}} a_{\vb*{i}} a^\dagger_{\vb*{j}} + a_{\vb*{i}} a^\dagger_{\vb*{j}} a^\dagger_{\vb*{j}} a_{\vb*{j}} + a^\dagger_{\vb*{i}} a^\dagger_{\vb*{i}} a_{\vb*{i}} a_{\vb*{j}} + a^\dagger_{\vb*{i}} a^\dagger_{\vb*{j}} a_{\vb*{j}} a_{\vb*{j}} ) \right).
    \end{autobreak}
\end{align}
上式中的第一项正是我们熟悉的基态能量,第二项是格点$\vb*{i}$上自旋偏离导致的局域能量,第三项给出不同格点的耦合。
剩下的各项是自旋波模式之间的耦合。

我们先只考虑二次型部分。和求解电子紧束缚模型类似,可以直接做傅里叶变换
\begin{equation}
    a_{\vb*{i}} = \frac{1}{\sqrt{N}} \sum_{\vb*{k}} \ee^{\ii \vb*{k} \cdot \vb*{R}_{\vb*{i}}} a_{\vb*{k}},
\end{equation}
得到
\begin{equation}
    H =  - \frac{1}{2} J S^2 N n_\text{bond} + \sum_{\vb*{k}} JS \left( n_\text{bond} - \sum_{\vb*{\delta}} \cos(\vb*{k} \cdot \vb*{\delta}) \right) a^\dagger_{\vb*{k}} a_{\vb*{k}},
\end{equation}
其中$\vb*{\delta}$指的是任意一个连接两个最近邻格点的矢量。
这样我们得到自旋波的色散关系
\begin{equation}
    \omega_{\vb*{k}} = JS \left( n_\text{bond} - \sum_{\vb*{\delta}} \cos(\vb*{k} \cdot \vb*{\delta}) \right).
\end{equation}



\subsection{反铁磁序}

与铁磁的情况不同,反铁磁序\emph{不是}海森堡模型的基态。直观地说,海森堡模型中同时有$S^z$和$S^x$,$S^y$算符,因此实际上存在很明显的量子涨落;我们只是足够幸运,哈密顿量作用到$\ket{\text{FM}}$上时有量子涨落的项正好给出零。
然而,在反铁磁的情况下,$E_\text{FM}$是非常大的能量,因此和模型的低能行为完全无关,这让我们无法利用$\ket{\text{FM}}$是能量本征态这一事实。
一些人猜测,海森堡模型中的量子涨落甚至可能在特定的晶格上完全破坏铁磁序!有关的情况见\autoref{chap:spin-liquid}。

\section{各向异性海森堡模型}

% TODO:XXZ模型等

\section{伊辛模型}

\concept{伊辛模型}是一种极端各向异性的自旋模型,只有$z$方向的自旋存在耦合,哈密顿量为
\begin{equation}
    H = - J \sum_{\pair{\vb*{i}, \vb*{j}}} S_{\vb*{i}}^z S_{\vb*{j}}^z ,
\end{equation}
在自旋$1/2$的情况下,很多时候我们会重新定义$J$而使用泡利矩阵给出哈密顿量:
\begin{equation}
    H = - J \sum_{\pair{\vb*{i}, \vb*{j}}} \sigma^z_{\vb*{i}} \sigma^z_{\vb*{j}}.
\end{equation}

\subsection{经典伊辛模型}

虽然各向异性通常会加大求解复杂程度,不过伊辛模型在没有外场或是外加磁场指向$z$方向时,即哈密顿量为
\begin{equation}
    H = - J \sum_{\pair{\vb*{i}, \vb*{j}}} \sigma^z_{\vb*{i}} \sigma^z_{\vb*{j}} + h \sum_{\vb*{i}} \sigma^z_{\vb*{i}}
\end{equation}
时是\emph{没有}量子涨落的,因而称为\concept{经典伊辛模型}。此时的伊辛模型是严格可解的,其基态、激发态非常清楚,甚至在一维和二维的时候,对应的配分函数都能解析计算。

\subsection{横场伊辛模型}

\concept{横场伊辛模型}通过引入$x$方向的磁场,制造了量子涨落,其哈密顿量为
\begin{equation}
    H = - J \sum_{\pair{\vb*{i}, \vb*{j}}} \sigma^z_{\vb*{i}} \sigma^x_{\vb*{j}} + h \sum_{\vb*{i}} \sigma^z_{\vb*{i}}.
\end{equation}

\chapter{自旋玻璃}
