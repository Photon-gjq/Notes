\chapter{弦网凝聚}

自旋液体的丰富演生行为——尤其是演生规范场——让我们

\section{Levin-Wen模型}

\concept{Levin-Wen模型}是\cite{Levin_2005}中提出的一类模型。

要从一个特定的Levin-Wen模型中提取出基态波函数中的拓扑序信息,可以做所谓的\emph{波函数重整化}:不断抹去局部的细节,知道达到一个不动点%
\footnote{
    需注意这和高能物理场论计算中的“波函数重整化”毫无关系,后者更确切的名字是“场重整化”。
}%
。这个操作和金斯堡-朗道理论中计算临界行为的步骤差不多,不同的是由于我们不需要保留动力学,操作的自由度可以大很多(但同时也缺乏系统的做重整化的步骤)。
由于我们只希望研究拓扑序,不动点处的哈密顿量应该是自由的,即应该是一系列局域的弦动能的和。
特别的,由于缺乏量子涨落,它应该是无阻挫的,从而能够将它划分成很多空间片段上的哈密顿之和,且在基态,每一个空间片段上的哈密顿量均被最小化。
设$\ket{\Phi_i}$是系统整体的基态$\ket{\Phi}$在$i$空间片段的投影,则只需要
\begin{equation}
    H_i \ket{\Phi_i} = E_\text{min} \ket{\Phi_i}
    \label{eq:string-local-constraint}
\end{equation}
我们就找到了基态。具体的$E_\text{min}$的数值等并不重要——重要的是,我们发现,系统基态可以通过施加一系列局域的约束而获得。
\eqref{eq:string-local-constraint}是线性的,因此这些局域约束也应该是线性的,具体来说,设$\{X_n\}$是空间片段$i$上的弦网构型,则\eqref{eq:string-local-constraint}实际上是关于$\{ \braket*{X_n}{\Phi_i} \}$的一个线性方程。
我们设$\Phi(X_n)$表示$\braket*{X_n}{\Phi_i}$,则局域约束就是关于$\{\Phi(X_n)\}$的线性约束,其中$\{X_n\}$是一系列通过$H_i$可以互相变换的同一个空间片段上的弦网构型。

指定这些局域约束实际上就是在定义怎样的基态波函数算是在不动点上的。
% TODO:说起来这里其实有一些彼此可能有冲突的要求……例如是不是总是能够找到又是自由的、无阻挫的,又是拓扑不变的哈密顿量呢?
