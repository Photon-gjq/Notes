\chapter{弦网凝聚}

自旋液体的丰富演生行为——尤其是演生规范场——让我们

\section{Levin-Wen模型}

\subsection{弦网系统的不动点和Levin-Wen模型}

\subsubsection{弦网系统}

\concept{Levin-Wen模型}是\cite{Levin_2005}中提出的一类模型。我们在Toric-code模型中已经看到过,在晶格的每个边上放置特定的自由度,则它们的特定取值可以排列出“弦”,这些弦或者两头连接着演生粒子,或者能够将一个演生粒子转移到另一个地方。
Levin-Wen模型\emph{不首先考虑}晶格的边上的自由度应该是什么,而是以这些弦为起点,定义希尔伯特空间和哈密顿量。

\Ztwo理论中的弦是没有定向的,但是一些其它的理论——比如说$U(1)$规范理论——中的弦就是有定向的。
一条弦的弦算符做共轭转置之后显然会给出一条定向相反的弦,而如果构成弦的局域算符不是厄米的,那么弦的类型也会发生变化。
此外,不妨稍微推广一下,允许弦“分支”,这样弦算符不仅能够转移粒子,还能够描述粒子的聚变或是裂变。
这样,一个弦网系统中的自由度由以下数据描述:
\begin{enumerate}
    \item 弦的类型:设共有$N$种弦,用$i=1, 2, \ldots, N$标记。
    \item 弦的共轭转置和定向:$i$类型的正向弦做共轭转置,得到$i^*$类型的反向弦;显然$i^{**}=i$。如果$i=i^*$,那么这种弦实际上是没有定向的。
    \item 分支规则:一系列三元组$\{i, j, k\}$,描述怎样的弦分支是可能的;需要注意有可能三个确定的$i, j, k$有不止一种连接方式,因此实际上在\emph{顶点}上有可能也需要放置一些自由度,用于确定连接到这个顶点上的三根弦的连接方式具体是哪一种。
\end{enumerate}
给定弦的类型的集合$\mathcal{I}$,$*: i \to i^*$运算,以及分支规则$\{ \{i, j, k\}, n \}$,一个弦网系统中允许的弦网构型就确定了,希尔伯特空间为
\begin{equation}
    \mathcal{H} = \otimes_s {\mathcal{H}_s} \otimes_v {\mathcal{H}_v}, \quad \mathcal{H}_s = \mathbb{C}^{\mathcal{I}}, \quad \mathcal{H}_v = \oplus_{i, j, k} \mathbb{C}^\text{ways of connection of $i, j, k$},
\end{equation}
其中$s$表示边而$v$表示格点。

可以看出前述定义中,完整描述一个弦网系统的数据实际上就是一个unitary dagger category $\mathcal{C}$,其对象为弦的末端的可能构型(单个点,两个点,多个点……),态射为弦,任意两个对象之间的态射是$\mathbb{C}$上的向量空间。
显然$\mathcal{C}$中的对象也可以使用弦来标记:将$i$类型弦的末端也表示为$i$即可,如果某个对象是多种弦的末端,把它表示成$i \otimes j \otimes k$即可。
注意在弦网模型中,弦一方面是算符,一方面,一个弦网算符作用在一个“空”态上就给出了一个弦网态,即弦也标记了系统的某一组基底。
这样就有
\begin{equation}
    \mathcal{H}_v = \otimes_{i, j, k} \hom(i \otimes j, k).
\end{equation}
我们目前还没有确定该范畴是不是张量范畴,或者说没有确定如此定义的$\otimes$是不是一个好的张量积,因为尚未确定$(i \otimes j) \otimes k \simeq i (j \otimes k)$之类的关系是不是成立。

以上对弦的作用的阐述——准粒子的产生、移动、聚变、裂变等——都是运动学;$\mathcal{C}$中的态射是否能够诠释为某些物理过程取决于哈密顿量是否提供了这样的相互作用通道。
弦网系统的哈密顿量大体上应该具有这样的形式:
\begin{equation}
    H = t H_t + U H_U,
\end{equation}
其中$H_U$是一个和(坐标表象下的;以下不注)弦网算符对易的厄米算符,较长的弦网算符倾向于具有较大的$U$,而$H_t$是一个和弦网算符不对易的算符,它的本征态是多个弦网态的线性组合。
$H_U$可以看成弦网的势能(而且还是弹性势能,可以认为是给弦一个张力),而$H_t$则是弦网的动能,因为它让弦网有量子涨落,也即,有时间演化。
这和粒子模型的哈密顿量的构造的思路是完全一样的。

我们已经见过一种这样的模型了:\Ztwo规范场。可以认为\Ztwo规范场中只有一种弦,是$\sigma^x$构成的,哈密顿量中,$\sum \sigma^x_{\vb*{i}}$项就是一个弦张力项,而$\sum_{\vb*{p}} \prod_{\vb*{i}} \sigma^z_{\vb*{i}}$项则是一个弦动能项。

$t \gg U$时系统基态是大量不同弦网态的线性组合,由于弦的非局域性,可以预期有系统拓扑保护的基态简并,而$U \gg t$时系统基态中没有多少弦。因此弦网模型总是有一个从弦网凝聚相到平凡相的量子相变。
弦网凝聚相显然是一个拓扑序:它有长程纠缠,并且有拓扑保护的基态简并。
弦网系统的哈密顿量可以是完全局域的、时间反演不变的,但是弦网凝聚态中存在非局域的弦网,并且准粒子——即弦的末端——的移动是被弦网构型记录下来的。
这暗示着弦网模型有着丰富的演生行为。

\subsubsection{波函数重整化}

要从一个特定的弦网模型中提取出基态波函数中的拓扑序信息,可以做所谓的\emph{波函数重整化}:不断抹去局部的细节,知道达到一个不动点%
\footnote{
    需注意这和高能物理场论计算中的“波函数重整化”毫无关系,后者更确切的名字是“场重整化”。
}%
。这个操作和金斯堡-朗道理论中计算临界行为的步骤差不多,不同的是由于我们不需要保留动力学,操作的自由性可以大很多,但同时也缺乏系统的做重整化的步骤,因此下面我们将只分析不动点处的哈密顿量和基态波函数应该满足什么性质。

由于我们只希望研究拓扑序,不动点处的哈密顿量应该是自由的,由于我们只关心弦网凝聚态,不动点处的哈密顿量即应该是一系列局域的弦动能的和。
特别的,由于缺乏量子涨落,它应该是无阻挫的,从而能够将它划分成很多空间片段上的哈密顿之和,且在基态,每一个空间片段上的哈密顿量均被最小化。
设$\ket{\Phi_i}$是系统整体的基态$\ket{\Phi}$在$i$空间片段的投影,则只需要
\begin{equation}
    H_i \ket{\Phi_i} = E_\text{min} \ket{\Phi_i}
    \label{eq:string-local-constraint}
\end{equation}
我们就找到了基态。具体的$E_\text{min}$的数值等并不重要——重要的是,我们发现,系统基态可以通过施加一系列局域的约束而获得。
\eqref{eq:string-local-constraint}是线性的,因此这些局域约束也应该是线性的,具体来说,设$\{X_n\}$是空间片段$i$上的弦网构型,则\eqref{eq:string-local-constraint}实际上是关于$\{ \braket*{X_n}{\Phi_i} \}$的一个线性方程。
我们设$\Phi(X_n)$表示$\braket*{X_n}{\Phi_i}$,则局域约束就是关于$\{\Phi(X_n)\}$的线性约束,其中$\{X_n\}$是一系列通过$H_i$可以互相变换的同一个空间片段上的弦网构型。
指定这些局域约束实际上就是在定义怎样的基态波函数算是在不动点上的。

这些约束在\cite{Levin_2005}中可以找到(方程(4)到(7))。

我们可能会担心这些要求是不是彼此冲突,例如是不是总是能够找到又是自由的、无阻挫的,又是拓扑不变的哈密顿量呢?
实际上是可以的,因为仅仅有这些条件是不足以确定$\mathcal{C}$是什么的,因此实际上,这些条件不仅没有彼此冲突,还留下了一定的调整的余地。
我们还可以再施加一些条件。一种常见的选择是要求弦的末端真的就是良定义的准粒子,这样$\otimes$是合理的张量积,从而
\[
    (i \otimes j) \otimes k \simeq i (j \otimes k),
\]
从而能够定义所谓的\concept{F-symbol}:
在没有指明其意义时,这个条件的形式看起来比较奇怪。实际上它也不是唯一的选择。

需注意可能并非所有弦网模型的波函数重整化不动点都满足这些条件。然而,满足以上条件的系统的行为已经足够丰富了,所以我们姑且暂时研究满足以上条件的系统。
给定一个unitary fusion category $\mathcal{C}$(无需是braided fusion category)就能够确定一个弦网模型在波函数重整化下的满足以上约束的不动点,且所有这类不动点都可以由某个unitary fusion category生成。

\subsubsection{Levin-Wen哈密顿量}

我们现在引入Levin-Wen模型的哈密顿量,它们是严格可解的,可以将一套指定的弦网系统实现出来。

\subsection{Levin-Wen模型的低能有效理论}

费米子、规范理论

\subsubsection{规范理论}

\subsubsection{Chern-Simons理论}

\subsection{边界态}
