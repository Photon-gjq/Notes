\section{较为寻常的金属和绝缘体}

在几节冗长的形式理论之后,我们开始分析一些实际的系统。本节讨论那些电子重整化后形成可以辨认的费米型准粒子的系统。
对这种系统,我们可以以某种电子气——自由电子气,或者紧束缚模型等——为自由理论,以库伦相互作用为微扰,或者,可以使用这种凝聚态场论简化后得到的动理学,通常就是玻尔兹曼方程,这样可以避开完整的Keldysh场论而讨论一些非平衡问题。
本节的内容将足够解释常见的关于金属和绝缘体的现象——实际上,“有费米面”几乎可以成为金属的一个定义。

\subsection{能带电子的波包动力学}

本节做如下近似:电子可以看成波包,要求它在动量空间比较局域,在实空间相对于晶格常数比较扩展,但是在宏观尺度上仍然足够局域。这通常要求原胞内的离子势没有剧烈变化,空间势的变化比较缓慢,外场没有破坏能带,外场变化足够慢,碰撞是瞬时发生的,散射较弱。 
它们的作用在后文的推导中可以看到。

\subsubsection{Bloch电子波包和波包动力学}\label{sec:bloch-wave-pocket}

考虑一个可以被看成晶格动量为$\vb*{k}_0$的Bloch电子波包。显然这个波包由同一个能带中的不同晶格动量成分组成(不同能带的电子具有非常不同的色散关系,它们各自运行的速度非常不同,用它们是无法组成一个比较稳定的波包的;以下略去能带标记$n$)。
注意波包的存在性本身就要求空间势的变化不能太剧烈,否则Bloch波函数在一个晶胞中将基本上定域在原子周围,无法形成比较“柔和”的波包。
波包中各个成分的晶格动量$\vb*{k}'$均很接近$\vb*{k}_0$,假定大体上
\begin{equation}
    - \frac{\Delta}{2} \leq k_i = k_{0i} - k'_i \leq \frac{\Delta}{2},
\end{equation}
从而可以做展开
\begin{equation}
    \epsilon_{\vb*{k}'} = \epsilon_{\vb*{k}_0} + \vb*{k} \cdot (\grad_{\vb*{k}} \epsilon_{\vb*{k}})|_{\vb*{k}_0}.
\end{equation}
在没有外场时Bloch波函数是能量本征态,其时间演化是平凡的,它们叠加而成的波包的形式形如
\[
    \psi(\vb*{r}, t) = \sum_{\vb*{k}'} a_{\vb*{k}} \psi_{\vb*{k}'}(\vb*{r}, t) = \sum_{\vb*{k}'} a_{\vb*{k}} \ee^{\ii (\vb*{k}' \cdot \vb*{r} - \epsilon_{\vb*{k}'})} u_{\vb*{k}'}(\vb*{r}, t),
\]
其中$a_{\vb*{k}}$满足归一化条件。代入$\epsilon_{\vb*{k}'}$的展开式,能够得到以下近似表达式
\[
    \begin{aligned}
        \psi(\vb*{r}, t) &= \ee^{\ii (\vb*{k}_0 \cdot \vb*{r} - \epsilon_{\vb*{k}_0} t)} \sum_{\vb*{k}} a_{\vb*{k}} u_{\vb*{k}'}(\vb*{r}) \ee^{\ii \vb*{k} \cdot (\vb*{r} - (\grad_{\vb*{k}} \epsilon_{\vb*{k}}) |_{\vb*{k}_0} t)} \\
        &= \ee^{\ii (\vb*{k}_0 \cdot \vb*{r} - \epsilon_{\vb*{k}_0} t)} u_{\vb*{k}_0}(\vb*{r}) \sum_{\vb*{k}} a_{\vb*{k}} \ee^{\ii \vb*{k} \cdot (\vb*{r} - (\grad_{\vb*{k}} \epsilon_{\vb*{k}}) |_{\vb*{k}_0} t)}  ,
    \end{aligned}
\]
因此可以看出,波包前进的速度是$\grad_{\vb*{k}} \epsilon_{\vb*{k}}$在$\vb*{k} = \vb*{k}_0$时的值。
这个对$\vb*{k}$的求和可以通过以下方式近似求出:我们认为% TODO

\begin{equation}
    \psi(\vb*{r}, t) = \ee^{\ii (\vb*{k}_0 \cdot \vb*{r} - \epsilon_{\vb*{k}_0} t)} u_{\vb*{k}_0}(\vb*{r}) % TODO
\end{equation}

$\Delta$是很小的,即$\Delta \ll 1 / a$,即
\begin{equation}
    \frac{1}{\Delta} \gg a,
\end{equation}
或者说,为了让波包描述有效,波包必须远大于原胞。这是合理的,因为如果使用波包的描述,我们是看不到晶格的,因此必须保证我们讨论的问题的尺度确实远大于晶格常数。

因此,结论是,一个“晶格动量大体上是$\vb*{k}_0$”的电子可以使用一个坐标$\vb*{r}$以及它的波函数的形状描述,其波函数是一个波包,在没有外场——从而系统的本征态就是Bloch电子——时,可以认为这个波包的运动速度由$\grad_{\vb*{k}} \epsilon_{\vb*{k}}$给出。

现在考虑施加了外场的情况。我们之前假设过外场的空间变化不大,那么外场可以近似通过以下势场引入:
\[
    V = \vb*{r} \cdot \vb*{F}.
\]
与自由电子的情况不同,Bloch波矢$\vb*{k}$不能写成一个解析形式非常简单的算符的本征值,因此我们采取一个略微迂回的方法,考虑某个初基格矢$\vb*{a}$方向上的平移算符
\[
    T \ket{\vb*{r}} = \ket{\vb*{r} + \vb*{a}},
\]
Bloch波函数是它的本征态,有
\[
    T \psi_{n \vb*{k}} = \ee^{\ii \vb*{k} \cdot \vb*{a}} \psi_{n \vb*{k}},
\]
于是通过计算$\expval{T}$的演化方程就能够计算出外场存在时波包的$\vb*{k}$的演化方程。
我们有
\[
    \dv{\expval{T}}{t} = \expval{\ii \comm*{H}{T}} = \ii \expval*{\comm*{V}{T}}。
\]
通过对$\ket{\vb*{r}}$的作用可以验证
\[
    \comm*{x_i}{T} = a T,
\]
其中$x_i$指的是$\vb*{a}$方向上的$\vb*{r}$分量;其它$\vb*{r}$分量和$T$对易。
于是
\[
    \comm*{V}{T} = F_i a T.
\]
由于我们假定了波包在实空间非常舒展、在倒空间比较狭窄,可以认为
\[
    \expval*{T} = \ee^{\ii \vb*{k} \cdot \vb*{a}},
\]
于是$\expval*{T}$的时间演化方程最终转化为
\[
    \dv{t} \ee^{\ii \vb*{k} \cdot \vb*{a}} = \ii F_i a \ee^{\ii \vb*{k} \cdot \vb*{a}},
\]
就有
\[
    \ii \vb*{a} \cdot \dot{\vb*{k}} = \ii F_i a = \ii \vb*{a} \cdot \vb*{F}.
\]
将$\vb*{a}$取遍三个初基格矢就得到
\[
    \dot{\vb*{k}} = \vb*{F}.
\]
于是,我们就得到了
\begin{equation}
    \vb*{v} = \grad_{\vb*{k}} \epsilon_{\vb*{k}}, \quad \dv{\vb*{k}}{t} = \vb*{F},
\end{equation}
联立求解这组方程就获得了外场下的能带电子运动情况。
实际上,我们有
\[
    \dv{\vb*{v}}{t} = \dv{t} \grad_{\vb*{k}} \epsilon_{\vb*{k}} = \dv{\vb*{k}}{t} \grad_{\vb*{k}} \grad_{\vb*{k}} \epsilon_{\vb*{k}},
\]
即
\begin{equation}
    \dv{\vb*{v}}{t} = \frac{1}{m^*} \vb*{F}.
\end{equation}
这里的$m^*$正是\eqref{eq:effective-mass}中的有效质量。上式是各向异性版本的牛顿第二定律,不过要注意$m^*$\emph{不是}把$[\pdv*[2]{\epsilon_{\vb*{k}}}{k_i}{k_j}]_{ij}$中的每个元素取倒数,并且可能有对$\vb*{k}$的依赖;此外这里的$\vb*{k}$一直就是Bloch波矢而不是广延波矢。

在本节给出的准经典理论中不存在从一个能带到另一个能带的跃迁。运动学上看,这是因为含有多个能带分量的波包会很快分裂开来,而只含有一个能带分量的波包则是相对稳定的。
考虑电子的能谱,这是因为不同能带之间的能隙可以看成一个势垒,从而,一个波包到了能带边界会很快被弹回来。

在一些情况下能带之间的隧穿是很重要的,此时本节给出的理论必须加以修正。

\subsubsection{外加电场下的电子振荡}

现在设一个恒定电场被加到晶体上,则会发生电子速度和晶格动量的振荡。
例如,考虑一个带底位于$\vb*{k}=0$处的能带,设$t=0$时电子在$\vb*{k}=0$处,由于是带底,有$\grad_{\vb*{k}} \epsilon_{\vb*{k}} = 0$,即电子无速度,而$m^* > 0$,于是电子速度增加。
而到了布里渊区边界,由于晶格周期势场,能带变平,此时为带顶,由于$\grad_{\vb*{k}} \epsilon_{\vb*{k}}$为零,电子速度为零,而由于是带顶,$m^* < 0$,于是电子被反向加速;注意此时$\vb*{k}$的变化方向还是沿着$\vb*{E}$的,$\vb*{k}$走出了第一布里渊区,或者等价地说,从第一布里渊区的另一侧进入了第一布里渊区。
这样,电子会从带底出发,首先被加速,到了一个$m^* = \infty$的地方时速度最大,然后减速,到了带顶速度正好降到零,然后$\vb*{k}$冲出第一布里渊区,而从第一布里渊区的另一侧回到第一布里渊区内,而$\vb*{v}$的方向则反了过来,最后回到带底。
在倒空间,电子从带底运动到带顶,然后从第一布里渊区的另一侧(也是带顶)回到带底;在实空间,出现能带倾斜,% TODO
这就是所谓的\concept{Bloch振荡}。

在实际实验中这个过程很难观察到,因为电子在运动中会不断受到介质中的其它准粒子和杂质的散射,这些本节没有考虑,而当电场强到电子运动得足够快,从而来不及被散射,让这种振荡得以发生时,往往或是材料被击穿,或是能带间隧穿会变得很明显,前一种情况下根本不能将介质看成晶体,后一种情况下波包动力学不可靠。

Bloch振荡的周期的量级和能带结构关系不大,而主要取决于晶格常数。电子在倒空间的运动速度为
\[
    \dv{k}{t} = e E,
\]
而第一布里渊区宽度大体上为$2 \pi / a$,因此
\begin{equation}
    \omega \sim e E a.
\end{equation}

\subsubsection{外加磁场下的电子振荡}

现在我们转而考虑有外加磁场时电子的响应,同样使用准经典的波包理论,并且忽略碰撞。
此时需要求解
\begin{equation}
    \vb*{v} = \grad_{\vb*{k}} \epsilon_{\vb*{k}}, \quad \dv{\vb*{k}}{t} = -e \vb*{v} \times \vb*{B}.
    \label{eq:calssical-magnetic-field}
\end{equation}
第二个方程意味着沿着磁场方向的$\vb*{k}$分量不发生改变,且由于洛伦兹力不做功,电子在$\vb*{k}$空间的运动轨迹是垂直磁场的平面与等能面的交线。
因此我们不做任何计算就确定了电子在倒空间中的行为。

例如,对自由电子可以解析求解以上方程,得到的是我们熟知的回转频率为
\begin{equation}
    \omega_0 = \frac{e B}{m}
\end{equation}
的螺旋运动。

\subsubsection{玻尔兹曼方程和电导}

现在将电子-电子相互作用纳入考虑。

\subsubsection{热力学性质}

金属中的准粒子主要就是电子和声子。按照\autoref{sec:lattice-special-heat},在极低温下声子能提供形如$T^3$的热容,而作为对比,极低温下电子热容正比于$T$(见\eqref{eq:free-electron-special-heat})。
因此在极低温下金属的比热主要由电子贡献,因为此时$T$相比$T^3$要明显。

\subsubsection{导电性分类}

\begin{itemize}
    \item \concept{绝缘体}就是有电子的能带被完全填满的系统。
    \item \concept{金属}就是有一些有电子的能带只是部分被填满的系统。
    \item \concept{半金属}是一些有电子的能带只有少量电子的系统,其导电性不好,但是仍然呈现一些金属的性质。
    \item \concept{半导体}是掺杂了的绝缘体,如果杂质能够形成能量大体上是在满带顶附近的局域空穴态,以及能量大体上是在空带底附近的局域电子态,那么热涨落就会让一些电子填充进空带中,而让满带中出现空穴,这就同时形成了两种载流子。
    不能依靠热涨落让电子从满带跳往空带,因为这要求费米温度量级的温度;我们需要掺杂才能够形成半导体。
\end{itemize}

能带论能够解释为什么绝缘体不导电。自由电子气模型无法解释为什么绝缘体不导电——绝缘体中同样有大量的电子,似乎本应该导电。

\subsection{强磁场下电子的量子理论}

磁场下电子实际上处于束缚态,因此可以预期,强磁场下将会出现明显偏离\eqref{eq:calssical-magnetic-field}的行为。

\subsubsection{朗道能级}

我们采用\concept{对称规范},设磁场方向为$z$方向,且设
\begin{equation}
    A_x = - \frac{1}{2} B y, \quad A_y = \frac{1}{2} B x,
\end{equation}
这样通过
\[
    \vb*{B} = \curl{\vb*{A}}
\]
就得到一个$z$方向上大小为$B$的磁场。系统的哈密顿量为
\begin{equation}
    {H} = \frac{\hbar^2}{2m} \left( - \ii \grad + \frac{\ii e}{c \hbar} \vb*{A} \right)^2.
    \label{eq:magnetic-hamiltonian}
\end{equation}
我们把$\hbar$放回了定义中,因为本节将需要频繁地用到这个能量尺度。
\eqref{eq:magnetic-hamiltonian}中的导数被协变导数替代了,以满足$U(1)$对称性。
由于$z$方向没有$\vb*{A}$分量,实际上可以将\eqref{eq:magnetic-hamiltonian}当成\emph{二维}电子气的哈密顿量,电子在$xy$平面上处于束缚态,而在$z$方向和自由电子没有区别。
因此接下来我们讨论二维电子气。

通过量纲分析可以发现\eqref{eq:magnetic-hamiltonian}中有一个特征长度
\begin{equation}
    l_0 = \sqrt{\frac{\hbar c}{e B}},
\end{equation}
同时可以看出磁通量子确实是磁通量的一个尺度:
\[
    \Phi_0 = \frac{h c}{e} \sim 2 B \pi l_0^2.
\]
这样,两个方向上的协变导数为
\begin{equation}
    D_x = \partial_x + \frac{\ii}{2l_0^2} y, \quad D_y = \partial_y - \frac{\ii}{2l_0^2}x.
\end{equation}
哈密顿量就转化为
\[
    {H} = - \frac{\hbar^2}{2m} (D_x^2 + D_y^2).
\]
容易验证$D_x$和$D_y$之间具有对易关系
现在定义
\begin{equation}
    D^\pm = D_x \pm \ii D_y,
\end{equation}
并定义复变量$z$为
\begin{equation}
    z = x + \ii y,
\end{equation}
则可以得到
\begin{equation}
    \comm*{D^-}{D^+} = - \frac{2}{l_0^2}.
\end{equation}
这意味着它们差一个常数就是一对升降算符。定义
\begin{equation}
    {a} = \ii \frac{l_0}{\sqrt{2}} D^-,
\end{equation}
就有
\begin{equation}
    {H} = \frac{\hbar^2}{m l_0^2} \left({a}^\dagger {a} + \frac{1}{2} \right).
\end{equation}
因此系统具有分立的能级,称为\concept{朗道能级}。% TODO:(Landau Lowest Level, LLL)
回过头看,$\frac{\hbar^2}{m l_0^2}$实际上是经典图景中电子在磁场下做匀速圆周运动的圆频率
% TODO
在磁场很弱的时候,这个能谱接近连续,此时可以使用\eqref{eq:magnetic-hamiltonian}。
把$z$方向的运动加上,就是
\begin{equation}
    {H} = \frac{\hbar^2}{m l_0^2} \left({a}^\dagger {a} + \frac{1}{2} \right) + \frac{p_z^2}{2m} .
\end{equation}

实际上,朗道能级对应的波函数可以写成一个全纯函数乘以一个高斯因子。设$\Psi$是基态波函数,我们做拟设
\begin{equation}
    \Psi(x, y) = f(z, z^*) \ee^{- z z^* / 4l_0^2},
    \label{eq:landau-wave-packet}
\end{equation}
由于
\[
    D^- \Psi = 0,
\]
展开计算可以发现
\[
    \pdv{f}{z^*} = 0,
\]
这表明$f$一定是全纯函数,即得所求证。

现在讨论朗道能级的简并度。在严格计算之前,可以大致地做一些数量级估计。
\eqref{eq:landau-wave-packet}给出了一个特征长度为$l_0$的波包。当然,将一个波包做平移之后还是可以得到一个波包。两个波包如果不重叠,就是彼此正交的系统的本征态。
这样,设体系总面积为$S$,我们可以认为体系被分割成了一系列大小为$\pi l_0^2$的圆,于是简并度为
\[
    \frac{S}{\pi l_0^2} \sim S \frac{eB}{h c} = \frac{\Phi}{\Phi_0}.
\]
其中定义
\begin{equation}
    \Phi_0 = \frac{h c}{e}
\end{equation}
为\concept{磁通量子}。

注意到由于$\Psi(\vb*{r})$可以被恰当地归一化,它在电子气占据的范围内不应该有任何奇点,于是可以做泰勒展开
\[
    f(z) = c_0 + c_1 z + c_2 z^2 + \cdots,
\]
实际上可以证明,$\{z^m \ee^{- \abs{z}^2 / 4 l_0^2}\}$构成一组完备正交基。
现在寻找波函数幅值最大的位置,即计算
\[
    \pdv{\abs{\Psi_n(\vb*{r})}^2}{r} = 0
\]
的解,其解为
\[
    r_n = \sqrt{2n} l_0.
\]
换而言之我们得到了一组年轮一样的基矢量。设基态简并度为$n$则
\[
    S = \sqrt{2n} l_0,
\]
从而
\[
    n = S \cdot \frac{eB}{\hbar c} = \frac{\Phi}{\Phi_0}.
\]

\subsection{高密度电子气的静电屏蔽和RPA近似}\label{sec:ext-e}

\subsubsection{托马斯-费米静态屏蔽势}

本节讨论一个可以忽略交换作用(什么时候交换相互作用可以忽略尚待分析)的电子气系统。由于可以忽略交换作用,可以完全照搬经典静电学,即认为一个电子的全部能量是其动能加上电势能,而不考虑任何交换能或者关联能。
想象我们在一个满足托马斯-费米近似的近独立电子气当中放入一个电荷。显然,异号电荷会移动到前者附近而产生一个屏蔽效应。
在电子气规模很大时(我们所研究的固体总是在热力学极限之下,因此这是成立的),屏蔽效应很强,这让外加电荷对电子气的扰动实际上局限在外加电荷的一个小邻域内。
不管怎么说求解外加电荷的影响就是求解以下自洽问题:
\begin{enumerate}
    \item 外加电荷导致外加电势能
    \begin{equation}
        \phi_\text{ext} = \frac{Q}{r},
    \end{equation}
    \item 外加电势与屏蔽电荷形成的电势$\var{\phi}$叠加,导致总的电势变化为
    \begin{equation}
        \phi_\text{eff} = \phi_\text{ext} + \var{\phi},
    \end{equation}
    \item 总电势变化$\phi_\text{eff}$导致了电荷密度变化$\var{n}$,电荷密度变化给出屏蔽电势$\var{\phi}$。
\end{enumerate}
自洽求解以上问题就可以确定所有物理量。

以上自洽求解步骤还需要一个方程:$V_\text{eff}$是怎么影响电荷密度的。
$V_\text{eff}$改变粒子排布的方式是这样的:单电子的能量$\epsilon_{\vb*{k}}$会由于$V_\text{eff}$的引入发生一个小的变化,且这个小的变化在不同位置通常是不一样的。
由于托马斯-费米近似成立,我们认为每个$\vb*{r}$位置附近的诸电子都可以认为是组成了一个正则系综%
\footnote{这么做总是可以的,因为我们讨论近平衡态理论,确实允许电荷密度在空间上出现分布,虽然完全的平衡态电子气应该是均匀的。}%
,且$\vb*{k}$和$\vb*{r}$可以同时确定,于是
\[
    n(\vb*{r}) = \sum_{\vb*{k}, \sigma} \frac{1}{1 + \ee^{\beta (\epsilon_{\vb*{k}}(\vb*{r}) - \mu)}} = 2 \sum_{\vb*{k}} \frac{1}{1 + \ee^{\beta (\epsilon_{\vb*{k}}(\vb*{r}) - \mu)}}, \quad \epsilon_{\vb*{k}}(\vb*{r}) = \epsilon_{\vb*{k}}^0 + V_\text{eff}(\vb*{r}),
\]
由于体系足够大,屏蔽作用让外加电荷的作用几乎是局域的,则$\var{n}(\vb*{r})$可以写成$V_\text{eff}(\vb*{r})$的函数而不涉及长程相互作用。
而由于外加电荷很小,可以采取线性近似,于是(第二个等号是因为费米-狄拉克分布的形式)
\[
    \var{n} = \dv{n}{V_\text{eff}} V_\text{eff} = \sum_{\vb*{k}} \pdv{n}{\epsilon_{\vb*{k}}} V_\text{eff} = - \pdv{n}{\mu} V_\text{eff},
\]
关于能量的态密度(即单位$\dd{\epsilon}$、单位体积中有多少可能的态)为
\begin{equation}
    N(\mu' - \mu) = \pdv{n\big|_{\mu \to \mu'}}{\mu'},
\end{equation}
其中能量从费米面量起,且我们将$n$的表达式中的$\mu$换成了变量$\mu'$,而使用$\mu$表示实际系统的化学势。使用该记号,则
\begin{equation}
    \var{n}(\vb*{r}) = - N(0) V_\text{eff}(\vb*{r}).
    \label{eq:from-veff-to-varn}
\end{equation}
这就是从$V_\text{eff}$计算$\var{n}$的方法。通过$\var{n}$计算$\var{\phi}$的方程为泊松方程
\[
    - \laplacian \phi_\text{eff} = 4 \pi \left( Q \delta(\vb*{r}) - e \var{n}(\vb*{r}) \right),
\]
其中我们不失一般性地将外加电荷放在原点上。使用傅里叶变换并且使用$\var{n}$和$V_\text{eff}$之间的线性关系\eqref{eq:from-veff-to-varn},可以得到
\[
    k^2 V_\text{eff}(\vb*{k}) = - 4\pi e Q - 4 \pi e^2 N(0) V_\text{eff}(\vb*{k}),
\]
解出$V_\text{eff}(\vb*{k})$为
\[
    V_\text{eff}(\vb*{k}) = - \frac{4 \pi e Q}{k^2 + \kappa
    _\text{TF}^2},
\]
做傅里叶逆变换得到
\begin{equation}
    V_\text{eff}(\vb*{r}) = - \frac{eQ}{r} \ee^{- \kappa_\text{TF} r},
    \label{eq:thomas-fermi-potential}
\end{equation}
其中
\begin{equation}
    \kappa_\text{TF}^2 = 4 \pi e^2 N(0).
\end{equation}
这个结果就是\concept{托马斯-费米屏蔽}。

\subsubsection{RPA近似下的动态屏蔽响应}

托马斯-费米屏蔽是静态的。现在考虑一般的近独立电子气的线性响应。
外加电势导致如下的相互作用哈密顿量:
\[
    {H}_\text{ext} = \int \dd[3]{\vb*{r}} {n}(\vb*{r}, t) V_\text{ext} (\vb*{r}, t),
\]
则它对电子数密度的影响可以使用(各向同性不含时体系的)推迟格林函数
\[
    G^\text{ret}_{nn}(\vb*{r}-\vb*{r}', t-t') = - \ii \theta(t-t') \expval*{\comm{\var{{n}}(\vb*{r}, t)}{\var{{n}}(\vb*{r}', t')}}
\]
描述,而响应就是%
\footnote{我们没有将$V_\text{ext}$写进${H}$中而是把它看成了外界扰动,这样电子气本身的哈密顿量仍然是空间平移不变的。}%
\begin{equation}
    \expval*{\var{{n}}(\vb*{r}, t)} = \int \dd{t'} G^\text{ret}_{nn}(\vb*{r}-\vb*{r}', t-t') V_\text{ext}(\vb*{r}', t'),
\end{equation}
在频域中就是
\begin{equation}
    \expval*{\var{{n}}(\vb*{k}, \omega)} = G^\text{ret}_{nn}(\vb*{k}, \omega) V_\text{ext}(\vb*{k}, \omega).
    \label{eq:green-function-electro-shielding}
\end{equation}
当然,在系统的构成已知的情况下,可以直接用费曼图计算出格林函数$G^\text{ret}_{nn}$,但是本节不这么处理问题。
但如果真的这么做很快就会遇到数学上的困难:费曼图求和需要经过一些特殊的重求和步骤才能够给出有意义的结果。
因此本节只是做一个近似,将最为奇异的那部分费曼图做一个求和,然后发现它确实能够给出有意义的结果。

为简化问题我们做\concept{RPA(Random Phase Approximation)近似}%
\footnote{这个名称最早来自核物理,在那里它是通过对某个随机相位的求和得到的。}%
,这个近似要求以下两点:
\begin{itemize}
    \item $\var{{n}}$和$V_\text{eff}$之间的响应函数近似是自由电子的$\var{{n}}$-$\var{{n}}$格林函数,即我们认为$V_\text{eff}$中来自电子屏蔽的部分在其它电子上的作用和一个外加势场完全没有区别;
    \item 只考虑经典静电能,即将电子数密度期望当成经典电子数密度,忽略所有交换能或关联能。
\end{itemize}
RPA近似看起来似乎没有什么明确的意义,但它实际上把最为奇异剧烈效应考虑在内,并且给出了定性上正确的屏蔽库伦势(从一个长程的相互作用变成了一个短程的相互作用)。如果RPA近似不准,只需要多算几个微扰就可以。
这样,自洽方程可以通过下面的方法给出。
\begin{enumerate}
    \item 外加电荷引入外加电势能,
    \item 外加电势能加上屏蔽电荷产生的电势能等于总的电势能变化$V_\text{ext}$,
    \item 总的电势能变化可以反推出屏蔽电荷。
\end{enumerate}

粒子数的平均值近似为一个经典的数密度,从而$\var{V}$可以使用泊松方程写成
\[
    - \laplacian \var{V}(\vb*{r}, t) = - 4\pi e^2 \expval*{\var{{n}}(\vb*{r}, t)},
\]
也即
\begin{equation}
    k^2 \var{V}(\vb*{k}, \omega) = 4\pi e^2 \expval*{\var{{n}}(\vb*{k}, \omega)}.
\end{equation}
于是第二步完成了。

第三步通过RPA近似的基本假设得到。这里我们用$G^\text{0, ret}_{nn} (\vb*{k}, \omega)$作为$G^\text{0, ret}_{\var{n}\var{n}} (\vb*{k}, \omega)$的简写:
\begin{equation}
    \expval*{\var{{n}}(\vb*{k}, \omega)} = G^\text{0, ret}_{nn} (\vb*{k}, \omega) V_\text{eff} (\vb*{k}, \omega).
    \label{eq:rpa-approximation}
\end{equation}
这样就得到自洽方程
\[
    V_\text{ext}(\vb*{k}, \omega) + \frac{4\pi e^2}{k^2} \expval*{\var{{n}}(\vb*{k}, \omega)} = (G^\text{0, ret}_{nn} (\vb*{k}, \omega))^{-1} \expval*{\var{{n}}(\vb*{k}, \omega)},
\]
从而
\begin{equation}
    G^\text{ret}_{nn}(\vb*{k}, \omega) = \frac{G^\text{0, ret}_{nn}(\vb*{k}, \omega)}{1 - \frac{4\pi e^2}{k^2} G^\text{0, ret}_{nn}(\vb*{k}, \omega)} = \frac{G^\text{0, ret}_{nn}(\vb*{k}, \omega)}{\epsilon(\vb*{k}, \omega)},
    \label{eq:ext-electron-ret-nn}
\end{equation}
其中$\epsilon(\vb*{k}, \omega)$是介电常数。
这样,计算出自由电子的密度-密度格林函数之后问题就完全解决了。

实际上从\eqref{eq:ext-electron-ret-nn}就可以看出RPA近似的实质。将\eqref{eq:ext-electron-ret-nn}做级数展开,我们有
\[
    G^\text{ret}_{nn}(\vb*{k}, \omega) = G^\text{0, ret}_{nn}(\vb*{k}, \omega) + G^\text{0, ret}_{nn}(\vb*{k}, \omega) \frac{4\pi e^2}{k^2} G^\text{0, ret}_{nn}(\vb*{k}, \omega) + \cdots,
\]
上式两边加上电子数密度平方,并注意到$G^\text{0, ret}_{nn}(\vb*{k}, \omega)$就是电子-空穴组成的环的贡献——依照定义,自由电子的密度-密度格林函数为
\[
    \begin{aligned}
        G_{nn}^\text{0}(\vb*{r}-\vb*{r}', \tau-\tau') &= - T_\tau \expval*{\var{{n}}(\vb*{r}, \tau) \var{{n}}(\vb*{r}', \tau')} \\
        &= - T_\tau \expval*{{{n}}(\vb*{r}, \tau) {{n}}(\vb*{r}', \tau')} + \expval*{{n}(\vb*{r}, \tau)} \expval*{{n}(\vb*{r}', \tau')} \\
        &= \sum_{\sigma, \sigma'} T_\tau \expval*{{\psi}_{\sigma'}(\vb*{r}', \tau') {\psi}^\dagger_\sigma(\vb*{r}, \tau)} T_\tau \expval*{ {\psi}_\sigma(\vb*{r}, \tau) {\psi}^\dagger_{\sigma'}(\vb*{r}', \tau') } ,
    \end{aligned}
\]
其中最后一个等号用到了Wick定理——我们就发现\eqref{eq:ext-electron-ret-nn}实际上就是环形图近似。
\eqref{eq:ext-electron-ret-nn}实际上给出了$G^\text{ret}_{nn}$中最为奇异的部分,因为环形图会产生次数非常高的$1/k^n$型的项,从而发散得最厉害;但是将所有这些发散的图求和起来却能够得到一个收敛的结果。
环形图近似的成立条件是粒子数密度很高,从而主要的相互作用形式是电子通过库仑相互作用,激发出一些电子-空穴对(即环形图中的环),即主要的相互作用是环形图。
也可以从另一个角度理解这件事:粒子数密度较高意味着屏蔽作用很强,所以$1/k^n$发散得不那么厉害的那些项没有什么贡献。
RPA近似下没有出现这样的图:一个电子回线内部有一条光子线,即在单电子回线的图中没有出现Fock项,这和托马斯-费米屏蔽的假设是一致的,因此确实可以预期,当$\omega \to 0$时RPA近似退化到托马斯-费米屏蔽上。

由于自旋守恒性,只有$\sigma = \sigma'$的项才非零,我们得到
\[
    G_{nn}^\text{0}(\vb*{r}-\vb*{r}', \tau-\tau') = \sum_\sigma G_\sigma(\vb*{r}' - \vb*{r}, \tau' - \tau) G_\sigma(\vb*{r} - \vb*{r}', \tau - \tau').
\]
上式具有卷积的形式,切换到频谱,就是
\[
    G_{nn}^\text{0}(\vb*{q}, \ii \omega_n) = \sum_{\sigma} \sum_{\vb*{k}} \frac{1}{\beta} \sum_{\nu_n} G_\sigma(\vb*{q}, \ii \nu_n) G_\sigma(\vb*{k} + \vb*{q}, \ii (\omega_n + \nu_n)),
\]
其中
\[
    \nu_n = \frac{\pi(2n+1)}{\beta}, \quad \omega_n = \frac{2\pi n}{\beta},
\]
前者是费米子的松原频率,后者则是玻色子的,因为电子配对形成玻色子。自由费米子的谱函数为
\[
    A(\vb*{k}, \omega) = \delta(\omega - \xi_{\vb*{k}}), \quad \xi_{\vb*{k}} = \frac{\vb*{k}^2}{2m} - \mu,
\]
格林函数为
\[
    G_\sigma(\vb*{r}, \ii \omega_n) = \frac{1}{\ii \omega_n - \xi_{\vb*{k}}}.
\]
于是,有
\[
    G_{nn}^\text{0}(\vb*{q}, \ii \omega_n) = 2 \sum_{\vb*{k}} \frac{1}{\beta} \sum_{\nu_n} \frac{1}{\ii \nu_n - \xi_{\vb*{k}}} \frac{1}{\ii \omega_n + \ii \nu_n - \xi_{\vb*{k}+\vb*{q}}}.
\]
要计算上式,常用的一个技巧是将求和转化为围道积分。注意到$\{\ii \nu_n\}$正是$f(z)$的全部极点,而且这些极点都是一阶极点。
容易看出这些极点的留数都是$-1/\beta$,于是对一个在虚轴上解析的函数$F(z)$,有
\[
    \frac{1}{\beta} \sum_{\nu_n} F(\ii \nu_n) = - \oint \frac{\dd{z}}{2\pi \ii} f(z) F(z),
\]
于是设围绕虚轴的环路为$C$,它实际上就是从$-\ii \infty + 0^+$到$\ii \infty + 0^+$,再从$\ii \infty - 0^+$到$-\ii \infty - 0^+$,
有
\[
    \begin{aligned}
        G_{nn}^\text{0}(\vb*{q}, \ii \omega_n) &= - 2 \sum_{\vb*{k}} \oint_C \frac{\dd{z}}{2\pi \ii} f(z) \frac{1}{z - \xi_{\vb*{k}}} \frac{1}{z +  \ii \omega_n - \xi_{\vb*{k}+\vb*{q}}} \\
        &= 2 \sum_{\vb*{k}} \left( \frac{f(\xi_{\vb*{k}})}{\xi_{\vb*{k}} + \ii \omega_n - \xi_{\vb*{k} + \vb*{q}}} + \frac{f(\xi_{\vb*{k} + \vb*{q}} - \ii \omega_n)}{\xi_{\vb*{k} + \vb*{q}} - \ii \omega_n - \xi_{\vb*{k}}} \right).
    \end{aligned}
\]
负号消失了是因为绕极点$z=\xi_{\vb*{k}}$和$z = \xi_{\vb*{k} + \vb*{q}} - \ii \omega_n$是顺时针而不是留数定理通常是的逆时针。
由$\{\omega_n\}$的定义,$f(\xi_{\vb*{k} + \vb*{q}} - \ii \omega_n)$就是$f(\xi_{\vb*{k} + \vb*{q}})$,于是得到松原格林函数
\begin{equation}
    G_{nn}^\text{0}(\vb*{q}, \ii \omega_n) = 2 \sum_{\vb*{k}} \frac{f(\xi_{\vb*{k}}) - f(\xi_{\vb*{k} + \vb*{q}})}{\ii \omega_n + \xi_{\vb*{k}} - \xi_{\vb*{k} + \vb*{q}}}.
\end{equation}
做解析延拓就得到推迟格林函数
\begin{equation}
    G_{nn}^\text{ret, 0}(\vb*{q}, \ii \omega_n) = 2 \sum_{\vb*{k}} \frac{f(\xi_{\vb*{k}}) - f(\xi_{\vb*{k} + \vb*{q}})}{\omega + \xi_{\vb*{k}} - \xi_{\vb*{k} + \vb*{q}} + \ii 0^+}.
    \label{eq:ext-electron-retarded-green-function}
\end{equation}
介电常数为
\begin{equation}
    \epsilon(\vb*{k}, \omega) = \frac{V_\text{ext}}{V_\text{eff}} = 1 - \frac{4 \pi e^2}{k^2} G^\text{ret, 0}_{nn}(\vb*{k}, \omega)
\end{equation}

\eqref{eq:ext-electron-retarded-green-function}可以推导出托马斯-费米屏蔽。
在$V_\text{eff}$在时间和空间上都变化非常缓慢时,只需要考虑$\vb*{q}$接近零的那部分傅里叶分量。
让$\omega$和$\vb*{q}$趋于零,我们有
\[
    G_{nn}^\text{ret, 0}(\vb*{q}, \ii \omega_n) = 2 \sum_{\vb*{k}} \dv{f(\xi_{\vb*{k}})}{\xi_{\vb*{k}}} = - 2 \sum_{\vb*{k}} \dv{f(\xi_{\vb*{k}})}{\mu} = - \pdv{n}{\mu} = - N(0),
\]
这正是我们在托马斯-费米屏蔽中做的近似:这表明RPA近似在低频下的确和托马斯-费米屏蔽一致,正如我们之前说明的那样。

\subsubsection{等离子体和等离激元}

考虑$\omega$非常高的极限,此时做展开
\[
    \frac{1}{\omega + \xi_{\vb*{k}} - \xi_{\vb*{k} + \vb*{q}} + \ii 0^+} = \frac{1}{\omega} - \frac{\xi_{\vb*{k}} - \xi_{\vb*{k} + \vb*{q}}}{\omega^2} + \cdots,
\]
展开到前两阶,代入\eqref{eq:ext-electron-retarded-green-function},得到
\[
    \begin{aligned}
        G_{nn}^\text{ret, 0}(\vb*{q}, \ii \omega_n) &= - \frac{2}{\omega^2} \sum_{\vb*{k}} (f(\xi_{\vb*{k}}) - f(\xi_{\vb*{k} + \vb*{q}})) (\xi_{\vb*{k}} - \xi_{\vb*{k} + \vb*{q}}) \\
        &= \frac{2 q^2}{m \omega^2} \sum_{\vb*{k}} f(\xi_{\vb*{k}}), 
    \end{aligned}
\]
于是就得到
\begin{equation}
    G_{nn}^\text{ret, 0}(\vb*{q}, \ii \omega_n) = \frac{q^2 n_\text{e}}{m \omega^2},
\end{equation}
相应的,
\[
    \epsilon(\vb*{k}, \omega \to 0) = 1 - \frac{4 \pi e^2 n_\text{e}}{m \omega^2},
\]
定义
\begin{equation}
    \omega_\text{p}^2 = \frac{4 \pi e^2 n_\text{e}}{m},
\end{equation}
就得到
\begin{equation}
    \epsilon(\vb*{k}, \omega \to 0) = 1 - \frac{\omega_\text{p}^2}{\omega^2}.
\end{equation}
考虑到\eqref{eq:ext-electron-ret-nn},当$\omega = \omega_\text{p}$时格林函数出现奇点,即此处有集体激发。
$G^\text{ret}_{nn}(\vb*{k}, \omega)$是一个密度-密度格林函数,对任意的动量,在$\omega = \omega_\text{p}$处都有一个密度波的集体激发模式。
我们此处的计算是在频率非常大,以至于介质几乎是等离子体时得到的,因此这种集体激发实际上正是\concept{等离激元}。
物理地看,等离激元是密度波这件事实际上说明了等离子体中的电子运动方式,即一个电子和一个空穴成对地振荡。
这个集体激发的频谱是平的,这当然是因为我们假定光速无穷大,完全使用库伦相互作用。如果考虑到光速有限,使用麦克斯韦方程可以推出
\begin{equation}
    \omega_{\vb*{k}}^2 = \omega_\text{p}^2 + c^2 k^2 / \epsilon(\infty),
\end{equation}
即光子的一个模式和固体发生耦合,打开了一个能隙。在假定光速无限时,丢弃不可见、恒定的无穷大项(相当于将元激发的产生湮灭算符的时间演化加上一个因子$\ee^{\ii c k t}$)就得到
\[
    \omega_{\vb*{k}} = \omega_\text{p},
\]
即我们算出的结果。

TODO:自能修正会影响吗?注意$-\Sigma$才是全体自能正规图的和。

\subsection{金属和朗道费米液体}\label{sec:landau-fermi-liquid}

金属中存在可以四处移动的巡游电子,并且库仑相互作用的量级通常和电子动能一样。
然而,金属的行为在很多方面非常像费米气体,这意味着实际上描述金属的低能有效理论基本上就是一个费米气体加上一定的相互作用。
\concept{朗道费米液体}是一个这样的低能有效理论,其基本假设为:
\begin{enumerate}
    \item 费米液体的状态可以和费米气体一样,使用某种费米型准粒子的占据数标记,并且系统能量可以写成占据数的函数。
    这个假设对哈密顿量的形式做出了很强的限制,因为此时的哈密顿量已经在粒子数表象下被严格对角化了。(但是这个哈密顿量仍然可以不是自由哈密顿量,因为可以有密度-密度相互作用)不过我们将会看到,这样的限制是合理的。
    \item 当相互作用趋于零(“被关闭”)时,费米液体态回退到实际的费米气体态,我们假设此时费米液体态的占据数和实际的费米气体态的占据数相同。
    这个假设实际上是说,费米液体中的占据数所描述的准粒子实际上是经过某种重整化的电子,和电子可以一一对应。
\end{enumerate}

可以通过这样的方式想象费米液体可以怎样产生:对无相互作用的能带电子%
\footnote{
    应当注意能带电子这个名称本身有些模糊不清。只计算周期性势场的影响可以得到能带,只计算经过平滑处理的赝势也可以得到能带,将一部分相互作用作为自能修正而留下最为显著的相互作用通道作为碰撞也可以得到能带(虽然还需要额外计算碰撞的影响,如费米液体中那样),对很多系统,其实相互作用全都考虑进去了还是可以得到能带——如同用DFT计算出来的那样。
    例如,在费米液体中,如果粒子数涨落不大,完全可以用粒子数平均值代替相互作用项中的准粒子占据数。
    即使相互作用大一些,也可以理解成“能带被拉歪了”。
}%
,我们可以浸染地将相互作用加入,如果没有出现诸如费米子配对之类的情况,那么第二个假设就已经满足了。
相互作用会导致电子出现自能修正,由于库仑相互作用的顶角是保持粒子数守恒的,不存在一个电子衰变成多个其它粒子的过程,因此自能修正是实数,即单电子态的确是本征态。%
\footnote{
    虽然如此,如果我们将高于费米面的电子称为电子而将低于费米面的电子称为空穴,那么的确可以出现一个电子和费米球中的空穴发生相互作用,而衰变成电子和空穴的过程。
    见\autoref{sec:fermi-liquid-ground}。
}%
在很多粒子数守恒的理论——如$\phi^4$理论或是QED——中,可以有稳定的单粒子、二粒子、三粒子……本征态,虽然$n$粒子态的能量不是$n$个粒子的能量简单相加,但是无论如何,一个本征态的能量可以写成各个动量上的粒子的占据数的函数,因此量子态可以用占据数标记,能量也可以用粒子数标记,即第一个假设也是成立的。

在什么情况下费米液体理论失效?如果相互作用是吸引的,那么低温下可能出现费米子配对,此时总是会出现相变。
还有一些情况下相互作用会导致电子在空间上排列成一定的模式,此时也会出现相变,因为出现了新的序。
我们将在\autoref{sec:low-and-super}中讨论这些情况。
如果相互作用较强,能带的图像甚至可能就完全不适用了,此时系统的元激发可能都不是经过修正的电子,也无所谓费米液体。
令人震惊的是,虽然大部分实际体系中库仑相互作用的确很强,并且显然不只有密度-密度相互作用通道,费米液体图像仍然是适用的。

由于费米液体中的准粒子是电子的单体激发,可以从费米液体中准粒子的行为中看到很明显的普通电子的影子,由于准粒子场和电子场的对称性相同,准粒子也是费米子,准粒子能谱和电子能谱形式类似,自旋均为$1/2$,带电量相同,等等。
唯象的讨论中可以直接将费米液体中的准粒子当成电子。

\subsubsection{基态附近的激发}\label{sec:fermi-liquid-ground}

与高能物理中不同,实际的凝聚态体系的基态中都有\emph{大量}准粒子。由于准粒子是费米子,系统基态中有一个费米球。
由于粒子数守恒,系统基态附近的激发态就是一些准粒子从费米球内部被打到费米球外部之后形成的,并且距离费米面均不远。
因此其实我们可以忽略费米海而认为系统中实际上有两种元激发:“电子”和“空穴”,而所谓的费米面就是区分电子和空穴的一个边界。

在费米液体理论中通常也只分析费米面附近的物理,部分原因在于费米海的结构可以非常复杂,因此只考虑费米面附近的物理是比较容易的,也是比较有实际意义的(因为是低温近似),部分原因在于只有这里才确定有稳定的准粒子——通常准粒子的寿命在接近费米面时比较长,因此看起来像是“真正的”粒子(否则会有非常明显的能级展宽)。

这件事的原因如下。首先应当注意,虽然准粒子的自能修正总是实数,从而,在没有其它任何准粒子时,单个准粒子可以永远稳定存在而不会衰变,但与高能物理中的情况不同,费米球的存在意味着如果费米面上方出现了一个准粒子,它会激发费米海内部的准粒子,从而产生一个空穴。
因此,在基态(不是真空态,而是带有费米球的态),费米面上方的准粒子的确会发生衰变。
将费米面上方的准粒子和费米球内部的空穴看成两种元激发可以更加清楚地看到这一点:此时费米面上方的准粒子的个数根本不是守恒的(从而,积掉费米海会产生一个非幺正的理论),从而其自能修正不可能是实数。
这是含有相互作用的多粒子态的一般特征:在这种态中没有非常“尖锐”的单粒子极点。

设准粒子寿命为$\tau$,则$\tau$反比于散射速率,而散射速率正比于库仑散射的强度。完整地做这个计算是很困难的,因为涉及到静电屏蔽等复杂的效应。
散射的过程可以概括为:一个动量为$\vb*{p}$的费米面外部的准粒子的能量降低,变成了动量为$\vb*{p}_1$的准粒子,同时激发了一个费米面内的动量为$\vb*{p}_2$的准粒子。
结果是,动量为$\vb*{p}$的费米面以外的准粒子衰变成了两个准粒子,动量分别为$\vb*{p}_1$和$\vb*{p} - \vb*{p}_1 + \vb*{p}_2$,还有一个动量为$\vb*{p}_2$的空穴。
如果这种散射实际上很少发生,那么就能够有稳定的准粒子。

设对某个费米型准粒子系统,准粒子之间的相互作用强度不非常强地依赖于动量(纯粹的库伦相互作用不满足这个条件,它强烈依赖入射电子的动量差,因此即使在费米面附近,相互作用强度也不能当作常数,因此能带电子加入库伦相互作用后并不严格就是稳定的准粒子,而仍然需要经过几轮重整化),不妨记作$M$。
在这样的系统中,动量分别为$\vb*{p}_1$和$\vb*{p} - \vb*{p}_1 + \vb*{p}_2$的两个准粒子和动量为$\vb*{p}_2$的空穴的联合态密度在当前温度下的期望值为$n$,由费米黄金法则有
\[
    \frac{1}{\tau} \propto \text{transition rate} \sim \abs{M}^2 n.
\]
由于系统中的准粒子非常多,不同能级上准粒子数的涨落可以略去,即认为不同能级上不多不少正好就有费米-狄拉克分布给出的粒子个数,%
\footnote{这是热力学系统的一般性质:系统规模大时涨落可略去。由于本文涉及的系统都是多体系统,总是可以做这样的近似。}%
那么就有
\[
    n = \int \dd[3]{\vb*{p}_1} \int \dd[3]{\vb*{p}_2} (1 - f(\epsilon_{\vb*{p}_1})) f(\epsilon_{\vb*{p}_2}) (1 - f(\epsilon_{\vb*{p} - \vb*{p}_1 + \vb*{p}_2})) \delta(\epsilon_{\vb*{p}} - \epsilon_{\vb*{p}_1} + \epsilon_{\vb*{p}_2} - \epsilon_{\vb*{p} - \vb*{p}_1 + \vb*{p}_2}).
\]
因子$(1-\epsilon_{\vb*{p}_1})$表示动量为$\vb*{p}_1$的粒子应该占据一个空态(或者说在接近零温时应该在费米面以外),因子$f(\epsilon_{\vb*{p}_2})$表示空穴一定来自一个已有的粒子,最后的$\delta$函数强制要求能量守恒。
我们不严格计算这个积分,而是做一些数量级估计。
由于$\vb*{p}_2$在费米面以下而$\vb*{p}- \vb*{p}_1 + \vb*{p}_2$在费米面以上,容易写出以下不等式
\[
    0 < \xi_{\vb*{p}_1} < \xi_{\vb*{p}}, \quad 0 < - \xi_{\vb*{p}_2} < \xi_{\vb*{p}} - \xi_{\vb*{p}_1} < \xi_{\vb*{p}},
\]
对$n$有贡献的$\vb*{p}_1$和$\vb*{p}_2$均满足这些不等式,这些不等式给出了两个宽度为$\xi_{\vb*{p}}$的球壳,因此
\[
    n \leq (4 \pi k_\text{F}^2 \xi_{\vb*{p}})^2,
\]
于是
\begin{equation}
    \frac{1}{\tau} \lesssim \xi_{\vb*{p}}^2.
\end{equation}
因此,如果准粒子非常接近费米液体的费米面,那么它是非常稳定的,因为此时$\xi_{\vb*{p}}$很小。从物理图像上看,此时的准粒子虽然会和费米海中的准粒子发生相互作用,但其能量不足以产生粒子-空穴对,因此也不会衰变。

总之,准粒子之间的碰撞不可忽略意味着费米面以上的准粒子会衰变。
然而,在费米面附近,这种衰变是非常缓慢的,以至于我们可以认为任何一种动量确定的准粒子分布都可以长期稳定存在,即哈密顿量在以动量标记的粒子数表象下是对角化的,从而费米液体理论的第一个假设成立。

% TODO:适用范围
最后分析费米液体适用条件。按照费米-狄拉克分布,$\expval*{{n}}$在$\epsilon_\text{F}$附近一个大约长为$T$的区域内明显偏离阶跃函数;另一方面,由于相互作用能量本身会有一个弥散,为
\[
    \Delta E \sim \frac{1}{\Delta t} \sim \frac{1}{\tau},
\]
其中$\tau$为准粒子平均自由时间。粒子稳定意味着
\[
    \Delta E \ll \frac{1}{T},
\]
也即
\[
    \frac{1}{\tau} \ll T.
\]
准粒子平均自由时间本身和温度有关,它大约是单位时间发生碰撞的概率的倒数,而只有费米子附近准粒子数明显偏离阶跃函数的那一部分准粒子比较有可能发生碰撞(费米球内部的准粒子不怎么会被激发,费米球外面根本没有准粒子),因此
\[
    \frac{1}{\tau} \sim T^2,
\]
最后就发现我们有
\[
    T \ll 1.
\]
因此费米液体图像只在低温下适用。

\subsubsection{哈密顿量和前向散射相互作用}

费米液体的能级结构和费米气体完全一致并不意味着电子之间的相互作用完全不产生任何影响,因为每个能级具体的能量大小可以经历修正。
此外,即使电子之间的相互作用可以忽略,向系统加入或取出电子也会让费米球发生变化,从而让系统总能量平摊到每个准粒子上的份额发生变化。
这意味着准粒子的哈密顿量并不是简单的
\[
    H = \sum_{\vb*{k}, \sigma} \epsilon_{\vb*{k}} n_{\vb*{k} \sigma},
\]
而有一些高阶项,它们表示密度-密度相互作用。

考虑一个能量本征态,其中准粒子在费米面之上的数量为$\var{n}$($\delta$表示相对基态的偏离,正表示有粒子,负表示有空穴)。
考虑到费米液体理论的第一条假设,把能量本征值相对于零温平衡态(由于费米海的结构可以非常复杂,零温能量反而是算不出来的,我们也不需要计算它)的变化写成以下级数展开($\vb*{k}$在费米面附近):
\begin{equation}
    \var{E} = \underbrace{\sum_{\vb*{k}, \sigma} \epsilon^0_{\vb*{k}} \var{n_{\vb*{k} \sigma}}}_{\var{E_1}} + \underbrace{\frac{1}{2V} \sum_{\vb*{k}, \vb*{k}', \sigma, \sigma'} f_{\sigma \sigma' \vb*{k} \vb*{k}'} \var{n_{\vb*{k} \sigma}} \var{n_{\vb*{k}' \sigma'}}}_{\var{E_2}},
    \label{eq:fermi-liquid-energy}
\end{equation}
其中$\var{n}$表示准粒子数目相对基态的变化。把能量写成粒子数的函数假定了自旋守恒。对动量做求和化积分,就得到
\begin{equation}
    \frac{\var{E}}{V} = \underbrace{\sum_{\sigma} \int \frac{\dd[3]{\vb*{k}}}{(2\pi)^3} \epsilon^0_{\vb*{k}} \var{n_{\vb*{k} \sigma}}}_{\var{E_1} / V} + \underbrace{\frac{1}{2} \sum_{\sigma, \sigma'} \int \frac{\dd[3]{\vb*{k}}}{(2\pi)^3} \int \frac{\dd[3]{\vb*{k}'}}{(2\pi)^3} f_{\sigma \sigma' \vb*{k} \vb*{k}'} \var{n_{\vb*{k} \sigma}} \var{n_{\vb*{k}' \sigma'}}}_{\var{E_2} / V}
\end{equation}
我们保留到二阶项,一阶项给出准粒子的近独立理论,二阶项给出准粒子的相互作用。
这种相互作用并不会让粒子数发生涨落或是让单个准粒子的动量发生涨落,而只会对能级做修正,是所谓的“前向散射”。

重整化群可能给出更高阶项,而也许二阶项实际上可以忽略,而绕了一大圈之后发现$\abs{M}$很小,从而实际上在很大的区域内费米液体图像均适用。
为了表明我们保留到二阶项是正确的,我们给出一个数量级估计,说明一阶项和二阶项的量级是同阶的,而低于更高阶项。
我们知道由于相互作用的存在,总能量$E$肯定不是单粒子哈密顿量(比如说$k^2/2m$这种形式)的期望值简单加起来的结果,但是显然能量具有可加性,设想我们改变了准粒子数分布,这样应该有
\[
    \var{E} = \sum_{\vb*{k}, \sigma} \epsilon_{\vb*{k} \sigma} \var{n_{\vb*{k} \sigma}},
\]
其中$\epsilon_{\vb*{k}}$是在有限温度下的近平衡态激发一个准粒子的能量,它的一部分是单准粒子能量,一部分是其它准粒子给它的相互作用能之和。
因此,$\epsilon_{\vb*{k}}$会依赖准粒子数分布。由于我们只研究二体相互作用,我们有
\[
    \epsilon \sim \sum_{\vb*{k}'} \text{something} \cdot n_{\vb*{k}'},
\]
于是设
\[
    \var{\epsilon_{\vb*{k} \sigma}} = \frac{1}{V} \sum_{\vb*{k}', \sigma'} f_{\sigma \sigma' \vb*{k} \vb*{k}'} \var{n_{\vb*{k}' \sigma'}},
\]
记$\epsilon_{\vb*{k}}^0$为$n_{\vb*{k}}$一概为零的$\epsilon_{\vb*{k}}$,代入$\var{E}$中就得到\eqref{eq:fermi-liquid-energy};第二项的$1/2$因子是因为一对粒子会被计数两次,所以要除以$2$;由于我们假定准粒子分布相对于零温只有微小的偏离,$\epsilon_{\vb*{k}}$被取为零温的值。
虽然$\epsilon^0 \var{n}$看起来比$f\var{n} \var{n}$大,但需要注意到我们在巨正则系综中工作,则真的有意义的应该是$E-\mu N$(且由于是近平衡,应有$\var{E} = \mu \var{N}$),而
\[
    \sum_{\vb*{k}} (\epsilon^0_{\vb*{k}} - \mu) \var{n_{\vb*{k}}} \sim \var{n}^2,
\]
于是$\epsilon^0 \var{n}$项和$f\var{n} \var{n}$项的贡献是同阶的,都需要考虑。
更高阶相互作用则涉及$\var{n}$的更高阶项,于是不考虑。因此,\eqref{eq:fermi-liquid-energy}的确是成立的。

在已经知道了$E$的表达式之后(比如说微扰计算出了体系能量),可以用变分计算出各个物理量:
\begin{equation}
    \epsilon_{\vb*{k}} = \fdv{E}{n_{\vb*{k} \sigma}} , \quad f_{\sigma \sigma' \vb*{k} \vb*{k}'} = V \frac{\var[2]{E}}{\var{n_{\vb*{k} \sigma}}\var{n_{\vb*{k}' \sigma'}}}, \quad \mu = \pdv{E}{N}.
\end{equation}

\eqref{eq:fermi-liquid-energy}中的一阶项可以看成一个等效的单粒子能量。由于只讨论费米面附近的理论,我们让能量从费米面量起,即使用$\xi$代替$\epsilon$,$k=k_\text{F}$时$\xi^0_{\vb*{k}}$就是零,在假定费米面具有旋转对称性的情况下可以做展开
\[
    \xi^0_{\vb*{k}} = \frac{k_\text{F}}{m^*} (k - k_\text{F}).
\]
我们仿照自由粒子的能量
\[
    \xi_{\vb*{k}} = \frac{k^2}{2m} - \frac{k_\text{F}^2}{2m} \approx \frac{k_\text{F}}{m} (k - k_\text{F})
\]
得到了一个等效质量$m^*$。能够像上面这样做的前提是准粒子能谱要足够光滑,如果像声子那样,就没法定义任何等效质量。%
\footnote{应注意此处的等效质量和“激发有能隙,是有质量的”中的“质量”是不同的;前者并不代表有一个能隙,而只是$\epsilon_{\vb*{k}}$的$k^2$项的系数而已。}%
如果温度很高,以至于不能保证有趣的行为仅仅发生在费米面附近,那有效质量的概念也没有什么用;实际上此时费米液体的理论就没有什么用。
请注意\eqref{eq:fermi-liquid-energy}完全是形式上的:诸如晶格势能造成的单粒子能量修正已经被纳入了$\var{E_1}$中,而只要费米面保持旋转对称性,就可以引入等效质量的概念,没有旋转对称性也可以引入有效质量张量。
并且,在只有费米面附近才有明显的激发的情况下,可以不失一般性地设
\[
    \epsilon_{\vb*{k}}^0 = \frac{k^2}{2m^*},
\]
因为真正有意义的是$\epsilon_{\vb*{k}} - \mu$,只需要同时调节$\epsilon_{\vb*{k}}$和$\mu$就可以让准粒子的$\epsilon_{\vb*{k}}$取自由粒子的形式。
再次强调,调节$\epsilon_{\vb*{k}}$和$\mu$之类的操作只适用于费米面附近;因此对一个费米液体我们通常避免讨论费米球深处有什么——这些东西对费米面附近的行为并不重要。

对二阶项,假定系统具有自旋旋转不变性,则$f$的取值完全由$f_{\uparrow \uparrow \vb*{k} \vb*{k}'}$和$f_{\uparrow \downarrow \vb*{k} \vb*{k}'}$决定。
实际上,由于$\vb*{k}$局限在费米面附近,我们有
\[
    f_{\alpha \beta \vb*{k} \vb*{k}'} = f_{\alpha \beta}(\vartheta),
\]
$\vartheta$是$\vb*{k}$和$\vb*{k}'$的夹角。这样,设
\begin{equation}
    \begin{aligned}
        f_{\uparrow \uparrow}(\vartheta) &= f^\text{S}(\vartheta) + f^\text{A}(\vartheta), \\
        f_{\uparrow \downarrow}(\vartheta) &= f^\text{S}(\vartheta) - f^\text{A}(\vartheta),
    \end{aligned}
\end{equation}
或者等价地可以设
\begin{equation}
    {f}(\vartheta) = f^\text{S}(\vartheta) + {\sigma} {\sigma}' f^\text{A}(\vartheta)
\end{equation}
从而将自旋守恒这一事实一并表示出来(${\sigma}^z$就是$z$方向的泡利矩阵),并将$f^\text{S}(\vartheta)$和$f^\text{A}(\vartheta)$展开成无量纲常数:
\begin{equation}
    \frac{k_\text{F} m^*}{\pi^2} f^\text{S,A}(\vartheta) = \sum_{l=0}^\infty F_l^\text{S,A} \legpoly (\cos \vartheta).
\end{equation}
于是,给定参数$m^*$,$k_\text{F}$以及$\{F_l^\text{S,A}\}$,费米液体服从的物理规律就给定了。
在这里,我们实际上又把准粒子当成了可以彼此散射、有相互作用的粒子,“准粒子动能”$k^2/2m^*$和“准粒子势能”$f_{\alpha \beta}(\vartheta)$是“单个准粒子能量”$\epsilon_{\vb*{k}}$的两部分;单单一个$k^2/2m^*$肯定和$\epsilon_{\vb*{k}}$是不一样的。

\subsubsection{费米液体的参数和裸电子的参数的对应关系}

在关闭相互作用时费米液体“无缝地”退化到实际的费米气体上,因此在无相互作用点处费米液体中准粒子的动量就是实际的能带电子的动量。
现在缓慢地加上相互作用,则费米液体准粒子的动量可以发生连续变化,但有限体系中动量实际上是分立的,从而动量只能不变。
因此,准粒子的动量实际上就是电子的动量。

需要注意的是动量守恒和粒子数没有关系,设$\ket{\Psi}$是费米液体的基态而$c^\dagger_{\vb*{k}}$是Bloch电子的产生算符,则$c^\dagger_{\vb*{k}} \ket{\Psi}$\emph{不是}$n_{\vb*{k}}$精确地加一的态,而是多出来一个电子位于$\vb*{k}$上、多出来两个电子位于$\vb*{k}$上和一个空穴位于$-\vb*{k}$上、多出来三个电子位于$\vb*{k}$上和两个空穴位于$-\vb*{k}$上这一系列态的叠加。
费米液体中的准粒子是能带电子重整化得到的,这个过程可以线性组合$c_{\vb*{k}}$和$c^\dagger_{\vb*{k}}$。

如果一个费米液体系统可以确定是一个实际的费米气体加入相互作用的结果,并且如前所述,能够保证准粒子个数和实际电子的个数一样,自旋相同,等等,并且保证自旋旋转不变性、空间平移不变性、空间各向同性,那么费米液体中准粒子的等效质量和实际电子的质量有一个简单的,使用$\{F_l^\text{S,A}\}$写出的关系。
由于
\[
    E - E_0 = \sum_{\vb*{k}} \var{n_{\vb*{k}}} \epsilon_{\vb*{k}},
\]
由动量为$\vb*{k}$的一个准粒子的运动速度为
\[
    \vb*{v}_i = \pdv{E}{\vb*{p}_i} = \pdv{\epsilon_{\vb*{k}}}{\vb*{k}},
\]
上式的量子版本就是
\[
    {\vb*{v}} = \pdv{{\epsilon}_{\vb*{k}}}{\vb*{k}}.
\]
我们于是可以将${\vb*{v}}$当成费米液体中准粒子的流速。设$\rho$是某种守恒荷的密度,则任意一个流算符的期望可以写成
\[
    \begin{aligned}
        \expval*{\rho \vb*{v}} = V \int \frac{\dd[3]{\vb*{k}}}{(2\pi)^3} \expval{\rho \pdv{{\epsilon}}{\vb*{k}}} , 
    \end{aligned}
\]
准粒子和实际的费米子数量相同,准粒子的粒子数流密度算符就是实际费米子的粒子数流密度算符,且由于动量守恒,一个准粒子的总动量就是与它对应的实际费米子的总动量,而实际费米子的总动量就是总质量流(因为$\vb*{p}=m\vb*{v}$);由于动量和自旋守恒,我们将费米子占据数算符用$\vb*{k}$和$\sigma$标记,于是我们有
\[
    \sum_{\vb*{k}, \sigma} \int \frac{\dd[3]{\vb*{k}}}{(2\pi)^3} \vb*{k} {n}_{\vb*{k} \sigma} = \trace \int \frac{\dd[3]{\vb*{k}}}{(2\pi)^3} m \pdv{{\epsilon}}{\vb*{k}} {n}.
\]
同样${\epsilon}$也适用一样的推导,于是就有
\[
    \int \frac{\dd[3]{\vb*{k}}}{(2\pi)^3} m \pdv{\epsilon_{\vb*{k} \sigma}}{\vb*{k}} n_{\vb*{k} \sigma} = \int \frac{\dd[3]{\vb*{k}}}{(2\pi)^3} \vb*{k} n_{\vb*{k} \sigma},
\]
对上式求变分,就有
\[
    \begin{aligned}
        \int \frac{\dd[3]{\vb*{k}}}{(2\pi)^3} \vb*{k} \var{n_{\vb*{k} \sigma}} &= \var \int \frac{\dd[3]{\vb*{k}}}{(2\pi)^3} m \pdv{\epsilon_{\vb*{k} \sigma}}{\vb*{k}} n_{\vb*{k} \sigma} \\
        &= m \int \frac{\dd[3]{\vb*{k}}}{(2\pi)^3} \pdv{\epsilon_{\vb*{k} \sigma}}{\vb*{k}} \var{n_{\vb*{k} \sigma}} + m \int \frac{\dd[3]{\vb*{k}}}{(2\pi)^3} \int \frac{\dd[3]{\vb*{k}'}}{(2\pi)^3} n_{\vb*{k} \sigma} \var{n_{\vb*{k}' \sigma}} \pdv{f_{\sigma}(\vartheta)}{\vb*{p}} \\
        &= m \int \frac{\dd[3]{\vb*{k}}}{(2\pi)^3} \pdv{\epsilon_{\vb*{k} \sigma}}{\vb*{k}} \var{n_{\vb*{k} \sigma}} - m \int \frac{\dd[3]{\vb*{k}}}{(2\pi)^3} \int \frac{\dd[3]{\vb*{k}'}}{(2\pi)^3} \var{n_{\vb*{k} \sigma}} \pdv{n_{\vb*{k}' \sigma}}{\vb*{k}'} f_{\sigma}(\vartheta) .
    \end{aligned}
\]
第三个等号交换了$\vb*{k}$和$\vb*{k}'$,但这是合理的,因为$f$只和这两者的夹角有关。
考虑到$\var{n}$的任意性,就有
\[
    \frac{\vb*{k}}{m} = \pdv{\epsilon_{\vb*{k} \sigma}}{\vb*{k}} - \int \frac{\dd[3]{\vb*{k}'}}{(2\pi)^3} \pdv{n_{\vb*{k}' \sigma}}{\vb*{k}'} f_{\sigma}(\vartheta).
\]
在上式两边点乘$\vb*{k}$,代入$n$是阶跃函数这一事实,并且注意到动量几乎总是在费米面上,从而$\vb*{k} = k_\text{F} \vb*{n}$,就得到
\begin{equation}
    \frac{1}{m} = \frac{1}{m^*} + \frac{k_\text{F}}{(2\pi)^3} \int \dd{\Omega} \cos \vartheta f_\sigma(\vartheta).
    \label{eq:m-and-m-star}
\end{equation}
上式的形式其实有些容易让人误解,毕竟,$f$和$m^*$都是加入相互作用之后重整化得到的有效参数,而上式看起来似乎是“$f$和$m$决定了$m^*$”。
这里还有一个疑难,就是在使用电子动量计算总动量时我们直接用了$p=mv$,而以费米液体的观点计算总动量时我们却没有这么做。
这是因为,只是根据\eqref{eq:fermi-liquid-energy}计算出的动量并不是真正的总动量,庞大的费米海的动量被忽略了:当一个费米液体中的准粒子被激发起来时,实际的系统中的费米海会受到扰动,从而会有额外的动量。
$f$通常和费米能级有关,因此可以提供一些关于“费米海有多重”的信息,这就是\eqref{eq:m-and-m-star}中会出现$f$的原因:\eqref{eq:m-and-m-star}来自动量守恒关系,动量守恒方程的一边含有$m$,另一边含有$m^*$和通过$f$反解出的费米海的动量,经过化简就得到\eqref{eq:m-and-m-star}。
当然,如果实际系统中根本没有电子间排斥,那么$f$肯定一直是零。

\subsubsection{费米液体的平衡态宏观性质}

使用以上参数:$m^*$,$k_\text{F}$以及$\{F_l^\text{S,A}\}$,可以计算费米液体的各种宏观性质。

首先考虑零温附近的比热。费米气体的比热在低温极限下正比于温度,费米液体实际上也一样。
能量由\eqref{eq:fermi-liquid-energy}给出,随着$T$增大,一些粒子从费米海溢出,从而能量增大,产生一个热容。
实际上,在零温极限附近,\eqref{eq:fermi-liquid-energy}中的$E_2$部分没有贡献。
这是因为
\[
    E_2 = \sum_{\sigma, \sigma'} \underbrace{\frac{1}{2V} \sum_{\vb*{k}} f_{\sigma \sigma'}(\theta) \var{n}_{\vb*{k} \sigma}}_{\text{constant}} \var{n}_{\vb*{k}' \sigma'},
\]
被大括号括起来的部分和$\vb*{k}$无关,而显然
\[
    \sum_{\vb*{k}} \var{n}_{\vb*{k} \sigma} = 0,
\]
因此$E_2$对总能量没有贡献。这样费米液体的热容和费米气体的热容就是完全一致的,为
\begin{equation}
    C_V = \frac{1}{3} m^* k_\text{F} T.
\end{equation}
这个公式在实验上非常重要,如果确定一个体系是费米液体(如发现低温下热容正比于温度),那么就可以据此测出粒子的有效质量。

也可以计算费米液体的磁化率。考虑弱场近似,则磁化率
\[
    \chi = \pdv{M}{H}
\]
近似为
\[
    \chi = \frac{M}{H},
\]
其中$M$表示磁化强度,$H$表示磁场强度(不是哈密顿量),而磁化强度为
\[
    M = \pdv{E}{H},
\]
于是得到
\[
    \frac{1}{\chi} = \pdv[2]{E}{M}.
\]
这样只需要使用$M$表示出$E$就可以了。
记自旋向上(以磁场方向为$z$轴)和向下的粒子数为$N_\uparrow$和$N_\downarrow$,则
\[
    M = \mu_\text{B} (N_\uparrow - N_\downarrow),
\]
其中$\mu_\text{B}$为玻尔磁子。磁场导致自旋向上和向下的粒子数发生变化的原因是,自旋和磁场一致的粒子的费米面会扩大,自旋和磁场相反的粒子的费米面会缩小,从而让$N_\uparrow$变大,$N_\downarrow$缩小。
由于粒子数不变,有
\[
    \var{N_\uparrow} = - \var{N_\downarrow},
\]
而没有磁场时向上和向下的粒子数一样,于是
\[
    M = 2 \mu_\text{B} \var{N}_\uparrow.
\]
$\var{N_\uparrow}$和费米动量的变化之间的关系是
\[
    \var{N_\uparrow} = \int_{k_\text{F} < k < k_\text{F} + \var{k_\text{F}}} \frac{V}{(2\pi)^3} \dd[3]{\vb*{k}} = \frac{V k_\text{F}^2 \var{k_\text{F}}}{2\pi^2}.
\]
现在可以将$M$用$\var{k_\text{F}}$表示出来了。接下来将能量写成$\var{k_\text{F}}$的函数。
对动能部分$E_1$,我们有
\[
    \var{E_1} = \sum_{\sigma, \vb*{k}} \frac{k_\text{F}}{m^*} (k - k_\text{F}) \var{n}_{\vb*{k} \sigma},
\]
$n_{\vb*{k} \uparrow}$仅有的变化是在$k_\text{F} < k < k_\text{F} + \var{k_\text{F}}$的区域内从$0$变成$1$,$n_{\vb*{k} \downarrow}$仅有的变化是在$k_\text{F} - \var{k_\text{F}} < k < k_\text{F}$的区域内从$1$变成$0$。
这样就有
\[
    \begin{aligned}
        \var{E_1} &= \int_{k_\text{F} < k < k_\text{F} + \var{k_\text{F}}} \frac{V}{(2\pi)^3} \dd[3]{\vb*{k}} \frac{k_\text{F}}{m^*} (k - k_\text{F}) + \int_{k_\text{F} - \var{k_\text{F}} < k < k_\text{F}} \frac{V}{(2\pi)^3} \dd[3]{\vb*{k}} \frac{k_\text{F}}{m^*} (k - k_\text{F}) (-1) \\
        &= \frac{V k_\text{F}^3}{2 \pi^2 m^*} (\var{k_\text{F}})^2.
    \end{aligned}
\]
最后,得到$\var{E_1}$和$M$的关系:
\[
    \var{E_1} = \frac{\pi^2}{2 m^* \mu_\text{B}^2 V k_\text{F}} M^2.
\]
同理,可以计算得到(计算的关键点在于意识到对全空间计算积分,则只有零阶勒让德多项式能够给出非零结果)
\[
    \var{E_2} = \frac{\pi^2}{2 m^* \mu_\text{B}^2 V k_\text{F}} F_0^\text{A} M^2.
\]
这样就得到了$\var{E}$关于$M$的表达式,从而最终得到
\begin{equation}
    \chi = \frac{1}{1 + F_0^\text{A}} \frac{m^* \mu_\text{B}^2 V k_\text{F}}{\pi^2}.
\end{equation}

\subsubsection{等效单粒子图像和动理学}

现在我们开始分析费米液体的动态响应。应当注意\eqref{eq:fermi-liquid-energy}在稍微远离费米面的地方就失效了,因为此时真的能够改变准粒子动量的碰撞是需要考虑的。
在碰撞有物理效应时,不能使用\eqref{eq:fermi-liquid-energy}作为哈密顿量而计算格林函数。

一种替代的方式是使用玻尔兹曼方程。

\subsubsection{费米液体理论的合理性}

确定一个系统是否适用费米液体理论首先需要我们确定系统的低能激发是否真的与电子的行为类似,因为原则上完全可以出现诸如电子配对等现象。
在确定系统的低能激发和电子行为类似之后,费米球的存在性,以及低能激发主要是费米面附近的准粒子和空穴就可以确定。
此时需要讨论的主要是相互作用的类型,即是否前向散射占据主要地位。

在重整化群计算中,一个二体散射项的强弱主要由这个散射项中允许的的粒子动量的取值多少确定,可能的动量占据的空间体积越大,对应的相互作用通道就越强。

懒得写了,总之就是BCS相互作用和密度-密度相互作用是特别强的两种。
% TODO:那Hubbard模型?

费米液体的思想是传统凝聚态物理的基础:在费米液体提出之后的很长一段时间,分析凝聚态系统的标准方式是将系统的基本激发当成某种经过重整化的电子(如费米液体理论中的准粒子),写出形式各异的无相互作用的能带,然后加入占主导地位的相互作用通道。
一些新兴的系统,如Luttinger液体,则无法归入这个框架:一点相互作用就足以破坏能带,从而让系统的低能自由度看起来完全不像电子。

\subsection{费米面的不稳定性}

一些与费米面的不稳定性有关的情况见\autoref{sec:low-and-super}。