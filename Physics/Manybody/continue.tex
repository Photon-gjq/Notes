\documentclass[hyperref, UTF8, a4paper]{ctexart}

\usepackage{geometry}
\usepackage{float}
\usepackage{titling}
\usepackage{titlesec}
\usepackage{paralist}
\usepackage{footnote}
\usepackage{enumerate}
\usepackage{amsmath, amssymb, amsthm}
\usepackage{bbm}
\usepackage{cite}
\usepackage{graphicx}
\usepackage{subfigure}
\usepackage{physics}
\usepackage{autobreak}
\usepackage{tikz}
\usepackage{tikz-feynhand}
\usepackage[colorlinks, linkcolor=black, anchorcolor=black, citecolor=black]{hyperref}
\usepackage{prettyref}

\geometry{left=3.18cm,right=3.18cm,top=2.54cm,bottom=2.54cm}
\titlespacing{\paragraph}{0pt}{1pt}{10pt}[20pt]
\setlength{\droptitle}{-5em}
\preauthor{\vspace{-10pt}\begin{center}}
\postauthor{\par\end{center}}

\DeclareMathOperator{\timeorder}{T}
\DeclareMathOperator{\diag}{diag}
\DeclareMathOperator{\legpoly}{P}
\DeclareMathOperator{\primevalue}{P}
\DeclareMathOperator{\sgn}{sgn}
\newcommand*{\ii}{\mathrm{i}}
\newcommand*{\ee}{\mathrm{e}}
\newcommand*{\const}{\mathrm{const}}
\newcommand*{\comment}{\paragraph{注记}}
\newcommand*{\suchthat}{\quad \text{s.t.} \quad}
\newcommand*{\argmin}{\arg\min}
\newcommand*{\argmax}{\arg\max}
\newcommand*{\normalorder}[1]{: #1 :}
\newcommand*{\pair}[1]{\langle #1 \rangle}
\newcommand*{\fd}[1]{\mathop{}\!\mathcal{D} #1}
\DeclareMathOperator{\bigO}{\mathcal{O}}
\newcommand*{\cexpval}[1]{\langle \langle #1 \rangle \rangle}  

\newrefformat{sec}{第\ref{#1}节}
\newrefformat{note}{注\ref{#1}}
\newrefformat{fig}{图\ref{#1}}
\renewcommand{\autoref}{\prettyref}

\usetikzlibrary{arrows,shapes,positioning}
\usetikzlibrary{arrows.meta}
\usetikzlibrary{decorations.markings}
\tikzstyle arrowstyle=[scale=1]
\tikzstyle directed=[postaction={decorate,decoration={markings,
    mark=at position .5 with {\arrow[arrowstyle]{>}}}}]
\tikzstyle ray=[directed, thick]
\tikzstyle dot=[anchor=base,fill,circle,inner sep=1pt]

\renewcommand{\emph}[1]{\textbf{#1}}
\newcommand*{\concept}[1]{\underline{\textbf{#1}}}

\title{连续自旋模型和它们的场论}
\author{吴何友}

\begin{document}

\maketitle

\section{非线性$\sigma$模型}

\concept{非线性$\sigma$模型}包含$N$个场,用下标$a$标记它们,配分函数为
\begin{equation}
    Z = \int \prod_{a=1}^N \fd{\phi_a} \delta(\sum_a \phi_a^2(\vb*{r}) - 1) \exp(- \frac{\beta}{2} \sum_a \int \dd[d]{\vb*{r}} (\grad{\phi_a})^2 ).
\end{equation}
看起来似乎各个场是彼此独立的,但是约束条件意味着它们实际上有耦合。
我们引入拉格朗日乘子,将约束写成
\[
    \delta(\sum_a \phi_a^2(\vb*{r}) - 1) \propto \int_{-\infty}^\infty \dd{\lambda} \exp(- \frac{\ii}{2} \lambda (\phi^2(\vb*{r}) - 1 )),
\]
每一点的拉格朗日乘子均可不同,这样我们得到一个辅助场:
\[
    Z = \int \prod_{a=1}^N \fd{\phi_a} \int \fd{\lambda} \exp(- \frac{1}{2} \int \dd[d]{\vb*{r}} \left( \beta (\grad{\phi})^2 + \ii \lambda (\phi^2 - 1) \right) ).
\]
我们对$\phi$做一个尺度变换并适当调整常数,就有
\[
    Z = \int \prod_{a=1}^N \fd{\phi_a} \int \fd{\lambda} \exp(- \frac{1}{2} \int \dd[d]{\vb*{r}} \left( \beta (\grad{\phi})^2 + \ii \lambda (\phi^2 - N) \right) ).
\]
由于不同的$\phi_a$之间并没有直接的相互作用,我们可以将上式写成
\begin{equation}
    Z = \int \fd{\phi} \int \fd{\lambda} \exp(- \frac{N}{2} \int \dd[d]{\vb*{r}} \left( \beta (\grad{\phi})^2 + \ii \lambda (\phi^2 - 1) \right) ).
    \label{eq:nonlinear-sigma-lambda}
\end{equation}

\subsection{大$N$展开}

\eqref{eq:nonlinear-sigma-lambda}的求解是非常不容易的,我们此处使用大$N$展开来处理它,即将$N$看成一个很大的量,即使它并不那么大。

\end{document}