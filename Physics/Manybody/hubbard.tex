\documentclass[hyperref, UTF8, a4paper]{ctexart}

\usepackage{geometry}
\usepackage{titling}
\usepackage{titlesec}
\usepackage{paralist}
\usepackage{footnote}
\usepackage{enumerate}
\usepackage{amsmath, amssymb, amsthm}
\usepackage{bbm}
\usepackage{cite}
\usepackage{graphicx}
\usepackage{subfigure}
\usepackage{physics}
\usepackage{tikz}
\usepackage{autobreak}
\usepackage[ruled, vlined, linesnumbered, noend]{algorithm2e}
\usepackage[colorlinks, linkcolor=black, anchorcolor=black, citecolor=black]{hyperref}
\usepackage{prettyref}

% Page style
\geometry{left=3.18cm,right=3.18cm,top=2.54cm,bottom=2.54cm}
\titlespacing{\paragraph}{0pt}{1pt}{10pt}[20pt]
\setlength{\droptitle}{-5em}
\preauthor{\vspace{-10pt}\begin{center}}
\postauthor{\par\end{center}}

% Math operators
\DeclareMathOperator{\timeorder}{T}
\DeclareMathOperator{\diag}{diag}
\DeclareMathOperator{\legpoly}{P}
\DeclareMathOperator{\primevalue}{P}
\DeclareMathOperator{\sgn}{sgn}
\newcommand*{\ii}{\mathrm{i}}
\newcommand*{\ee}{\mathrm{e}}
\newcommand*{\const}{\mathrm{const}}
\newcommand*{\comment}{\paragraph{注记}}
\newcommand*{\suchthat}{\quad \text{s.t.} \quad}
\newcommand*{\argmin}{\arg\min}
\newcommand*{\argmax}{\arg\max}
\newcommand*{\normalorder}[1]{: #1 :}
\newcommand*{\pair}[1]{\langle #1 \rangle}
\newcommand*{\fd}[1]{\mathcal{D} #1}
\DeclareMathOperator{\bigO}{\mathcal{O}}

% prettyref setting
\newrefformat{sec}{第\ref{#1}节}
\newrefformat{note}{注\ref{#1}}
\newrefformat{fig}{图\ref{#1}}
\newrefformat{alg}{算法\ref{#1}}
\renewcommand{\autoref}{\prettyref}

% TikZ setting
\usetikzlibrary{arrows,shapes,positioning}
\usetikzlibrary{arrows.meta}
\usetikzlibrary{decorations.markings}
\tikzstyle arrowstyle=[scale=1]
\tikzstyle directed=[postaction={decorate,decoration={markings,
    mark=at position .5 with {\arrow[arrowstyle]{stealth}}}}]
\tikzstyle ray=[directed, thick]
\tikzstyle dot=[anchor=base,fill,circle,inner sep=1pt]

% Algorithm setting
\renewcommand{\algorithmcfname}{算法}
% Python-style code
\SetKwIF{If}{ElseIf}{Else}{if}{:}{elif:}{else:}{}
\SetKwFor{For}{for}{:}{}
\SetKwFor{While}{while}{:}{}
\SetKwInput{KwData}{输入}
\SetKwInput{KwResult}{输出}
\SetArgSty{textnormal}

\renewcommand{\emph}[1]{\textbf{#1}}
\newcommand*{\concept}[1]{\underline{\textbf{#1}}}

\title{Hubbard模型}
\author{吴何友}

\begin{document}

\maketitle

\section{Hubbard模型的定义}

\concept{Hubbard模型}是一种常见的强关联电子模型,它是一个定义在点阵上的模型,以下我们照惯例用$i, j$等表示格点坐标。
不包含化学势的哈密顿量为
\begin{equation}
    \hat{H} = \underbrace{-t \sum_{\pair{i, j}, \sigma} \hat{c}_{i\sigma}^\dagger \hat{c}_{j\sigma} + \text{h.c.}}_{\hat{H}_0} + \underbrace{U \sum_i \hat{n}_{i \uparrow} \hat{n}_{i \downarrow}}_{\hat{H}_\text{I}}.
\end{equation}
或者,为了后面蒙特卡洛模拟的方便,重新定义化学势,也可以有
\begin{equation}
    \hat{H} = -t \sum_{\pair{i, j}, \sigma} \hat{c}_{i\sigma}^\dagger \hat{c}_{j\sigma} + \text{h.c.} 
    + U \sum_i \left(\hat{n}_{i\uparrow} - \frac{1}{2}\right) \left(\hat{n}_{i\downarrow} - \frac{1}{2}\right).
\end{equation}

\section{Hubbard模型的量子蒙特卡洛模拟}

\subsection{Trotter分解和辅助场引入}

下面我们尝试对Hubbard模型做Trotter分解。设虚时间间隔为$\Delta\tau$,总共有$m$个虚时间点,$\tau=m\Delta \tau$。
对Hubbard模型,有一种特殊的分解方法:
\begin{equation}
    \ee^{-\Delta \tau \hat{H}_\text{I}} = \gamma \sum_{s_1, s_2, \ldots, s_N = \pm 1} \ee^{\alpha \sum_i s_i (\hat{n}_{i\uparrow} - \hat{n}_{i \downarrow})}, 
    \quad \gamma = \frac{1}{2^N} \ee^{\Delta \tau U N / 4}, \quad \cosh(\alpha) = \ee^{\Delta \tau U / 2},
\end{equation}
可以看到$\gamma$是一个和辅助场$\{s_i\}$(照惯例我们下面记它的时间线为$\vb{s}$)无关的量,考虑到配分函数的常数因子无关紧要,略去此因子,则配分函数为
\[
    \begin{aligned}
        Z &= \trace \prod_{n=1}^m \sum_{\vb{s}_{n}} \ee^{\alpha \sum_i s_i (\hat{n}_{i\uparrow} - \hat{n}_{i \downarrow})} \ee^{\Delta \tau t \sum_{\pair{i, j}, \sigma} \hat{c}_{i\sigma}^\dagger \hat{c}_{j\sigma} + \text{h.c.}} \\
        &= \sum_{\vb{s}} \prod_{n=1}^m \ee^{\alpha \hat{c}^\dagger_{\uparrow} \diag{\vb{s}_n} \hat{c}_{\uparrow}} \ee^{- \alpha \hat{c}^\dagger_{\downarrow} \diag{\vb{s}_n} \hat{c}_{\downarrow}} \ee^{- \Delta \tau \hat{c}_\uparrow^\dagger \vb{T} \hat{c}_\uparrow} \ee^{- \Delta \tau \hat{c}_\downarrow^\dagger \vb{T} \hat{c}_\downarrow},
    \end{aligned}
\]
其中我们指定$\vb{T}$是动能部分$\hat{H}_0$在单粒子表象下的系数矩阵,即
\begin{equation}
    T_{ij} = \begin{cases}
        -t, \quad &\pair{i, j}, \\
        0, \quad &\text{otherwise}.
    \end{cases}
\end{equation}
应用公式
\begin{equation}
    \trace(\ee^{- \sum_{i, j} \hat{c}_i^\dagger A_{ij} \hat{c}_j} \ee^{- \sum_{i, j} \hat{c}_i^\dagger B_{ij} \hat{c}_j} \cdots) = \det(1 + \ee^{- \vb{A}}\ee^{- \vb{B}} \cdots),
    \label{eq:trace-to-det}
\end{equation}
我们积掉费米子自由度,得到
\[
    Z = \sum_{\vb{s}} \det(1 + \prod_{n=1}^m \exp(\alpha \diag{\vb{s}_n \oplus (-\vb{s}_n)}) \exp( -\Delta \tau \pmqty{\dmat{\vb{T}, \vb{T}}})).
\]
上式中出现了矩阵拼接,因为电子的量子数同时包括位置和自旋,因此需要$2N \times 2N$的矩阵(在$2N$维中,前$N$维对应自旋向上的态,后$N$维对应自旋向下的态)。
然而,Hubbard模型的自选旋转不变性意味着以上矩阵是分块对角的,从而可以拆分开来,得到下式:
\begin{equation}
    Z = \det(1 + \prod_{\sigma=\uparrow, \downarrow} \prod_{n=1}^m \vb{B}_{\vb{s}}^\sigma(\tau) ),
\end{equation}
其中
\begin{equation}
    \vb{B}^\uparrow_{\vb{s}}(\tau) = \ee^{\alpha \diag \vb{s}_n} \ee^{-\Delta \tau \vb{T}}, \quad \vb{B}^\downarrow_{\vb{s}}(\tau) = \ee^{- \alpha \diag \vb{s}_n} \ee^{-\Delta \tau \vb{T}}.
\end{equation}
所有$\vb{B}_{\sigma}$都是一个$N \times N$矩阵,而不是$2N \times 2N$的矩阵。

\end{document}
