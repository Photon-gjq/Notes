\documentclass[hyperref, UTF8, a4paper]{ctexart}

\usepackage{geometry}
\usepackage{titling}
\usepackage{titlesec}
\usepackage{paralist}
\usepackage{footnote}
\usepackage{enumerate}
\usepackage{amsmath, amssymb, amsthm}
\usepackage{bbm}
\usepackage{cite}
\usepackage{graphicx}
\usepackage{subfigure}
\usepackage{physics}
\usepackage{tikz}
\usepackage{autobreak}
\usepackage[ruled, vlined, linesnumbered, noend]{algorithm2e}
\usepackage[colorlinks, linkcolor=black, anchorcolor=black, citecolor=black]{hyperref}
\usepackage{prettyref}

% Page style
\geometry{left=3.18cm,right=3.18cm,top=2.54cm,bottom=2.54cm}
\titlespacing{\paragraph}{0pt}{1pt}{10pt}[20pt]
\setlength{\droptitle}{-5em}
\preauthor{\vspace{-10pt}\begin{center}}
\postauthor{\par\end{center}}

% Math operators
\DeclareMathOperator{\timeorder}{T}
\DeclareMathOperator{\diag}{diag}
\DeclareMathOperator{\legpoly}{P}
\DeclareMathOperator{\primevalue}{P}
\DeclareMathOperator{\sgn}{sgn}
\newcommand*{\ii}{\mathrm{i}}
\newcommand*{\ee}{\mathrm{e}}
\newcommand*{\const}{\mathrm{const}}
\newcommand*{\comment}{\paragraph{注记}}
\newcommand*{\suchthat}{\quad \text{s.t.} \quad}
\newcommand*{\argmin}{\arg\min}
\newcommand*{\argmax}{\arg\max}
\newcommand*{\normalorder}[1]{: #1 :}
\newcommand*{\pair}[1]{\langle #1 \rangle}
\newcommand*{\fd}[1]{\mathcal{D} #1}
\DeclareMathOperator{\bigO}{\mathcal{O}}

% prettyref setting
\newrefformat{sec}{第\ref{#1}节}
\newrefformat{note}{注\ref{#1}}
\newrefformat{fig}{图\ref{#1}}
\newrefformat{alg}{算法\ref{#1}}
\renewcommand{\autoref}{\prettyref}

% TikZ setting
\usetikzlibrary{arrows,shapes,positioning}
\usetikzlibrary{arrows.meta}
\usetikzlibrary{decorations.markings}
\tikzstyle arrowstyle=[scale=1]
\tikzstyle directed=[postaction={decorate,decoration={markings,
    mark=at position .5 with {\arrow[arrowstyle]{stealth}}}}]
\tikzstyle ray=[directed, thick]
\tikzstyle dot=[anchor=base,fill,circle,inner sep=1pt]

% Algorithm setting
\renewcommand{\algorithmcfname}{算法}
% Python-style code
\SetKwIF{If}{ElseIf}{Else}{if}{:}{elif:}{else:}{}
\SetKwFor{For}{for}{:}{}
\SetKwFor{While}{while}{:}{}
\SetKwInput{KwData}{输入}
\SetKwInput{KwResult}{输出}
\SetArgSty{textnormal}

\renewcommand{\emph}[1]{\textbf{#1}}
\newcommand*{\concept}[1]{\underline{\textbf{#1}}}
\newcommand*{\Ztwo}{$\mathbb{Z}_2$}

\title{凝聚态体系中的拓扑}
\author{吴何友}

\begin{document}

\maketitle

一些系统处在的底流形具有非平凡的拓扑性质。依照定义,在重整化群下由拓扑导致的项并不会发生跑动。%(?存疑)

\section{拓扑相变}

\subsection{KT相变}

\subsubsection{二维经典XY模型的反常相变}

考虑一个二维经典XY模型:
\begin{equation}
    H = - J \sum_{\pair{i, j}} \vb*{S}_i \cdot \vb*{S}_j,
\end{equation}
使用$\theta_i$表示每个格点的自旋指向,并且将自旋长度整合进$J$中,就有
\begin{equation}
    H = - J \sum_{\pair{i, j}} \cos(\theta_i - \theta_j).
\end{equation}
如果假定诸$\theta_i$在空间上变化不大,那么就有
\[
    H = - J \sum_{\pair{i, j}} \left( 1 - \frac{(\theta_i - \theta_j)^2}{2} \right).
\]
在高温下可以出现波长非常短的$\theta$涨落,因此此时相邻自旋彼此方向相反也是没有关系的,而低温下,只有长波涨落才能够保留下来,因此下面的操作都是在低温极限下得到的。
设晶格常数为$a$,格点总数为$N$,则将上式粗粒化之后得到
\begin{equation}
    H = E_0 + \frac{J}{2} \int \dd[2]{\vb*{r}} (\grad{\theta})^2, \quad E_0 = 2 J N.
    \label{eq:smooth-xy}
\end{equation}
\eqref{eq:smooth-xy}的鞍点近似满足二维平面上的拉普拉斯方程
\begin{equation}
    \laplacian{\theta} = 0.
\end{equation}

计算低温下的关联函数。我们有
\[
    \expval*{\vb*{S}(\vb*{r}) \cdot \vb*{S}(0)} = \expval*{\cos(\theta(0) - \theta(\vb*{r}))} = \Re \expval*{\ee^{\ii (\theta(0) - \theta(\vb*{r}))}} = \Re \exp(-\frac{1}{2} \expval*{(\theta(0) - \theta(\vb*{r}))^2}).
\]
二维下$\theta$的关联函数正比于$\ln r$,具体来说是
\[
    \expval*{\theta(\vb*{r} \theta(0))} = \frac{1}{\beta J} \int_{\frac{1}{L}}^{\frac{1}{a}} \frac{\dd[2]{\vb*{k}}}{(2\pi)^2} \frac{\ee^{\ii \vb*{k} \cdot \vb*{r}}}{k^2} 
    % = \frac{1}{2\pi \beta J} \ln(\frac{r}{a}),
\]
因此
\begin{equation}
    \expval*{\vb*{S}(\vb*{r}) \cdot \vb*{S}(0)} \sim \ee^{- \ln (r/a) / (2 \pi \beta J)} \sim \left(\frac{a}{r}\right)^{1/2\pi \beta J}.
\end{equation}
我们观察到低温下二维XY模型存在(准)长程序:$\theta$关联函数以对数方式衰减,而自旋关联函数则以幂律衰减。另一方面高温下二维XY模型中不存在任何长程关联。这暗示降低温度时存在一个相变。

看起来这里有一个矛盾。Mermin–Wagner定理说,二维及以下的场论不能出现连续对称性的对称性自发破缺,因为此时无质量的Goldstone玻色子的格林函数将无法良好定义。
二维经典XY模型是$O(2)$模型,具有连续的自旋旋转对称性,因此不应该出现对称性自发破缺,也无法定义序参量。
因此其相变——如果有的话——只能够来自一个不同的机制。

\subsubsection{拓扑激发}

由于$\theta$是一个角度,它具有多值性,或者,等价地说,$\theta$允许的值形成了一个环,因此每一点的$\theta$的取值范围组成了一个拓扑非平庸的空间。
整个场构型就是实空间到这个环的映射。

如果我们讨论一个闭合回路上的$\theta$,那么可能的场构型可以根据环的第一同伦群
\[
    \pi_1(S^1) = \mathbb{Z}
\]
分类,分类这些场构型的整数就是绕数。
从比较分析的角度看,如果场构型中存在路径,其上的$\vb*{S}_i$首尾连成一个环(也即,场构型中有一个涡旋而这个环包围着涡旋),那么如果我们要求保留$\theta$的连续性,就必须允许它具有多值性,而一个闭合回路上的$\theta$只能走过整数圈,这个整数就是绕数。

绕数实际上给出了闭合路径中涡旋的数目。涡旋是场构型中的奇点%
\footnote{
    这是物理中少数奇点可以直接出现在场构型中并且占据主导地位的情况。这没有违反我们观察做的“一切都可导”的假设,因为此处的场构型是一个离散系统连续化之后得到的,而奇点来自场量的取值空间(而非底流形)的内禀的拓扑结构而不是函数性质不良好。
    设$M$是底流形上的一个紧致子流形,考虑$M$上的场构型,如果$M$对应的场量取值空间的同伦群是非平凡的,并且$M$上的场构型对应一个非平凡的同伦群群元,那么当我们缩小$M$时$M$上的场构型始终对应一个非平凡的同伦群群元(缩小$M$等价于对$M$上的场构型做一个光滑的变形),那么当$M$缩成一个点(假定这是可以的)时我们发现这个点上的场值不能唯一确定,因此就出现了一个奇点。
    奇点出现意味着这里不再能够用连续的场作为自由度,晶格变得重要起来。
    因此,如果$M$上的场构型对应一个非平凡的同伦群群元,那么$M$之内\emph{一定}有奇点。同伦群的群元通常表示了这个奇点的“强度”,称为\concept{拓扑荷}。
}%
,设$C$是闭合路径,设$n$为$C$中的涡旋数目(逆时针涡旋(一般直接称为涡旋)数目减去顺时针涡旋即\concept{反涡旋}的数目),$C_i$为围绕着第$i$个涡旋的小围道,由于$\theta$是调和场显然有
\[
    \oint_{C \cup ( \cup_i C_i)} \dd{\vb*{r}} \cdot \grad{\theta} = 0,
\]
而按照涡旋的定义环绕涡旋计算$\grad{\theta}$的线积分就得到$2\pi$,于是
\begin{equation}
    \oint \dd{\vb*{r}} \cdot \grad{\theta} = 2 \pi n.
\end{equation}
因此我们发现一个围道$C$上$\theta$的绕数不是别的,就是$C$中的净涡旋数目。

如果要求$\theta$具有单值性,就必须做适当的割线:如果有偶数个涡旋则应当将它们两两配对并将一对涡旋之间的连线上的点割除,如果有奇数个涡旋则还应该作一条通向无穷远点的割线。
等价地看,我们认为这些割线上有等效的冲击载荷,使得$\laplacian{\theta}$在这些割线上为某个$\delta$函数而不为零。

设有一群总计$n$个的涡旋,它们周围有
\[
    2 \pi n \sim 2 \pi r \abs*{\grad{\theta}},
\]
从而
\[
    \abs*{\grad{\theta}} \sim \frac{n}{r}.
\]
因此涡旋实际上是相当非局域的激发,衰减是很慢的。
代入哈密顿量,有
\[
    E \sim E_0 + \frac{J}{2} 2 \pi \int r \dd{r} \left( \frac{n}{r} \right)^2 = E_0 + \pi J n^2 \int \dd{r} \ln r.
\]
能量同时出现了红外发散和紫外发散。考虑到坐标尺度的上下界分别被系统尺寸和晶格常数截断,有
\begin{equation}
    E \sim E_0 + \pi J n^2 \ln \frac{L}{a}.
\end{equation}
在一个比较大的系统中,即使单个涡旋也具有很大能量。

\section{自旋链中的拓扑项}

% TODO:瞬子;瞬子是时空上的场构型,因此无所谓“瞬子的时间演化”

% 参考文献:http://www.mit.edu/~levitov/8.334/notes/XYnotes1.pdf
% Phase Transitions and Collective Phenomena https://www.tcm.phy.cam.ac.uk/~bds10/phase.html

\end{document}