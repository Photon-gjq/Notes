\part{流体}

\chapter{牛顿流体的流体动力学}

对流体,我们有\concept{纳维-斯托克斯方程}
\begin{equation}
    \rho \left( \pdv{\vb*{v}}{t} + \vb*{v} \cdot \grad{\vb*{v}} \right) = - \grad{P} + \vb*{f}.
\end{equation}

\subsection{声波}

当速度的时间变化相比于空间输运非常大时,即
\[
    \pdv{t} \gg \vb*{v} \cdot \grad
\]
时,近似有
\begin{equation}
    \rho \pdv{\vb*{v}}{t} = - \grad{p},
    \label{eq:ns-eq-small-v}
\end{equation}
两边计算散度,并利用输运方程\eqref{eq:transportation-eq},得到
\[
    \laplacian{p} = \pdv[2]{\rho}{t},
\]
再假定压强变化不大,有
\[
    \rho = \eval{\pdv{\rho}{P}}_{P_0} (P - P_0) = \eval{\pdv{\rho}{P}}_{P_0} p,
\]
于是就得到波动方程
\begin{equation}
    \frac{1}{c^2} \pdv[2]{p}{t} = \laplacian{p},
    \label{eq:sound-wave-fluid}
\end{equation}
其中
\begin{equation}
    \frac{1}{c^2} = \eval{\pdv{\rho}{P}}_{P_0}.
\end{equation}
这就是说,快速振动而振幅不大的流体中会有线性机械波,这就是\concept{声波}。

声波一定是横波,因为\eqref{eq:ns-eq-small-v}两边同取旋度,就有
\[
    \pdv{t} \curl{\vb*{v}} = 0, 
\]
即$\curl{\vb*{v}}$不随时间变化。由于\eqref{eq:sound-wave-fluid}是线性的,我们可以只取其小幅振动的解,即从一个一般的解中剥离整体平移运动的部分。这种小幅振动的解的基底不妨取为平面波,而对每个平面波,都有$\curl{\vb*{v}}=0$,于是我们就得出结论:声波是无旋的。
实际上可以直接从压强的性质出发得到这个结论,因为横波要求剪力而压强不能提供剪力。