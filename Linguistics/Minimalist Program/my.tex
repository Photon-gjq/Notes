\documentclass[a4paper]{article}

\usepackage[utf8]{inputenc}
\usepackage{geometry}
\usepackage{caption}
\usepackage{subcaption}
\usepackage{abstract}
% \usepackage{paralist}
\usepackage{amsmath, amssymb}
\usepackage{tikz}
\usepackage{qtree}
\usepackage{gb4e}
\usepackage[colorlinks, urlcolor=cyan]{hyperref}
\usepackage{prettyref}

% TikZ setting
\usetikzlibrary{arrows,shapes,positioning}
\usetikzlibrary{arrows.meta}
\usetikzlibrary{decorations.markings}
\tikzstyle arrowstyle=[scale=1]
\tikzstyle directed=[postaction={decorate,decoration={markings,
    mark=at position .5 with {\arrow[arrowstyle]{stealth}}}}]
\tikzstyle ray=[directed, thick]
\tikzstyle dot=[anchor=base,fill,circle,inner sep=1pt]

\newcommand*{\concept}[1]{{\textbf{#1}}}

\title{The Most Generalized Framework of Minimalist Syntax}
\author{Jinyuan Wu}

\begin{document}

\maketitle

This article is about a most generalized framework of Minimalist syntax. We know that Minimalist syntax 
is a highly fragmented field, where almost every linguists has his/her own analysis
framework, and the boring battle around which framework is the best continues constantly.
In this article I want to give a most generalized framework of Minimalist syntax. All other
frameworks can be seen as a constrained version of the framework in this article.

\section{The generalized framework}

The basic structure of the syntax is below.
\begin{exe}
    \ex\label{ex:structure} Structure of syntax 
    

\tikzset{every picture/.style={line width=0.75pt}} %set default line width to 0.75pt        

\begin{tikzpicture}[x=0.75pt,y=0.75pt,yscale=-1,xscale=1]
%uncomment if require: \path (0,335); %set diagram left start at 0, and has height of 335

%Straight Lines [id:da8133557183786788] 
\draw    (111,174) -- (597,174) ;
\draw [shift={(599,174)}, rotate = 180] [fill={rgb, 255:red, 0; green, 0; blue, 0 }  ][line width=0.08]  [draw opacity=0] (12,-3) -- (0,0) -- (12,3) -- cycle    ;
%Straight Lines [id:da9139623062429578] 
\draw    (180,95) -- (162.44,173.05) ;
\draw [shift={(162,175)}, rotate = 282.68] [fill={rgb, 255:red, 0; green, 0; blue, 0 }  ][line width=0.08]  [draw opacity=0] (12,-3) -- (0,0) -- (12,3) -- cycle    ;
%Straight Lines [id:da03470199826802922] 
\draw    (192,94) -- (216.4,172.09) ;
\draw [shift={(217,174)}, rotate = 252.65] [fill={rgb, 255:red, 0; green, 0; blue, 0 }  ][line width=0.08]  [draw opacity=0] (12,-3) -- (0,0) -- (12,3) -- cycle    ;
%Straight Lines [id:da44105963725312125] 
\draw    (282,92) -- (282.98,172) ;
\draw [shift={(283,174)}, rotate = 269.3] [fill={rgb, 255:red, 0; green, 0; blue, 0 }  ][line width=0.08]  [draw opacity=0] (12,-3) -- (0,0) -- (12,3) -- cycle    ;
%Straight Lines [id:da049869658230754954] 
\draw    (211,95) -- (368.2,172.12) ;
\draw [shift={(370,173)}, rotate = 206.13] [fill={rgb, 255:red, 0; green, 0; blue, 0 }  ][line width=0.08]  [draw opacity=0] (12,-3) -- (0,0) -- (12,3) -- cycle    ;
%Straight Lines [id:da31651729521748106] 
\draw    (301,91) -- (448.25,172.04) ;
\draw [shift={(450,173)}, rotate = 208.83] [fill={rgb, 255:red, 0; green, 0; blue, 0 }  ][line width=0.08]  [draw opacity=0] (12,-3) -- (0,0) -- (12,3) -- cycle    ;
%Straight Lines [id:da6422754162098989] 
\draw    (205.13,210.65) -- (266.87,300.35) ;
\draw [shift={(268,302)}, rotate = 235.47] [fill={rgb, 255:red, 0; green, 0; blue, 0 }  ][line width=0.08]  [draw opacity=0] (12,-3) -- (0,0) -- (12,3) -- cycle    ;
\draw [shift={(204,209)}, rotate = 55.47] [fill={rgb, 255:red, 0; green, 0; blue, 0 }  ][line width=0.08]  [draw opacity=0] (12,-3) -- (0,0) -- (12,3) -- cycle    ;
%Straight Lines [id:da8528940161688809] 
\draw    (342.17,202.82) -- (298.83,297.18) ;
\draw [shift={(298,299)}, rotate = 294.66] [fill={rgb, 255:red, 0; green, 0; blue, 0 }  ][line width=0.08]  [draw opacity=0] (12,-3) -- (0,0) -- (12,3) -- cycle    ;
\draw [shift={(343,201)}, rotate = 114.66] [fill={rgb, 255:red, 0; green, 0; blue, 0 }  ][line width=0.08]  [draw opacity=0] (12,-3) -- (0,0) -- (12,3) -- cycle    ;

% Text Node
\draw (192,59) node   [align=left] {abstract\\lexicon\\(list A)};
% Text Node
\draw (138,177) node [anchor=north west][inner sep=0.75pt]   [align=left] {Merge};
% Text Node
\draw (204,177) node [anchor=north west][inner sep=0.75pt]   [align=left] {Merge};
% Text Node
\draw (258.5,177) node [anchor=north west][inner sep=0.75pt]   [align=left] {Spellout};
% Text Node
\draw (288,66) node   [align=left] {phonetic\\forms (list B)};
% Text Node
\draw (349,177) node [anchor=north west][inner sep=0.75pt]   [align=left] {Merge};
% Text Node
\draw (420.5,177) node [anchor=north west][inner sep=0.75pt]   [align=left] {Spellout};
% Text Node
\draw (240,307) node [anchor=north west][inner sep=0.75pt]   [align=left] {meaning (list C)};


\end{tikzpicture}

\end{exe}
The list A, list B and list C in \eqref{ex:structure} are terms commonly used in Distributed Morphology.
The Merge operation is simply gluing two elements into an endocentric tree:
\begin{exe}
    \ex\label{ex:merge} \concept{Merge} 
    

\tikzset{every picture/.style={line width=0.75pt}} %set default line width to 0.75pt        

\begin{tikzpicture}[x=0.75pt,y=0.75pt,yscale=-1,xscale=1]
%uncomment if require: \path (0,300); %set diagram left start at 0, and has height of 300

%Straight Lines [id:da17269891356456113] 
\draw    (308,151) -- (366,127) ;
%Straight Lines [id:da3274513862294177] 
\draw    (424,151) -- (366,127) ;

% Text Node
\draw (366,124) node [anchor=south] [inner sep=0.75pt]   [align=left] {$<$};
% Text Node
\draw (244.91,142) node [anchor=east] [inner sep=0.75pt]    {$\alpha \ \ +\ \ \beta $};
% Text Node
\draw (249,138.4) node [anchor=north west][inner sep=0.75pt]    {$\longrightarrow $};
% Text Node
\draw (308,154.4) node [anchor=north] [inner sep=0.75pt]    {$\alpha $};
% Text Node
\draw (424,154.4) node [anchor=north] [inner sep=0.75pt]    {$\beta $};


\end{tikzpicture}

\end{exe}
The Spellout operation is to give a % TODO

Lexicon terms, or \concept{morphemes}, may be \concept{functional categories} or \concept{roots}. 
It should be noted that there is no clear distinction between these two terms, as the latter 
can be bleached \emph{gradually} into the former. \concept{Syntactic derivation}, i.e. the process in \eqref{ex:structure}, denotes the process of taking morphemes out of the lexicon (called \concept{numeration}), merging them into words and constituents, and transferring them into phonetic forms. 
In each stage of derivation, there is one or more syntactic object on building.
The set of all syntactic objects completely describe the state of the current state, so we call the set \concept{workspace}.
In each stage of derivation, there may be one or more syntactic object on building.

What lexicon terms enter the derivation is solely determined by the interaction between syntax and meaning. 
No step in \eqref{ex:structure} is guaranteed to pass. If one step is impossible, then the derivation 
\emph{crashes}. It may also be possible to assign a probability to each step. Note that the syntactic derivation
may \emph{not} be a Markov process, because interaction with meaning introduces some memory effects.
Nor is it local to a small area of the syntactic trees.

It can be seen that the \emph{narrow} syntax is very, very small. There is no explicit syntax rules, because they
emerge from the interplay of functional categories and roots. There is also no explicit constraints.
All constraints are imposed by the interface between syntax and other (much bigger) components.

We can use ``renormalization of syntactic theories'' to arrive other frameworks from our truly minimal framework.
A syntactic tree containing some hidden functional categories can be seen as a feature bundle.
Composite operations, like Move, work just like fundamental operations. 
In the language of physics, we are dealing with effective interactions and dressed states. 
We begin to work on these effective theories from now on.

\section{Notes about traditional fundamental Minimalist operations}

\paragraph{Selection} The selection mechanism - a constituent selects another - is not explicitly required,
because if $[_< \ \alpha \ \ \beta]$ cannot be spelt out, then effectively, we can say $\alpha$ \emph{selects}
$\beta$. 

\paragraph{Agree} We can mimic orthodox Agree by selection. For example, instead of saying ``the verb assigns 
an accusative case to the object'', we can say ``the verb selects an object with an accusative case'', which 
in turns means ``a VP where the object is in accusative case does not spell out''.

Note that this means actually a feature in a maximal projection can spread into the whole maximal projection, 
since Agree actually happens in the spelling out step, which will not happen before the maximal projection is 
completed. Therefore, Agree may not be single-probe-single-goal. However, since spellout and Merge happens 
cyclicly, usually there will not be many unvalued features in an Agree step, so we still have an 
effective single-probe-single-goal Agree.

\begin{exe}
    \ex Equivalence between a single-probe-single-goal Agree, feature bundle model and a multiple-goal Agree, distributed morphology (therefore multiple functional categories) model. % TODO:
\end{exe}

\paragraph{Movement} 

\paragraph{The copy theory and the multi-domain theory of movement}

\section{The shape of a syntactic tree}

\paragraph{Optional arguments and adjuncts} 

\paragraph{The lexical hypothesis, or morphosyntax} span-spellout (e.g. spellout of C+T+v) = a theory of feature bundles

\paragraph{Where are the features?} 

\section{Notes about morphism}

\section{The structure of syntax: Y-structure, the lexicon, parameters, and more}

\end{document}
