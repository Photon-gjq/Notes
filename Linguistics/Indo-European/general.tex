\documentclass[a4paper]{article}

\usepackage{geometry}
\usepackage{caption}
\usepackage{subcaption}
\usepackage{abstract}
% \usepackage{paralist}
\usepackage{amsmath, amssymb}
\usepackage{qtree}
\usepackage{gb4e}
\usepackage[colorlinks, urlcolor=cyan]{hyperref}
\usepackage{prettyref}

\geometry{left=3.18cm,right=3.18cm,top=2.54cm,bottom=2.54cm}

\newcommand{\cswiki}{\href{https://en.wikipedia.org/wiki/Centum_and_satem_languages}{Wikipedia}}
%\newcommand{\peopledoc}{\href{../../Anthropology/evolution/人类起源的故事.md}{this note}}

\title{Indo-European Languages}
\author{Jinyuan Wu}

\begin{document}

\maketitle

\section{Classification}

\subsection{Centum and satem languages}

From \cswiki:
\begin{quote}
    Languages of the Indo-European family are classified as either centum languages or satem languages according to how the dorsal consonants (sounds of "K" and "G" type) of the reconstructed Proto-Indo-European language (PIE) developed. An example of the different developments is provided by the words for "hundred" found in the early attested Indo-European languages (which is where the two branches get their names). In centum languages, they typically began with a /k/ sound (Latin centum was pronounced with initial /k/), but in satem languages, they often began with /s/ (the example satem comes from the Avestan language of Zoroastrian scripture).
\end{quote}

\section{Syntax}

\subsection{Overview of approaches}

\cite{lehmann1974proto}

\bibliographystyle{plain}
\bibliography{indo-european}

\end{document}