\documentclass[hyperref, a4paper]{article}

\usepackage{geometry}
\usepackage{titling}
\usepackage{titlesec}
% No longer needed, since we will use enumitem package
% \usepackage{paralist}
\usepackage{enumitem}
\usepackage{footnote}
% Conflicts with enumitem
%\usepackage{enumerate}
\usepackage{amsmath, amssymb, amsthm}
\usepackage{mathtools}
\usepackage{bbm}
\usepackage{cite}
\usepackage{graphicx}
\usepackage{subfigure}
\usepackage{physics}
\usepackage{tensor}
\usepackage{siunitx}
\usepackage[version=4]{mhchem}
\usepackage{tikz}
\usepackage{xcolor}
\usepackage{listings}
\usepackage{autobreak}
\usepackage[ruled, vlined, linesnumbered]{algorithm2e}
\usepackage{nameref,zref-xr}
\zxrsetup{toltxlabel}
\zexternaldocument*[optics-]{../optics/optics}[optics.pdf]
\zexternaldocument*[solid-]{../solid/solid}[solid.pdf]
\usepackage[colorlinks,unicode]{hyperref} % , linkcolor=black, anchorcolor=black, citecolor=black, urlcolor=black, filecolor=black
\usepackage[most]{tcolorbox}
\usepackage{prettyref}

% Page style
\geometry{left=3.18cm,right=3.18cm,top=2.54cm,bottom=2.54cm}
\titlespacing{\paragraph}{0pt}{1pt}{10pt}[20pt]
\setlength{\droptitle}{-5em}
\preauthor{\vspace{-10pt}\begin{center}}
\postauthor{\par\end{center}}

% More compact lists 
\setlist[itemize]{
    itemindent=17pt, 
    leftmargin=1pt,
    listparindent=\parindent,
    parsep=0pt,
}

% Math operators
\DeclareMathOperator{\timeorder}{\mathcal{T}}
\DeclareMathOperator{\diag}{diag}
\DeclareMathOperator{\legpoly}{P}
\DeclareMathOperator{\primevalue}{P}
\DeclareMathOperator{\sgn}{sgn}
\newcommand*{\ii}{\mathrm{i}}
\newcommand*{\ee}{\mathrm{e}}
\newcommand*{\const}{\mathrm{const}}
\newcommand*{\suchthat}{\quad \text{s.t.} \quad}
\newcommand*{\argmin}{\arg\min}
\newcommand*{\argmax}{\arg\max}
\newcommand*{\normalorder}[1]{: #1 :}
\newcommand*{\pair}[1]{\langle #1 \rangle}
\newcommand*{\fd}[1]{\mathcal{D} #1}
\DeclareMathOperator{\bigO}{\mathcal{O}}

% TikZ setting
\usetikzlibrary{arrows,shapes,positioning}
\usetikzlibrary{calc}
\usetikzlibrary{arrows.meta}
\usetikzlibrary{decorations.markings}
\tikzstyle arrowstyle=[scale=1]
\tikzstyle directed=[postaction={decorate,decoration={markings,
    mark=at position .5 with {\arrow[arrowstyle]{stealth}}}}]
\tikzstyle ray=[directed, thick]
\tikzstyle dot=[anchor=base,fill,circle,inner sep=1pt]

% Algorithm setting
% Julia-style code
\SetKwIF{If}{ElseIf}{Else}{if}{}{elseif}{else}{end}
\SetKwFor{For}{for}{}{end}
\SetKwFor{While}{while}{}{end}
\SetKwProg{Function}{function}{}{end}
\SetArgSty{textnormal}

\newcommand*{\concept}[1]{{\textbf{#1}}}

% Embedded codes
\lstset{basicstyle=\ttfamily,
  showstringspaces=false,
  commentstyle=\color{gray},
  keywordstyle=\color{blue}
}

% Support for tensor double arrows.
\renewcommand{\tensor}[1]{ \stackrel{\leftrightarrow}{\vb*{#1}}}

% Reference formatting
\newrefformat{fig}{Figure~\ref{#1} on page~\pageref{#1}}

% Color boxes
\tcbuselibrary{skins, breakable, theorems}
\newtcbtheorem[number within=section]{warning}{Warning}%
  {colback=orange!5,colframe=orange!65,fonttitle=\bfseries, breakable}{warn}
\newtcbtheorem[number within=section]{note}{Note}%
  {colback=green!5,colframe=green!65,fonttitle=\bfseries, breakable}{note}

\title{Advanced Electrodynamics, Homework 3}
\author{Jinyuan Wu}

\begin{document}

\maketitle

\paragraph{2D Green function} (a) Derive the 2D Green function in polar coordinates.

\paragraph{Solution} \begin{itemize}
\item[(a)] The 2D Green function is given by the solution of the two dimensional version of Helmholtz equation
with an external source: 
\begin{equation}
    (\laplacian + k^2) G_0(\vb*{r} - \vb*{r}') = - \delta^{(2)}(\vb*{r} - \vb*{r}').
\end{equation}  
The solution, in terms of Fourier transformation, is 
\[
    G_0(\vb*{R}) = - \int \frac{\dd[2]{\vb*{p}}}{(2\pi)^2} \frac{\ee^{\ii \vb*{p} \cdot \vb*{R}}}{k^2 - {\vb*{p}}^2 + \ii 0^+}.
\]
In polar coordinates where we consider the direction of $\vb*{R}$ to be the $\theta = 0$ axis, we have 
% &= \frac{1}{(2\pi)^2} \frac{1}{2} \int_{0}^{2\pi} \dd{\theta} \int_0^\infty \dd{p}
%         \left( \frac{1}{p + k - \ii 0^+} + \frac{1}{p - k - \ii 0^+} \right) 
%         \ee^{\ii p \abs*{\vb*{R}} \cos \theta} \\
%         &= \frac{1}{2 (2\pi)^2} \left(
%             \int_{-\pi/2}^{\pi/2} \dd{\theta} \times 2 \pi \ii \ee^{\ii k \abs*{\vb*{R}} \cos \theta}
%             + \int_{\pi/2}^{3\pi/2} \dd{\theta} \times 2 \pi \ii \ee^{- \ii k \abs*{\vb*{R}} \cos \theta}
%         \right) \\
%         &= \frac{\ii}{4\pi} ((\pi J_0(k \abs*{\vb*{R}}) + \ii \pi \mathbf{H}(k \abs*{\vb*{R}}))
%         + (\pi J_0(k \abs*{\vb*{R}}) - \ii \pi \mathbf{H}(k \abs*{\vb*{R}}))) \\
%         &= \frac{\ii}{4\pi} \times 2 \pi J_0(k \abs*{\vb*{R}}).
\[
    \begin{aligned}
        G_0(\vb*{R}) &= - \frac{1}{(2\pi)^2} \int_{0}^\infty p \dd{p} \int_{0}^{2\pi} \dd{\theta} 
        \frac{\ee^{\ii p \abs*{\vb*{R}} \cos \theta}}{k^2 - p^2 + \ii 0^+} \\
        &= \frac{1}{(2\pi)^2} \int_0^\infty p \dd{p} \times \frac{1}{p^2 - k^2 - \ii 0^+} 
        \times 2\pi J_0(p \abs*{\vb*{R}}) \\
        &= \frac{1}{2\pi} \times K_0\left(\frac{\abs*{\vb*{R}}}{\sqrt{- 1 / k^2}}\right) \\
        &= \frac{1}{2\pi} K_0(- \ii k \abs*{\vb*{R}}) 
        = \frac{1}{2\pi} \times \frac{\pi}{2} \ii H_0^{(1)}(k \abs*{\vb*{R}}) ,
    \end{aligned}
\]
where the third line is obtained using Mathematica, and the fourth line comes from well-known properties 
of Bessel functions \cite{besselwiki}. So we get 
\begin{equation}
    G_0(\vb*{R}) = \frac{\ii}{4} H_0^{(1)}(k \abs*{\vb*{R}}). 
\end{equation}

\end{itemize}

\paragraph{}

\paragraph{Dyadic green function in Fourier space} (a) Show that in vacuum the Maxwell equations can be rephrased 
into 
\begin{equation}
    \vb*{M}^{2}\left[\begin{array}{l}
        \vb*{E} \\
        \vb*{H}
        \end{array}\right]=\left[\begin{array}{cc}
        c^{2} \vb*{k} \cdot \vb*{k}-c^{2} \vb*{k} \vb*{k} & 0 \\
        0 & c^{2} \vb*{k} \cdot \vb*{k}-c^{2} \vb*{k} \vb*{k}
        \end{array}\right]\left[\begin{array}{c}
        \vb*{E} \\
        \vb*{H}
        \end{array}\right]=\omega^{2}\left[\begin{array}{c}
        \vb*{E} \\
        \vb*{H}
        \end{array}\right].
        \label{eq:maxwell-eigen}
\end{equation}
(b) Find the eigenvalues and eigenvectors. (c) Derive the Green function in the Fourier space, 
and show why longitude modes are absent.

\paragraph{Solution} \begin{itemize}
\item[(a)] In the Fourier space the Maxwell equations are 
\[
    \begin{aligned}
        \vb*{k} \cdot \vb*{E} &= 0, \\ 
        \vb*{k} \times \vb*{E} &= \omega \vb*{B}, \\
        \vb*{k} \cdot \vb*{B} &= 0, \\
        \vb*{k} \times \vb*{B} &= - \frac{1}{c^2} \omega \vb*{E}.
    \end{aligned}
\]
where $\vb*{E}$ and $\vb*{B}$ are actually $\vb*{\mathcal{E}}(\vb*{k}, \omega)$ 
and $\vb*{\mathcal{B}}(\vb*{k}, \omega)$, respectively. The first and the third equations have 
no $\omega$ dependence and therefore cannot be a part of the eigenvalue problem. 
From the second and the fourth equations we have 
\[
    \begin{aligned}
        \omega \vb*{k} \times \vb*{B} &= \vb*{k} \times (\vb*{k} \times \vb*{E}) \\
        &= (\vb*{k} \cdot \vb*{E}) \vb*{k} - \vb*{k}^2 \vb*{E} \\
        &= (\vb*{k} \vb*{k} - \vb*{k} \cdot \vb*{k}) \vb*{E},
    \end{aligned}
\]
and therefore 
\begin{equation}
    - \frac{\omega^2}{c^2} \vb*{E} = (\vb*{k} \vb*{k} - \vb*{k} \cdot \vb*{k}) \vb*{E}.
    \label{eq:e-eigen}
\end{equation}
Similarly we have 
\[
    \begin{aligned}
        \frac{\omega}{c^2} \vb*{k} \times \vb*{E} &= - \vb*{k} \times (\vb*{k} \times \vb*{B}) \\
        &= - (\vb*{k} \cdot \vb*{B}) \vb*{k} + \vb*{k}^2 \vb*{B} \\
        &= (- \vb*{k} \vb*{k} + \vb*{k} \cdot \vb*{k}) \vb*{B}, 
    \end{aligned}
\]
and therefore 
\begin{equation}
    \frac{\omega^2}{c^2} \vb*{B} = (- \vb*{k} \vb*{k} + \vb*{k} \cdot \vb*{k}) \vb*{B}.
    \label{eq:b-eigen}
\end{equation}
From \eqref{eq:e-eigen} and \eqref{eq:b-eigen} we have 
\begin{equation}
    \pmqty{\dmat{ \vb*{k} \cdot \vb*{k} - \vb*{k} \vb*{k} , \vb*{k} \cdot \vb*{k} - \vb*{k} \vb*{k} }} 
    \pmqty{\vb*{E} \\ \vb*{B}} = \frac{\omega^2}{c^2} \pmqty{\vb*{E} \\ \vb*{B}}. 
\end{equation}
Note that $B = \mu_0 \vb*{H}$, we find this is just \eqref{eq:maxwell-eigen}.

\item[(b)] What we need to do is to solve the equation 
\[
    \det( c^2 \vb*{k} \cdot \vb*{k} - c^2 \vb*{k} \vb*{k} - \omega^2)^2 = 0,
\] 
or in other words 
\[
    \det\pmqty{
        k_y^2 + k_z^2 - \frac{\omega^2}{c^2} & - k_x k_y & - k_x k_z \\
        - k_x k_y & k_x^2 + k_z^2 - \frac{\omega^2}{c^2} & - k_y k_z \\
        - k_x k_z & - k_y k_z & k_x^2 + k_y^2 - \frac{\omega^2}{c^2} 
    }^2 = 0.
\]
Factoring the LHS of the equation using Mathematica, we have 
\[
    \frac{\omega^4}{c^4} \left( k_x^2 + k_y^2 + k_z^2 - \frac{\omega^2}{c^2} \right)^4 = 0,
\]
so we have six eigen values, which are 
\begin{equation}
    \left(\frac{\omega}{c}\right)^2_1 = \left(\frac{\omega}{c}\right)^2_2 = 0, \quad 
    \left(\frac{\omega}{c}\right)^2_3 = \left(\frac{\omega}{c}\right)^2_4 
    = \left(\frac{\omega}{c}\right)^2_5 = \left(\frac{\omega}{c}\right)^2_6 = \vb*{k}^2.
\end{equation}

We get six modes, and now we find the eigenmodes. 
For mode 1 and mode 2, we have 
\[
    \pmqty{
        k_y^2 + k_z^2 & - k_x k_y & - k_x k_z \\
        - k_x k_y & k_x^2 + k_z^2 & - k_y k_z \\
        - k_x k_z & - k_y k_z & k_x^2 + k_y^2 
    } \pmqty{ E_x \\ E_y ^2 \\ E_z^2 } = 0
\]
and the same equation holds for $\vb*{H}$, and a complete set of linear independent non-zero solutions are
\begin{equation}
    \pmqty{\vb*{E} \\ \vb*{H}} = \pmqty{k_x \\ k_y \\ k_z \\ 0 \\ 0 \\ 0},  
    \pmqty{0 \\ 0 \\ 0 \\ k_x \\ k_y \\ k_z}.
    \label{eq:invariant-1}
\end{equation}
For mode 3, 4, 5, and 6, we have 
\[
    \pmqty{
        - k_z^2 & - k_x k_y & - k_x k_z \\
        - k_x k_y & - k_x^2 & - k_y k_z \\
        - k_x k_z & - k_y k_z & - k_y^2 
    } \pmqty{ E_x \\ E_y ^2 \\ E_z^2 } = 0,
\]
and the same equation holds for $\vb*{H}$, and a complete set of linear independent non-zero solutions are 
\begin{equation}
    \pmqty{\vb*{E} \\ \vb*{H}} = \pmqty{k_z \\ 0 \\ - k_x \\ 0 \\ 0 \\ 0}, 
    \pmqty{0 \\ 0 \\ 0 \\ k_z \\ 0 \\ - k_x }, \pmqty{-k_y \\ k_x \\ 0 \\ 0 \\ 0 \\ 0},
    \pmqty{ 0 \\ 0 \\ 0 \\ -k_y \\ k_x \\ 0}.
    \label{eq:invariant-2}
\end{equation}

\item[(c)] We do Schmidt orthogonalization and then invoke spectrum decomposition.
We let 
\[
    \pmqty{k_x \\ k_y \\ k_z} = k \pmqty{\sin \theta \cos \varphi \\ \sin \theta \sin \varphi \\ \cos \theta},  
\]
then the invariant space \eqref{eq:invariant-2} may be spanned by the direct product of the orthogonal uniform basis
\begin{equation}
    \pmqty{\cos \theta \cos \varphi \\ \cos \theta \sin \varphi \\ - \sin \theta}, 
    \pmqty{- \sin \varphi \\ \cos \varphi \\ 0}
    \label{eq:small-invariant}
\end{equation}
and $\{(1, 0), (0, 1)\}$. The projector operator into \eqref{eq:small-invariant} is 
\[
    \pmqty{\cos \theta \cos \varphi \\ \cos \theta \sin \varphi \\ - \sin \theta} 
    \pmqty{\cos \theta \cos \varphi & \cos \theta \sin \varphi & - \sin \theta} + 
    \pmqty{- \sin \varphi \\ \cos \varphi \\ 0} \pmqty{- \sin \varphi & \cos \varphi & 0} .
\]
Replacing $\theta, \varphi$ by $k_x, k_y, k_z$ using the equations 
\[
    \sin \varphi = \frac{k_y}{\sqrt{k_x^2 + k_y^2}}, \quad 
    \cos \varphi = \frac{k_x}{\sqrt{k_x^2 + k_y^2}}, \quad 
    \sin \theta = \frac{\sqrt{k_x^2 + k_y^2}}{\sqrt{k_x^2 + k_y^2 + k_z^2}}, \quad 
    \cos \theta = \frac{k_z}{\sqrt{k_x^2 + k_y^2 + k_z^2}},
\]
the projector operator into \eqref{eq:invariant-2} is the product between
\[
    \pmqty{ \frac{k_y^2}{k_x^2 + k_y^2} & - \frac{k_x k_y}{k_x^2 + k_y^2} & 0
    \\ - \frac{k_x k_y}{k_x^2 + k_y^2} & \frac{k_x^2}{k_x^2 + k_y^2} & 0 \\
    0 & 0 & 0} +
    \pmqty{
        \frac{k_z^2}{k_x^2 + k_y^2 + k_z^2} \frac{k_y^2}{k_x^2 + k_y^2} &
        \frac{k_z^2}{k_x^2 + k_y^2 + k_z^2} \frac{k_x k_y}{k_x^2 + k_y^2} &
        - \frac{k_x k_z}{k_x^2 + k_y^2 + k_z^2} \\
        \frac{k_z^2}{k_x^2 + k_y^2 + k_z^2} \frac{k_x k_y}{k_x^2 + k_y^2} &
        \frac{k_z^2}{k_x^2 + k_y^2 + k_z^2} \frac{k_y^2}{k_x^2 + k_y^2} & 
        - \frac{k_y k_z}{k_x^2 + k_y^2 + k_z^2} \\
        - \frac{k_x k_z}{k_x^2 + k_y^2 + k_z^2} & 
        - \frac{k_y k_z}{k_x^2 + k_y^2 + k_z^2} &
        \frac{k_x^2 + k_y^2}{k_x^2 + k_y^2 + k_z^2}
    } 
\]
and $I^{2 \times 2}$. So by spectrum decomposition
\begin{equation}
    \tensor{\vb*{G}} = \sum_n \frac{\vb*{u}_n \vb*{u}_n^\dagger}{\lambda_n - \lambda},
\end{equation}
we have 
\begin{equation}
    \begin{aligned}
        \tensor{\vb*{G}} &= - \frac{1}{\omega^2} \frac{1}{k^2} \pmqty{\dmat{
            k_x^2 & k_x k_y & k_x k_z \\ k_x k_y & k_y^2 & k_y k_z \\ k_x k_z & k_y k_z & k_z^2,
            k_x^2 & k_x k_y & k_x k_z \\ k_x k_y & k_y^2 & k_y k_z \\ k_x k_z & k_y k_z & k_z^2
        }} \\
        &\quad+ \frac{1}{c^2 k^2 - \omega^2} \pmqty{\dmat{
            \frac{k_y^2}{k_x^2 + k_y^2} & - \frac{k_x k_y}{k_x^2 + k_y^2} & 0
            \\ - \frac{k_x k_y}{k_x^2 + k_y^2} & \frac{k_x^2}{k_x^2 + k_y^2} & 0 \\
            0 & 0 & 0,
            \frac{k_y^2}{k_x^2 + k_y^2} & - \frac{k_x k_y}{k_x^2 + k_y^2} & 0
            \\ - \frac{k_x k_y}{k_x^2 + k_y^2} & \frac{k_x^2}{k_x^2 + k_y^2} & 0 \\
            0 & 0 & 0
        }} \\
        &\quad+ \frac{1}{c^2 k^2 - \omega^2} \pmqty{\dmat{
            \frac{k_z^2}{k_x^2 + k_y^2 + k_z^2} \frac{k_y^2}{k_x^2 + k_y^2} &
            \frac{k_z^2}{k_x^2 + k_y^2 + k_z^2} \frac{k_x k_y}{k_x^2 + k_y^2} &
            - \frac{k_x k_z}{k_x^2 + k_y^2 + k_z^2} \\
            \frac{k_z^2}{k_x^2 + k_y^2 + k_z^2} \frac{k_x k_y}{k_x^2 + k_y^2} &
            \frac{k_z^2}{k_x^2 + k_y^2 + k_z^2} \frac{k_y^2}{k_x^2 + k_y^2} & 
            - \frac{k_y k_z}{k_x^2 + k_y^2 + k_z^2} \\
            - \frac{k_x k_z}{k_x^2 + k_y^2 + k_z^2} & 
            - \frac{k_y k_z}{k_x^2 + k_y^2 + k_z^2} &
            \frac{k_x^2 + k_y^2}{k_x^2 + k_y^2 + k_z^2}, 
            \frac{k_z^2}{k_x^2 + k_y^2 + k_z^2} \frac{k_y^2}{k_x^2 + k_y^2} &
            \frac{k_z^2}{k_x^2 + k_y^2 + k_z^2} \frac{k_x k_y}{k_x^2 + k_y^2} &
            - \frac{k_x k_z}{k_x^2 + k_y^2 + k_z^2} \\
            \frac{k_z^2}{k_x^2 + k_y^2 + k_z^2} \frac{k_x k_y}{k_x^2 + k_y^2} &
            \frac{k_z^2}{k_x^2 + k_y^2 + k_z^2} \frac{k_y^2}{k_x^2 + k_y^2} & 
            - \frac{k_y k_z}{k_x^2 + k_y^2 + k_z^2} \\
            - \frac{k_x k_z}{k_x^2 + k_y^2 + k_z^2} & 
            - \frac{k_y k_z}{k_x^2 + k_y^2 + k_z^2} &
            \frac{k_x^2 + k_y^2}{k_x^2 + k_y^2 + k_z^2}
            }}
        \end{aligned}
\end{equation}
 
\end{itemize}

\bibliographystyle{plain}
\bibliography{3} 

\end{document}