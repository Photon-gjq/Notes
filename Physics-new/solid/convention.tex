\section*{前置知识}

真正重要的前置知识只有量子力学的基本框架(有态矢量、有希尔伯特空间、有算符),哈密顿量和拉格朗日量是什么,以及在此基础上的一些自然推论,比如说量子场论和统计场论的框架,包括格林函数、费曼图、重整化群等。
关于平衡态格林函数和费曼图,\cite{wen2004quantum}的前几章是很好的介绍(但出于某些原因,文小刚使用零温场论的形式表述所有问题,即使是平衡态有限温问题)。

一定的“具体物理”还是需要的,包括单粒子量子力学,主要是关于$\comm*{x_i}{p_i} = \ii \delta_{ij}$和自旋自由度的那些东西,以及经典和量子电动力学,当然还有它的低能极限——库伦相互作用。
这些可以在任何教科书中找到。

一些好用的技巧,诸如Hubbard-Stratonovich变换,Bogoliubov变换等,可以在各种专门的凝聚态场论教科书中找到,如\cite{altland2010condensed}。

由于以上形式理论非常容易在各种教科书中找到,本文不再介绍它们,而是将注意力集中在凝聚态问题上。
必要的前置知识——尤其是那些涉及记号约定的——将通过“背景知识”块引入。

\section*{记号约定}

\subsection*{术语与表达式记号}

“自由”一词可能具有两个意思:一个表示\emph{自由理论},即系统的场论哈密顿量中只有场算符的二次型,一个表示\emph{自由电子},即能量就是$\vb*{k}^2/2m$的电子。
前者包括后者但是不完全就是后者,因为能带电子、紧束缚模型等也属于前者。

本文提到的费米子主要是电子,使用一个三维坐标$\vb*{r}$(或者三位动量$\vb*{p}$),以及只有向上和向下两种选择的自旋就可以描述一个电子。
单电子自旋算符为
\begin{equation}
    {\vb*{S}} = \sum_{\alpha, \beta} \ket{\alpha} \vb*{\sigma}_{\alpha \beta} \bra{\beta},
\end{equation}
其中$\alpha$和$\beta$取遍$\uparrow$和$\downarrow$,$\sigma$为泡利矩阵。
在不涉及自旋-轨道耦合的场合,在书写哈密顿量时我们直接略去自旋的下标,这是合理的,因为只需要把不考虑自旋的哈密顿量中的各个产生湮灭算符根据自旋守恒的性质机械地加上自旋下标再求和就能够得到完整的哈密顿量。
在需要实际计算粒子数时就不能这么做了;需要计算总能量时当然也不能这么做。

由于本文不涉及相对论性过程,设$\vb*{a}$为一个矢量,则使用$a$表示其模长。

在符号够用时,用$\ii \omega_n$表示松原频率,在符号不够用时,设一个场已有动量标记$\vb*{k}$,用$\ii k^0$表示松原频率。
此时用$k$表示$(\vb*{k}, \ii k^0)$。

对离散格点系统,使用$\pair{i, j}$表示最接近的一对格点。(只求和一次,即认为$\pair{i, j}$和$\pair{j, i}$相同)

我们用$\{\alpha | \vb*{a} \}$表示三维欧几里得群$E$的成员,它定义为
\begin{equation}
    \{\alpha | \vb*{a} \} \vb*{r} = \alpha \vb*{r} + \vb*{a},
\end{equation}
其乘法为
\begin{equation}
    \{\alpha_i | \vb*{a}_i \} \{\alpha_j | \vb*{a}_j \} = \{ \alpha_i \alpha_j | \alpha_i \vb*{a}_j + \vb*{a}_i \},
\end{equation}
从而
\begin{equation}
    \{\alpha_i | \vb*{a}_i \}^{-1} = \{ \alpha_i^{-1} | - \alpha_i^{-1} \vb*{a}_i \}.
\end{equation}

\subsection*{单位制}

本文取普朗克单位制,认为$\hbar=c=1$,且$4\pi\epsilon_0=1$,$k_\text{B}=1$。
将本文的计算结果恢复到国际单位制需要遵循以下规则:
\begin{itemize}
    \item 将本文中的$T$替换为$k_\text{B} T$;
    \item 
\end{itemize}
需要注意“比率”(如热容、态密度等)的\emph{定义式}中的$T, \dd[3]{\vb*{k}}$等不需要替换。

$\text{h.c.}$表示厄米共轭,$\text{c.c.}$表示复共轭。

若无特殊说明,$f(z)$定义为近独立费米子的分布函数,即
\[
    f(z) = \frac{1}{\ee^{\beta z} + 1}.
\]

\subsection*{主要的字母符号}

费米子的产生湮灭算符为${c}^\dagger$和${c}$,而如果是关于位置的产生湮灭算符,则为${\psi}^\dagger$和${\psi}$。
