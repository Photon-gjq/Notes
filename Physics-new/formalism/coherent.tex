\begin{back}{相干态和相干态路径积分}{coherent-state}
    设动力学变量$x$和它的正则动量满足正则对易关系
    \begin{equation}
        \comm{x}{p} = \ii,
    \end{equation}
    引入产生湮灭算符
    \begin{equation}
        a = \frac{1}{\sqrt{2}} (x + \ii p), \quad a^\dagger = \frac{1}{\sqrt{2}} (x - \ii p),
    \end{equation}
    则有正确的对易关系
    \begin{equation}
        \comm{a}{a^\dagger} = 1.
    \end{equation}
    我们现在考虑怎么将一个密度矩阵写成某种“准概率分布”的形式,即能够将它和一个函数$W(x, p)$或者$W(a, a^*)$建立线性关系。

    考虑用复参数定义的相干态
    \begin{equation}
        \ket*{\alpha} = \ee^{- \abs*{\alpha}^2 / 2} \sum_{n = 0}^\infty \frac{\alpha^n}{\sqrt{n!}} \ket*{0},
    \end{equation}
    其中$\ket*{n}$是谐振子哈密顿量
    \begin{equation}
        H = \sum_{n \geq 0} \hbar \omega \left( a^\dagger a + \frac{1}{2} \right) 
    \end{equation}
    的第$n$激发态。
    能够证明完备条件
    \begin{equation}
        \frac{1}{\pi} \int \dd[2]{\alpha} \dyad{\alpha} = 1
    \end{equation}
    成立,这里积分测度为
    \begin{equation}
        \dd[2]{\alpha} = \frac{\dd{\alpha^*} \wedge \dd{\alpha}}{2 \ii} = R \dd{R} \wedge \dd{\theta},
    \end{equation}
    其中$R$是$\alpha$的长度而$\theta$是相角,$\dd[2]{\alpha}$是复平面上的积分测度。
    由于不同$\alpha$的相干态彼此不正交——实际上,我们有
    \begin{equation}
        \braket*{\alpha}{\beta} = \ee^{- (\abs{\alpha}^2 + \abs{\beta}^2) / 2 + \alpha^* \beta},
    \end{equation}
    全体相干态实际上是超完备的。

\end{back}