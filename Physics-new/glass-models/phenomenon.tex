\documentclass[hyperref, a4paper]{article}

\usepackage{geometry}
\usepackage{titling}
\usepackage{titlesec}
% No longer needed, since we will use enumitem package
% \usepackage{paralist}
\usepackage{enumitem}
\usepackage{footnote}
\usepackage{enumerate}
\usepackage{amsmath, amssymb, amsthm}
\usepackage{mathtools}
\usepackage{bbm}
\usepackage{cite}
\usepackage{graphicx}
\usepackage{subfigure}
\usepackage{physics}
\usepackage{tensor}
\usepackage{siunitx}
\usepackage[version=4]{mhchem}
\usepackage{tikz}
\usepackage{xcolor}
\usepackage{listings}
\usepackage{autobreak}
\usepackage[ruled, vlined, linesnumbered]{algorithm2e}
\usepackage{nameref,zref-xr}
\zxrsetup{toltxlabel}
\usepackage[colorlinks,unicode]{hyperref} % , linkcolor=black, anchorcolor=black, citecolor=black, urlcolor=black, filecolor=black
\usepackage{prettyref}

% Page style
\geometry{left=3.18cm,right=3.18cm,top=2.54cm,bottom=2.54cm}
\titlespacing{\paragraph}{0pt}{1pt}{10pt}[20pt]
\setlength{\droptitle}{-5em}
\preauthor{\vspace{-10pt}\begin{center}}
\postauthor{\par\end{center}}

% More compact lists 
%\setlist[itemize]{
    %itemindent=17pt, 
    %leftmargin=1pt,
    %listparindent=\parindent,
    %parsep=0pt,
%}

% Math operators
\DeclareMathOperator{\timeorder}{\mathcal{T}}
\DeclareMathOperator{\diag}{diag}
\DeclareMathOperator{\legpoly}{P}
\DeclareMathOperator{\primevalue}{P}
\DeclareMathOperator{\sgn}{sgn}
\newcommand*{\ii}{\mathrm{i}}
\newcommand*{\ee}{\mathrm{e}}
\newcommand*{\const}{\mathrm{const}}
\newcommand*{\suchthat}{\quad \text{s.t.} \quad}
\newcommand*{\argmin}{\arg\min}
\newcommand*{\argmax}{\arg\max}
\newcommand*{\normalorder}[1]{: #1 :}
\newcommand*{\pair}[1]{\langle #1 \rangle}
\newcommand*{\fd}[1]{\mathcal{D} #1}
\DeclareMathOperator{\bigO}{\mathcal{O}}

% TikZ setting
\usetikzlibrary{arrows,shapes,positioning}
\usetikzlibrary{arrows.meta}
\usetikzlibrary{decorations.markings}
\tikzstyle arrowstyle=[scale=1]
\tikzstyle directed=[postaction={decorate,decoration={markings,
    mark=at position .5 with {\arrow[arrowstyle]{stealth}}}}]
\tikzstyle ray=[directed, thick]
\tikzstyle dot=[anchor=base,fill,circle,inner sep=1pt]

% Algorithm setting
% Julia-style code
\SetKwIF{If}{ElseIf}{Else}{if}{}{elseif}{else}{end}
\SetKwFor{For}{for}{}{end}
\SetKwFor{While}{while}{}{end}
\SetKwProg{Function}{function}{}{end}
\SetArgSty{textnormal}

\newcommand*{\concept}[1]{{\textbf{#1}}}

% Embedded codes
\lstset{basicstyle=\ttfamily,
  showstringspaces=false,
  commentstyle=\color{gray},
  keywordstyle=\color{blue}
}

\newcommand{\opticsdoc}{\href{../optics/optics}{the optics note}}
\newcommand{\soliddoc}{\href{../solid/solid}{the solid state physics note}}

\newrefformat{fig}{Figure~\ref{#1} on page~\pageref{#1}}
\newrefformat{sec}{Section~\ref{#1}}

\title{Phenomenology of the Glass Transition}
\author{Jinyuan Wu}

\begin{document}

\maketitle

This article and other articles in this folder are mainly informed by \cite{uni-chem-review}.

\section{Strong and fragile glass}

Glasses are often said to be very, very thick liquid. This claim is not that correct, actually.
Viscosities of liquids usually obey the Arrhenius equation \cite{viscosity-liquid}
\begin{equation}
    \eta = A \exp \left(\dfrac{E_a}{RT} \right) ,
    \label{eq:arrhenius}
\end{equation}
and the $\log \eta$ - $T_\text{g} / T$ relation is a straight line. 
Some glasses indeed have such a behavior and we call them \concept{strong glasses}. 
It should be noted, however, that glasses may \emph{break} under a external force. Liquids do not break.
They just absorb any amount of energy injected into them and exhaust the amount of energy via viscosity.
Brittleness, under this view, may be regarded as the fact that the system is unable to ``digest'' the input 
energy and transform them into heat, so the system gets tore apart.
So we find that a \concept{fragile glass} usually has a smaller $\eta$ compared to a similar strong glass 

\begin{figure}
    \centering
    

\tikzset{every picture/.style={line width=0.75pt}} %set default line width to 0.75pt        

\begin{tikzpicture}[x=0.75pt,y=0.75pt,yscale=-1,xscale=1]
%uncomment if require: \path (0,300); %set diagram left start at 0, and has height of 300

%Straight Lines [id:da1597530183407554] 
\draw [color={rgb, 255:red, 208; green, 2; blue, 27 }  ,draw opacity=1 ]   (182,226.33) -- (473.43,45.44) ;
%Curve Lines [id:da7858735050814729] 
\draw [color={rgb, 255:red, 74; green, 144; blue, 226 }  ,draw opacity=1 ]   (182,226.33) .. controls (288.43,227.44) and (456.43,106.44) .. (473.43,45.44) ;
%Straight Lines [id:da13335799595525732] 
\draw    (182,226.33) -- (480.79,226.33) ;
\draw [shift={(482.79,226.33)}, rotate = 180] [fill={rgb, 255:red, 0; green, 0; blue, 0 }  ][line width=0.08]  [draw opacity=0] (12,-3) -- (0,0) -- (12,3) -- cycle    ;
%Straight Lines [id:da08630284642228214] 
\draw    (182,226.33) -- (182,51.06) ;
\draw [shift={(182,49.06)}, rotate = 450] [fill={rgb, 255:red, 0; green, 0; blue, 0 }  ][line width=0.08]  [draw opacity=0] (12,-3) -- (0,0) -- (12,3) -- cycle    ;

% Text Node
\draw (180,49.06) node [anchor=east] [inner sep=0.75pt]    {$\log \eta $};
% Text Node
\draw (484.79,226.33) node [anchor=west] [inner sep=0.75pt]    {$T_\text{g} / T$};
% Text Node
\draw (288.92,87.5) node [anchor=north west][inner sep=0.75pt]   [align=left] {strong glass};
% Text Node
\draw (385.42,146.5) node [anchor=north west][inner sep=0.75pt]   [align=left] {fragile glass};


\end{tikzpicture}

    \caption{Viscosities of strong glasses and fragile glasses. Figure taken from Figure~\cite{uni-chem-review}.}
    \label{fig:viscosity}
\end{figure}

\section{}

\bibliographystyle{plain}
\bibliography{glass} 

\end{document}