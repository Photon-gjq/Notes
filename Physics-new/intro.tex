\documentclass[hyperref, a4paper]{article}

\usepackage{geometry}
\usepackage{titling}
\usepackage{titlesec}
% No longer needed, since we will use enumitem package
% \usepackage{paralist}
\usepackage{enumitem}
\usepackage{footnote}
\usepackage{enumerate}
\usepackage{amsmath, amssymb, amsthm}
\usepackage{mathtools}
\usepackage{bbm}
\usepackage{cite}
\usepackage{graphicx}
\usepackage{subfigure}
\usepackage{physics}
\usepackage{tensor}
\usepackage{siunitx}
\usepackage[version=4]{mhchem}
\usepackage{tikz}
\usepackage{xcolor}
\usepackage{listings}
\usepackage{autobreak}
\usepackage[ruled, vlined, linesnumbered]{algorithm2e}
\usepackage{nameref,zref-xr}
\zxrsetup{toltxlabel}
\zexternaldocument*[optics-]{./optics/optics}[optics.pdf]
\zexternaldocument*[solid-]{./solid/solid}[solid.pdf]
\zexternaldocument*[info-]{./information/quantum-circuit}[quantum-circuit.pdf]
\usepackage[colorlinks,unicode]{hyperref} % , linkcolor=black, anchorcolor=black, citecolor=black, urlcolor=black, filecolor=black
\usepackage{prettyref}

% Page style
\geometry{left=3.18cm,right=3.18cm,top=2.54cm,bottom=2.54cm}
\titlespacing{\paragraph}{0pt}{1pt}{10pt}[20pt]
\setlength{\droptitle}{-5em}
\preauthor{\vspace{-10pt}\begin{center}}
\postauthor{\par\end{center}}

% More compact lists 
%\setlist[itemize]{
    %itemindent=17pt, 
    %leftmargin=1pt,
    %listparindent=\parindent,
    %parsep=0pt,
%}

% Math operators
\DeclareMathOperator{\timeorder}{\mathcal{T}}
\DeclareMathOperator{\diag}{diag}
\DeclareMathOperator{\legpoly}{P}
\DeclareMathOperator{\primevalue}{P}
\DeclareMathOperator{\sgn}{sgn}
\newcommand*{\ii}{\mathrm{i}}
\newcommand*{\ee}{\mathrm{e}}
\newcommand*{\const}{\mathrm{const}}
\newcommand*{\suchthat}{\quad \text{s.t.} \quad}
\newcommand*{\argmin}{\arg\min}
\newcommand*{\argmax}{\arg\max}
\newcommand*{\normalorder}[1]{: #1 :}
\newcommand*{\pair}[1]{\langle #1 \rangle}
\newcommand*{\fd}[1]{\mathcal{D} #1}
\DeclareMathOperator{\bigO}{\mathcal{O}}

% TikZ setting
\usetikzlibrary{arrows,shapes,positioning}
\usetikzlibrary{arrows.meta}
\usetikzlibrary{decorations.markings}
\tikzstyle arrowstyle=[scale=1]
\tikzstyle directed=[postaction={decorate,decoration={markings,
    mark=at position .5 with {\arrow[arrowstyle]{stealth}}}}]
\tikzstyle ray=[directed, thick]
\tikzstyle dot=[anchor=base,fill,circle,inner sep=1pt]

% Algorithm setting
% Julia-style code
\SetKwIF{If}{ElseIf}{Else}{if}{}{elseif}{else}{end}
\SetKwFor{For}{for}{}{end}
\SetKwFor{While}{while}{}{end}
\SetKwProg{Function}{function}{}{end}
\SetArgSty{textnormal}

\newcommand*{\concept}[1]{{\textbf{#1}}}

% Embedded codes
\lstset{basicstyle=\ttfamily,
  showstringspaces=false,
  commentstyle=\color{gray},
  keywordstyle=\color{blue}
}

\newcommand{\opticsdoc}{\href{../optics/optics}{the optics note}}
\newcommand{\soliddoc}{\href{../solid/solid}{the solid state physics note}}

\newrefformat{fig}{Figure~\ref{#1} on page~\pageref{#1}}
\newrefformat{sec}{Section~\ref{#1}}

\newenvironment{qanda}{\setlength{\parindent}{0pt}}{\bigskip}
\newcommand{\Q}{\bigskip\bfseries Q: }
\newcommand{\A}{\par\textbf{A:} \normalfont}

\title{A Travel Guide}
\author{Jinyuan Wu}

\begin{document}

\maketitle

\section{What to expect in courses}

\section{Confusion in formalisms and jargons}

Top rules:
\begin{itemize}
    \item Never pay too much attention to formalisms. Never, never, pay too much attention to generalization of formalisms.
    \item Physics is built up by \emph{cases}. Always calculate important cases.
    \item Formalisms are built to deal with important cases. If you face obstacles when trying to generalize one theory, it is almost always the case that the generalization is related to some highly non-trivial phenomena.
    For example, generalization of quantum field theories into an arbitrary spacetime is related to a self-consistent theory on quantum gravity, which remains an open question with no hope to be answered in the foreseeable future.
\end{itemize}

\subsection{Application of math in physics}

\subsection{Quantum theories}

This part is about the basic formalisms of any quantum theory, including the meaning of states, operators and other.

\begin{qanda}

\Q Why is it legit to quantize a theory? Is there always a duality between a classical theory and a quantum theory?
\A Actually there is no \emph{quantization}. We only have \emph{classicalization}. The quantization procedures shown in quantum mechanics and quantum field theory textbooks are just pedagogical \emph{arguments} that make readers accept the quantum theories introduced.
Some classical theories give weird results when quantized. For example, a certain string theory lives in a 11-dimensional spacetime or otherwise we have Lorentz anomalies.
Some quantum theories just do not have an obvious classical counterpart, for example certain conformal field theories.

\Q What does it mean by the term \concept{quantum fluctuation}?
\A 

\Q Are ladder operators always possible to define in any operator algebra?
\A In principle you can always find ladder operators or an adapted version of them to resolve the problem of degeneracy.
But generally speaking there are no such beautiful expressions as is the case in linear oscillators or spins.
These two systems have ladder operators with simple analytic forms because of their good Lie algebra structures.

We should not worry about this, however, as roughly speaking, the only systems physicists know are oscillators and spins.
Generalization is not that important. Again, \emph{specific yet thought-provoking cases} are what really matter.

\Q The concept of measurement makes me nervous ...
\A Entanglement

It should be noted, however, that any feasible exact calculations are done in a setting where a quantum system is embedded into a classical background.
It is easy to notice that there is almost no measurement in the sense of quantum mechanics in quantum field theories, because 

\end{qanda}

\subsection{From quantum to classical}

\begin{qanda}

\Q Is taking the classical limit just taking the $\hbar \to 0$ limit?
\A Things are more tricky than a simple $\hbar \to 0$ limit. When we are talking about \emph{classical statistical mechanics}, we calculate the partition function as if all operators commute, which is % TODO: 

\Q Why are classical kinetic theories or fluid dynamics or things like that used in systems that are quantum enough?
\A Because the derivations of these theories are actually not restricted to classical theories, though they are often carried out in classical systems.

\end{qanda}

\subsection{Low energy effective theories}

\begin{qanda}

\Q It seems people always taking a low energy effective theory for granted and do not reason about whether there is a well-defined one ...
\A The mere existence of a low energy effective theory can indeed be taken for granted because in principle, we can just analyze the low-energy subspace of a Hamiltonian and find some labels to label them, thus effectively writing down a low energy effective theory.

What cannot be taken for granted is \emph{which} labels they are. The low energy effective theory of an interacting particle system is not always a theory about particles - for example in vicinity of a critical point we may get a conformal field theory, the correlation functions of which lack clear poles and therefore there is no well-defined particles.
Sometimes formally we can write down an effective theory concerning several specific degrees of freedom, but in this situation sometimes the degrees of freedom fluctuate strongly and the theory is not well-defined (or in the language of field theories, have infinitely high order terms). 
When choosing the degree of freedoms in the low energy effective theory, we are making very strong assumption about the system under investigation, which relies on intuition and numerical results.

\Q It seems the analytic approaches to obtain an EFT are not many ...
\A Unfortunately, yes, especially in condensed matter physics. What we do have are:
\begin{itemize}
    \item Hubbard-Stratonovich transformation. 
    \item Parton constructions.
\end{itemize}
All of these approaches have the deficient we mentioned above that we do not know if a degree of freedom fluctuate wildly.
Formally several Hubbard-Stratonovich transformations can be applied to a field theory with a $\bar{\psi} \bar{\psi} \psi \psi$ interaction term, but most of the Hubbard-Stratonovich parameters have large fluctuation.

\end{qanda}

\subsection{Quantum field theories}

\begin{qanda}

\Q The second quantization procedure that somehow ``promotes'' the wave function of a single particle to a quantum field does not seem to make sense.
\A The term \concept{second quantization} has several different meanings, all of which can be found in textbooks.
    
\Q Is any quantum theory of fields a theory of particles? Or in other words, are quantum fields just convenient ways to represent particles' behaviors?
\A The two claims are only true in \emph{perturbative quantum field theories}, which are the only \emph{quantum field theories} used in particle physics.
There are occasions, however, when quantum fields themselves have physical meanings.
The conformal field theory is an example. Another example is the lattice field theory approach, for example lattice QCD.
In the strong coupling regime, new emergent phenomena occur and it is not correct to assume that the system of interest is made of particles corresponding to the fields appearing in the Lagrangian.

The term \concept{quantum field theory}, therefore, is kind of confusing because it may be used to denote the \emph{quantum mechanics of fields}, where fields themselves are what we are interested in, or the perturbative theory used in particle physics, when a theory can actually be defined with just Feynman rules without mentioning fields. 
Griffith's textbook on particle physics is a good example of \emph{quantum field theory (in the latter sense) without fields}.
Normal textbooks whose titles include the word ``field'' will first introduce the quantum theory of fields in the former sense and then secretly slip to the latter view of point.

\Q Is the Lagrangian really necessary considering the fact that perturbative quantum field theories can be defined purely in terms of Feynman diagrams and that some field theories do not have a Lagrangian? Or is the Hamiltonian really necessary?
\A Since everything we are interested in in a quantum field theory can be computed using correlation functions, a quantum field theory can be seen as a (semi-)probabilistic theory where we are interested in the moments (i.e. correlation functions) of variables.
So yes, in principle we can define a quantum field theory using purely correlation functions, and if we want to impose certain constraints we can rephrase them in terms of the relations between different correlation functions.
For example, a free theory may be defined as a theory where the Wick theorem holds.
This approach is called \concept{bootstrap}.

It should be pointed out that bootstrap is not practical for most theories we will encounter. Conformal field theories and topological field theories can be bootstrapped because they are highly constrained, but it is impossible to bootstrap, say, QED, from properties like symmetry.
In this case we simply say it is not possible to define the field theory in question using bootstrap.
The idea that the Lagrangian or the Hamiltonian should be used less and we should focus more on correlation functions and amplitudes, however, is important, which gives rise to the so-called \concept{amplitudology}.



\end{qanda}

\section{Approximations}

Suppose 

\section{Tricks to remember the notations and conventions}

\subsection{Fourier transformation}

We have 
\[
    \braket*{\vb*{x}}{\vb*{p}} = \ee^{\ii \vb*{p} \cdot \vb*{x}},
\]
and therefore 
\[
    \braket*{\vb*{x}}{\psi} \sim \int \dd[n]{\vb*{p}} \ee^{\ii \vb*{p} \cdot \vb*{x}} \braket*{\vb*{p}}{\psi}.
\]
That is why in physics we usually use the following convention when doing Fourier transformation in the spatial coordinates:
\begin{equation}
    f(\vb*{x}) \sim \int \dd[n]{\vb*{p}} \ee^{\ii \vb*{p} \cdot \vb*{x}} f(\vb*{p}), \quad f(\vb*{p}) \sim \int \dd[n]{\vb*{x}} \ee^{- \ii \vb*{p} \cdot \vb*{x}} f(\vb*{x}).
\end{equation}
On the other hand, in the Schrödinger picture we have 
\[
    \ket*{\psi(t)} \sim \ket*{\psi(0)} \ee^{- \ii E t},
\]
and therefore in the temporal coordinate we do Fourier transformation in the following way:
\begin{equation}
    f(t) \sim \int \dd{\omega} f(\omega) \ee^{- \ii \omega t}, \quad f(\omega) \sim \int \dd{t} \ee^{\ii \omega t} f(t).
\end{equation}
Therefore, when we do Fourier transformation in all coordinates of a covariant relativistic system we have 
\begin{equation}
    f(x) \sim \int \dd{p} f(p) \ee^{- \ii p \cdot x}, \quad f(p) \sim \int \dd{x} f(x) \ee^{\ii p \cdot x}.
\end{equation}
where we use the $(+, -, -, -)$ metric.

\subsection{Wick rotation}

\end{document}