\documentclass[hyperref, a4paper]{article}

\usepackage{geometry}
\usepackage{marginnote}
\usepackage{titling}
\usepackage{titlesec}
% No longer needed, since we will use enumitem package
% \usepackage{paralist}
\usepackage{enumitem}
\usepackage{footnote}
% Conflicts with enumitem
%\usepackage{enumerate}
\usepackage{amsmath, amssymb, amsthm}
\usepackage{mathtools}
\usepackage{bbm}
\usepackage{cite}
\usepackage{graphicx}
\usepackage{subfigure}
\usepackage{physics}
\usepackage{tensor}
\usepackage{siunitx}
\usepackage[version=4]{mhchem}
\usepackage{tikz}
\usepackage{xcolor}
\usepackage{listings}
\usepackage{autobreak}
\usepackage[ruled, vlined, linesnumbered]{algorithm2e}
\usepackage{nameref,zref-xr}
\zxrsetup{toltxlabel}
\zexternaldocument*[optics-]{../optics/optics}[optics.pdf]
\zexternaldocument*[solid-]{../solid/solid}[solid.pdf]
\zexternaldocument*[qft-]{../relativistic-qft/relativistic-qft}[relativistic-qft.pdf]
\usepackage[colorlinks,unicode]{hyperref} % , linkcolor=black, anchorcolor=black, citecolor=black, urlcolor=black, filecolor=black
\usepackage[most]{tcolorbox}
\usepackage{prettyref}

% Page style
\geometry{left=3.18cm,right=3.18cm,top=2.54cm,bottom=2.54cm}
\titlespacing{\paragraph}{0pt}{1pt}{10pt}[20pt]
\setlength{\droptitle}{-5em}
\preauthor{\vspace{-10pt}\begin{center}}
\postauthor{\par\end{center}}

% More compact lists 
\setlist[itemize]{
    itemindent=17pt, 
    leftmargin=1pt,
    listparindent=\parindent,
    parsep=0pt,
}

% Math operators
\DeclareMathOperator{\timeorder}{\mathcal{T}}
\DeclareMathOperator{\diag}{diag}
\DeclareMathOperator{\legpoly}{P}
\DeclareMathOperator{\primevalue}{P}
\DeclareMathOperator{\sgn}{sgn}
\newcommand*{\ii}{\mathrm{i}}
\newcommand*{\ee}{\mathrm{e}}
\newcommand*{\const}{\mathrm{const}}
\newcommand*{\suchthat}{\quad \text{s.t.} \quad}
\newcommand*{\argmin}{\arg\min}
\newcommand*{\argmax}{\arg\max}
\newcommand*{\normalorder}[1]{: #1 :}
\newcommand*{\pair}[1]{\langle #1 \rangle}
\newcommand*{\fd}[1]{\mathcal{D} #1}
\DeclareMathOperator{\bigO}{\mathcal{O}}

% TikZ setting
\usetikzlibrary{arrows,shapes,positioning}
\usetikzlibrary{calc}
\usetikzlibrary{arrows.meta}
\usetikzlibrary{decorations.markings}
\tikzstyle arrowstyle=[scale=1]
\tikzstyle directed=[postaction={decorate,decoration={markings,
    mark=at position .5 with {\arrow[arrowstyle]{stealth}}}}]
\tikzstyle ray=[directed, thick]
\tikzstyle dot=[anchor=base,fill,circle,inner sep=1pt]

% Algorithm setting
% Julia-style code
\SetKwIF{If}{ElseIf}{Else}{if}{}{elseif}{else}{end}
\SetKwFor{For}{for}{}{end}
\SetKwFor{While}{while}{}{end}
\SetKwProg{Function}{function}{}{end}
\SetArgSty{textnormal}

\newcommand*{\concept}[1]{{\textbf{#1}}}

% Embedded codes
\lstset{basicstyle=\ttfamily,
  showstringspaces=false,
  commentstyle=\color{gray},
  keywordstyle=\color{blue}
}

% Reference formatting
\newrefformat{fig}{Figure~\ref{#1} on page~\pageref{#1}}

% Color boxes
\tcbuselibrary{skins, breakable, theorems}
\newtcbtheorem[number within=section]{warning}{Warning}%
  {colback=orange!5,colframe=orange!65,fonttitle=\bfseries, breakable}{warn}
\newtcbtheorem[number within=section]{note}{Note}%
  {colback=green!5,colframe=green!65,fonttitle=\bfseries, breakable}{note}

\newcommand{\ktnote}{\href{../topological-phases-reading-notes/kt.pdf}{this note}}

\title{Lorentz Group and Its Representations}
\author{Jinyuan Wu}

\begin{document}

\maketitle

This article is mainly based on Prof. Dingyu Shao's lecture on relativistic QFT at Fudan University.
It also summarize how these things can be found in different QFT textbooks.

\section{The geometry of the Lorentz group}

The \concept{Lorentz group} is defined to be the metric-keeping   

TODO: Parity and time reversal
Double covering

\section{Generators of the Lorentz group}

In Section~\ref{qft-sec:symmetry} in \href{../relativistic-qft/relativistic-qft.pdf}{this note}, we have 
introduced spins with the approach in \emph{Physics from Symmetry}. Here we redo the work in a more swift 
way in Peskin Section~3.1.

We already know the angular momentum operators are  
\begin{equation}
    \vb*{J} = \vb*{x} \times \vb*{p} = - \ii \vb*{x} \times \grad,
\end{equation}
which form a pseudovector. According to differential geometry, we can use the Hodge star operator to map it 
into a rank $3-2=2$ tensor, which is 
\begin{equation}
    \begin{aligned}
        J^{ij} &= \epsilon^{ijk} J^k = - \ii \epsilon^{ijk} \epsilon^{lmk} x^l \partial^m \\
        &= - \ii (\delta^{il} \delta^{jm} - \delta^{im} \delta^{jl}) x^l \partial^m \\
        &= - \ii (x^i \partial^j - x^j \partial^i),
    \end{aligned}
    \label{eq:jij-def}
\end{equation}
from which we have 
\[
    \begin{aligned}
        \epsilon^{lmi} J^{ij} &= \epsilon^{lmi} \epsilon^{ijk} J^k \\
        &= (\delta^{lj} \delta^{mk} - \delta^{lk} \delta^{jm}) J^k \\
        &= \delta^{lj} J^m - \delta^{jm} J^l,
    \end{aligned}
\]
and therefore 
\[
    \epsilon^{lji} J^{ij} = \delta^{mj} (\delta^{lj} J^m - \delta^{jm} J^l) = J^l - J^l \delta^{jj} = - 2 J^l,
\]
so
\begin{equation}
    J^l = \frac{1}{2} \epsilon^{lij} J^{ij}.    
\end{equation}

We can try to extend the antisymmetric tensor \eqref{eq:jij-def} into a 4-antisymmetric tensor, which is 
\begin{equation}
    J^{\mu \nu} = \ii (x^\mu \partial^\nu - x^\nu \partial^\mu).
    \label{eq:j-scalar-field}
\end{equation}
Note that $\partial^\mu = (\partial_t, - \grad)$. Now we find $J^{0 i}$ are just the boost generators.
Therefore, $J^{\mu \nu}$ - an antisymmetric tensor with 6 independent components - gives all rotational and 
boost generators of the Lorentz group. The $J^{0 i}$ components are boost generators, and the $J^{ij}$ 
components are rotational generators. A general Lorentz transformation is 
\begin{equation}
    U = \ee^{- \frac{\ii}{2} \omega_{\mu \nu} J^{\mu \nu}},
\end{equation}
where $\omega_{\mu \nu}$ is an antisymmetric matrix.

\eqref{eq:j-scalar-field} is actually the scalar field representation of the Lie algebra of the Lorentz group.
We need to find the commutation rules from \eqref{eq:j-scalar-field}, to find more representations of the 
Lorentz group. We have 
\begin{equation}
    \left[J^{\mu \nu}, J^{\rho \sigma}\right]=\ii \left(g^{\nu \rho} J^{\mu \sigma}-g^{\mu \rho} J^{\nu \sigma}-g^{\nu \sigma} J^{\mu \rho}+g^{\mu \sigma} J^{\nu \rho}\right).
    \label{eq:lorentz-algebra}
\end{equation}
The derivation of this equation is just to expand the LHS and sort all terms according to the metric factor.

The most frequently used representation is the representation on 4-vectors, which is 
\begin{equation}
    (\mathcal{J}^{\mu \nu})_{\alpha \beta} = \ii (\delta^{\mu}_\alpha \delta^{\nu}_\beta - \delta^\mu_\beta \delta^\nu_\alpha).
    \label{eq:four-vec-lorentz}
\end{equation}

Since the Poincare group preserves the metrics, we can regard any transformation in the active perspective, i.e.
the only change being 
\begin{equation}
    \psi(x) \longrightarrow \Lambda_{\psi} \psi(\Lambda^{-1} x), \quad \mathcal{L}(x) \longrightarrow \mathcal{L}(\Lambda^{-1}).
\end{equation}
Now we derive the Noether current of Lorentz generators. We have 
\[
    \var{\psi}^\mu = - \frac{\ii \var{\omega_{\rho \sigma}}}{2} (J^{\rho \sigma})\indices{^\mu_\nu} \psi^\nu
    + \frac{\ii \var{\omega_{\rho \sigma}}}{2} (J^{\rho \sigma})\indices{^\nu_\delta} x^\delta \partial_\nu \psi^\mu,
\]
and since $\partial_\nu x^\delta = \delta^\delta_\nu$, and $J\indices{^\nu_\delta}$ is antisymmetric, we have
\[
    \var{\mathcal{L}} = \frac{\ii \var{\omega_{\rho \sigma}}}{2} (J^{\rho \sigma})\indices{^\nu_\delta} x^\delta \partial_\nu \mathcal{L} = \partial_\nu \underbrace{\left( \frac{\ii \var{\omega_{\rho \sigma}}}{2} (J^{\rho \sigma})\indices{^\nu_\delta} x^\delta  \mathcal{L} \right)}_{\mathcal{J}^\mu}.
\]
Therefore, the Noether current is 
\[
    \partial_\mu\left( \pdv{\mathcal{L}}{\partial_\mu \psi^\nu} \var{\psi^\nu} - \mathcal{J}^\mu \right) = 0.
\]
Expanding the expression, we find the following conservation current:
\begin{equation}
    (J^{\rho \sigma})^\mu = (L^{\rho \sigma})^\mu + (S^{\rho \sigma})^\mu, \quad \partial_\mu (J^{\rho \sigma})^\mu = 0,
\end{equation}
where 
\begin{equation}
    (L^{\rho \sigma})^\mu = (J^{\rho \sigma})\indices{^\alpha_\delta} x^\delta T\indices{^\mu_\alpha},
\end{equation}
and 
\begin{equation}
    (S^{\rho \sigma})^\mu = - \pdv{\mathcal{L}}{\partial_\mu \psi^\nu} (J^{\rho \sigma})\indices{^\nu_\delta} \psi^\delta.
    \label{eq:spin-angular}
\end{equation}

\section{Dirac algebra}

Consider we have a representation $S^{\mu \nu}$ of \eqref{eq:lorentz-algebra}. Note that if we assume 
\begin{equation}
    S^{\mu \nu} = \frac{\ii}{4} \comm*{\gamma^\mu}{\gamma^\nu},
    \label{eq:j-gamma-def}
\end{equation}
where we have the \concept{Dirac algebra}
\begin{equation}
    \acomm*{\gamma^\mu}{\gamma^\nu} = 2 \eta^{\mu \nu}, 
\end{equation}
then \eqref{eq:lorentz-algebra} holds. The calculation is tedious. Here we list the main steps. First we have 
\[
    \begin{aligned}
        \comm*{S^{\mu \nu}}{S^{\rho \sigma}} &= - \frac{1}{16} \comm*{\comm*{\gamma^\mu}{\gamma^\nu}}{\comm*{\gamma^\rho}{\gamma^\sigma}} \\
        &= - \frac{1}{16} (\comm*{\comm*{\comm*{\gamma^\mu}{\gamma^\nu}}{\gamma^\rho}}{\gamma^\sigma} 
        + \comm*{\comm*{\comm*{\gamma^\nu}{\gamma^\rho}}{\gamma^\sigma}}{\gamma^\mu} \\
        &\quad \quad + \comm*{\comm*{\comm*{\gamma^\rho}{\gamma^\sigma}}{\gamma^\mu}}{\gamma^\nu} 
        + \comm*{\comm*{\comm*{\gamma^\sigma}{\gamma^\mu}}{\gamma^\nu}}{\gamma^\rho}) \\
        &= 
    \end{aligned}
\]
We can then prove 
\begin{equation}
    \comm*{\gamma^\mu}{S^{\rho \sigma}} = (\mathcal{J}^{\rho \sigma}) \indices{^\mu_\nu} \gamma^\nu,
\end{equation}
and then by inserting \eqref{eq:j-gamma-def} 

\begin{note*}{}{}
    We have the following formulae:
    \[
        \begin{aligned}
            &\{A,BC\}=\{A,B\}C-B[A,C], \\ 
            &\{AB,C\}=A\{B,C\}-[A,C]B, \\ 
            &[AB,C]=A\{B,C\}-\{A,C\}B, \\
            &[[A,C],[B,D]]=[[[A,B],C],D]+[[[B,C],D],A]+[[[C,D],A],B]+[[[D,A],B],C].
        \end{aligned}
    \]
\end{note*}

\begin{equation}
    S^{\rho \sigma} = \frac{\ii}{4} \comm*{\gamma^\rho}{\gamma^\sigma}.
\end{equation}

TODO: Can every representation be rewritten into the form of gamma matrices?

\section{Dirac spinors}

A possible of the Dirac algebra in the four-dimensional Minkowski space is
\begin{equation}
    \gamma^0 = \pmqty{0 & 1 \\ 1 & 0}, \quad \gamma^i = \pmqty{ 0 & \sigma^i \\ - \sigma^i & 0 },
\end{equation}
which is called the \concept{Weyl} or \concept{chiral} representation. This name comes from the fact that the 
we can find $u(p)$ and $v(p)$ are eigenvectors of the \emph{helicity} operator 
\begin{equation}
    h \coloneqq \hat{\vb*{p}} \cdot \vb*{S} = \frac{1}{2} \hat{p}_i \pmqty{\dmat{\sigma^i, \sigma^i}}.
\end{equation}
A particle with $h = 1/2$ is called \concept{right-handed}, and a particle with $h = - 1/2$ is called 
\concept{left handed}. Note that the notion of helicity is orthogonal to the notion of particle and antiparticle.
In this representation, we have 
\begin{equation}
    S^{0 i} = - \frac{\ii}{2} \pmqty{\dmat{\sigma^i, - \sigma^i}}, \quad 
    S^{ij} = \frac{1}{2} \epsilon^{ijk} \pmqty{\dmat{\sigma^k, \sigma^k}} \eqqcolon \frac{1}{2} \epsilon^{ijk} \Sigma^k.
    \label{eq:weyl-lorentz}
\end{equation}

\section{Lorentz transformation on Dirac spinors}



\end{document}