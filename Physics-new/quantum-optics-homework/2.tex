\documentclass[hyperref, a4paper]{article}

\usepackage{geometry}
\usepackage{float}
\usepackage{titling}
\usepackage{titlesec}
% No longer needed, since we will use enumitem package
% \usepackage{paralist}
\usepackage{enumitem}
\usepackage{footnote}
\usepackage{enumerate}
\usepackage{amsmath, amssymb, amsthm}
\usepackage{mathtools}
\usepackage{bbm}
\usepackage{cite}
\usepackage{graphicx}
\usepackage{subcaption}
\usepackage{physics}
\usepackage{tensor}
\usepackage{siunitx}
\usepackage{booktabs}
\usepackage[version=4]{mhchem}
\usepackage{tikz}
\usepackage{xcolor}
\usepackage{listings}
\usepackage{autobreak}
\usepackage[ruled, vlined, linesnumbered]{algorithm2e}
\usepackage{xr-hyper}
\usepackage[colorlinks,unicode]{hyperref} % , linkcolor=black, anchorcolor=black, citecolor=black, urlcolor=black, filecolor=black
\usepackage{prettyref}

% Page style
\geometry{left=3.18cm,right=3.18cm,top=2.54cm,bottom=2.54cm}
\titlespacing{\paragraph}{0pt}{1pt}{10pt}[20pt]
\setlength{\droptitle}{-5em}
\preauthor{\vspace{-10pt}\begin{center}}
\postauthor{\par\end{center}}

% More compact lists 
\setlist[itemize]{itemindent=17pt, leftmargin=1pt}

% Math operators
\DeclareMathOperator{\timeorder}{\mathcal{T}}
\DeclareMathOperator{\diag}{diag}
\DeclareMathOperator{\legpoly}{P}
\DeclareMathOperator{\primevalue}{P}
\DeclareMathOperator{\sgn}{sgn}
\newcommand*{\ii}{\mathrm{i}}
\newcommand*{\ee}{\mathrm{e}}
\newcommand*{\const}{\mathrm{const}}
\newcommand*{\suchthat}{\quad \text{s.t.} \quad}
\newcommand*{\argmin}{\arg\min}
\newcommand*{\argmax}{\arg\max}
\newcommand*{\normalorder}[1]{: #1 :}
\newcommand*{\pair}[1]{\langle #1 \rangle}
\newcommand*{\fd}[1]{\mathcal{D} #1}
\DeclareMathOperator{\bigO}{\mathcal{O}}
\DeclareMathOperator{\object}{Ob}
\DeclareMathOperator{\morphism}{Hom}

% TikZ setting
\usetikzlibrary{arrows,shapes,positioning}
\usetikzlibrary{arrows.meta}
\usetikzlibrary{decorations.markings}
\tikzstyle arrowstyle=[scale=1]
\tikzstyle directed=[postaction={decorate,decoration={markings,
    mark=at position .5 with {\arrow[arrowstyle]{stealth}}}}]
\tikzstyle ray=[directed, thick]
\tikzstyle dot=[anchor=base,fill,circle,inner sep=1pt]

% Algorithm setting
% Julia-style code
\SetKwIF{If}{ElseIf}{Else}{if}{}{elseif}{else}{end}
\SetKwFor{For}{for}{}{end}
\SetKwFor{While}{while}{}{end}
\SetKwProg{Function}{function}{}{end}
\SetArgSty{textnormal}

\newcommand*{\concept}[1]{{\textbf{#1}}}

\newrefformat{fig}{Figure~\ref{#1}}

% Embedded codes
\lstset{basicstyle=\ttfamily,
  showstringspaces=false,
  commentstyle=\color{gray},
  keywordstyle=\color{blue}
}

\title{Quantum Optics, Homework 2}
\author{Jinyuan Wu}

\begin{document}

\maketitle

\paragraph{Details in the HBT experiment} 

\paragraph{Solution}

\paragraph{}

\paragraph{Conditional generation of single photon pulses} Many research and applications in quantum optics needs single photon pulses, that is, a wave packet of light that contains exactly one single photon. Such a single photon pulse can be generated in two ways: The deterministic approach via single atom emission, and the so-called heralded approach. This problem discusses a simplified version of the later.
Consider a bi-photon generation process described by the Hamiltonian  
\begin{equation}
    H=\beta a_{k}^\dagger b_{k^{\prime}}^\dagger+ \text{h.c.}.
\end{equation}
Here $a_{k}^\dagger, b_{k^{\prime}}^\dagger$ are creation operators of photons into the $k, k'$ propagation modes respectively. Such process can be realized for example in a frequency down conversion experiment, where a single photon is ``split'' into two in a nonlinear optical crystal, or a 4-wave mixing experiment where two incoming photons are converted into two output photons in an atomic gas.
(a) Consider initially light is in vacuum state $|\psi(0)\rangle=|V\rangle$. Consider that the bi-photon generation process is switched on for time $\tau$ and then off, with $\xi=\beta \tau \ll 1$. Integrate the Schrodinger equation to obtain $|\psi(\tau)\rangle$, that is, the photon state after the interaction. (b) Consider a photon detector positioned $L$ meters away from the bi-photon generation device along the $k'$ propagation pathway. The time interval that the detector can detect a $b_{k'}$ photon is $[L / c, L / c+\tau]$ (we ignore any change of light speed within the experiment). For an ideal photon detector, what is the probability of detecting 1 photon, and detecting 2 photons during this time interval? If one photon is detected along $k'$, what is the photon state in the $k$ path? The strategy is the so called heralded single photon generation: a nearly perfect single photon pulse in the $k$ mode is heralded by the detection of a single photon in the $k'$-mode.

\paragraph{Solution} \begin{itemize}
    \item[(a)] In the interaction picture, the time evolution of the state is given by
    \[
        \ii \dv{t} \ket*{\psi(t)} = H \ket*{\psi(t)} = h(t) \left( \beta a^\dagger_{k} b_{k'}^\dagger + \text{h.c.} \right) \ket*{\psi(t)},
    \] 
    where $h(t)$ is one when $t \in [0, \tau]$ and zero otherwise.
    Formally we have 
    \[
        \ket*{\psi(t)} = \timeorder \exp(- \ii \int_0^t \dd{t'} h(t') (\beta a^\dagger_k b^\dagger_{k'} + \text{h.c.})) \ket*{\psi(0)}.
    \]
    Since $\xi \ll 1$, the operators approximately do not have time evolution, and thus we have 
    \begin{equation}
        \begin{aligned}
            \ket*{\psi(\tau)} &= \exp(- \ii \tau (\beta a^\dagger_k b^\dagger_{k'} + \text{h.c.})) \ket*{\psi(0)} \\
            &= \ee^{- \ii \xi (a^\dagger_k b^\dagger_{k'} + a_k b_{k'})} \ket*{0} .
        \end{aligned}
        \label{eq:final-state-tau}
    \end{equation}
    \item[(b)] Since the time interval is very short, we can view the measurement as simply measuring the state \eqref{eq:final-state-tau} as the pulse comes across the detector.
    Expanding \eqref{eq:final-state-tau} we have 
    \[
        \begin{aligned}
            \ket*{\psi(\tau)} &= \ket*{0} - \ii \xi (a^\dagger_{k} b^\dagger_{k'} + \text{h.c.}) \ket*{0} + \frac{1}{2} (- \ii \xi)^2 (a^\dagger_{k} b^\dagger_{k'} + \text{h.c.})^2 \ket*{0} + \cdots \\
            &= \left( 1 - \frac{\xi^2}{2} + \cdots \right) \ket*{0} - (\ii \xi + \cdots) \ket*{n_k = 1, n_{k'} = 1} - \left( \frac{1}{2} \xi^2 + \cdots \right) \ket*{n_k = 2, n_{k'} = 2} + \cdots.  
        \end{aligned}
    \] 
    Taking only the leading order terms, we have 
    \begin{equation}
        P(n_k = 1) = \xi^2, \quad P(n_k = 2) = \frac{\xi^4}{4}.
    \end{equation}

    It can be seen that in $\ket*{\psi(\tau)}$ we always have $n_k = n_{k'}$, and therefore if one photon is detected along $k'$, the photon state in the $k$ path is $\ket*{n_k = 1}$.
    Therefore if we placed a baffle in path $k$, which is removed when $n_{k'}$ is detected to be $1$, whenever a pulse is generated, it is a single-photon one.
\end{itemize}

\begin{figure}
    \centering
    

\tikzset{every picture/.style={line width=0.75pt}} %set default line width to 0.75pt        

\begin{tikzpicture}[x=0.75pt,y=0.75pt,yscale=-1,xscale=1]
%uncomment if require: \path (0,300); %set diagram left start at 0, and has height of 300

%Shape: Rectangle [id:dp9291562475809372] 
\draw  [draw opacity=0][fill={rgb, 255:red, 0; green, 0; blue, 0 }  ,fill opacity=0.36 ] (144,109) -- (165.71,109) -- (165.71,177.67) -- (144,177.67) -- cycle ;
%Straight Lines [id:da8833130318893516] 
\draw    (69.71,142.67) -- (144.71,142.67) ;
\draw [shift={(107.21,142.67)}, rotate = 180] [fill={rgb, 255:red, 0; green, 0; blue, 0 }  ][line width=0.08]  [draw opacity=0] (12,-3) -- (0,0) -- (12,3) -- cycle    ;
%Straight Lines [id:da6688023135314647] 
\draw    (164.85,143.34) -- (239.71,106.67) ;
\draw [shift={(202.28,125.01)}, rotate = 513.9] [fill={rgb, 255:red, 0; green, 0; blue, 0 }  ][line width=0.08]  [draw opacity=0] (12,-3) -- (0,0) -- (12,3) -- cycle    ;
%Straight Lines [id:da8843982354889979] 
\draw    (164.85,143.34) -- (278,186.29) ;
\draw [shift={(221.43,164.81)}, rotate = 200.79] [fill={rgb, 255:red, 0; green, 0; blue, 0 }  ][line width=0.08]  [draw opacity=0] (12,-3) -- (0,0) -- (12,3) -- cycle    ;
%Shape: Chord [id:dp20648494743318113] 
\draw   (283.31,171.99) .. controls (291.94,175.15) and (296.75,183.92) .. (294.07,191.77) .. controls (291.36,199.74) and (281.96,203.75) .. (273.08,200.72) -- cycle ;

% Text Node
\draw (241.71,103.27) node [anchor=south west] [inner sep=0.75pt]    {$k$};
% Text Node
\draw (234.71,174.07) node [anchor=north west][inner sep=0.75pt]    {$k'$};
% Text Node
\draw (152.38,103.74) node [anchor=south] [inner sep=0.75pt]   [align=left] {SPDC};


\end{tikzpicture}

    \caption{Light circuit in the heralded approach}
    \label{fig:heralded-circuit}
\end{figure}

\paragraph{}

\paragraph{Details in the NLS process} Analyze the NLS process in detail. 
The circuit is shown in \prettyref{fig:nls}, and the two beam splitters are represented as 
\begin{equation}
    S_1 = \pmqty{\cos \theta & \sin \theta \\
    - \sin \theta & \cos \theta}, \quad S_2 = \pmqty{\cos \sigma & \sin \sigma \\
    - \sin \sigma & \cos \sigma}
\end{equation}
(a) Derive the output quantum state before the measurement.
(b) Find the conditional quantum state with the measurement results shown in \prettyref{fig:nls}.
(c) Find when the NLS process works in terms of $\theta$ and $\sigma$, and the probability of a successful NLS.

\paragraph{Solution} \begin{itemize}
    \item[(a)] 
\end{itemize}

\begin{figure}
    \centering
    

\tikzset{every picture/.style={line width=0.75pt}} %set default line width to 0.75pt        

\begin{tikzpicture}[x=0.75pt,y=0.75pt,yscale=-1,xscale=1]
%uncomment if require: \path (0,300); %set diagram left start at 0, and has height of 300

%Shape: Square [id:dp5812640844481787] 
\draw   (206,108) -- (232,108) -- (232,134) -- (206,134) -- cycle ;
%Straight Lines [id:da23406010312221892] 
\draw    (206,134) -- (232,108) ;

%Straight Lines [id:da12470247976338489] 
\draw    (105,121) -- (219,121) ;
\draw [shift={(162,121)}, rotate = 180] [fill={rgb, 255:red, 0; green, 0; blue, 0 }  ][line width=0.08]  [draw opacity=0] (12,-3) -- (0,0) -- (12,3) -- cycle    ;
%Straight Lines [id:da8231064877854108] 
\draw    (219,121) -- (333,121) ;
%Straight Lines [id:da03344738041998285] 
\draw    (219,121) -- (219,211.33) ;
\draw [shift={(219,166.17)}, rotate = 90] [fill={rgb, 255:red, 0; green, 0; blue, 0 }  ][line width=0.08]  [draw opacity=0] (12,-3) -- (0,0) -- (12,3) -- cycle    ;
%Straight Lines [id:da3041250062025471] 
\draw    (219,53.33) -- (219,121) ;
\draw [shift={(219,87.17)}, rotate = 90] [fill={rgb, 255:red, 0; green, 0; blue, 0 }  ][line width=0.08]  [draw opacity=0] (12,-3) -- (0,0) -- (12,3) -- cycle    ;
%Shape: Square [id:dp9371797224374869] 
\draw   (320,108) -- (346,108) -- (346,134) -- (320,134) -- cycle ;
%Straight Lines [id:da05948010816759908] 
\draw    (320,134) -- (346,108) ;

%Straight Lines [id:da4093214516440322] 
\draw    (333,121) -- (447,121) ;
\draw [shift={(390,121)}, rotate = 180] [fill={rgb, 255:red, 0; green, 0; blue, 0 }  ][line width=0.08]  [draw opacity=0] (12,-3) -- (0,0) -- (12,3) -- cycle    ;
%Straight Lines [id:da3614482552806686] 
\draw    (333,121) -- (333,211.33) ;
\draw [shift={(333,166.17)}, rotate = 90] [fill={rgb, 255:red, 0; green, 0; blue, 0 }  ][line width=0.08]  [draw opacity=0] (12,-3) -- (0,0) -- (12,3) -- cycle    ;
%Straight Lines [id:da7606039605325201] 
\draw    (333,53.33) -- (333,121) ;
\draw [shift={(333,87.17)}, rotate = 90] [fill={rgb, 255:red, 0; green, 0; blue, 0 }  ][line width=0.08]  [draw opacity=0] (12,-3) -- (0,0) -- (12,3) -- cycle    ;
%Straight Lines [id:da4133297014948891] 
\draw    (219,53.33) -- (274,53.33) ;
%Straight Lines [id:da6242316516385615] 
\draw    (333,53.33) -- (388,53.33) ;
%Shape: Chord [id:dp8695097730726187] 
\draw   (274.11,38.08) .. controls (283.31,38.1) and (290.82,44.7) .. (290.98,53) .. controls (291.14,61.42) and (283.68,68.4) .. (274.3,68.58) -- cycle ;
%Shape: Chord [id:dp8381933152038832] 
\draw   (388.11,38.08) .. controls (397.31,38.1) and (404.82,44.7) .. (404.98,53) .. controls (405.14,61.42) and (397.68,68.4) .. (388.3,68.58) -- cycle ;

% Text Node
\draw (6,94.4) node [anchor=north west][inner sep=0.75pt]    {$\alpha \ket{0} +\beta \ket{1} +\gamma \ket{2}$};
% Text Node
\draw (204,214.4) node [anchor=north west][inner sep=0.75pt]    {$\ket{1}$};
% Text Node
\draw (322,212.4) node [anchor=north west][inner sep=0.75pt]    {$\ket{0}$};
% Text Node
\draw (265,18) node [anchor=north west][inner sep=0.75pt]   [align=left] {``1''};
% Text Node
\draw (387,18) node [anchor=north west][inner sep=0.75pt]   [align=left] {``0''};
% Text Node
\draw (107,124) node [anchor=north west][inner sep=0.75pt]   [align=left] {input 1};
% Text Node
\draw (221,208.33) node [anchor=south west] [inner sep=0.75pt]   [align=left] {input 2};
% Text Node
\draw (335,208.33) node [anchor=south west] [inner sep=0.75pt]   [align=left] {input 3};
% Text Node
\draw (203,32) node [anchor=north west][inner sep=0.75pt]   [align=left] {output 1};
% Text Node
\draw (327,32) node [anchor=north west][inner sep=0.75pt]   [align=left] {output 2};
% Text Node
\draw (386,97) node [anchor=north west][inner sep=0.75pt]   [align=left] {output 3};


\end{tikzpicture}

    \caption{The NLS circuit}
    \label{fig:nls}
\end{figure}

\paragraph{}

\end{document}