\documentclass[hyperref, UTF8, a4paper]{ctexart}

\usepackage{geometry}
\usepackage{titling}
\usepackage{titlesec}
\usepackage{paralist}
\usepackage{footnote}
\usepackage{enumerate}
\usepackage{amsmath, amssymb, amsthm}
\usepackage{bbm}
\usepackage{cite}
\usepackage{graphicx}
\usepackage{subfigure}
\usepackage{physics}
\usepackage{siunitx}
\usepackage{tikz}
\usepackage[compat=1.1.0]{tikz-feynhand}
\usepackage{autobreak}
\usepackage[ruled, vlined, linesnumbered, noend]{algorithm2e}
\usepackage[colorlinks, linkcolor=black, anchorcolor=black, citecolor=black]{hyperref}
\usepackage{prettyref}

% Page style
\geometry{left=3.18cm,right=3.18cm,top=2.54cm,bottom=2.54cm}
\titlespacing{\paragraph}{0pt}{1pt}{10pt}[20pt]
\setlength{\droptitle}{-5em}
\preauthor{\vspace{-10pt}\begin{center}}
\postauthor{\par\end{center}}

% Math operators
\DeclareMathOperator{\timeorder}{T}
\DeclareMathOperator{\diag}{diag}
\DeclareMathOperator{\legpoly}{P}
\DeclareMathOperator{\primevalue}{P}
\DeclareMathOperator{\sgn}{sgn}
\newcommand*{\ii}{\mathrm{i}}
\newcommand*{\ee}{\mathrm{e}}
\newcommand*{\const}{\mathrm{const}}
\newcommand*{\comment}{\paragraph{注记}}
\newcommand*{\suchthat}{\quad \text{s.t.} \quad}
\newcommand*{\argmin}{\arg\min}
\newcommand*{\argmax}{\arg\max}
\newcommand*{\normalorder}[1]{: #1 :}
\newcommand*{\pair}[1]{\langle #1 \rangle}
\newcommand*{\fd}[1]{\mathcal{D} #1}
\DeclareMathOperator{\bigO}{\mathcal{O}}

% prettyref setting
\newrefformat{sec}{第\ref{#1}节}
\newrefformat{note}{注\ref{#1}}
\newrefformat{fig}{图\ref{#1}}
\newrefformat{alg}{算法\ref{#1}}
\renewcommand{\autoref}{\prettyref}

% TikZ setting
\usetikzlibrary{arrows,shapes,positioning}
\usetikzlibrary{arrows.meta}
\usetikzlibrary{decorations.markings}
\tikzstyle arrowstyle=[scale=1]
\tikzstyle directed=[postaction={decorate,decoration={markings,
    mark=at position .5 with {\arrow[arrowstyle]{stealth}}}}]
\tikzstyle ray=[directed, thick]
\tikzstyle dot=[anchor=base,fill,circle,inner sep=1pt]

% Algorithm setting
\renewcommand{\algorithmcfname}{算法}
% Python-style code
\SetKwIF{If}{ElseIf}{Else}{if}{:}{elif:}{else:}{}
\SetKwFor{For}{for}{:}{}
\SetKwFor{While}{while}{:}{}
\SetKwInput{KwData}{输入}
\SetKwInput{KwResult}{输出}
\SetArgSty{textnormal}

\renewcommand{\emph}[1]{\textbf{#1}}
\newcommand*{\concept}[1]{\underline{\textbf{#1}}}
\newcommand*{\Ztwo}{$\mathbb{Z}_2$}

\title{三角晶格反铁磁序的相变}
\author{吴晋渊}

\begin{document}

\maketitle

\paragraph{(a)} 设晶格被往右移动了一个晶格常数的距离,这相当于标有$(a, b, c)$的三角形观察窗被向左移动了一个晶格常数的距离,从\prettyref{fig:lattice-trans}可以看出,设晶格移动后的$a, b, c$点为$a', b', c'$,则
\[
    a' = c, \quad b' = a, \quad c' = b,
\]
即
\[
    m_1' = m_3, \quad m_2' = m_1, \quad m_3' = m_2.
\]
于是
\[
    \begin{aligned}
        \psi' &= m_1' + m_2' \ee^{\ii 4 \pi / 3} + m_3' \ee^{- \ii 4 \pi / 3} \\
        &= m_3 + m_1 \ee^{\ii 4 \pi / 3} + m_2 \ee^{- \ii 4 \pi / 3} \\
        &= \ee^{\ii 4 \pi / 3} \psi,
    \end{aligned}
\]
则对应的用$m$和$\theta$表示的变换是
\begin{equation}
    m' = m, \quad \theta' = \theta + \frac{4\pi}{3}.
    \label{eq:cyc1}
\end{equation}
类似的,设晶格被往斜上方\SI{60}{\degree}的方向移动,此时发生的变换为
\[
    c' = a, \quad b' = c, \quad a' = b,
\]
然后
\[
    \begin{aligned}
        \psi' &= m_1' + m_2' \ee^{\ii 4 \pi / 3} + m_3' \ee^{- \ii 4 \pi / 3} \\
        &= m_2 + m_3 \ee^{\ii 4 \pi / 3} + m_1 \ee^{- \ii 4 \pi / 3} \\
        &= \ee^{- \ii 4 \pi / 3} \psi,
    \end{aligned} 
\] 
即
\begin{equation}
    m' = m, \quad \theta' = \theta - \frac{4\pi}{3}.
    \label{eq:cyc2}
\end{equation}
一个一般的晶格平移变换是以上两种平移变换的组合,设平移矢量为
\begin{equation}
    \vb*{l} = n_1 \vb*{e}_1 + n_2 \vb*{e}_2, \quad n_1, n_2 \in \mathbb{Z},
\end{equation}
则$\psi$的变换为
\begin{equation}
    m' = m, \quad \theta' = \theta + n_1 \frac{4 \pi}{3} - n_2 \frac{4\pi}{3}.
    \label{eq:transition}
\end{equation}

本模型允许三种类型的旋转。旋转中心可以设置在格点上,可以设置在三角形的中心,也可以设置在边的中点上。
如果旋转中心设置在格点上,可能的旋转角度为\SI{60}{\degree}的倍数。为简便起见下面仅考虑晶格被顺时针旋转\SI{60}{\degree}的情况,其它角度的情况只需要重复若干次\SI{60}{\degree}的旋转即可。
如果旋转中心在$a$点上,晶格被顺时针旋转\SI{60}{\degree}即相当于观察窗被逆时针旋转了\SI{60}{\degree},即相当于交换了$b$和$c$,于是$\psi$的变换为
\[
    \psi' = m_1 + m_3 \ee^{\ii 4 \pi / 3} + m_2 \ee^{- \ii 4 \pi /3},
\]
即
\begin{equation}
    \psi' = \psi^*.
    \label{eq:change-bc}
\end{equation}
类似的如果旋转中心设置在$b$点上,变换为交换$a$和$c$,即
\begin{equation}
    \psi' = \ee^{- \ii 4 \pi / 3} \psi^*,
    \label{eq:change-ca}
\end{equation}
而如果旋转中心设置在$c$点上则
\begin{equation}
    \psi' = \ee^{\ii 4 \pi / 3} \psi^*.
    \label{eq:change-ab}
\end{equation}

如果旋转中心设置在三角形中心,旋转就是$a, b, c$的轮换,因此此时$\psi$的变换和平移完全相同。
此时只有顺时针旋转\SI{120}{\degree}和逆时针旋转\SI{120}{\degree}两种可能,晶格被逆时针旋转\SI{120}{\degree}时有\eqref{eq:cyc1},晶格被顺时针旋转\SI{120}{\degree}时有\eqref{eq:cyc2}。

如果旋转中心设置在边的中心上,那么旋转角度只有\SI{180}{\degree},因此相当于交换$ab$,$bc$,$ca$中的任意两个,从而如果旋转中心在$ab$边上$\psi$的变换就是\eqref{eq:change-ab},如果旋转中心在$bc$边上那么$\psi$的变换就是\eqref{eq:change-bc},如果旋转中心在$ca$边上那么$\psi$的变换就是\eqref{eq:change-ca}。

在变换$\sigma \to - \sigma$下$\psi$的变换就是简单的$\psi' = - \psi$,即
\begin{equation}
    m' = m, \quad \theta' = \theta + \pi.
    \label{eq:z2}
\end{equation}

综上,本模型在$xy$平面上的对称性是\eqref{eq:transition}和\eqref{eq:change-bc}的任意复合,完整的对称性是$xy$平面上的对称性、$z$方向平移和自旋翻转\eqref{eq:z2}的任意复合。

\begin{figure}
    \centering
    \tikzset{every picture/.style={line width=0.75pt}} %set default line width to 0.75pt    
    \begin{tikzpicture}[x=0.75pt,y=0.75pt,yscale=-1,xscale=1]
        %uncomment if require: \path (0,235); %set diagram left start at 0, and has height of 235
        
        %Shape: Triangle [id:dp8748753639518245] 
        \draw   (323.25,113.67) -- (357,172) -- (289.5,172) -- cycle ;
        %Shape: Triangle [id:dp4663705272068355] 
        \draw   (255.75,113.67) -- (289.5,172) -- (222,172) -- cycle ;
        %Shape: Triangle [id:dp8849769619211774] 
        \draw   (289.5,55.33) -- (323.25,113.67) -- (255.75,113.67) -- cycle ;
        %Shape: Triangle [id:dp5240106325187708] 
        \draw   (459.25,113.67) -- (493,172) -- (425.5,172) -- cycle ;
        %Shape: Triangle [id:dp6098007350668448] 
        \draw   (391.75,113.67) -- (425.5,172) -- (358,172) -- cycle ;
        %Shape: Triangle [id:dp579965896101847] 
        \draw   (425.5,55.33) -- (459.25,113.67) -- (391.75,113.67) -- cycle ;
        %Straight Lines [id:da48488062655384634] 
        \draw    (323.25,113.67) -- (391.75,113.67) ;
        %Shape: Triangle [id:dp9032926223600755] 
        \draw   (358,55.33) -- (391.75,113.67) -- (324.25,113.67) -- cycle ;
        %Straight Lines [id:da3541365346986354] 
        \draw    (289.5,55.33) -- (425.5,55.33) ;
        
        %Straight Lines [id:da11061956849946131] 
        \draw    (147.5,172.33) -- (182.5,172.33) ;
        \draw [shift={(184.5,172.33)}, rotate = 180] [color={rgb, 255:red, 0; green, 0; blue, 0 }  ][line width=0.75]    (10.93,-3.29) .. controls (6.95,-1.4) and (3.31,-0.3) .. (0,0) .. controls (3.31,0.3) and (6.95,1.4) .. (10.93,3.29)   ;
        %Straight Lines [id:da21963418761371867] 
        \draw    (147.5,172.33) -- (165.5,141.07) ;
        \draw [shift={(166.5,139.33)}, rotate = 479.93] [color={rgb, 255:red, 0; green, 0; blue, 0 }  ][line width=0.75]    (10.93,-3.29) .. controls (6.95,-1.4) and (3.31,-0.3) .. (0,0) .. controls (3.31,0.3) and (6.95,1.4) .. (10.93,3.29)   ;

        % Text Node
        \draw (386,122.4) node [anchor=north west][inner sep=0.75pt]  {$a$};
        % Text Node
        \draw (353,153.4) node [anchor=north west][inner sep=0.75pt]  {$b$};
        % Text Node
        \draw (405,157) node [anchor=north west][inner sep=0.75pt]    {$c$};
        % Text Node
        \draw (298,156) node [anchor=north west][inner sep=0.75pt]    {$a$};
        % Text Node
        \draw (317,122.4) node [anchor=north west][inner sep=0.75pt]    {$c$};

        % Text Node
        \draw (169,135) node [anchor=north west][inner sep=0.75pt]    {$\vb*{e}_{2}$};
        % Text Node
        \draw (187,161.4) node [anchor=north west][inner sep=0.75pt]    {$\vb*{e}_{1}$};
        
        \end{tikzpicture}
    \caption{晶格平移示意}
    \label{fig:lattice-trans}
\end{figure}

\paragraph{(b)} 由于自旋翻转对称性,有效自由能中只能出现$\psi$的偶数阶项。显然$\abs*{\grad{\psi}}^2$是一个允许出现的项,因为它在$\psi$的任何相位变化、取共轭、正负号变化之下都不变。
由于系统中的相互作用高度局域,有效自由能中导数算符的阶数应该较低,于是以下始终假定含有导数算符的项只有$\abs*{\grad{\psi}}^2$。在不含导数算符的项中:
\begin{enumerate}
    \item $\psi^2$阶满足对称性要求的只有$\abs*{\psi}^2$;
    \item $\psi^4$阶满足对称性要求的只有$\abs*{\psi}^4$;
    \item 由于$6$含有因子$3$,而对称性并不要求有效自由能在任意的相位变化下都不变,只需要在$4 \pi / 3$的倍数的相位变化下不变即可,$\psi^6$阶项可以不是$\abs*{\psi}$的函数,只需要所有$\psi^6$阶项之和的虚部为零即可。
    满足条件的项包括$\abs*{\psi}^6$和$\psi^6 + (\psi^*)^6$。
\end{enumerate}
取到$\psi$的六阶项,并照惯例做适当的单位变换使$\abs*{\grad{\psi}}^2$项系数为$1$,有
\begin{equation}
    F = \int \dd[d]{\vb*{r}} (\abs*{\grad{\psi}}^2 + r \abs*{\psi}^2 + u \abs*{\psi}^4 + v \abs*{\psi}^6 + \frac{1}{2} v' (\psi^6 + (\psi^*)^6)).
    \label{eq:gl-with-psi}
\end{equation}
换用$m$和$\theta$为场变量,就是
\begin{equation}
    F = \int \dd[d]{\vb*{r}} (\underbrace{(\grad{m})^2 + m^2 (\grad{\theta})^2}_{\abs*{\grad{\psi}}^2} + r m^2 + u m^4 + v m^6 + v' m^6 \cos(6 \theta)).
\end{equation}

\paragraph{(c)} \eqref{eq:gl-with-psi}中有四种相互作用顶角。为了方便起见切换到动量表象下,场的变换为
\[
    \bar{\psi}(\vb*{k}) = \int \dd[3]{\vb*{r}} \ee^{\ii \vb*{k} \cdot \vb*{r}} \bar{\psi}(\vb*{r}), \quad \psi(\vb*{k}) = \int \dd[3]{\vb*{r}} \ee^{- \ii \vb*{k} \cdot \vb*{r}} \psi(\vb*{r}),
\]
费曼规则为:
\begin{itemize}
    \item 粒子线带有箭头。
    \item 有三种相互作用顶角,一种是二进二出,一种是三进三出,一种是六个粒子一起湮灭,一种是六个粒子一起创生。
    第一种顶角对应因子$-4u$,第二种顶角对应因子$-(3!)^2 v$,后两种顶角分别对应因子$-6! v'/2$;多出来的数值因子来自入射粒子线和出射粒子线和外线之间的对应方式数目。
    \item 传播子对应$1/(k^2+r)$。
    \item 对所有不确定的动量应该做积分$\int \dd[d]{\vb*{k}} / (2\pi)^d$,动量上下限由需要积掉的动量区间确定;对每个顶角应该施加动量守恒因子$(2\pi)^d \delta(\sum \vb*{k})$。
    \item 闭合粒子线不贡献对称性因子,图形的整体对称贡献对称性因子。
\end{itemize}

\begin{figure}
    \centering
    \subfigure[$\abs{\psi}^4$顶角,入射粒子线可交换,出射粒子线可交换]{
        \begin{tikzpicture}
            \begin{feynhand}
                \vertex (a) at (-1.5, -1.5) {$\vb*{k}_1$};
                \vertex (b) at (-1.5, 1.5) {$\vb*{k}_2$};
                \vertex (c) at (1.5, -1.5) {$\vb*{k}_3$};
                \vertex (d) at (1.5, 1.5) {$\vb*{k}_4$};
                \vertex (o) at (0, 0) ;
                \propag [fermion] (a) to (o);
                \propag [fermion] (b) to (o);
                \propag [fermion] (o) to (c);
                \propag [fermion] (o) to (d);
            \end{feynhand}
        \end{tikzpicture}
    }
    \subfigure[$\abs*{\psi}^6$顶角,入射粒子线可交换,出射粒子线可交换]{
        \begin{tikzpicture}
            \begin{feynhand}
                \vertex (a) at (-2.1, 0) {$\vb*{k}_2$};
                \vertex (e) at (-1.2, 1.8) {$\vb*{k}_1$};
                \vertex (f) at (-1.2, -1.8) {$\vb*{k}_3$};
                \vertex (b) at (1.2, 1.8) {$\vb*{k}_4$};
                \vertex (c) at (1.2, -1.8) {$\vb*{k}_6$};
                \vertex (d) at (2.1, 0) {$\vb*{k}_5$};
                \vertex (o) at (0, 0) ;
                \propag [fermion] (a) to (o);
                \propag [fermion] (e) to (o);
                \propag [fermion] (f) to (o);
                \propag [antfer] (b) to (o);
                \propag [antfer] (c) to (o);
                \propag [antfer] (d) to (o);
            \end{feynhand}
        \end{tikzpicture}
    }
    \vfill
    \subfigure[$\psi^6$顶角,六条粒子线可交换]{
        \begin{tikzpicture}
            \begin{feynhand}
                \vertex (a) at (-2.1, 0) {$\vb*{k}_2$};
                \vertex (e) at (-1.2, 1.8) {$\vb*{k}_1$};
                \vertex (f) at (-1.2, -1.8) {$\vb*{k}_3$};
                \vertex (b) at (1.2, 1.8) {$\vb*{k}_4$};
                \vertex (c) at (1.2, -1.8) {$\vb*{k}_6$};
                \vertex (d) at (2.1, 0) {$\vb*{k}_5$};
                \vertex (o) at (0, 0) ;
                \propag [fermion] (a) to (o);
                \propag [fermion] (e) to (o);
                \propag [fermion] (f) to (o);
                \propag [fermion] (b) to (o);
                \propag [fermion] (c) to (o);
                \propag [fermion] (d) to (o);
            \end{feynhand}
        \end{tikzpicture}
    }
    \subfigure[$\bar{\psi}^6$顶角,六条粒子线可交换]{
        \begin{tikzpicture}
            \begin{feynhand}
                \vertex (a) at (-2.1, 0) {$\vb*{k}_2$};
                \vertex (e) at (-1.2, 1.8) {$\vb*{k}_1$};
                \vertex (f) at (-1.2, -1.8) {$\vb*{k}_3$};
                \vertex (b) at (1.2, 1.8) {$\vb*{k}_4$};
                \vertex (c) at (1.2, -1.8) {$\vb*{k}_6$};
                \vertex (d) at (2.1, 0) {$\vb*{k}_5$};
                \vertex (o) at (0, 0) ;
                \propag [antfer] (a) to (o);
                \propag [antfer] (e) to (o);
                \propag [antfer] (f) to (o);
                \propag [antfer] (b) to (o);
                \propag [antfer] (c) to (o);
                \propag [antfer] (d) to (o);
            \end{feynhand}
        \end{tikzpicture}
    }
    \caption{\eqref{eq:gl-with-psi}中的相互作用顶角}
    \label{fig:vertex-intepretation}
\end{figure}

仅计算顶角个数小于等于$2$的树图和一圈图。对单顶角图,有
\begin{equation}
    \begin{aligned}
        \begin{tikzpicture}
            \begin{feynhand}
                \vertex (a) at (-1.5, 0);
                \vertex (b) at (-0.0001, 0);
                \propag [fermion, style=thick] (a) to [edge label={$\vb*{k}$}] (b) ;
                \vertex (c) at (0.0001, 0);
                \vertex (d) at (1.5, 0);
                \propag [fermion] (c) to [edge label={$\vb*{k}$}] (d) ;
                \propag [fermion] (b) to [in=45, out=135, looseness=20000, edge label={$\vb*{q}$}] (c);
            \end{feynhand}
        \end{tikzpicture} &= - 4 u \int^{\Lambda / \zeta} \frac{\dd[d]{\vb*{k}}}{(2\pi)^d} \bar{\psi}(\vb*{k}) \psi(\vb*{k}) \int_{\Lambda / \zeta}^\Lambda \frac{\dd[d]{\vb*{q}}}{(2\pi)^d} \frac{1}{q^2 + r},
    \end{aligned}
    \label{eq:correction-r}
\end{equation}
这是对$r$的修正。同样
\begin{equation}
    \begin{aligned}
        &\quad \begin{tikzpicture}
            \begin{feynhand}
                \vertex (a) at (-1.5, 0) {$\vb*{k}_1$};
                \vertex (f) at (-1, -1.4) {$\vb*{k}_2$};
                \vertex (c) at (1, -1.4) {$\vb*{k}_3$};
                \vertex (d) at (1.5, 0) {$\vb*{k}_4$};
                \vertex (o) at (0, 0) ;
                \vertex (op) at (0.0001, 0) ;
                \propag [fermion, line width=1mm] (a) to (o);
                \propag [fermion] (f) to (o);
                \propag [antfer] (c) to (o);
                \propag [antfer] (d) to (o);
                \propag [fermion] (o) to [in=50, out=130, looseness=40000, edge label={$\vb*{q}$}] (op);
            \end{feynhand}
        \end{tikzpicture} \\
        &= - \frac{1}{4} (3!)^2 v \prod_{i=1}^3 \int^{\Lambda / \zeta} \frac{\dd[d]{\vb*{k}_i}}{(2\pi)^d} \bar{\psi}(\vb*{k}_3) \bar{\psi}(\vb*{k}_1+\vb*{k}_2-\vb*{k}_3) \psi(\vb*{k}_1) \psi(\vb*{k}_2) \int_{\Lambda/\zeta}^\Lambda \frac{\dd[d]{\vb*{q}}}{(2\pi)^d} \frac{1}{q^2+r},
    \end{aligned}
    \label{eq:correction-u-1}
\end{equation}
因子$1/4$来自$\vb*{k}_1$外腿和$\vb*{k}_2$外腿可交换、$\vb*{k}_3$外腿和$\vb*{k}_4$外腿可交换。
这是对$u$的修正。其余的单顶角图或者为二圈图或者对低能过程没有修正。

二顶角树图可以通过任意画两个顶角,然后将一个入射粒子线和一个出射粒子线连接得到,圈图可以通过将两对入射粒子线和出射粒子线连接得到。
因此,含有两个六条腿的顶角的树图会给出一个十条腿的等效顶角而含有两个六条腿的顶角的一圈图会给出一个八条腿的等效顶角。
由于我们只考虑六阶及以下的项,有意义的二顶角树图中至少有一个顶角是$\abs*{\psi}^4$。

\begin{figure}
    \centering
    
    \subfigure[树图,修正$v$;由于动量守恒基本上对relevant的项没有贡献]{
        \begin{tikzpicture}
            \begin{feynhand}
                \vertex (a) at (-1.7, 0) ;
                \vertex (b) at (0.65, 1.8);
                \vertex (c) at (0.65, -1.8);
                \vertex (d) at (1.7, 0);
                \vertex (e) at (0, 0) ;
                \vertex (f) at (-2.6, 1.8);
                \vertex (g) at (-2.6, -1.8);
                \vertex (h) at (-3.4, 0);
                \propag [fermion] (a) to (e);
                \propag [fermion] (b) to (e);
                \propag [antfer] (c) to (e);
                \propag [antfer] (d) to (e);
                \propag [fermion] (f) to (a);
                \propag [fermion] (g) to (a);
                \propag [antfer] (h) to (a);
            \end{feynhand}
        \end{tikzpicture}
        \label{fig:no-contribution-tree}
    }
    \subfigure[一圈图,修正$r$;不满足动量守恒关系]{
        \begin{tikzpicture}
            \begin{feynhand}
                \vertex (a) at (0, 0);
                \vertex (b) at (0.75, 1.8);
                \vertex (c) at (0.75, -1.8);
                \vertex (d) at (2.1, 0);
                \vertex (e) at (-1.5, 0);
                \vertex (f) at (-1.499, 0);
                \vertex (g) at (-3, 0);
                \propag[fermion] (a) to (b);
                \propag[fermion] (a) to (c);
                \propag[fermion] (a) to (d);
                \propag[fermion] (f) to (a);
                \propag[fermion] (g) to (e);
                \propag[fermion] (e) to [in=45, out=135, looseness=8000] (f);
            \end{feynhand}
        \end{tikzpicture}
        \label{fig:one-loop-removed}
    }
    \subfigure[一圈图,修正$r$]{
        \begin{tikzpicture}
            \begin{feynhand}
                \vertex (a) at (1, 0);
                \vertex (b) at (1.75, 1.8);
                \vertex (c) at (1.75, -1.8);
                \vertex (d) at (-1, 0);
                \vertex (e) at (-1.75, 1.8);
                \vertex (f) at (-1.75, -1.8);
                \propag[fermion] (a) to (b);
                \propag[fermion] (a) to (c);
                \propag[fermion] (e) to (d);
                \propag[fermion] (f) to (d);
                \propag[fermion] (d) to [in=120, out=60, looseness=2] (a);
                \propag[fermion] (d) to [in=240, out=300, looseness=2] (a);
            \end{feynhand}
        \end{tikzpicture}
    }
    \subfigure[一圈图,修正$r$]{
        \begin{tikzpicture}
            \begin{feynhand}
                \vertex (a) at (1, 0);
                \vertex (b) at (1.75, 1.8);
                \vertex (c) at (1.75, -1.8);
                \vertex (d) at (-1, 0);
                \vertex (e) at (-1.75, 1.8);
                \vertex (f) at (-1.75, -1.8);
                \propag[fermion] (b) to (a);
                \propag[fermion] (a) to (c);
                \propag[fermion] (e) to (d);
                \propag[fermion] (d) to (f);
                \propag[fermion] (d) to [in=120, out=60, looseness=2] (a);
                \propag[antfer] (d) to [in=240, out=300, looseness=2] (a);
            \end{feynhand}
        \end{tikzpicture}
    }
    \caption{由$\abs*{\psi}^4$项形成的二顶角图}
    \label{fig:second-order-correction-psi4}
\end{figure}

两个$\abs*{\psi}^4$顶角形成的图为\autoref{fig:second-order-correction-psi4},其中\autoref{fig:one-loop-removed}不满足动量守恒关系,因为一个低动量的粒子被第一个顶角散射为了一个高动量的粒子。(将\autoref{fig:one-loop-removed}中的所有粒子线调转方向可以得到一个不同的图,但是这个图同样没有贡献)
对$v$的修正为
\[
    \begin{aligned}
        &\quad \begin{tikzpicture}
            \begin{feynhand}
                \vertex (a) at (-1.5, 0);
                \vertex (b) at (0.5, 1.2) {$\vb*{k}_4$};
                \vertex (c) at (0.5, -1.2) {$\vb*{k}_5$};
                \vertex (d) at (1.5, 0) {$\vb*{k}_1$};
                \vertex (e) at (0, 0) ;
                \vertex (f) at (-2.1, 1.2) {$\vb*{k}_2$};
                \vertex (g) at (-2.1, -1.2) {$\vb*{k}_3$};
                \vertex (h) at (-3, 0) {$\vb*{k}_6$};
                \propag [fermion] (a) to (e);
                \propag [antfer] (b) to (e);
                \propag [antfer] (c) to (e);
                \propag [fermion] (d) to (e);
                \propag [fermion] (f) to (a);
                \propag [fermion] (g) to (a);
                \propag [antfer] (h) to (a);
            \end{feynhand}
        \end{tikzpicture} \\
        &= \frac{1}{4} (- 4 u)^2 \prod_{i=1}^5 \int^{\Lambda / \zeta} \frac{\dd[d]{\vb*{k}_i}}{(2\pi)^d} \bar{\psi}(\vb*{k}_4) \bar{\psi}(\vb*{k}_5) \bar{\psi}(\sum \vb*{k} - \vb*{k}_4 - \vb*{k}_5) \psi(\vb*{k}_1) \psi(\vb*{k}_2) \psi(\vb*{k}_3) \\
        & \quad \quad \times \frac{1}{(\vb*{k}_4 + \vb*{k}_5 - \vb*{k}_1)^2+r} \\
        &\approx \frac{1}{4} (-4u)^2 \frac{1}{r} \prod_{i=1}^5 \int^{\Lambda / \zeta} \frac{\dd[d]{\vb*{k}_i}}{(2\pi)^d} \bar{\psi}(\vb*{k}_4) \bar{\psi}(\vb*{k}_5) \bar{\psi}(\sum \vb*{k} - \vb*{k}_4 - \vb*{k}_5) \psi(\vb*{k}_1) \psi(\vb*{k}_2) \psi(\vb*{k}_3),
    \end{aligned}
\]
其中$1/4$的因子来自$\vb*{k}_2$外腿和$\vb*{k}_3$外腿可交换、$\vb*{k}_4$外腿和$\vb*{k}_5$外腿可交换。
第一个等号后面给出的项实际上有动量依赖,但是既然只有没有动量依赖的$\abs*{\psi}^6$项才是重要的,我们直接忽略这个动量依赖;但是这相当于假定了中间态粒子动量为$0$,而中间态应该在高能自由度上,因此这个图没有修正。
图(c)给出的$u$的修正为
\begin{equation}
    \begin{aligned}
        &\quad \begin{tikzpicture}
            \begin{feynhand}
                \vertex (a) at (0.6, 0);
                \vertex (b) at (1.3, 1.3) {$\vb*{k}_3$};
                \vertex (c) at (1.3, -1.3) {$\vb*{k}_4$};
                \vertex (d) at (-0.6, 0);
                \vertex (e) at (-1.3, 1.3) {$\vb*{k}_1$};
                \vertex (f) at (-1.3, -1.3) {$\vb*{k}_2$};
                \propag[fermion] (a) to (b);
                \propag[fermion] (a) to (c);
                \propag[fermion] (e) to (d);
                \propag[fermion] (f) to (d);
                \propag[fermion] (d) to [in=120, out=60, looseness=2, edge label={${\vb*{q}}$}] (a);
                \propag[fermion] (d) to [in=240, out=300, looseness=2, edge label={$\vb*{p}$}] (a);
            \end{feynhand}
        \end{tikzpicture} \\
        &= \frac{1}{2^3} (-4 u)^2 \prod_{i=1}^3 \int_{\Lambda}^{0} \frac{\dd[d]{\vb*{k}_i}}{(2\pi)^d} \bar{\psi}(\vb*{k}_3) \bar{\psi}(\sum \vb*{k} - \vb*{k}_3) \psi(\vb*{k}_1) \psi(\vb*{k}_2) \int_{\Lambda / \zeta}^{\Lambda} \frac{\dd[d]{\vb*{q}}}{(2\pi)^d} \frac{1}{q^2 + r} \frac{1}{(\vb*{k}_1 + \vb*{k}_2 - \vb*{q})^2 + r} \\
        &\approx \frac{1}{2^3} (-4 u)^2 \prod_{i=1}^3 \int_{\Lambda}^{0} \frac{\dd[d]{\vb*{k}_i}}{(2\pi)^d} \bar{\psi}(\vb*{k}_3) \bar{\psi}(\sum \vb*{k} - \vb*{k}_3) \psi(\vb*{k}_1) \psi(\vb*{k}_2) \int_{\Lambda / \zeta}^{\Lambda} \frac{\dd[d]{\vb*{q}}}{(2\pi)^d} \frac{1}{(q^2 + r)^2},
    \end{aligned}
    \label{eq:correction-u-2}
\end{equation}
这里略去了对$\vb*{k}_1$和$\vb*{k}_2$的全部依赖,因为只有没有动量依赖的$\abs*{\psi}^4$项才是重要的。
图(d)给出的修正和图(c)基本一致,只是对称性因子不同:
\begin{equation}
    \begin{aligned}
        &\quad \begin{tikzpicture}
            \begin{feynhand}
                \vertex (a) at (0.6, 0);
                \vertex (b) at (1.3, 1.3) {$\vb*{k}_2$};
                \vertex (c) at (1.3, -1.3) {$\vb*{k}_4$};
                \vertex (d) at (-0.6, 0);
                \vertex (e) at (-1.3, 1.3) {$\vb*{k}_1$};
                \vertex (f) at (-1.3, -1.3) {$\vb*{k}_3$};
                \propag[fermion] (b) to (a);
                \propag[fermion] (a) to (c);
                \propag[fermion] (e) to (d);
                \propag[fermion] (d) to (f);
                \propag[fermion] (d) to [in=120, out=60, looseness=2, edge label={${\vb*{q}}$}] (a);
                \propag[antfer] (d) to [in=240, out=300, looseness=2, edge label={$\vb*{p}$}] (a);
            \end{feynhand}
        \end{tikzpicture} \\
        &= (-4 u)^2 \prod_{i=1}^3 \int_{\Lambda}^{0} \frac{\dd[d]{\vb*{k}_i}}{(2\pi)^d} \bar{\psi}(\vb*{k}_3) \bar{\psi}(\sum \vb*{k} - \vb*{k}_3) \psi(\vb*{k}_1) \psi(\vb*{k}_2) \int_{\Lambda / \zeta}^{\Lambda} \frac{\dd[d]{\vb*{q}}}{(2\pi)^d} \frac{1}{q^2 + r} \frac{1}{(\vb*{k}_1 + \vb*{k}_2 - \vb*{q})^2 + r} \\
        &\approx (-4 u)^2 \prod_{i=1}^3 \int_{\Lambda}^{0} \frac{\dd[d]{\vb*{k}_i}}{(2\pi)^d} \bar{\psi}(\vb*{k}_3) \bar{\psi}(\sum \vb*{k} - \vb*{k}_3) \psi(\vb*{k}_1) \psi(\vb*{k}_2) \int_{\Lambda / \zeta}^{\Lambda} \frac{\dd[d]{\vb*{q}}}{(2\pi)^d} \frac{1}{(q^2 + r)^2}.
    \end{aligned}
    \label{eq:correction-u-3}
\end{equation}

\begin{figure}
    \centering

    \subfigure[一圈图,修正$v$;不满足动量守恒关系]{
        \begin{tikzpicture}
            \begin{feynhand}
                \vertex (a) at (0, 0);
                \vertex (b) at (0.75, 1.8);
                \vertex (c) at (0.75, -1.8);
                \vertex (d) at (2.1, 0);
                \vertex (e) at (-1.8, 0);
                \vertex (f) at (-1.799, 0);
                \vertex (g) at (-3, 0);
                \vertex (h) at (-0.75, 1.8);
                \vertex (i) at (-0.75, -1.8);
                \propag[fermion] (a) to (b);
                \propag[fermion] (a) to (c);
                \propag[fermion] (h) to (a);
                \propag[fermion] (i) to (a);
                \propag[fermion] (a) to (d);
                \propag[fermion] (f) to (a);
                \propag[fermion] (g) to (e);
                \propag[fermion] (e) to [in=45, out=135, looseness=8000] (f);
            \end{feynhand}
        \end{tikzpicture}
    }
    \subfigure[一圈图,修正$v$,作用粒子-空穴变换还可以得到一个独立的图]{
        \begin{tikzpicture}
            \begin{feynhand}
                \vertex (a) at (-1.5, 0) ;
                \vertex (b) at (-0.75, -1.8);
                \vertex (c) at (0.75, -1.8);
                \vertex (d) at (2.1, 0);
                \vertex (e) at (0, 0) ;
                \vertex (ep) at (0.001, 0);
                \vertex (f) at (-2.5, 1.8);
                \vertex (g) at (-2.5, -1.8);
                \vertex (h) at (-3, 0);
                \propag [fermion] (e) to [in=45, out=135, looseness=10000] (ep);
                \propag [fermion] (a) to (e);
                \propag [fermion] (b) to (e);
                \propag [antfer] (c) to (e);
                \propag [antfer] (d) to (e);
                \propag [fermion] (f) to (a);
                \propag [fermion] (g) to (a);
                \propag [antfer] (h) to (a);
            \end{feynhand}
        \end{tikzpicture}
    }
    \subfigure[一圈图,修正$v$,作用粒子-空穴变换还可以得到一个独立的图]{
        \begin{tikzpicture}
            \begin{feynhand}
                \vertex (a) at (-1.5, 0) ;
                \vertex (b) at (-0.75, -1.8);
                \vertex (c) at (0.75, -1.8);
                \vertex (i) at (0.75, 1.8);
                \vertex (d) at (2.1, 0);
                \vertex (e) at (0, 0) ;
                \vertex (g) at (-2.5, -1.8);
                \vertex (h) at (-3, 0);
                \propag [antfer] (e) to [in=60, out=120, looseness=2] (a);
                \propag [fermion] (e) to (i);
                \propag [fermion] (a) to (e);
                \propag [fermion] (b) to (e);
                \propag [antfer] (c) to (e);
                \propag [antfer] (d) to (e);
                \propag [fermion] (g) to (a);
                \propag [fermion] (h) to (a);
            \end{feynhand}
        \end{tikzpicture}
    }
    \subfigure[一圈图,修正$v$]{
        \begin{tikzpicture}
            \begin{feynhand}
                \vertex (a) at (-1.5, 0) ;
                \vertex (b) at (-0.75, -1.8);
                \vertex (c) at (0.75, -1.8);
                \vertex (i) at (0.75, 1.8);
                \vertex (d) at (2.1, 0);
                \vertex (e) at (0, 0) ;
                \vertex (g) at (-2.5, -1.8);
                \vertex (h) at (-3, 0);
                \propag [fermion] (e) to [in=60, out=120, looseness=2] (a);
                \propag [fermion] (e) to (i);
                \propag [fermion] (a) to (e);
                \propag [fermion] (b) to (e);
                \propag [antfer] (c) to (e);
                \propag [fermion] (d) to (e);
                \propag [fermion] (g) to (a);
                \propag [antfer] (h) to (a);
            \end{feynhand}
        \end{tikzpicture}
    }
    \caption{由$\abs*{\psi}^4$项和$\abs*{\psi}^6$形成的二顶角图}
    \label{fig:second-order-correction-psi4andpsi6}
\end{figure}

下面考虑$\abs*{\psi}^4$项和$\abs*{\psi}^6$形成的二顶角图。这些图如果是树图将会导致$\psi^8$阶的项,所以我们只考虑一圈图,展示如\autoref{fig:second-order-correction-psi4andpsi6}。
实际上将\autoref{fig:second-order-correction-psi4andpsi6}中的图中的所有箭头都倒转可以得到另一些图,但是因为只需要做粒子-空穴变换就可以得到这些图,无需单独考虑它们。
其中,图(a)是不满足动量守恒的,因为最左端的入射粒子是低动量的但是被第一个相互作用顶角散射到了一个高动量模式上。

图(b)会给出$v$的修正的另一个贡献:
\[
    \begin{aligned}
        &\quad \begin{tikzpicture}
            \begin{feynhand}
                \vertex (a) at (-1.2, 0) ;
                \vertex (b) at (-0.75, -1.3) {$\vb*{k}_3$};
                \vertex (c) at (0.75, -1.3) {$\vb*{k}_6$};
                \vertex (d) at (1.5, 0) {$\vb*{k}_4$};
                \vertex (e) at (0, 0) ;
                \vertex (ep) at (0.001, 0);
                \vertex (f) at (-2.0, 1.3) {$\vb*{k}_2$};
                \vertex (g) at (-2.0, -1.3) {$\vb*{k}_1$};
                \vertex (h) at (-2.7, 0) {$\vb*{k}_5$};
                \propag [fermion] (e) to [in=60, out=120, looseness=5000, edge label={$\vb*{q}$}] (ep);
                \propag [fermion] (a) to (e);
                \propag [fermion] (b) to (e);
                \propag [antfer] (c) to (e);
                \propag [antfer] (d) to (e);
                \propag [fermion] (f) to (a);
                \propag [fermion] (g) to (a);
                \propag [antfer] (h) to (a);
            \end{feynhand}
        \end{tikzpicture} \\
        &= \frac{1}{4} (-4 u) (- (3!)^2 v) \prod_{i=1}^5 \int^{\Lambda / \zeta} \frac{\dd[d]{\vb*{k}_i}}{(2\pi)^d} \bar{\psi}(\sum \vb*{k} - \vb*{k}_4 - \vb*{k}_5) \bar{\psi}(\vb*{k}_4) \bar{\psi}(\vb*{k}_5) \psi(\vb*{k}_1) \psi(\vb*{k}_2) \psi(\vb*{k}_3) \\
        &\quad \quad \times \frac{1}{(\vb*{k}_1 + \vb*{k}_2 - \vb*{k}_5)^2 + r} \int_{\Lambda / \zeta}^\Lambda \frac{\dd[d]{\vb*{q}}}{(2\pi)^d} \frac{1}{q^2 + r} \\
        &\approx \frac{1}{4} (-4 u) (- (3!)^2 v) \frac{1}{r} \prod_{i=1}^5 \int^{\Lambda / \zeta} \frac{\dd[d]{\vb*{k}_i}}{(2\pi)^d} \bar{\psi}(\sum \vb*{k} - \vb*{k}_4 - \vb*{k}_5) \bar{\psi}(\vb*{k}_4) \bar{\psi}(\vb*{k}_5) \psi(\vb*{k}_1) \psi(\vb*{k}_2) \psi(\vb*{k}_3) \\
        &\quad \quad \times \int_{\Lambda / \zeta}^\Lambda \frac{\dd[d]{\vb*{q}}}{(2\pi)^d} \frac{1}{q^2 + r}.
    \end{aligned}
\]
在最后的对$v$的修正中上式应该乘以$2$,因为将上式中所有箭头颠倒过来可以得到另一个图,它有一样的对$v$的修正。
然而,实际上上式和\autoref{fig:no-contribution-tree}类似,由于我们仅保留对relevant的项的修正,我们要求$\vb*{k}_2+\vb*{k}_1-\vb*{k}_5$为零,但是这和中间态粒子应该是高能模式矛盾。因此上式实际上没有贡献。
对图(c),有
\begin{equation}
    \begin{aligned}
        &\quad \begin{tikzpicture}
            \begin{feynhand}
                \vertex (a) at (-1.5, 0) ;
                \vertex (b) at (-0.6, -1.3) {$\vb*{k}_3$};
                \vertex (c) at (0.6, -1.3) {$\vb*{k}_6$};
                \vertex (i) at (0.6, 1.3) {$\vb*{k}_4$};
                \vertex (d) at (1.6, 0) {$\vb*{k}_5$};
                \vertex (e) at (0, 0) ;
                \vertex (g) at (-1.8, -1.2) {$\vb*{k}_2$};
                \vertex (h) at (-2.6, 0) {$\vb*{k}_1$};
                \propag [antfer] (e) to [in=70, out=110, looseness=2, edge label={$\vb*{q}$}] (a);
                \propag [fermion] (e) to (i);
                \propag [fermion] (a) to [edge label={$\vb*{p}$}] (e);
                \propag [fermion] (b) to (e);
                \propag [antfer] (c) to (e);
                \propag [antfer] (d) to (e);
                \propag [fermion] (g) to (a);
                \propag [fermion] (h) to (a);
            \end{feynhand}
        \end{tikzpicture} \\
        &= \frac{1}{2 \times 2 \times 3!} (-4 u)(-(3!)^2 v) \prod_{i=1}^5 \int^{\Lambda / \zeta} \frac{\dd[d]{\vb*{k}_i}}{(2\pi)^d} \bar{\psi}(\sum \vb*{k} - \vb*{k}_4 - \vb*{k}_5) \bar{\psi}(\vb*{k}_4) \bar{\psi}(\vb*{k}_5) \psi(\vb*{k}_1) \psi(\vb*{k}_2) \psi(\vb*{k}_3) \\
        &\quad \quad \times \int_{\Lambda / \zeta}^\Lambda \frac{\dd[d]{\vb*{q}}}{(2\pi)^d} \frac{1}{q^2 + r} \frac{1}{(\vb*{k}_1 + \vb*{k}_2 - \vb*{q})^2 + r} \\
        &\approx \frac{1}{2 \times 2 \times 3!} (-4 u)(-(3!)^2 v) \prod_{i=1}^5 \int^{\Lambda / \zeta} \frac{\dd[d]{\vb*{k}_i}}{(2\pi)^d} \bar{\psi}(\sum \vb*{k} - \vb*{k}_4 - \vb*{k}_5) \bar{\psi}(\vb*{k}_4) \bar{\psi}(\vb*{k}_5) \psi(\vb*{k}_1) \psi(\vb*{k}_2) \psi(\vb*{k}_3) \\
        &\quad \quad \times \int_{\Lambda / \zeta}^\Lambda \frac{\dd[d]{\vb*{q}}}{(2\pi)^d} \frac{1}{(q^2 + r)^2}.
    \end{aligned}
    \label{eq:correction-v-2}
\end{equation}
同样在最后的对$v$的修正中上式应该乘以$2$,因为将上式中所有箭头颠倒过来可以得到另一个图,它有一样的对$v$的修正。
图(d)带来的修正和图(c)大体上是一样的,但是对称性因子有变化,为
\begin{equation}
    \begin{aligned}
        &\quad \frac{1}{2 \times 2} (-4 u)(-(3!)^2 v) \prod_{i=1}^5 \int^{\Lambda / \zeta} \frac{\dd[d]{\vb*{k}_i}}{(2\pi)^d} \bar{\psi}(\sum \vb*{k} - \vb*{k}_4 - \vb*{k}_5) \bar{\psi}(\vb*{k}_4) \bar{\psi}(\vb*{k}_5) \psi(\vb*{k}_1) \psi(\vb*{k}_2) \psi(\vb*{k}_3) \\
        &\quad \quad \times \int_{\Lambda / \zeta}^\Lambda \frac{\dd[d]{\vb*{q}}}{(2\pi)^d} \frac{1}{q^2 + r} \frac{1}{(\vb*{k}_1 + \vb*{k}_2 - \vb*{q})^2 + r} \\
        &\approx \frac{1}{2 \times 2} (-4 u)(-(3!)^2 v) \prod_{i=1}^5 \int^{\Lambda / \zeta} \frac{\dd[d]{\vb*{k}_i}}{(2\pi)^d} \bar{\psi}(\sum \vb*{k} - \vb*{k}_4 - \vb*{k}_5) \bar{\psi}(\vb*{k}_4) \bar{\psi}(\vb*{k}_5) \psi(\vb*{k}_1) \psi(\vb*{k}_2) \psi(\vb*{k}_3) \\
        &\quad \quad \times \int_{\Lambda / \zeta}^\Lambda \frac{\dd[d]{\vb*{q}}}{(2\pi)^d} \frac{1}{(q^2 + r)^2}.
    \end{aligned}
    \label{eq:correction-v-3}
\end{equation}

\begin{figure}
    \centering
    \subfigure[一圈图,修正$v'$;不满足动量守恒关系]{
        \begin{tikzpicture}
            \begin{feynhand}
                \vertex (a) at (0, 0);
                \vertex (b) at (0.75, 1.8);
                \vertex (c) at (0.75, -1.8);
                \vertex (d) at (2.1, 0);
                \vertex (e) at (-1.8, 0);
                \vertex (f) at (-1.799, 0);
                \vertex (g) at (-3, 0);
                \vertex (h) at (-0.75, 1.8);
                \vertex (i) at (-0.75, -1.8);
                \propag[fermion] (b) to (a);
                \propag[fermion] (c) to (a);
                \propag[fermion] (h) to (a);
                \propag[fermion] (i) to (a);
                \propag[fermion] (d) to (a);
                \propag[fermion] (f) to (a);
                \propag[fermion] (g) to (e);
                \propag[fermion] (e) to [in=45, out=135, looseness=8000] (f);
            \end{feynhand}
        \end{tikzpicture}
    }
    \subfigure[一圈图,修正$v'$,作用粒子-空穴变换还可以得到一个独立的图]{
        \begin{tikzpicture}
            \begin{feynhand}
                \vertex (a) at (-1.5, 0) ;
                \vertex (b) at (-0.75, -1.8);
                \vertex (c) at (0.75, -1.8);
                \vertex (i) at (0.75, 1.8);
                \vertex (d) at (2.1, 0);
                \vertex (e) at (0, 0) ;
                \vertex (g) at (-2.5, -1.8);
                \vertex (h) at (-3, 0);
                \propag [antfer] (e) to [in=60, out=120, looseness=2] (a);
                \propag [fermion] (i) to (e);
                \propag [fermion] (a) to (e);
                \propag [fermion] (b) to (e);
                \propag [fermion] (c) to (e);
                \propag [fermion] (d) to (e);
                \propag [fermion] (g) to (a);
                \propag [fermion] (h) to (a);
            \end{feynhand}
        \end{tikzpicture}
    }
    \caption{由$\psi^6$项和$\abs*{\psi}^4$形成的二顶角图}
    \label{fig:second-order-correction-psi4andvp}
\end{figure}

由于$\psi^6$项和$\bar{\psi}^6$项之间具有对称性,并且两者对对方的一圈重整化群流没有贡献,只需要算其中一个就可以得知另一个。\autoref{fig:second-order-correction-psi4andvp}给出了可能的图,其中(a)由于前述的低动量电子被散射到高动量模式上没有任何贡献,只需要计算(b),它给出
\begin{equation}
    \begin{aligned}
        &\quad \begin{tikzpicture}
            \begin{feynhand}
                \vertex (a) at (-1.5, 0) ;
                \vertex (b) at (-0.6, -1.3) {$\vb*{k}_3$};
                \vertex (c) at (0.6, -1.3) {$\vb*{k}_6$};
                \vertex (i) at (0.6, 1.3) {$\vb*{k}_4$};
                \vertex (d) at (1.6, 0) {$\vb*{k}_5$};
                \vertex (e) at (0, 0) ;
                \vertex (g) at (-1.8, -1.2) {$\vb*{k}_2$};
                \vertex (h) at (-2.6, 0) {$\vb*{k}_1$};
                \propag [antfer] (e) to [in=70, out=110, looseness=2, edge label={$\vb*{q}$}] (a);
                \propag [fermion] (i) to (e);
                \propag [fermion] (a) to [edge label={$\vb*{p}$}] (e);
                \propag [fermion] (b) to (e);
                \propag [antfer] (e) to (c);
                \propag [antfer] (e) to (d);
                \propag [fermion] (g) to (a);
                \propag [fermion] (h) to (a);
            \end{feynhand}
        \end{tikzpicture} \\
        &= \frac{1}{2 \times 2 \times 4!} (-4 u)(-6! v'/ 2) \prod_{i=1}^5 \int^{\Lambda / \zeta} \frac{\dd[d]{\vb*{k}_i}}{(2\pi)^d} {\psi}(\sum \vb*{k} - \vb*{k}_4 - \vb*{k}_5) {\psi}(\vb*{k}_4) {\psi}(\vb*{k}_5) \psi(\vb*{k}_1) \psi(\vb*{k}_2) \psi(\vb*{k}_3) \\
        &\quad \quad \times \int_{\Lambda / \zeta}^\Lambda \frac{\dd[d]{\vb*{q}}}{(2\pi)^d} \frac{1}{q^2 + r} \frac{1}{(\vb*{k}_1 + \vb*{k}_2 - \vb*{q})^2 + r} \\
        &\approx \frac{1}{2 \times 2 \times 4!} (-4 u)(-6! v'/2) \prod_{i=1}^5 \int^{\Lambda / \zeta} \frac{\dd[d]{\vb*{k}_i}}{(2\pi)^d} {\psi}(\sum \vb*{k} - \vb*{k}_4 - \vb*{k}_5) {\psi}(\vb*{k}_4) {\psi}(\vb*{k}_5) \psi(\vb*{k}_1) \psi(\vb*{k}_2) \psi(\vb*{k}_3) \\
        &\quad \quad \times \int_{\Lambda / \zeta}^\Lambda \frac{\dd[d]{\vb*{q}}}{(2\pi)^d} \frac{1}{(q^2 + r)^2}.
    \end{aligned}
    \label{eq:correction-vp}
\end{equation}
在最后对$v'$的修正中上式应该乘以2,因为将所有箭头颠倒过来可以得到另一个图,它给出的修正是完全一样的。

做量纲分析,取坐标的倒数为$1$,则有
\[
    \left[ \int \dd[d]{\vb*{r}} (\grad{\psi})^2 \right] = 1,
\]
即
\begin{equation}
    [\psi] = \frac{d-2}{2}.
\end{equation}
从而根据自由能的形式,有
\begin{equation}
    [r] = 2, \quad [u] = 4 - d, \quad [v] = [v'] = 6 - 2d.
\end{equation}
这样就可以写出参数跑动的方式。对$r$,根据\eqref{eq:correction-r}有
\begin{equation}
    r(\zeta) = \zeta^2 \left( r + 4 u \int_{\Lambda/\zeta}^\Lambda \frac{\dd[d]{\vb*{q}}}{(2\pi)^d} \frac{1}{q^2+r} \right).
\end{equation}
对$u$,根据\eqref{eq:correction-u-1},\eqref{eq:correction-u-2}和\eqref{eq:correction-u-3}有
\begin{equation}
    \begin{aligned}
        u(\zeta) &= \zeta^{4-d} \Bigl( u + \frac{1}{4} (3!)^2 v \int_{\Lambda / \zeta}^\Lambda \frac{\dd[d]{\vb*{q}}}{(2\pi)^d} \frac{1}{q^2 + r} - \frac{1}{2^3} (-4u)^2 \int_{\Lambda / \zeta}^\Lambda \frac{\dd[d]{\vb*{q}}}{(2\pi)^d} \frac{1}{(q^2 + r)^2} \\
        &- (-4u)^2 \int_{\Lambda / \zeta}^\Lambda \frac{\dd[d]{\vb*{q}}}{(2\pi)^d} \frac{1}{(q^2 + r)^2} \Bigl) \\
        &= \zeta^{4-d} \left( u + 9 v \int_{\Lambda / \zeta}^\Lambda \frac{\dd[d]{\vb*{q}}}{(2\pi)^d} \frac{1}{q^2 + r} - 18 u^2 \int_{\Lambda / \zeta}^\Lambda \frac{\dd[d]{\vb*{q}}}{(2\pi)^d} \frac{1}{(q^2 + r)^2} \right).
    \end{aligned}
\end{equation}
对$v$的修正来自\eqref{eq:correction-v-2},\eqref{eq:correction-v-3},为
\begin{equation}
    \begin{aligned}
        v(\zeta) &= \zeta^{6-2d} \Bigl( v - 2 \times \frac{1}{2^2 \times 3!} (-4 u)(-(3!)^2 v) \int^\Lambda_{\Lambda/\zeta} \frac{\dd[d]{\vb*{q}}}{(2\pi)^d} \frac{1}{(q^2+r)^2} \\
        &- \frac{1}{2^2} (-4 u)(-(3!)^2 v) \int^\Lambda_{\Lambda/\zeta} \frac{\dd[d]{\vb*{q}}}{(2\pi)^d} \frac{1}{(q^2+r)^2} \Bigl) \\
        &= \zeta^{6-2d} \left( v - 48 u v \int^\Lambda_{\Lambda/\zeta} \frac{\dd[d]{\vb*{q}}}{(2\pi)^d} \frac{1}{(q^2+r)^2} \right).
    \end{aligned}
\end{equation}
对$v'$的修正来自\eqref{eq:correction-vp},为
\begin{equation}
    \begin{aligned}
        v'(\zeta) &= \zeta^{6-2d} \left( v' - 2 \times \frac{1}{2 \times 2 \times 4!} (-4 u)(-6! v'/2) \int_{\Lambda / \zeta}^\Lambda \frac{\dd[d]{\vb*{q}}}{(2\pi)^d} \frac{1}{(q^2 + r)^2} \right) \\
        &= \zeta^{6-2d} \left( v' - 30 u v' \int_{\Lambda / \zeta}^\Lambda \frac{\dd[d]{\vb*{q}}}{(2\pi)^d} \frac{1}{(q^2 + r)^2} \right).
    \end{aligned}
\end{equation}

以上方程都是在$\zeta$接近$1$的近似下成立的,根据它们计算$\beta$函数可以得到(下面已经做了适当的单位变换,使得$\Lambda=1$)
\begin{equation}
    \beta(r) = \dv{r}{\ln \zeta} = 2 r + 4u  \frac{S_{d-1}}{(2\pi)^d (1 + r)},
\end{equation}
\begin{equation}
    \beta(u) = \dv{u}{\ln \zeta} = (4-d) u + 9v \frac{S_{d-1}}{(2\pi)^d (1+r)} - 18 u^2 \frac{S_{d-1}}{(2\pi)^d (1 + r)^2}, 
\end{equation}
\begin{equation}
    \beta(v) = \dv{v}{\ln \zeta} = (6-2d) v - 48 uv \frac{S_{d-1}}{(2\pi)^d (1+r)^2}, 
\end{equation}
\begin{equation}
    \beta(v') = \dv{v'}{\ln \zeta} = (6-2d) v' - 30 uv' \frac{S_{d-1}}{(2\pi)^d (1+r)^2}.
\end{equation}
以上四个方程给出了重整化群方程,其中$S_{d-1}$表示半径为$1$的$d-1$维球面的表面积。

下面做$4-\epsilon$展开以求解不动点。令所有$\beta$函数为零,注意到$u=r=v=v'=0$是一个不动点,Wilson-Fisher不动点中各个参数的量级均应为$\epsilon$阶或者更高阶小量。
计算可知在$d=4-\epsilon$时
\[
    \frac{S_{d-1}}{(2\pi)^d} = \frac{1}{(2\pi)^d} \frac{2 \pi^{d/2}}{\Gamma(d/2)} = \frac{1}{8 \pi^2} + \bigO(\epsilon),
\]
既然$S^{d-1}$因子在重整化群方程中都是和参数乘在一起的,而参数至少为$\epsilon$阶,可以直接略去上式中的小量部分,从而得到方程组
\[
    \begin{aligned}
        &2 r + 4u \frac{1}{8\pi^2} \frac{1}{1+r} = 0, \\
        &\epsilon u + 9 v \frac{1}{8\pi^2} \frac{1}{1+r} - 18 u^2 \frac{1}{8\pi^2} \frac{1}{(1+r)^2} = 0, \\
        &(2\epsilon-2) v - 48 uv \frac{1}{8\pi^2} \frac{1}{(1+r)^2} = 0, \\
        &(2\epsilon-2) v' - 30 u v' \frac{1}{8\pi^2} \frac{1}{(1+r)^2} = 0.
    \end{aligned}
\]
解之,得到三组解,一组是所有参数都是零的解,这我们已经知道了,一组是有限大小的解,这组解应该舍去,因为这不是Wilson-Fisher不动点。
Wilson-Fisher不动点,取到$\epsilon$的一阶项,为
\begin{equation}
    r^\star = - \frac{\epsilon}{9}, \quad u^\star = \frac{4}{9} \pi^2 \epsilon, \quad v = v' = 0.
\end{equation}
下面考虑此不动点附近的重整化群流,设$r = r^\star + \var{r}$,等等,取到$\epsilon$的一阶项,有
\[
    \begin{aligned}
        \beta(r) &= \left( 2 - \frac{2\epsilon}{9} \right) \var{r} + \left( \frac{1}{2\pi^2} + \frac{\epsilon}{18 \pi^2} \right) \var{u}, \\
        \beta(u) &= - \epsilon \var{u} + \frac{9 + \epsilon}{8 \pi^2} \var{v}, \\
        \beta(v) &= - \left( 2 + \frac{2 \epsilon}{3} \right) \var{v}, \\
        \beta(v') &= - \left( 2 - \frac{\epsilon}{3} \right) \var{v'},
    \end{aligned}
\]
因此$v$和$v'$都是irrelevant的;但$v'$是重整化群的本征方向而$v$不是。对以上方程右边的系数矩阵计算本征值,得到以下本征方向:
\[
    \var{r} \sim \zeta^{2 - \frac{2}{9} \epsilon}, \quad - \frac{9 + \epsilon}{2\pi^2 (18 + 7 \epsilon)} \var{r} + \var{u} \sim \zeta^{-\epsilon}, \quad - \frac{3(9+\epsilon)}{64\pi^4(\epsilon-6)} \var{r} - \frac{3(9+\epsilon)}{8\pi^2(\epsilon-6)} \var{u} + \var{v} \sim \zeta^{-2-\frac{2}{3} \epsilon}.
\]
因此可以看出我们有
\begin{equation}
    \var{r} \sim \zeta^{2-\frac{2}{9}\epsilon}, \quad \var{u} \sim \zeta^{2 - \frac{2}{9}\epsilon}.
\end{equation}
因此,只要$\epsilon$充分小,使得$-\epsilon<0$,$-2-2\epsilon/3<0$,那么$u$和$r$是relevant的而其它项都不是relevant的。
$\epsilon=1$显然满足这一条件,因此在三维情况下,红外不动点处的有效理论应该形如
\begin{equation}
    F = \abs*{\grad{\psi}}^2 + r \abs*{\psi}^2 + u \abs*{\psi}^4.
\end{equation}
由于$\psi$是一个复场,设其实部和虚部分别为$\psi_1$和$\psi_2$,代入上式,得到
\begin{equation}
    F = \int \dd[d]{\vb*{r}} ((\grad{\psi_1})^2 + (\grad{\psi_2})^2 + r (\psi_1^2 + \psi_2^2) + u (\psi_1^2 + \psi_2^2)^2),
\end{equation}
这是一个$O(2)$模型,因此本文分析的模型\eqref{eq:gl-with-psi}和经典XY模型的普适类是一样的。

\end{document}