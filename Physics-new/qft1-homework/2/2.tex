\documentclass[hyperref, a4paper]{article}

\usepackage{geometry}
\usepackage{titling}
\usepackage{titlesec}
% No longer needed, since we will use enumitem package
% \usepackage{paralist}
\usepackage{enumitem}
\usepackage{footnote}
\usepackage{enumerate}
\usepackage{amsmath, amssymb, amsthm}
\usepackage{mathtools}
\usepackage{bbm}
\usepackage{cite}
\usepackage{graphicx}
\usepackage{subfigure}
\usepackage{physics}
\usepackage{tensor}
\usepackage{siunitx}
\usepackage[version=4]{mhchem}
\usepackage{tikz}
\usepackage{xcolor}
\usepackage{listings}
\usepackage{autobreak}
\usepackage[ruled, vlined, linesnumbered]{algorithm2e}
\usepackage{xr-hyper}
\usepackage[colorlinks,unicode]{hyperref} % , linkcolor=black, anchorcolor=black, citecolor=black, urlcolor=black, filecolor=black
\usepackage{prettyref}

% Page style
\geometry{left=3.18cm,right=3.18cm,top=2.54cm,bottom=2.54cm}
\titlespacing{\paragraph}{0pt}{1pt}{10pt}[20pt]
\setlength{\droptitle}{-5em}
\preauthor{\vspace{-10pt}\begin{center}}
\postauthor{\par\end{center}}

% More compact lists 
\setlist[itemize]{
    itemindent=17pt, 
    leftmargin=1pt,
    listparindent=\parindent,
    parsep=0pt,
}

% Math operators
\DeclareMathOperator{\timeorder}{\mathcal{T}}
\DeclareMathOperator{\diag}{diag}
\DeclareMathOperator{\legpoly}{P}
\DeclareMathOperator{\primevalue}{P}
\DeclareMathOperator{\sgn}{sgn}
\newcommand*{\ii}{\mathrm{i}}
\newcommand*{\ee}{\mathrm{e}}
\newcommand*{\const}{\mathrm{const}}
\newcommand*{\suchthat}{\quad \text{s.t.} \quad}
\newcommand*{\argmin}{\arg\min}
\newcommand*{\argmax}{\arg\max}
\newcommand*{\normalorder}[1]{: #1 :}
\newcommand*{\pair}[1]{\langle #1 \rangle}
\newcommand*{\fd}[1]{\mathcal{D} #1}
\DeclareMathOperator{\bigO}{\mathcal{O}}

% TikZ setting
\usetikzlibrary{arrows,shapes,positioning}
\usetikzlibrary{arrows.meta}
\usetikzlibrary{decorations.markings}
\tikzstyle arrowstyle=[scale=1]
\tikzstyle directed=[postaction={decorate,decoration={markings,
    mark=at position .5 with {\arrow[arrowstyle]{stealth}}}}]
\tikzstyle ray=[directed, thick]
\tikzstyle dot=[anchor=base,fill,circle,inner sep=1pt]

% Algorithm setting
% Julia-style code
\SetKwIF{If}{ElseIf}{Else}{if}{}{elseif}{else}{end}
\SetKwFor{For}{for}{}{end}
\SetKwFor{While}{while}{}{end}
\SetKwProg{Function}{function}{}{end}
\SetArgSty{textnormal}

\newcommand*{\concept}[1]{{\textbf{#1}}}

% Embedded codes
\lstset{basicstyle=\ttfamily,
  showstringspaces=false,
  commentstyle=\color{gray},
  keywordstyle=\color{blue}
}

\newrefformat{fig}{Figure~\ref{#1} on page~\pageref{#1}}

\title{QFT I, Homework 2}
\author{Jinyuan Wu}

\begin{document}

\maketitle

\paragraph{The complex scalar field} This is problem $2.2$ on p. 33 of Peskin.
Consider the field theory of a complex scalar field obeying the Klein-Gordon equation. The action of this theory is
\begin{equation}
  S=\int \dd^{4} x\left(\partial_{\mu} \phi^{*} \partial^{\mu} \phi-m^{2} \phi^{*} \phi\right).
  \label{eq:prob-1-1}
\end{equation}
It is convenient to analyze this theory by considering $\phi$ and $\phi^{*}$, rather than the real and imaginary parts of $\phi=\left(\phi_{1}+ \ii \phi_{2}\right) / \sqrt{2}$, as the basic dynamical variables.
\begin{itemize}
  \item[(a)] Find the conjugate momenta to $\phi(x)$ and $\phi^{*}(x)$ and the canonical commutation relations. Show that the Hamiltonian is
  \begin{equation}
    H=\int \dd^{3} x\left(\pi^{*} \pi+\nabla \phi^{*} \cdot \nabla \phi+m^{2} \phi^{*} \phi\right).
    \label{eq:prob-1-2}
  \end{equation}
  Compute the Heisenberg equation of motion for $\phi(x)$ and show that it is indeed the Klein-Gordon equation.
  \item[(b)] Diagonalize $H$ by introducing creation and annihilation operators. Show that the theory contains two sets of particles of mass $m$.
  \item[(c)] Rewrite the conserved charge
  \begin{equation}
    Q=\int \dd^{3} x \frac{\ii}{2}\left(\phi^{*} \pi^{*}-\pi \phi\right)
    \label{eq:prob-1-3}
  \end{equation}
  in terms of creation and annihilation operators, and evaluate the charge of the particles of each type.
  \item[(d)] Consider the case of two complex Klein-Gordon fields with the same mass. Label the fields as $\phi_{a}(x)$, where $a=1,2$. Show that there are now four conserved charges, one given by the generalization of part (c), and the other three given by
  \begin{equation}
    Q^{i}=\int \dd^{3} x \frac{\ii}{2}\left(\phi_{a}^{*}\left(\sigma^{i}\right)_{a b} \pi_{b}^{*}-\pi_{a}\left(\sigma^{i}\right)_{a b} \phi_{b}\right),
    \label{eq:prob-1-4}
  \end{equation}
  where $\sigma^{i}$ are the Pauli sigma matrices. Show that these three charges have the commutation relations of angular momentum $(S U(2))$. Generalize these results to the case of $n$ identical complex scalar fields.
\end{itemize}

\paragraph{Solution}
\begin{itemize}
    \item[(a)]  From \eqref{eq:prob-1-1} we have 
    \begin{equation}
        \pi = \pdv{\mathcal{L}}{\partial_0 \phi} = \partial^0 \phi^* = \dot{\phi}^*,
    \end{equation}
    and
    \begin{equation}
        \pi^* = \pdv{\mathcal{L}}{\partial_0 \phi^*} = \partial^0 \phi = \dot{\phi}.
    \end{equation}
    The canonical commutation relations are 
    \begin{equation}
        \begin{aligned}
            \comm*{\phi(\vb*{x}, t)}{\pi(\vb*{y}, t)} = \ii \delta^{(3)}(\vb*{x} - \vb*{y}), \quad \comm*{\phi^*(\vb*{x}, t)}{\pi^*(\vb*{y}, t)} = \ii \delta^{(3)}(\vb*{x} - \vb*{y}), \\
            \comm*{\phi(\vb*{x}, t)}{\phi(\vb*{y}, t)} = \comm*{\phi(\vb*{x}, t)}{\phi^*(\vb*{y}, t)} = \comm*{\phi^*(\vb*{x}, t)}{\phi^*(\vb*{y}, t)} = \comm*{\phi(\vb*{x}, t)}{\pi^*(\vb*{y}, t)} = 0, \\
            \comm*{\pi(\vb*{x}, t)}{\pi(\vb*{y}, t)} = \comm*{\pi(\vb*{x}, t)}{\pi^*(\vb*{y}, t)} = \comm*{\pi^*(\vb*{x}, t)}{\pi^*(\vb*{y}, t)} = \comm*{\pi(\vb*{x}, t)}{\phi^*(\vb*{y}, t)} = 0.
        \end{aligned}
        \label{eq:canonical-relation}
    \end{equation}
    The Hamiltonian is therefore 
    \[
        \begin{aligned}
            H &= \int \dd[3]{\vb*{x}} (\pi \partial_0 \phi + \pi^* \partial_0 \phi^* - \mathcal{L}) \\
            &= \int \dd[3]{\vb*{x}} ( \partial_0 \phi \partial^0 \phi^* + \partial_0 \phi^* \partial^0 \phi - (\partial_0 \phi \partial^0 \phi^* - \grad{\phi} \cdot \grad{\phi^*}) + m^2 \phi^* \phi) \\
            &= \int \dd[3]{\vb*{x}} (\dot{\phi} \dot{\phi}^* + \grad{\phi} \cdot \grad{\phi^*} + m^2 \phi \phi^*) \\
            &= \int \dd[3]{\vb*{x}} (\pi \pi^* + \grad{\phi} \cdot \grad{\phi^*} + m^2 \phi \phi^*),
        \end{aligned}
    \]
    which is exactly \eqref{eq:prob-1-2}.

    Now we try to derive equations of motion from \eqref{eq:prob-1-2}.
    The equation of motion for $\phi$ is 
    \[
        \begin{aligned}
            \dot{\phi}(\vb*{y}, t) &= \frac{1}{\ii} \comm*{\phi(\vb*{y}, t)}{H} \\
            &= \frac{1}{\ii} \int \dd[3]{\vb*{x}} (\comm*{\phi(\vb*{y}, t)}{\pi(\vb*{x}, t) \pi^*(\vb*{x}, t)} \\
            &\quad \quad + \comm*{\phi(\vb*{y}, t)}{\grad{\phi(\vb*{x}, t)} \cdot \grad{\phi^*(\vb*{x}, t)} + m^2 \phi(\vb*{x}, t) \phi^*(\vb*{x}, t)}) \\
            &= \frac{1}{\ii} \int \dd[3]{\vb*{x}} \comm*{\phi(\vb*{y}, t)}{\pi(\vb*{x}, t)} \pi^*(\vb*{x}, t) = \frac{1}{\ii} \int \dd[3]{\vb*{x}} \ii \delta^{(3)}(\vb*{x} - \vb*{y}) \pi^*(\vb*{x} , t) \\
            &= \pi^*(\vb*{y} , t), 
        \end{aligned}
    \]
    and the equation of motion for $\pi^*$ is 
    \[
        \begin{aligned}
            \dot{\pi}^*(\vb*{y}, t) &= \frac{1}{\ii} \comm*{\pi^*(\vb*{y}, t)}{H} \\
            &= \frac{1}{\ii} \int \dd[3]{\vb*{x}}  (\comm*{\pi^*(\vb*{y}, t)}{\pi(\vb*{x}, t) \pi^*(\vb*{x}, t)} \\
            &\quad \quad + \comm*{\pi^*(\vb*{y}, t)}{\grad{\phi(\vb*{x}, t)} \cdot \grad{\phi^*(\vb*{x}, t)} + m^2 \phi(\vb*{x}, t) \phi^*(\vb*{x}, t)}) \\
            &= \frac{1}{\ii} \int \dd[3]{\vb*{x}} (\grad_{\vb*{x}}{\phi(\vb*{x}, t)} \cdot \grad_{\vb*{x}}{\comm*{\pi^*(\vb*{y}, t)}{\phi^*(\vb*{x}, t)}} + m^2 \phi(\vb*{x}, t) \comm*{\pi^*(\vb*{y}, t)}{\phi^*(\vb*{x}, t)}) \\
            &= \frac{1}{\ii} \int \dd[3]{\vb*{x}} ( \grad_{\vb*{x}} \phi(\vb*{x}, t ) \grad_{\vb*{x}} (-\ii \delta^{(3)}(\vb*{x} - \vb*{y})) + m^2 \phi(\vb*{x}, t) (- \ii \delta^{(3)}(\vb*{x} - \vb*{y})) ) \\
            &= \int \dd[3]{\vb*{x}} ( - \grad_{\vb*{x}} \phi(\vb*{x}, t ) \grad_{\vb*{x}} \delta^{(3)}(\vb*{x} - \vb*{y}) - m^2 \phi(\vb*{x}, t) \delta^{(3)}(\vb*{x} - \vb*{y}) ) \\
            &= \int \dd[3]{\vb*{x}} ( \laplacian_{\vb*{x}} \phi(\vb*{x}, t ) \delta^{(3)}(\vb*{x} - \vb*{y}) - m^2 \phi(\vb*{x}, t) \delta^{(3)}(\vb*{x} - \vb*{y}) ) \\
            &= \laplacian_{\vb*{y}} \phi(\vb*{y}, t) - m^2 \phi(\vb*{y}, t),
        \end{aligned}
    \]
    so putting the two equations of motion together, we have 
    \[
        \partial_t^2 \phi(\vb*{y}, t) = \partial_t \pi^*(\vb*{y}, t) = \laplacian_{\vb*{y}} \phi(\vb*{y}, t) - m^2 \phi(\vb*{y}, t),
    \]
    or in other words
    \[
        (\partial_t^2 - \laplacian) \phi + m^2 \phi = 0,
    \]
    which is indeed the Klein-Gordon equation.
    \item[(b)] We make the following Fourier expansion at a given time $t$:
    \begin{equation}
        \phi(t, \vb*{x}) = \int \frac{\dd[3]{\vb*{p}}}{(2\pi)^3} \frac{1}{\sqrt{2 \omega_{\vb*{p}}}} (a_{\vb*{p}}(t) \ee^{\ii \vb*{p} \cdot \vb*{x}} + b^\dagger_{\vb*{p}}(t) \ee^{- \ii \vb*{p} \cdot \vb*{x}}).
        \label{eq:expansion-origin-1}
    \end{equation}
    The Klein-Gordon equation therefore reads 
    \[
        \begin{aligned}
            0 &= (\partial^2 + m^2) \phi(t, \vb*{x}) \\
            &= \int \frac{\dd[3]{\vb*{p}}}{(2\pi)^3} \frac{1}{\sqrt{2 \omega_{\vb*{p}}}} \left( (\partial_t^2 - \laplacian + m^2) a_{\vb*{p}}(t) \ee^{\ii \vb*{p} \cdot \vb*{x}} + (\partial_t^2 - \laplacian + m^2) b^\dagger_{\vb*{p}}(t) \ee^{- \ii \vb*{p} \cdot \vb*{x}} \right) \\ 
            &= \int \frac{\dd[3]{\vb*{p}}}{(2\pi)^3} \frac{1}{\sqrt{2 \omega_{\vb*{p}}}} \left( (\partial_t^2 + \vb*{p}^2 + m^2) a_{\vb*{p}}(t) \ee^{\ii \vb*{p} \cdot \vb*{x}} + (\partial_t^2 + \vb*{p}^2 + m^2) b^\dagger_{\vb*{p}}(t) \ee^{- \ii \vb*{p} \cdot \vb*{x}} \right),
        \end{aligned}
    \]
    so therefore we have 
    \[
        (\partial_t^2 + \vb*{p}^2 + m^2) a_{\vb*{p}}(t) = (\partial_t^2 + \vb*{p}^2 + m^2) b^\dagger_{\vb*{p}}(t) = 0,
    \]
    the solution of which are 
    \begin{equation}
        a_{\vb*{p}}(t) = \ee^{\ii \omega_{\vb*{p}} t} a_{\vb*{p} 1} + \ee^{- \ii \omega_{\vb*{p}} t} a_{\vb*{p} 2}, \quad b^\dagger_{\vb*{p}}(t) = \ee^{\ii \omega_{\vb*{p}} t} b^\dagger_{\vb*{p} 1} + \ee^{- \ii \omega_{\vb*{p}} t} b^\dagger_{\vb*{p} 2},
        \label{eq:expansion-origin-2}
    \end{equation}
    where 
    \[
        -\omega_{\vb*{p}}^2 + \vb*{p}^2 + m^2 = 0. 
    \]
    So by \eqref{eq:expansion-origin-1} and \eqref{eq:expansion-origin-2} we have 
    \[
        \begin{aligned}
            \phi(t, \vb*{x}) &= \int \frac{\dd[3]{\vb*{p}}}{(2\pi)^3} \frac{1}{\sqrt{2 \omega_{\vb*{p}}}} \left(  (\ee^{\ii \omega_{\vb*{p}} t} a_{\vb*{p} 1} + \ee^{- \ii \omega_{\vb*{p}} t} a_{\vb*{p} 2}) \ee^{\ii \vb*{p} \cdot \vb*{x}} + (\ee^{\ii \omega_{\vb*{p}} t} b^\dagger_{\vb*{p} 1} + \ee^{- \ii \omega_{\vb*{p}} t} b^\dagger_{\vb*{p} 2}) \ee^{- \ii \vb*{p} \cdot \vb*{x}} \right) \\
            &= \int \frac{\dd[3]{\vb*{p}}}{(2\pi)^3} \frac{1}{\sqrt{2 \omega_{\vb*{p}}}} \left( (a_{\vb*{p}2} + b^\dagger_{- \vb*{p}, 2}) \ee^{-\ii \omega_{\vb*{p}} t + \ii \vb*{p} \cdot \vb*{x}} + (a_{-\vb*{p} ,1} + b^\dagger_{\vb*{p} 1}) \ee^{\ii \omega_{\vb*{p}} t - \ii \vb*{p} \cdot \vb*{x}} \right).
        \end{aligned}
    \]
    Therefore by redefining $a$ and $b$ operators we have the expansion
    \begin{equation}
        \phi(x) = \int \frac{\dd[3]{\vb*{p}}}{(2\pi)^3} \frac{1}{\sqrt{2 \omega_{\vb*{p}}}} (a_{\vb*{p}} \ee^{- \ii p \cdot x} + b^\dagger_{\vb*{p}} \ee^{\ii p \cdot x}),
    \end{equation} 
    so the Fourier expansion of $\phi^*$ is
    \begin{equation}
        \phi^*(x) = \int \frac{\dd[3]{\vb*{p}}}{(2\pi)^3} \frac{1}{\sqrt{2\omega_{\vb*{p}}}} (a^\dagger_{\vb*{p}} \ee^{\ii p \cdot x} + b_{\vb*{p}} \ee^{- \ii p \cdot x}).
    \end{equation} 
    where $p$ is on-shell, i.e.
    \begin{equation}
        p^\mu = (\pm \omega_{\vb*{p}}, \vb*{p}), \quad p^2 = m^2, \quad \omega_{\vb*{p}} = \omega_{-\vb*{p}}= \sqrt{{\vb*{p}}^2 + m^2}.
    \end{equation}
    and the Fourier expansion of the two conjugate momenta are 
    \begin{equation}
        \pi(x) = \dot{\phi}^*(x) = \int \frac{\dd[3]{\vb*{p}}}{(2\pi)^3} \ii \sqrt{\frac{\omega_{\vb*{p}}}{2}} (a^\dagger_{\vb*{p}} \ee^{\ii p \cdot x} - b_{\vb*{p}} \ee^{- \ii p \cdot x}),
    \end{equation}
    and 
    \begin{equation}
        \pi^*(x) = \dot{\phi}(x) = \int \frac{\dd[3]{\vb*{p}}}{(2\pi)^3} \ii \sqrt{\frac{\omega_{\vb*{p}}}{2}} (- a_{\vb*{p}} \ee^{- \ii p \cdot x} + b^\dagger_{\vb*{p}} \ee^{\ii p \cdot x}).
    \end{equation}

    Now we try to derive the commutation relations of the operators $a_{\vb*{p}}$ and the operators $b_{\vb*{p}}$s.
    We will insert 
    \begin{equation}
        \begin{aligned}
            \comm*{a_{\vb*{p}}}{a^\dagger_{\vb*{p}'}} = (2\pi)^3 \delta^{(3)}(\vb*{p} - \vb*{p}'), \quad \comm*{b_{\vb*{p}}}{b^\dagger_{\vb*{p}'}} = (2\pi)^3 \delta^{(3)}(\vb*{p} - \vb*{p}') , \\
            \comm*{a_{\vb*{p}}}{a_{\vb*{p}'}} = \comm*{b_{\vb*{p}}}{b_{\vb*{p}'}} = \comm*{a^\dagger_{\vb*{p}}}{a^\dagger_{\vb*{p}'}} = \comm*{b^\dagger_{\vb*{p}}}{b^\dagger_{\vb*{p}'}} = \comm*{a_{\vb*{p}}}{b_{\vb*{p}'}}  = \comm*{a^\dagger_{\vb*{p}}}{b^\dagger_{\vb*{p}'}} = \comm*{a_{\vb*{p}}}{b^\dagger_{\vb*{p}'}} = \comm*{a_{\vb*{p}}^\dagger}{b_{\vb*{p}'}} = 0 
        \end{aligned}
        \label{eq:a-b-relation}
    \end{equation}
    into these four expansions, and verify whether \eqref{eq:canonical-relation} holds.
    If so, then \eqref{eq:a-b-relation} holds also because the relation between the creation and annihilation operators and the field operators are linear and therefore \eqref{eq:a-b-relation} and \eqref{eq:canonical-relation} must be equivalent if one of them implies the other.
    Assuming \eqref{eq:a-b-relation} to be true, we have 
    \[
        \begin{aligned}
            &\quad \comm*{\phi(\vb*{x}, t)}{\pi(\vb*{y}, t)} \\
            &= \int \frac{\dd[3]{\vb*{p}}}{(2\pi)^3} \int \frac{\dd[3]{\vb*{p}'}}{(2\pi)^3} \ii \frac{1}{\sqrt{2 \omega_{\vb*{p}}}} \sqrt{\frac{\omega_{\vb*{p}'}}{2}} \left( \comm*{a_{\vb*{p}}}{ a^\dagger_{\vb*{p}'}} \ee^{- \ii p \cdot x + \ii p' \cdot y} + \comm*{b^\dagger_{\vb*{p}}}{- b_{\vb*{p}'}} \ee^{\ii p \cdot x - \ii p' \cdot y} \right) \\
            &= \int \frac{\dd[3]{\vb*{p}}}{(2\pi)^3} \int \frac{\dd[3]{\vb*{p}'}}{(2\pi)^3} \ii \frac{1}{\sqrt{2 \omega_{\vb*{p}}}} \sqrt{\frac{\omega_{\vb*{p}'}}{2}} \left( (2\pi)^3 \delta^{(3)}(\vb*{p} - \vb*{p}') \ee^{\ii \vb*{p} \cdot \vb*{x} - \ii \vb*{p}' \cdot \vb*{y}} + (2\pi)^3 \delta^{(3)}(\vb*{p} - \vb*{p}') \ee^{- \ii \vb*{p} \cdot \vb*{x} + \ii \vb*{p}' \cdot \vb*{y}} \right) \\
            &= \int \frac{\dd[3]{\vb*{p}}}{(2\pi)^3} \ii \frac{1}{\sqrt{2\omega_{\vb*{p}}}} \sqrt{\frac{\omega_{\vb*{p}}}{2}} (\ee^{\ii \vb*{p} \cdot (\vb*{x} - \vb*{y})} + \ee^{- \ii \vb*{p} \cdot (\vb*{x} - \vb*{y})}) \\
            &= \frac{\ii}{2} \int \frac{\dd[3]{\vb*{p}}}{(2\pi)^3} (\ee^{\ii \vb*{p} \cdot (\vb*{x} - \vb*{y})} + \ee^{- \ii \vb*{p} \cdot (\vb*{x} - \vb*{y})}) = \frac{\ii}{2} (\delta^{(3)}(\vb*{x} - \vb*{y}) + \delta^{(3)}(\vb*{x} - \vb*{y})) \\
            &= \ii \delta^{(3)}(\vb*{x} - \vb*{y}),
        \end{aligned}
    \]
    The first equation holds because of the second line in \eqref{eq:a-b-relation}.
    Similarly we can show that $\comm*{\phi^*(\vb*{x}, t)}{\pi^*(\vb*{y}, t)} = \ii \delta^{(3)}(\vb*{x} - \vb*{y})$.
    Simply by the fact that $a$ and $b^\dagger$ commutes we find that 
    \[
        \comm*{\phi(\vb*{x}, t)}{\phi(\vb*{y}, t)} = \comm*{\phi^*(\vb*{x}, t)}{\phi^*(\vb*{y}, t)} = \comm*{\pi(\vb*{x}, t)}{\pi(\vb*{y}, t)} = \comm*{\pi^*(\vb*{x}, t)}{\pi^*(\vb*{y}, t)}  = 0.
    \]
    Besides, we have 
    \[
        \begin{aligned}
            &\quad \comm*{\phi(\vb*{x}, t)}{\phi^*(\vb*{y}, t)} \\
            &=  \int \frac{\dd[3]{\vb*{p}}}{(2\pi)^3} \int \frac{\dd[3]{\vb*{p}'}}{(2\pi)^3} \frac{1}{\sqrt{2\omega_{\vb*{p}'}}} \frac{1}{\sqrt{2 \omega_{\vb*{p}}}} (\comm*{a_{\vb*{p}}}{a^\dagger_{\vb*{p}'}} \ee^{- \ii p \cdot x + \ii p' \cdot y} + \comm*{b^\dagger_{\vb*{p}}}{b_{\vb*{p}'}} \ee^{\ii p \cdot x- \ii p' \cdot y})  \\
            &= \int \frac{\dd[3]{\vb*{p}}}{(2\pi)^3} \int \frac{\dd[3]{\vb*{p}'}}{(2\pi)^3} \frac{1}{\sqrt{2\omega_{\vb*{p}'}}} \frac{1}{\sqrt{2 \omega_{\vb*{p}}}} ( (2\pi)^3 \delta^{(3)}(\vb*{p} - \vb*{p}') \ee^{\ii \vb*{p} \cdot \vb*{x} - \ii \vb*{p}' \cdot \vb*{y}} - (2\pi)^3 \delta^{(3)}(\vb*{p} - \vb*{p}') \ee^{- \ii \vb*{p} \cdot \vb*{x} + \ii \vb*{p}' \cdot \vb*{y}}) \\
            &=  \int \frac{\dd[3]{\vb*{p}}}{(2\pi)^3}  \frac{1}{2 \omega_{\vb*{p}}} (\ee^{\ii \vb*{p} (\vb*{x} - \vb*{y})} - \ee^{- \ii \vb*{p} \cdot (\vb*{x} - \vb*{y})}) \\
            &=  \int \frac{\dd[3]{\vb*{p}}}{(2\pi)^3}  \frac{1}{2 \omega_{\vb*{p}}} (\ee^{\ii \vb*{p} (\vb*{x} - \vb*{y})} - \ee^{\ii \vb*{p} \cdot (\vb*{x} - \vb*{y})}) = 0.
        \end{aligned}
    \]
    Replacing $1 / \sqrt{2 \omega_{\vb*{p}}}$ with $\ii \sqrt{\omega_{\vb*{p}} / 2}$, we have also shown that $\comm*{\pi(\vb*{x}, t)}{\pi^*(\vb*{y} ,t)} = 0$.
    Now the only remaining commutators in \eqref{eq:canonical-relation} not shown are 
    \[
        \comm*{\phi(\vb*{x}, t)}{\pi^*(\vb*{y}, t)} = \comm*{\phi^*(\vb*{x}, t)}{\pi(\vb*{y}, t)} = 0.
    \]
    the proof of which are trivial because $a_{\vb*{p}}$ and $b^\dagger_{\vb*{p}}$ commute.
    So now we find that \eqref{eq:canonical-relation} hold under the assumption that \eqref{eq:a-b-relation} hold, so \eqref{eq:a-b-relation} and \eqref{eq:canonical-relation} are equivalent.

    Inserting the four expansions into \eqref{eq:prob-1-2} we obtain 
    \[
        \begin{aligned}
            H &= \int \dd[3]{\vb*{x}} \left(\pi^{*} \pi+\nabla \phi^{*} \cdot \nabla \phi+m^{2} \phi^{*} \phi\right) \\
            &= \int \dd[3]{\vb*{x}} \int \frac{\dd[3]{\vb*{p}}}{(2\pi)^3} \int \frac{\dd[3]{\vb*{p}'}}{(2\pi)^3} \Big( \ii \sqrt{\frac{\omega_{\vb*{p}}}{2}} (a^\dagger_{\vb*{p}} \ee^{\ii p \cdot x} - b_{\vb*{p}} \ee^{- \ii p \cdot x}) \times \ii \sqrt{\frac{\omega_{\vb*{p}'}}{2}} (- a_{\vb*{p}'} \ee^{- \ii p' \cdot x} + b^\dagger_{\vb*{p}'} \ee^{\ii p' \cdot x})  \\
            &\quad \quad + \frac{1}{\sqrt{2\omega_{\vb*{p}}}} ( - \ii \vb*{p} a^\dagger_{\vb*{p}} \ee^{\ii p \cdot x} + \ii \vb*{p} b_{\vb*{p}} \ee^{- \ii p \cdot x}) \cdot \frac{1}{\sqrt{2\omega_{\vb*{p}'}}} ( \ii \vb*{p}' a_{\vb*{p}'} \ee^{- \ii p' \cdot x} - \ii \vb*{p}' b^\dagger_{\vb*{p}'} \ee^{\ii p' \cdot x})  \\
            &\quad \quad + m^2 \frac{1}{\sqrt{2\omega_{\vb*{p}}}} (a^\dagger_{\vb*{p}} \ee^{\ii p \cdot x} + b_{\vb*{p}} \ee^{- \ii p \cdot x}) \times \frac{1}{\sqrt{2\omega_{\vb*{p}'}}} (a_{\vb*{p}'} \ee^{- \ii p' \cdot x} + b^\dagger_{\vb*{p}'} \ee^{\ii p' \cdot x}) \Big) \\
            &= \int \frac{\dd[3]{\vb*{p}}}{(2\pi)^3} \Big(- \frac{\omega_{\vb*{p}}}{2} (- a^\dagger_{\vb*{p}} a_{\vb*{p}} +a^\dagger_{\vb*{p}} b^\dagger_{-\vb*{p}} - b_{\vb*{p}} b^\dagger_{\vb*{p}} + b_{\vb*{p}} a_{-\vb*{p}} )  \\
            &\quad \quad + \frac{1}{2\omega_{\vb*{p}}} \vb*{p}^2 (a^\dagger_{\vb*{p}} a_{\vb*{p}} + a^\dagger_{\vb*{p}} b^\dagger_{-\vb*{p}} + b_{\vb*{p}} a_{-\vb*{p}} + b_{\vb*{p}} b^\dagger_{\vb*{p}}) + \frac{m^2}{2 \omega_{\vb*{p}}} (a_{\vb*{p}}^\dagger a_{\vb*{p}} + a^\dagger_{\vb*{p}} b^\dagger_{- \vb*{p}} + b_{\vb*{p}} a_{- \vb*{p}} + b_{\vb*{p}} b^\dagger_{\vb*{p}} ) \Big) \\
            &= \int \frac{\dd[3]{\vb*{p}}}{(2\pi)^3} \frac{\omega_{\vb*{p}}}{2} ( a^\dagger_{\vb*{p}} a_{\vb*{p}} - a^\dagger_{\vb*{p}} b^\dagger_{-\vb*{p}} + b_{\vb*{p}} b^\dagger_{\vb*{p}} - b_{\vb*{p}} a_{-\vb*{p}} + a_{\vb*{p}}^\dagger a_{\vb*{p}} + a^\dagger_{\vb*{p}} b^\dagger_{- \vb*{p}} + b_{\vb*{p}} a_{- \vb*{p}} + b_{\vb*{p}} b^\dagger_{\vb*{p}} ) \\
            &= \int \frac{\dd[3]{\vb*{p}}}{(2\pi)^3} \omega_{\vb*{p}} (a^\dagger_{\vb*{p}} a_{\vb*{p}} + b_{\vb*{p}} b^\dagger_{\vb*{p}}).
        \end{aligned}
    \]
    The third equation is obtained by integrating $x$ and then integrating $\vb*{p}$: since all operators are defined on the same time point, we have 
    \[
        \int \dd[3]{\vb*{x}} \ee^{\ii (p - p') \cdot x} = (2\pi)^3 \delta^{(3)}(\vb*{p} - \vb*{p}'),
    \]
    and then by integrating $\vb*{p}$ we just replace all $\vb*{p}'$ with $\vb*{p}$.
    Now we exchange $b_{\vb*{p}}$ and $b^\dagger_{\vb*{p}}$, we have 
    \begin{equation}
        \begin{aligned}
            H &= \int \frac{\dd[3]{\vb*{p}}}{(2\pi)^3} \omega_{\vb*{p}} (a^\dagger_{\vb*{p}} a_{\vb*{p}} + b^\dagger_{\vb*{p}} b_{\vb*{p}} + \comm*{b_{\vb*{p}}}{b^\dagger_{\vb*{p}}}) \\
            &= \int \frac{\dd[3]{\vb*{p}}}{(2\pi)^3} \omega_{\vb*{p}} (a^\dagger_{\vb*{p}} a_{\vb*{p}} + b^\dagger_{\vb*{p}} b_{\vb*{p}} + (2\pi)^3 \delta^{(3)}(0)).
        \end{aligned}
    \end{equation}
    This is the diagonalized Hamiltonian. It can be seen that we have two sets of particles here, and both of them have $\omega_{\vb*{p}}$ as the energy spectrum, whose mass is $m$.

    \item[(c)] We have 
    \[
        \begin{aligned}
            Q &= \int \dd[3]{\vb*{x}} \frac{\ii}{2} (\phi^* \pi^* - \pi \phi) \\
            &= \int \dd[3]{\vb*{x}} \int \frac{\dd[3]{\vb*{p}}}{(2\pi)^3}  \int \frac{\dd[3]{\vb*{p}'}}{(2\pi)^3}\frac{\ii}{2} \Big( \ii \frac{1}{\sqrt{2 \omega_{\vb*{p}}}} \sqrt{\frac{\omega_{\vb*{p}'}}{2}} (a^\dagger_{\vb*{p}} \ee^{\ii p \cdot x} + b_{\vb*{p}} \ee^{- \ii p \cdot x} ) (- a_{\vb*{p}'} \ee^{- \ii p' \cdot x} + b^\dagger_{\vb*{p}'} \ee^{\ii p' \cdot x}) \Big) + \text{h.c.} \\
            &= \int \frac{\dd[3]{\vb*{p}}}{(2\pi)^3} \left( - \frac{1}{4} \right) ( - a_{\vb*{p}}^\dagger a_{\vb*{p}} + b_{\vb*{p}} b^\dagger_{\vb*{p}} - b_{\vb*{p}} a_{- \vb*{p}} + a^\dagger_{\vb*{p}} b^\dagger_{- \vb*{p}} ) + \text{h.c.} \\
            &= \int \frac{\dd[3]{\vb*{p}}}{(2\pi)^3} \left( - \frac{1}{4} \right) ( - a_{\vb*{p}}^\dagger a_{\vb*{p}} + b_{\vb*{p}} b^\dagger_{\vb*{p}} - b_{\vb*{p}} a_{- \vb*{p}} + a^\dagger_{\vb*{p}} b^\dagger_{- \vb*{p}}  - a_{\vb*{p}}^\dagger a_{\vb*{p}} + b_{\vb*{p}} b^\dagger_{\vb*{p}} - a_{- \vb*{p}}^\dagger b^\dagger_{\vb*{p}} + b_{- \vb*{p}} a_{\vb*{p}} ) \\
            &= \int \frac{\dd[3]{\vb*{p}}}{(2\pi)^3} \left( - \frac{1}{4} \right) ( - a_{\vb*{p}}^\dagger a_{\vb*{p}} + b_{\vb*{p}} b^\dagger_{\vb*{p}} - b_{\vb*{p}} a_{- \vb*{p}} + a^\dagger_{\vb*{p}} b^\dagger_{- \vb*{p}}  - a_{\vb*{p}}^\dagger a_{\vb*{p}} + b_{\vb*{p}} b^\dagger_{\vb*{p}} - a_{\vb*{p}}^\dagger b^\dagger_{- \vb*{p}} + b_{ \vb*{p}} a_{- \vb*{p}} ) \\
            &= \frac{1}{2} \int \frac{\dd[3]{\vb*{p}}}{(2\pi)^3} (a^\dagger_{\vb*{p}} a_{\vb*{p}} - b_{\vb*{p}} b^\dagger_{\vb*{p}}),
        \end{aligned}
    \] 
    where we have used a procedure similar to the one we use to derive the Hamiltonian.
    Again, by exchange $b_{\vb*{p}}$ and $b^\dagger_{\vb*{p}}$, we have 
    \begin{equation}
        \begin{aligned}
            Q &= \frac{1}{2} \int \frac{\dd[3]{\vb*{p}}}{(2\pi)^3} (a^\dagger_{\vb*{p}} a_{\vb*{p}} - b_{\vb*{p}}^\dagger b_{\vb*{p}} - \comm*{b_{\vb*{p}}}{b^\dagger_{\vb*{p}}} ) \\
            &= \frac{1}{2} \int \frac{\dd[3]{\vb*{p}}}{(2\pi)^3} (a^\dagger_{\vb*{p}} a_{\vb*{p}} - b_{\vb*{p}}^\dagger b_{\vb*{p}} - (2\pi)^3 \delta^{(3)}(0) ) 
        \end{aligned}.
    \end{equation}
    So this is how $Q$ can be recast in terms of creation and annihilation operators, and we can immediately see that $a$ particles carry a charge of $1/2$ while $b$ particles carry a charge of $-1/2$.
    \item[(d)] Now the action is 
    \begin{equation}
        S = \int \dd[4]{x} (\partial_\mu \phi^*_a \partial^\mu \phi_a - m^2 \phi^*_a \phi_a).
    \end{equation} 
    The theory has $U(2)$ symmetry, i.e. when $U_{ij} \in U(2)$, we have 
    \[
        \phi'_a = U_{ab} \phi_b, \quad \mathcal{L}(\phi', \partial_\mu \phi') = \mathcal{L}(\phi, \partial_\mu \phi),
    \]
    where there is no coordinate transformation.
    The generators of $U(2)$ are just $\sigma^\mu, \mu = 0, 1, 2, 3$, therefore an infinitesimal transformation can be written as 
    \[
        \var{\phi_a} = - \frac{\ii}{2} \sigma\indices{^\mu_{ab}} \phi_b \var{a_\mu}, \quad \var{\phi^*_a} = \frac{\ii}{2} \sigma\indices{^\mu_{ab}} \phi^*_b \var{a_\mu}, \quad \var{x^\mu} = 0.
    \]
    Noether's theorem therefore tells us that 
    \[
        \begin{aligned}
            0 &= \partial_\mu \left( \pdv{\mathcal{L}}{\partial_\mu \phi_a} \var{\phi_a} + \var{\phi^*_a} \pdv{\mathcal{L}}{\partial_\mu \phi^*_a} + \mathcal{L} \var{x^\mu} \right) \\
            &= \partial_\mu \left( - \frac{\ii}{2} \partial^\mu \phi_a^* \sigma\indices{^\nu_{ab}} \phi_b \var{a_\nu} + \frac{\ii}{2} \sigma\indices{^\nu_{ab}} \phi^*_b \var{a_\nu}\partial^\mu \phi_a \right) \\
            &= \var{a_\nu} \times \frac{\ii}{2} \partial_\mu \left( \phi^*_b \sigma\indices{^\nu_{ab}} \partial^\mu \phi_a - \partial^\mu \phi_a^* \sigma\indices{^\nu_{ab}} \phi_b \right),
        \end{aligned}
    \]
    so now we have four Noether's currents, i.e.
    \begin{equation}
        j\indices{^\nu^\mu} = \frac{\ii}{2} (\phi^*_b \sigma\indices{^\nu_{ab}} \partial^\mu \phi_a - \partial^\mu \phi_a^* \sigma\indices{^\nu_{ab}} \phi_b), \quad \partial_\mu j\indices{^\nu^\mu} = 0.
    \end{equation}
    The conserved charges are 
    \begin{equation}
        \begin{aligned}
            Q^\nu &= \int \dd[3]{\vb*{x}} j\indices{^\nu_0} = \int \dd[3]{\vb*{x}} \frac{\ii}{2} (\phi_b^* \sigma\indices{^\nu_{ab}} \dot{\phi}_a - \dot{\phi}^*_a \sigma\indices{^\nu_{ab}} \phi_b) \\
            &= \int \dd[3]{\vb*{x}} \frac{\ii}{2} (\phi^*_b \sigma\indices{^\nu_{ab}} \pi^*_a - \pi_a \sigma\indices{^\nu_{ab}} \phi_b).
        \end{aligned}
        \label{eq:generalized-charge}
    \end{equation}
    Let $\nu = 0$, and we get 
    \begin{equation}
        Q^0 = \int \dd[3]{\vb*{x}} \frac{\ii}{2} (\pi^*_a \phi^*_a - \pi_a \phi_a),
    \end{equation}
    which is a generalization of \eqref{eq:prob-1-3}.

    Since all the four $Q^\nu$ are quadratic forms of the field operators and the coefficient matrices are Pauli matrices, it is already sufficient to see that they satisfy the angular momentum algebra.
    First note that 
    \[
        \begin{aligned}
            &\quad \comm*{(\pi_a A_{ab} \phi_b)|_{t, \vb*{x}}}{(\pi_c B_{cd} \phi_d)_{t, \vb*{y}}} \\
            &= \pi_a \comm*{A_{ab} \phi_b}{\pi_c B_{cd}} \phi_d + \pi_c \comm*{\pi_a A_{ab}}{B_{cd} \phi_d} \phi_b \\
            &= A_{ab} B_{cd} \pi_a \phi_d \times \ii \delta^{(3)}(\vb*{x} - \vb*{y}) \delta_{bc} - A_{ab} B_{cd} \pi_c \phi_b \times \ii \delta^{(3)}(\vb*{x} - \vb*{y}) \delta_{ad} \\
            &= \ii \delta^{(3)}(\vb*{x} - \vb*{y}) (\pi_a A_{ab} B_{bd} \phi_d - \pi_c B_{cd} A_{db} \phi_b) \\
            &= \ii \delta^{(3)}(\vb*{x} - \vb*{y}) \pi_a \phi_b (A_{ac} B_{cb} - B_{ac} A_{cb}),
        \end{aligned}
    \]
    so hence we have 
    \begin{equation}
        \comm*{(\pi_a A_{ab} \phi_b)|_{t, \vb*{x}}}{(\pi_c B_{cd} \phi_d)_{t, \vb*{y}}} = \ii \delta^{(3)}(\vb*{x} - \vb*{y}) [\pi_a]^\top(\vb*{x}) \comm*{A}{B} [\phi_b](\vb*{y}),
    \end{equation}
    and a similar equation holds for $\phi^*$ and $\pi^*$ that is 
    \begin{equation}
        \comm*{(\phi^*_b A_{ab} \pi^*_a)|_{t,\vb*{x}}}{(\phi^*_d B_{cd} \pi^*_c)|_{t, \vb*{y}}} = \ii \delta^{(3)}(\vb*{x} - \vb*{y}) [\phi^*_a]^\top \comm*{A^\top}{B^\top} [\pi^*_b].
    \end{equation}
    Thus we have 
    \[
        \begin{aligned}
            \comm*{Q^0}{Q^i} &= - \frac{1}{4} \int \dd[3]{\vb*{x}} \int \dd[3]{\vb*{y}} \Big( \ii \delta^{(3)}(\vb*{x} - \vb*{y}) [\pi^*_a]^\top(\vb*{x}) \comm*{I}{(\sigma^i)^\top} [\phi^*_b](\vb*{y}) \\
            &\quad \quad \quad + \ii \delta^{(3)}(\vb*{x} - \vb*{y}) [\pi_a]^\top(\vb*{x}) \comm*{I}{\sigma^i} [\phi_b](\vb*{y}) \Big) \\
            &= 0 + 0 = 0,
        \end{aligned}
    \]
    and 
    \[
        \begin{aligned}
            \comm*{Q^i}{Q^j} &= - \frac{1}{4} \int \dd[3]{\vb*{x}} \int \dd[3]{\vb*{y}} \Big( \ii \delta^{(3)}(\vb*{x} - \vb*{y}) [\pi^*_a]^\top(\vb*{x}) \comm*{(\sigma^i)^\top}{(\sigma^j)^\top} [\phi^*_b](\vb*{y}) \\
            &\quad \quad \quad + \ii \delta^{(3)}(\vb*{x} - \vb*{y}) [\pi_a]^\top(\vb*{x}) \comm*{\sigma^i}{\sigma^j} [\phi_b](\vb*{y}) \Big) \\
            &= - \frac{\ii}{4} \int \dd[3]{\vb*{x}} \Big( [\pi^*_a]^\top \times (- 2 \ii \epsilon_{ijk} (\sigma^k)^\top) \times [\phi^*_b] \\
            &\quad \quad \quad + [\pi_a]^\top \times 2 \ii \epsilon_{ijk} \sigma^k \times [\phi_b] \Big) \\
            &= \frac{-1}{2} \epsilon_{ijk} \int \dd[3]{\vb*{x}} ( [\pi^*_a]^\top (\sigma^k)^\top [\phi^*_b] - [\pi_a]^\top \sigma^k [\phi_b]) \\
            &= \ii \epsilon_{ijk} Q^k.
        \end{aligned}
    \]
    So now we have verified that the charges $Q^a$ satisfy the momentum algebra.

    Note that our derivation uses nothing special about $U(2)$. 
    \eqref{eq:generalized-charge} still works when the theory is generalized to $n$ fields, as long as the matrices $\sigma\indices{^\nu_{ab}}$ are now generators of $U(N)$. 
\end{itemize}

\paragraph{}

\paragraph{Correlation function of real scalar field} This is problem $2.3$ on p. 34 of Peskin. Evaluate the function. 
\[
\langle 0|\phi(x) \phi(y)| 0\rangle=D(x-y)=\int \frac{\dd^{3} p}{(2 \pi)^{3}} \frac{1}{2 E_{\vb*{p}}} e^{-\ii p \cdot (x-y)}
\]
for $(x-y)$ spacelike so that $(x-y)^{2}=-r^{2}$, explicitly in terms of Bessel functions.

\paragraph{Solution} Since $D(x - y)$ is invariant under any Lorentz transformation, we can just assume that $x^0 = y^0$, so that 
\begin{equation}
    \abs*{\vb*{x} - \vb*{y}}^2 = r^2.
\end{equation}
Therefore we have 
\[
    \begin{aligned}
        D(x - y) &= \int \frac{\dd[3]{\vb*{p}}}{(2\pi)^3} \frac{1}{2 E_{\vb*{p}}} \ee^{\ii \vb*{p} \cdot (\vb*{x} - \vb*{y})} = \int \frac{\dd[3]{\vb*{p}}}{(2\pi)^3} \frac{1}{2 \sqrt{m^2 + \vb*{p}^2}} \ee^{\ii \vb*{p} \cdot (\vb*{x} - \vb*{y})} \\
        &= \int \frac{\dd[3]{\vb*{p}}}{(2\pi)^3} \frac{1}{2 \sqrt{m^2 + p^2}} \ee^{\ii p r \cos \theta } \\
        &= \int \frac{p^2 \dd{p} \sin \theta \dd{\theta} \dd{\varphi}}{(2\pi)^3} \frac{1}{2 \sqrt{p^2 + m^2}} \ee^{\ii p r \cos \theta} \\
        &= \frac{2\pi}{(2\pi)^3} \int_0^\infty p^2 \dd{p} \frac{1}{2 \sqrt{p^2 + m^2}} \int_0^\pi \sin \theta \dd{\theta} \ee^{\ii p r \cos \theta} \\
        &= \frac{1}{(2\pi)^2} \int_0^\infty p^2 \dd{p} \frac{1}{2 \sqrt{p^2 + m^2}} \frac{\ee^{\ii p r} - \ee^{- \ii p r}}{\ii p r} \\
        &= - \frac{\ii}{2 (2\pi)^2 r} \int_{-\infty}^\infty p \dd{p} \frac{\ee^{\ii p r}}{\sqrt{p^2 + m^2}},
    \end{aligned}
\]
where $p$ denotes $\abs*{\vb*{p}}$ instead of $p^\mu$.
By changing the integral contour (shown in \prettyref{fig:the-integral-contour}) we have 
\begin{equation}
    \begin{aligned}
        D(x-y) &= - \frac{\ii}{2 (2\pi)^2 r} \int^\infty_{-\infty} p \dd{p} \frac{\ee^{\ii p r}}{\sqrt{p^2 + m^2}} \\
        &= - \frac{\ii}{2 (2\pi)^2 r} \times 2 \times \int_{\ii m}^{\ii \infty} p \dd{p} \frac{\ee^{\ii p r}}{\sqrt{p^2 + m^2}} \\
        &= - \frac{\ii}{4 \pi^2 r} \int_m^\infty \ii \rho \times  \ii \dd{\rho} \frac{\ee^{\ii (\ii \rho) r}}{\sqrt{(\ii \rho)^2 + m^2}} \\
        &= \frac{1}{4\pi^2 r} \int_m^\infty \dd{\rho} \frac{\rho \ee^{-\rho r}}{\sqrt{\rho^2 - m^2}},
    \end{aligned}
    \label{eq:deriving-d-xy}
\end{equation}
and by Mathematica we have 
\begin{equation}
    D(x-y) = \frac{m}{4\pi^2 r} K_1(mr),
\end{equation}
where $K_n(x)$ denotes the Bessel $K$ function, i.e. the modified Bessel function of the second kind.

\begin{figure}
    \centering
    

\tikzset{every picture/.style={line width=0.75pt}} %set default line width to 0.75pt        

\begin{tikzpicture}[x=0.75pt,y=0.75pt,yscale=-1,xscale=1]
%uncomment if require: \path (0,355); %set diagram left start at 0, and has height of 355

%Straight Lines [id:da9384661794436353] 
\draw    (172,192) -- (387,192) ;
\draw [shift={(389,192)}, rotate = 180] [fill={rgb, 255:red, 0; green, 0; blue, 0 }  ][line width=0.08]  [draw opacity=0] (12,-3) -- (0,0) -- (12,3) -- cycle    ;
%Straight Lines [id:da7960229970616877] 
\draw    (280.5,333.67) -- (280.5,41.67) ;
\draw [shift={(280.5,39.67)}, rotate = 450] [fill={rgb, 255:red, 0; green, 0; blue, 0 }  ][line width=0.08]  [draw opacity=0] (12,-3) -- (0,0) -- (12,3) -- cycle    ;
%Straight Lines [id:da020774989692481816] 
\draw    (280.25,151) ;
\draw [shift={(280.25,151)}, rotate = 45] [color={rgb, 255:red, 0; green, 0; blue, 0 }  ][line width=0.75]    (-5.59,0) -- (5.59,0)(0,5.59) -- (0,-5.59)   ;
%Straight Lines [id:da4874392386138706] 
\draw    (280.75,230.5) ;
\draw [shift={(280.75,230.5)}, rotate = 45] [color={rgb, 255:red, 0; green, 0; blue, 0 }  ][line width=0.75]    (-5.59,0) -- (5.59,0)(0,5.59) -- (0,-5.59)   ;
%Straight Lines [id:da5385174147320926] 
\draw    (280.5,49.67) .. controls (282.16,51.34) and (282.16,53) .. (280.49,54.67) .. controls (278.82,56.34) and (278.82,58) .. (280.48,59.67) .. controls (282.14,61.34) and (282.13,63.01) .. (280.46,64.67) .. controls (278.79,66.34) and (278.79,68) .. (280.45,69.67) .. controls (282.11,71.34) and (282.11,73) .. (280.44,74.67) .. controls (278.77,76.34) and (278.77,78) .. (280.43,79.67) .. controls (282.09,81.34) and (282.08,83.01) .. (280.41,84.67) .. controls (278.74,86.34) and (278.74,88) .. (280.4,89.67) .. controls (282.06,91.34) and (282.06,93) .. (280.39,94.67) .. controls (278.72,96.34) and (278.72,98) .. (280.38,99.67) .. controls (282.04,101.34) and (282.03,103.01) .. (280.36,104.67) .. controls (278.69,106.34) and (278.69,108) .. (280.35,109.67) .. controls (282.01,111.34) and (282.01,113) .. (280.34,114.67) .. controls (278.67,116.34) and (278.67,118) .. (280.33,119.67) .. controls (281.99,121.34) and (281.98,123.01) .. (280.31,124.67) .. controls (278.64,126.34) and (278.64,128) .. (280.3,129.67) .. controls (281.96,131.34) and (281.96,133) .. (280.29,134.67) .. controls (278.62,136.34) and (278.62,138) .. (280.28,139.67) .. controls (281.94,141.34) and (281.94,143) .. (280.27,144.67) .. controls (278.6,146.33) and (278.59,148) .. (280.25,149.67) -- (280.25,151) -- (280.25,151) ;
%Straight Lines [id:da06703860214630253] 
\draw    (280.75,232.33) .. controls (282.41,234) and (282.41,235.67) .. (280.74,237.33) .. controls (279.07,239) and (279.07,240.66) .. (280.73,242.33) .. controls (282.39,244) and (282.38,245.67) .. (280.71,247.33) .. controls (279.04,249) and (279.04,250.66) .. (280.7,252.33) .. controls (282.36,254) and (282.36,255.66) .. (280.69,257.33) .. controls (279.02,259) and (279.02,260.66) .. (280.68,262.33) .. controls (282.34,264) and (282.33,265.67) .. (280.66,267.33) .. controls (278.99,269) and (278.99,270.66) .. (280.65,272.33) .. controls (282.31,274) and (282.31,275.66) .. (280.64,277.33) .. controls (278.97,279) and (278.97,280.66) .. (280.63,282.33) .. controls (282.29,284) and (282.28,285.67) .. (280.61,287.33) .. controls (278.94,289) and (278.94,290.66) .. (280.6,292.33) .. controls (282.26,294) and (282.26,295.66) .. (280.59,297.33) .. controls (278.92,299) and (278.92,300.66) .. (280.58,302.33) .. controls (282.24,304) and (282.23,305.67) .. (280.56,307.33) .. controls (278.89,309) and (278.89,310.66) .. (280.55,312.33) .. controls (282.21,314) and (282.21,315.66) .. (280.54,317.33) .. controls (278.87,319) and (278.87,320.66) .. (280.53,322.33) .. controls (282.19,324) and (282.19,325.66) .. (280.52,327.33) .. controls (278.85,328.99) and (278.84,330.66) .. (280.5,332.33) -- (280.5,333.67) -- (280.5,333.67) ;
%Straight Lines [id:da6548908870907941] 
\draw [color={rgb, 255:red, 74; green, 144; blue, 226 }  ,draw opacity=1 ]   (179,185) -- (369,185) ;
%Straight Lines [id:da968493002036825] 
\draw [color={rgb, 255:red, 74; green, 144; blue, 226 }  ,draw opacity=1 ]   (270,185) -- (274,185) ;
\draw [shift={(276,185)}, rotate = 180] [fill={rgb, 255:red, 74; green, 144; blue, 226 }  ,fill opacity=1 ][line width=0.08]  [draw opacity=0] (12,-3) -- (0,0) -- (12,3) -- cycle    ;

%Straight Lines [id:da31931157297735724] 
\draw [color={rgb, 255:red, 208; green, 2; blue, 27 }  ,draw opacity=1 ]   (270.05,53) -- (270.05,154.3) ;
%Shape: Arc [id:dp3939141183172141] 
\draw  [draw opacity=0] (290.88,154.26) .. controls (290.88,154.26) and (290.88,154.26) .. (290.88,154.26) .. controls (290.95,157.54) and (286.34,160.23) .. (280.59,160.26) .. controls (274.84,160.3) and (270.12,157.68) .. (270.05,154.4) .. controls (270.05,154.34) and (270.05,154.28) .. (270.05,154.23) -- (280.46,154.33) -- cycle ; \draw  [color={rgb, 255:red, 208; green, 2; blue, 27 }  ,draw opacity=1 ] (290.88,154.26) .. controls (290.88,154.26) and (290.88,154.26) .. (290.88,154.26) .. controls (290.95,157.54) and (286.34,160.23) .. (280.59,160.26) .. controls (274.84,160.3) and (270.12,157.68) .. (270.05,154.4) .. controls (270.05,154.34) and (270.05,154.28) .. (270.05,154.23) ;
%Straight Lines [id:da08081592705578022] 
\draw [color={rgb, 255:red, 208; green, 2; blue, 27 }  ,draw opacity=1 ]   (290.88,52.96) -- (290.88,154.26) ;

%Straight Lines [id:da1747284277267389] 
\draw [color={rgb, 255:red, 208; green, 2; blue, 27 }  ,draw opacity=1 ]   (270.05,119.23) -- (270.05,125.03) ;
\draw [shift={(270.05,127.03)}, rotate = 270] [fill={rgb, 255:red, 208; green, 2; blue, 27 }  ,fill opacity=1 ][line width=0.08]  [draw opacity=0] (12,-3) -- (0,0) -- (12,3) -- cycle    ;


% Text Node
\draw (265.25,149) node [anchor=east] [inner sep=0.75pt]    {$\mathrm{i} m$};
% Text Node
\draw (270.25,227) node [anchor=east] [inner sep=0.75pt]    {$-\mathrm{i} m$};


\end{tikzpicture}

    \caption{The integral contour. The blue line corresponds to the first line of \eqref{eq:deriving-d-xy}, the red line the second line of \eqref{eq:deriving-d-xy}. }
    \label{fig:the-integral-contour}
\end{figure}

\end{document}