\documentclass[hyperref, a4paper]{article}

\usepackage{geometry}
\usepackage{titling}
\usepackage{titlesec}
% No longer needed, since we will use enumitem package
% \usepackage{paralist}
\usepackage{enumitem}
\usepackage{footnote}
\usepackage{enumerate}
\usepackage{amsmath, amssymb, amsthm}
\usepackage{mathtools}
\usepackage{bbm}
\usepackage{cite}
\usepackage{graphicx}
\usepackage{subfigure}
\usepackage{physics}
\usepackage{tensor}
\usepackage{siunitx}
\usepackage[version=4]{mhchem}
\usepackage{tikz}
\usepackage{xcolor}
\usepackage{listings}
\usepackage{autobreak}
\usepackage[ruled, vlined, linesnumbered]{algorithm2e}
\usepackage{nameref,zref-xr}
\zxrsetup{toltxlabel}
\zexternaldocument*[optics-]{../../optics/optics}[optics.pdf]
\zexternaldocument*[solid-]{../../solid/solid}[solid.pdf]
\usepackage[colorlinks,unicode]{hyperref} % , linkcolor=black, anchorcolor=black, citecolor=black, urlcolor=black, filecolor=black
\usepackage[most]{tcolorbox}
\usepackage{prettyref}

% Page style
\geometry{left=3.18cm,right=3.18cm,top=2.54cm,bottom=2.54cm}
\titlespacing{\paragraph}{0pt}{1pt}{10pt}[20pt]
\setlength{\droptitle}{-5em}
\preauthor{\vspace{-10pt}\begin{center}}
\postauthor{\par\end{center}}

% More compact lists 
\setlist[itemize]{
    itemindent=17pt, 
    leftmargin=1pt,
    listparindent=\parindent,
    parsep=0pt,
}

% Math operators
\DeclareMathOperator{\timeorder}{\mathcal{T}}
\DeclareMathOperator{\diag}{diag}
\DeclareMathOperator{\legpoly}{P}
\DeclareMathOperator{\primevalue}{P}
\DeclareMathOperator{\sgn}{sgn}
\newcommand*{\ii}{\mathrm{i}}
\newcommand*{\ee}{\mathrm{e}}
\newcommand*{\const}{\mathrm{const}}
\newcommand*{\suchthat}{\quad \text{s.t.} \quad}
\newcommand*{\argmin}{\arg\min}
\newcommand*{\argmax}{\arg\max}
\newcommand*{\normalorder}[1]{: #1 :}
\newcommand*{\pair}[1]{\langle #1 \rangle}
\newcommand*{\fd}[1]{\mathcal{D} #1}
\DeclareMathOperator{\bigO}{\mathcal{O}}

% TikZ setting
\usetikzlibrary{arrows,shapes,positioning}
\usetikzlibrary{arrows.meta}
\usetikzlibrary{decorations.markings}
\tikzstyle arrowstyle=[scale=1]
\tikzstyle directed=[postaction={decorate,decoration={markings,
    mark=at position .5 with {\arrow[arrowstyle]{stealth}}}}]
\tikzstyle ray=[directed, thick]
\tikzstyle dot=[anchor=base,fill,circle,inner sep=1pt]

% Algorithm setting
% Julia-style code
\SetKwIF{If}{ElseIf}{Else}{if}{}{elseif}{else}{end}
\SetKwFor{For}{for}{}{end}
\SetKwFor{While}{while}{}{end}
\SetKwProg{Function}{function}{}{end}
\SetArgSty{textnormal}

\newcommand*{\concept}[1]{{\textbf{#1}}}

% Embedded codes
\lstset{basicstyle=\ttfamily,
  showstringspaces=false,
  commentstyle=\color{gray},
  keywordstyle=\color{blue}
}

% Reference formatting
\newrefformat{fig}{Figure~\ref{#1} on page~\pageref{#1}}

% Color boxes
\tcbuselibrary{skins, breakable, theorems}
\newtcbtheorem[number within=section]{warning}{Warning}%
  {colback=orange!5,colframe=orange!65,fonttitle=\bfseries, breakable}{warn}

\title{QFT I, Homework 4}
\author{Jinyuan Wu}

\begin{document}

\maketitle

\paragraph{Scalar QED} Consider the theory of a complex scalar field $\phi$ interacting with the electromagnetic field $A^{\mu}$. The Lagrangian is
\begin{equation}
    \mathcal{L}=-\frac{1}{4} F_{\mu \nu} F^{\mu \nu}+\left(D_{\mu} \phi\right)^{*} D^{\mu} \phi-m^{2} \phi^{*} \phi.
    \label{eq:scalar-qed}
\end{equation}
where $D_{\mu}=\partial_{\mu}+ \ii e A_{\mu}$ is the usual gauge covaraint derivative.
\begin{itemize}
    \item[(a)] Show the Lagrangian is invariant under the gauge transformations
    \begin{equation}
        \phi(x) \rightarrow \ee^{-\ii \alpha(x)} \phi(x), \quad A_{\mu}(x) \rightarrow A_{\mu}(x)+\frac{1}{e} \partial_{\mu} \alpha(x).
        \label{eq:gauge}
    \end{equation}
    \item[(b)] Derive the Feynman rules for the interaction between photons and scalar particles.
    \item[(c)] Draw all the leading-order Feynman diagrams and compute the amplitude for the process $\gamma \gamma \rightarrow \phi \phi^{*}$.
    \item[(d)] Compute the differential cross section $\dd \sigma / \dd \cos \theta$. You can take an average over all initial state polarizations. For simplicity, you can restrict your calculation in the limit $m=0$.
    \item[(e)] Draw all leading order Feynman diagrams, that contribute to the Compton scattering process $\gamma \phi \rightarrow \gamma \phi$ and compute the differential cross section $\dd \sigma / \dd \cos \theta$ with $m=0$.
\end{itemize}

\paragraph{Solution} \begin{itemize}
\item[(a)] Under the gauge transformation \eqref{eq:gauge}, we have 
\[
    F_{\mu \nu}  \to 
    F'_{\mu \nu} = \partial_\mu A'_\nu - \partial_\nu A'_\mu
    = \partial_\mu \left(A_\nu + \frac{1}{e} \partial_\nu \alpha\right) 
    - \partial_\nu \left(A_\mu + \frac{1}{e} \partial_\mu \alpha\right)
    = \partial_\mu A_\nu - \partial_\nu A_\mu = F_{\mu \nu},
\] 
\end{itemize}
so the first term in \eqref{eq:scalar-qed} remains the same. It is obvious that under \eqref{eq:gauge}
\[
    \phi^* \phi \to \phi'^* \phi' = \ee^{\ii \alpha} \phi^* \ee^{- \ii \alpha} \phi = \phi^* \phi,
\]
so the third term in \eqref{eq:scalar-qed} is also invariant. Also we have
\[
    \begin{aligned}
        D^\mu \phi \to (\partial^\mu + \ii e A'^\mu) \phi' 
        &= (\partial^\mu + \ii e A^\mu + \ii \partial^\mu \alpha) \ee^{- \ii \alpha} \phi  \\
        &= \ee^{- \ii \alpha} (\partial^\mu - \ii \partial^\mu \alpha 
        + \ii e A^\mu + \ii \partial^\mu \alpha) \phi \\
        &= \ee^{- \ii \alpha} D^\mu \phi,
    \end{aligned}
\]
and also 
\[
    (D^\mu \phi)^* = \ee^{\ii \alpha} D^\mu \phi, 
\]
so $D^\mu \phi (D^\mu \phi)^*$ is also invariant. 
Therefore \eqref{eq:scalar-qed} is invariant under \eqref{eq:gauge}.

\item[(b)] Expanding \eqref{eq:gauge} we have 
\begin{equation}
    \mathcal{L} = \mathcal{L}_\text{scalar} + \mathcal{L}_\text{vector} + \mathcal{L}_\text{scalarQED},
\end{equation} 
where $\mathcal{L}_\text{scalar}$ and $\mathcal{L}_\text{vector}$ are Lagrangians of free scalar field and 
free massless vector field, and 
\begin{equation}
    \mathcal{L}_\text{scalarQED} = 
\end{equation}

\end{document}