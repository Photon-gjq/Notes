\documentclass[hyperref, a4paper]{article}

\usepackage{geometry}
\usepackage{titling}
\usepackage{titlesec}
% No longer needed, since we will use enumitem package
% \usepackage{paralist}
\usepackage{enumitem}
\usepackage{footnote}
\usepackage{enumerate}
\usepackage{amsmath, amssymb, amsthm}
\usepackage{mathtools}
\usepackage{bbm}
\usepackage{cite}
\usepackage{graphicx}
\usepackage{subfigure}
\usepackage{physics}
\usepackage{tensor}
\usepackage{siunitx}
\usepackage[version=4]{mhchem}
\usepackage{tikz}
\usepackage{xcolor}
\usepackage{listings}
\usepackage{autobreak}
\usepackage[ruled, vlined, linesnumbered]{algorithm2e}
\usepackage{nameref,zref-xr}
\zxrsetup{toltxlabel}
\zexternaldocument*[hw2-]{../2/2}[2.pdf]
\usepackage[colorlinks,unicode]{hyperref} % , linkcolor=black, anchorcolor=black, citecolor=black, urlcolor=black, filecolor=black
\usepackage[most]{tcolorbox}
\usepackage{prettyref}

% Page style
\geometry{left=3.18cm,right=3.18cm,top=2.54cm,bottom=2.54cm}
\titlespacing{\paragraph}{0pt}{1pt}{10pt}[20pt]
\setlength{\droptitle}{-5em}
\preauthor{\vspace{-10pt}\begin{center}}
\postauthor{\par\end{center}}

% More compact lists 
\setlist[itemize]{
    itemindent=17pt, 
    leftmargin=1pt,
    listparindent=\parindent,
    parsep=0pt,
}

% Math operators
\DeclareMathOperator{\timeorder}{\mathcal{T}}
\DeclareMathOperator{\diag}{diag}
\DeclareMathOperator{\legpoly}{P}
\DeclareMathOperator{\primevalue}{P}
\DeclareMathOperator{\sgn}{sgn}
\newcommand*{\ii}{\mathrm{i}}
\newcommand*{\ee}{\mathrm{e}}
\newcommand*{\const}{\mathrm{const}}
\newcommand*{\suchthat}{\quad \text{s.t.} \quad}
\newcommand*{\argmin}{\arg\min}
\newcommand*{\argmax}{\arg\max}
\newcommand*{\normalorder}[1]{: #1 :}
\newcommand*{\pair}[1]{\langle #1 \rangle}
\newcommand*{\fd}[1]{\mathcal{D} #1}
\DeclareMathOperator{\bigO}{\mathcal{O}}

% TikZ setting
\usetikzlibrary{arrows,shapes,positioning}
\usetikzlibrary{arrows.meta}
\usetikzlibrary{decorations.markings}
\tikzstyle arrowstyle=[scale=1]
\tikzstyle directed=[postaction={decorate,decoration={markings,
    mark=at position .5 with {\arrow[arrowstyle]{stealth}}}}]
\tikzstyle ray=[directed, thick]
\tikzstyle dot=[anchor=base,fill,circle,inner sep=1pt]

% Algorithm setting
% Julia-style code
\SetKwIF{If}{ElseIf}{Else}{if}{}{elseif}{else}{end}
\SetKwFor{For}{for}{}{end}
\SetKwFor{While}{while}{}{end}
\SetKwProg{Function}{function}{}{end}
\SetArgSty{textnormal}

\newcommand*{\concept}[1]{{\textbf{#1}}}

% Embedded codes
\lstset{basicstyle=\ttfamily,
  showstringspaces=false,
  commentstyle=\color{gray},
  keywordstyle=\color{blue}
}

% Reference formatting
\newrefformat{fig}{Figure~\ref{#1} on page~\pageref{#1}}

% Color boxes
\tcbuselibrary{skins, breakable, theorems}
\newtcbtheorem[number within=section]{warning}{Warning}%
  {colback=orange!5,colframe=orange!65,fonttitle=\bfseries, breakable}{warn}
\newtcbtheorem[number within=section]{note}{Note}%
  {colback=green!5,colframe=green!65,fonttitle=\bfseries, breakable}{note}
\newtcbtheorem[number within=section]{info}{Info}%
  {colback=blue!5,colframe=blue!65,fonttitle=\bfseries, breakable}{info}

\newcommand{\hwtwo}{\href{../2/2.pdf}{Homework 2}}

\title{QFT I, Homework 4}
\author{Jinyuan Wu}

\begin{document}

\maketitle

\paragraph{Scalar QED} Consider the theory of a complex scalar field $\phi$ interacting with the electromagnetic field $A^{\mu}$. The Lagrangian is
\begin{equation}
    \mathcal{L}=-\frac{1}{4} F_{\mu \nu} F^{\mu \nu}+\left(D_{\mu} \phi\right)^{*} D^{\mu} \phi-m^{2} \phi^{*} \phi.
    \label{eq:scalar-qed}
\end{equation}
where $D_{\mu}=\partial_{\mu}+ \ii e A_{\mu}$ is the usual gauge covaraint derivative.
\begin{itemize}
    \item[(a)] Show the Lagrangian is invariant under the gauge transformations
    \begin{equation}
        \phi(x) \rightarrow \ee^{-\ii \alpha(x)} \phi(x), \quad A_{\mu}(x) \rightarrow A_{\mu}(x)+\frac{1}{e} \partial_{\mu} \alpha(x).
        \label{eq:gauge}
    \end{equation}
    \item[(b)] Derive the Feynman rules for the interaction between photons and scalar particles.
    \item[(c)] Draw all the leading-order Feynman diagrams and compute the amplitude for the process $\gamma \gamma \rightarrow \phi \phi^{*}$.
    \item[(d)] Compute the differential cross section $\dd \sigma / \dd \cos \theta$. You can take an average over all initial state polarizations. For simplicity, you can restrict your calculation in the limit $m=0$.
    \item[(e)] Draw all leading order Feynman diagrams, that contribute to the Compton scattering process $\gamma \phi \rightarrow \gamma \phi$ and compute the differential cross section $\dd \sigma / \dd \cos \theta$ with $m=0$.
\end{itemize}

\paragraph{Solution} \begin{itemize}
\item[(a)] Under the gauge transformation \eqref{eq:gauge}, we have 
\[
    F_{\mu \nu}  \to 
    F'_{\mu \nu} = \partial_\mu A'_\nu - \partial_\nu A'_\mu
    = \partial_\mu \left(A_\nu + \frac{1}{e} \partial_\nu \alpha\right) 
    - \partial_\nu \left(A_\mu + \frac{1}{e} \partial_\mu \alpha\right)
    = \partial_\mu A_\nu - \partial_\nu A_\mu = F_{\mu \nu},
\] 
so the first term in \eqref{eq:scalar-qed} remains the same. It is obvious that under \eqref{eq:gauge}
\[
    \phi^* \phi \to \phi'^* \phi' = \ee^{\ii \alpha} \phi^* \ee^{- \ii \alpha} \phi = \phi^* \phi,
\]
so the third term in \eqref{eq:scalar-qed} is also invariant. Also we have
\[
    \begin{aligned}
        D^\mu \phi \to (\partial^\mu + \ii e A'^\mu) \phi' 
        &= (\partial^\mu + \ii e A^\mu + \ii \partial^\mu \alpha) \ee^{- \ii \alpha} \phi  \\
        &= \ee^{- \ii \alpha} (\partial^\mu - \ii \partial^\mu \alpha 
        + \ii e A^\mu + \ii \partial^\mu \alpha) \phi \\
        &= \ee^{- \ii \alpha} D^\mu \phi,
    \end{aligned}
\]
and also 
\[
    (D^\mu \phi)^* = \ee^{\ii \alpha} D^\mu \phi, 
\]
so $D^\mu \phi (D^\mu \phi)^*$ is also invariant. 
Therefore \eqref{eq:scalar-qed} is invariant under \eqref{eq:gauge}.

\item[(b)] We make the following expansion of Fourier transformation. For the complex scalar field we have 
\begin{equation}
    \phi(x) = \int \frac{\dd[3]{\vb*{p}}}{(2\pi)^3} \frac{1}{\sqrt{2 \omega_{\vb*{p}}}} (a_{\vb*{p}} \ee^{- \ii p \cdot x} + b^\dagger_{\vb*{p}} \ee^{\ii p \cdot x}).
\end{equation}
which was proved in \eqref{hw2-eq:complex-scalar-expansion} in \hwtwo. The vector field is expanded as 
\begin{equation}
    A_\mu(x) = \int \frac{\dd[3]{\vb*{p}}}{(2\pi)^3} \frac{1}{\sqrt{2 \omega_{\vb*{p}}}} \sum_{r=1}^2 \epsilon_\mu^r(\vb*{p}) \left({a}_{\vb*{p}, r}^\dagger \ee^{ \ii p \cdot x} + {a}_{\vb*{p}, r} \ee^{ - \ii p \cdot x} \right).
\end{equation}

Expanding \eqref{eq:gauge} we have 
\begin{equation}
    \mathcal{L} = \mathcal{L}_\text{scalar} + \mathcal{L}_\text{vector} + \mathcal{L}_\text{scalarQED},
\end{equation} 
where $\mathcal{L}_\text{scalar}$ and $\mathcal{L}_\text{vector}$ are Lagrangians of free scalar field and 
free massless vector field, and 
\begin{equation}
    \begin{aligned}
        \mathcal{L}_\text{scalarQED} &= (D_{\mu} \phi)^{*} D^{\mu} \phi - (\partial_\mu \phi)^* \partial^\mu \phi \\
        &= e^2 \eta_{\mu \nu} A^\mu A^\nu \phi^* \phi - \ii e A_\mu \phi^* \partial^\mu \phi + \ii e \partial_\mu \phi^* A^\mu \phi .
    \end{aligned}
\end{equation}
The first term has no derivatives. Therefore it gives the following (momentum space) vertex:
\begin{equation}
    \begin{gathered}
        \begin{tikzpicture}[x=0.75pt,y=0.75pt,yscale=-1,xscale=1]
            %uncomment if require: \path (0,300); %set diagram left start at 0, and has height of 300
            
            %Straight Lines [id:da20598725730998768] 
            \draw    (100,124) .. controls (102.36,124) and (103.54,125.18) .. (103.54,127.54) .. controls (103.54,129.89) and (104.72,131.07) .. (107.07,131.07) .. controls (109.43,131.07) and (110.61,132.25) .. (110.61,134.61) .. controls (110.61,136.96) and (111.79,138.14) .. (114.14,138.14) .. controls (116.5,138.14) and (117.68,139.32) .. (117.68,141.68) .. controls (117.68,144.03) and (118.86,145.21) .. (121.21,145.21) .. controls (123.57,145.21) and (124.75,146.39) .. (124.75,148.75) .. controls (124.75,151.1) and (125.93,152.28) .. (128.28,152.28) .. controls (130.64,152.28) and (131.82,153.46) .. (131.82,155.82) .. controls (131.82,158.18) and (133,159.36) .. (135.36,159.36) .. controls (137.71,159.36) and (138.89,160.54) .. (138.89,162.89) .. controls (138.89,165.25) and (140.07,166.43) .. (142.43,166.43) .. controls (144.78,166.43) and (145.96,167.61) .. (145.96,169.96) .. controls (145.96,172.32) and (147.14,173.5) .. (149.5,173.5) -- (153,177) -- (153,177) ;
            %Straight Lines [id:da2719113994092306] 
            \draw    (153,177) .. controls (153,179.36) and (151.82,180.54) .. (149.46,180.54) .. controls (147.11,180.54) and (145.93,181.72) .. (145.93,184.07) .. controls (145.93,186.43) and (144.75,187.61) .. (142.39,187.61) .. controls (140.04,187.61) and (138.86,188.79) .. (138.86,191.14) .. controls (138.86,193.5) and (137.68,194.68) .. (135.32,194.68) .. controls (132.97,194.68) and (131.79,195.86) .. (131.79,198.21) .. controls (131.79,200.57) and (130.61,201.75) .. (128.25,201.75) .. controls (125.9,201.75) and (124.72,202.93) .. (124.72,205.28) .. controls (124.72,207.64) and (123.54,208.82) .. (121.18,208.82) .. controls (118.82,208.82) and (117.64,210) .. (117.64,212.36) .. controls (117.64,214.71) and (116.46,215.89) .. (114.11,215.89) .. controls (111.75,215.89) and (110.57,217.07) .. (110.57,219.43) .. controls (110.57,221.78) and (109.39,222.96) .. (107.04,222.96) .. controls (104.68,222.96) and (103.5,224.14) .. (103.5,226.5) -- (102,228) -- (102,228) ;
            %Straight Lines [id:da11896663722164491] 
            \draw    (153,177) -- (205,229) ;
            %Straight Lines [id:da6755259756022818] 
            \draw    (205,125) -- (153,177) ;
            
            % Text Node
            \draw (98,120.6) node [anchor=south east] [inner sep=0.75pt]    {$\mu $};
            % Text Node
            \draw (100,231.4) node [anchor=north east] [inner sep=0.75pt]    {$\nu $};
            \end{tikzpicture}
    \end{gathered} = 2 \ii e^2 \eta_{\mu \nu},
    \label{eq:vertex-1}
\end{equation}
where the factor $\ii$ comes from the time evolution operator and the factor $2$ comes from the fact that there 
are two identical photon lines. The two $\phi$ lines can be any of the following four:
\[
    \begin{tikzpicture}[x=0.75pt,y=0.75pt,yscale=-1,xscale=1]
        %uncomment if require: \path (0,300); %set diagram left start at 0, and has height of 300
        
        %Straight Lines [id:da5333941910049727] 
        \draw    (102,110) -- (158,110) ;
        %Straight Lines [id:da5053748028189731] 
        \draw    (117,97) -- (141,97) ;
        \draw [shift={(143,97)}, rotate = 180] [fill={rgb, 255:red, 0; green, 0; blue, 0 }  ][line width=0.08]  [draw opacity=0] (12,-3) -- (0,0) -- (12,3) -- cycle    ;
        %Straight Lines [id:da35597292464392005] 
        \draw    (138,110) ;
        \draw [shift={(138,110)}, rotate = 180] [fill={rgb, 255:red, 0; green, 0; blue, 0 }  ][line width=0.08]  [draw opacity=0] (12,-3) -- (0,0) -- (12,3) -- cycle    ;
        
        %Straight Lines [id:da5339059893857614] 
        \draw    (198,110) -- (254,110) ;
        %Straight Lines [id:da4261501273289108] 
        \draw    (215,97) -- (239,97) ;
        \draw [shift={(213,97)}, rotate = 0] [fill={rgb, 255:red, 0; green, 0; blue, 0 }  ][line width=0.08]  [draw opacity=0] (12,-3) -- (0,0) -- (12,3) -- cycle    ;
        %Straight Lines [id:da8010910938646716] 
        \draw    (234,110) ;
        \draw [shift={(234,110)}, rotate = 180] [fill={rgb, 255:red, 0; green, 0; blue, 0 }  ][line width=0.08]  [draw opacity=0] (12,-3) -- (0,0) -- (12,3) -- cycle    ;
        
        %Straight Lines [id:da7123771813668587] 
        \draw    (289,110) -- (345,110) ;
        %Straight Lines [id:da48052315279755575] 
        \draw    (306,97) -- (330,97) ;
        \draw [shift={(304,97)}, rotate = 0] [fill={rgb, 255:red, 0; green, 0; blue, 0 }  ][line width=0.08]  [draw opacity=0] (12,-3) -- (0,0) -- (12,3) -- cycle    ;
        %Straight Lines [id:da4510882786256236] 
        \draw    (309,110) ;
        \draw [shift={(309,110)}, rotate = 0] [fill={rgb, 255:red, 0; green, 0; blue, 0 }  ][line width=0.08]  [draw opacity=0] (12,-3) -- (0,0) -- (12,3) -- cycle    ;
        
        %Straight Lines [id:da5794703722656782] 
        \draw    (383,110) -- (439,110) ;
        %Straight Lines [id:da4043268933923083] 
        \draw    (398,97) -- (422,97) ;
        \draw [shift={(424,97)}, rotate = 180] [fill={rgb, 255:red, 0; green, 0; blue, 0 }  ][line width=0.08]  [draw opacity=0] (12,-3) -- (0,0) -- (12,3) -- cycle    ;
        %Straight Lines [id:da19763943611515544] 
        \draw    (403,110) ;
        \draw [shift={(403,110)}, rotate = 0] [fill={rgb, 255:red, 0; green, 0; blue, 0 }  ][line width=0.08]  [draw opacity=0] (12,-3) -- (0,0) -- (12,3) -- cycle    ;
        
        
        % Text Node
        \draw (172,106) node [anchor=north west][inner sep=0.75pt]   [align=left] {,};
        % Text Node
        \draw (265,106) node [anchor=north west][inner sep=0.75pt]   [align=left] {,};
        % Text Node
        \draw (355,106) node [anchor=north west][inner sep=0.75pt]   [align=left] {,};
        % Text Node
        \draw (451,107) node [anchor=north west][inner sep=0.75pt]   [align=left] {.};
        \end{tikzpicture}        
\]
The second term gives
\[
    - \ii e A_\mu \phi^* \partial^\mu\phi \sim - \ii e A_{\mu} 
    (a^\dagger_{\vb*{p}} \ee^{\ii p \cdot x} + b_{\vb*{p}} \ee^{- \ii p \cdot x}) 
    (- \ii (p' \cdot x) a_{\vb*{p}'} \ee^{- \ii p' \cdot x} + \ii (p' \cdot x) b^\dagger_{\vb*{p}'} \ee^{\ii p' \cdot x}),
\]
and the third term is its complex conjugate. Therefore, the $a^\dagger a$ term in the Lagrangian is 
\[
    \sim - e(p_1 + p_2)_\mu A^\mu a^\dagger_{\vb*{p}_1} a_{\vb*{p}_2},
\]
so after adding the $\ii$ factor from the time evolution operator we have  
\begin{equation}
    \begin{gathered}
        \begin{tikzpicture}[x=0.75pt,y=0.75pt,yscale=-1,xscale=1]
            %uncomment if require: \path (0,300); %set diagram left start at 0, and has height of 300
            
            %Straight Lines [id:da26529668927633354] 
            \draw    (123,138) .. controls (124.67,136.33) and (126.33,136.33) .. (128,138) .. controls (129.67,139.67) and (131.33,139.67) .. (133,138) .. controls (134.67,136.33) and (136.33,136.33) .. (138,138) .. controls (139.67,139.67) and (141.33,139.67) .. (143,138) .. controls (144.67,136.33) and (146.33,136.33) .. (148,138) .. controls (149.67,139.67) and (151.33,139.67) .. (153,138) .. controls (154.67,136.33) and (156.33,136.33) .. (158,138) .. controls (159.67,139.67) and (161.33,139.67) .. (163,138) .. controls (164.67,136.33) and (166.33,136.33) .. (168,138) .. controls (169.67,139.67) and (171.33,139.67) .. (173,138) .. controls (174.67,136.33) and (176.33,136.33) .. (178,138) .. controls (179.67,139.67) and (181.33,139.67) .. (183,138) .. controls (184.67,136.33) and (186.33,136.33) .. (188,138) -- (190,138) -- (190,138) ;
            %Straight Lines [id:da8078184451432662] 
            \draw    (190,138) -- (242,190) ;
            \draw [shift={(216,164)}, rotate = 225] [fill={rgb, 255:red, 0; green, 0; blue, 0 }  ][line width=0.08]  [draw opacity=0] (12,-3) -- (0,0) -- (12,3) -- cycle    ;
            %Straight Lines [id:da015273692780729986] 
            \draw    (242,86) -- (190,138) ;
            \draw [shift={(216,112)}, rotate = 315] [fill={rgb, 255:red, 0; green, 0; blue, 0 }  ][line width=0.08]  [draw opacity=0] (12,-3) -- (0,0) -- (12,3) -- cycle    ;
            %Straight Lines [id:da8273592738769799] 
            \draw    (219,94) -- (201.41,111.59) ;
            \draw [shift={(200,113)}, rotate = 315] [fill={rgb, 255:red, 0; green, 0; blue, 0 }  ][line width=0.08]  [draw opacity=0] (12,-3) -- (0,0) -- (12,3) -- cycle    ;
            %Straight Lines [id:da2568084829615431] 
            \draw    (197,159) -- (215.59,177.59) ;
            \draw [shift={(217,179)}, rotate = 225] [fill={rgb, 255:red, 0; green, 0; blue, 0 }  ][line width=0.08]  [draw opacity=0] (12,-3) -- (0,0) -- (12,3) -- cycle    ;
            
            % Text Node
            \draw (121,138) node [anchor=east] [inner sep=0.75pt]    {$\mu $};
            % Text Node
            \draw (207.5,100.1) node [anchor=south east] [inner sep=0.75pt]    {$p$};
            % Text Node
            \draw (202,169.4) node [anchor=north east] [inner sep=0.75pt]    {$q$};
            \end{tikzpicture}
    \end{gathered} = - \ii e (p_\mu + q_\mu),
    \label{eq:vertex-2}
\end{equation}
and we can change the direction of a momentum line and a $\phi$-particle line arbitrarily; if a momentum line 
goes in contrast to the corresponding particle line, then we need to add a minus sign to the corresponding 
momentum. For example we have
\begin{equation}
    \begin{gathered}
        \begin{tikzpicture}[x=0.75pt,y=0.75pt,yscale=-1,xscale=1]
            %uncomment if require: \path (0,300); %set diagram left start at 0, and has height of 300
            
            %Straight Lines [id:da26529668927633354] 
            \draw    (123,138) .. controls (124.67,136.33) and (126.33,136.33) .. (128,138) .. controls (129.67,139.67) and (131.33,139.67) .. (133,138) .. controls (134.67,136.33) and (136.33,136.33) .. (138,138) .. controls (139.67,139.67) and (141.33,139.67) .. (143,138) .. controls (144.67,136.33) and (146.33,136.33) .. (148,138) .. controls (149.67,139.67) and (151.33,139.67) .. (153,138) .. controls (154.67,136.33) and (156.33,136.33) .. (158,138) .. controls (159.67,139.67) and (161.33,139.67) .. (163,138) .. controls (164.67,136.33) and (166.33,136.33) .. (168,138) .. controls (169.67,139.67) and (171.33,139.67) .. (173,138) .. controls (174.67,136.33) and (176.33,136.33) .. (178,138) .. controls (179.67,139.67) and (181.33,139.67) .. (183,138) .. controls (184.67,136.33) and (186.33,136.33) .. (188,138) -- (190,138) -- (190,138) ;
            %Straight Lines [id:da8078184451432662] 
            \draw    (190,138) -- (242,190) ;
            \draw [shift={(216,164)}, rotate = 225] [fill={rgb, 255:red, 0; green, 0; blue, 0 }  ][line width=0.08]  [draw opacity=0] (12,-3) -- (0,0) -- (12,3) -- cycle    ;
            %Straight Lines [id:da015273692780729986] 
            \draw    (242,86) -- (190,138) ;
            \draw [shift={(216,112)}, rotate = 315] [fill={rgb, 255:red, 0; green, 0; blue, 0 }  ][line width=0.08]  [draw opacity=0] (12,-3) -- (0,0) -- (12,3) -- cycle    ;
            %Straight Lines [id:da8273592738769799] 
            \draw    (217.59,95.41) -- (200,113) ;
            \draw [shift={(219,94)}, rotate = 135] [fill={rgb, 255:red, 0; green, 0; blue, 0 }  ][line width=0.08]  [draw opacity=0] (12,-3) -- (0,0) -- (12,3) -- cycle    ;
            %Straight Lines [id:da2568084829615431] 
            \draw    (198.41,160.41) -- (217,179) ;
            \draw [shift={(197,159)}, rotate = 45] [fill={rgb, 255:red, 0; green, 0; blue, 0 }  ][line width=0.08]  [draw opacity=0] (12,-3) -- (0,0) -- (12,3) -- cycle    ;
            
            % Text Node
            \draw (121,138) node [anchor=east] [inner sep=0.75pt]    {$\mu $};
            % Text Node
            \draw (207.5,100.1) node [anchor=south east] [inner sep=0.75pt]    {$p$};
            % Text Node
            \draw (202,169.4) node [anchor=north east] [inner sep=0.75pt]    {$q$};
            \end{tikzpicture}
    \end{gathered} = \ii e (p_\mu + q_\mu).
\end{equation}
There are four vertices in this type in total.

\begin{note*}{}{}
    Here we follow the notation of Peskin, i.e. using the \emph{momentum} arrow to denote whether this line
    represents creation or annihilation and using the arrow \emph{on} a particle line to show whether this line 
    represents a particle (if the direction of the particle line is parallel to the direction of the momentum line)
    or a antiparticle (otherwise). The real direction of a 4-momentum is \emph{not} represented in any arrow.
\end{note*} 

\item[(c)] We enumerate over all possible diagrams. The vertex \eqref{eq:vertex-1} itself is a diagram:
\begin{equation}
    \begin{gathered}
        \begin{tikzpicture}[x=0.75pt,y=0.75pt,yscale=-1,xscale=1]
            %uncomment if require: \path (0,300); %set diagram left start at 0, and has height of 300
            
            %Straight Lines [id:da21978589206393995] 
            \draw    (111,112) .. controls (113.36,112) and (114.54,113.18) .. (114.54,115.54) .. controls (114.54,117.89) and (115.72,119.07) .. (118.07,119.07) .. controls (120.43,119.07) and (121.61,120.25) .. (121.61,122.61) .. controls (121.61,124.96) and (122.79,126.14) .. (125.14,126.14) .. controls (127.5,126.14) and (128.68,127.32) .. (128.68,129.68) .. controls (128.68,132.03) and (129.86,133.21) .. (132.21,133.21) .. controls (134.57,133.21) and (135.75,134.39) .. (135.75,136.75) .. controls (135.75,139.1) and (136.93,140.28) .. (139.28,140.28) .. controls (141.64,140.28) and (142.82,141.46) .. (142.82,143.82) .. controls (142.82,146.18) and (144,147.36) .. (146.36,147.36) .. controls (148.71,147.36) and (149.89,148.54) .. (149.89,150.89) .. controls (149.89,153.25) and (151.07,154.43) .. (153.43,154.43) .. controls (155.78,154.43) and (156.96,155.61) .. (156.96,157.96) .. controls (156.96,160.32) and (158.14,161.5) .. (160.5,161.5) -- (164,165) -- (164,165) ;
            %Straight Lines [id:da9856052975518923] 
            \draw    (164,165) .. controls (164,167.36) and (162.82,168.54) .. (160.46,168.54) .. controls (158.11,168.54) and (156.93,169.72) .. (156.93,172.07) .. controls (156.93,174.43) and (155.75,175.61) .. (153.39,175.61) .. controls (151.04,175.61) and (149.86,176.79) .. (149.86,179.14) .. controls (149.86,181.5) and (148.68,182.68) .. (146.32,182.68) .. controls (143.97,182.68) and (142.79,183.86) .. (142.79,186.21) .. controls (142.79,188.57) and (141.61,189.75) .. (139.25,189.75) .. controls (136.9,189.75) and (135.72,190.93) .. (135.72,193.28) .. controls (135.72,195.64) and (134.54,196.82) .. (132.18,196.82) .. controls (129.82,196.82) and (128.64,198) .. (128.64,200.36) .. controls (128.64,202.71) and (127.46,203.89) .. (125.11,203.89) .. controls (122.75,203.89) and (121.57,205.07) .. (121.57,207.43) .. controls (121.57,209.78) and (120.39,210.96) .. (118.04,210.96) .. controls (115.68,210.96) and (114.5,212.14) .. (114.5,214.5) -- (113,216) -- (113,216) ;
            %Straight Lines [id:da9772050670458003] 
            \draw    (164,165) -- (216,217) ;
            \draw [shift={(190,191)}, rotate = 225] [fill={rgb, 255:red, 0; green, 0; blue, 0 }  ][line width=0.08]  [draw opacity=0] (12,-3) -- (0,0) -- (12,3) -- cycle    ;
            %Straight Lines [id:da4232647913189236] 
            \draw    (216,113) -- (164,165) ;
            \draw [shift={(190,139)}, rotate = 315] [fill={rgb, 255:red, 0; green, 0; blue, 0 }  ][line width=0.08]  [draw opacity=0] (12,-3) -- (0,0) -- (12,3) -- cycle    ;
            %Straight Lines [id:da02356337350673887] 
            \draw    (171,190) -- (189.59,208.59) ;
            \draw [shift={(191,210)}, rotate = 225] [fill={rgb, 255:red, 0; green, 0; blue, 0 }  ][line width=0.08]  [draw opacity=0] (12,-3) -- (0,0) -- (12,3) -- cycle    ;
            %Straight Lines [id:da21663694749778983] 
            \draw    (172,143) -- (189.59,125.41) ;
            \draw [shift={(191,124)}, rotate = 135] [fill={rgb, 255:red, 0; green, 0; blue, 0 }  ][line width=0.08]  [draw opacity=0] (12,-3) -- (0,0) -- (12,3) -- cycle    ;
            %Straight Lines [id:da8405940618431673] 
            \draw    (116,198) -- (133.59,180.41) ;
            \draw [shift={(135,179)}, rotate = 135] [fill={rgb, 255:red, 0; green, 0; blue, 0 }  ][line width=0.08]  [draw opacity=0] (12,-3) -- (0,0) -- (12,3) -- cycle    ;
            %Straight Lines [id:da6272624926312453] 
            \draw    (116,135) -- (134.59,153.59) ;
            \draw [shift={(136,155)}, rotate = 225] [fill={rgb, 255:red, 0; green, 0; blue, 0 }  ][line width=0.08]  [draw opacity=0] (12,-3) -- (0,0) -- (12,3) -- cycle    ;
            
            % Text Node
            \draw (109,108.6) node [anchor=south east] [inner sep=0.75pt]    {$k,\epsilon _{\mu }^{\sigma }$};
            % Text Node
            \draw (111,219.4) node [anchor=north east] [inner sep=0.75pt]    {$k',\epsilon _{\nu }^{\sigma '}$};
            % Text Node
            \draw (179.5,130.1) node [anchor=south east] [inner sep=0.75pt]    {$p$};
            % Text Node
            \draw (176,201.4) node [anchor=north east] [inner sep=0.75pt]    {$q$};
            \end{tikzpicture}                     
    \end{gathered} = (\epsilon^\sigma)^\mu (\epsilon^{\sigma'})^\nu \times 2 \ii e^2 \eta_{\mu \nu} \eqqcolon \ii \mathcal{M}_4. 
\end{equation} 
Combining two \eqref{eq:vertex-2}-type vertices we have a $t$-channel
\begin{equation}
    \begin{gathered}
        \begin{tikzpicture}[x=0.75pt,y=0.75pt,yscale=-1,xscale=1]
            %uncomment if require: \path (0,300); %set diagram left start at 0, and has height of 300
            
            %Straight Lines [id:da7518773750951211] 
            \draw    (170,205.67) .. controls (170.78,203.44) and (172.28,202.72) .. (174.51,203.5) .. controls (176.74,204.28) and (178.24,203.56) .. (179.01,201.33) .. controls (179.8,199.11) and (181.3,198.39) .. (183.52,199.17) .. controls (185.75,199.95) and (187.25,199.23) .. (188.03,197) .. controls (188.8,194.77) and (190.3,194.05) .. (192.53,194.83) .. controls (194.75,195.61) and (196.25,194.89) .. (197.04,192.67) .. controls (197.81,190.44) and (199.31,189.72) .. (201.54,190.5) .. controls (203.77,191.28) and (205.27,190.56) .. (206.05,188.33) .. controls (206.84,186.11) and (208.34,185.39) .. (210.56,186.17) .. controls (212.79,186.95) and (214.29,186.23) .. (215.06,184) .. controls (215.85,181.78) and (217.35,181.06) .. (219.57,181.84) -- (222,180.67) -- (222,180.67) ;
            %Straight Lines [id:da023689323523824912] 
            \draw    (275,96.67) -- (222,122.67) ;
            \draw [shift={(248.5,109.67)}, rotate = 333.87] [fill={rgb, 255:red, 0; green, 0; blue, 0 }  ][line width=0.08]  [draw opacity=0] (12,-3) -- (0,0) -- (12,3) -- cycle    ;
            %Straight Lines [id:da3034609552464371] 
            \draw    (222,122.67) -- (222,180.67) ;
            \draw [shift={(222,151.67)}, rotate = 270] [fill={rgb, 255:red, 0; green, 0; blue, 0 }  ][line width=0.08]  [draw opacity=0] (12,-3) -- (0,0) -- (12,3) -- cycle    ;
            %Straight Lines [id:da9685178679936943] 
            \draw    (166,100.67) .. controls (168.16,99.72) and (169.71,100.33) .. (170.65,102.49) .. controls (171.6,104.65) and (173.15,105.26) .. (175.31,104.32) .. controls (177.47,103.38) and (179.02,103.99) .. (179.96,106.15) .. controls (180.91,108.31) and (182.46,108.92) .. (184.62,107.98) .. controls (186.78,107.04) and (188.33,107.65) .. (189.27,109.81) .. controls (190.21,111.97) and (191.76,112.58) .. (193.92,111.64) .. controls (196.08,110.69) and (197.63,111.3) .. (198.58,113.46) .. controls (199.52,115.62) and (201.07,116.23) .. (203.23,115.29) .. controls (205.39,114.35) and (206.94,114.96) .. (207.88,117.12) .. controls (208.83,119.28) and (210.38,119.89) .. (212.54,118.95) .. controls (214.7,118.01) and (216.25,118.62) .. (217.19,120.78) .. controls (218.14,122.94) and (219.69,123.55) .. (221.85,122.61) -- (222,122.67) -- (222,122.67) ;
            %Straight Lines [id:da7135958731836463] 
            \draw    (275,206.67) -- (222,180.67) ;
            \draw [shift={(248.5,193.67)}, rotate = 206.13] [fill={rgb, 255:red, 0; green, 0; blue, 0 }  ][line width=0.08]  [draw opacity=0] (12,-3) -- (0,0) -- (12,3) -- cycle    ;
            %Straight Lines [id:da342051397453359] 
            \draw    (174,114.67) -- (199.17,125.85) ;
            \draw [shift={(201,126.67)}, rotate = 203.96] [fill={rgb, 255:red, 0; green, 0; blue, 0 }  ][line width=0.08]  [draw opacity=0] (12,-3) -- (0,0) -- (12,3) -- cycle    ;
            %Straight Lines [id:da6252414291688255] 
            \draw    (176,192) -- (200.17,181.47) ;
            \draw [shift={(202,180.67)}, rotate = 156.45] [fill={rgb, 255:red, 0; green, 0; blue, 0 }  ][line width=0.08]  [draw opacity=0] (12,-3) -- (0,0) -- (12,3) -- cycle    ;
            %Straight Lines [id:da5509398554375176] 
            \draw    (269.23,112.09) -- (246,124.17) ;
            \draw [shift={(271,111.17)}, rotate = 152.53] [fill={rgb, 255:red, 0; green, 0; blue, 0 }  ][line width=0.08]  [draw opacity=0] (12,-3) -- (0,0) -- (12,3) -- cycle    ;
            %Straight Lines [id:da663637802656273] 
            \draw    (274.23,197.24) -- (251,185.17) ;
            \draw [shift={(276,198.17)}, rotate = 207.47] [fill={rgb, 255:red, 0; green, 0; blue, 0 }  ][line width=0.08]  [draw opacity=0] (12,-3) -- (0,0) -- (12,3) -- cycle    ;
            %Straight Lines [id:da14906801778839696] 
            \draw    (233,141) -- (233,167.67) ;
            \draw [shift={(233,169.67)}, rotate = 270] [fill={rgb, 255:red, 0; green, 0; blue, 0 }  ][line width=0.08]  [draw opacity=0] (12,-3) -- (0,0) -- (12,3) -- cycle    ;
            
            % Text Node
            \draw (277,93.27) node [anchor=south west] [inner sep=0.75pt]    {$p$};
            % Text Node
            \draw (278,201.57) node [anchor=north west][inner sep=0.75pt]    {$q$};
            % Text Node
            \draw (237,144.4) node [anchor=north west][inner sep=0.75pt]    {$q'$};
            % Text Node
            \draw (164,97.27) node [anchor=south east] [inner sep=0.75pt]    {$k,\epsilon _{\mu }^{\sigma }$};
            % Text Node
            \draw (168,209.07) node [anchor=north east] [inner sep=0.75pt]    {$k',\epsilon _{\nu }^{\sigma '}$};
            \end{tikzpicture}            
    \end{gathered} 
    \begin{aligned}[t]
        &= \epsilon^\sigma_\mu e^{\sigma'}_\nu \times \ii e (p - (k - p))^\mu \times \ii e (- (k - p) - q)^\nu \times \frac{\ii}{(k - p)^2 - m^2 + \ii 0^+} \\
        &= - \ii e^2 \epsilon^\sigma_\mu (2 p - k)^\mu \epsilon^{\sigma'}_\nu (-k+p-q)^\nu \frac{1}{(k-p)^2 - m^2 + \ii 0^+} \eqqcolon \ii \mathcal{M}_t,
    \end{aligned}
\end{equation}
and a $u$-channel
\begin{equation}
    \begin{gathered}
        \begin{tikzpicture}[x=0.75pt,y=0.75pt,yscale=-1,xscale=1]
            %uncomment if require: \path (0,300); %set diagram left start at 0, and has height of 300
            
            %Straight Lines [id:da3057302086045661] 
            \draw    (152,184.67) .. controls (151.45,182.38) and (152.31,180.95) .. (154.6,180.4) .. controls (156.89,179.85) and (157.76,178.42) .. (157.21,176.13) .. controls (156.65,173.84) and (157.52,172.41) .. (159.81,171.86) .. controls (162.1,171.31) and (162.97,169.88) .. (162.41,167.59) .. controls (161.86,165.3) and (162.73,163.87) .. (165.02,163.32) .. controls (167.31,162.77) and (168.18,161.34) .. (167.62,159.05) .. controls (167.06,156.76) and (167.93,155.33) .. (170.22,154.78) .. controls (172.51,154.23) and (173.38,152.8) .. (172.82,150.51) .. controls (172.27,148.22) and (173.14,146.8) .. (175.43,146.25) .. controls (177.72,145.7) and (178.59,144.27) .. (178.03,141.98) .. controls (177.47,139.69) and (178.34,138.26) .. (180.63,137.71) .. controls (182.92,137.16) and (183.79,135.73) .. (183.24,133.44) .. controls (182.68,131.15) and (183.55,129.72) .. (185.84,129.17) .. controls (188.13,128.62) and (189,127.19) .. (188.44,124.9) .. controls (187.89,122.61) and (188.76,121.18) .. (191.05,120.63) .. controls (193.34,120.08) and (194.21,118.65) .. (193.65,116.36) .. controls (193.09,114.07) and (193.96,112.64) .. (196.25,112.09) .. controls (198.54,111.54) and (199.4,110.12) .. (198.85,107.83) .. controls (198.3,105.54) and (199.17,104.11) .. (201.46,103.56) -- (202,102.67) -- (202,102.67) ;
            %Shape: Circle [id:dp7275452294524412] 
            \draw  [draw opacity=0][fill={rgb, 255:red, 255; green, 255; blue, 255 }  ,fill opacity=1 ] (172,130) .. controls (172,125.58) and (175.58,122) .. (180,122) .. controls (184.42,122) and (188,125.58) .. (188,130) .. controls (188,134.42) and (184.42,138) .. (180,138) .. controls (175.58,138) and (172,134.42) .. (172,130) -- cycle ;
            %Straight Lines [id:da9256849462699361] 
            \draw    (255,76.67) -- (202,102.67) ;
            \draw [shift={(228.5,89.67)}, rotate = 333.87] [fill={rgb, 255:red, 0; green, 0; blue, 0 }  ][line width=0.08]  [draw opacity=0] (12,-3) -- (0,0) -- (12,3) -- cycle    ;
            %Straight Lines [id:da488527311343778] 
            \draw    (202,102.67) -- (202,160.67) ;
            \draw [shift={(202,131.67)}, rotate = 270] [fill={rgb, 255:red, 0; green, 0; blue, 0 }  ][line width=0.08]  [draw opacity=0] (12,-3) -- (0,0) -- (12,3) -- cycle    ;
            %Straight Lines [id:da2615197234092139] 
            \draw    (146,80.67) .. controls (148.32,81.08) and (149.28,82.44) .. (148.87,84.76) .. controls (148.46,87.08) and (149.41,88.45) .. (151.73,88.86) .. controls (154.05,89.27) and (155.01,90.64) .. (154.6,92.96) .. controls (154.19,95.28) and (155.15,96.64) .. (157.47,97.05) .. controls (159.79,97.46) and (160.75,98.83) .. (160.34,101.15) .. controls (159.93,103.47) and (160.88,104.83) .. (163.2,105.24) .. controls (165.52,105.65) and (166.48,107.02) .. (166.07,109.34) .. controls (165.66,111.66) and (166.62,113.03) .. (168.94,113.44) .. controls (171.26,113.85) and (172.22,115.21) .. (171.81,117.53) .. controls (171.4,119.85) and (172.35,121.22) .. (174.67,121.63) .. controls (176.99,122.04) and (177.95,123.4) .. (177.54,125.72) .. controls (177.13,128.04) and (178.09,129.41) .. (180.41,129.82) .. controls (182.73,130.23) and (183.69,131.6) .. (183.28,133.92) .. controls (182.87,136.24) and (183.82,137.6) .. (186.14,138.01) .. controls (188.46,138.42) and (189.42,139.79) .. (189.01,142.11) .. controls (188.6,144.43) and (189.56,145.8) .. (191.88,146.21) .. controls (194.2,146.62) and (195.15,147.98) .. (194.74,150.3) .. controls (194.33,152.62) and (195.29,153.99) .. (197.61,154.4) .. controls (199.93,154.81) and (200.89,156.17) .. (200.48,158.49) -- (202,160.67) -- (202,160.67) ;
            %Straight Lines [id:da7955957386194441] 
            \draw    (255,186.67) -- (202,160.67) ;
            \draw [shift={(228.5,173.67)}, rotate = 206.13] [fill={rgb, 255:red, 0; green, 0; blue, 0 }  ][line width=0.08]  [draw opacity=0] (12,-3) -- (0,0) -- (12,3) -- cycle    ;
            %Straight Lines [id:da8700765164776401] 
            \draw    (145,98.67) -- (162.77,121.42) ;
            \draw [shift={(164,123)}, rotate = 232.02] [fill={rgb, 255:red, 0; green, 0; blue, 0 }  ][line width=0.08]  [draw opacity=0] (12,-3) -- (0,0) -- (12,3) -- cycle    ;
            %Straight Lines [id:da7164688914224506] 
            \draw    (147,170) -- (162.89,146.33) ;
            \draw [shift={(164,144.67)}, rotate = 123.86] [fill={rgb, 255:red, 0; green, 0; blue, 0 }  ][line width=0.08]  [draw opacity=0] (12,-3) -- (0,0) -- (12,3) -- cycle    ;
            %Straight Lines [id:da4544265427173353] 
            \draw    (249.23,92.09) -- (226,104.17) ;
            \draw [shift={(251,91.17)}, rotate = 152.53] [fill={rgb, 255:red, 0; green, 0; blue, 0 }  ][line width=0.08]  [draw opacity=0] (12,-3) -- (0,0) -- (12,3) -- cycle    ;
            %Straight Lines [id:da7941011992391758] 
            \draw    (254.23,177.24) -- (231,165.17) ;
            \draw [shift={(256,178.17)}, rotate = 207.47] [fill={rgb, 255:red, 0; green, 0; blue, 0 }  ][line width=0.08]  [draw opacity=0] (12,-3) -- (0,0) -- (12,3) -- cycle    ;
            %Straight Lines [id:da44874471775014224] 
            \draw    (213,121) -- (213,147.67) ;
            \draw [shift={(213,149.67)}, rotate = 270] [fill={rgb, 255:red, 0; green, 0; blue, 0 }  ][line width=0.08]  [draw opacity=0] (12,-3) -- (0,0) -- (12,3) -- cycle    ;
            
            % Text Node
            \draw (257,73.27) node [anchor=south west] [inner sep=0.75pt]    {$p$};
            % Text Node
            \draw (258,181.57) node [anchor=north west][inner sep=0.75pt]    {$q$};
            % Text Node
            \draw (217,124.4) node [anchor=north west][inner sep=0.75pt]    {$q'$};
            % Text Node
            \draw (144,77.27) node [anchor=south east] [inner sep=0.75pt]    {$k,\epsilon _{\mu }^{\sigma }$};
            % Text Node
            \draw (150,188.07) node [anchor=north east] [inner sep=0.75pt]    {$k',\epsilon _{\nu }^{\sigma '}$};
            \end{tikzpicture}            
    \end{gathered} 
    \begin{aligned}[t]
        &= \epsilon^\sigma_\mu \epsilon^{\sigma'}_\nu \times \ii e (p - (q-k))^\nu \times \ii e (- q - (q-k))^\mu \times \frac{\ii}{(q-k)^2 - m^2 + \ii 0^+} \\
        &= \ii e^2 \epsilon^\sigma_\mu (2q-k)^\mu \epsilon^{\sigma'}_\nu (p-q+k)^\nu \frac{1}{(q-k)^2 - m^2 + \ii 0^+} \eqqcolon \ii \mathcal{M}_u.
    \end{aligned}
\end{equation}

\begin{note*}{}{}
    We \emph{do not} need to distinguish the direction of the $q'$ momentum line. This line can be either 
    a particle line or an antiparticle line, but since the ordinary propagator $\ii / (p^2 - m^2 + \ii 0^+)$ 
    is obtained by summing up the two cases, when we write down this propagator, we have automatically 
    considered both processes. 
\end{note*}

Summing everything up, we have 
\begin{equation}
    \begin{aligned}
        \ii \mathcal{M}(\gamma \gamma \to \phi \phi^*) &= \ii (\mathcal{M}_4 + \mathcal{M}_t + \mathcal{M}_u) \\
        &= \ii e^2 (\epsilon^\sigma)^\mu (\epsilon^{\sigma'})^\nu \left( 2 \eta_{\mu \nu} + \frac{(k-2p)_\mu (k'-2q)_\nu}{t - m^2} + \frac{(k-2q)^\mu (k'-2q)^\nu}{u - m^2} \right),
    \end{aligned}
\end{equation}
where 
\begin{equation}
    t = (k-p)^2, \quad u = (q-k)^2.
\end{equation}

\item[(d)] The massless limit can be calculated with Eq.~(4.85) in Peskin, which is 
\begin{equation}
    \left(\dv{\sigma}{\Omega}\right)_\text{CM} = \frac{\abs*{\mathcal{M}}^2}{64 \pi^2 E_\text{CM}^2},
\end{equation} 
What we need is $\abs*{\mathcal{M}}^2$. We have 


\end{itemize}

\end{document}