\documentclass[hyperref, a4paper]{article}

\usepackage{geometry}
\usepackage{titling}
\usepackage{titlesec}
% No longer needed, since we will use enumitem package
% \usepackage{paralist}
\usepackage{enumitem}
\usepackage{footnote}
\usepackage{enumerate}
\usepackage{amsmath, amssymb, amsthm}
\usepackage{mathtools}
\usepackage{bbm}
\usepackage{cite}
\usepackage{graphicx}
\usepackage{subfigure}
\usepackage{physics}
\usepackage{tensor}
\usepackage{siunitx}
\usepackage[version=4]{mhchem}
\usepackage{tikz}
\usepackage{xcolor}
\usepackage{listings}
\usepackage{autobreak}
\usepackage[ruled, vlined, linesnumbered]{algorithm2e}
\usepackage{xr-hyper}
\usepackage[colorlinks,unicode]{hyperref} % , linkcolor=black, anchorcolor=black, citecolor=black, urlcolor=black, filecolor=black
\usepackage{prettyref}

% Page style
\geometry{left=3.18cm,right=3.18cm,top=2.54cm,bottom=2.54cm}
\titlespacing{\paragraph}{0pt}{1pt}{10pt}[20pt]
\setlength{\droptitle}{-5em}
\preauthor{\vspace{-10pt}\begin{center}}
\postauthor{\par\end{center}}

% More compact lists 
\setlist[itemize]{itemindent=17pt, leftmargin=1pt}

% Math operators
\DeclareMathOperator{\timeorder}{\mathcal{T}}
\DeclareMathOperator{\diag}{diag}
\DeclareMathOperator{\legpoly}{P}
\DeclareMathOperator{\primevalue}{P}
\DeclareMathOperator{\sgn}{sgn}
\newcommand*{\ii}{\mathrm{i}}
\newcommand*{\ee}{\mathrm{e}}
\newcommand*{\const}{\mathrm{const}}
\newcommand*{\suchthat}{\quad \text{s.t.} \quad}
\newcommand*{\argmin}{\arg\min}
\newcommand*{\argmax}{\arg\max}
\newcommand*{\normalorder}[1]{: #1 :}
\newcommand*{\pair}[1]{\langle #1 \rangle}
\newcommand*{\fd}[1]{\mathcal{D} #1}
\DeclareMathOperator{\bigO}{\mathcal{O}}

% TikZ setting
\usetikzlibrary{arrows,shapes,positioning}
\usetikzlibrary{arrows.meta}
\usetikzlibrary{decorations.markings}
\tikzstyle arrowstyle=[scale=1]
\tikzstyle directed=[postaction={decorate,decoration={markings,
    mark=at position .5 with {\arrow[arrowstyle]{stealth}}}}]
\tikzstyle ray=[directed, thick]
\tikzstyle dot=[anchor=base,fill,circle,inner sep=1pt]

% Algorithm setting
% Julia-style code
\SetKwIF{If}{ElseIf}{Else}{if}{}{elseif}{else}{end}
\SetKwFor{For}{for}{}{end}
\SetKwFor{While}{while}{}{end}
\SetKwProg{Function}{function}{}{end}
\SetArgSty{textnormal}

\newcommand*{\concept}[1]{{\textbf{#1}}}

% Embedded codes
\lstset{basicstyle=\ttfamily,
  showstringspaces=false,
  commentstyle=\color{gray},
  keywordstyle=\color{blue}
}

\title{QFT I, Homework 2}
\author{Jinyuan Wu}

\begin{document}

\maketitle

\paragraph{The complex scalar field} This is problem $2.2$ on p. 33 of Peskin.
Consider the field theory of a complex scalar field obeying the Klein-Gordon equation. The action of this theory is
\begin{equation}
  S=\int \dd^{4} x\left(\partial_{\mu} \phi^{*} \partial^{\mu} \phi-m^{2} \phi^{*} \phi\right).
  \label{eq:prob-1-1}
\end{equation}
It is convenient to analyze this theory by considering $\phi$ and $\phi^{*}$, rather than the real and imaginary parts of $\phi=\left(\phi_{1}+ \ii \phi_{2}\right) / \sqrt{2}$, as the basic dynamical variables.
\begin{itemize}
  \item[(a)] Find the conjugate momenta to $\phi(x)$ and $\phi^{*}(x)$ and the canonical commutation relations. Show that the Hamiltonian is
  \begin{equation}
    H=\int \dd^{3} x\left(\pi^{*} \pi+\nabla \phi^{*} \cdot \nabla \phi+m^{2} \phi^{*} \phi\right).
    \label{eq:prob-1-2}
  \end{equation}
  Compute the Heisenberg equation of motion for $\phi(x)$ and show that it is indeed the Klein-Gordon equation.
  \item[(b)] Diagonalize $H$ by introducing creation and annihilation operators. Show that the theory contains two sets of particles of mass $m$.
  \item[(c)] Rewrite the conserved charge
  \begin{equation}
    Q=\int \dd^{3} x \frac{\ii}{2}\left(\phi^{*} \pi^{*}-\pi \phi\right)
    \label{eq:prob-1-3}
  \end{equation}
  in terms of creation and annihilation operators, and evaluate the charge of the particles of each type.
  \item[(d)] Consider the case of two complex Klein-Gordon fields with the same mass. Label the fields as $\phi_{a}(x)$, where $a=1,2$. Show that there are now four conserved charges, one given by the generalization of part (c), and the other three given by
  \begin{equation}
    Q^{i}=\int \dd^{3} x \frac{\ii}{2}\left(\phi_{a}^{*}\left(\sigma^{i}\right)_{a b} \pi_{b}^{*}-\pi_{a}\left(\sigma^{i}\right)_{a b} \phi_{b}\right),
    \label{eq:prob-1-4}
  \end{equation}
  where $\sigma^{i}$ are the Pauli sigma matrices. Show that these three charges have the commutation relations of angular momentum $(S U(2))$. Generalize these results to the case of $n$ identical complex scalar fields.
\end{itemize}

\paragraph{Solution}
\begin{itemize}
    \item[(a)]  From \eqref{eq:prob-1-1} we have 
    \begin{equation}
        \pi = \pdv{\mathcal{L}}{\partial_0 \phi} = \partial^0 \phi^* = \dot{\phi}^*,
    \end{equation}
    and
    \begin{equation}
        \pi^* = \pdv{\mathcal{L}}{\partial_0 \phi^*} = \partial^0 \phi = \dot{\phi}.
    \end{equation}
    The canonical commutation relations are 
    \begin{equation}
        \begin{aligned}
            \comm*{\phi(\vb*{x}, t)}{\pi(\vb*{y}, t)} = \ii \delta^{(3)}(\vb*{x} - \vb*{y}), \quad \comm*{\phi^*(\vb*{x}, t)}{\pi^*(\vb*{y}, t)} = \ii \delta^{(3)}(\vb*{x} - \vb*{y}), \\
            \comm*{\phi(\vb*{x}, t)}{\phi(\vb*{y}, t)} = \comm*{\phi(\vb*{x}, t)}{\phi^*(\vb*{y}, t)} = \comm*{\phi^*(\vb*{x}, t)}{\phi^*(\vb*{y}, t)} = \comm*{\phi(\vb*{x}, t)}{\pi^*(\vb*{y}, t)} = 0, \\
            \comm*{\pi(\vb*{x}, t)}{\pi(\vb*{y}, t)} = \comm*{\pi(\vb*{x}, t)}{\pi^*(\vb*{y}, t)} = \comm*{\pi^*(\vb*{x}, t)}{\pi^*(\vb*{y}, t)} = \comm*{\pi(\vb*{x}, t)}{\phi^*(\vb*{y}, t)} = 0.
        \end{aligned}
        \label{eq:canonical-relation}
    \end{equation}
    The Hamiltonian is therefore 
    \[
        \begin{aligned}
            H &= \int \dd[3]{\vb*{x}} (\pi \partial_0 \phi + \pi^* \partial_0 \phi^* - \mathcal{L}) \\
            &= \int \dd[3]{\vb*{x}} ( \partial_0 \phi \partial^0 \phi^* + \partial_0 \phi^* \partial^0 \phi - (\partial_0 \phi \partial^0 \phi^* - \grad{\phi} \cdot \grad{\phi^*}) + m^2 \phi^* \phi) \\
            &= \int \dd[3]{\vb*{x}} (\dot{\phi} \dot{\phi}^* + \grad{\phi} \cdot \grad{\phi^*} + m^2 \phi \phi^*) \\
            &= \int \dd[3]{\vb*{x}} (\pi \pi^* + \grad{\phi} \cdot \grad{\phi^*} + m^2 \phi \phi^*),
        \end{aligned}
    \]
    which is exactly \eqref{eq:prob-1-2}.

    Now we try to derive equations of motion from \eqref{eq:prob-1-2}.
    The equation of motion for $\phi$ is 
    \[
        \begin{aligned}
            \dot{\phi}(\vb*{y}, t) &= \frac{1}{\ii} \comm*{\phi(\vb*{y}, t)}{H} \\
            &= \frac{1}{\ii} \int \dd[3]{\vb*{x}} (\comm*{\phi(\vb*{y}, t)}{\pi(\vb*{x}, t) \pi^*(\vb*{x}, t)} \\
            &\quad \quad + \comm*{\phi(\vb*{y}, t)}{\grad{\phi(\vb*{x}, t)} \cdot \grad{\phi^*(\vb*{x}, t)} + m^2 \phi(\vb*{x}, t) \phi^*(\vb*{x}, t)}) \\
            &= \frac{1}{\ii} \int \dd[3]{\vb*{x}} \comm*{\phi(\vb*{y}, t)}{\pi(\vb*{x}, t)} \pi^*(\vb*{x}, t) = \frac{1}{\ii} \int \dd[3]{\vb*{x}} \ii \delta^{(3)}(\vb*{x} - \vb*{y}) \pi^*(\vb*{x} , t) \\
            &= \pi^*(\vb*{y} , t), 
        \end{aligned}
    \]
    and the equation of motion for $\pi^*$ is 
    \[
        \begin{aligned}
            \dot{\pi}^*(\vb*{y}, t) &= \frac{1}{\ii} \comm*{\pi^*(\vb*{y}, t)}{H} \\
            &= \frac{1}{\ii} \int \dd[3]{\vb*{x}}  (\comm*{\pi^*(\vb*{y}, t)}{\pi(\vb*{x}, t) \pi^*(\vb*{x}, t)} \\
            &\quad \quad + \comm*{\pi^*(\vb*{y}, t)}{\grad{\phi(\vb*{x}, t)} \cdot \grad{\phi^*(\vb*{x}, t)} + m^2 \phi(\vb*{x}, t) \phi^*(\vb*{x}, t)}) \\
            &= \frac{1}{\ii} \int \dd[3]{\vb*{x}} (\grad_{\vb*{x}}{\phi(\vb*{x}, t)} \cdot \grad_{\vb*{x}}{\comm*{\pi^*(\vb*{y}, t)}{\phi^*(\vb*{x}, t)}} + m^2 \phi(\vb*{x}, t) \comm*{\pi^*(\vb*{y}, t)}{\phi^*(\vb*{x}, t)}) \\
            &= \frac{1}{\ii} \int \dd[3]{\vb*{x}} ( \grad_{\vb*{x}} \phi(\vb*{x}, t ) \grad_{\vb*{x}} (-\ii \delta^{(3)}(\vb*{x} - \vb*{y})) + m^2 \phi(\vb*{x}, t) (- \ii \delta^{(3)}(\vb*{x} - \vb*{y})) ) \\
            &= \int \dd[3]{\vb*{x}} ( - \grad_{\vb*{x}} \phi(\vb*{x}, t ) \grad_{\vb*{x}} \delta^{(3)}(\vb*{x} - \vb*{y}) - m^2 \phi(\vb*{x}, t) \delta^{(3)}(\vb*{x} - \vb*{y}) ) \\
            &= \int \dd[3]{\vb*{x}} ( \laplacian_{\vb*{x}} \phi(\vb*{x}, t ) \delta^{(3)}(\vb*{x} - \vb*{y}) - m^2 \phi(\vb*{x}, t) \delta^{(3)}(\vb*{x} - \vb*{y}) ) \\
            &= \laplacian_{\vb*{y}} \phi(\vb*{y}, t) - m^2 \phi(\vb*{y}, t),
        \end{aligned}
    \]
    so putting the two equations of motion together, we have 
    \[
        \partial_t^2 \phi(\vb*{y}, t) = \partial_t \pi^*(\vb*{y}, t) = \laplacian_{\vb*{y}} \phi(\vb*{y}, t) - m^2 \phi(\vb*{y}, t),
    \]
    or in other words
    \[
        (\partial_t^2 - \laplacian) \phi + m^2 \phi = 0,
    \]
    which is indeed the Klein-Gordon equation.
    \item[(b)] We make the following Fourier expansion at a given time $t$:
    \begin{equation}
        \phi(t, \vb*{x}) = \int \frac{\dd[3]{\vb*{p}}}{(2\pi)^3} \frac{1}{\sqrt{2 \omega_{\vb*{p}}}} (a_{\vb*{p}}(t) \ee^{\ii \vb*{p} \cdot \vb*{x}} + b^\dagger_{\vb*{p}}(t) \ee^{- \ii \vb*{p} \cdot \vb*{x}}).
        \label{eq:expansion-origin-1}
    \end{equation}
    The Klein-Gordon equation therefore reads 
    \[
        \begin{aligned}
            0 &= (\partial^2 + m^2) \phi(t, \vb*{x}) \\
            &= \int \frac{\dd[3]{\vb*{p}}}{(2\pi)^3} \frac{1}{\sqrt{2 \omega_{\vb*{p}}}} \left( (\partial_t^2 - \laplacian + m^2) a_{\vb*{p}}(t) \ee^{\ii \vb*{p} \cdot \vb*{x}} + (\partial_t^2 - \laplacian + m^2) b^\dagger_{\vb*{p}}(t) \ee^{- \ii \vb*{p} \cdot \vb*{x}} \right) \\ 
            &= \int \frac{\dd[3]{\vb*{p}}}{(2\pi)^3} \frac{1}{\sqrt{2 \omega_{\vb*{p}}}} \left( (\partial_t^2 + \vb*{p}^2 + m^2) a_{\vb*{p}}(t) \ee^{\ii \vb*{p} \cdot \vb*{x}} + (\partial_t^2 + \vb*{p}^2 + m^2) b^\dagger_{\vb*{p}}(t) \ee^{- \ii \vb*{p} \cdot \vb*{x}} \right),
        \end{aligned}
    \]
    so therefore we have 
    \[
        (\partial_t^2 + \vb*{p}^2 + m^2) a_{\vb*{p}}(t) = (\partial_t^2 + \vb*{p}^2 + m^2) b^\dagger_{\vb*{p}}(t) = 0,
    \]
    the solution of which are 
    \begin{equation}
        a_{\vb*{p}}(t) = \ee^{\ii \omega_{\vb*{p}} t} a_{\vb*{p} 1} + \ee^{- \ii \omega_{\vb*{p}} t} a_{\vb*{p} 2}, \quad b^\dagger_{\vb*{p}}(t) = \ee^{\ii \omega_{\vb*{p}} t} b^\dagger_{\vb*{p} 1} + \ee^{- \ii \omega_{\vb*{p}} t} b^\dagger_{\vb*{p} 2},
        \label{eq:expansion-origin-2}
    \end{equation}
    where 
    \[
        -\omega_{\vb*{p}}^2 + \vb*{p}^2 + m^2 = 0. 
    \]
    So by \eqref{eq:expansion-origin-1} and \eqref{eq:expansion-origin-2} we have 
    \[
        \begin{aligned}
            \phi(t, \vb*{x}) &= \int \frac{\dd[3]{\vb*{p}}}{(2\pi)^3} \frac{1}{\sqrt{2 \omega_{\vb*{p}}}} \left(  (\ee^{\ii \omega_{\vb*{p}} t} a_{\vb*{p} 1} + \ee^{- \ii \omega_{\vb*{p}} t} a_{\vb*{p} 2}) \ee^{\ii \vb*{p} \cdot \vb*{x}} + (\ee^{\ii \omega_{\vb*{p}} t} b^\dagger_{\vb*{p} 1} + \ee^{- \ii \omega_{\vb*{p}} t} b^\dagger_{\vb*{p} 2}) \ee^{- \ii \vb*{p} \cdot \vb*{x}} \right) \\
            &= \int \frac{\dd[3]{\vb*{p}}}{(2\pi)^3} \frac{1}{\sqrt{2 \omega_{\vb*{p}}}} \left( (a_{\vb*{p}2} + b^\dagger_{- \vb*{p}, 2}) \ee^{-\ii \omega_{\vb*{p}} t + \ii \vb*{p} \cdot \vb*{x}} + (a_{-\vb*{p} ,1} + b^\dagger_{\vb*{p} 1}) \ee^{\ii \omega_{\vb*{p}} t - \ii \vb*{p} \cdot \vb*{x}} \right).
        \end{aligned}
    \]
    Therefore by redefining $a$ and $b$ operators we have the expansion
    \begin{equation}
        \phi(x) = \int \frac{\dd[3]{\vb*{p}}}{(2\pi)^3} \frac{1}{\sqrt{2 \omega_{\vb*{p}}}} (a_{\vb*{p}} \ee^{- \ii p \cdot x} + b^\dagger_{\vb*{p}} \ee^{\ii p \cdot x}),
    \end{equation} 
    so the Fourier expansion of $\phi^*$ is
    \begin{equation}
        \phi^*(x) = \int \frac{\dd[3]{\vb*{p}}}{(2\pi)^3} \frac{1}{\sqrt{2\omega_{\vb*{p}}}} (a^\dagger_{\vb*{p}} \ee^{\ii p \cdot x} + b_{\vb*{p}} \ee^{- \ii p \cdot x}).
    \end{equation} 
    where $p$ is on-shell, i.e.
    \begin{equation}
        p^\mu = (\pm \omega_{\vb*{p}}, \vb*{p}), \quad p^2 = m^2, \quad \omega_{\vb*{p}} = \omega_{-\vb*{p}}= \sqrt{{\vb*{p}}^2 + m^2}.
    \end{equation}
    and the Fourier expansion of the two conjugate momenta are 
    \begin{equation}
        \pi(x) = \dot{\phi}^*(x) = \int \frac{\dd[3]{\vb*{p}}}{(2\pi)^3} \ii \sqrt{\frac{\omega_{\vb*{p}}}{2}} (a^\dagger_{\vb*{p}} \ee^{\ii p \cdot x} - b_{\vb*{p}} \ee^{- \ii p \cdot x}),
    \end{equation}
    and 
    \begin{equation}
        \pi^*(x) = \dot{\phi}(x) = \int \frac{\dd[3]{\vb*{p}}}{(2\pi)^3} \ii \sqrt{\frac{\omega_{\vb*{p}}}{2}} (- a_{\vb*{p}} \ee^{- \ii p \cdot x} + b^\dagger_{\vb*{p}} \ee^{\ii p \cdot x}).
    \end{equation}

    Now we try to derive the commutation relations of $a_{\vb*{p}}$s and $b_{\vb*{p}}$s.
    We will insert 
    \begin{equation}
        \begin{aligned}
            \comm*{a_{\vb*{p}}}{a^\dagger_{\vb*{p}'}} = (2\pi)^3 \delta^{(3)}(\vb*{p} - \vb*{p}'), \quad \comm*{b_{\vb*{p}}}{b^\dagger_{\vb*{p}'}} = (2\pi)^3 \delta^{(3)}(\vb*{p} - \vb*{p}') , \\
            \comm*{a_{\vb*{p}}}{a_{\vb*{p}'}} = \comm*{b_{\vb*{p}}}{b_{\vb*{p}'}} = \comm*{a^\dagger_{\vb*{p}}}{a^\dagger_{\vb*{p}'}} = \comm*{b^\dagger_{\vb*{p}}}{b^\dagger_{\vb*{p}'}} = \comm*{a_{\vb*{p}}}{b_{\vb*{p}'}}  = \comm*{a^\dagger_{\vb*{p}}}{b^\dagger_{\vb*{p}'}} = \comm*{a_{\vb*{p}}}{b^\dagger_{\vb*{p}'}} = \comm*{a_{\vb*{p}}^\dagger}{b_{\vb*{p}'}} = 0 
        \end{aligned}
        \label{eq:a-b-relation}
    \end{equation}
    into these four expansions, and verify whether \eqref{eq:canonical-relation} holds.
    If so, then \eqref{eq:a-b-relation} holds also because the relation between the creation and annihilation operators and the field operators are linear and therefore \eqref{eq:a-b-relation} and \eqref{eq:canonical-relation} must be equivalent if one of them implies the other.
    Assuming \eqref{eq:a-b-relation} to be true, we have 
    \[
        \begin{aligned}
            &\quad \comm*{\phi(\vb*{x}, t)}{\pi(\vb*{y}, t)} \\
            &= \int \frac{\dd[3]{\vb*{p}}}{(2\pi)^3} \int \frac{\dd[3]{\vb*{p}'}}{(2\pi)^3} \ii \frac{1}{\sqrt{2 \omega_{\vb*{p}}}} \sqrt{\frac{\omega_{\vb*{p}'}}{2}} \left( \comm*{a_{\vb*{p}}}{ a^\dagger_{\vb*{p}'}} \ee^{- \ii p \cdot x + \ii p' \cdot y} + \comm*{b^\dagger_{\vb*{p}}}{- b_{\vb*{p}'}} \ee^{\ii p \cdot x - \ii p' \cdot y} \right) \\
            &= \int \frac{\dd[3]{\vb*{p}}}{(2\pi)^3} \int \frac{\dd[3]{\vb*{p}'}}{(2\pi)^3} \ii \frac{1}{\sqrt{2 \omega_{\vb*{p}}}} \sqrt{\frac{\omega_{\vb*{p}'}}{2}} \left( (2\pi)^3 \delta^{(3)}(\vb*{p} - \vb*{p}') \ee^{\ii \vb*{p} \cdot \vb*{x} - \ii \vb*{p}' \cdot \vb*{y}} + (2\pi)^3 \delta^{(3)}(\vb*{p} - \vb*{p}') \ee^{- \ii \vb*{p} \cdot \vb*{x} + \ii \vb*{p}' \cdot \vb*{y}} \right) \\
            &= \int \frac{\dd[3]{\vb*{p}}}{(2\pi)^3} \ii \frac{1}{\sqrt{2\omega_{\vb*{p}}}} \sqrt{\frac{\omega_{\vb*{p}}}{2}} (\ee^{\ii \vb*{p} \cdot (\vb*{x} - \vb*{y})} + \ee^{- \ii \vb*{p} \cdot (\vb*{x} - \vb*{y})}) \\
            &= \frac{\ii}{2} \int \frac{\dd[3]{\vb*{p}}}{(2\pi)^3} (\ee^{\ii \vb*{p} \cdot (\vb*{x} - \vb*{y})} + \ee^{- \ii \vb*{p} \cdot (\vb*{x} - \vb*{y})}) = \frac{\ii}{2} (\delta^{(3)}(\vb*{x} - \vb*{y}) + \delta^{(3)}(\vb*{x} - \vb*{y})) \\
            &= \ii \delta^{(3)}(\vb*{x} - \vb*{y}),
        \end{aligned}
    \]
    The first equation holds because of the second line in \eqref{eq:a-b-relation}.
    Similarly we can show that $\comm*{\phi^*(\vb*{x}, t)}{\pi^*(\vb*{y}, t)} = \ii \delta^{(3)}(\vb*{x} - \vb*{y})$.
    Simply by the fact that $a$ and $b^\dagger$ commutes we find that 
    \[
        \comm*{\phi(\vb*{x}, t)}{\phi(\vb*{y}, t)} = \comm*{\phi^*(\vb*{x}, t)}{\phi^*(\vb*{y}, t)} = \comm*{\pi(\vb*{x}, t)}{\pi(\vb*{y}, t)} = \comm*{\pi^*(\vb*{x}, t)}{\pi^*(\vb*{y}, t)}  = 0.
    \]
    Besides, we have 
    \[
        \begin{aligned}
            &\quad \comm*{\phi(\vb*{x}, t)}{\phi^*(\vb*{y}, t)} \\
            &=  \int \frac{\dd[3]{\vb*{p}}}{(2\pi)^3} \int \frac{\dd[3]{\vb*{p}'}}{(2\pi)^3} \frac{1}{\sqrt{2\omega_{\vb*{p}'}}} \frac{1}{\sqrt{2 \omega_{\vb*{p}}}} (\comm*{a_{\vb*{p}}}{a^\dagger_{\vb*{p}'}} \ee^{- \ii p \cdot x + \ii p' \cdot y} + \comm*{b^\dagger_{\vb*{p}}}{b_{\vb*{p}'}} \ee^{\ii p \cdot x- \ii p' \cdot y})  \\
            &= \int \frac{\dd[3]{\vb*{p}}}{(2\pi)^3} \int \frac{\dd[3]{\vb*{p}'}}{(2\pi)^3} \frac{1}{\sqrt{2\omega_{\vb*{p}'}}} \frac{1}{\sqrt{2 \omega_{\vb*{p}}}} ( (2\pi)^3 \delta^{(3)}(\vb*{p} - \vb*{p}') \ee^{\ii \vb*{p} \cdot \vb*{x} - \ii \vb*{p}' \cdot \vb*{y}} - (2\pi)^3 \delta^{(3)}(\vb*{p} - \vb*{p}') \ee^{- \ii \vb*{p} \cdot \vb*{x} + \ii \vb*{p}' \cdot \vb*{y}}) \\
            &=  \int \frac{\dd[3]{\vb*{p}}}{(2\pi)^3}  \frac{1}{2 \omega_{\vb*{p}}} (\ee^{\ii \vb*{p} (\vb*{x} - \vb*{y})} - \ee^{- \ii \vb*{p} \cdot (\vb*{x} - \vb*{y})}) \\
            &=  \int \frac{\dd[3]{\vb*{p}}}{(2\pi)^3}  \frac{1}{2 \omega_{\vb*{p}}} (\ee^{\ii \vb*{p} (\vb*{x} - \vb*{y})} - \ee^{\ii \vb*{p} \cdot (\vb*{x} - \vb*{y})}) = 0.
        \end{aligned}
    \]
    Replacing $1 / \sqrt{2 \omega_{\vb*{p}}}$ with $\ii \sqrt{\omega_{\vb*{p}} / 2}$, we have also shown that $\comm*{\pi(\vb*{x}, t)}{\pi^*(\vb*{y} ,t)} = 0$.
    Now the only remaining commutators in \eqref{eq:canonical-relation} not shown are 
    \[
        \comm*{\phi(\vb*{x}, t)}{\pi^*(\vb*{y}, t)} = \comm*{\phi^*(\vb*{x}, t)}{\pi(\vb*{y}, t)} = 0.
    \]
    the proof of which are trivial because $a_{\vb*{p}}$ and $b^\dagger_{\vb*{p}}$ commute.
    So now we find that \eqref{eq:canonical-relation} hold under the assumption that \eqref{eq:a-b-relation} hold, so \eqref{eq:a-b-relation} and \eqref{eq:canonical-relation} are equivalent.

    Inserting the four expansions into \eqref{eq:prob-1-2} we obtain 
    \[
        \begin{aligned}
            H &= \int \dd[3]{\vb*{x}} \left(\pi^{*} \pi+\nabla \phi^{*} \cdot \nabla \phi+m^{2} \phi^{*} \phi\right) \\
            &= \int \dd[3]{\vb*{x}} \int \frac{\dd[3]{\vb*{p}}}{(2\pi)^3} \int \frac{\dd[3]{\vb*{p}'}}{(2\pi)^3} \Big( \ii \sqrt{\frac{\omega_{\vb*{p}}}{2}} (a^\dagger_{\vb*{p}} \ee^{\ii p \cdot x} - b_{\vb*{p}} \ee^{- \ii p \cdot x}) \times \ii \sqrt{\frac{\omega_{\vb*{p}'}}{2}} (- a_{\vb*{p}'} \ee^{- \ii p' \cdot x} + b^\dagger_{\vb*{p}'} \ee^{\ii p' \cdot x})  \\
            &\quad \quad + \frac{1}{\sqrt{2\omega_{\vb*{p}}}} ( - \ii \vb*{p} a^\dagger_{\vb*{p}} \ee^{\ii p \cdot x} + \ii \vb*{p} b_{\vb*{p}} \ee^{- \ii p \cdot x}) \cdot \frac{1}{\sqrt{2\omega_{\vb*{p}'}}} ( \ii \vb*{p}' a_{\vb*{p}'} \ee^{- \ii p' \cdot x} - \ii \vb*{p}' b^\dagger_{\vb*{p}'} \ee^{\ii p' \cdot x})  \\
            &\quad \quad + m^2 \frac{1}{\sqrt{2\omega_{\vb*{p}}}} (a^\dagger_{\vb*{p}} \ee^{\ii p \cdot x} + b_{\vb*{p}} \ee^{- \ii p \cdot x}) \times \frac{1}{\sqrt{2\omega_{\vb*{p}'}}} (a_{\vb*{p}'} \ee^{- \ii p' \cdot x} + b^\dagger_{\vb*{p}'} \ee^{\ii p' \cdot x}) \Big) \\
            &= \int \frac{\dd[3]{\vb*{p}}}{(2\pi)^3} \Big(- \frac{\omega_{\vb*{p}}}{2} (- a^\dagger_{\vb*{p}} a_{\vb*{p}} +a^\dagger_{\vb*{p}} b^\dagger_{-\vb*{p}} - b_{\vb*{p}} b^\dagger_{\vb*{p}} + b_{\vb*{p}} a_{-\vb*{p}} )  \\
            &\quad \quad + \frac{1}{2\omega_{\vb*{p}}} \vb*{p}^2 (a^\dagger_{\vb*{p}} a_{\vb*{p}} + a^\dagger_{\vb*{p}} b^\dagger_{-\vb*{p}} + b_{\vb*{p}} a_{-\vb*{p}} + b_{\vb*{p}} b^\dagger_{\vb*{p}}) + \frac{m^2}{2 \omega_{\vb*{p}}} (a_{\vb*{p}}^\dagger a_{\vb*{p}} + a^\dagger_{\vb*{p}} b^\dagger_{- \vb*{p}} + b_{\vb*{p}} a_{- \vb*{p}} + b_{\vb*{p}} b^\dagger_{\vb*{p}} ) \Big) \\
            &= \int \frac{\dd[3]{\vb*{p}}}{(2\pi)^3} \frac{\omega_{\vb*{p}}}{2} ( a^\dagger_{\vb*{p}} a_{\vb*{p}} - a^\dagger_{\vb*{p}} b^\dagger_{-\vb*{p}} + b_{\vb*{p}} b^\dagger_{\vb*{p}} - b_{\vb*{p}} a_{-\vb*{p}} + a_{\vb*{p}}^\dagger a_{\vb*{p}} + a^\dagger_{\vb*{p}} b^\dagger_{- \vb*{p}} + b_{\vb*{p}} a_{- \vb*{p}} + b_{\vb*{p}} b^\dagger_{\vb*{p}} ) \\
            &= \int \frac{\dd[3]{\vb*{p}}}{(2\pi)^3} \omega_{\vb*{p}} (a^\dagger_{\vb*{p}} a_{\vb*{p}} + b_{\vb*{p}} b^\dagger_{\vb*{p}}).
        \end{aligned}
    \]
    The third equation is obtained by integrating $x$ and then integrating $\vb*{p}$: since all operators are defined on the same time point, we have 
    \[
        \int \dd[3]{\vb*{x}} \ee^{\ii (p - p') \cdot x} = (2\pi)^3 \delta^{(3)}(\vb*{p} - \vb*{p}'),
    \]
    and then by integrating $\vb*{p}$ we just replace all $\vb*{p}'$ with $\vb*{p}$.
    Now we exchange $b_{\vb*{p}}$ and $b^\dagger_{\vb*{p}}$, we have 
    \begin{equation}
        \begin{aligned}
            H &= \int \frac{\dd[3]{\vb*{p}}}{(2\pi)^3} \omega_{\vb*{p}} (a^\dagger_{\vb*{p}} a_{\vb*{p}} + b^\dagger_{\vb*{p}} b_{\vb*{p}} + \comm*{b_{\vb*{p}}}{b^\dagger_{\vb*{p}}}) \\
            &= \int \frac{\dd[3]{\vb*{p}}}{(2\pi)^3} \omega_{\vb*{p}} (a^\dagger_{\vb*{p}} a_{\vb*{p}} + b^\dagger_{\vb*{p}} b_{\vb*{p}} + (2\pi)^3 \delta^{(3)}(0)).
        \end{aligned}
    \end{equation}
    This is the diagonalized Hamiltonian. It can be seen that we have two sets of particles here, and both of them have $\omega_{\vb*{p}}$ as the energy spectrum, whose mass is $m$.

    \item[(c)] We have 
    \[
        \begin{aligned}
            Q &= \int \dd[3]{\vb*{x}} \frac{\ii}{2} (\phi^* \pi^* - \pi \phi) \\
            &= \int \dd[3]{\vb*{x}} \int \frac{\dd[3]{\vb*{p}}}{(2\pi)^3}  \int \frac{\dd[3]{\vb*{p}'}}{(2\pi)^3}\frac{\ii}{2} \Big( \ii \frac{1}{\sqrt{2 \omega_{\vb*{p}}}} \sqrt{\frac{\omega_{\vb*{p}'}}{2}} (a^\dagger_{\vb*{p}} \ee^{\ii p \cdot x} + b_{\vb*{p}} \ee^{- \ii p \cdot x} ) (- a_{\vb*{p}'} \ee^{- \ii p' \cdot x} + b^\dagger_{\vb*{p}'} \ee^{\ii p' \cdot x}) \Big) + \text{h.c.} \\
            &= \int \frac{\dd[3]{\vb*{p}}}{(2\pi)^3} \left( - \frac{1}{4} \right) ( - a_{\vb*{p}}^\dagger a_{\vb*{p}} + b_{\vb*{p}} b^\dagger_{\vb*{p}} - b_{\vb*{p}} a_{- \vb*{p}} + a^\dagger_{\vb*{p}} b^\dagger_{- \vb*{p}} ) + \text{h.c.} \\
            &= \int \frac{\dd[3]{\vb*{p}}}{(2\pi)^3} \left( - \frac{1}{4} \right) ( - a_{\vb*{p}}^\dagger a_{\vb*{p}} + b_{\vb*{p}} b^\dagger_{\vb*{p}} - b_{\vb*{p}} a_{- \vb*{p}} + a^\dagger_{\vb*{p}} b^\dagger_{- \vb*{p}}  - a_{\vb*{p}}^\dagger a_{\vb*{p}} + b_{\vb*{p}} b^\dagger_{\vb*{p}} - a_{- \vb*{p}}^\dagger b^\dagger_{\vb*{p}} + b_{- \vb*{p}} a_{\vb*{p}} ) \\
            &= \int \frac{\dd[3]{\vb*{p}}}{(2\pi)^3} \left( - \frac{1}{4} \right) ( - a_{\vb*{p}}^\dagger a_{\vb*{p}} + b_{\vb*{p}} b^\dagger_{\vb*{p}} - b_{\vb*{p}} a_{- \vb*{p}} + a^\dagger_{\vb*{p}} b^\dagger_{- \vb*{p}}  - a_{\vb*{p}}^\dagger a_{\vb*{p}} + b_{\vb*{p}} b^\dagger_{\vb*{p}} - a_{\vb*{p}}^\dagger b^\dagger_{- \vb*{p}} + b_{ \vb*{p}} a_{- \vb*{p}} ) \\
            &= \frac{1}{2} \int \frac{\dd[3]{\vb*{p}}}{(2\pi)^3} (a^\dagger_{\vb*{p}} a_{\vb*{p}} - b_{\vb*{p}} b^\dagger_{\vb*{p}}),
        \end{aligned}
    \] 
    where we have used a procedure similar to the one we use to derive the Hamiltonian.
    Again, by exchange $b_{\vb*{p}}$ and $b^\dagger_{\vb*{p}}$, we have 
    
\end{itemize}

\paragraph{}

\paragraph{Correlation function of real scalar field} This is problem $2.3$ on p. 34 of Peskin. Evaluate the function. 
\[
\langle 0|\phi(x) \phi(y)| 0\rangle=D(x-y)=\int \frac{\dd^{3} p}{(2 \pi)^{3}} \frac{1}{2 E_{\vb*{p}}} e^{-\ii p \cdot (x-y)}
\]
for $(x-y)$ spacelike so that $(x-y)^{2}=-r^{2}$, explicitly in terms of Bessel functions.

\end{document}