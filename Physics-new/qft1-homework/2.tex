\documentclass[hyperref, a4paper]{article}

\usepackage{geometry}
\usepackage{titling}
\usepackage{titlesec}
% No longer needed, since we will use enumitem package
% \usepackage{paralist}
\usepackage{enumitem}
\usepackage{footnote}
\usepackage{enumerate}
\usepackage{amsmath, amssymb, amsthm}
\usepackage{mathtools}
\usepackage{bbm}
\usepackage{cite}
\usepackage{graphicx}
\usepackage{subfigure}
\usepackage{physics}
\usepackage{tensor}
\usepackage{siunitx}
\usepackage[version=4]{mhchem}
\usepackage{tikz}
\usepackage{xcolor}
\usepackage{listings}
\usepackage{autobreak}
\usepackage[ruled, vlined, linesnumbered]{algorithm2e}
\usepackage{xr-hyper}
\usepackage[colorlinks,unicode]{hyperref} % , linkcolor=black, anchorcolor=black, citecolor=black, urlcolor=black, filecolor=black
\usepackage{prettyref}

% Page style
\geometry{left=3.18cm,right=3.18cm,top=2.54cm,bottom=2.54cm}
\titlespacing{\paragraph}{0pt}{1pt}{10pt}[20pt]
\setlength{\droptitle}{-5em}
\preauthor{\vspace{-10pt}\begin{center}}
\postauthor{\par\end{center}}

% More compact lists 
\setlist[itemize]{itemindent=17pt, leftmargin=1pt}

% Math operators
\DeclareMathOperator{\timeorder}{\mathcal{T}}
\DeclareMathOperator{\diag}{diag}
\DeclareMathOperator{\legpoly}{P}
\DeclareMathOperator{\primevalue}{P}
\DeclareMathOperator{\sgn}{sgn}
\newcommand*{\ii}{\mathrm{i}}
\newcommand*{\ee}{\mathrm{e}}
\newcommand*{\const}{\mathrm{const}}
\newcommand*{\suchthat}{\quad \text{s.t.} \quad}
\newcommand*{\argmin}{\arg\min}
\newcommand*{\argmax}{\arg\max}
\newcommand*{\normalorder}[1]{: #1 :}
\newcommand*{\pair}[1]{\langle #1 \rangle}
\newcommand*{\fd}[1]{\mathcal{D} #1}
\DeclareMathOperator{\bigO}{\mathcal{O}}

% TikZ setting
\usetikzlibrary{arrows,shapes,positioning}
\usetikzlibrary{arrows.meta}
\usetikzlibrary{decorations.markings}
\tikzstyle arrowstyle=[scale=1]
\tikzstyle directed=[postaction={decorate,decoration={markings,
    mark=at position .5 with {\arrow[arrowstyle]{stealth}}}}]
\tikzstyle ray=[directed, thick]
\tikzstyle dot=[anchor=base,fill,circle,inner sep=1pt]

% Algorithm setting
% Julia-style code
\SetKwIF{If}{ElseIf}{Else}{if}{}{elseif}{else}{end}
\SetKwFor{For}{for}{}{end}
\SetKwFor{While}{while}{}{end}
\SetKwProg{Function}{function}{}{end}
\SetArgSty{textnormal}

\newcommand*{\concept}[1]{{\textbf{#1}}}

% Embedded codes
\lstset{basicstyle=\ttfamily,
  showstringspaces=false,
  commentstyle=\color{gray},
  keywordstyle=\color{blue}
}

\title{QFT I, Homework 2}
\author{Jinyuan Wu}

\begin{document}

\maketitle

\paragraph{The complex scalar field} This is problem $2.2$ on p. 33 of Peskin.
Consider the field theory of a complex scalar field obeying the Klein-Gordon equation. The action of this theory is
\[
S=\int d^{4} x\left(\partial_{\mu} \phi^{*} \partial^{\mu} \phi-m^{2} \phi^{*} \phi\right)
\]
It is convenient to analyze this theory by considering $\phi$ and $\phi^{*}$, rather than the real and imaginary parts of $\phi=\left(\phi_{1}+i \phi_{2}\right) / \sqrt{2}$, as the basic dynamical variables.
\begin{itemize}
  \item[(a)] Find the conjugate momenta to $\phi(x)$ and $\phi^{*}(x)$ and the canonical commutation relations. Show that the Hamiltonian is
  \[
  H=\int d^{3} x\left(\pi^{*} \pi+\nabla \phi^{*} \cdot \nabla \phi+m^{2} \phi^{*} \phi\right)
  \]
  Compute the Heisenberg equation of motion for $\phi(x)$ and show that it is indeed the Klein-Gordon equation.
  \item[(b)] Diagonalize $H$ by introducing creation and annihilation operators. Show that the theory contains two sets of particles of mass $m$.
  \item[(c)] Rewrite the conserved charge
  \[
  Q=\int d^{3} x \frac{i}{2}\left(\phi^{*} \pi^{*}-\pi \phi\right)
  \]
  in terms of creation and annihilation operators, and evaluate the charge of the particles of each type.
  \item[(d)] Consider the case of two complex Klein-Gordon fields with the same mass. Label the fields as $\phi_{a}(x)$, where $a=1,2$. Show that there are now four conserved charges, one given by the generalization of part (c), and the other three given by
  \[
  Q^{i}=\int d^{3} x \frac{i}{2}\left(\phi_{a}^{*}\left(\sigma^{i}\right)_{a b} \pi_{b}^{*}-\pi_{a}\left(\sigma^{i}\right)_{a b} \phi_{b}\right)
  \]
  where $\sigma^{i}$ are the Pauli sigma matrices. Show that these three charges have the commutation relations of angular momentum $(S U(2))$. Generalize these results to the case of $n$ identical complex scalar fields.
\end{itemize}

\paragraph{Solution}

\paragraph{}



\end{document}