\chapter{牛顿力学的量子化}

\section{对称性分析}

量子化的基本手续是将出现在哈密顿量中的动力学自由度替换成厄米算符,将泊松括号替换成对易子。

实验表明,电子的自由度包括轨道自由度和自旋自由度,自旋$1/2$。

\section{薛定谔方程}

轨道部分在坐标表象下,坐标和动理动量分别是
\begin{equation}
    {\vb*{r}} = \vb*{r}, \quad {\vb*{p}} = -\ii \hbar \grad.
    \label{eq:single-electron-schodinger}
\end{equation}
轨道部分的动力学由单电子薛定谔方程描述。在没有磁场时,单电子的薛定谔方程是
\begin{equation}
    \ii \hbar \pdv{\psi}{t} = - \frac{\hbar^2 \laplacian}{2m} \psi + V \psi,
\end{equation}
从中可以得到概率密度为
\begin{equation}
    \rho = \abs*{\psi}^2,
\end{equation}
概率流密度为
\begin{equation}
    \vb*{j} = \frac{\ii \hbar}{2m} (\psi \grad{\psi^*} - \psi^* \grad{\psi}).
\end{equation}

\subsection{波列}

在$V$为常数时,\eqref{eq:single-electron-schodinger}的平面波解和色散关系如下:
\begin{equation}
    \psi = \ee^{\ii(\vb*{k} \cdot \vb*{r} - \omega t)}, \quad \hbar \omega = \frac{\hbar^2 k^2}{2m} + V.
    \label{eq:plain-wave-solution}
\end{equation}
可以看到,波矢$\vb*{k}$和动量$\vb*{p}$只差了一个因子$\hbar$。
这当然是正确的,实际上从一个局域近似为平面波的波形(如球面波、柱面波)中提取波矢的方式就是作用算符$-\grad$然后除以原波形。
\eqref{eq:plain-wave-solution}是一个定态解,其能量就是$\hbar \omega$。
\eqref{eq:plain-wave-solution}的波长称为\concept{德布罗意波长},为
\begin{equation}
    \lambda = \frac{2\pi}{k} = \frac{h}{p} = \frac{h}{\sqrt{2m(E-V)}}.
\end{equation}
在我们讨论的问题的尺度远大于德布罗意波长时,\eqref{eq:plain-wave-solution}的波动性实际上并不明显,此时可以使用经典理论处理问题,否则必须使用量子力学。

位置和动量的对易关系意味着在一次测量中不可能同时确定地得到它们的值,因为这两个算符没有共同本征态,从而
\begin{equation}
    \Delta x \Delta p_x \geq \frac{\hbar}{2}.
    \label{eq:position-momentum-uncertainty}
\end{equation}
虽然时间不是算符,但如果由于相互作用等原因,能级发生展宽,那么相应的,系统停留在这个能级上的时间也不会是无限大的,当然也不会是无限小的,由傅里叶变换关于函数定域性的理论可以推导出
\begin{equation}
    \Delta t \Delta E \geq \frac{\hbar}{2}.
    \label{eq:time-energy-uncertainty}
\end{equation}
实际上,也可以从傅里叶变换的观点推导出\eqref{eq:position-momentum-uncertainty},但是要注意它和\eqref{eq:time-energy-uncertainty}的物理意义是不同的,因为没有对应于时间的算符。

\subsection{常用常数}

普朗克常数
\[
    h = \SI{6.62607004e-34}{J.s}, \quad \hbar = \SI{1.05457266e-34}{J.s},
\]
元电荷
\[
    e = \SI{1.602176634e-19}{C},
\]
电子质量
\[
    m_\text{e} = \SI{9.10938356e-31}{kg},
\]
质子质量
\[
    m_\text{p} = \SI{1.6726219e-27}{kg},
\]
中子质量
\[
    m_\text{n} = \SI{1.674927471e-27}{kg}
\]

\subsection{斯特恩-盖拉赫实验}

考虑一束带有磁矩的粒子穿过一个基本上只在$z$方向有梯度的磁场(这样的一个装置称为一个\concept{SG仪器}),显然,由于磁矩的存在,在经典图景下会引入这样一个磁场力:
\begin{equation}
    \vb*{F} = \pdv{(\vb*{\mu} \cdot \vb*{B})}{z} \vb*{e}_z = \mu_z \pdv{B_z}{z} \vb*{e}_z.
\end{equation}
如果粒子束的生成装置当中不存在特殊的磁矩方向(通常会用一个炉子加热原子,然后产生一个射流,因此的确不存在特殊的磁矩方向),那么,预期的结果应该是经过磁场之后的粒子束弥散为一个连续的竖条。
但实际情况并非如此:束流劈裂成了两束,一束向上,一束向下,即似乎$z$方向磁矩只有两个离散的取值。
由于磁矩正比于自旋,显然$z$方向自旋也应该只有两个离散的取值。

现在只取一束束流,不妨认为其自旋向上,那么,将它通过一个有$z$方向磁场的SG仪器,束流按理来说不会进一步劈裂——实际上也的确如此。
但如果将它通过一个有$x$方向磁场的SG仪器,在经典图像下由于束流不存在$x$方向的磁矩,应该什么也不会发生,但实际上实验结果是,束流在$x$方向发生了劈裂。
显然,$z$方向自旋全部指向一个方向的束流的$x$方向自旋还是可以有两个取值。

也许有这样的可能,就是自旋在三个方向上都有两种取值,这样,$z$方向自旋全部指向一个方向的束流在$x$方向上的自旋可以同时有两个取值。
那么,将一个粒子束依次通过$z$方向的SG装置和$x$方向的SG装置应该能够得到$x$方向和$z$方向自旋都完全确定的束流。
实际情况并非如此:一个粒子束依次通过$z$方向的SG装置和$x$方向的SG装置之后得到的束流再次通过一个$z$方向的SG装置,照样出现了劈裂。

在量子力学的框架下解释这样的现象就非常容易。SG装置实际上造成了自旋和轨道的耦合,它是一个使用粒子的轨道去测量粒子自己的自旋的装置。
因此,当一个粒子经过一个某一方向SG装置后,它的自旋角动量塌缩到这一方向的自旋角动量分量的本征态上。
自旋算符的本征值是分立的,自然造成分立的测量结果,也就是粒子束流劈裂。
由于不同方向的自旋算符不对易,$z$方向的自旋算符的本征态不是$x$方向的自旋算符的本征态,从而造成之前看到的$z$方向自旋确定的束流经过$x$方向的SG装置之后再次发生劈裂的情况。
先经过$z$方向的SG装置再经过$x$方向的SG装置的束流的每个粒子都塌缩到了$x$方向自旋算符的本征态上,于是经过$z$方向的SG装置之后再次发生劈裂。

\subsection{电子-光子过程}

能标从低到高有三种过程:光电效应、康普顿散射和电子对产生。

经典电动力学不能够解释为什么一些情况下向自由电子入射光会得到频率不同的反射光——在镜子里,蓝色的衣服当然不可能是红色的。但使用X射线照射一些物质后,确实能够得到频率发生变化的散射光。这个效应就是\concept{康普顿效应}。
如果认为电磁波实际上是粒子,并且使用经典狭义相对论的动量和动能公式,设入射光子和出射光子的动量分别是$\vb*{p}_0$和$\vb*{p}$,两者夹角为$\theta$,碰撞之后电子的速度为$\vb*{v}$,则
\[
    \vb*{p}_0 = \vb*{p} + \frac{m \vb*{v}}{\sqrt{1 - v^2/c^2}}, \quad \abs{\vb*{p}_0} c + m c^2 = \abs{\vb*{p}} c + \frac{m c^2}{\sqrt{1 - v^2/c^2}},
\]
考虑到波长为
\[
    \lambda = \frac{h}{p},
\]
得到
\begin{equation}
    \lambda - \lambda_0 = \frac{h}{mc} (1 - \cos \theta),
    \label{eq:compton-movement}
\end{equation}
设散射强的光子能量为$\epsilon_0$,则散射之后的光子能量为
\begin{equation}
    \epsilon = \frac{\epsilon_0}{1 + \frac{\epsilon_0}{m c^2} (1 - \cos \theta)} \geq \frac{\epsilon_0}{1 + \frac{2\epsilon_0}{m c^2}},
\end{equation}
反冲电子动能为
\begin{equation}
    T = \epsilon_0 - \epsilon = \epsilon_0 \frac{\frac{\epsilon_0}{mc^2}(1-\cos \theta)}{1+\frac{\epsilon_0}{mc^2}(1-\cos \theta)} \leq \epsilon_0 \frac{\frac{2 \epsilon_0}{mc^2}}{1+\frac{2 \epsilon_0}{mc^2}}.
\end{equation}
显然,只有在狭义相对论需要被完整地考虑的时候,才会出现康普顿效应\eqref{eq:compton-movement},在牛顿时空观已经够用的时候,取$c \to \infty$,康普顿效应就消失了。
\eqref{eq:compton-movement}给出了一个自然的长度尺度,即\concept{康普顿波长}:
\begin{equation}
    \lambda = \frac{h}{mc},
\end{equation}
它是粒子之间的物理过程中相对论效应需要被完整考虑的长度尺度,低于这个长度,相对论效应非常明显,从而实际上需要使用量子电动力学分析问题。
德布罗意波长给出了实物粒子的量子效应开始变得明显的长度尺度,而康普顿波长给出了相对论效应开始变得明显的长度尺度。

我们会注意到我们假设电子和入射光子发生相互作用之后一定会产生一个出射光子。实际上,不可能发生没有出射光子的过程,因为一个入射光子完全被电子吸收的过程需要满足
\[
    \vb*{p}_0 = \frac{m \vb*{v}}{\sqrt{1 - v^2/c^2}}, \quad p_0 c = \frac{m c^2}{\sqrt{1 - v^2 / c^2}},
\]
这个方程组意味着$v=c$,而这是不可能的——电子不可能被加速到光速。

需要注意的是光电效应和上述结论没有矛盾,因为发生光电效应时电子处于束缚态,因此电子吸收不了的能量可以用于挣脱束缚态。
“自由”和“束缚”的概念在这里并不是绝对的:如果入射光的单个光子的能量足够强,那么即使一开始电子处于束缚态,束缚能也是可以忽略的。
这也就是如果使用普通的晶体做靶标,通常要发生康普顿效应需要X射线的原因。
实际上,用X射线做的康普顿效应实验通常也会得到波长没有发生变化,只是传播方向发生了变化的波,因为X射线不能电离内层电子,因此光子和内层电子发生弹性散射。
相反,康普顿效应改变了光子的能量,因此对光子而言它是\concept{非相干散射},也是\concept{非弹性散射}。

\chapter{束缚态单电子系统}

\section{一维势阱中的电子}

首先考虑如下一维无限高,无限厚的方势阱:
\begin{equation}
    V(x) = \begin{cases}
        0, \quad &0 < x < a, \\
        \infty, \quad &\text{otherwise}.
    \end{cases}
\end{equation}
代入薛定谔方程可以发现在势能无限大的地方要让方程成立只能够让$\psi=0$。
于是一维方势阱中的定态电子的波函数由以下定解问题确定:
\[
    - \hbar^2 \dv[2]{\psi}{x} = E \psi, \quad \psi\big|_{x=0} = \psi\big|_{x=a} = 0.
\]
解之,即得到
\begin{equation}
    \psi_n(x) = \begin{cases}
        \sqrt{\frac{2}{a}} \sin(\frac{n \pi x}{a}), \quad &0 < x < a, \\
        0, \quad &\text{otherwise},
    \end{cases}
\end{equation}
能量为
\begin{equation}
    E_n = \frac{n^2 h^2}{8 m a^2}.
\end{equation}
这个系统没有散射态解,因为势阱以外的地方波函数全部都是零,不可能出现无穷远处还有非零波函数值的情况。

有限高,无限厚的势阱:粒子可以有隧穿,但是在无穷远处衰减为零。

\section{有心力场中的单电子}

\subsection{哈密顿量和薛定谔方程}

有心力场中的单个电子的哈密顿量为
\begin{equation}
    {H} \psi = \frac{{p}^2}{2m} \psi + V(r) \psi.
\end{equation}
求解此问题等价于求解坐标表象下的定态方程
\begin{equation}
    \frac{{p}^2}{2m} \psi + V(r) \psi = - \frac{\hbar^2}{2m} \laplacian \psi + V(r) \psi = E \psi,
    \label{eq:centered-force-eq}
\end{equation}
使用球坐标系,以$\theta$为和$z$轴的夹角,$\varphi$为$x$-$y$平面上的转角,则
\[
    {p}^2 = -\hbar^2 \laplacian = - \hbar^2 \left( \frac{1}{r^2} \pdv{r} r^2 \pdv{r} + \frac{1}{r^2 \sin \theta} \pdv{\theta} \sin \theta \pdv{\theta} + \frac{1}{r^2 \sin^2 \theta} \pdv[2]{\varphi} \right) .
\]
可以直接分离变量,但让我们首先采取一种物理意义比较明显的变形。注意到轨道角动量算符的平方为
\[
    {L}^2 = {\vb*{r}}^2 {\vb*{p}}^2 - ({\vb*{r}} \cdot {\vb*{p}}) ({\vb*{p}} \cdot {\vb*{r}}),
\]
在球坐标系下$\vb*{r}$只在$r$方向上有分量,于是上式变成
\[
    {L}^2 = r^2 {p}^2 - r \left( - \hbar^2 \pdv[2]{r} \right) r,
\]
注意到(实际上这是动量-坐标对易关系的自然推论)
\[
    r \pdv[2]{r} r = \pdv{r} r^2 \pdv{r},
\]
% 此处有误
这就给出了角动量长度平方的一个简洁的表达式:
\begin{equation}
    {L}^2 = - \hbar^2 \left( \frac{1}{\sin \theta} \pdv{\theta} \sin \theta \pdv{\theta} + \frac{1}{\sin^2 \theta} \pdv[2]{\varphi} \right),
\end{equation}
当然也可以按照定义直接计算出这个表达式。相应的,哈密顿量在球坐标系下就是
\[
    {H} = - \frac{\hbar^2}{2 m r} \pdv[2]{r} r + \frac{{L}^2}{2 m r^2} + V(r).
\]
以上对哈密顿量的改写和经典情况可以类比。经典情况下,径向运动是有效势阱中的一维运动,且满足
\[
    \frac{p_r^2}{2m} + \frac{L^2}{2mr^2} - \frac{Ze^2}{4\pi \epsilon_0 r} = E,
\]
正好是将所有算符替换成实数之后的结果——本该如此。

现在,定态薛定谔方程成为
\[
    - \frac{\hbar^2}{2 m r} \pdv[2]{r} (r \psi) + \frac{{L}^2}{2 m r^2} \psi + V(r) \psi = E \psi,
\]
算符${L}^2$仅仅和$\varphi$和$\theta$有关,因此我们可以将径向部分和角向部分做分离变量。
设
\[
    \psi = R(r) Y(\theta, \varphi),
\]
则径向部分的方程是
\begin{equation}
    \left( - \frac{\hbar^2}{2m r^2} \dv{r} r^2 \dv{r} + \frac{\hbar^2}{2m r^2} \alpha + V(r) \right) R = E R,
    \label{eq:original-r-equation}
\end{equation}
角向部分的方程为
\begin{equation}
    {L}^2 Y = \alpha \hbar^2 Y.
    \label{eq:angle-equation}
\end{equation}
其中$\alpha$为常数。

\subsection{角向部分的方程的求解}

径向部分的方程包含一个可变的$V(r)$项,而角向部分的方程可以直接解出。
请注意角向部分实际上是一个拉普拉斯方程的角向部分,因此其解为球谐函数。
下面简单地展示其求解过程。设
\[
    Y(\theta, \varphi) = \Theta(\theta) \Phi(\varphi),
\]
代入
\[
    - \left( \frac{1}{\sin \theta} \pdv{\theta} \sin \theta \pdv{\theta} + \frac{1}{\sin^2 \theta} \pdv[2]{\varphi} \right) Y = \alpha Y,
\]
得到两个方程
\[
    \frac{1}{\Phi} \dv[2]{\Phi}{\varphi} = c,
\]
以及
\[
    - \left( \frac{1}{\sin \theta} \dv{\theta} \sin \theta \dv{\theta} + \frac{1}{\sin^2 \theta} c \right) \Theta = \alpha \Theta.
\]
$\Phi$必须是单值的,因为波函数应该是单值的,于是
\[
    c = m^2, \quad m = 0, 1, 2, \ldots,
\]
而
\[
    - \left( \dv{\cos \theta} (1 - \cos^2 \theta) \dv{\cos \theta} + \frac{1}{1 - \cos^2 \theta} m^2 \right) \Theta = \alpha \Theta.
\]
这是一个连带勒让德方程,为了保证$\Theta(\theta)$单值且有界,应有
\[
    \alpha = l(l+1), \quad l \geq \abs{m}, \quad l = 0, 1, 2, \ldots,
\]
于是\eqref{eq:angle-equation}的一组正交解为
\[
    Y(\theta, \varphi) = \ee^{\ii m \varphi} \legpoly_l^m(\cos \theta),
\]
其中$\legpoly_l^m$表示缔合勒让德多项式。归一化使用球坐标系的角向积分
\[
    \int \sin \theta \dd{\theta} \dd{\varphi},
\]
最后得到正交归一化的球谐函数
% TODO:(-1)^m因子?
\begin{equation}
    \begin{split}
        Y_{lm}(\theta, \varphi) = (-1)^m \sqrt{\frac{(2l+1)(l-\abs{m})!}{4\pi (l+\abs{m})!}} \legpoly_l^\abs{m} (\cos \theta) \ee^{\ii m \varphi}, \\
        l = 0, 1, \ldots, \quad m = \pm l, \pm (l-1), \ldots, 0 .
    \end{split}
\end{equation}

球谐函数是${L}^2$的本征函数,但它带有两个量子数$l$和$m$,这意味着${L}^2$的本征函数存在简并;$l,m$分别标记了${L}^2$的本征值和某个不确定的可观察量的本征值。
注意到球坐标系中
\[
    {L}_z = - \ii \hbar \pdv{\varphi},
\]
它正是$\Phi(\varphi)$满足的本征方程中的那个算符,因此$m$标记的是${L}_z$的本征值,$l$标记的是${L}^2$的本征值。
这是正确的,因为
\[
    m = \pm l, \pm (l-1), \ldots, 0
\]
正是角动量代数的重要性质。我们下面记此处的$m$为$m_l$,与自旋角动量在$z$方向上的取值$m_s = \pm \frac{1}{2}$相区分。
球谐函数$Y_{lm}(\theta, \varphi)$同时是${L}^2$和$L_z$的本征函数,相应的本征值为
\begin{equation}
    {L}^2 Y_{lm} = l(l+1) \hbar^2 Y_{lm}, \quad {L}_z Y_{lm} = m \hbar Y_{lm}.
    \label{eq:orbital-angular-momentum}
\end{equation}

\subsection{量子数}\label{sec:quantum-number}

现在径向方程\eqref{eq:original-r-equation}成为了
\begin{equation}
    \left( - \frac{\hbar^2}{2m r^2} \dv{r} r^2 \dv{r} + \frac{\hbar^2}{2m r^2} l(l+1) + V(r) \right) R = E R.
    \label{eq:r-equation}
\end{equation}
这是一个单变量的本征值问题,它还会产生一个(而且也只有一个,因为一旦$E$确定了,$R$就确定了,不存在简并)量子数$n$,它标记不同的能量。
请注意由于角动量部分被分离变量出去了,实际上多出来了一个有效势$l(l+1)$项。
更加变于求解的一种形式是设$u=rR$,则有
\begin{equation}
    - \frac{\hbar^2}{2m} \dv[2]{u}{r} + \frac{\hbar^2}{2m r^2} l(l+1) u + V(r) u = Eu, \quad R = \frac{u}{r}.
\end{equation}

这样,\eqref{eq:centered-force-eq}有一组由$n, l, m_l$标记的正交归一化解,再考虑到自旋自由度,电子的状态就完全确定了。
通过求解过程可以看出前三个量子数标记了${H}, {L}^2, {L}_z$的本征值;
它们是通过分离变量求解坐标空间中的薛定谔方程得到的,因此随着时间变化,它们对应的物理量是守恒的且彼此对易,能够找到这样三个守恒且彼此对易的物理量当然是因为有心力系统的对称性。
目前尚未引入任何涉及自旋的机制,因此自旋也是恒定不变的。
于是我们找到了四个标记电子的好量子数:
\begin{enumerate}
    \item 主量子数$n$,它标记不同的能量,它是分立的,因为电子陷在一个势阱中,从而是离散谱;
    \item 角量子数$l$,它标记不同的角动量大小,对一部分势场形式,也参与标记不同的能量,它是分立的,因为波函数在角方向上是单值的;
    \item 磁量子数$m_l = 0, \pm 1, \ldots, \pm l$,它标记$z$轴上的角动量分量,同样也是分立的;
    \item 自旋量子数$m_s = \pm \frac{1}{2}$,它来自电子的内禀旋转自由度;自旋量子数和我们刚才讨论的轨道空间无关。
\end{enumerate}

这四个量子数直接决定了波函数的形状。主量子数决定了径向概率分布,角量子数和磁量子数决定了波函数的角向概率分布。
这四个量子数还可以确定其它一些量子数。例如,做宇称变换
\[
    (r, \theta, \varphi) \longrightarrow (r, \pi - \theta, \pi + \varphi),
\]
径向部分始终不变,若$l$为奇数则球谐函数会差一个负号,从而波函数为奇宇称,若$l$为偶数则球谐函数不变,从而波函数为偶宇称。

纯量子的理论展现出了和经典理论很不同的一些特性。请注意量子理论中电子可以完全没有角动量,这在经典理论下是不可能的——电子会直接落入有心力的力心,比如说原子核。
然而,哈密顿量\eqref{eq:columb-electron-hamiltonian}中各项不对易从而有量子涨落,因此如果角动量确定为零,那么电子的位置就不能够确定,因此电子并不会落入原子核。

\subsection{角动量代数}\label{sec:algebra-of-angular}

本节从对称性的角度来分析有心力场中的电子的角量子数。
有心力场各向同性的性质意味着系统具有$SO(3)$对称性,从而$SO(3)$的李代数中的每个生成元都是守恒量,当然这就是大名鼎鼎的角动量守恒。使用${J}$表示这些生成元。
$SO(3)$的李代数的对易关系就是角动量算符的对易关系,即
\begin{equation}
    {\vb*{J}} \times {\vb*{J}} = \ii \hbar {\vb*{J}},
\end{equation}
或者也可以写成显式的李括号的形式:
\begin{equation}
    \comm*{{J}_i}{{J}_j} = \ii \hbar \epsilon_{ijk} {J}_k.
\end{equation}
从这个李代数可以马上发现${\vb*{J}}^2$是卡西米尔元。
对每个不可约表示,三个方向的角动量互不对易,因此只需要取一个角动量,列出其本征态即可完整地描述角动量空间。
这样就可以使用${\vb*{J}}^2$和${J}_z$的共同本征态作为角动量空间的基底,无论是不是不可约表示。

对$SO(3)$的不可约表示,设$j$是角量子数,有
\begin{equation}
    {\vb*{J}}^2 \ket{j,m_j} = j(j+1)\hbar^2 \ket{j,m_j}, \quad {J}_z \ket{j,m_j} = m_j \hbar \ket{j,m_j},
    \label{eq:eigenvalue-of-angular}
\end{equation}
其中$j$和$m_j$都是整数或者半整数。我们知道$j$唯一地标记了每个有限维不可约表示,而
\begin{equation}
    m_j = -j, -j+1, \ldots, j.
\end{equation}
要证明\eqref{eq:eigenvalue-of-angular}(并做一些别的计算),通常需要定义产生湮灭算符,具体来说,定义
\begin{equation}
    {J}_\pm = \frac{1}{\sqrt{2}} ({J}_x \pm \ii {J}_y),
\end{equation}
可以计算得到对易关系
\begin{equation}
    \comm*{{J}_z}{{J}_\pm} = \hbar {J}_\pm,
\end{equation}
因此${J}_+$和${J}_-$就是$\hbar m_j$的升降算符,每作用一次增加或者减少$\hbar$的$z$方向角动量。
如果是有限维表示,必定有
\begin{equation}
    {J}_\pm \ket{j, m_j} = \sqrt{\frac{(j \mp m_j)(j \pm m_j + 1)}{2}} \hbar \ket{j, m_j \pm 1}.
\end{equation}
从上式也可以推导出
\begin{equation}
    \begin{aligned}
        {J}_x \ket{j, m_j} &= \frac{1}{2} \sqrt{(j-m_j)(j+m_j+1)} \hbar \ket{j, m_j+1} \\
        &+ \frac{1}{2} \sqrt{(j+m_j)(j-m_j+1)} \hbar \ket{j, m_j-1} , \\
        {J}_y \ket{j, m_j} &= \frac{1}{2} \ii \sqrt{(j+m_j)(j-m_j+1)} \hbar \ket{j, m_j-1} \\
        &- \frac{1}{2} \ii \sqrt{(j-m_j)(j+m_j+1)} \hbar \ket{j, m_j+1}.
    \end{aligned}
    \label{eq:apply-jx-jy-to-angular-state}
\end{equation}

现在回过头和\eqref{eq:orbital-angular-momentum}中求解出来的轨道角动量做对比,会发现中心场下的束缚态电子的轨道角动量的确是一个$SO(3)$的有限维表示。
电子对应的量子场是一个有质量的旋量场,它的内禀自由度是$SO(3)$的一个二维不可约表示,因此也就有两个自旋量子数:$1/2$和$-1/2$。
实际上,如果依次取自旋空间的基底为$\ket{1/2}$和$\ket{-1/2}$,则${S}_z$的矩阵形式为
\[
    {S}_z = \pmqty{\frac{1}{2} \hbar & 0 \\ 0 & -\frac{1}{2} \hbar}.
\]
定义泡利矩阵为
\begin{equation}
    \sigma_x = \pmqty{0 & 1 \\ 1 & 0}, \quad \sigma_y = \pmqty{0 & -\ii \\ \ii & 0}, \quad \sigma_z = \pmqty{1 & 0 \\ 0 & -1},
\end{equation}
使用\eqref{eq:apply-jx-jy-to-angular-state}可以计算出
\begin{equation}
    {S}_i = \frac{\hbar}{2} \sigma_i.
\end{equation}

\chapter{弹性散射}

\section{散射过程}

考虑在一个势场中的粒子,具有如下哈密顿量:
\begin{equation}
    \hat{H} = \underbrace{- \frac{\hbar^2 \laplacian}{2m}}_{\hat{H}_0} + V(\vb*{r}).
\end{equation}
所谓散射是指这样一个过程:在极限$t\to \infty$和$t \to -\infty$下,系统状态可以认为是自由的,即在这两个极限下可以认为$V=0$。
我们假定$V(\vb*{r})$在无穷远处为零,并且衰减的速率要大于$1/r$——这是为了让我们要处理的问题性质足够良好。当然,这就意味着库伦势需要额外讨论。
这样,如果粒子确实有散射态,那么散射态满足$E > 0$。($V(\vb*{r})$在远处衰减为零的条件很重要,否则$E$会需要整体加上一个常数)
在无穷远处$V=0$,所以动量本征态就是粒子的定态;又由于无穷远处只有动能而能量守恒,入射粒子动量大小等于出射粒子动量大小。
由于入射、出射动量大小相同,可以用球坐标系中出射动量的两个角参数$\theta$和$\varphi$来标记。

散射过程的实验可观测量通常是\concept{微分散射截面}。我们制备一束很密集(从而可以很好地引入粒子数密度的概念)但是又不过于密集(从而可以忽视粒子间的散射)的束流,将它射入势场$V(\vb*{r})$中,然后以势场为中心,观察被散射到不同立体角中的粒子个数。
在经典理论中微分散射截面定义为
\begin{equation}
    \sigma = \dv{A}{\Omega},
    \label{eq:classical-cross-section}
\end{equation}
即入射平面中面元$\dd{A}$中的粒子的轨道最后终结于立体角元$\dd{\Omega}$中。
设入射粒子数面密度在$\dd{A}$区域中为$F \dd{t}$,单位时间中散射到$\dd{\Omega}$中的粒子总数为$\dd{N}$,则由粒子数守恒我们有
\[
    \dd{N} \dd{t} = F \dd{t} \dd{A},
\]
从而\eqref{eq:classical-cross-section}等价于
\begin{equation}
    \dd{N} = F \sigma \dd{\Omega}.
    \label{eq:quantum-cross-section}
\end{equation}
在量子物理中由于轨道的概念没有意义,\eqref{eq:classical-cross-section}不适用,但是\eqref{eq:quantum-cross-section}仍然是适用的,从而还是可以定义一个散射截面。

需要注意的是无论是经典理论还是量子理论中,散射截面要有意义实际上都要求散射过程发生得足够快。
如果散射实际上很慢,那么显然,$t_1$时刻出现在面积元$\dd{A}$中的粒子要经过一个时间延迟之后,在$t_2$时刻出现在$\dd{\Omega}$中。
但是,很多时候这个时间延迟都是可以忽略的。

散射截面的计算可以使用跃迁率。我们知道
\[
    w(i \to f) = \frac{2\pi}{\hbar} \abs*{T_{if}}^2 \rho(E_i), \quad E_i = E_f,
\]
这里$T$由李普曼-施温格方程给出,满足
\[
    \hat{T} = \hat{H}' + \hat{H}' \frac{1}{E - \hat{H}_0} \hat{H}' + \cdots, \quad \hat{S} = 1 - 2 \pi \ii \delta(E_f - E_i) \hat{T},
\]
散射定态$\ket{\psi_i}$满足
\[
    \hat{T} \ket{i} = \hat{H}' \ket{\psi_i}.
\]
由于无穷远处的能量本征态就是(连续的)动量本征态,上式可以写成
\[
    w(\vb*{p}_i \to \vb*{p}_j) = \frac{2\pi}{\hbar} \abs*{T_{if}}^2 \rho(E_{\vb*{p}_i}).
\]
为了避免系数出错(因为涉及狄拉克$\delta$函数),我们假定粒子被局限在一个边长为$L$的很大的盒子中,从而获得离散的动量,但是又能够表现出散射的行为。
入射动量和出射动量的长度可以相差一个微小的,由于离散化动量空间而产生的值$\Delta p$,相应的能量差为$\Delta E$,等等。
此时我们设$\vb*{n}$是三个分量均为整数的格矢,则
\[
    \vb*{k} = \frac{2\pi}{L} \vb*{n}, \quad E = \frac{\hbar^2 k^2}{2m} = \frac{\hbar^2}{2m} \left( \frac{2\pi}{L} \right)^2 \abs*{\vb*{n}}^2,
\]
于是
\[
    \Delta E = \frac{\hbar^2}{m} \left( \frac{2\pi}{L} \right)^2 \abs*{\vb*{n}} \Delta \abs*{\vb*{n}}.
\]
另一方面,动量大小为$\vb*{p}$到$\vb*{p}+\Delta \vb*{p}$,动量指向在$\dd{\Omega}$内的区域内的状态数为
\[
    \Delta N = \abs*{\vb*{n}}^2 \Delta \abs*{\vb*{n}} \dd{\Omega},
\]
则态密度为
\[
    \frac{\Delta N}{\Delta E} = k \frac{m}{\hbar^2} \left( \frac{L}{2\pi} \right)^3 \dd{\Omega},
\]
于是
\begin{equation}
    w(\vb*{p}_i \to \vb*{p}_j) = \frac{mL^3}{(2\pi)^2 \hbar^3} k \abs*{T_{if}}^2 \dd{\Omega}.
\end{equation}
现在假定有$N$个粒子组成一束横截面上均匀的束流,我们有
\[
    \dd{N} = N w(\vb*{p}_i \to \vb*{p}_j),
\]
而
\[
    F = \frac{N}{L^3} v = \frac{N}{L^3} \frac{\hbar k}{m},
\]
最终我们就得到
\[
    \sigma = \left( \frac{m L^3}{2\pi \hbar^2} \right)^2 \abs*{T_{if}}^2.
\]
上面我们为了把微分散射截面作为一个独立的量给了它一个符号$\sigma$,但是更加通用的符号实际上是$\dd{\sigma}/\dd{\Omega}$,于是
\begin{equation}
    \dv{\sigma}{\Omega} = \left( \frac{m L^3}{2\pi \hbar^2} \right)^2 \abs*{T_{if}}^2.
    \label{eq:quantum-cross-section-and-t}
\end{equation}
因此,接下来要做的就是计算$T$矩阵。请注意散射截面实际上是乘上了某个因子的概率分布,因此如果有多个散射道而它们彼此没有相干
% TODO

\section{无穷远处的散射定态}

最一般的计算$T$矩阵的方法显然是直接做展开,但由于是单粒子问题,通过计算散射定态来求解$T$还是可行的,于是我们设散射定态为$\varphi(\vb*{r})$,要求解本征值方程
\[
    \left( - \frac{\hbar^2 \laplacian}{2m} + V(\vb*{r}) \right) \varphi = E \varphi.
\]
引入以下参数:
\begin{equation}
    E = \frac{\hbar^2 k^2}{2m}, \quad V(\vb*{r}) = \frac{\hbar^2 U(\vb*{r})}{2m},
\end{equation}
这里我们隐含地假定了$E>0$。此时散射定态的本征值问题就是
\begin{equation}
    (\laplacian + k^2 - U(\vb*{r})) \varphi_k = 0.
    \label{eq:scattering-eigen}
\end{equation}

\eqref{eq:scattering-eigen}的解可以非常丰富,除了物理的散射定态以外肯定还有非物理的一些解,所以要讨论我们需要的解是什么样的。
不失一般性地,我们认为入射粒子从$z$轴负半轴入射,从而入射态为
\[
    \varphi^\text{in} = \frac{1}{L^{3/2}} \ee^{\ii k z}.
\]
这样就可以将以指向$z$轴正半轴、大小为$k$的动量入射的粒子的散射振幅写成$f_k(\theta, \varphi)$。
设自由粒子的出射格林函数为$G_k(\vb*{r}, \vb*{r}')$,则按照李普曼-施温格方程我们有
\[
    \begin{aligned}
        \varphi_k(\vb*{r}) &= \frac{1}{L^{3/2}} \ee^{\ii k z} + \mel{\vb*{r}}{\frac{1}{E - \hat{H}_0 + \ii 0^+} \hat{T}}{\vb*{k}} \\
        &= \frac{1}{L^{3/2}} \ee^{\ii k z} + \sum_{\vb*{k}'} \int \dd[3]{\vb*{r}'} G_k(\vb*{r}, \vb*{r}') \frac{1}{L^{3/2}} \ee^{\ii \vb*{k}' \cdot \vb*{r}'} \mel*{\vb*{k}'}{\hat{T}}{\vb*{k}}.
    \end{aligned}
\]
这个方程有非常明确的物理意义:如果我们制备一个足够像平面波的入射态,那么应该预期一段时间后在动量$\vb*{k}'$上观察到权重为$\mel{n}{\hat{T}}{i}$的出射态。
这个结论实际上是非常一般的,不仅仅限于此处的单粒子量子力学,如光学中有完全一样的操作:我们解出一个散射定态,然后认为如果向一个光学系统中打入一个平面波,很快将在无穷远处得到散射定态中的出射波,而忽略“弛豫”的时间。

$G_k(\vb*{r}, \vb*{r}')$本身是容易计算的。最简单的做法是直接求解含源的亥姆霍兹方程
\[
    \left(E + \frac{\hbar^2 \laplacian}{2m} \right) G_k(\vb*{r}, \vb*{r}') = \delta(\vb*{r} - \vb*{r}'),
\]
也即
\[
    (\laplacian + k^2) G_k(\vb*{r}, \vb*{r}') = \frac{2m}{\hbar^2} \delta(\vb*{r} - \vb*{r}),
\]
使用(物理意义非常清楚)的傅里叶变换解法我们有
\[
    G_k^+(\vb*{r}, \vb*{r}') = - \frac{2m}{\hbar^2} \frac{1}{4\pi} \frac{\ee^{\ii k \abs*{\vb*{r} - \vb*{r}'}}}{\abs*{\vb*{r} - \vb*{r}'}},
\]
其中上标$+$表示这是一个推迟格林函数。在$\vb*{r}$趋于无穷远时,我们可以做近似
\[
    \abs*{\vb*{r} - \vb*{r}'} = r - \frac{\vb*{r}}{r} \cdot \vb*{r}' = r - \hat{\vb*{r}} \cdot \vb*{r}',
\]
从而
\[
    \ee^{\ii k \abs*{\vb*{r} - \vb*{r}'}} = \ee^{\ii k r} \ee^{- \ii k \hat{\vb*{r}} \cdot \vb*{r}'},
\]
而分母上的$\abs*{\vb*{r} - \vb*{r}'}$则可以简单地近似为$r$。这样就有
\[
    \varphi_k(\vb*{r}) = \frac{1}{L^{3/2}} \left( \ee^{\ii k z} - \frac{m L^3}{2\pi \hbar^2} \frac{\ee^{\ii k r}}{r} \mel{k \hat{\vb*{r}}}{\hat{T}}{\vb*{k}} \right) , \quad \text{as $r \to \infty$}.
\]
设
\begin{equation}
    f(\vb*{k}, k \hat{\vb*{r}}) = f_k(\theta, \varphi) = - \frac{m L^3}{2\pi \hbar^2} \mel{k \hat{\vb*{r}}}{\hat{T}}{\vb*{k}} ,
    \label{eq:scattering-amp}
\end{equation}
其中$(\theta, \varphi)$是$\hat{\vb*{r}}$(相对于$\vb*{k}$)的球坐标方位角,我们就得到物理意义非常清晰的渐进形式
\begin{equation}
    \varphi_k(\vb*{r}) = \frac{1}{L^{3/2}} \left( \ee^{\ii k z} + \frac{\ee^{\ii k r}}{r} f_k(\theta, \varphi) \right) , \quad \text{as $r \to \infty$}.
    \label{eq:infty-in-box}
\end{equation}
可以看到散射定态分为两项,一项是一个入射的平面波,另一项是一个加入了某种各向异性修正(肯定要加入因为入射方向是空间中的一个特殊方向)的球面波。$f_k(\theta, \varphi)$实际上就是(可能差一个常数因子的)散射振幅,并且按照\eqref{eq:quantum-cross-section-and-t},我们有
\begin{equation}
    \dv{\sigma}{\Omega} = \abs*{f_k(\theta, \varphi)}^2,
    \label{eq:quantum-cross-section-and-f}
\end{equation}
也即,实际上散射截面的计算只用到了无穷远处的散射定态的信息,这当然是合理的,因为正如我们在定义散射截面时看到的那样,散射截面对散射过程中间态是完全不关注的。
另外可以发现观测点$\vb*{r}$和这一点预期的出射态的动量方向一致,这也表面经典粒子图像在散射问题中有时仍然适用,如果我们研究的问题中粒子适当地定域,但是相对散射势场又足够像平面波,那么就可以把散射过程当成一个碰撞“黑箱”,其余部分全部使用经典理论。

\eqref{eq:infty-in-box}中的因子$L^{-3/2}$当然是来自归一化因子,于是在$L \to \infty$时我们可以简单地写出
\begin{equation}
    \varphi_k(\vb*{r}) = \ee^{\ii k z} + \frac{\ee^{\ii k r}}{r} f_k(\theta, \varphi) , \quad \text{as $r \to \infty$}.
    \label{eq:wave-function-k-f}
\end{equation}
由于$f$本质上是散射定态在无穷远处中球面波成分的权重除以平面波成分的权重,归一化方式不改变它,从而\eqref{eq:quantum-cross-section-and-f}仍然成立。

由于我们在本节中使用了非相对论性的动能和时空观(如对$\vb*{p}$的积分的归一化常数是$(2\pi)^3$而不是$2E_{\vb*{p}}$),本节的理论实际上仅仅适用于非相对论性的(以轨道-动量为自由度的)粒子。

总之,只需要计算出散射定态,从中提取出$f_k(\theta, \varphi)$,就能够计算出散射截面。

\section{玻恩近似}

本节给出一种通过积分方程得到的级数解法。
从\eqref{eq:scattering-eigen}可以得到如下积分方程
\begin{equation}
    \varphi_k(\vb*{r}) = \ee^{\ii \vb*{k} \cdot \vb*{r}} - \int \dd[3]{\vb*{r}'} \frac{1}{4\pi} \frac{\ee^{\ii k \abs*{\vb*{r} - \vb*{r}'}}}{\abs*{\vb*{r} - \vb*{r}'}} U(\vb*{r}') \varphi_k(\vb*{r}'),
    \label{eq:born-eq}
\end{equation}
其中$\vb*{k}$,如前所述,通常取为指向$z$轴正方向。实际上这个积分方程就是李普曼-施温格方程,在这里我们使用了关系
\[
    \hat{T} \ket*{\vb*{k}} = \hat{U} \ket*{\varphi_k},
\]
并且在坐标表象下写出了分量。

我们假定$U$很小,从而可以使用微扰论求解\eqref{eq:scattering-eigen},也即,微扰计算\eqref{eq:born-eq}。形式上可以写出以下无穷级数:
\[
    \begin{aligned}
        \varphi_k(\vb*{r}) &= \ee^{\ii \vb*{k} \cdot \vb*{r}} - \int \dd[3]{\vb*{r}'} \frac{1}{4\pi} \frac{\ee^{\ii k \abs*{\vb*{r} - \vb*{r}'}}}{\abs*{\vb*{r} - \vb*{r}'}} U(\vb*{r}') \ee^{\ii \vb*{k} \cdot \vb*{r}'} \\
        &+ \int \dd[3]{\vb*{r}'} \int \dd[3]{\vb*{r}''} \frac{1}{4\pi} \frac{\ee^{\ii k \abs*{\vb*{r} - \vb*{r}'}}}{\abs*{\vb*{r} - \vb*{r}'}} U(\vb*{r}') \frac{1}{4\pi} \frac{\ee^{\ii k \abs*{\vb*{r}' - \vb*{r}''}}}{\abs*{\vb*{r}' - \vb*{r}''}} \ee^{\ii \vb*{k} \cdot \vb*{r}''} + \cdots,
    \end{aligned}
\]
直观地看就是粒子被散射了一次,两次,三次……的波函数之和,画成费曼图就是粒子自由运动,粒子先自由运动然后被散射然后再自由运动……的和。这就是\concept{玻恩级数}。
当然,如果$U$很大,那么以上级数就是发散的,而即使$U$比较小,如果其形式不正确,以上级数仍然可能是渐进级数。
无论如何,如果我们希望计算前几阶微扰而得到一个可靠的结果,应该要求\eqref{eq:born-eq}右边的第二项相比第一项足够小。
在两种情况下我们可以放心地做微扰论。设$U$的力程为$a$。首先,如果粒子动量不是很大,使得$k a \lesssim 1$,则相位因子$\ee^{\ii k r}$不重要,此时微扰论适用的条件为
\[
    1 \gg \int \dd[3]{\vb*{r}} \frac{1}{4\pi} \frac{1}{\abs*{\vb*{r} - \vb*{r}'}} \abs*{U},
\]
也即
\begin{equation}
    \abs*{V} \ll \frac{\hbar^2}{a^2 m} , \quad k a \lesssim 1.
    \label{eq:slow-pertubation}
\end{equation}
这个条件的意义是比较明显的,第一个不等式的右边实际上是粒子被束缚在空间尺度为$a$的空间中的动能尺度,因此它实际上就是要求粒子运动不是很快,同时势场不足以产生束缚态。
另一种情况是高能极限,即$ka \gg 1$,此时相位因子会导致被积函数快速振荡,我们有
\[
    \int \dd[3]{\vb*{r}} \frac{\ee^{\ii k \abs*{\vb*{r} - \vb*{r}'}}}{\abs*{\vb*{r} - \vb*{r}'}} U(\vb*{r}') \ee^{\ii \vb*{k} \cdot \vb*{r}'} \sim \frac{1}{k} \int \dd{z} U(\vb*{r}) \sim \frac{a}{k} \abs*{U},
\]
于是微扰论适用的条件就是
\begin{equation}
    \abs*{V} \ll \frac{\hbar^2}{a^2 m } ka , \quad k a \gg 1.
    \label{eq:fast-pertubation}
\end{equation}
\eqref{eq:fast-pertubation}对势场的要求弱于\eqref{eq:slow-pertubation},因此如果低能极限下可以使用微扰论,那么高能极限下也可以。

特别地,如果只取一阶近似(即所谓\concept{一阶玻恩近似},在不至于引起混淆的情况下也称为玻恩近似),就有
\begin{equation}
    \varphi_k(\vb*{r}) = \ee^{\ii \vb*{k} \cdot \vb*{r}} - \int \dd[3]{\vb*{r}'} \frac{1}{4\pi} \frac{\ee^{\ii k \abs*{\vb*{r} - \vb*{r}'}}}{\abs*{\vb*{r} - \vb*{r}'}} U(\vb*{r}') \ee^{\ii \vb*{k} \cdot \vb*{r}'},
\end{equation}
在无穷远处使用近似
\[
    \abs*{\vb*{r}-\vb*{r}'} = r - \hat{\vb*{r}} \cdot \vb*{r}',
\]
和上一节推导无穷远处的$\varphi_k$的方法类似,得到
\[
    \varphi_k(\vb*{r}) = \ee^{\ii \vb*{k} \cdot \vb*{r}} - \frac{\ee^{\ii k r}}{4 \pi r} \int \dd[3]{\vb*{r}'} U(\vb*{r}') \ee^{\ii \vb*{r}' \cdot (\vb*{k} - k \hat{\vb*{r}})},
\]
记动量转移为
\begin{equation}
    \vb*{q} = \vb*{k}' - \vb*{k} = k \hat{\vb*{r}} - \vb*{k},
\end{equation}
散射振幅
\begin{equation}
    f_k(\theta, \varphi) = - \frac{m}{2\pi \hbar^2} \int \dd[3]{\vb*{r}} V(\vb*{r}) \ee^{- \ii \vb*{q} \cdot \vb*{r}}, 
\end{equation}
从而散射截面为
\begin{equation}
    \dv{\sigma}{\Omega} = \frac{m^2}{4\pi^2 \hbar^4} \left( \int \dd[3]{\vb*{r}} V(\vb*{r}) \ee^{- \ii \vb*{q} \cdot \vb*{r}} \right)^2.
\end{equation}

玻恩近似的适用条件是

在玻恩近似的慢碰近似——也即,$k \to 0$而与此同时玻恩近似仍然适用的情况下——我们有
\begin{equation}
    \dv{\sigma}{\Omega} = 4 a^2,
\end{equation}
其中
\begin{equation}
    a = \frac{m}{4\pi \hbar^2} \int \dd[3]{\vb*{r}} V(\vb*{r})
\end{equation}
称为\concept{散射长度}。

\section{分波法}

在系统具有特殊的对称性,从而可以分离变量求解时,\concept{分波法}——即以系统的对称群的表示为基底展开波函数——是另一种常用的方法。
由于我们总是假定势场局限在小的空间区域中,以下仅仅讨论球对称的情况,即$V = V(r)$,或者说我们认为角动量守恒。
需要注意的是库伦势没有特征长度,不能认为局限在小的空间区域内,不能使用分波法。

对足够局域的势场,总是可以把空间分成三部分:最内部的是散射区,需要求解完整的带有势场的薛定谔方程;较外层的是中间区,散射势可以忽略,但是等效角动量势垒不能忽略;最外层的是辐射区,可以认为是自由粒子。

由于势场具有球对称性,在中间区和辐射区尝试用球贝塞尔函数展开径向波函数;我们总是可以将$z$轴设置在入射$\vb*{k}$的方向上,于是系统具有$z$轴旋转对称性,从而$m=0$,只需要考虑角量子数,于是写出波函数通解:
\[
    \varphi(\vb*{r}) = \int \dd[3]{\vb*{k}} \sum_{l} (a_l(\vb*{k}) \hankelone_l(kr) + b_l(\vb*{k}) \hankeltwo_l(kr)) \legpoly_l(\cos \theta).
\]
在无穷远处,$\hankelone_l(kr)$是一个出射球面波而$\hankeltwo_l(kr)$是一个入射球面波。
散射定态中有一系列出射波和一个入射平面波。出射波肯定可以写成$\hankelone_l(kr) \legpoly_l (\cos \theta)$的线性组合,而对平面波,我们有\concept{瑞利公式}
\begin{equation}
    \ee^{\ii k z} = \sum_{l=0}^\infty \ii^l (2l+1) \mathrm{j}_l (kr) \legpoly_l(\cos \theta),
\end{equation}
因此它的确可以用汉克尔函数表示。于是散射定态应该形如
\[
    \varphi(\vb*{r}) = \int \dd[3]{\vb*{k}} A(\vb*{k}) \left( \ee^{\ii k z} + \sum_l a_l(\vb*{k}) \hankelone_l(kr)  \right).
\]
而由于我们仅考虑单色入射波,应有
\begin{equation}
    \varphi(\vb*{r}) = A\left( \ee^{\ii k z} + k \sum_l \ii^{l+1} (2l+1) a_l \hankelone_l(kr) \legpoly_l(\cos \theta) \right).
\end{equation}
这里我们为了后面求解的方便特意重新定义了一些常数。于是,解薛定谔方程就变成了算系数。

在讨论怎么算系数之前,先来看看算出系数后怎么得到散射振幅。在无穷远处
\[
    \hankelone_l(kr) \approx (-\ii)^{l+1} \frac{\ee^{\ii k r}}{kr},
\]
于是
\begin{equation}
    f_k(\theta, \varphi) = \sum_l (2l+1) a_l \legpoly_l(\cos \theta).
    \label{eq:scattering-amp-waves}
\end{equation}
于是总散射截面就是
\begin{equation}
    \sigma_\text{total} = 4\pi \sum_{l=0}^\infty (2l+1) \abs*{a_l}^2. 
\end{equation}

这个级数是否收敛得足够快?我们有一定理由认为确实如此,设势场的特征长度为$a$,一个动量为$\hbar \vb*{k}$的波入射,在远离$a$处不会散射,因此能够散射的波有一个特征角动量,大体上是$\hbar k a$。
角动量大的波更加不容易发生散射,因此$a_l$在快速衰减。特别的,在$k a \ll 1$时,基本上只有$l=0$的分波是重要的。
此时可以做拟设
\begin{equation}
    \varphi(\vb*{r}) = A (\mathrm{j}_0(kr) + \ii k a_0 \hankelone_0(kr)) \legpoly_0(\cos \theta),
\end{equation}
入射波中其它的分波都可以忽略,因为它们不会产生任何散射。
由于$l$和$k$都是守恒量,在实际计算可以每次只将下面的拟设
\begin{equation}
    R(r) \propto \mathrm{j}_l (kr) + \ii^{l} k a_l \hankelone_l(kr)
\end{equation}
代入径向部分的方程中。

\section{光学定理}

采用与上文一致的系数约定,$S$矩阵为
\[
    \hat{S} = 1 - 2 \pi \ii \delta(E_f - E_i) \hat{T},
\]
于是根据$\hat{S}$的幺正性条件得到
\[
    2 \Im (- 2 \pi) \mel*{f}{\hat{T}}{i} = \sum_k (- 2 \pi)^2 \mel*{f}{\hat{T}}{k} \mel*{k}{\hat{T}}{i},
\]
我们考虑所谓\concept{前向散射}的散射振幅,也即$f$和$i$相同的情况,就有
\[
    \Im \mel*{\vb*{k}}{\hat{T}}{\vb*{k}} = - \pi \sum_{\vb*{q}} \abs*{\mel*{\vb*{k}}{\hat{T}}{\vb*{q}}}^2.
\]
显然,由无穷远处动能相等,$\abs*{\vb*{q}} = \abs*{\vb*{k}}$。
代入\eqref{eq:scattering-amp},就得到
\[
    - \frac{2 \pi \hbar^2}{m L^3} \Im f(\vb*{k}, \vb*{k}) = - \pi \sum_{\vb*{k}'} \left( \frac{2 \pi \hbar^2}{m L^3} \right)^2 \abs*{f(\vb*{k}, \vb*{k}')}^2 \delta_{\vb*{k} \vb*{k}'},
\]
于是
\[
    \begin{aligned}
        \Im f(\vb*{k}, \vb*{k}) &= \frac{2\pi^2 \hbar^2}{m} \frac{1}{L^3} \sum_{\vb*{k}'} \abs*{f(\vb*{k}, \vb*{k}')}^2 \delta_{E' E} \\
        &= \frac{2\pi^2 \hbar^2}{m} \int \frac{\dd[3]{\vb*{k}'}}{(2\pi)^3} \abs*{f(\vb*{k}, \vb*{k}')}^2 \delta\left(\frac{k'^2}{2m} - \frac{k^2}{2m}\right) \\
        &= \frac{2\pi^2 \hbar^2}{m} \frac{1}{(2\pi)^3} \int k'^2 \dd{k'} \delta\left(\frac{\hbar^2 k'^2}{2m} - \frac{\hbar^2 k^2}{2m}\right) \int \dd{\Omega} \abs*{f(\vb*{k}, \vb*{k}')}^2 \\
        &= \frac{2\pi^2 \hbar^2}{m} \frac{1}{(2\pi)^3} k^2 \frac{1}{\hbar^2 k / m} \int \dd{\Omega} \abs*{f_k(\theta, \varphi)}^2 ,
    \end{aligned}
\]
最后一步给出了总散射截面的形式,于是我们得到
\begin{equation}
    \Im f(\vb*{k}, \vb*{k}) = \Im f_k(0, 0) = \frac{k}{4\pi} \sigma_\text{total}.
\end{equation}
这个结论称为\concept{光学定理}。最早它是在计算电磁波散射截面时得到的,在那里幺正性的要求来自电磁波能量归一化,因此称为这个名字。前向散射条件$f_k(0, 0)$在极坐标系下就是$\theta=0$。

实际上光学定理也可以通过复变函数的方法直接计算得到。

\section{相移法}

实际上还有一种使用入射波和反射波之间的相位差快速计算$a_l$从而快速计算散射截面的方法。
将$\ee^{\ii k z}$的瑞利展开公式代入\eqref{eq:wave-function-k-f}中,并利用$\mathrm{j}_l(kr)$的渐进性质
\[
    \mathrm{j}_l(kr) = \frac{1}{2} (\hankelone_l(kr) + \hankeltwo_l(kr)) \approx \frac{1}{2} \left( \frac{(-\ii)^{l+1} \ee^{\ii k r}}{kr} + \frac{\ii^{l+1} \ee^{- \ii k r}}{kr} \right),
\]
就得到
\[
    \varphi_k(\vb*{k}) \approx \frac{1}{2\ii k r} \sum_{l=0}^\infty (2l+1) \legpoly_l(\cos \theta) (\ee^{\ii k r} - (-1)^l \ee^{- \ii k r}) + \frac{\ee^{\ii k r}}{r} \sum_{l=0}^\infty (2l+1) a_l \legpoly_l(\cos \theta),
\]
成立条件是$r \to \infty$。可以看到$\varphi_k(\vb*{r})$由很多入射球面波和出射球面波构成。由于散射定态(在归一化之后)也是一个正常的波函数,它必定满足概率流守恒条件,因此每个$l$对应的入射球面波和出射球面波的系数的大小必须相同,即
\[
    \abs{\frac{1}{2\ii k} + a_l} = \abs{\frac{1}{2\ii k} (-1)^{l+1}}.
\]
当然,如果完全没有散射,也就是说$a_l=0$,这个条件也是成立的。我们设散射的存在让角量子数为$l$的出射波的相位发生了$2\delta$的变化,即做拟设
\begin{equation}
    \varphi_k(\vb*{r}) \approx \frac{1}{2\ii k r} \sum_{l=0}^\infty (2l+1) \legpoly_l(\cos \theta) (\ee^{2\ii \delta_l} \ee^{\ii k r} - (-1)^l \ee^{- \ii k r}),
    \label{eq:phase-shift}
\end{equation}
则
\[
    \frac{1}{2\ii k} + a_l = \frac{\ee^{2\ii \delta_l}}{2\ii k},
\]
即
\begin{equation}
    a_l = \frac{\ee^{\ii \delta_l}}{k} \sin \delta_l.
    \label{eq:a-and-delta}
\end{equation}
因此求出$\delta_l$之后整个散射问题就求解完毕了。例如,散射截面就是
\begin{equation}
    \sigma_\text{total} = \frac{4\pi}{k^2} \sum_{l=0}^\infty (2l+1) \sin^2 \delta_l.
\end{equation}
因此,我们可以用$\hankelone_l(kr)$和$\hankeltwo_l(kr)$展开散射定态,计算完毕后令$r\to \infty$然后和\eqref{eq:phase-shift}作比较,计算出出射波的相位偏移,就得到了所有需要计算的东西。这称为\concept{相移法}。

上面的推导用到了概率流守恒的条件,意味着实际上相移法的成立是幺正性的结论。的确,实际上我们可以从$S$矩阵的幺正性出发得到相移法。

同样,不出意外的,使用相移法可以很容易地证明光学定理,因为光学定理也是幺正性的自然要求。
将\eqref{eq:a-and-delta}代入\eqref{eq:scattering-amp-waves}中,我们有
\[
    f_k(\theta) = \sum_{l=0}^\infty (2l+1) \frac{\ee^{\ii \delta_l}}{k} \sin \delta_l \legpoly_l(\cos \theta),
\]
于是
\[
    \begin{aligned}
        \Im f_k(0) &= \Im \sum_{l=0}^\infty (2l+1) \frac{\ee^{\ii \delta_l}}{k} \sin \delta_l \legpoly_l(0) \\
        &= \sum_{l=0}^\infty (2l+1) \frac{\sin \delta_l}{k} \sin \delta_l \\
        &= \frac{k}{4\pi} \sigma_\text{total},
    \end{aligned}
\]
就得到光学定理。

\section{全同粒子散射}
