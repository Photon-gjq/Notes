\documentclass[hyperref, UTF8, a4paper, oneside]{ctexbook}

\usepackage{geometry}
\usepackage{titling}
\usepackage{titlesec}
\usepackage{paralist}
\usepackage{footnote}
\usepackage{enumerate}
\usepackage{autobreak}
\usepackage{amsmath, amssymb, amsthm}
\usepackage{mathtools}
\usepackage{bbm}
\usepackage{cite}
\usepackage{graphicx}
\usepackage{subfigure}
\usepackage{physics}
\usepackage{siunitx}
\usepackage{tikz}
\usepackage[compat=1.1.0]{tikz-feynhand}
\usepackage[ruled, vlined, linesnumbered, noend]{algorithm2e}
\usepackage{xr-hyper}
\usepackage[colorlinks, linkcolor=black, anchorcolor=black, citecolor=black, filecolor=black]{hyperref}
\usepackage[most]{tcolorbox}
\usepackage{caption}
\usepackage{prettyref}

\externaldocument[optics-]{../optics/optics}[optics.pdf]
\externaldocument[vasp-]{../cond-comp/vasp/vasp}[vasp.pdf]
\externaldocument[qft-]{../relativistic-qft/relativistic-qft}[relativistic-qft.pdf]
\externaldocument[soft-]{../soft/soft}[soft.pdf]

\geometry{left=3.18cm,right=3.18cm,top=2.54cm,bottom=2.54cm}
\titlespacing{\paragraph}{0pt}{1pt}{10pt}[20pt]
\setlength{\droptitle}{-5em}
\preauthor{\vspace{-10pt}\begin{center}}
\postauthor{\par\end{center}}

\DeclareMathOperator{\timeorder}{\mathcal{T}}
\DeclareMathOperator{\diag}{diag}
\DeclareMathOperator{\legpoly}{P}
\DeclareMathOperator{\primevalue}{P}
\DeclareMathOperator{\sgn}{sgn}
\newcommand*{\ii}{\mathrm{i}}
\newcommand*{\ee}{\mathrm{e}}
\newcommand*{\const}{\mathrm{const}}
\newcommand*{\suchthat}{\quad \text{s.t.} \quad}
\newcommand*{\argmin}{\arg\min}
\newcommand*{\argmax}{\arg\max}
\newcommand*{\normalorder}[1]{: #1 :}
\newcommand*{\pair}[1]{\langle #1 \rangle}
\newcommand*{\fd}[1]{\mathcal{D} #1}

\newcommand*{\st}{\quad \text{s.t.} \quad}
\newcommand*{\natnums}{\mathbb{N}}
\newcommand*{\reals}{\mathbb{R}}
\newcommand*{\complexes}{\mathbb{C}}
\newcommand*{\ogroup}[1]{\mathrm{O}(#1)}
\newcommand*{\sogroup}[1]{\mathrm{SO}(#1)}\DeclareMathOperator{\laguerre}{L}
\newcommand*{\lsterm}[3]{$^{#1}${#2}$_{#3}$}
\newcommand*{\nuclear}[3]{^{#2}_{#3}\text{#1}}

\newenvironment{bigcase}{\left\{\quad\begin{aligned}}{\end{aligned}\right.}

\newrefformat{chap}{第\ref{#1}章}
\newrefformat{sec}{第\ref{#1}节}
\newrefformat{note}{注\ref{#1}}
\newrefformat{fig}{图\ref{#1}}
\newrefformat{alg}{算法\ref{#1}}
\newrefformat{back}{背景知识\ref{#1}}
\newrefformat{info}{资料框\ref{#1}}
\newrefformat{warn}{注意事项\ref{#1}}
\renewcommand{\autoref}{\prettyref}

\usetikzlibrary{arrows,shapes,positioning}
\usetikzlibrary{arrows.meta}
\usetikzlibrary{decorations.markings}
\tikzstyle arrowstyle=[scale=1]
\tikzstyle directed=[postaction={decorate,decoration={markings,
    mark=at position .5 with {\arrow[arrowstyle]{stealth}}}}]
\tikzstyle ray=[directed, thick]
\tikzstyle dot=[anchor=base,fill,circle,inner sep=1pt]

% Algorithm setting
\renewcommand{\algorithmcfname}{算法}
% Python-style code
\SetKwIF{If}{ElseIf}{Else}{if}{:}{elif:}{else:}{}
\SetKwFor{For}{for}{:}{}
\SetKwFor{While}{while}{:}{}
\SetKwInput{KwData}{输入}
\SetKwInput{KwResult}{输出}
\SetArgSty{textnormal}

\tcbuselibrary{skins, breakable, theorems}

\newtcbtheorem[number within=chapter]{back}{背景知识}%
  {colback=blue!5,colframe=blue!65,fonttitle=\bfseries, breakable}{back}
\newtcbtheorem[number within=chapter]{info}{资料框}%
  {colback=blue!5,colframe=blue!65,fonttitle=\bfseries, breakable}{info}
\newtcbtheorem[number within=chapter]{warning}{注意事项}%
  {colback=orange!5,colframe=orange!65,fonttitle=\bfseries, breakable}{warn}

\renewcommand{\emph}[1]{\textbf{#1}}
\newcommand*{\concept}[1]{\underline{\textbf{#1}}}

\numberwithin{equation}{chapter}

\newcommand{\hmn}[1]{% Hermann-Maguin notation
  \ensuremath{\begingroup\setupHMN #1\endgroup}%
}

\newcommand{\setupHMN}{%
  \doHMN{-}{\HMNoverline}%
  \doHMN{*}{\HMNminverse}%
  \doHMN{i}{\infty}
}

\newcommand{\doHMN}[2]{%
  \begingroup\lccode`~=`#1
  \lowercase{\endgroup\let~}#2%
  \mathcode`#1="8000
}

\newcommand{\HMNminverse}[1]{\frac{#1}{m}}
\newcommand{\HMNoverline}[1]{\mkern1mu\overline{\mkern-1mu#1\mkern-1mu}\mkern1mu}

\newcommand{\Ztwo}{$\mathbb{Z}_2$}

\newcommand{\bigO}[1]{\mathcal{O}(#1)}

\newcommand{\vaspdoc}{\href{../computational/vasp/vasp.pdf}{VASP笔记}}
\newcommand{\opticsdoc}{\href{../optics/optics}{光学笔记}}
\newcommand{\softdoc}{\href{../soft/soft}{流体和软物质物理笔记}}
\newcommand{\qftdoc}{\href{../relativistic-qft/relativistic-qft}{相对论性量子场论笔记}}

\title{牛顿力学和非相对论量子力学}
\author{吴晋渊}

\begin{document}

\maketitle

\part{牛顿力学}

还需要做的事:
\begin{enumerate}
    \item 欧拉方程的导出
    \item 欧拉角
    \item 广义本征值法
\end{enumerate}

\chapter{对称性分析导出牛顿力学}

\section{自由粒子}

\subsection{伽利略对称性下自由粒子的拉氏量}

假定在牛顿力学中时间是底流形的坐标——也就是说,没有必要一开始就把它看成一个广义坐标
(当然,总是可以使用扩展拉格朗日量的方法把它当成广义坐标)。
再假设单个粒子的位置可以使用一个$\reals^3$中的矢量表示。
因此,单个粒子的广义坐标可以是三个直角坐标(当然也可以是等价的极坐标等)。
这样一来它的拉氏量就可以写成
\[
    L = L(\vb*{r}, \vb*{v}, t)
\]
其中$\vb*{v}$就是$\vb*{r}$对$t$的导数。

$\vb*{r}$和$\vb*{v}$显然需要通过和某个“参考点”的比较确定,也就是说我们需要一个\concept{参照物}才能够写下$L$的形式。我们还需要一些具体的坐标将$\vb*{r}$写出来,因此还需要一个坐标系。坐标系与参照物合称为\concept{参考系}。
显然,不同的参考系中$\vb*{r}$的形式可以很不一样。

关于参考系要额外提一句:因为参考系中定义了基矢量,在计算矢量导数的时候就产生了一个问题:这些基矢量会不会随时间变化?实际上根本不可能判断这一点。因此,实际上每个参考系都有它自己的“对时间求导”的算符。当然我们可以明确地判断两个参考系的基矢量是不是相对静止的。
在实际计算中,通常取一个参考系而假定其基矢量不变,称为\concept{实验室参考系}。

这导致了另一个微妙的地方:在列写欧拉-拉格朗日方程的时候我们有
\begin{equation}
    \pdv{L}{q} - \dv{t} \pdv{L}{\dot{q}} = 0,
    \label{eq:el-eq}
\end{equation}
从而有
\begin{equation}
    \pdv{L}{\vb*{r}} - \dv{t} \pdv{L}{\dot{\vb*{r}}} = 0.
    \label{eq:vector-el-eq}
\end{equation}
无论我们选取的参考系基矢量相对实验室参考系的基矢量有没有发生运动,\eqref{eq:vector-el-eq}中的时间导数都\concept{不能}考虑基矢量的变化,因为向量形式的\eqref{eq:vector-el-eq}只不过是将向量各分量当成独立坐标代入\eqref{eq:el-eq}之后拼起来的结果。%
\footnote{一个可能的疑惑:向量各分量不是要满足升降指标关系和坐标变换关系?为什么我们把它们当成了标量处理?要注意在处理欧拉-拉格朗日方程的时候我们实际上只关心“某些量在底流形上怎么分布”,而底流形只有一个维度就是时间。空间坐标之间的指标变换等关系相当于一个额外附加给“场量”的约束。由于牛顿力学中假定可以找到欧氏空间的惯性系,只要在欧氏空间中使用笛卡尔坐标系写出各坐标并构造合适的拉氏量,切换到其它坐标系中就不会有困难。}

首先考虑粒子自由运动或者说“不受到任何相互作用时”的运动情况。应当指出“不受相互作用”实际上没有良好定义——怎样确定一个物体有没有受到相互作用呢?
比如说,一个弹簧振子应该算是受到了相互作用吗?一方面似乎是的,因为有一个线性回复力作用在了振子上;
但另一方面似乎不是,因为弹簧振子可以就这样一直振动下去而不对外界产生影响,也不受外界影响。
因此实际上“不受相互作用”只是“最简单的系统”的一种直观说法。当然,通常要求“不受相互作用”的诸系统的拉氏量具有可加性。

\concept{假定}总是有一个参考系,在其中不受相互作用的粒子具有空间平移和时间平移不变性,以及空间的各向同性。
这个参考系就是一个\concept{惯性系}。
从空间和时间的平移不变性得到
\[
    L = L(\vb*{v})
\]
由于空间是各向同性的,则$L$不应该依赖于$\vb*{v}$的方向,这样就有
\[
    L = L(v^2)
\]
这样还是不能确定$L$的具体形式。
通过经验结果,\concept{假定}在一个惯性系中匀速运动的某个点构成了另一个惯性系,且不同惯性系中的拉氏量关于此惯性系中的速度的形式相同。
现在设$\vb*{v}_1$和$\vb*{v}_2$分别是两个惯性系中的速度,则一定可以找到一个常数$\vb*{v}_0$使
\[
    \vb*{v}_1 = \vb*{v}_2 + \vb*{v}_0
\]
并且$L(v_1^2)$和$L(v_2^2)$分别是两个恰当地描写了系统的拉氏量,则$L((\vb*{v}_1 - \vb*{v}_0)^2)$也恰当地描写了系统。
也就是说$L(v^2)$在对$\vb*{v}$做平移变换前后等价,从而
\[
    L((\vb*{v} + \delta \vb*{v})^2) - L(v^2) = \dv{f}{t},
\]
也就是
\[
    2 \pdv{L}{v^2} \vb*{v} \cdot \delta v = \dv{f}{t}
\]
注意到$\vb*{v} \cdot \delta v$本身已经是某个函数对时间的全导数了,则必须有
\[
    \pdv{L}{v^2} = \const
\]
因此$L$形如
\begin{equation}
    L = \frac{1}{2} m v^2.
    \label{eq:free-newton}
\end{equation}
如果我们认为作用量有极小值,那么一定有$m>0$。$m$就是\concept{质量}。

\eqref{eq:free-newton}给出了牛顿力学中自由粒子的拉氏量。当系统中有多个粒子且彼此间有相互作用时拉氏量成为
\begin{equation}
    L = \sum_i \frac{1}{2} m_i v_i^2 - U(\vb*{r}, \vb*{v}, t)
    \label{eq:interaction}
\end{equation}

\subsection{对称性和守恒量}

\subsubsection{平移不变性和动量守恒}

假定系统在方向$\vb*{n}$上具有平移不变性,也就是做变换
\[
    \vb*{r}^{(i)} \longrightarrow \vb*{r}^{(i)} + \epsilon \vb*{n}
\]
之后,拉氏量和原来的拉氏量等价。

引入力
\begin{equation}
    \vb*{F} = \dv{t} \pdv{U}{\dot{\vb*{r}}} - \pdv{U}{\vb*{r}}
\end{equation}

\subsection{达朗贝尔原理}

设$\vb*{F}$为
\begin{equation}
    \vb*{F}^{(i)} = \dv{t} \pdv{U}{\vb*{v}^{(i)}} - \pdv{U}{\vb*{r}^{(i)}} + F^\text{(nc)} + F^\text{(cons)}
\end{equation}
虚位移$\var{\vb*{r}^{(i)}}$满足约束方程,且为理想约束,则
\begin{equation}
    \sum_i (\vb*{F}^{(i)} - m_i \ddot{\vb*{r}}^{(i)}) \cdot \var{\vb*{r}^{(i)}} = 0
\end{equation}
即达朗贝尔原理。由于变分的任意性,只需要它就可以推导出所有的运动方程。

\chapter{二体问题}

\section{两体中心势}\label{sec:two-body}

\subsection{运动分析}

两体中心势在讨论天体力学问题时非常常用。相对而言天体力学是比较“干净”的系统,因为其中的各种相互作用可以比较明确地写出来,且没有复杂而混乱的粘滞、摩擦等现象。

设系统中有两质点质量为$m_1,m_2$,其位置矢量为$\vb*{r}_1, \vb*{r}_2$。假定系统的拉氏量形如
\begin{equation}
    L = \frac{1}{2} m_1 v_1^2 + \frac{1}{2} m_2 v_2^2 - U(r), \quad \vb*{r} = \vb*{r}_1 - \vb*{r}_2.
    \label{eq:two-body-original}
\end{equation}
这个拉氏量的形式等价于下面的断言:每一个质点都受到一个大小只依赖于两质点间距、方向与两质点连线共线的力,且这两个力大小相等方向相反。

引入质心
\begin{equation}
    \vb*{R} = \frac{m_1 \vb*{r}_1 + m_2 \vb*{r}_2}{m_1 + m_2}
\end{equation}
可以将拉氏量约化为
\[
    L = \frac{1}{2} M V^2 + \frac{1}{2} \mu v^2 - U(r), 
\]
其中$\mu$定义为
\begin{equation}
    \quad \frac{1}{\mu} = \frac{1}{m_1} + \frac{1}{m_2}
\end{equation}
称为\concept{约化质量}。
容易注意到$\vb*{V}$各分量为循环坐标,也即$\vb*{V}$始终不变,于是得出结论:二体中心势问题中质心速度不变。
那么拉氏量可以等价地改写为
\[
    L = \frac{1}{2} \mu v^2 - U(r)
\]
取球坐标$(r, \theta, \phi)$来描述$\vb*{r}$,拉氏量成为
\[
    L = \frac{1}{2} \mu \dot{r}^2 + \frac{1}{2} \mu r^2 \sin^2 \theta \dot{\phi}^2 + \frac{1}{2} \mu r^2 \dot{\theta}^2 - U(r)
\]
考虑关于$\theta$的欧拉-拉格朗日方程,有
\[
    2 r \dot{r} \dot{\theta} + r^2 \ddot{\theta} - r^2 \sin \theta \cos \theta \dot{\phi}^2 = 0,
\]
将$\theta=\pi/2, \dot{\theta}=0$代入上式得到$\ddot{\theta}=0$。
因此,$\theta=\pi/2$是一个解。
考虑初始条件就是$\theta|_{t=0}=\pi/2, \dot{\theta}|_{t=0}=0$,我们发现在此初始条件之下恒有$\theta=\pi/2$。
于是关于$r, \theta, \phi$拉氏量转化为关于$r, \phi$的拉氏量
\[
    L = \frac{1}{2} \mu (\dot{r}^2 + r^2 \dot{\phi}^2) - U(r)
\]
然后讨论关于$\dot{\phi}$的欧拉-拉格朗日方程
\[
    \dv{t} (\mu r^2 \dot{\phi}) = 0
\]
于是引入\concept{角动量}
\begin{equation}
    l_0 = \mu r^2 \dot{\phi} = \const
\end{equation}
这个运动积分包含了一个坐标的时间导数,因此不能直接用它在拉氏量中消去$\dot{\phi}$。于是我们讨论关于$r, \phi$的哈密顿量。计算广义动量
\[
    p_r = \pdv{L}{\dot{r}} = \mu \dot{r}, \quad p_\phi = \pdv{L}{\dot{\theta}} = \mu r^2 \dot{\phi}
\]
于是写出哈密顿量
\[
    H = p_r \dot{r} + p_\phi \dot{\phi} - L = \frac{p_r^2}{2\mu} + \frac{p_\phi^2}{2\mu r^2} + U(r)
\]
$p_\phi$就是$l_0$,因此
\begin{equation}
    H = \frac{p_r^2}{2\mu} + U_\text{eff} (r),
    \label{eq:hamiltionian-about-r}
\end{equation}
其中
\begin{equation}
    U_\text{eff} (r) = U(r) + \frac{l_0^2}{2\mu r^2}.
\end{equation}
这样就获得了仅仅关于$r$的动力学方程。按照\eqref{eq:hamiltionian-about-r}写出正则方程,有
\[
    \dot{r} = \pdv{H}{p_r} = \frac{p_r}{\mu}, \quad \dot{p_r} = - \pdv{H}{r} = - \pdv{U_\text{eff}}{r},
\]
也就是
\[
    \mu \ddot{r} = - \pdv{U_\text{eff}}{r}.
\]
这个方程就是直接使用牛顿力学列出的方程。此方程两边对$r$积分,或者,更加直截了当的,注意到\eqref{eq:hamiltionian-about-r}不显含时间因此$H$守恒,有
\[
    \frac{1}{2 \mu} p_r^2 + U_\text{eff}(r) = \frac{1}{2} \mu \dot{r}^2 + U_\text{eff}(r) = E_0 = \const
\]
从而
\begin{equation}
    \dot{r} = \pm \sqrt{\frac{2}{\mu} (E_0 - U_\text{eff}(r))}
    \label{eq:two-body-movement}
\end{equation}
因此一定有$E_0 \geq U_\text{eff}$。
并且,如果$r$有极大值或是极小值,那么此时一定有$E_0 = U_\text{eff}(r)$,因为$r$取极值时$\dot{r}$应为零。
上面的推导不能确定\eqref{eq:two-body-movement}中的正负号,而实际上也不应该确定——$r$会在轨道的一些地方增大、另一些地方减小。
\eqref{eq:two-body-movement}有形式解
\begin{equation}
    t = \sqrt{\frac{\mu}{2}} \int \frac{\pm \dd r}{\sqrt{E_0 - U_\text{eff}(r)}} + \const.
\end{equation}

\subsection{轨道方程}

很多时候我们并不关心具体的$r(t)$关系,而希望知道关于轨道的信息。使用
\[
    \dot{\phi} = \frac{l_0}{\mu r^2}
\]
以及\eqref{eq:two-body-movement},立即得到
\begin{equation}
    \dv{\phi}{r} = \pm \sqrt{\frac{l_0^2}{2\mu}} \frac{1}{r^2 \sqrt{E_0 - U_\text{eff}(r)}}
    \label{eq:two-body-orbit}
\end{equation}
这里的正负号也没有确定下来。这也是合理的,因为轨道在$r-\phi$平面上是一个闭合的曲线,不能使用一个单一的函数描述,而至少需要两条首尾相连的函数曲线。分别取正号和负号,正好得到这两条函数曲线。

以上推导只是确定了有一个大小有限的轨道,却不能说明轨道是否闭合。
轨道闭合意味着当$r$回到$r_\text{min}$时$\phi$的增量和$2\pi$之比必须为有理数。
通过变量代换
\begin{equation}
    r = \frac{1}{u}
\end{equation}
可以得到
\[
    \Delta \phi = - 2 \mp \int_{1/r_\text{max}}^{1/r_\text{min}} \frac{l_0 \dd u}{\sqrt{2\mu (E_0 - U_\text{eff}(1/u))}}
\]
Bertrand定理:
当且仅当$U \propto 1/r^2$和$U \propto r$时轨道封闭。

以上所有的论述都使用了有心势,但是也可以使用有心力,此时会产生二阶微分方程。

Binet方程:
\begin{equation}
    \dv[2]{u}{\phi} + u = - \frac{\mu}{l_0^2 u^2} f(1/u)
\end{equation}

\section{弹性散射}

所谓弹性散射指的是下面的一个场景:
\begin{enumerate}
    \item 一开始,一个粒子$m_1$从无穷远处以速度$\vb*{v}_i$入射,粒子$m_2$保持静止(不静止也没有关系,因为可以把参考系设在一开始的$m_2$上);
    \item 粒子$m_1$与$m_2$按照\eqref{eq:two-body-original}发生相互作用;
    \item 粒子$m_1$的运动方向发生偏转,粒子$m_2$开始运动,两个粒子再次相距无穷远。
\end{enumerate}
我们并不关心散射过程中相互作用的细节,而只关心终态,也就是两个粒子散射后的速度。
也就是说,要求出速度的大小和方向。

\subsection{单粒子散射}

讨论两个粒子的运动会带来数学处理上的麻烦。
因此我们首先在质心系中讨论问题,然后再切换回实验室坐标系。

实验室坐标系中质心速度为
\[
    \vb*{V} = \frac{m_1}{m_1 + m_2} \vb*{v}_\text{i}
\]
这个式子可以用来做速度转化。
在质心系中有
\[
    \vb*{r}_1' = \vb*{r}_1 - \vb*{R} = \frac{m_2}{m_1 + m_2} \vb*{r}, 
    \quad \vb*{r}_2' = \vb*{r}_2 - \vb*{R} = - \frac{m_1}{m_1 + m_2} \vb*{r}
\]
于是写出系统初态
\[
    \begin{split}
        \vb*{v}_{1\text{i}}' = \vb*{v}_\text{i} - \vb*{V} = \frac{m_2}{m_1 + m_2} \vb*{v}_\text{i}, \\
        \vb*{v}_{2\text{i}}' = 0 - \vb*{V} = - \vb*{V}.
    \end{split}
\]
由于系统不受外力,质心系中能量、动量守恒,于是
\[
    \begin{split}
        m_1 \vb*{v}'_{1\text{i}} + m_2 \vb*{v}'_{2\text{i}} = m_1 \vb*{v}'_{1\text{f}} + m_2 \vb*{v}'_{2\text{f}}, \\
        \frac{1}{2} m_1 v_\text{i}'^2 + \frac{1}{2} m_2 v_{2\text{i}}'^2 = \frac{1}{2} m_1 v_{1\text{f}}'^2 + \frac{1}{2} m_2 v_{2\text{f}}'^2
    \end{split}
\]
这两个方程是完整的相互作用\eqref{eq:two-body-original}的推论,而它们本身不能推出\eqref{eq:two-body-original},
因此必然有一些运动的细节不能够使用这两个方程得出。
它们的作用在于让我们对相互作用后粒子的运动状态有一个初步的了解。
我们不知道碰撞之后$\vb*{v}_1'$的方向转向了哪里,于是设
\[
    \vb*{v}_{1\text{f}}' = v_{1\text{f}}' \vb*{n},
\]
其中$\vb*{n}$为一个方向未定的单位矢量。
可以证明
\[
    v_{1\text{i}}' = v_{1\text{f}}', \quad v_{2\text{i}} = v_{2\text{f}}
\]
也就是散射结束后$m_1$和$m_2$的速度大小均不变,速度方向相反。
因此两个粒子散射后速度和散射前速度的夹角相等,设为$\theta$。

以上推导已经使用了能量、动量守恒方程提供的所有信息。因此可以看到,这两个方程没有提供的关于运动细节的信息就是$\theta$。

最后我们切换回实验室参考系。设$\Theta_1, \Theta_2$分别为实验室参考系中两个粒子速度方向的改变。
可以计算得到
\begin{equation}
    \tan \Theta_1 = \frac{\sin \theta}{m_1 / m_2 + \cos \theta}, \quad \Theta_2 = \frac{\pi - \theta}{2}
\end{equation}

因此只需要计算质心系中的散射角$\theta$,就能够完全确定实验室参考系中的散射情况。
而$\theta$是$\vb*{v}_{1}'$的偏转角,从而是$m_2 \dot{\vb*{r}} / (m_1 + m_2)$的偏转角,所以就是$\dot{\vb*{r}}$的偏转角。
在$t=-\infty$时$\vb*{r}$在无穷远处,此时$\phi=-\pi$,而散射结束,即$t=+\infty$时$\vb*{r}$再次到达无穷远处,此时设$\phi = \phi_\text{m}$。
TODO:画图
这个角度的公式是现成的——由\eqref{eq:two-body-orbit}可以直接写出
\[
    \begin{aligned}
        \pi - \theta &= \int_\text{whole orbit} \abs{ \frac{l_0}{r^2} \frac{\dd r}{\sqrt{2\mu (E_0 - U_\text{eff})}}} \\
        &= \left( - \int_{-\infty}^{r_\text{min}} + \int_{r_\text{min}}^\infty \right) \frac{l_0}{r^2} \frac{\dd r}{\sqrt{2\mu (E_0 - U_\text{eff})}} \\
        &= 2 \int_{r_\text{min}}^\infty \frac{l_0}{r^2} \frac{\dd r}{\sqrt{2\mu(E_0 - U(r)) - l_0^2 / r^2}}
    \end{aligned}
\]
其中$r_\text{min}$就是分母根号中的表达式的根。正负号是因为粒子从无穷远处到达$r_\text{min}$时$r$减小而$\phi$增大,粒子从$r_\text{min}$处到达无穷远处时$r$增大而$\phi$增大。
在无穷远处分析得到
\[
    \begin{split}
        l_0 = m v_\text{i} b, \\
        E_0 = \frac{1}{2} m v_\text{i}^2 + U(\infty)
    \end{split},
\]
总之,只需要计算积分
\begin{equation}
    \begin{bigcase}
        \pi - \theta = 2 \int_{r_\text{min}}^\infty \frac{l_0}{r^2} \frac{\dd r}{\sqrt{2\mu(E_0 - U(r)) - l_0^2 / r^2}}, \\
        l_0 = m v_\text{i} b, \\
        E_0 = \frac{1}{2} m v_\text{i}^2 + U(\infty)
    \end{bigcase}
    \label{eq:scattering-angle}
\end{equation}
就能够得到$\theta$和$b$之间的关系。至此两个粒子相互散射的问题已经完全解决。


\subsection{散射截面}

散射截面:入射粒子没有相互作用;靶粒子固定。
设在某个参考系当中一束截面为$A$的粒子入射之后被散射到了$\Omega$的立体角中。入射粒子相对靶粒子的纵向距离为$b$,相应的偏转角为$\theta$。
这里的$\theta$可以是质心系中的$\theta$,也可以是实验室参考系中的$\Theta$,又或者是别的什么参考系中的偏转角——只要这个参考系是惯性系并且相对实验室坐标系的运动速度和粒子入射速度同向,轴也同向(TODO)。
微分散射截面定义为这两者之间的线性关联函数,也就是
\[
    A = \int b \dd b \dd \chi = \int b \abs{\pdv{(b, \chi)}{(\theta, \chi)}} \dd \theta \dd \chi = \int \frac{b}{\sin \theta} \abs{\pdv{b}{\theta}} \sin \theta \dd \theta \dd \chi = \int \frac{b}{\sin \theta} \abs{\pdv{b}{\theta}} \dd \Omega
\]
$\theta$可以写成$b$的函数,于是$b$也可以写成$\theta$的函数,于是可以将偏导数写成全导数,得到
微分散射截面
\begin{equation}
    \sigma = \dv{A}{\Omega} = \frac{b(\theta)}{\sin \theta} \abs{\dv{b}{\theta}}.
    \label{eq:cross-section}
\end{equation}
当更换参考系导致偏转角的形式发生变化时,注意到
\[
    \dv{\Omega}{\Omega'} = \frac{\sin \theta \dd \theta}{\sin \theta' \dd \theta'}
\]
于是得到散射截面换算公式
\begin{equation}
    \frac{\sigma(\theta')}{\sigma(\theta)} = \frac{\sin \theta \dd \theta}{\sin \theta' \dd \theta'}
\end{equation}
例如实验室参考系中的散射截面为
\begin{equation}
    \sigma(\Theta) = \sigma(\theta) \frac{\sin \theta \dd \theta}{\sin \Theta \dd \Theta}
\end{equation}

现在考虑总截面,也就是
\begin{equation}
    S = \int_{\text{all}} \sigma \dd \Omega = 2\pi \int_0^{\pi} \sigma(\theta) \dd \theta
    \label{eq:total-crosssection}
\end{equation}
表面上看,由于入射面积可以任意大,\eqref{eq:total-crosssection}必然是发散的。
但实际上对很多相互作用而言\eqref{eq:total-crosssection}收敛。例如硬球散射
\[
    \sigma = \frac{1}{4} a^2
\]
的总截面就是$\pi a^2$。这是为什么?实际上,在\eqref{eq:scattering-angle}中会发现,随着$b$增大,$l_0$增大,$\theta$会减小。如果$\theta$减小得足够快,以至于在$b$较大的区域$\theta$的微小变动就会导致$b$的巨大变动(也就是说:很大一块入射区域上的入射粒子都被散射到了非常小的立体角中),那么对应的$\sigma(\theta)$就会变得很大。
将以上情况推至极端,让同一个$\theta$对应连续的一段$b$,此时$\sigma$发散,产生类似于$\delta$函数的结果。
由于我们计算散射截面时使用的都是完全连续的微积分工具,这些让$\sigma(\theta)$没有良定义的$\theta - b$值实际上全部被弃去了。容易看出这些被弃去的$\theta$只构成一个零测集。
这样一来,
\[
    \int \sigma \dd {\Omega}
\]
就给出“能够让$\sigma$有良定义的入射区域的面积”,也就是“入射粒子不会全部被挤压到同一个$\theta$的入射区域的面积”。
例如,在计算硬球散射时,大于球半径$a$的$b$全部被映射到$\theta=0$上,此时$\sigma(\theta)$无良定义,我们直接不考虑这些点。注意到$\theta=0$对应的立体角区域是一个零测集,因此我们有
\[
    \int \sigma \dd{\Omega} = \frac{1}{4} a^2 \cdot 4\pi = \pi a^2,
\]
这正是硬球的截面积。没有打到硬球上的入射粒子全部对应$\theta=0$,打到硬球上的入射粒子则必定被散射,因此硬球的截面积确实是入射粒子不会全部被挤压到同一个$\theta$的入射区域的面积。

如果总截面是发散的,说明它描述的相互作用即使对非常远处的物体也具有非常明显的作用,即这种相互作用是长程力。反之,它描述的相互作用是短程力。

\chapter{微振动}

许多力学体系具有一个或多个平衡位形。当体系略微偏离平衡位形时,体系可能在平衡位形附近微振动。

本节的研究局限于这样的系统:
\begin{enumerate}
    \item 不需要任何显含时间的广义坐标变换;
    \item 质点受力全保守,因此只需要一个拉氏量就能够描写系统,而不需要引入$Q_i$,并且拉氏量不显含时间;
    \item 体系不受约束或者只收到完整约束,从而,可以将受限坐标从拉氏量中消去,得到约化拉氏量。
\end{enumerate}

在直角坐标之下系统的拉氏量写成
\[
    L = \frac{1}{2} \sum_i M_i s_i^2 - U(s)
\]
但是我们使用的坐标$q$却未必使用直角坐标表示。由于假定了坐标变换$q\longrightarrow s$不显含时间,本节接下来讨论下面的拉氏量:
\begin{equation}
    L = \frac{1}{2} \sum_{k,l} m_{kl} \dot{q}_k \dot{q}_l - U(q) 
\end{equation}
其中
\begin{equation}
    m = \left(\pdv{s}{q}\right)^\top M \left(\pdv{s}{q}\right)  
\end{equation}
$m$显然是对称的、正定的,因为$M$是对称、正定的。

\[
    \dv{t} \sum_{i} m_{ij}(q) \dot{q}_j - \frac{1}{2} \sum_{j, k} \pdv{m_{jk}}{q_i} \dot{q}_j \dot{q}_k - \pdv{U}{q_i} = 0
\]

系统平衡意味着下面的解
\[
    q_i(t) = q_{i0}, \quad \dot{q}_i(t) = 0
\]
是一个解,也就是说
平衡的充要条件为
\begin{equation}
    \pdv{U}{q} = 0
\end{equation}

定义$r = q - q^{(0)}$,则从$s$到$r$的变换是非奇异的。

\[
    \begin{aligned}
        L &= \frac{1}{2} \sum_{i,j} m_{ij} (q^{(0)}+r) \dot{r}_i \dot{r}_j - U(q^{(0)}+r) \\
        &= \frac{1}{2} \sum_{i,j} m_{ij}(q^{(0)}) \dot{r}_i \dot{r}_j - U(q^{(0)}) - \sum_{i} \eval{\pdv{U}{q_i}}_{q=q_0} r_i 
    \end{aligned}
\]

\begin{equation}
    \sum_{j} (T_{ij} \ddot{r}_j + V_{ij} r_j) = 0
    \label{eq:viberating-equation}
\end{equation}

\[
    T \ddot{r} + V r = 0
\]
要将$T,V$同时做对角化。

具体的方法是广义本征值法,
\begin{equation}
    V z^{(k)} = \theta_k T z^{(k)}
\end{equation}

在能够确定系统的每一个自由度都在做振动时,其实直接对\eqref{eq:viberating-equation}做傅里叶变换就可以了。设
\begin{equation}
    r_i = A_i \ee^{\ii \omega t}
\end{equation}

那么就有
\begin{equation}
    (-\omega^2 T + V) A = 0
    \label{eq:amplitude}
\end{equation}
求解
\begin{equation}
    \det (-\omega^2 T + V) A = 0
\end{equation}

A是振动模式,$\omega$是圆频率。

要求系统的每一个自由度都在做振动是为了确保傅里叶变换的合法性,
否则某个自由度在时间趋于正负无穷时发散,则傅里叶变换不存在。
然而,此时仍然可以假定$\omega$有虚部而做拉普拉斯变换,其结果的形式完全一样。

初始条件决定了什么样的频率被激发出来。

\chapter{刚体}

\section{刚体的运动学}

刚体指的是这样的一个系统:其中任何两个粒子之间的距离都保持不变。设我们已经有一个惯性系,称为实验室坐标系,其中各点由矢量$\vb*{x}$标记,则其中的刚体需要且只需要满足
\begin{equation}
    \dv{t} \abs{\vb*{x}^{(i)} - \vb*{x}^{(j)}} = 0
    \label{eq:rigid-body-def}
\end{equation}
这当然就等价于
\[
    \dv{t} (\vb*{x}^{(i)} - \vb*{x}^{(j)})^2 = 0,
\]
也就是
\begin{equation}
    (\vb*{x}^{(i)} - \vb*{x}^{(j)}) \cdot (\dot{\vb*{x}}^{(i)} - \dot{\vb*{x}}^{(j)}) = 0.
    \label{eq:rigid-body-motion-constraint}
\end{equation}
方程\eqref{eq:rigid-body-motion-constraint}的导出每一步都是等价的,因此它实际上是刚体的一个等价定义。

下面我们要讨论刚体如何运动。所谓运动,就是从一个初始态到终末态的可微的变换。刚体就是定义在三维欧氏空间中、满足\eqref{eq:rigid-body-def}的质点系,因此刚体所有可能的运动就是满足\eqref{eq:rigid-body-def}的可微变换。
直觉上看刚体的运动可以分成两部分:远距离看也能够看到的“平动”,以及需要走近追踪每一个点运动过程才能测定的转动。
基于这样的直觉,我们首先分析“转动”,也就是刚体中不同点之间的相对位置的变化情况。
我们在刚体中选择一个点$\vb*{x}^{(0)}$,称其为\concept{基点},然后将参考系选择在它上面但不改变基矢量的方向,
也就是做代换$\vb*{x} - \vb*{x}^{(0)} \longrightarrow \vb*{r}$,
那么就有
\begin{equation}
    \vb*{r}^{(i)} \cdot {\dot{\vb*{r}}}^{(i)} = 0, \quad (\vb*{r}^{(i)} - \vb*{r}^{(j)}) \cdot (\dot{\vb*{r}}^{(i)} - \dot{\vb*{r}}^{(j)}) = 0
    \label{eq:around-a-point}
\end{equation}
这样的运动我们称其为\concept{定点转动}。为什么这是“转动”接下来马上介绍。

首先指出一个事实:对整个刚体有一个线性变换$A$,使得刚体上任意一点相对基点的位矢$\vb*{r}$转动前后的值由$A$联系。
要看出这是为什么,不失一般性地假定刚体中的质点数目大于等于3(为什么不失一般性在本节的结尾可以看到),设我们从刚体中选取了三个各不相同且不共线的点,记它们相对基点的位矢为$\vb*{r}_1, \vb*{r}_2, \vb*{r}_3$。
现在设变换前某一个点的位矢为$\vb*{r}$,变换之后为$\vb*{r}'$,那么就有
\[
    \begin{aligned}
        (\vb*{r} - \vb*{r}_1)^2 &= (\vb*{r}' - \vb*{r}_1')^2, \\
        (\vb*{r} - \vb*{r}_2)^2 &= (\vb*{r}' - \vb*{r}_2')^2, \\
        (\vb*{r} - \vb*{r}_3)^2 &= (\vb*{r}' - \vb*{r}_3')^2.
    \end{aligned}
\]
由\eqref{eq:around-a-point}得到
\[
    \vb*{r}_1^2 = \vb*{r}_1'^2, \quad \vb*{r}_2^2 = \vb*{r}_2'^2, \quad \vb*{r}_3^2 = \vb*{r}_3'^2, \quad \vb*{r}^2 = \vb*{r}'^2
\]
于是得到
\[
    \begin{aligned}
        \vb*{r}' \cdot \vb*{r}_1' &= \vb*{r} \cdot \vb*{r}_1, \\
        \vb*{r}' \cdot \vb*{r}_2' &= \vb*{r} \cdot \vb*{r}_2, \\
        \vb*{r}' \cdot \vb*{r}_3' &= \vb*{r} \cdot \vb*{r}_3
    \end{aligned} 
\]
由于$\vb*{r}_1, \vb*{r}_2, \vb*{r}_3$不共线,我们可以知道$\vb*{r}'_1, \vb*{r}'_2, \vb*{r}'_3$也不共线,不然就不能保证变换前后所有点的距离相等。那么,我们就得到了三个关于$\vb*{r}'$的彼此无关的线性方程,它们也关于$\vb*{r}$线性。
因此可以将$\vb*{r}'$写成$\vb*{r}$的一个线性变换:
\[
    \vb*{r}' = A \vb*{r}
\]
注意到我们只需要从刚体中的所有点中取出不共线的三个,就能够使用完全一样的方程算出所有点变换之后的位置,因此刚体上每一点的运动都可以用这个线性变换描述。

现在的问题是,怎样的$A$是被允许的?
由于
\[
    \abs{\vb*{r}'} = \abs{A \vb*{r}} = \abs{\vb*{r}},
\]
$A$应该是一个正交矩阵。
也就是说,刚体的定点转动是$\ogroup{3}$群的成员。
然而,并非所有$\ogroup{3}$群中的变换都是可能的转动。实际上,行列式为$-1$的正交矩阵都不是合法的转动,因为可以做分解
\[
    A = (-I) (-A), \quad \det (-I) = -1, \; \det (-A) > 0
\]
而$-I$是一个反射变换,它不可微——我们不可能在一瞬间将所有的粒子对着基点做反射变换!
这件事的深层根源在于$\ogroup{3}$并不是一个李群——我们不能寻找一个“很小的”反射变换。
在要求了所有变换都可微的前提下,我们注意到\eqref{eq:around-a-point}等价于$A$是正交矩阵,
因此\eqref{eq:around-a-point}——从而\eqref{eq:rigid-body-motion-constraint}——等价于$A$是一个正交矩阵并且可微,
也就是说,$A$是$\sogroup{3}$的成员。
于是我们就得到了结论:所有可能的刚体定点转动就是$\sogroup{3}$。
因此定点转动确实是转动——它就是转动群$\sogroup{3}$。

在定点转动中有一类特殊的转动称为\concept{定轴转动}。所谓定轴转动指的是能够找到一个空间方向$\vb*{n}$使得沿着这个方向的$\vb*{r}$不会被改变,也就是说,
\[
    A \vb*{n} = \vb*{n}
\]
我们称这样的$\vb*{n}$指示了这个转动的\concept{轴}。
这是一个本征值问题,它有非零解意味着
\[
    \det (A - I) = 0
\]
从而
\[
    \begin{aligned}
        \det (A - I) &= \det A(I - A^{-1}) \\
        &= \det A \det (I - A^{-1}) \\
        &= \det (I - A^{-1}) \\
        &= \det (I - A^\top) \\
        &= \det (I - A)
    \end{aligned}
\]
我们在三维空间下,因此
\[
    \det (A - I) = \det (I - A) = (-1)^3 \det (A - I)
\]
因此$(A-I)$行列式确实是零,因此一定能够找到一个方向$\vb*{n}$使得$A$是以它为转轴的定轴转动。可以看出这个结论对空间的维度是由要求的。

不过需要指出的是,设转动$A=BC$,那么$B$、$C$对应的轴未必相同,它们的轴和$A$的轴也没有特别必然的关系。
因此我们只能够对着一段有限时间内的转动说“它可以看成沿着某根轴$\vb*{n}$的转动”,却不能保证这一段转动内部的每一小段转动都沿着$\vb*{n}$。

现在已经知道了刚体在有限一段时间内运动的方式。在实际求解刚体运动时往往需要使用微分方程,因此有必要讨论无穷小转动的形式。所有允许的无穷小转动就是$\sogroup{3}$的无穷小生成元,也就是所有的反对称矩阵,形如
\[
    \bmqty{0 & a & b \\ -a & 0 & c \\ -b & -c & 0}
\]
我们会发现将这样一个矩阵作用在一个矢量上面就相当于找了三个量$a_1, a_2, a_3$然后计算$\epsilon_{ijk} a_i r_j \vb*{e}_k$,
并且能够证明,当对坐标系做转动时$a_1, a_2, a_3$的变换合于矢量的坐标变换关系(在坐标系做反演变换时这三个量的正负号会改变),
因此它们构成一个轴矢量。
因此无穷小转动形如$\epsilon \vb*{\omega} \times \vb*{r}$,其中$\vb*{\omega}$是一个轴矢量。
由于转动矩阵$A$在刚体上任何一点都是一样的,$\vb*{\omega}$在刚体上任何一点也都是一样的。因此我们有
\begin{equation}
    \dot{\vb*{r}}^{(i)} = \vb*{\omega} \times \vb*{r}^{(i)}
    \label{eq:body-frame-motion}
\end{equation}
其中$\vb*{\omega}$在整个刚体上都是一样的。
这样一来就能够写出
\begin{equation}
    \dot{\vb*{x}}^{(i)} - \dot{\vb*{x}}^{(j)} = \vb*{\omega} \times (\vb*{x}^{(i)} - \vb*{x}^{(j)})
\end{equation}
现在可以注意到另一个非常有趣的地方。设$\vb*{\omega}_1$和$\vb*{\omega}_2$分别是两个基点对应的$\vb*{\omega}$,那么同时有
\[
    \dot{\vb*{x}}^{(i)} - \dot{\vb*{x}}^{(j)} = \vb*{\omega}_1 \times (\vb*{x}^{(i)} - \vb*{x}^{(j)}) = \vb*{\omega}_2 \times (\vb*{x}^{(i)} - \vb*{x}^{(j)})
\]
\[
    (\vb*{x}^{(i)} - \vb*{x}^{(j)}) \times (\vb*{\omega}_1 - \vb*{\omega}_2) = 0
\]
由于刚体中的质点数多于两个,以上方程等价于数目在3以上的独立方程,但它确实有解,因此唯一的可能就是$\vb*{\omega}_1 = \vb*{\omega}_2$。
因此角速度和基点的选择没有关系。

注意到虽然$\sogroup{3}$并不是一个交换群,它的李代数却(按照李群一般的性质)对加法交换,也就是说虽然转动不可交换,但是无穷小转动确实可以交换。
作为对比,无穷小平动可以交换(李代数的性质),而有限小的平动也可以交换。
另一个值得注意的点是,通过合适的坐标变换,$\vb*{\omega} \times \vb*{r}$实际上对应着一个定轴转动。
如果$\vb*{r}$的方式和$\vb*{\omega}$一致,那么就有$\vb*{\omega}\times \vb*{r} \dd t =0$,因此$\vb*{r}$在变换之后不会产生任何变化。
于是称$\vb*{\omega}$的方向为\concept{瞬时轴},
无穷小转动$\vb*{\omega} \dd t = \omega_1 \vb*{e}_1 + \omega_2 \vb*{e}_2 + \omega_3 \vb*{e}_3$
相当于分别在在$\vb*{e}_1, \vb*{e}_2, \vb*{e}_3$方向上转动$\omega_1 \dd t, \omega_2 \dd t, \omega_3 \dd t$。
原本这三个转动操作的复合和次序有关,但在无穷小转动的情况下,它们的复合和顺序就没有关系。

然后我们讨论两个质点的情况,此时只需要将基点选择在其中一个上面,自然可以证明以上所有的结论都适用。至此刚体的运动方式完全确定。

最后再提一个有用的技巧:实际上,完全不需要把基点选择在刚体系统内部。随意寻找一个空间点,设它在实验室坐标系中的位置为$\vb*{x}_0$,则下面的方程组
\begin{equation}
    \begin{split}
        \vb*{x}|_{t=0} = \vb*{x}_0, \\
        \dot{\vb*{x}} = \dot{\vb*{r}} + \vb*{\omega} \times (\vb*{x} - \vb*{r})
    \end{split}
    \label{eq:extended-part}
\end{equation}
必定有解,其中$\vb*{r}$是刚体中的一个质点。
现在将$\vb*{x}$加入刚体质点系内,则容易验证加入之后的质点系还是一个刚体,
而由于它的坐标不会出现在系统的拉氏量内,系统的动力学不会发生丝毫改变,因此将$\vb*{x}(t)$加入质点系中不改变其行为。
因此永远可以将一系列虚设的、位置随着时间变化使得它离其余质点的距离始终不变的点加入刚体质点系中而不改变系统的行为,
而我们自然可以将基点选取在这些点上,而仍然保持\eqref{eq:around-a-point}成立,从而从它推导出来的一切结论成立。
特别的,基点常常被选择在刚体的质心$\vb*{R}$上——不难检验质心满足\eqref{eq:extended-part}。

\subsection{刚体姿态与欧拉角}\label{sec:euler-angle}

我们选择一个“固着”在刚体上面的坐标系,称为\concept{本体系},在其中刚体完全静止。
这个坐标系是不是存在姑且还需要讨论。
设它的基点在实验室坐标系中为$\vb*{x}_B$,并且我们就令$\vb*{r}$为以它为基点的位矢。
本体系的三个基矢量为$\vb*{e}'_1, \vb*{e}'_2, \vb*{e}'_3$,组成一个正交右手系。刚体上各点到此基点的位矢为$\vb*{r}^{(i)}$。
显然,参考系是本体系的充要条件是
\begin{equation}
    \vb*{r}^{(i)} = 
    \eval{r^{(i)}_1}_{t=0} \vb*{e}'_1 + \eval{r^{(i)}_2}_{t=0} \vb*{e}'_2 + \eval{r^{(i)}_3}_{t=0} \vb*{e}'_3
    \label{eq:position-vector-in-body-frame}
\end{equation}
在计算此矢量的导数时有两种方法,一种是考虑各个基矢量$\vb*{e}'_1, \vb*{e}'_2, \vb*{e}'_3$在实验室参考系中的变化,
另一种是将各个基矢量看成不变的,也就是说在本体系中计算导数。
此时\eqref{eq:body-frame-motion}应当写成
\[
    \left( \dv{\vb*{r}^{(i)}}{t} \right)_\text{lab} = \vb*{\omega} \times \vb*{r}^{(i)}
\]
考虑到\eqref{eq:position-vector-in-body-frame}中的分量均不会发生变化,\eqref{eq:position-vector-in-body-frame}等价于
\begin{equation}
    \left( \dv{\vb*{e}_i'}{t} \right)_\text{lab} = \vb*{\omega} \times \vb*{e}'_i, \quad i = 1, 2, 3
    \label{eq:basis-of-body-frame}
\end{equation}
给定初始条件——通常就将$t=0$时的$\vb*{e}$规定为$\vb*{e}$也就是实验室坐标系中的基矢量,以上方程一定有解,并且正交归一性能够保持。
因此本体系总是存在的。同样方程组\eqref{eq:basis-of-body-frame}和基点规定无关,因此本体系的坐标轴方向和基点选择无关。\footnote{到现在为止我们已经有了三种参考系:首先是实验室坐标系,其中的位矢为$\vb*{x}$,基矢量为$\vb*{e}_1, \vb*{e}_2, \vb*{e}_3$;其次是平动系,也就是基点选在刚体上或者刚体的延拓部分上的参考系,其中的位矢为$\vb*{r}$,基矢量为$\vb*{e}_1, \vb*{e}_2, \vb*{e}_3$;最后是本体系,位矢也是$\vb*{r}$,但是基矢量为“固着”在刚体上的$\vb*{e}'_1, \vb*{e}'_2, \vb*{e}'_3$。}

我们会注意到,一旦选定了一个基点,而又知道了$\vb*{e}'$的表达式,
那么根据\eqref{eq:position-vector-in-body-frame}就能够写出每一时刻的刚体中所有质点的位置。
$\vb*{e}'$和基点选择无关,
而基点只需要选择在刚体上任意一点(或者根据\eqref{eq:extended-part}确定的刚体想象中的附带部分)就可以了,同样和$\vb*{e}'$无关。
进一步的,$\vb*{e}'$可以写成$\vb*{e}$通过一个$\sogroup{3}$的元素变换之后的结果
(\eqref{eq:basis-of-body-frame}给出了$\vb*{e}'$的无穷小变化,而这个无穷小变化由叉乘给出,正是$\sogroup{3}$的无穷小生成元),
而一个$\sogroup{3}$元素总可以使用三个参数表示。
因此在刚体的几何构型完全确定,从而$\vb*{r}$完全确定的情况下,需要三个参数(基点在实验室坐标系中的位置)来描述基点的位置,
三个参数来描述$\vb*{e}'$,所以我们说:刚体运动的自由度最多为六个。

我们会怎样选择这三个描述$\vb*{e}'$的参数呢?通常的方法是写出三个绕着不同的轴做旋转的矩阵,

\section{刚体动力学}

\subsection{刚体的拉氏量}

现在尝试导出刚体的拉氏量。这个拉氏量本身形如\eqref{eq:interaction},在实验室参考系中就是
\[
    L = \sum_i \frac{1}{2} m_i (\dot{\vb*{x}}^{(i)})^2 - U
\]
但是它并没有什么用,因为还有约束\eqref{eq:rigid-body-def}。注意到这个约束实际上是一个理想约束:它可以写成
\[
    \abs{\vb*{r}^{(i)} - \vb*{r}^{(j)}} = \eval{r_{ij}}_{t=0}
\]
这样的形式,因而不显含时间和速度。
因此必定可以将\eqref{eq:interaction}和\eqref{eq:rigid-body-def}结合起来来获得一个约化拉氏量。
我们知道刚体有6个自由度,因此这个约化拉氏量一定只显含6个独立的坐标。

现在选择一个合适的基点,就可以有
\[
    \dot{\vb*{x}}^{(i)} = \vb*{V} + \vb*{\omega} \times \vb*{r}^{(i)}
\]
从而
\[
    \sum_i \frac{1}{2} m_i (\dot{\vb*{x}}^{(i)})^2 = 
    \frac{1}{2} \sum_i m_i V^2 + \vb*{V} \cdot \sum_i m_i \vb*{\omega} \times \vb*{r}^{(i)} 
    + \frac{1}{2} \sum_i m_i (\vb*{\omega} \times \vb*{r}^{(i)})^2
\]
现在如果将基点选在质心,就有
\[
    \sum_i m_i \vb*{r}^{(i)} = 0
\]
从而
\[
    \sum_i \frac{1}{2} m_i (\dot{\vb*{x}}^{(i)})^2 = \frac{1}{2} \sum_i m_i V^2 
    + \frac{1}{2} \sum_i m_i (\vb*{\omega} \times \vb*{r}^{(i)})^2
\]
这就可以看出刚体动力学中质心的作用——它大大简化了问题。另一种选择基点的方式是将它选在实验室坐标系中某个静止不动的东西上面,此时$\vb*{V}=0$,同样可以消去交叉项。
\[
    \begin{aligned}
        \sum_i m_i (\vb*{\omega} \times \vb*{r}^{(i)})^2 
        &= \sum_i m_i ((r^{(i)})^2 \omega^2 - (\vb*{\omega} \cdot \vb*{r}^{(i)})) \\
        &= \vb*{\omega} \cdot \left( \sum_i m_i ((r^{(i)})^2 \vb*{\delta} - \vb*{r}^{(i)} \vb*{r}^{(i)} ) \right)
    \end{aligned}
\]
其中$\vb*{\delta}$为单位张量。于是定义\concept{惯性张量}
\begin{equation}
    \vb*{I} = \sum_i m_i {(r^{(i)}}^2 \vb*{\delta} - \vb*{r}^{(i)} \vb*{r}^{(i)} ) 
    \label{eq:ineria-tensor}
\end{equation}
它确实是一个张量——它的定义完全是坐标系无关的。
使用惯性张量,刚体在实验室坐标系中的动能就是
\begin{equation}
    T = \frac{1}{2} M V^2 + \frac{1}{2} \vb*{\omega} \cdot \vb*{I} \cdot \vb*{\omega}
    \label{eq:kinetic-energy-rigid-body}
\end{equation}
动能已经写成了关于刚体质心速度$\vb*{V}$和角速度$\vb*{\omega}$的二次型。很容易可以验证,在约束\eqref{eq:rigid-body-def}下这两个矢量的各个分量可以独立变化,也就是说$\vb*{V}$和$\vb*{\omega}$对应着6个自由度。
$\vb*{V}$对应的坐标是$\vb*{R}$,也就是质心位置,$\vb*{\omega}$对应的坐标则为$\int \vb*{\omega} \dd t$的各分量——注意它们并不是欧拉角,虽然它们与欧拉角之间很容易相互转换。

改用质心位置和$\int \vb*{\omega} \dd t$表示$U$,此时拉氏量成为
\begin{equation}
    L = \frac{1}{2} M v^2 + \frac{1}{2} \vb*{\omega} \cdot \vb*{I} \cdot \vb*{\omega} - U
    \label{eq:rigid-body-lagrangian}
\end{equation}
这个拉氏量已经是约化拉氏量了,因为它只含有自由坐标。$M, \vb*{I}$由于不随时间变化,可以当成是给定量。

\subsection{惯性张量的性质与刚体主轴}

惯性张量在任意一个坐标系下的表示就是\concept{惯量矩阵}。现在我们需要讨论怎样能够让惯量矩阵对角化,也就是要求\eqref{eq:ineria-tensor}的某个对角矩阵。
注意到惯性张量是一个正定、对称的张量,那么就可以做特征分解,得到一组单位正交基,在这组基下惯量矩阵是对角化的;

\begin{equation}
    \vb*{I} = I_1 \vb*{e}_1 \vb*{e}_1 + I_2 \vb*{e}_2 \vb*{e}_2 + I_3 \vb*{e}_3 \vb*{e}_3
\end{equation}

$\vb*{r}' = \vb*{A} \cdot \vb*{r}$其中$\vb*{A}$为正交张量。
也就是,惯性张量的主轴就构成了本体系的一组基。当然也可以把它固着在其它参照物上,形成非本体系的参考系。

当本体系以惯性张量的主轴为基时,惯性张量的各个分量不会发生变化。这是因为
\[
    I_{jk} = \sum_i m_i \vb*{e}_j \cdot ({r^{(i)}}^2 \vb*{\delta} - \vb*{r}^{(i)}\vb*{r}^{i}) \cdot \vb*{e}_k = \sum_i m_i ({r^{(i)}}^2 \delta_{ij} - \eval{r^{(i)}_j r^{(i)}_k}_{t=0}),
\]
其中的所有量都不随时间变化。

\begin{equation}
    \vb*{I}' = \vb*{A} \cdot \vb*{I} \cdot \vb*{A}^\top
\end{equation}

平行轴、垂直轴等

如果刚体有对称面,那么一定有一根主轴垂直于对称面;如果刚体有对称轴,那么对称轴是一根主轴,并且在垂直于对称轴的平面内,任何两条相互垂直的轴都构成主轴,且对应的主惯量相等。

\subsection{欧拉方程}

设角速度对应的坐标为$\vb*{\varphi}$,即
\[
    \vb*{\varphi} = \int \vb*{\omega} \dd t
\]
回顾$\vb*{R}$和$\vb*{\omega}$的表达式,
% \[
%     \vb*{x}^{(i)} = \underbrace{\vb*{R}}_\text{与$\vb*{R}$有关} + 
%     \underbrace{\timeorder \exp (\int \Omega \dd t) \vb*{x}^{(i)}_0}_\text{与$\vb*{\varphi}$有关}
% \]
\[
    \dot{\vb*{x}}^{(i)} = \vb*{\omega} \times (\vb*{x}^{(i)} - \vb*{R}) + \vb*{V}
\]
\[
    \dd{\vb*{x}^{(i)}} = \vb*{\varphi} \times (\vb*{x}^{(i)} - \vb*{R}) + \dd{\vb*{R}}
\]
这是一个可溯约束但不是完整约束,因此不能直接使用欧拉拉格朗日方程,而要使用达朗贝尔原理。

从拉氏量\eqref{eq:rigid-body-lagrangian}得到运动方程
\[
    \pdv{L}{\vb*{R}} = \sum_i \pdv{\vb*{x}^{(i)}}{\vb*{R}} \cdot \dv{t} \pdv{L}{\dot{\vb*{x}}^{(i)}}, \quad \pdv{L}{\vb*{\varphi}} = \sum_i \pdv{\vb*{x}^{(i)}}{\vb*{\varphi}} \cdot \dv{t} \pdv{L}{\dot{\vb*{x}}^{(i)}}
\]
\[
    \pdv{\vb*{R}} \left( \frac{1}{2} M V^2 - U \right) = \dv{t} \left(M \vb*{V} - \pdv{U}{\vb*{V}}\right), \quad \pdv{\vb*{\varphi}} \left( \frac{1}{2} \vb*{\omega} \cdot \vb*{I} \cdot \vb*{\omega} - U \right) = \dv{t} \left( \vb*{I} \cdot \vb*{\omega} - \pdv{U}{\vb*{\omega}} \right)
\]
我们实际上只知道$U$和$\vb*{x}$之间的关系,所以需要使用链式法则。

由于$M$是常量,直接得到关于$\vb*{V}$的方程为
\[
    M \dv{t} \vb*{V} = \dv{t} \pdv{U}{\vb*{V}} - \pdv{U}{\vb*{R}}
\]
计算$U$的导数项得到
\[
    \begin{aligned}
        \dv{t} \pdv{U}{\vb*{V}} - \pdv{U}{\vb*{R}} &=
        \dv{t} \sum_i \pdv{\dot{\vb*{x}}^{(i)}}{\vb*{V}} \cdot \pdv{U}{\dot{\vb*{x}}^{(i)}} 
        - \sum_i \pdv{\vb*{x}^{(i)}}{\vb*{R}} \cdot \pdv{U}{\vb*{x}^{(i)}}  \\
        &= \dv{t} \sum_i \pdv{U}{\dot{\vb*{x}}^{(i)}} - \sum_i \pdv{U}{\vb*{x}^{(i)}} \\
        &= \sum_i \vb*{F}^{(i)}
    \end{aligned}
\]
于是得到质心运动方程
\begin{equation}
    M \dv{t} \vb*{V} = \sum_i \vb*{F}^{(i)}.
    \label{eq:mass-center-motion}
\end{equation}

关于$\vb*{\omega}$的方程要略微复杂一些,因为$\vb*{I}$在实验室坐标系中会发生变化。
\[
    \dv{t} \left( \vb*{I} \cdot \vb*{\omega} \right) - \frac{1}{2} \pdv{\vb*{\varphi}} (\vb*{\omega} \cdot \vb*{I} \cdot \vb*{\omega}) = \vb*{I} \cdot \dot{\vb*{\omega}} + \dot{\vb*{I}} \cdot \vb*{\omega} = \dv{t} \pdv{U}{\vb*{\omega}} - \pdv{U}{\vb*{\varphi}}
\]
考虑无穷小转动
\[
    \vb*{A} = \vb*{\delta} + \epsilon_{ijk} \omega_i \vb*{e}_k \vb*{e}_j
\]
可以得到$\vb*{I}$的无穷小变化
\[
    \begin{aligned}
        \vb*{I} + \dd \vb*{I} &= \vb*{A} \cdot \vb*{I} \cdot \vb*{A}^\top - \vb*{I} \\
        &= \vb*{I} + \epsilon_{ijk} \omega_i I_{jl} \vb*{e}_k \vb*{e}_l \dd t + \epsilon_{ijk} \omega_i I_{lk} \vb*{e}_l \vb*{e}_j \dd t
    \end{aligned}
\]
从而计算出其导数和$\vb*{\omega}$点乘之后的结果
\[
    \begin{aligned}
        \dv{\vb*{I}}{t} \cdot \vb*{\omega} &= (\epsilon_{ijk} \omega_i I_{jl} \vb*{e}_k \vb*{e}_l + \epsilon_{ijk} \omega_i I_{lk} \vb*{e}_l \vb*{e}_j) \cdot \omega_m \vb*{e}_m \\
        &= \epsilon_{ijk} \omega_i (I_{jl} \omega_l) \vb*{e}_k \\
        &= \vb*{\omega} \times (\vb*{I} \cdot \vb*{\omega})
    \end{aligned}
\]
同样
\[
    \pdv{\vb*{\varphi}} (\vb*{\omega} \cdot \vb*{I} \cdot \vb*{\omega}) = \vb*{\omega} \cdot \pdv{\vb*{I}}{\vb*{\varphi}} \cdot \vb*{\omega},
\]
而
\[
    \pdv{\vb*{I}}{\varphi_i} = \epsilon_{ijk} (I_{jl} \vb*{e}_k \vb*{e}_l + I_{lk} \vb*{e}_l \vb*{e}_j),
\]
\[
    \pdv{\vb*{I}}{\vb*{\varphi}} = \epsilon_{ijk} \vb*{e}_i (I_{jl} \vb*{e}_k \vb*{e}_l + I_{lk} \vb*{e}_l \vb*{e}_j) 
\]
可以验证
\[
    \vb*{\omega} \cdot \pdv{\vb*{I}}{\vb*{\varphi}} \cdot \vb*{\omega} = 0.
\]
实际上可以直接通过$\vb*{I}$在主轴本体系下的展开来导出以上两式。

于是得到关于$\vb*{\omega}$的方程
\begin{equation}
    \vb*{I} \cdot \dot{\vb*{\omega}} + \vb*{\omega} \times (\vb*{I} \cdot \vb*{\omega}) = \dv{t} \pdv{U}{\vb*{\omega}} - \pdv{U}{\vb*{\varphi}}
    \label{eq:origin-omega-motion}
\end{equation}

由于难以从编时算符中恢复出$\vb*{\varphi}$,取下面形式的微分,保持$\vb*{R},\vb*{V}, \vb*{\omega}$不变:
$\dd \vb*{x}^{(i)} = \dd \vb*{\varphi} \times (\vb*{x}^{(i)} - \vb*{R})$
则
\[
    \begin{aligned}
        \dd U &= \sum_i \pdv{U}{\vb*{x}^{(i)}} \cdot \dd \vb*{x}^{(i)} = \sum_i  \pdv{U}{\vb*{x}^{(i)}} \cdot (\dd \vb*{\varphi} \times (\vb*{x}^{(i)} - \vb*{R})) \\
        &= \sum_i \dd \vb*{\varphi} \cdot \left( (\vb*{x}^{(i)} - \vb*{R}) \times \pdv{U}{\vb*{x}^{(i)}} \right)
    \end{aligned}
\]

\[
    \begin{aligned}
        \dv{t} \pdv{U}{\vb*{\omega}} - \pdv{U}{\vb*{\varphi}} &= \sum_i \left(\dv{t} \left(\pdv{\dot{\vb*{x}}^{(i)}}{\vb*{\omega}} \cdot \pdv{U}{\dot{\vb*{x}}^{(i)}}\right) - (\vb*{x}^{(i)} - \vb*{R}) \times \pdv{U}{\vb*{x}^{(i)}}\right) \\
        &= \sum_i \left(\dv{t} \left((\vb*{x}^{(i)} - \vb*{R}) \times \pdv{U}{\dot{\vb*{x}}^{(i)}}\right) - (\vb*{x}^{(i)} - \vb*{R}) \times \pdv{U}{\vb*{x}^{(i)}}\right) \\
        &= 
    \end{aligned}
\]

稳定性
\begin{equation}
    \begin{bigcase}
        I_1 \dot{\omega}_1 &= \omega_2 \omega_3 (I_2 - I_3), \\
        I_2 \dot{\omega}_2 &= \omega_3 \omega_1 (I_3 - I_1), \\
        I_3 \dot{\omega}_2 &= \omega_1 \omega_2 (I_1 - I_2)
    \end{bigcase}
\end{equation}


\part{单粒子量子力学}

\chapter{量子化的光场}

\section{线性介质中的光场量子化}

电磁场足够强以至于难以看到单光子效应,而又足够弱以至于能量不至于强到需要考虑量子电动力学的圈图修正,这样就可以使用经典电动力学描述整个系统。
为了讨论电磁波的量子涨落(在分析诸如腔内辐射场,或是非线性光学中的DFG过程时非常重要,在做高精度测量时有时也要考虑),即使没有圈图效应,我们也要做光场的量子化。
这个做法的必要性将在后续的章节中多次体现出来,我们这里只是讨论量子化技术本身,暂时不考虑光的量子性在哪些情境下最为明显。

\subsection{真空}\label{sec:quantization-in-vacuum}

我们首先考虑真空中的光场的量子化,此时我们无非是在重复QED中的运算,实际上是在重复无质量矢量场的量子化(见\qftdoc中的\ref{qft-sec:massless-vector-quantize}节)。
QED中矢量场展开为
\begin{equation}
    A_\mu(\vb*{x}, t) = (\frac{\varphi}{c}, - \vb*{A}) = \int \frac{\dd[3]{\vb*{k}}}{(2\pi)^3} \sqrt{\frac{\hbar}{2\omega_{\vb*{k}} \epsilon_0}} \sum_{\sigma=1}^2 \left( a_{\vb*{k} \sigma} \epsilon_\mu^\sigma(\vb*{k}) \ee^{\ii \vb*{k} \cdot \vb*{x} - \ii \omega_{\vb*{k}} t} + a_{\vb*{k} \sigma}^\dagger \epsilon_\mu^\sigma(\vb*{k})^* \ee^{- \ii \vb*{k} \cdot \vb*{x} + \ii \omega_{\vb*{k}} t} \right),
    \label{eq:vector-field-components}
\end{equation}
其中电磁场模式为平面波。
取费曼规范,做一些分部积分并去掉表面项,得到
\begin{equation}
    \mathcal{L} = - \frac{1}{2 \mu_0} \partial_\mu A_\nu \partial^\mu A^\nu,
\end{equation}
从而正则动量为
\begin{equation}
    \pi^\mu = \pdv{\mathcal{L}}{\partial_0 A_\mu} = - \partial^0 A^\mu,
\end{equation}
可以据此写出正则量子化条件,即时间相同时,$A^\mu$同$A^\nu$对易,而
\begin{equation}
    [A^\mu(\vb*{x}, t), \pi^\mu(\vb*{y}, t)] = \ii \eta^{\mu \nu} \delta^{(3)}(\vb*{x} - \vb*{y}).
\end{equation}
哈密顿量为
\[
    \begin{aligned}
        H &= \int \dd[3]{\vb*{r}} (\pi^\mu \partial_0 A_\mu - \mathcal{L}) \\
        &= \int \dd[3]{\vb*{r}} \left( - \frac{1}{c^2} (\partial_t A^\mu)^2 + \frac{1}{2} \partial_\mu A_\nu \partial^\mu A^\nu \right),
    \end{aligned}
\]
这里要注意$x^0 = c t$。代入$A_\mu$的展开式计算得到
\begin{equation}
    H = \sum_{\sigma=1}^2 \int \frac{\dd[2]{\vb*{k}}}{(2\pi)^3} \hbar \omega_{\vb*{k}} \left( a^\dagger_{\vb*{k} \sigma} a_{\vb*{k} \sigma} + \frac{1}{2} \right), \quad \omega_{\vb*{k}} = c \abs*{\vb*{k}}.
\end{equation}
这就得到了量子化的能量。
在量子化过程中我们已经通过限制$\sigma$而施加了规范,不过这个规范并不是辐射规范,而是洛伦兹规范。横波条件通过$\epsilon$矢量和波矢垂直而保证。
不过,既然我们只关心电偶极辐射而有关的相互作用哈密顿量可以完全写成$\vb*{E}$,这也不重要。

从四维矢量计算电场,得到
\begin{equation}
    \vb*{E}(\vb*{r}, t) = \int \frac{\dd[3]{\vb*{k}}}{(2\pi)^3} \sqrt{\frac{\hbar}{2\omega_{\vb*{k}} \epsilon_0}} \sum_{\sigma=1}^2 \left( (- \ii \vb*{k} \epsilon_0^\sigma(\vb*{k}) + \ii \omega_{\vb*{k}} \vb*{\epsilon}^\sigma(\vb*{k})) a_{\vb*{k} \sigma} \ee^{\ii \vb*{k} \cdot \vb*{r} - \ii \omega_{\vb*{k}} t} + \text{h.c.} \right),
\end{equation}
以及
\begin{equation}
    \vb*{B}(\vb*{r}, t) = \int \frac{\dd[3]{\vb*{k}}}{(2\pi)^3} \sqrt{\frac{\hbar}{2\omega_{\vb*{k}} \epsilon_0}} \sum_{\sigma=1}^2 \left( \ii \vb*{k} \times \vb*{\epsilon}_\sigma a_{\vb*{k} \sigma} \ee^{\ii \vb*{k} \cdot \vb*{r} - \ii \omega_{\vb*{k}} t } + \text{h.c.} \right).
\end{equation}
其中
\begin{equation}
    \epsilon^\mu_\sigma = (\epsilon_\sigma^0, \vb*{\epsilon}_\sigma), \quad \frac{\omega_{\vb*{k}}}{c} \epsilon_0^\sigma - \vb*{k} \cdot \vb*{\epsilon}_\sigma = 0, \quad \abs*{\epsilon_\sigma^0}^2 - \abs*{\vb*{\epsilon}_\sigma}^2 = 1.
\end{equation}
通过以上公式,能够验证以下哈密顿量形式:
\begin{equation}
    H = \int \dd[3]{\vb*{r}} \left( \frac{\epsilon_0}{2} \vb*{E}^2 + \frac{1}{2\mu_0} \vb*{B}^2 \right).
    \label{eq:e-and-b-hamiltonian}
\end{equation}
这正是电动力学中常见的形式。因此实际上我们也可以直接用\eqref{eq:vector-field-components}写出$\vb*{E}$和$\vb*{B}$并代入\eqref{eq:e-and-b-hamiltonian}。

在本文讨论的光学问题中,我们可以使用一种对具体计算更加友好的形式,即采用\concept{辐射规范}。
在辐射规范之下,我们有
\begin{equation}
    \begin{aligned}
        \mathcal{L} &= - \frac{1}{4 \mu_0} (\partial_\mu A_\nu - \partial_\nu A_\mu) (\partial^\mu A^\nu - \partial^\nu A^\mu) \\
        &= \frac{1}{2 \mu_0} \frac{1}{c^2} (\partial_t \vb*{A})^2 - \frac{1}{4\mu_0} (\partial_i A_j - \partial_j A_i) (\partial^i A^j - \partial^j A^i) \\
        &= \frac{\epsilon_0}{2} ((\dot{\vb*{A}})^2 - c^2 (\curl{\vb*{A}})^2).
    \end{aligned}
\end{equation}
以这个拉氏量为出发点做正则量子化。做展开
\begin{equation}
    \vb*{A}(\vb*{r}, t) = \int \frac{\dd[3]{\vb*{k}}}{(2\pi)^3} \sqrt{\frac{\hbar}{2 \omega_{\vb*{k}} \epsilon_0}} \sum_{\sigma=1}^2 (a_{\vb*{k} \sigma} \vu*{e}^\sigma \ee^{\ii \vb*{k} \cdot \vb*{r} - \ii \omega_{\vb*{k}} t} + a^\dagger_{\vb*{k} \sigma} (\vu*{e}^\sigma)^* \ee^{- \ii \vb*{k} \cdot \vb*{r} + \ii \omega_{\vb*{k}} t}),
\end{equation}
从而电场和磁场分别为
\begin{equation}
    \vb*{E}(\vb*{r}, t) = \ii \int \frac{\dd[3]{\vb*{k}}}{(2\pi)^3} \sqrt{\frac{\hbar \omega_{\vb*{k}}}{2 \epsilon_0}} \sum_{\sigma=1}^2 (a_{\vb*{k} \sigma} \vu*{e}^\sigma \ee^{\ii \vb*{k} \cdot \vb*{r} - \ii \omega_{\vb*{k}} t} - a^\dagger_{\vb*{k} \sigma} (\vu*{e}^\sigma)^* \ee^{- \ii \vb*{k} \cdot \vb*{r} + \ii \omega_{\vb*{k}} t})
    \label{eq:vacuum-e-field}
\end{equation}
和
\begin{equation}
    \vb*{B}(\vb*{r}, t) = \ii \int \frac{\dd[3]{\vb*{k}}}{(2\pi)^3} \sqrt{\frac{\hbar}{2 \omega_{\vb*{k}} \epsilon_0}} \sum_{\sigma=1}^2 (a_{\vb*{k} \sigma} \vb*{k} \times \vu*{e}_\sigma \ee^{\ii \vb*{k} \cdot \vb*{r} - \ii \omega_{\vb*{k}} t} - a^\dagger_{\vb*{k} \sigma} \vb*{k} \times \vu*{e}_\sigma^* \ee^{- \ii \vb*{k} \cdot \vb*{r} + \ii \omega_{\vb*{k}} t}).
    \label{eq:vacuum-b-field}
\end{equation}
正则动量为
\begin{equation}
    \vb*{\pi} = \epsilon_0 \dot{\vb*{A}},
\end{equation}
施加正则对易关系,会得到正确的
\begin{equation}
    \comm*{a_{\vb*{k} \sigma}}{a_{\vb*{k}' \sigma'}} = (2\pi)^3 \delta(\vb*{k} - \vb*{k}') \delta_{\sigma \sigma'},
\end{equation}
而哈密顿量为
\begin{equation}
    \begin{aligned}
        H &= \int \dd[3]{\vb*{r}} \left(\vb*{\pi} \cdot \pdv{\vb*{A}}{t} - \mathcal{L} \right) = \int \dd[3]{\vb*{r}} \left( \frac{\epsilon_0}{2} \left(\pdv{\vb*{A}}{t}\right)^2 + \frac{\epsilon_0}{2} c^2 (\curl{\vb*{A}})^2 \right) \\
        &= \int \dd[3]{\vb*{r}} \left( \frac{\epsilon_0}{2} \vb*{E}^2 + \frac{1}{2\mu_0} \vb*{B}^2 \right) \\
        &= \sum_{\sigma=1}^2 \int \frac{\dd[3]{\vb*{k}}}{(2\pi)^3} \hbar \omega_{\vb*{k}} \left(a^\dagger_{\vb*{k} \sigma} a_{\vb*{k} \sigma} + \frac{1}{2} \right).
    \end{aligned}
\end{equation}
因此,辐射规范给出的结果和完整的QED计算是完全一致的。
在辐射规范中我们还可以证明一个在横场条件成立时也成立,并且在一般的QED中很难计算的公式:
\begin{equation}
    \comm*{E^i(\vb*{r}, t)}{B^j(\vb*{r}', t)} = - \frac{\ii \hbar}{\epsilon_0} \pdv{x^k} \delta(\vb*{r} - \vb*{r}')
\end{equation}
其中$i, j, k$是$x, y, z$的轮换排列;其它情况下对易子为零。
还能够发现电场和自己的对易子始终为零,磁场亦然。因此电场的三个分量可以同时确定地被测量,磁场亦然。
但是不能同时准确测出$\vb*{E}$和$\vb*{B}$。
由于$(\vb*{E}, \vb*{B})$,$\vb*{A}$和$a_{\vb*{k} \sigma}$之间的关系是线性的,$a$的产生湮灭算符对易关系、$\vb*{A}$和$\vb*{\pi}$的正则对易关系以及$\vb*{E}$和$\vb*{B}$的对易关系是彼此等价的。

\begin{back}{粒子数表象和升降算符}{ladder-operator-particle-number}
    如果
    \begin{equation}
        n = a^\dagger a,
    \end{equation}
    且
    \begin{equation}
        \comm*{a}{a^\dagger} = 1,
    \end{equation}
    那么
    \begin{equation}
        a = \sum_{n} \sqrt{n} \ket*{n-1} \bra{n}, \quad a^\dagger = \sum_{n} \sqrt{n} \ket*{n} \bra*{n-1},
    \end{equation}
    或者说
\end{back}

\subsection{长波光子和介质中的麦克斯韦方程}\label{sec:long-wavelength-photon-maxwell-general}

下面我们讨论和\eqref{eq:material-hamiltonian}匹配的对易关系,以及它对角化之后将给出什么样的能谱。
应当指出,此时真空中的那些对易关系——$\vb*{A}$和$\epsilon_0 \dot{\vb*{A}}$之间的对易关系,$\vb*{E}$和$\vb*{B}$之间的对易关系——可能不能够直接适用。
这是因为正则量子化中,积掉自由度会导致哈密顿量的本征态的意义发生变化,从而算符的意义会发生变化。在高能物理中这导致场强重整化,在本文讨论的量子光学中则还会让对易关系发生变化。
同样这也会让横波条件的形式发生变化——介质中横波条件是$\div{\vb*{\epsilon} \cdot \vb*{E}} = 0$。

我们需要直接从\eqref{eq:material-hamiltonian}计算正则动量。同样取辐射规范,以$\vb*{A}$为基本自由度,则\eqref{eq:material-hamiltonian}就是
\begin{equation}
    H = \int \dd[3]{\vb*{r}} \left( \frac{1}{2} \dot{\vb*{A}} \cdot \vb*{\epsilon} \cdot \dot{\vb*{A}} + \frac{1}{2} (\curl{\vb*{A}}) \cdot (\vb*{\mu}^{-1}) \cdot (\curl{\vb*{A}}) \right),
\end{equation}
于是正则动量为
\begin{equation}
    \vb*{\pi} = \vb*{\epsilon} \cdot \dot{\vb*{\vb*{A}}}.
\end{equation}
我们现在需要展开$\vb*{A}$。此时空间平移不变性不能保持,我们不能使用动量来标记电场的振动模式,
我们将\eqref{eq:photon-in-material}右边的$\vb*{j}$取为零——我们此处在对线性介质做正则量子化,暂时不考虑电流——那就得到了一个广义本征值问题。
这就意味着,我们可以求解出一整套本征函数,它们由下式
\begin{equation}
    \curl{(\mu^{-1} \cdot \curl{\vb*{u}_n})} - \omega_n^2 \epsilon \cdot \vb*{u}_n = 0
\end{equation}
确定,其中$\omega_n$对应着能够在系统中稳定传播的电磁波模式的频率,且有正交归一关系
\begin{equation}
    \int_V \dd[3]{\vb*{r}} \vb*{u}^*_m \cdot \vb*{\epsilon} \cdot \vb*{u}_n = \delta_{mn}, 
\end{equation}
请注意由于$\vb*{A}$的厄米性,$\vb*{u}_m^*$一般是另一个$\vb*{u}_n$。
正交归一关系又意味着
\begin{equation}
    \omega_n^2 \delta_{mn} = \int \dd[3]{\vb*{r}} (\curl{\vb*{u}_m^*}) \cdot (\vb*{\mu}^{-1}) \cdot (\curl{\vb*{u}_n}) + \int \dd{\vb*{S}} \cdot (\vb*{u}_m^* \times ((\vb*{\mu}^{-1}) \cdot (\curl{\vb*{u}_n}))),
\end{equation}
在自由空间中等式右边第二项可以略去,在一个反射性能尚可的反射腔体(如果我们只讨论有限空间中的问题,那么基本上这个问题需要放在一个腔体中,否则无法忽视外界影响)中可以把第二项当成微扰。
本节仅仅给出最为简单的理论,暂时不考虑第二项。
用这组基底$\{\vb*{u}_n\}$做展开
\begin{equation}
    \vb*{A}(\vb*{r}, t) = \sum_n \ii \sqrt{\frac{\hbar}{2\omega_{n}}} \vb*{u}_n(\vb*{r}) a_n \ee^{- \ii \omega_n t} + \text{h.c.},
\end{equation}
得到
\begin{equation}
    - \vb*{E} = \dot{\vb*{A}} = \sum_n \sqrt{\frac{\hbar \omega_n}{2}} \vb*{u}_n(\vb*{r}) a_n \ee^{-\ii \omega_n t} + \text{h.c.},
\end{equation}
以及
\begin{equation}
    \vb*{B} = \curl{\vb*{A}} = \sum_n \ii \sqrt{\frac{\hbar}{2\omega_n}} \curl{\vb*{u}_n(\vb*{r})} a_n \ee^{-\ii \omega_n t} + \text{h.c.}.
\end{equation}
施加正则对易关系
\begin{equation}
    \comm*{A^i(\vb*{r}, t)}{\pi^j(\vb*{r}', t)} = \ii \hbar \delta(\vb*{r} - \vb*{r}') \delta^{ij},
\end{equation}
我们发现我们能够得到我们想要的产生湮灭算符对易关系
\begin{equation}
    \comm*{a_n}{a_m^\dagger} = \delta_{mn}.
\end{equation}
然后,可以计算出哈密顿量为
\begin{equation}
    H = \sum_n \hbar \omega_n \left( a^\dagger_n a_n + \frac{1}{2} \right).
\end{equation}
这个哈密顿量的形式和真空中完全一样,不同的地方在于$\omega_{\vb*{k}}$被$\omega_n$取代,色散关系可能变得非常不一样。

既然$\epsilon$和$\mu$的概念对长波光子在量子情况下仍然适用,反射、折射等概念对长波光子仍然有意义,且和经典情况非常类似。
特别的,光场可能被约束在一个四面都是反射镜的腔体中,此时的光场被所谓的\concept{cavity QED}或者简写为\concept{cQED}描述。

一个介质系统中的量子化光场的自由哈密顿量就是普通的谐振子哈密顿量。
使用本质上是经典的方程\eqref{eq:photon-in-material},得到一系列振动模式,其频率即为这个介质系统中的量子化光场中的模式的频率,振动模式的场强分布就是\eqref{eq:photon-in-material}给出的本征模式。

\subsection{归一化和单光子电场}

现在设想我们在一个有限大小的空间中做光场量子化,不过该空间中还是能够良定义波矢。
这样我们只需要在\autoref{sec:quantization-in-vacuum}中做代换
\[
    \int \frac{\dd[3]{\vb*{k}}}{(2\pi)^3} \longrightarrow \frac{1}{V} \sum_{\vb*{k}},
\]
于是电场和磁场分别为
\begin{equation}
    \vb*{E}(\vb*{r}, t) = \frac{\ii}{V} \sum_{\vb*{k}} \sqrt{\frac{\hbar \omega_{\vb*{k}}}{2 \epsilon_0}} \sum_{\sigma=1}^2 (a_{\vb*{k} \sigma} \vu*{e}^\sigma \ee^{\ii \vb*{k} \cdot \vb*{r} - \ii \omega_{\vb*{k}} t} - a^\dagger_{\vb*{k} \sigma} (\vu*{e}^\sigma)^* \ee^{- \ii \vb*{k} \cdot \vb*{r} + \ii \omega_{\vb*{k}} t})
    \label{eq:vacuum-e-field-cavity-origin}
\end{equation}
和
\begin{equation}
    \vb*{B}(\vb*{r}, t) = \frac{\ii}{V} \sum_{\vb*{k}} \sqrt{\frac{\hbar}{2 \omega_{\vb*{k}} \epsilon_0}} \sum_{\sigma=1}^2 (a_{\vb*{k} \sigma} \vb*{k} \times \vu*{e}_\sigma \ee^{\ii \vb*{k} \cdot \vb*{r} - \ii \omega_{\vb*{k}} t} - a^\dagger_{\vb*{k} \sigma} \vb*{k} \times \vu*{e}_\sigma^* \ee^{- \ii \vb*{k} \cdot \vb*{r} + \ii \omega_{\vb*{k}} t}).
    \label{eq:vacuum-b-field-cavity-origin}
\end{equation}
不过我们注意到,此时哈密顿量为
\[
    H = \frac{1}{V} \sum_{\vb*{k}, \sigma} \hbar \omega_{\vb*{k}} \left( a^\dagger_{\vb*{k} \sigma} a_{\vb*{k} \sigma} + \frac{1}{2} \right),
\]
多出来一个不太美观的因子$V$;此外对易关系也是
\[
    \comm*{a_{\vb*{k} \sigma}}{a_{\vb*{k}' \sigma'}^\dagger} = (2\pi)^3 \delta(\vb*{k} - \vb*{k}') \delta_{\sigma \sigma'} \longrightarrow \frac{1}{V} \delta_{\vb*{k} \vb*{k}'} \delta_{\sigma \sigma'}.
\]
为此我们如下重新定义产生湮灭算符:
\begin{equation}
    a_{\vb*{k} \sigma} \longrightarrow \sqrt{V} a_{\vb*{k} \sigma},
\end{equation}
于是电场和磁场分别为
\begin{equation}
    \vb*{E}(\vb*{r}, t) = \ii \sum_{\vb*{k}} \sqrt{\frac{\hbar \omega_{\vb*{k}}}{2 \epsilon_0 V}} \sum_{\sigma=1}^2 (a_{\vb*{k} \sigma} \vu*{e}^\sigma \ee^{\ii \vb*{k} \cdot \vb*{r} - \ii \omega_{\vb*{k}} t} - a^\dagger_{\vb*{k} \sigma} (\vu*{e}^\sigma)^* \ee^{- \ii \vb*{k} \cdot \vb*{r} + \ii \omega_{\vb*{k}} t})
    \label{eq:vacuum-e-field-cavity-1}
\end{equation}
和
\begin{equation}
    \vb*{B}(\vb*{r}, t) = \ii \sum_{\vb*{k}} \sqrt{\frac{\hbar}{2 \omega_{\vb*{k}} \epsilon_0 V}} \sum_{\sigma=1}^2 (a_{\vb*{k} \sigma} \vb*{k} \times \vu*{e}_\sigma \ee^{\ii \vb*{k} \cdot \vb*{r} - \ii \omega_{\vb*{k}} t} - a^\dagger_{\vb*{k} \sigma} \vb*{k} \times \vu*{e}_\sigma^* \ee^{- \ii \vb*{k} \cdot \vb*{r} + \ii \omega_{\vb*{k}} t}),
    \label{eq:vacuum-b-field-cavity-1}
\end{equation}
哈密顿量为
\begin{equation}
    H = \sum_{\vb*{k}, \sigma} \hbar \omega_{\vb*{k}} \left( a^\dagger_{\vb*{k} \sigma} a_{\vb*{k} \sigma} + \frac{1}{2} \right),
\end{equation}
对易关系为
\begin{equation}
    \comm*{a_{\vb*{k} \sigma}}{a_{\vb*{k}' \sigma'}^\dagger} = \delta_{\vb*{k} \vb*{k}'} \delta_{\sigma \sigma'}.
\end{equation}
我们定义
\begin{equation}
    \mathcal{E}_{\vb*{k} \sigma} = \sqrt{\frac{\hbar \omega_{\vb*{k}}}{2 \epsilon_0 V}},
\end{equation}
则
\begin{equation}
    \vb*{E} = \sum_{\vb*{k}, \sigma} \mathcal{E}_{\vb*{k} \sigma} \vb*{f}_{\vb*{k} \sigma} a_{\vb*{k} \sigma} \ee^{- \ii \omega_{\vb*{k}} t} + \text{h.c.},
\end{equation}
其中
\begin{equation}
    \vb*{f}_{\vb*{k} \sigma} = \ii \ee^{\ii \vb*{k} \cdot \vb*{r}} \vu*{e}^\sigma,
\end{equation}
满足
\begin{equation}
    \frac{1}{V} \int \dd[3]{\vb*{r}} \vb*{f}^*_{\vb*{k} \sigma} \cdot \vb*{f}_{\vb*{k}' \sigma'} = \delta_{\vb*{k} \vb*{k}'} \delta_{\sigma \sigma'}.
\end{equation}
我们称$\mathcal{E}_{\vb*{k} \sigma}$为\concept{单光子电场},因为它给出了增多一个电子,电场大体上增多的幅度,而$\vb*{f}$则称为\concept{模式函数},它给出了光场的稳定振动方式,且不显含体积(体积出现在了积分前的归一化常数中)。

对一个一般的体系,我们会定义
\begin{equation}
    \mathcal{E}_{n} = \sqrt{\frac{\hbar \omega_n}{2 \epsilon_0 V}},
\end{equation}
从而电场为
\begin{equation}
    \vb*{E} = \sum_n \mathcal{E}_n \vb*{f}_{n} a_n \ee^{- \ii \omega_n t} + \text{h.c.},
    \label{eq:general-optical-field}
\end{equation}
其中
\begin{equation}
    \vb*{f}_n = \sqrt{\epsilon_0 V} \vb*{u}_n,
\end{equation}
且
\begin{equation}
    \frac{1}{V} \int \dd[3]{\vb*{r}} \vb*{f}_m^* \cdot \frac{\vb*{\epsilon}}{\epsilon_0} \cdot \vb*{f}_n = \delta_{mn}.
\end{equation}

实际上从这里我们可以看出,经典的麦克斯韦方程本身已经是一个具有一定量子特性的理论了——“单光子波函数”(虽然没有良定义的单光子量子力学,但是我们不妨这么指代$\mel*{0}{A^i(\vb*{r})}{\psi}$)服从的方程就是麦克斯韦方程。
也可以从另一个角度看这件事:在麦克斯韦方程两边乘上$\hbar$,由于$E \sim \hbar \omega$,得到的理论看上去就是一个量子理论。
将光场量子化引入的新物理只有两件事:光束由分立的光子构成,以及存在光子数的量子涨落,但是,在没有非线性光学效应的情况下,光子数目守恒,第一件事完全可以通过手动引入“光子”的概念并指派其波函数为(经过适当归一化的)经典电磁场来做到。
光的量子性只有在下面的地方才会变得重要:
\begin{itemize}
    \item 光子生灭明显,一些光子模式上原本没有电子而一段时间后有光子产生时,即处理非线性光学时,因为此时会有一些原本完全没有光子分布的模式上出现了光子。经典处理只能在有种子光的时候处理光子的产生——并不奇怪,因为光子从零到一产生的过程涉及一个极为弱的场强,弱到经典场论不再使用。
    \item 纠缠重要时。经典电动力学面对“光子增多了”的描述方法是更大的场强,而没有直积的希尔伯特空间这样的概念,从而无法捕捉到纠缠。
\end{itemize}
可以看到这些光的量子性变得明显的情况都涉及多光子Fock态。
我们其实可以在这里看到一个相当有趣的情况:当所研究的问题中涉及非常弱的光场(如特定频率的光子一开始没有,但是一段时间后被产生)时,经典电动力学就失效了,然而经典电动力学的形式却又很像是在处理“单光子波函数”。
当然,这两者并没有矛盾,本质上是因为经典电动力学无法正确处理“多光子形成的多体波函数”:“单光子波函数”不涉及多体波函数,它给出的所有物理就是一个麦克斯韦方程,正好和经典电动力学一致;
相干态下电场的标准差相比于电场期望值本身比较小,从而电场能够被看成经典的量,同样可以被经典电动力学处理。
经典电动力学无法正确处理“多光子形成的多体波函数”,因为没有Fock空间;但是这并不是说经典电动力学就缺乏(相比于经典质点动力学的)一切量子性,如坐标和动量的不确定性等。
实际上,对缺乏纠缠、缺乏粒子生灭和碰撞、粒子数大的有质量粒子系统,单粒子波函数乘以适当的因子也可以诠释为“粒子数的平方根”。在这个意义上它和电磁场的地位是类似的。
当然,有质量粒子系统中有大量的碰撞,其宏观理论通常是动理学方程,且“经典费米场”不是一个物理意义特别明确的东西,因此我们很少看到“电子的宏观场”。
然而在电动力学以外的系统中,由于系统非常接近相干态,量子场能够被看成经典场的例子也是有的,如超流和超导中的序参量。

从本节的计算中也可以看到为什么很多时候经典电动力学已经够用了,因为一般的偶极辐射产生的就是相干光。要产生明显偏离相干光的辐射场实际上是很不容易的。

在本文中“电磁场”可能代表量子化的场算符,也可能代表经典电磁场,也可能代表“单光子波函数”。在不考虑非线性效应时这三者的时间演化是相同的。
在$\vb*{E}$被认为是经典场时,$\vb*{E}^2$——从而$I$——在形式上对应于“光子出现的概率”。
光场中被传输的不是$I$,电磁场的相位信息是很重要的,正如量子力学中叠加的不是概率而是概率振幅一样。
对非相干光(后文将讨论),可以直接将$I$相加,正如高度混合态的系统可以直接使用经典概率论处理一样。

\section{相干态和Wigner波函数}

如果我们直接计算电场在光子的Fock态下的期望值,无疑会得到零,因为$\vb*{E}$算符正比于单个产生算符和湮灭算符之和,从而,设$\ket*{\psi}$是一个光子Fock态,$\mel{\psi}{E_n}{\psi}$中,$E_n$作用到右边后模式$n$上的光子数目发生变化,因此电场期望值为零。

这看起来真是匪夷所思,因为这似乎说明Fock态中没有光子,没有能量,而这当然不正确。
例如,计算电场平方的期望值却又会得到非零结果。电场平方正比于该点的“电场能”,因此Fock态是有能量的。

问题的核心在于Fock态\emph{不是}电场的本征态。我们现在要找到一个和经典的电场接近的量子态,并且考虑如何用一种经典意义明显的方式表示一个任意的量子光学中的光场波函数。

\subsection{相干态}

\begin{back}{相干态和相干态路径积分}{coherent-state}
    设动力学变量$x$和它的正则动量满足正则对易关系
    \begin{equation}
        \comm{x}{p} = \ii,
    \end{equation}
    引入产生湮灭算符
    \begin{equation}
        a = \frac{1}{\sqrt{2}} (x + \ii p), \quad a^\dagger = \frac{1}{\sqrt{2}} (x - \ii p),
    \end{equation}
    则有正确的对易关系
    \begin{equation}
        \comm{a}{a^\dagger} = 1.
    \end{equation}
    我们现在考虑怎么将一个密度矩阵写成某种“准概率分布”的形式,即能够将它和一个函数$W(x, p)$或者$W(a, a^*)$建立线性关系。

    考虑用复参数定义的相干态
    \begin{equation}
        \ket*{\alpha} = \ee^{- \abs*{\alpha}^2 / 2} \sum_{n = 0}^\infty \frac{\alpha^n}{\sqrt{n!}} \ket*{0},
    \end{equation}
    其中$\ket*{n}$是谐振子哈密顿量
    \begin{equation}
        H = \sum_{n \geq 0} \hbar \omega \left( a^\dagger a + \frac{1}{2} \right) 
    \end{equation}
    的第$n$激发态。
    能够证明完备条件
    \begin{equation}
        \frac{1}{\pi} \int \dd[2]{\alpha} \dyad{\alpha} = 1
    \end{equation}
    成立,这里积分测度为
    \begin{equation}
        \dd[2]{\alpha} = \frac{\dd{\alpha^*} \wedge \dd{\alpha}}{2 \ii} = R \dd{R} \wedge \dd{\theta},
    \end{equation}
    其中$R$是$\alpha$的长度而$\theta$是相角,$\dd[2]{\alpha}$是复平面上的积分测度。
    由于不同$\alpha$的相干态彼此不正交——实际上,我们有
    \begin{equation}
        \braket*{\alpha}{\beta} = \ee^{- (\abs{\alpha}^2 + \abs{\beta}^2) / 2 + \alpha^* \beta},
    \end{equation}
    全体相干态实际上是超完备的。

\end{back}

我们在构造路径积分时遇到过形式最为一般的相干态。在量子光学中相干态还有特殊的意义。

设想空间中有一个振动频率给定的电偶极子
\begin{equation}
    \vb*{d}(t) = \vb*{d} \ee^{- \ii \omega t } + \text{h.c.},
\end{equation}
我们考虑空间中的光场的演化情况。设光场波函数为$\ket*{\psi}$,取偶极辐射近似。
在相互作用绘景中有
\begin{equation}
    \ii \pdv{t} \ket*{\psi} = - \vb*{d}(t) \cdot \vb*{E} \ket*{\psi}.
\end{equation}
假定$t=0$时没有任何光子,因此,等价地我们可以认为从$t=0$开始电偶极子才开始振动,在此之前系统中一直没有任何光子。
我们有形式解
\[
    \begin{aligned}
        \ket*{\psi(t)} &= \exp(\frac{\ii}{\hbar} \int_0^t \dd{t'} \sum_n (\vb*{d} \ee^{- \ii \omega t } + \text{h.c.}) \cdot (\mathcal{E}_n \vb*{f}_n a_n \ee^{- \ii \omega_n t'} + \text{h.c.}) ) \ket*{0} \\
        &= \exp(\sum_n (\alpha_n a_n^\dagger - \alpha_n^* a_n) ) \ket*{0},
    \end{aligned}
\]
这里我们定义
\begin{equation}
    \alpha_n \coloneqq \frac{1}{\hbar} \int_0^t \dd{t'} \mathcal{E}_n \vb*{f}_n^* \ee^{-\ii \omega_n t'} \cdot (\vb*{d} \ee^{- \ii \omega t} + \text{h.c.}).
\end{equation}
由于
\[
    \comm*{a_n}{\comm*{a_n}{a^\dagger_n}} = \comm*{a^\dagger_n}{\comm*{a_n}{a^\dagger_n}} = 0,
\]
我们有
\[
    \begin{aligned}
        \exp(\alpha_n a_n^\dagger - \alpha_n^* a_n) &= \ee^{\alpha_n a_n^\dagger} \ee^{- \alpha_n^* a_n} \ee^{\frac{1}{2} \comm*{\alpha_n a_n^\dagger}{\alpha_n^* a_n}} \\
        &= \ee^{\alpha_n a_n^\dagger} \ee^{- \alpha_n^* a_n} \ee^{\frac{1}{2} \abs{\alpha_n}},
    \end{aligned}
\]
定义其为\concept{位移算符}
\begin{equation}
    D(\{\alpha_n\}) \coloneqq \prod_n \ee^{\alpha_n a_n^\dagger} \ee^{- \alpha_n^* a_n} \ee^{\frac{1}{2} \abs{\alpha_n}},
\end{equation}
它作用在真空态上给出
\begin{equation}
    \ket*{\{\alpha_n\}} \coloneqq D(\{\alpha_n\}) \ket*{0} = \exp(\sum_n (\alpha_n a_n^\dagger - \alpha_n^* a_n) ) \ket*{0} = \prod_n \ee^{- \frac{\abs{\alpha_n}^2}{2}} \ee^{\alpha_n a^\dagger_n} \ket*{0_n}.
\end{equation}
将上式中的$\ee^{\alpha_n a^\dagger_n}$展开,我们发现
\begin{equation}
    \ket*{\{\alpha_n\}} = \prod_n \ee^{- \frac{\abs{\alpha_n}^2}{2}} \sum_{i_n=0}^\infty \frac{\alpha_n^{i_n}}{\sqrt{i_n!}} \ket*{0_n},
\end{equation}
因此它的确就是路径积分量子化中定义的相干态。

现在我们回来看看$\{\alpha_n\}$到底是什么,或者说相干态到底是什么。
实际上在路径积分量子化中我们已经知道系统处在相干态上意味着系统大体上服从经典场论了,不过还是做一些具体计算来更清楚地展示这一点。

为了更加简便我们考虑一个\concept{单模光场},即只考虑\eqref{eq:general-optical-field}中的一个模式。
这是合理的,因为一个线性体系中的所有模式之间不存在任何关系。
对某个特定的模式对应的产生湮灭算符$(a, a^\dagger)$定义
\begin{equation}
    X_1 = \frac{1}{2} (a + a^\dagger), \quad X_2 = \frac{1}{2\ii} (a - a^\dagger),
\end{equation}
则
\begin{equation}
    \vb*{E} = 2 \mathcal{E} (\vb*{f} X_1 \cos(\omega t) + \vb*{f}^* X_2 \sin(\omega t)).
\end{equation}
经典电动力学中系统总是在相干态附近,将$a$替换成$\alpha$就能够从量子光学过渡到经典光学。
在做了这个过渡之后,$X_1$大体上是$\Re{\alpha}$而$X_2$大体上就是$\Im{\alpha}$。
于是我们就看到了相干态$\ket*{\alpha}$的经典意义:一个相干态$\ket*{\alpha}$对应一个正弦振动的经典电场,其幅度为
\begin{equation}
    A = 2 \mathcal{E} \abs{\vb*{f}} \sqrt{X_1^2 + X_2^2} = 2 \mathcal{E} \abs{\vb*{f}} \abs{\alpha},
\end{equation}
其相位则是
\begin{equation}
    \varphi = \arg \alpha = \arctan \frac{X_2}{X_1}.
\end{equation}

$X_1$和$X_2$当然是不对易的,因此即使是相干态下(当然也包括其它相干态下),实际上电场
\begin{equation}
    E = A \ee^{\ii \varphi} \ee^{- \omega t} + \text{h.c.}
\end{equation}
是不能够完全确定的。或者也可以说,$A$(实际上就是粒子数开平方)和$\varphi$是不能同时完全确定的。

相干态下的光子数服从泊松分布。
因此,相干态的平均值和经典电场类似,但是存在涨落,其涨落行为和真空类似。

\subsection{准概率分布函数}

\begin{back}{量子力学中的相空间和准概率分布函数}{phase-wigner}
    虽然在量子力学中坐标和动量不能同时确定,从而看起来似乎不能够有良定义的相空间,不过注意到,算符$O$的矩阵元一般来说形如$\mel{x}{O}{x'}$,即需要两个标签——$x$和$x'$——标记一个矩阵元,那么我们将$x$和$x'$线性组合一下,将其中一个坐标做傅里叶变换切换到动量空间,似乎还是能够写出$O(x, p)$这样的式子,从而有一个等效的相空间。
    当然,这样得到的“相空间中的分布函数”,即满足下述条件的函数$f(x, p)$:
    \[
        \expval*{O} = \int \dd{x} \dd{p} O(x, p) f(x, p)
    \]
    未必能够赋予概率的意义,因为它可以取负值甚至虚数值。

    我们现在考虑几种$f(x, p)$的定义。最为知名的可能是Wigner函数,它基本上是经典力学中的粒子分布函数的量子推广。
    设$\rho$是一个密度矩阵,\concept{Wigner函数}定义为
    \begin{equation}
        W(x, p) = \frac{1}{2\pi \hbar} \int \dd{y} \mel{x - \frac{y}{2}}{\rho}{x + \frac{y}{2}} \ee^{\ii p y / \hbar}.
    \end{equation}
    当然也可以用$\alpha$和$\alpha^*$——或者说,用二维相平面上的复数$\alpha$——做Wigner函数的宗量。
    将算符$O$写成关于$a$和$a^\dagger$的\emph{对称排序}$O_\text{S}$之后,我们有
    \begin{equation}
        \expval*{O} = \int \dd[2]{\alpha} W(\alpha, \alpha^*) O_\text{S}(\alpha, \alpha^*).
    \end{equation}
    例如,
    \begin{equation}
        \frac{1}{2} \expval*{a a^\dagger + a^\dagger a} = \int \dd[2]{\alpha} W(\alpha, \alpha^*) \alpha \alpha^*.
    \end{equation}
    
    Wigner分布函数可以验证是实数,不过有正有负。能够证明Wigner函数取负值的区域不会太大——在几个$\hbar$以内——这直观地展示了从量子过渡到经典的过程。
    与路径积分类似,Wigner函数也并非只能在坐标-动量构成的相空间中定义——任何有广义坐标和广义动量的能够用哈密顿动力学描述的系统中的密度矩阵都能够用Wigner函数等价地给出。
    因此,Wigner函数实际上可以构成量子力学的另一种形式理论:所谓量子力学,就是将普通的概率分布拓展为Wigner函数的物理理论。

    我们当然也可以寻找一种函数$P(\alpha, \alpha^*)$使得
    \begin{equation}
        \expval*{O} = \int \dd[2]{\alpha} P(\alpha, \alpha^*) O_\text{N}(\alpha, \alpha^*),
    \end{equation}
    这里$O_\text{N}$表示算符$O$的\emph{正则排序},即将湮灭算符排在右边,产生算符排在左边。
    我们有
    \[
        \begin{aligned}
            \expval*{O} &= \sum_{m, n} c_{mn} \trace((a^\dagger)^m a^n \rho) \\
            &= \sum_{m, n} c_{mn} \int \dd[2]{\alpha} \trace((\alpha^*)^m \alpha^n \rho \delta(\alpha^* - a^\dagger) \delta(\alpha - a)) \\
            &= \int \dd[2]{\alpha} \sum_{m,n} c_{mn} (\alpha^*)^m \alpha^n \trace(\rho \delta(\alpha^* - a^\dagger) \delta(\alpha - a)),
        \end{aligned}
    \]
    上式最后一行的形式正是$O_\text{N}$乘以某个分布函数,于是我们得到
    \begin{equation}
        P(\alpha, \alpha^*) = \trace(\rho \delta(\alpha^* - a^\dagger) \delta(\alpha - a)).
    \end{equation}
    简单地验证会发现
    \begin{equation}
        \rho=\int \dd^{2} \alpha P\left(\alpha, \alpha^{*}\right)|\alpha\rangle\langle\alpha| .
    \end{equation}

    我们当然还可以让寻找一种函数$Q(\alpha, \alpha^*)$使得
    \begin{equation}
        \expval*{O} = \int \dd[2]{\alpha} Q(\alpha, \alpha^*) O_\text{A}(\alpha, \alpha^*),
    \end{equation}
    这里$O_\text{A}$表示算符$O$的\emph{反正则排序},即湮灭算符排在左边而产生算符排在右边。和$P$函数一样如法炮制,能够得到
    \begin{equation}
        Q\left(\alpha, \alpha^{*}\right)=\operatorname{tr}\left[\rho \delta(\alpha-a) \delta\left(\alpha^{*}-a^{\dagger}\right)\right].
    \end{equation}
    由于
    \[
        \begin{aligned}
            Q\left(\alpha, \alpha^{*}\right) &=\frac{1}{\pi} \operatorname{tr} \int d^{2} \alpha^{\prime}\left[\rho \delta(\alpha-a)\left|\alpha^{\prime}\right\rangle\left\langle\alpha^{\prime}\right| \delta\left(\alpha^{*}-a^{\dagger}\right)\right] \\
            &=\frac{1}{\pi} \operatorname{tr} \int d^{2} \alpha^{\prime}\left\{\rho \delta\left(\alpha-\alpha^{\prime}\right)\left|\alpha^{\prime}\right\rangle\left\langle\alpha^{\prime}\right| \delta\left[\alpha^{*}-\left(\alpha^{\prime}\right)^{*}\right]\right\} \\
            &=\frac{1}{\pi} \operatorname{tr}(\rho|\alpha\rangle\langle\alpha|) \\
            &=\frac{1}{\pi}\langle\alpha|\rho| \alpha\rangle,
            \end{aligned}
    \]
    我们有
    \begin{equation}
        Q(\alpha, \alpha^*) = \frac{1}{\pi} \mel{\alpha}{\rho}{\alpha},
    \end{equation}

    $P$函数,$W$函数,$Q$函数都能够写成某些算符的期望值的某种积分变换。
    我们有
    \begin{equation}
        \delta\left(\alpha^{*}-a^{\dagger}\right) \delta(\alpha-a) =\frac{1}{\pi^{2}} \int \dd^{2} \beta \exp \left[-\beta\left(\alpha^{*}-a^{\dagger}\right)\right] \exp \left[\beta^{*}(\alpha-a)\right],
    \end{equation}
    于是
    \begin{equation}
        P\left(\alpha, \alpha^{*}\right)=\frac{1}{\pi^{2}} \int \dd^{2} \beta \ee^{- \ii \beta \alpha^{*}- \ii \beta^{*} \alpha} C_\text{N}\left(\beta, \beta^{*}\right),
    \end{equation}
    其中
    \begin{equation}
        C_\text{N}\left(\beta, \beta^{*}\right)=\operatorname{tr}\left(\ee^{\ii \beta a^{\dagger}} \ee^{\ii \beta^{*} a} \rho\right).
    \end{equation}
    类似的
    \begin{equation}
        Q\left(\alpha, \alpha^{*}\right)=\frac{1}{\pi^{2}} \int \dd^{2} \beta \ee^{- \ii \beta \alpha^{*}- \ii \beta^{\cdot} \alpha} C_\text{A}\left(\beta, \beta^{*}\right),
    \end{equation}
    其中
    \begin{equation}
        C_\text{A}\left(\beta, \beta^{*}\right)=\operatorname{tr}\left(\ee^{\ii \beta^{*} a} \ee^{\ii \beta a^{\dagger}} \rho\right).
    \end{equation}
    对Wigner函数也有类似的公式:
    \begin{equation}
        W\left(\alpha, \alpha^{*}\right)=\frac{1}{\pi^{2}} \int \dd^{2} \beta \ee^{-\ii \beta \alpha^{*}- \ii \beta^{*} \alpha} C_\text{S}\left(\beta, \beta^{*}\right),
    \end{equation}
    其中
    \begin{equation}
        C_\text{S}\left(\beta, \beta^{*}\right)=\operatorname{tr}\left(\ee^{\ii \beta a^{\dagger}+ \ii \beta^{*} a} \rho\right).
    \end{equation}

    作为一个例子,相干态的三种准
    $Q$函数相比Wigner函数要宽一些;$P$函数是最窄的。
\end{back}

\section{光的探测}

实用的光探测器基本上都是根据偶极相互作用哈密顿量$- \vb*{d} \cdot \vb*{E}$工作的,即一个凝聚态系统——这里是光传感器——吸收了一个光子,将电子激发出来,然后探测由此产生的光电流。
由于光传感器中电子的能量通常很低,不会激发出多少光子,实验中置于光场中的探测器探测到光信号的概率几乎完全由“光场中一个光子湮灭,光场掉落到一个能量更低的状态”这样的过程贡献。
因此,能够对实验中探测器探测到光信号的概率有贡献的关联函数基本上都是正规序的,即它们均形如
\[
    \mel{\psi}{a^\dagger_{\vb*{q}_1} a^\dagger_{\vb*{q}_2} \cdots a_{\vb*{p}_2 a_{\vb*{p}_2}}}{\psi}.
\]
从而,% TODO
其中
\begin{equation}
    \vb*{E}^{+} (\vb*{r}, t) = \sum_n \mathcal{E}_n \vb*{f}_n(\vb*{r}) a_n \ee^{-\ii \omega_n t} , \quad \vb*{E}^{-} (\vb*{r}, t) = \sum_n \mathcal{E}_n \vb*{f}^*_n(\vb*{r}) a^\dagger_n \ee^{\ii \omega_n t}, \quad \vb*{E} = \vb*{E}^+ + \vb*{E}^-.
\end{equation}

\chapter{单光子现象}

\section{单光子器件}

分束器

\section{证明单光子量子性的实验}

\subsection{Aspect实验}

要证明单光子具有量子性,最好的办法是使用一些这样的实验:它的一种版本能够证明光子的粒子性,它的另一个仅仅做了少许修正的版本(比如说探测器被移动到别的位置)能够证明光子的波动性。
两个版本区别很小这件事能够排除实验装置和光的复杂相互作用显著地改变了光的行为这样的说法,而粒子性和波动性同时出现则强烈暗示需要量子理论描述光。
1986年的Aspect实验是这种实验的一个典范。

\subsection{Zeilinger实验}

A.Zeilinger
一种更加简明的实验是这样的:同样使用分束器和反射镜,构造这样的光路:

\subsection{如果狄拉克是错的……}

\subsubsection{两个激光器产生的光束的干涉}

两个激光器产生的光似乎是相干的?

实际上激光器产生的是相干态光而不是光子数确定的多光子玻色波函数。

\subsubsection{玻色-爱因斯坦凝聚态中的干涉}

\subsection{Hanbury Brown和Twiss效应}

\subsubsection{天狼星上的光的疑似量子性}

\subsubsection{经典电动力学解释}

\chapter{纠缠光}

\part{原子物理}

\documentclass[UTF8, a4paper]{ctexart}

\usepackage{geometry}
\usepackage{titling}
\usepackage{titlesec}
\usepackage{paralist}
\usepackage{footnote}
\usepackage{enumerate}
\usepackage{amsmath, amssymb, amsthm}
\usepackage{mathtools}
\usepackage{cite}
\usepackage{graphicx}
\usepackage{subfigure}
\usepackage{physics}
\usepackage{siunitx}
\usepackage[colorlinks, linkcolor=black, anchorcolor=black, citecolor=black]{hyperref}
\usepackage{prettyref}

\geometry{left=3.28cm,right=3.28cm,top=2.54cm,bottom=2.54cm}
\titlespacing{\paragraph}{0pt}{1pt}{10pt}[20pt]
\setlength{\droptitle}{-5em}
\preauthor{\vspace{-10pt}\begin{center}}
\postauthor{\par\end{center}}

\newcommand*{\ee}{\mathrm{e}}
\newcommand*{\ii}{\mathrm{i}}
\newcommand*{\st}{\quad \text{s.t.} \quad}
\newcommand*{\const}{\mathrm{const}}
\newcommand*{\natnums}{\mathbb{N}}
\newcommand*{\reals}{\mathbb{R}}
\newcommand*{\complexes}{\mathbb{C}}
\DeclareMathOperator{\timeorder}{T}
\newcommand*{\ogroup}[1]{\mathrm{O}(#1)}
\newcommand*{\sogroup}[1]{\mathrm{SO}(#1)}
\DeclareMathOperator{\legpoly}{P}
\DeclareMathOperator{\laguerre}{L}
\newcommand*{\lsterm}[3]{$^{#1}${#2}$_{#3}$}
\newcommand*{\nuclear}[3]{^{#2}_{#3}\text{#1}}

\newrefformat{sec}{第\ref{#1}节}
\newrefformat{note}{注\ref{#1}}
\renewcommand{\autoref}{\prettyref}
\newcommand*{\concept}[1]{\underline{\textbf{#1}}}

\title{原子物理}
\author{吴何友}

\begin{document}

\maketitle

\section{单电子量子力学}

\subsection{薛定谔方程}

实验表明,电子的自由度包括轨道自由度和自旋自由度,自旋$1/2$。
轨道部分在坐标表象下,坐标和动理动量分别是
\begin{equation}
    \hat{\vb*{r}} = \vb*{r}, \quad \hat{\vb*{p}} = -\ii \hbar \grad.
    \label{eq:single-electron-schodinger}
\end{equation}
轨道部分的动力学由单电子薛定谔方程描述。在没有磁场时,单电子的薛定谔方程是
\begin{equation}
    \ii \hbar \pdv{\psi}{t} = - \frac{\hbar^2 \laplacian}{2m} \psi + V \psi,
\end{equation}
从中可以得到概率密度为
\begin{equation}
    \rho = \abs*{\psi}^2,
\end{equation}
概率流密度为
\begin{equation}
    \vb*{j} = \frac{\ii \hbar}{2m} (\psi \grad{\psi^*} - \psi^* \grad{\psi}).
\end{equation}

\subsection{波列}

在$V$为常数时,\eqref{eq:single-electron-schodinger}的平面波解和色散关系如下:
\begin{equation}
    \psi = \ee^{\ii(\vb*{k} \cdot \vb*{r} - \omega t)}, \quad \hbar \omega = \frac{\hbar^2 k^2}{2m} + V.
    \label{eq:plain-wave-solution}
\end{equation}
可以看到,波矢$\vb*{k}$和动量$\vb*{p}$只差了一个因子$\hbar$。
这当然是正确的,实际上从一个局域近似为平面波的波形(如球面波、柱面波)中提取波矢的方式就是作用算符$-\grad$然后除以原波形。
\eqref{eq:plain-wave-solution}是一个定态解,其能量就是$\hbar \omega$。
\eqref{eq:plain-wave-solution}的波长称为\concept{德布罗意波长},为
\begin{equation}
    \lambda = \frac{2\pi}{k} = \frac{h}{p} = \frac{h}{\sqrt{2m(E-V)}}.
\end{equation}
在我们讨论的问题的尺度远大于德布罗意波长时,\eqref{eq:plain-wave-solution}的波动性实际上并不明显,此时可以使用经典理论处理问题,否则必须使用量子力学。

位置和动量的对易关系意味着在一次测量中不可能同时确定地得到它们的值,因为这两个算符没有共同本征态,从而
\begin{equation}
    \Delta x \Delta p_x \geq \frac{\hbar}{2}.
    \label{eq:position-momentum-uncertainty}
\end{equation}
虽然时间不是算符,但如果由于相互作用等原因,能级发生展宽,那么相应的,系统停留在这个能级上的时间也不会是无限大的,当然也不会是无限小的,由傅里叶变换关于函数定域性的理论可以推导出
\begin{equation}
    \Delta t \Delta E \geq \frac{\hbar}{2}.
    \label{eq:time-energy-uncertainty}
\end{equation}
实际上,也可以从傅里叶变换的观点推导出\eqref{eq:position-momentum-uncertainty},但是要注意它和\eqref{eq:time-energy-uncertainty}的物理意义是不同的,因为没有对应于时间的算符。

\subsection{常用常数}

普朗克常数
\[
    h = \SI{6.62607004e-34}{J.s}, \quad \hbar = \SI{1.05457266e-34}{J.s},
\]
元电荷
\[
    e = \SI{1.602176634e-19}{C},
\]
电子质量
\[
    m_\text{e} = \SI{9.10938356e-31}{kg},
\]
质子质量
\[
    m_\text{p} = \SI{1.6726219e-27}{kg},
\]
中子质量
\[
    m_\text{n} = \SI{1.674927471e-27}{kg}
\]

\subsection{斯特恩-盖拉赫实验}

考虑一束带有磁矩的粒子穿过一个基本上只在$z$方向有梯度的磁场(这样的一个装置称为一个\concept{SG仪器}),显然,由于磁矩的存在,在经典图景下会引入这样一个磁场力:
\begin{equation}
    \vb*{F} = \pdv{(\vb*{\mu} \cdot \vb*{B})}{z} \vb*{e}_z = \mu_z \pdv{B_z}{z} \vb*{e}_z.
\end{equation}
如果粒子束的生成装置当中不存在特殊的磁矩方向(通常会用一个炉子加热原子,然后产生一个射流,因此的确不存在特殊的磁矩方向),那么,预期的结果应该是经过磁场之后的粒子束弥散为一个连续的竖条。
但实际情况并非如此:束流劈裂成了两束,一束向上,一束向下,即似乎$z$方向磁矩只有两个离散的取值。
由于磁矩正比于自旋,显然$z$方向自旋也应该只有两个离散的取值。

现在只取一束束流,不妨认为其自旋向上,那么,将它通过一个有$z$方向磁场的SG仪器,束流按理来说不会进一步劈裂——实际上也的确如此。
但如果将它通过一个有$x$方向磁场的SG仪器,在经典图像下由于束流不存在$x$方向的磁矩,应该什么也不会发生,但实际上实验结果是,束流在$x$方向发生了劈裂。
显然,$z$方向自旋全部指向一个方向的束流的$x$方向自旋还是可以有两个取值。

也许有这样的可能,就是自旋在三个方向上都有两种取值,这样,$z$方向自旋全部指向一个方向的束流在$x$方向上的自旋可以同时有两个取值。
那么,将一个粒子束依次通过$z$方向的SG装置和$x$方向的SG装置应该能够得到$x$方向和$z$方向自旋都完全确定的束流。
实际情况并非如此:一个粒子束依次通过$z$方向的SG装置和$x$方向的SG装置之后得到的束流再次通过一个$z$方向的SG装置,照样出现了劈裂。

在量子力学的框架下解释这样的现象就非常容易。SG装置实际上造成了自旋和轨道的耦合,它是一个使用粒子的轨道去测量粒子自己的自旋的装置。
因此,当一个粒子经过一个某一方向SG装置后,它的自旋角动量塌缩到这一方向的自旋角动量分量的本征态上。
自旋算符的本征值是分立的,自然造成分立的测量结果,也就是粒子束流劈裂。
由于不同方向的自旋算符不对易,$z$方向的自旋算符的本征态不是$x$方向的自旋算符的本征态,从而造成之前看到的$z$方向自旋确定的束流经过$x$方向的SG装置之后再次发生劈裂的情况。
先经过$z$方向的SG装置再经过$x$方向的SG装置的束流的每个粒子都塌缩到了$x$方向自旋算符的本征态上,于是经过$z$方向的SG装置之后再次发生劈裂。

\subsection{电子-光子过程}

能标从低到高有三种过程:光电效应、康普顿散射和电子对产生。

经典电动力学不能够解释为什么一些情况下向自由电子入射光会得到频率不同的反射光——在镜子里,蓝色的衣服当然不可能是红色的。但使用X射线照射一些物质后,确实能够得到频率发生变化的散射光。这个效应就是\concept{康普顿效应}。
如果认为电磁波实际上是粒子,并且使用经典狭义相对论的动量和动能公式,设入射光子和出射光子的动量分别是$\vb*{p}_0$和$\vb*{p}$,两者夹角为$\theta$,碰撞之后电子的速度为$\vb*{v}$,则
\[
    \vb*{p}_0 = \vb*{p} + \frac{m \vb*{v}}{\sqrt{1 - v^2/c^2}}, \quad \abs{\vb*{p}_0} c + m c^2 = \abs{\vb*{p}} c + \frac{m c^2}{\sqrt{1 - v^2/c^2}},
\]
考虑到波长为
\[
    \lambda = \frac{h}{p},
\]
得到
\begin{equation}
    \lambda - \lambda_0 = \frac{h}{mc} (1 - \cos \theta),
    \label{eq:compton-movement}
\end{equation}
设散射强的光子能量为$\epsilon_0$,则散射之后的光子能量为
\begin{equation}
    \epsilon = \frac{\epsilon_0}{1 + \frac{\epsilon_0}{m c^2} (1 - \cos \theta)} \geq \frac{\epsilon_0}{1 + \frac{2\epsilon_0}{m c^2}},
\end{equation}
反冲电子动能为
\begin{equation}
    T = \epsilon_0 - \epsilon = \epsilon_0 \frac{\frac{\epsilon_0}{mc^2}(1-\cos \theta)}{1+\frac{\epsilon_0}{mc^2}(1-\cos \theta)} \leq \epsilon_0 \frac{\frac{2 \epsilon_0}{mc^2}}{1+\frac{2 \epsilon_0}{mc^2}}.
\end{equation}
显然,只有在狭义相对论需要被完整地考虑的时候,才会出现康普顿效应\eqref{eq:compton-movement},在牛顿时空观已经够用的时候,取$c \to \infty$,康普顿效应就消失了。
\eqref{eq:compton-movement}给出了一个自然的长度尺度,即\concept{康普顿波长}:
\begin{equation}
    \lambda = \frac{h}{mc},
\end{equation}
它是粒子之间的物理过程中相对论效应需要被完整考虑的长度尺度,低于这个长度,相对论效应非常明显,从而实际上需要使用量子电动力学分析问题。
德布罗意波长给出了实物粒子的量子效应开始变得明显的长度尺度,而康普顿波长给出了相对论效应开始变得明显的长度尺度。

我们会注意到我们假设电子和入射光子发生相互作用之后一定会产生一个出射光子。实际上,不可能发生没有出射光子的过程,因为一个入射光子完全被电子吸收的过程需要满足
\[
    \vb*{p}_0 = \frac{m \vb*{v}}{\sqrt{1 - v^2/c^2}}, \quad p_0 c = \frac{m c^2}{\sqrt{1 - v^2 / c^2}},
\]
这个方程组意味着$v=c$,而这是不可能的——电子不可能被加速到光速。

需要注意的是光电效应和上述结论没有矛盾,因为发生光电效应时电子处于束缚态,因此电子吸收不了的能量可以用于挣脱束缚态。
“自由”和“束缚”的概念在这里并不是绝对的:如果入射光的单个光子的能量足够强,那么即使一开始电子处于束缚态,束缚能也是可以忽略的。
这也就是如果使用普通的晶体做靶标,通常要发生康普顿效应需要X射线的原因。
实际上,用X射线做的康普顿效应实验通常也会得到波长没有发生变化,只是传播方向发生了变化的波,因为X射线不能电离内层电子,因此光子和内层电子发生弹性散射。
相反,康普顿效应改变了光子的能量,因此对光子而言它是\concept{非相干散射},也是\concept{非弹性散射}。

\section{束缚态单电子系统}

\subsection{一维势阱中的电子}

首先考虑如下一维无限高,无限厚的方势阱:
\begin{equation}
    V(x) = \begin{cases}
        0, \quad &0 < x < a, \\
        \infty, \quad &\text{otherwise}.
    \end{cases}
\end{equation}
代入薛定谔方程可以发现在势能无限大的地方要让方程成立只能够让$\psi=0$。
于是一维方势阱中的定态电子的波函数由以下定解问题确定:
\[
    - \hbar^2 \dv[2]{\psi}{x} = E \psi, \quad \psi\big|_{x=0} = \psi\big|_{x=a} = 0.
\]
解之,即得到
\begin{equation}
    \psi_n(x) = \begin{cases}
        \sqrt{\frac{2}{a}} \sin(\frac{n \pi x}{a}), \quad &0 < x < a, \\
        0, \quad &\text{otherwise},
    \end{cases}
\end{equation}
能量为
\begin{equation}
    E_n = \frac{n^2 h^2}{8 m a^2}.
\end{equation}
这个系统没有散射态解,因为势阱以外的地方波函数全部都是零,不可能出现无穷远处还有非零波函数值的情况。

有限高,无限厚的势阱:粒子可以有隧穿,但是在无穷远处衰减为零。

\subsection{有心力场中的单电子}

\subsubsection{哈密顿量和薛定谔方程}

有心力场中的单个电子的哈密顿量为
\begin{equation}
    \hat{H} \psi = \frac{\hat{p}^2}{2m} \psi + V(r) \psi.
\end{equation}
求解此问题等价于求解坐标表象下的定态方程
\begin{equation}
    \frac{\hat{p}^2}{2m} \psi + V(r) \psi = - \frac{\hbar^2}{2m} \laplacian \psi + V(r) \psi = E \psi,
    \label{eq:centered-force-eq}
\end{equation}
使用球坐标系,以$\theta$为和$z$轴的夹角,$\varphi$为$x$-$y$平面上的转角,则
\[
    \hat{p}^2 = -\hbar^2 \laplacian = - \hbar^2 \left( \frac{1}{r^2} \pdv{r} r^2 \pdv{r} + \frac{1}{r^2 \sin \theta} \pdv{\theta} \sin \theta \pdv{\theta} + \frac{1}{r^2 \sin^2 \theta} \pdv[2]{\varphi} \right) .
\]
可以直接分离变量,但让我们首先采取一种物理意义比较明显的变形。注意到轨道角动量算符的平方为
\[
    \hat{L}^2 = \hat{\vb*{r}}^2 \hat{\vb*{p}}^2 - (\hat{\vb*{r}} \cdot \hat{\vb*{p}}) (\hat{\vb*{p}} \cdot \hat{\vb*{r}}),
\]
在球坐标系下$\vb*{r}$只在$r$方向上有分量,于是上式变成
\[
    \hat{L}^2 = r^2 \hat{p}^2 - r \left( - \hbar^2 \pdv[2]{r} \right) r,
\]
注意到(实际上这是动量-坐标对易关系的自然推论)
\[
    r \pdv[2]{r} r = \pdv{r} r^2 \pdv{r},
\]
% 此处有误
这就给出了角动量长度平方的一个简洁的表达式:
\begin{equation}
    \hat{L}^2 = - \hbar^2 \left( \frac{1}{\sin \theta} \pdv{\theta} \sin \theta \pdv{\theta} + \frac{1}{\sin^2 \theta} \pdv[2]{\varphi} \right),
\end{equation}
当然也可以按照定义直接计算出这个表达式。相应的,哈密顿量在球坐标系下就是
\[
    \hat{H} = - \frac{\hbar^2}{2 m r} \pdv[2]{r} r + \frac{\hat{L}^2}{2 m r^2} + V(r).
\]
以上对哈密顿量的改写和经典情况可以类比。经典情况下,径向运动是有效势阱中的一维运动,且满足
\[
    \frac{p_r^2}{2m} + \frac{L^2}{2mr^2} - \frac{Ze^2}{4\pi \epsilon_0 r} = E,
\]
正好是将所有算符替换成实数之后的结果——本该如此。

现在,定态薛定谔方程成为
\[
    - \frac{\hbar^2}{2 m r} \pdv[2]{r} (r \psi) + \frac{\hat{L}^2}{2 m r^2} \psi + V(r) \psi = E \psi,
\]
算符$\hat{L}^2$仅仅和$\varphi$和$\theta$有关,因此我们可以将径向部分和角向部分做分离变量。
设
\[
    \psi = R(r) Y(\theta, \varphi),
\]
则径向部分的方程是
\begin{equation}
    \left( - \frac{\hbar^2}{2m r^2} \dv{r} r^2 \dv{r} + \frac{\hbar^2}{2m r^2} \alpha + V(r) \right) R = E R,
    \label{eq:original-r-equation}
\end{equation}
角向部分的方程为
\begin{equation}
    \hat{L}^2 Y = \alpha \hbar^2 Y.
    \label{eq:angle-equation}
\end{equation}
其中$\alpha$为常数。

\subsubsection{角向部分的方程的求解}

径向部分的方程包含一个可变的$V(r)$项,而角向部分的方程可以直接解出。
请注意角向部分实际上是一个拉普拉斯方程的角向部分,因此其解为球谐函数。
下面简单地展示其求解过程。设
\[
    Y(\theta, \varphi) = \Theta(\theta) \Phi(\varphi),
\]
代入
\[
    - \left( \frac{1}{\sin \theta} \pdv{\theta} \sin \theta \pdv{\theta} + \frac{1}{\sin^2 \theta} \pdv[2]{\varphi} \right) Y = \alpha Y,
\]
得到两个方程
\[
    \frac{1}{\Phi} \dv[2]{\Phi}{\varphi} = c,
\]
以及
\[
    - \left( \frac{1}{\sin \theta} \dv{\theta} \sin \theta \dv{\theta} + \frac{1}{\sin^2 \theta} c \right) \Theta = \alpha \Theta.
\]
$\Phi$必须是单值的,因为波函数应该是单值的,于是
\[
    c = m^2, \quad m = 0, 1, 2, \ldots,
\]
而
\[
    - \left( \dv{\cos \theta} (1 - \cos^2 \theta) \dv{\cos \theta} + \frac{1}{1 - \cos^2 \theta} m^2 \right) \Theta = \alpha \Theta.
\]
这是一个连带勒让德方程,为了保证$\Theta(\theta)$单值且有界,应有
\[
    \alpha = l(l+1), \quad l \geq \abs{m}, \quad l = 0, 1, 2, \ldots,
\]
于是\eqref{eq:angle-equation}的一组正交解为
\[
    Y(\theta, \varphi) = \ee^{\ii m \varphi} \legpoly_l^m(\cos \theta),
\]
其中$\legpoly_l^m$表示缔合勒让德多项式。归一化使用球坐标系的角向积分
\[
    \int \sin \theta \dd{\theta} \dd{\varphi},
\]
最后得到正交归一化的球谐函数
% TODO:(-1)^m因子?
\begin{equation}
    \begin{split}
        Y_{lm}(\theta, \varphi) = (-1)^m \sqrt{\frac{(2l+1)(l-\abs{m})!}{4\pi (l+\abs{m})!}} \legpoly_l^\abs{m} (\cos \theta) \ee^{\ii m \varphi}, \\
        l = 0, 1, \ldots, \quad m = \pm l, \pm (l-1), \ldots, 0 .
    \end{split}
\end{equation}

球谐函数是$\hat{L}^2$的本征函数,但它带有两个量子数$l$和$m$,这意味着$\hat{L}^2$的本征函数存在简并;$l,m$分别标记了$\hat{L}^2$的本征值和某个不确定的可观察量的本征值。
注意到球坐标系中
\[
    \hat{L}_z = - \ii \hbar \pdv{\varphi},
\]
它正是$\Phi(\varphi)$满足的本征方程中的那个算符,因此$m$标记的是$\hat{L}_z$的本征值,$l$标记的是$\hat{L}^2$的本征值。
这是正确的,因为
\[
    m = \pm l, \pm (l-1), \ldots, 0
\]
正是角动量代数的重要性质。我们下面记此处的$m$为$m_l$,与自旋角动量在$z$方向上的取值$m_s = \pm \frac{1}{2}$相区分。
球谐函数$Y_{lm}(\theta, \varphi)$同时是$\hat{L}^2$和$L_z$的本征函数,相应的本征值为
\begin{equation}
    \hat{L}^2 Y_{lm} = l(l+1) \hbar^2 Y_{lm}, \quad \hat{L}_z Y_{lm} = m \hbar Y_{lm}.
    \label{eq:orbital-angular-momentum}
\end{equation}

\subsubsection{量子数}\label{sec:quantum-number}

现在径向方程\eqref{eq:original-r-equation}成为了
\begin{equation}
    \left( - \frac{\hbar^2}{2m r^2} \dv{r} r^2 \dv{r} + \frac{\hbar^2}{2m r^2} l(l+1) + V(r) \right) R = E R.
    \label{eq:r-equation}
\end{equation}
这是一个单变量的本征值问题,它还会产生一个(而且也只有一个,因为一旦$E$确定了,$R$就确定了,不存在简并)量子数$n$,它标记不同的能量。
请注意由于角动量部分被分离变量出去了,实际上多出来了一个有效势$l(l+1)$项。
更加变于求解的一种形式是设$u=rR$,则有
\begin{equation}
    - \frac{\hbar^2}{2m} \dv[2]{u}{r} + \frac{\hbar^2}{2m r^2} l(l+1) u + V(r) u = Eu, \quad R = \frac{u}{r}.
\end{equation}

这样,\eqref{eq:centered-force-eq}有一组由$n, l, m_l$标记的正交归一化解,再考虑到自旋自由度,电子的状态就完全确定了。
通过求解过程可以看出前三个量子数标记了$\hat{H}, \hat{L}^2, \hat{L}_z$的本征值;
它们是通过分离变量求解坐标空间中的薛定谔方程得到的,因此随着时间变化,它们对应的物理量是守恒的且彼此对易,能够找到这样三个守恒且彼此对易的物理量当然是因为有心力系统的对称性。
目前尚未引入任何涉及自旋的机制,因此自旋也是恒定不变的。
于是我们找到了四个标记电子的好量子数:
\begin{enumerate}
    \item 主量子数$n$,它标记不同的能量,它是分立的,因为电子陷在一个势阱中,从而是离散谱;
    \item 角量子数$l$,它标记不同的角动量大小,对一部分势场形式,也参与标记不同的能量,它是分立的,因为波函数在角方向上是单值的;
    \item 磁量子数$m_l = 0, \pm 1, \ldots, \pm l$,它标记$z$轴上的角动量分量,同样也是分立的;
    \item 自旋量子数$m_s = \pm \frac{1}{2}$,它来自电子的内禀旋转自由度;自旋量子数和我们刚才讨论的轨道空间无关。
\end{enumerate}

这四个量子数直接决定了波函数的形状。主量子数决定了径向概率分布,角量子数和磁量子数决定了波函数的角向概率分布。
这四个量子数还可以确定其它一些量子数。例如,做宇称变换
\[
    (r, \theta, \varphi) \longrightarrow (r, \pi - \theta, \pi + \varphi),
\]
径向部分始终不变,若$l$为奇数则球谐函数会差一个负号,从而波函数为奇宇称,若$l$为偶数则球谐函数不变,从而波函数为偶宇称。

纯量子的理论展现出了和经典理论很不同的一些特性。请注意量子理论中电子可以完全没有角动量,这在经典理论下是不可能的——电子会直接落入有心力的力心,比如说原子核。
然而,哈密顿量\eqref{eq:columb-electron-hamiltonian}中各项不对易从而有量子涨落,因此如果角动量确定为零,那么电子的位置就不能够确定,因此电子并不会落入原子核。

\subsubsection{角动量代数}\label{sec:algebra-of-angular}

本节从对称性的角度来分析有心力场中的电子的角量子数。
有心力场各向同性的性质意味着系统具有$SO(3)$对称性,从而$SO(3)$的李代数中的每个生成元都是守恒量,当然这就是大名鼎鼎的角动量守恒。使用$\hat{J}$表示这些生成元。
$SO(3)$的李代数的对易关系就是角动量算符的对易关系,即
\begin{equation}
    \hat{\vb*{J}} \times \hat{\vb*{J}} = \ii \hbar \hat{\vb*{J}},
\end{equation}
或者也可以写成显式的李括号的形式:
\begin{equation}
    \comm*{\hat{J}_i}{\hat{J}_j} = \ii \hbar \epsilon_{ijk} \hat{J}_k.
\end{equation}
从这个李代数可以马上发现$\hat{\vb*{J}}^2$是卡西米尔元。
对每个不可约表示,三个方向的角动量互不对易,因此只需要取一个角动量,列出其本征态即可完整地描述角动量空间。
这样就可以使用$\hat{\vb*{J}}^2$和$\hat{J}_z$的共同本征态作为角动量空间的基底,无论是不是不可约表示。

对$SO(3)$的不可约表示,设$j$是角量子数,有
\begin{equation}
    \hat{\vb*{J}}^2 \ket{j,m_j} = j(j+1)\hbar^2 \ket{j,m_j}, \quad \hat{J}_z \ket{j,m_j} = m_j \hbar \ket{j,m_j},
    \label{eq:eigenvalue-of-angular}
\end{equation}
其中$j$和$m_j$都是整数或者半整数。我们知道$j$唯一地标记了每个有限维不可约表示,而
\begin{equation}
    m_j = -j, -j+1, \ldots, j.
\end{equation}
要证明\eqref{eq:eigenvalue-of-angular}(并做一些别的计算),通常需要定义产生湮灭算符,具体来说,定义
\begin{equation}
    \hat{J}_\pm = \frac{1}{\sqrt{2}} (\hat{J}_x \pm \ii \hat{J}_y),
\end{equation}
可以计算得到对易关系
\begin{equation}
    \comm*{\hat{J}_z}{\hat{J}_\pm} = \hbar \hat{J}_\pm,
\end{equation}
因此$\hat{J}_+$和$\hat{J}_-$就是$\hbar m_j$的升降算符,每作用一次增加或者减少$\hbar$的$z$方向角动量。
如果是有限维表示,必定有
\begin{equation}
    \hat{J}_\pm \ket{j, m_j} = \sqrt{\frac{(j \mp m_j)(j \pm m_j + 1)}{2}} \hbar \ket{j, m_j \pm 1}.
\end{equation}
从上式也可以推导出
\begin{equation}
    \begin{aligned}
        \hat{J}_x \ket{j, m_j} &= \frac{1}{2} \sqrt{(j-m_j)(j+m_j+1)} \hbar \ket{j, m_j+1} \\
        &+ \frac{1}{2} \sqrt{(j+m_j)(j-m_j+1)} \hbar \ket{j, m_j-1} , \\
        \hat{J}_y \ket{j, m_j} &= \frac{1}{2} \ii \sqrt{(j+m_j)(j-m_j+1)} \hbar \ket{j, m_j-1} \\
        &- \frac{1}{2} \ii \sqrt{(j-m_j)(j+m_j+1)} \hbar \ket{j, m_j+1}.
    \end{aligned}
    \label{eq:apply-jx-jy-to-angular-state}
\end{equation}

现在回过头和\eqref{eq:orbital-angular-momentum}中求解出来的轨道角动量做对比,会发现中心场下的束缚态电子的轨道角动量的确是一个$SO(3)$的有限维表示。
电子对应的量子场是一个有质量的旋量场,它的内禀自由度是$SO(3)$的一个二维不可约表示,因此也就有两个自旋量子数:$1/2$和$-1/2$。
实际上,如果依次取自旋空间的基底为$\ket{1/2}$和$\ket{-1/2}$,则$\hat{S}_z$的矩阵形式为
\[
    \hat{S}_z = \pmqty{\frac{1}{2} \hbar & 0 \\ 0 & -\frac{1}{2} \hbar}.
\]
定义泡利矩阵为
\begin{equation}
    \sigma_x = \pmqty{0 & 1 \\ 1 & 0}, \quad \sigma_y = \pmqty{0 & -\ii \\ \ii & 0}, \quad \sigma_z = \pmqty{1 & 0 \\ 0 & -1},
\end{equation}
使用\eqref{eq:apply-jx-jy-to-angular-state}可以计算出
\begin{equation}
    \hat{S}_i = \frac{\hbar}{2} \sigma_i.
\end{equation}

\subsubsection{库伦势场和类氢原子}

本节讨论库伦势场中的电子运动情况。首先我们表明,原子的的确确是某个带正电荷而非常重的核约束了一些电子而得到的体系。
通过阴极射线实验可以证实原子中确确实实有电子,但是正电荷的分布是不清楚的。
采用半经典模型,设一个$\alpha$粒子与原子发生散射,则势能为
\begin{equation}
    V(r) = \frac{1}{4\pi \epsilon_0} \frac{Z_1 Z_2 e^2}{r}, 
\end{equation}
散射角为
\[
    \theta = \pi - 2 \int_{r_\text{min}}^\infty \frac{b \dd{r}}{r^2 \sqrt{1 - V(r)/E - b^2/r^2}},  
\]
而散射截面为
\[
    \dd{\sigma} = \frac{b(\theta)}{\sin \theta} \abs{\dv{b}{\theta}} \dd{\Omega},
\]
最终可以计算出
\begin{equation}
    \dd{\sigma} = \left(\frac{a}{4}\right)^2 \sin^{-4}\frac{\theta}{2} \dd{\Omega}, \quad a = \frac{Z_1 Z_2 e^2}{4\pi \epsilon_0 E}.    
\end{equation}
如果我们使用非常薄的金属箔作为靶标,并假定不同原子核分布非常稀疏(由于原子核非常小,这是正确的),从而不同层的原子核都是错开的,没有互相遮挡,设金属箔数密度为$n$,厚度为$t$,则被散射到立体角$\dd{\Omega}$中的$\alpha$粒子数满足
\[
    \frac{\dd{N}}{N} = \frac{n A t \dd{\sigma}}{A} = nt \dd{\sigma},
\]
于是
\begin{equation}
    \frac{\dd{N}}{N \dd{\Omega}} = n t \left(\frac{a}{4}\right)^2 \sin^{-4}\frac{\theta}{2}.
\end{equation}
此即\concept{卢瑟福散射公式}。卢瑟福散射公式成立的条件包括:
\begin{enumerate}
    \item 非相对论近似,因为使用了牛顿力学的动能公式;
    \item 大角度散射,也就是说瞄准距离$b$比较小,因为只有这样入射的$\alpha$粒子才能够充分接近原子核,从而可以像我们做的那样,忽略外层电子的屏蔽效应;
    \item $r_\text{min}$要大于原子核半径,从而不会发生核反应,且可以将原子核看成点电荷;
    \item 在满足第一个和第三个条件的前提下,入射$\alpha$粒子动能尽可能大,从而外层电子的屏蔽作用可以忽略。
\end{enumerate}
这个公式和实验结果一致,说明原子结构中确实有原子核。
再往下,经典理论就会造成著名的疑难,就是既然电子绕着原子核运动,那么必然会发出辐射而损失能量。
因此我们接下来使用量子力学来分析原子内部电子的运动情况。

库伦势场中的单个电子的哈密顿量为
\begin{equation}
    \hat{H} \psi = \frac{\hat{p}^2}{2m} \psi - \frac{Z}{4\pi \epsilon_0} \frac{e^2}{\abs{\vb*{r}}} \psi.
    \label{eq:columb-electron-hamiltonian}
\end{equation}
我们常常将这样的体系称为类氢原子,因为它和氢原子的结构除了$Z$可能不一样以外完全一致。
此时\eqref{eq:r-equation}为
\[
    \left( - \frac{\hbar^2}{2m r^2} \dv{r} r^2 \dv{r} + \frac{\hbar^2}{2m r^2} l(l+1) - \frac{1}{4\pi \epsilon_0} \frac{e^2}{r} \right) R = E R,
\]
显然这是一个束缚在势阱中的电子的方程,它必定有束缚态解,从而可以提供我们需要的主量子数。

为了获取一些灵感,首先考虑$r\to \infty$的极限,得到渐进解
\[
    R = \exp(- \sqrt{- \frac{2 m E}{\hbar^2}} r).
\]
设
\[
    k^2 = - \frac{2 m E}{\hbar^2},
\]
并令
\[
    \rho = 2 k r, \quad \gamma = \frac{m Z e^2}{4\pi \epsilon_0 k \hbar^2}, 
\]
取试探解
\[
    R(\rho) = \ee^{- \rho / 2} F(\rho),
\]
得到
\[
    \dv[2]{F}{\rho} + \left( \frac{2}{\rho} - 1 \right) \dv{F}{\rho} + \left( \frac{\gamma - 1}{\rho} - \frac{l(l+1)}{\rho^2} \right) F = 0.
\]
这仍然是一个本征值问题,本征值由$\gamma$标记。通过广义幂级数展开可以知道$\gamma$应当为整数,这样就得到了本征值
\begin{equation}
    E_n = - \frac{1}{2} m Z^2 \left( \frac{e^2}{4\pi \epsilon_0 \hbar} \right)^2 \frac{1}{n^2} = - \frac{1}{2} m Z^2 (\alpha c)^2 \frac{1}{n^2}.
\end{equation}
通过广义幂级数展开还可以发现广义幂级数形如
\[
    F(\rho) = \rho^l \sum_j a_j \rho^j.
\]
我们会发现除了$\rho^l$以外的$F(\rho)$的因子实际上服从合流拉盖尔方程,于是得到
\[
    R(\rho) = \ee^{-\rho/2} \rho^l \laguerre_{n+l}^{2l+1}(\rho).
\]
定义\concept{第一波尔半径}
\begin{equation}
    a_1 = \frac{4\pi \epsilon_0 \hbar^2}{m e^2},
\end{equation}
我们下面会看到它是经典原子模型(电子绕着正电荷匀速圆周运动)给出的轨道半径,则能量可以写成
\begin{equation}
    E_n = - \frac{Z^2 e^2}{4 \pi \epsilon_0 } \frac{1}{2 a_1} \frac{1}{n^2} = \frac{E_1}{n^2}, \quad E_1 = - \frac{1}{4\pi \epsilon_0} \frac{e^2}{2 a_1}.
    \label{eq:hydrogen-energy}
\end{equation}
将各个常数放回上式并归一化就得到
\begin{equation}
    R_{n, l} = - \sqrt{\left( \frac{2 Z}{n a_1} \right)^3 \frac{(n-(l+1))!}{2n ((n + l)!)^3}} \exp(- \frac{Z r}{n a_1}) \left( \frac{2 Z r}{n a_1} \right)^l \laguerre_{n+l}^{2l+1}\left(\frac{2 Z r}{n a_1}\right),
\end{equation}
其中$\laguerre_{n+l}^{2l+1}$是合流拉盖尔多项式。
$n-l$决定了径向峰值的数目。为了让合流拉盖尔多项式有良定义,我们有
\begin{equation}
    l = 0, 1, 2, \ldots, n-1.
\end{equation}
至此,库伦中心势下的电子运动情况完全确定。

可以依稀从量子力学中的氢原子看出一些经典的图像。在经典的原子模型中,粒子可以做椭圆运动,但是运动的能量仅仅关乎一个参数即椭圆的半长轴$a$,而和半短轴$b$无关;半短轴$b$则决定角动量等。
因此能量和角动量是分开的。角动量可以有不同的指向,因此角动量长度和它在$z$轴上的投影也没有必然的关系(当然,角动量在$z$轴上的投影不可能超过总的角动量长度)
实际上,如果我们假定:
\begin{enumerate}
    \item 核外电子绕着原子核运动遵循牛顿定律;
    \item 在稳定的轨道上运动的电子不会发射或者吸收电磁波;
    \item 角动量量子化,即
    \begin{equation}
        L = m v r = n \hbar,
        \label{eq:quantum-angular-momentum}
    \end{equation}
\end{enumerate}
我们也可以得到类氢原子能级(这称为\concept{波尔模型})。
做受力分析
\[
    m \frac{v^2}{r} = \frac{1}{4\pi \epsilon_0} \frac{Z e^2}{r^2},
\]
并结合\eqref{eq:quantum-angular-momentum},可以得到
\begin{equation}
    r_n = \frac{4\pi \epsilon_0 \hbar^2}{Zme^2} n^2.
\end{equation}

求解出氢原子的薛定谔方程的完整解之后,可以发现主量子数为$n$的能级上有$n$个可能的角量子数,每个角量子数又允许$2l+1$个磁量子数,而最后还有两个自旋量子数,因此主量子数为$n$的能级上有
\[
    \frac{1}{2} (1 + (2(n-1)+1)) n = n^2
\]
个轨道,有$2n^2$个电子。

我们注意到,一般来说,能量和$n, l$都有关系(和$m$确定没有关系,因为自旋旋转不变性),但在库伦势场中能量和$l$实际上并没有关系。
这意味着库伦势场中其实还有一个隐藏的对称性。从半经典模型的考虑,角动量由半短轴决定,但是能量只和半长轴有关,因此这并不出乎意料。
当然,电子的轨道可以不垂直于我们选取的$xOy$平面,从而角动量的$z$轴投影可能变动,但能量和坐标轴选取无关。

\subsection{跃迁和偶极辐射}\label{sec:electro-dipole}

电子和电磁场耦合,因此可以在不同能级之间跃迁而发射或吸收光子。
跃迁包括受激跃迁(电子首先吸收光子,然后发生跃迁)以及自发跃迁(电子直接发生跃迁)。
对这一过程的完整计算涉及量子电动力学的束缚态,但通常对能标不是非常高的过程,使用量子化的原子和经典电动力学就足够计算一些问题。
受激跃迁只需要量子化原子加上经典电动力学即可完全解释,而自发跃迁不能使用经典电动力学解释而必须将光场量子化,因为“自发”意味着光场的真空涨落,这要使用量子理论处理。

\subsubsection{光子}

我们首先从经典电动力学中的平面波来探讨问题,这是可以的,因为实际上光子的产生湮灭算符对应着量子化的电磁场的傅里叶分量的振幅,于是单个光子的经典对应就是一个平面波。
考虑以下单色平面波
\begin{equation}
    \vb*{E} = \vb*{E}_0 \ee^{\ii(\vb*{k} \cdot \vb*{r} - \omega t)}, \quad \vb*{E}_0 = \sum_{m_s} E_{m_s} \vb*{e}_{m_s}, \quad m_s = -1, 0, 1,
\end{equation}
这里的$m_s$指的就是自旋。光场是矢量场,因此自旋为1,这样其内禀旋转自由度有三个方向,正好和矢量有$x, y, z$三个方向对应。
定义
\begin{equation}
    \vb*{e}_{\pm 1} = \mp \frac{1}{\sqrt{2}} (\vb*{e}_x \pm \ii \vb*{e}_y), \quad \vb*{e}_0 = \vb*{e}_z,
\end{equation}
容易看出它们正交,且$\vb*{e}_1$对应着电场垂直于$z$轴,且绕$z$轴逆时针旋转,$\vb*{e}_{-1}$对应着电场垂直于$z$轴,且绕$z$轴顺时针旋转,因此$\vb*{e}_{\pm 1}$是垂直于$z$轴的圆偏振基,称它们为$\sigma^\pm$基,而$\vb*{e}_0$是平行于$z$轴的线偏振基,称为$\pi$基。

照惯例,取$\vb*{k}$的方向为$z$轴,电磁波只有横波模式没有纵波模式(这是$U(1)$规范场的性质:总是可以选取一个规范让纵波消失),因此$\pi$基没有复振幅,光从来不在$\pi$基上有偏振,或者等价地说,光子在其传播方向上的自旋只有$\pm 1$,没有$0$。纵光子是观测不到的。

原子发射光子时,由对称性分析,如果能够良定义一个光子位置$\vb*{r}$,且将位矢零点放在原子中心,那么有
\[
    \vb*{r} \times \vb*{k} = 0,
\]
因此将位矢放在原子中心时由原子发射的光子没有轨道角动量。于是光子的角动量仅含有自旋角动量。

\subsubsection{爱因斯坦的唯象理论}\label{sec:einstein-phonomenon}

考虑温度为$T$的空腔中有大量相同的原子,显然处于定态$i$和$j$的原子需要满足玻尔兹曼分布率
\[
    N_i \propto \ee^{-\frac{E_i}{k_\text{B} T}},
\]
或者写成
\[
    \frac{N_j}{N_i} = \ee^{-\hbar \omega_{ji} / kT}.
\]
能够达到热力学平衡意味着电子需要在不同能级之间跃迁。
电子和电磁场有耦合,因此电子在不同能级上跃迁确实是可以的。跃迁发生的机制可能有这么几种:
\begin{enumerate}
    \item 自发发射,也就是电子放出一个光子,跃迁到较低的能级;
    \item 受激发射,即电子先吸收一个光子再放出一个光子,然后发生跃迁;
    \item 吸收,即电子吸收一个光子然后跃迁到较高的轨道上。
\end{enumerate}

当然,在经典电动力学的框架下,只应该出现后两种情况。但是这样一来系统实际上不能够达到玻尔兹曼分布。

受激发射和吸收可以使用量子化的原子和一个经典电磁场耦合来计算,但是自发发射在这个框架下是很难解释的,因为一个激发态的原子放在完全没有电磁场的空间内照样会有自发发射。
对这一现象的完整解释显然涉及真空涨落,因此需要量子电动力学。

设温度为$T$的光场中频率为$\omega$附近的能量密度为$u(\omega, T)$。设有两个能级$i$和$j$,且$E_j > E_i$。
我们假定(之后会通过量子力学严格证明)自发发射的跃迁率和$u$无关,而受激发射和吸收的跃迁率正比于$u(\omega_{ji}, T)$,其中
\begin{equation}
    \hbar \omega_{ji} = E_j - E_i,
    \label{eq:photon-energy}
\end{equation}
这样在这两个能级之间的自发发射、受激发射、吸收的跃迁率分别是
\[
    A_{ji} N_j, \quad B_{ji} N_j u(\omega_{ji}, T), \quad C_{ij} N_i u(\omega_{ji}, T).
\]
能级$i$向上跃迁到$j$的跃迁率为
\[
    \lambda_{ij} = C_{ij} u(\omega_{ji}, T),
\]
能级$j$向下跃迁到$i$的跃迁率为
\[
    \lambda_{ji} = B_{ji} u(\omega_{ji}, T) + A_{ji}.
\]
平衡时两者相等,即有
\[
    N_i C_{ij} u(\omega_{ji}, T) = N_j (B_{ji} u(\omega_{ji}, T) + A_{ji})
\]
$T \to \infty$,不同能级上原子分布的个数差别变得很小,$u \to \infty$,而上式仍然成立,因此$C_{ij} = B_{ji}$%
\footnote{请注意对温度的依赖被完全归入$u(\omega, T)$中,系数$C$和$B$由电子和光场耦合的方式决定,因此不依赖温度。}%
,这样就有
\[
    u(\omega_{ji}) = \frac{A_{ji} / B_{ji}}{\ee^{\omega_{ji} \hbar / k T} - 1}.
\]
由于原子能级可以随意调整,我们有
\[
    u(\omega) = \frac{A_{ji} / B_{ji}}{\ee^{\omega \hbar / k T} - 1}.
\]
而由于空腔内的辐射能量密度为
\[
    u = \frac{\hbar \omega^3}{\pi^2 c^3} \frac{1}{\ee^{\hbar \omega / kT} - 1}
\]
自发发射跃迁率为
\[
    A_{ji} = \frac{\omega_{ji}^3}{3\pi \epsilon_0 \hbar c^3}
\]

\subsubsection{跃迁系数的推导}\label{sec:electro-dipole-hopping}

现在尝试从头计算跃迁率。如前所述,热平衡(实际上不仅仅是热平衡)时受激发射和吸收的跃迁率相等,这是由细致平衡条件以及高温下辐射密度趋于无限大这两个事实保证的,没有用到任何关于辐射机制的细节。
这样就只需要计算自发发射和受激发射的跃迁率。
在上一节中我们用到了空腔内的辐射能量密度,通过经典电动力学推导出的辐射能量密度公式是错误的(红外灾难和紫外灾难),这意味着完整地讨论跃迁需要量子电动力学。
不过,受激发射还是可以使用经典电动力学得到。

仅考虑电偶极辐射,则哈密顿量中需要加入这样一项:
\begin{equation}
    \hat{H}_\text{DE} = - \hat{\vb*{d}} \cdot \vb*{E}, \quad \hat{\vb*{d}} = -e \hat{\vb*{r}}.
\end{equation}
由于是偶极辐射,电场近似认为不存在空间变动,即可以展开成以下仅显含时间的傅里叶分量:
\[
    \vb*{E} = \int \dd{\omega} \vb*{E}(\omega) \ee^{- \ii \omega t}.
\]
实际上,由于电子的运动会释放光子,终究不能够将电场$\vb*{E}$看成一个外部给定的场,而必须把它的值看成是系统状态的一部分,电场的状态和电子的状态直积得到系统状态。
我们接下来将只考虑电子的状态,换而言之,将电场的状态迹掉了,因此电子和不同频率的电场的相互作用得到的概率振幅应当被非相干叠加,即使用经典概率的方法叠加。%
\footnote{实际上,如果完整地使用cQED做计算,把光子和原子全部计入态矢量中,那么的确在计算过程中不会出现混合态,但是当我们开始略去光子的状态而只考虑原子的跃迁时,就已经隐含地迹掉了光子,从而原子的状态就成为混合态了。
在推导原子的跃迁时必然要在某个阶段引入混合态,从而概率振幅非相干叠加,因为\autoref{sec:einstein-phonomenon}中的能量密度是光子的热系综的能量密度,因此必须能够保证我们的推导对热系综也适用。}%
在经典极限下平面波的偏振方向可以取任何方向,因为光子一方面单个能量很弱,另一方面总数又很大,的确可以让电场振动方向指向任意方向。
总之,相互作用哈密顿量为
\[
    \hat{H}_\text{DE} = \int \dd{\omega} e \hat{\vb*{r}} \cdot \vb*{E}(\omega),
\]
且每个$\vb*{E}(\omega)$又允许任意的偏振方向取向。换而言之电子能够和$\omega$、偏振方向随意取的平面波电场模式(实际上是光子模式的经典极限)发生相互作用。

对每个电场模式,使用一阶微扰计算跃迁振幅。一阶含时微扰论的概率振幅为
\[
    \braket{n}{\psi(t)} = - \frac{\ii}{\hbar} \ee^{- \ii E_n t / \hbar} \sum_m \int_0^t \dd{t'} \mel{n}{\hat{H}_\text{DE}}{m} \ee^{\ii \omega_{nm} t} \braket{m}{\psi(0)},
\]
其中
\[
    \hbar \omega_{nm} = E_n - E_m,
\]
$m, n$等表示一组正交基——在这里就是使用$(n, l, m_l, m_s)$表示出的原子态。
这一组原子态是一组偏好基,它们之间的跃迁概率就是以上概率振幅的模长平方。
考虑这样一个过程:一开始原子处于态$\ket{i}$上,然后和频率为$\omega$的电场发生相互作用,跃迁到态$\ket{j}$上。
这个过程的振幅为
\[
    \begin{aligned}
        \braket{j}{\psi(t)} &= - \frac{\ii}{\hbar} \ee^{- \ii E_n t / \hbar} \int_0^t \dd{t'} \mel{j}{e \hat{\vb*{r}} \cdot \vb*{E}(\omega) \ee^{- \ii \omega t'}}{i} \ee^{\ii \omega_{ji} t'} \\
        &= - \frac{\ii}{\hbar} \ee^{- \ii E_n t / \hbar} e \vb*{E}(\omega) \cdot \mel{j}{\hat{\vb*{r}}}{i} \frac{\ee^{\ii (\omega_{ji} - \omega) t} - 1}{\ii (\omega_{ji} - \omega)} \\
        &= - \frac{\ii}{\hbar} \ee^{- \ii E_n t / \hbar} e E(\omega) \mel{j}{\hat{\vb*{r}}}{i} \cos \theta \frac{\ee^{\ii (\omega_{ji} - \omega) t} - 1}{\ii (\omega_{ji} - \omega)},
    \end{aligned}
\]
其中$\theta$是电场偏振方向和$\mel{j}{\hat{\vb*{r}}}{i}$的夹角。

电子从$\ket{i}$到$\ket{j}$的过程可以通过和不同波长、不同偏振方向的电磁波相互作用而发生。由对称性分析,不同偏振方向的电磁波出现的可能性是一样的,也就是说偏振方向在立体角$\dd{\Omega}$中,频率出现在$\omega$到$\omega+\dd{\omega}$的电磁波出现的概率为
\[
    P \dd{\Omega} = p(\omega) \frac{\dd{\Omega}}{4\pi} \dd{\omega},
\]
于是
\[
    \begin{aligned}
        P_{i \to j} &= \int p(\omega) \dd{\omega} \int \frac{\dd{\Omega}}{4\pi} \abs*{\braket{j}{\psi(t)}}^2 \\
        &= \frac{e^2}{\hbar^2} \int \frac{\dd{\Omega}}{4\pi} \cos^2 \theta \int \dd{\omega} p(\omega) E(\omega)^2 \abs*{\mel{j}{\hat{\vb*{r}}}{i}}^2 \frac{\sin^2((\omega_{ji} - \omega) t / 2)}{(\omega_{ji}-\omega)^2} \\
        &= \frac{e^2}{3 \hbar^2} \abs*{\mel{j}{\hat{\vb*{r}}}{i}}^2 \int \dd{\omega} p(\omega) E(\omega)^2 \frac{\sin^2((\omega_{ji} - \omega) t / 2)}{(\omega_{ji}-\omega)^2}.
    \end{aligned}
\]
我们关心的时间尺度通常比较大,而随着$t$增大,$\frac{\sin^2((\omega_{ji} - \omega) t / 2)}{(\omega_{ji}-\omega)^2}$会变得越来越尖锐,只在$\omega = \omega_{ji}$附近有比较明显的非零值,于是近似有(这个近似表明跃迁几乎总是发出频率就是$\omega_{ji}$的电磁波,这当然是正确的)
\[
    \begin{aligned}
        P_{i \to j} &= \frac{e^2}{3 \hbar^2} \abs*{\mel{j}{\hat{\vb*{r}}}{i}}^2 p(\omega_{ji}) E(\omega_{ji})^2 \int_{-\infty}^\infty \dd{\omega} \frac{\sin^2((\omega_{ji} - \omega) t / 2)}{(\omega_{ji}-\omega)^2} \\
        &= \frac{e^2}{3 \hbar^2} \abs*{\mel{j}{\hat{\vb*{r}}}{i}}^2 p(\omega_{ji}) E(\omega_{ji})^2 \pi t.
    \end{aligned}
\]
可见跃迁几率随着时间增大而线性增大。单位时间的跃迁几率为
\[
    \Gamma_{i \to j} = \dv{P_{i \to j}}{t} = \frac{\pi e^2}{3 \hbar^2} \abs*{\mel{j}{\hat{\vb*{r}}}{i}}^2 p(\omega_{ji}) E(\omega_{ji})^2.
\]
考虑到频段$\omega$到$\omega+\dd{\omega}$上的电磁场能量(它是电场能量的两倍)为
\begin{equation}
    u(\omega) = \epsilon_0 p(\omega) E(\omega)^2,
\end{equation}
最后得到
\begin{equation}
    \Gamma_{i \to j} = \frac{\pi e^2 \expval*{\vb*{r}_{ji}}^2}{3 \epsilon_0 \hbar^2} u(\omega_{ji}), \quad B_{ji} = \frac{\pi e^2}{3 \epsilon_0 \hbar^2},
\end{equation}
其中
\begin{equation}
    e \expval*{\vb*{r}_{ji}} = e \int \dd[3]{\vb*{r}} \psi_i^*(\vb*{r}) \vb*{r} \psi_j(\vb*{r})
    \label{eq:electro-dipole}
\end{equation}
为电偶极矩的期望值。

以上是国际单位制的推导,如果改用高斯单位制,由于电磁能量的形式会发生变化,将得到
\begin{equation}
    B_{ji} = \frac{4\pi^2 e^2}{3 \hbar^2} \abs{\expval*{\vb*{r}_{ji}}}^2.
\end{equation}

\subsubsection{选择定则}

\eqref{eq:electro-dipole}中的波函数对$(r, \theta, \varphi)$是分离变量的,而
\[
    \begin{aligned}
        r_x &= r \sin \theta \cos \varphi, \\
        r_y &= r \sin \theta \sin \varphi, \\
        r_z &= r \cos \theta 
    \end{aligned}
\]
也是分离变量的。
记$\psi_1$和$\psi_2$的量子数分别是$n_1, l_1, m_1$和$n_2, l_2, m_2$。
\eqref{eq:electro-dipole}给出非零结果的必要条件是其角部分均不为零。
在$\varphi$方向上,积分是
\[
    \int_0^{2\pi} \dd{\varphi} \ee^{ - \ii m_1 \varphi} \cos \varphi \ee^{\ii m_2 \varphi} \vb*{e}_x + \int_0^{2\pi} \dd{\varphi} \ee^{ - \ii m_1 \varphi} \sin \varphi \ee^{\ii m_2 \varphi} \vb*{e}_y + \int_0^{2\pi} \dd{\varphi} \ee^{ - \ii m_1 \varphi} \ee^{\ii m_2 \varphi} \vb*{e}_z,
\]
让三个分量不全为零的可能取值是:
\[
    m_2 - m_1 = \pm 1, 0.
\]
在$\theta$方向上,积分是
\[
    \int_0^\pi \dd{\theta} \sin \theta \legpoly_{l_1}^{m_1} (\cos \theta) \legpoly_{l_2}^{m_2} (\cos \theta) (\vb*{e}_x + \vb*{e}_y) + \int_0^\pi \dd{\theta} \cos \theta \legpoly_{l_1}^{m_1} (\cos \theta) \legpoly_{l_2}^{m_2} (\cos \theta) \vb*{e}_z,
\]
使用勒让德多项式的性质,可以证明让三个分量不全为零的可能取值为
\[
    l_2 - l_1 = \pm 1.
\]
总之,要让受激发射系数不为零,需要
\begin{equation}
    \Delta m = 0, \pm 1, \quad \Delta l = \pm 1.
\end{equation}
这就是\concept{单电子原子跃迁的选择定则}。
实际上,也可以通过守恒量分析得到这个结论。
由于光子为自旋1的粒子,%
\footnote{这里可能会遇到一个疑难:光子的自旋角动量只在其前进方向上有投影,且只有$\pm 1$两种取值,那么似乎$z$方向角动量守恒意味着只能有$\Delta m = \pm 1$。
然而,光子的前进方向和我们选取的电子$z$方向未必相同,因此光子的自旋角动量投影在$z$方向上还是会有$0, \pm 1$三种取值。
}%

不满足选择定则的跃迁称为\concept{禁戒跃迁}。通过磁偶极跃迁、电四极子跃迁、双光子跃迁甚至原子和原子之间的碰撞等方法,禁戒跃迁也是可以发生的,但是相对来说发生概率不大,从而对应的能级为亚稳态——在电偶极跃迁比较频繁的时间尺度上它不会发生,但是在更长的时间尺度上它的确会发生。

我们通常会讨论的和原子相互作用的电磁场都是比较弱的,因此原子对光子的吸收和发射和其它过程——如原子在外加电场、磁场下的变化——都没有耦合。
这意味着原子光谱提供了一种非常好的、不受其它实验手段影响的检查原子内部能级发生了什么变动的方式。
将原子置于外场中而产生的各种效应基本上都可以通过光谱体现出来。

\subsection{磁矩和磁场作用}

到目前为止的讨论,磁量子数$m_l$都是能量简并的,而如果加入一个磁场,那么就会有一个特定的空间方向,从而破缺$m_l$简并,导致能级进一步分裂。

\subsubsection{磁矩}

一些系统在外加静磁场时能量会增加一项
\[
    E_\text{M} = - \vb*{\mu} \cdot \vb*{B},
\]
其中的矢量$\vb*{\mu}$就称为磁矩。
磁矩和电荷的周期性运动具有非常密切的关系。一个没有内部结构的电荷做周期性运动相当于产生了一个环状电流,因此会产生一个磁矩,称为\concept{轨道磁矩}。
首先采用经典理论分析轨道磁矩。电子轨道角动量的公式为
\[
    \vb*{L} = m_\text{e} \vb*{r} \times \dv{\vb*{r}}{t} = 2 m_\text{e} \dv{\vb*{S}}{t},
\]
电子被束缚在原子核周围时做平面周期性运动,这样它就产生了一个大小为
\[
    I = - \frac{e}{\tau}
\]
的电流,其中$\tau$是运动周期。
在电磁学中,一个电流为$I$,围绕的面积为$\vb*{S}$的平面线圈的磁矩为
\[
    \vb*{\mu} = I \vb*{S} = - \frac{e \vb*{S}}{\tau},
\]
而由于电子做周期性运动,由角动量守恒我们有
\[
    \dv{\vb*{S}}{t} = \frac{\vb*{S}}{\tau},
\]
这样轨道磁矩就是
\[
    \vb*{\mu} = - \frac{e}{2m_\text{e}} \vb*{L}.
\]
负号的出现是因为电子携带负电荷,后面推导核子的磁矩时就没有这个负号。
为了与原子物理的背景保持一致,引入下标$l$表示这是来自轨道角动量的磁矩,并且设
\begin{equation}
    \mu_\text{B} = \frac{e\hbar}{2m_\text{e}}
\end{equation}
称为\concept{玻尔磁子},于是
\begin{equation}
    \vb*{\mu}_l = - \frac{\mu_\text{B}}{\hbar} \vb*{L}.
    \label{eq:orbit-magnetic-moment}
\end{equation}
虽然\eqref{eq:orbit-magnetic-moment}是在经典力学中导出的,但它也适用于量子理论。

量子理论中还有自旋角动量,这是不是会引入自旋磁矩?确实会,不过自旋磁矩的值和把电子当成带电小球计算出来的值并不相同。实际上,自旋磁矩是
\begin{equation}
    \vb*{\mu}_s = - \frac{e}{m_\text{e}} \vb*{S}.
\end{equation}
当然这也不奇怪,因为自旋在粒子图像中并没有经典对应。实际上自旋磁矩的严格计算直接来自QED。

总之,原子的总磁矩为
\begin{equation}
    \hat{\vb*{\mu}} = - \frac{\mu_\text{B}}{\hbar} (\vb*{L} + 2\vb*{S}),
\end{equation}
而对应的哈密顿量为
\begin{equation}
    \hat{H}_\text{mag} = - \hat{\vb*{\mu}} \cdot \vb*{B}.
    \label{eq:magnetic-hamiltonian}
\end{equation}

\subsubsection{半经典图像}

轨道角动量有明确的经典意义,可以使用半经典理论描述它。例如如果角动量的长度和$z$轴分量保持不变,那么就会发生\concept{拉莫尔进动},即角动量矢量在一个对称轴就是$z$轴的锥面上运动,长度保持不变。

磁矩对电子运动的影响无非是让电子受力(即破缺平移不变性)或是受力矩(即破缺旋转不变性)。
如果磁场是均匀的,那么电子肯定不会受力,但会受到一个力矩,因为磁矩的方向的变动会让$\vb*{\mu} \cdot \vb*{B}$发生变化,即
\[
    \pdv{(\vb*{\mu} \cdot \vb*{B})}{\vb*{\varphi}} \neq 0,
\]
但是电子位置的变动当然不会让$\vb*{\mu}$发生任何变化。
如果磁场是不均匀的,那么磁矩不仅受到力矩还受力,因为空间平移不变性被破缺了,或者说
\[
    \pdv{(\vb*{\mu} \cdot \vb*{B})}{\vb*{r}} \neq 0.
\]

\subsection{相对论修正}

本节讨论相对论修正导致的能级的小幅变化。这会导致环绕原子核的库伦势场中的电子的能级发生小的分裂。

\subsubsection{精细结构}

本节讨论\concept{精细结构},即相对论修正中最大的两个项。它们分别来自相对论动能和自旋-轨道耦合。
分裂出的两个能级很难发生彼此之间的跃迁,因为两者的角量子数和磁量子数都一样,因此两者之间的跃迁违背选择定则。

首先,我们知道,相对论情况下动能为
\[
    E_\text{k} = \frac{m c^2}{\sqrt{1 - v^2 / c^2}},
\]
用动量表示出来,展开到第二阶,得到
\begin{equation}
    E_\text{k} = \frac{p^2}{2m} - \frac{p^4}{8 m^3 c^2} + \cdots.
\end{equation}
第一项当然就是经典动能,第二项给出了一个微扰。注意到这个微扰仍然具有全部的旋转不变性,因此角动量平方和$z$方向角动量在它之下仍然守恒。
这样虽然类氢原子的波函数有简并,在$\psi_{nlm}$下仍然可以把它们当成非简并的。
一阶微扰为
\begin{equation}
    E^{\text{kin},(1)}_{n, l} = - \frac{(E^{(0)}_n)^2}{2 m c^2} \left( \frac{4 n}{l + 1/2} - 3 \right).
    \label{eq:kinetic-energy-relativity-correction}
\end{equation}
$l$的简并解除了——本该如此,实际上$l$会有简并单纯是库伦场的额外对称性的结果。这大约是$E_n^{(0)}$的$10^{-5}$量级。
$m$仍然有简并,从而对不同的$m$对应的波函数做线性组合,得到的仍然是能量本征态。这件事很重要,因为我们马上要引入一个不能保持$m$守恒的修正。

即使没有外加磁场,电子的自旋和轨道角动量仍然会出现小的耦合。这是相对论效应的结果。本节将给出对这一现象的一个半经典讨论。
为方便起见,以下称相对于原子实静止的参考系为$O$系,相对于某一时刻的电子静止的参考系(这仍然是一个惯性系,因为它并不是每一时刻都和电子保持静止)为$e$系。
设电子在$O$系中运行速度为$\vb*{v}$,$O$系中原子实施加给电子一个静电场$\vb*{E}_0$,则$e$系中$\vb*{E}_0$将变换成如下磁场:
\[
    \vb*{B}' = \frac{\vb*{E}_0 \times \vb*{v}}{\sqrt{c^2 - v^2}},
\]
由于$\vb*{v}$相对于光速很小,有
\begin{equation}
    \vb*{B}' = \frac{1}{c^2} \vb*{E}_0 \times \vb*{v}.
\end{equation}
$e$系中任何一个角动量的进动都是% TODO:经典力学
\[
    \vb*{\omega}' = \frac{e}{m_\text{e}} \vb*{B}',
\]
而在$O$系中$e$系的坐标轴以
\[
    \vb*{\omega}_T = - \frac{e}{2 m_\text{e}} \vb*{B}'
\]
的角速度进动,因此最后$O$系中任何一个角动量都在以
\begin{equation}
    \vb*{\omega} = \frac{e}{2 m_\text{e}} \vb*{B}'
\end{equation}
的角速度进动。这等价于$O$系中多出来了一个磁场,% 这也太扯了。。我觉得比较好的推导是,证明$e$系中哈密顿量会多出来一项,然后这一项和$O$系是相同的,当然哈密顿量未必是洛伦兹标量,所以也很麻烦。。。
因此需要在哈密顿量中引入一项
\begin{equation}
    \hat{H}_{LS} = - \vb*{\mu} \cdot \vb*{B}_\text{eff} = \frac{1}{2c^2} \frac{e}{m} \vb*{S} \cdot (\vb*{E}_0 \times \vb*{v}).
\end{equation}
如果是类氢原子,有
\begin{equation}
    \hat{H}_{LS} = \frac{Ze^2}{8 \pi \epsilon_0 c^2 m^2} \frac{\vb*{S} \cdot \vb*{L}}{r^3}.
    \label{eq:spin-ortibal-coupling}
\end{equation}
因此轨道角动量和自旋角动量是有耦合的,即所谓\concept{自旋-轨道耦合},两者同向时能量较高,两者反向时能量较低。
这同样会带来一个能级分裂。由于
\[
    \vb*{S} \cdot \vb*{L} = \frac{1}{2} (J^2 - L^2 - S^2),
\]
应该使用做了L-S耦合(详情见多电子系统的一般讨论\autoref{sec:ls-coupling},这里就是将轨道角动量和自旋角动量做了一个复合)的波函数$\psi_{njl m_j}$,请注意这族波函数也是相对论性动能修正下的能量本征态。
计算得到
\begin{equation}
    E^{\text{LS}, (1)}_{njl} = \frac{(E^{(0)}_n)^2}{mc^2} \frac{2n (j(j+1) - l(l+1) - 3/4)}{l(l+1)(2l+1)}.
\end{equation}
于是最后能级修正为
\begin{equation}
    E^{(1)}_{nj} = \frac{(E^{(0)}_n)^2}{mc^2} \left( \frac{3}{2} - \frac{4 n}{2 j + 1} \right).
\end{equation}
引入无量纲的\concept{精细结构常数}
\begin{equation}
    \alpha = \frac{e^2}{4\pi \epsilon_0 \hbar c} \approx \frac{1}{137},
\end{equation}
做了一阶修正之后的能级为
\begin{equation}
    E_{n j} = \frac{E_1^{(0)}}{n^2} \left( 1 + \left( \frac{Z \alpha}{n} \right)^2 \left(  \frac{2 n}{2 j + 1} - \frac{3}{4} \right) \right).
\end{equation}

\subsubsection{更精细的物理}

% TODO

实际上,以上讨论还是不能够覆盖所有的物理现象。核磁矩的存在、电四极辐射以及其它一些机制会导致能级进一步分裂,产生\concept{超精细结构}。
这些效应都远远小于精细结构,如核磁矩相比电子磁矩是很小的,

使用狄拉克方程,因为兰姆位移。

\section{多电子系统}

多电子处于同一系统时会产生更多有趣的结果。由于电子是费米子,体系的波函数(同时包括轨道部分和自旋部分)一定是交换反对称的。
如果不考虑轨道-自旋耦合,为了保证反对称性,轨道部分对称则自旋部分反对称;轨道部分反对称则自旋部分对称。

\subsection{双电子原子}

\subsubsection{反对称化}

首先考虑一个双电子原子。如果两个电子的轨道运动相同,那么轨道部分的波函数一定是对称的(如果是反对称的就变成零了),那么自旋部分的波函数一定是反对称的,并且两个电子的自旋一定不相同(否则所有状态都相同,违反泡利不相容原理)。
这样,自旋部分的波函数就是
\begin{equation}
    \chi = \frac{1}{\sqrt{2}} (\chi_{\uparrow 1} \chi_{\downarrow 2} - \chi_{\uparrow 2} \chi_{\downarrow 1}).
    \label{eq:asym-spin}
\end{equation}
不需要其它任何条件,自旋部分的波函数就完全确定了(可以差一个因子但这无关紧要)。
因此轨道部分相同的两个电子的自旋角动量代数是单态的,即$s=0, m_s=0$。

如果两个电子的轨道运动不同,那么轨道部分的波函数可以是对称的也可以是反对称的。
假定它是对称的,那么自旋部分的波函数一定反对称,因此两个电子的自旋不可能相等。这就意味着自旋部分的波函数还是\eqref{eq:asym-spin}。
而如果轨道部分的波函数是反对称的,那么自旋部分的波函数是对称的。下面我们考虑自旋本征态。
两个电子的自旋如果相等,那么自旋波函数就是以下二者之一:
\[
    \chi_{\uparrow 1} \chi_{\uparrow 2}, \quad \chi_{\downarrow 1} \chi_{\downarrow 2}.
\]
自旋波函数当然也可以是两个不同的自旋的线性组合,即它是$\chi_{\uparrow 1} \chi_{\downarrow 2}$及其交换的线性组合,并且满足对称条件,从而为
\[
    \chi = \frac{1}{\sqrt{2}} (\chi_{\uparrow 1} \chi_{\downarrow 2} + \chi_{\uparrow 2} \chi_{\downarrow 1}).
\]
因此自旋本征态为
\begin{equation}
    \chi = \chi_{\uparrow 1} \chi_{\uparrow 2}, \quad \chi_{\downarrow 1} \chi_{\downarrow 2}, \quad \frac{1}{\sqrt{2}} (\chi_{\uparrow 1} \chi_{\downarrow 2} + \chi_{\uparrow 2} \chi_{\downarrow 1}).
    \label{eq:sym-spin}
\end{equation}
\eqref{eq:sym-spin}是一个三重态,$s=1, m_s=0, \pm 1$。

从角动量代数的角度,两个电子放在一起,它们的角动量代数的复合是两个$s=1/2$的角动量代数的复合,所得结果的$s$取值范围为$0, 1$,和刚才推导得到的一致。
具体$s$取多少由轨道波函数的情况决定。轨道部分如果对称,那么自旋部分必须反对称%
\footnote{注意这是交换对称不是空间对称,空间对称由宇称描述。}%
,这对应$s=0$;反之,轨道部分反对称,则自旋部分必须对称,对应$s=1$。

\subsubsection{交换能}

双电子原子的波函数必须满足交换对称或者反对称条件还意味着,电子之间的库伦能也会发生改变。
将两个电子之间的库伦相互作用看成微扰,计算该微扰造成的能量本征值变化的一阶修正,就是
\begin{equation}
    E = \int \dd[3]{\vb*{r}_1} \dd[3]{\vb*{r}_2} \psi^*(\vb*{r}_1, \vb*{r}_2) \frac{1}{4\pi \epsilon_0} \frac{e^2}{\abs{\vb*{r}_1 - \vb*{r}_2}} \psi(\vb*{r}_1, \vb*{r}_2).
\end{equation}
对对称态或者反对称态
\[
    \psi(\vb*{r}_1, \vb*{r}_2) = \frac{1}{\sqrt{2}} (\psi_1(\vb*{r}_1) \psi_2(\vb*{r}_2) \pm \psi_2(\vb*{r}_1) \psi_1(\vb*{r}_2)),
\]
我们有
\begin{equation}
    \begin{aligned}
        E &= \int \dd[3]{\vb*{r}_1} \dd[3]{\vb*{r}_2} \psi_1^*(\vb*{r}_1) \psi_1(\vb*{r}_1) \frac{1}{4\pi \epsilon_0} \frac{e^2}{\abs{\vb*{r}_1 - \vb*{r}_2}} \psi_2^*(\vb*{r}_2) \psi_2(\vb*{r}_2) \\
        &\pm \int \dd[3]{\vb*{r}_1} \dd[3]{\vb*{r}_2} \psi_2^*(\vb*{r}_1) \psi_1(\vb*{r}_1) \frac{1}{4\pi \epsilon_0} \frac{e^2}{\abs{\vb*{r}_1 - \vb*{r}_2}} \psi_1^*(\vb*{r}_2) \psi_2(\vb*{r}_2).
    \end{aligned}
\end{equation}
等式右边第一项就是将电子云密度看成电荷密度计算出来的库伦能,第二项则是一个没有经典对应的项,称为\concept{交换能}。
可以看到在两个电子的波函数没有很大重叠时交换能可以略去,这也是合理的。

交换能意味着两个电子的自旋角动量发生了耦合。要看出这是为什么,定义
\begin{equation}
    J = \int \dd[3]{\vb*{r}_1} \dd[3]{\vb*{r}_2} \psi_2^*(\vb*{r}_1) \psi_1(\vb*{r}_1) \frac{1}{4\pi \epsilon_0} \frac{e^2}{\abs{\vb*{r}_1 - \vb*{r}_2}} \psi_1^*(\vb*{r}_2) \psi_2(\vb*{r}_2),
\end{equation}
若$l=1$则轨道部分反对称,交换能为$-J$,若$l=0$则轨道部分对称,交换能为$J$。这样,交换能在哈密顿量中就引入这样一项:
\[
    \hat{V}_\text{ex} = -J ( \dyad{\uparrow \uparrow} + \dyad{\downarrow \downarrow} + \frac{1}{2} (\ket{\uparrow \downarrow} + \ket{\downarrow \uparrow}) (\bra{\uparrow \downarrow} + \bra{\downarrow \uparrow}) ) + J \frac{1}{2} (\ket{\uparrow \downarrow} - \ket{\downarrow \uparrow}) (\bra{\uparrow \downarrow} - \bra{\downarrow \uparrow}),
\]
可以验证,这实际上就是
\begin{equation}
    \hat{V}_\text{ex} = - \frac{1}{2} J (1 + 4 \hat{\vb*{s}}_1 \cdot \hat{\vb*{s}}_2),
\end{equation}
现在我们看出,交换能实际上会让电子的自旋倾向于趋于一致。

实际上,以上推导没有用到任何关于库伦相互作用的信息,因此可以看出,只要两个电子之间有相互作用,就肯定会有交换相互作用。

\subsection{中心场近似下的多电子原子}

本节中单电子物理量用小写,整个原子的物理量(通常就是对应的单电子物理量之和)用大写。

\subsubsection{中心场近似}\label{sec:centric-field}

将一个电子受到其它电子的作用看成一个平均场,即认为原子核受到的屏蔽作用是固定不变的。对称性分析表明这个平均场一定是一个有心力场$S(r)$,因此称之为\concept{中心场}。
除了中心场以外的相互作用称为\concept{剩余相互作用}。

在哈密顿量的势能项当中加入中心场之后,会发现能量和角量子数有关,这是因为$n$相同$l$不同的原子径向分布不同,因此受到的屏蔽也不同。角动量大的电子近核概率小,屏蔽效应强,能量高。
$n, l$完全决定了波函数的径向部分。(见\autoref{sec:quantum-number})

原子中所有电子在单电子能级上的分布情况称为\concept{电子组态},它给出了全部电子的能级的组合,也即,给出了$n$和$l$的组合。$n$和$l$相同的电子称为\concept{同科电子}。
不需要知道完整的电子状态就可以得到电子组态。
电子组态可以使用标准的spdf记号给出。
我们称不同的主量子数对应的全部电子组成一个\concept{壳层}。$n=1, 2, 3, \ldots$对应着K,L,M,N,O,P等壳层,但是现在很少用这些字母符号了。
在每一个壳层内部,不同的角量子数$l$给出不同的\concept{支壳层}。
在中心场近似下,每个支壳层内部的电子能量都是一样的。
支壳层内部的$m_l$可以发生变化,每个$m_l$给出一个\concept{轨道},每个轨道容纳自旋不同的两个电子。%
\footnote{当然,这是以$n,l,m_l,m_s$为好量子数之后的半经典叙述。实际上电子可以处于这组表象下的叠加态。}%
$l$支壳层有$2(2l+1)$个电子,即$2l+1$个轨道,$2$个自旋,因此$n$壳层有
\[
    \sum_{l=0}^{n-1} 2(2l+1) = 2n^2
\]
个电子。
外层电子的能量主要由$l$决定;这就导致了所谓的能级交错现象,即主量子数小的支壳层如果角量子数适当能量反而比较大。\concept{洪特规则}给出了不同支壳层能量的大小顺序。

以上图景可以解释一些实验中观察到的现象。
首先是\concept{原子幻数},即
\[
    Z=2, 10, 18, 36, 54, 86, \ldots
\]
时第一电离能位于峰值,然后一下子到达谷值。原子序数为原子幻数的元素即为稀有气体。
稀有气体非常稳定是因为它有电子的最高的能级都是$n$p支壳层,且全满,而$n$p与$(n+1)$s有较大能隙,p支壳层全满的原子不容易激发;此外内满壳层电子云的电荷分布球对称,对价电子吸引强。
为什么碱金属容易电离是因为原子实中电荷均匀球对称分布,几乎就是一个单独的正电荷,因此价电子受到的束缚非常弱。
同理为什么卤素容易接受电子是因为容易失去一个空穴。

\[
    \sum_{m=-l}^l \abs{Y_{lm}(\theta, \varphi)}^2 = \frac{2l+1}{4\pi},
\]
从而
\[
    \rho(\vb*{r}) = -2(2l+1) \frac{e}{4\pi} \chi_{nl}(\vb*{r})
\]

\subsubsection{化学反应}

吸能:失去电子

放能:得到电子、正负电荷中心接近从而降低库伦能

\subsubsection{原子态和光谱}

由于能量和$m_l$、$m_s$无关,如果一个支壳层非空而非全满,那么就有能量简并。$C_{2(2l+1)}^N$

\subsubsection{角动量的合成}

一个很自然的问题是,如何从不同电子的角动量代数推导出整个原子的角动量代数。
这样做的目的至少有两个,首先,在不涉及原子失去或者得到电子的情况下,只需要将原子作为整体讨论其角动量就可以;其次,在中心场近似的基础上将电子之间的相互作用作为微扰引入时,需要使用角动量的合成来获得一组基(见\autoref{sec:ls-coupling}和\autoref{sec:jj-coupling}),这组基中每一个的标签(如总角动量等)在加入微扰前后都是守恒量,这样可以省去简并微扰论需要的繁琐的行列式计算。

角动量代数的合成实际上就是将$SO(3)$的表示的直积(显然也是$SO(3)$的某个表示)分解成一系列不可约表示的直和的过程。
我们已经在\autoref{sec:algebra-of-angular}中讨论了其不可约表示的结构,现在需要分析怎么做分解。

首先讨论两个角动量代数的合成。设我们有算符$\hat{\vb*{J}}_1$和$\hat{\vb*{J}}_2$,它们各自的角量子数和磁量子数为$j_1, m_1$和$j_2, m_2$(略去了$m_j$的下标$j$)。
简单地将两个空间直积起来,可以得到一组正交归一化基
\begin{equation}
    \ket{j_1 j_2 m_1 m_2} = \ket{j_1 m_1} \otimes \ket{j_2 m_2}.
\end{equation}
下面假定$j_1$和$j_2$已经给定,也就是说被合成的是两个不可约有限维表示。这个假设是不失一般性的,因为如果需要合成两个可约表示,总是可以把它们包含的不可约表示分别合成。
这样我们有
\[
    \sum_{m_1,m_2} \dyad{j_1 j_2 m_1 m_2} = 1.
\]
设合成之后的角量子数为$j$(取值不再仅限于一个),磁量子数为$m$,则我们要做的就是计算出$\braket{jm}{j_1 j_2 m_1 m_2}$,它们称为\concept{CG系数}。

首先注意到由于
\[
    \hat{J}_z = \hat{J}_{z1} + \hat{J}_{z2},
\]
可以得到
\[
    m \hbar \ket{jm} = \sum_{m_1, m_2} (m_1 + m_2) \hbar \ket{j_1 j_2 m_1 m_2} \braket{j_1 j_2 m_1 m_2}{j m}, 
\]
从而
\[
    (m - m_1 - m_2) \braket{j_1 j_2 m_1 m_2}{j m} = 0, 
\]
或者也可以写成
\begin{equation}
    \braket{j_1 j_2 m_1 m_2}{j m} = \braket{j_1 j_2 m_1 (m - m_1)}{j m} \delta_{m, m_1 + m_2}.
    \label{eq:m-is-the-sum-of-m1-and-m2}
\end{equation}
换而言之,合成之后一定有
\begin{equation}
    m = m_1 + m_2.
\end{equation}
这当然是角动量叠加的定义导致的结果。这又意味着,让$m_1$和$m_2$扫过它们可以取的值,会得到以下结果:
\begin{itemize}
    \item $m$扫过$-(j_1 + j_2)$到$j_1 + j_2$;
    \item $m$扫过$-(j_1 + j_2 + 1)$到$j_1 + j_2 + 1$;
    \item ……
    \item $m$扫过$-\abs*{j_1 - j_2}$到$\abs*{j_1 - j_2}$。
\end{itemize}
这意味着合成之后得到的角动量代数实际上是
\begin{equation}
    j = \abs*{j_1 - j_2}, \abs*{j_1 - j_2} + 1, \ldots, j_1 + j_2 - 1, j_1 + j_2
\end{equation}
的不可约表示的直和。

\eqref{eq:m-is-the-sum-of-m1-and-m2}意味着
\[
    \ket{jm} = \sum_{\max(-j_1, m-j_2) \leq m_1 \leq \min(j_1, m+j_2)} \ket{j_1 j_2 m_1 (m - m_1)} \braket{j_1 j_2 m_1 (m - m_1)}{jm}.
\]
不断将升降算符作用在上式左右,就能够计算出CG系数。

如果被合成的若干个角动量代数的$j$都是相同的,

如果需要将两个可约表示做合成,由于不同的$j_1$和$j_2$对应的不可约表示可以合成得到具有相同的$j$的表示,需要使用$j_1$和$j_2$区分这些表示。
这样,合成之后的好量子数为$j_1, j_2, j, m$。
可以重复以上步骤将$n$个可约表示做合成,得到的好量子数为$\{j_i\} j m$。

需要格外注意的是,无论怎么将不同的电子的角动量代数做合成,最后得到的本征态必须整体是反对称的,而以上给出的合成方法不能够保证这一点。
如果参与合成的电子全部不同科,那么不会有任何问题,因为能够保证所有电子的单电子态矢量都是不一样的,因此可以随意做对称化和反对称化;
如果参与合成的电子有同科电子,那么由泡利不相容原理,同科电子磁量子数相同必定意味着轨道部分是对称的,从而自旋部分必须是反对称的。

% 偶数定则:交换两个电子会产生一个$(-1)^{l_1 + l_2 - l}$的因子

\subsection{相互作用带来的能量修正}

中心场近似中各个电子之间没有相互作用。实际上,各个电子之间有两种主要的相互作用。
其一是\concept{剩余相互作用}$\hat{H}_1$,即中心场以外的电子间剩余库伦作用(交换能、关联能等全部被收入这一项),其二是自旋轨道耦合$\hat{H}_2$。
这两者分别可以写成
\begin{equation}
    \hat{H}_1 = \frac{1}{2} \sum_{\vb*{r}_1, \vb*{r}_2} \int \dd[3]{\vb*{r}_1} \dd[3]{\vb*{r}_2} \frac{1}{4 \pi \epsilon_0} \frac{e^2}{\abs{\vb*{r}_1 - \vb*{r}_2}} - S(r),
\end{equation}
以及
\begin{equation}
    \hat{H}_2 = \sum_{i=1}^N \xi(r_i) \vb*{l}_i \cdot \vb*{s}_i.
    \label{eq:spin-orbital-many}
\end{equation}
上式中$\xi$的宗量为$\vb*{r}$的模长,这是空间各向同性的推论。

可以估计出
\[
    H_1 \sim Z, \quad H_2 \sim Z^4,
\]
且电子相距越近,剩余相互作用越明显。下面我们会发现,如果剩余相互作用远大于自旋轨道相互作用,则会导致L-S耦合,反之会导致j-j耦合,因此大部分元素的基态、轻元素的低激发态适用L-S耦合,重元素的激发态适用j-j耦合。
在$\hat{H}_1$和$\hat{H}_2$同阶时,会出现中间耦合,这适用于轻元素的高激发态和中等元素的激发态。

\subsubsection{L-S耦合}\label{sec:ls-coupling}

% TODO:同科电子的合成。在将轨道角动量和自旋角动量合成成总角动量时,如果不涉及同科电子,那么只需要放心大胆地根据角动量代数的合成法则做就可以,因为总角动量算符的每一个基矢量都可以做反对称化而不会变成零。但有同科电子时,必须把违反全同性的那些态排除掉。。TODO

如果剩余相互作用的修正远大于自旋-轨道相互作用,即
\begin{equation}
    H_2 \ll H_1 \ll H_0,
\end{equation}
就首先考虑$\hat{H}_1$的作用。此时两个电子的自旋角动量之间存在耦合,从而不同电子的轨道角动量之间也存在耦合,但是自旋角动量和轨道角动量之间尚无耦合,这称为\concept{L-S耦合}。
这样,在加入了$\hat{H}_1$之后的好量子数由$\hat{H}_0, \hat{L}^2, \hat{L}_z, \hat{S}^2, \hat{S}_z$给出,它们具有共同本征函数系。
于是把$\{n_i l_i m_{li} m_{si}\}$表象线性变换成$\{n_i l_i\} L M_L S M_S$表象,$\hat{H}_1$带来的修正在这一组表象下应当是对角的。%
\footnote{由于多个电子角动量合成之后的角动量代数是可约的,即使每个电子的$l_i$都知道了,还是必须明确给出$L$和$S$,从经典图像上说,这是因为我们不知道各个电子的角动量的指向如何。}%
这样,把$\hat{H}_1$当成微扰项,那么能量的一阶修正为
\begin{equation}
    E^{\text{rem}, (1)}_{\{n_i l_i\} LS} = \sum_{s_{z1}, s_{z2}, \ldots, s_{zN}} \int \prod \dd[3]{\vb*{r}_i} \Psi^{(0) *}_{\{n_i l_i\} L M_L S M_S}(\{\vb*{r}_i\}) \hat{H}_1 \Psi^{(0)}_{\{n_i l_i\} L M_L S M_S}(\{\vb*{r}_i\}).
    \label{eq:ls-1}
\end{equation}
中心场近似的能量$E^{(0)}_{\{n_i l_i\}}$仅仅和原子组态有关;实际上,$E^{(1)}_{LS}$只和原子组态以及$L,S$有关。
这是系统的对称性决定的,加入$\hat{H}_1$之后系统同时具有自旋旋转不变性(剩余相互作用中的交换能部分)和轨道旋转不变性(剩余相互作用中的库伦能部分),因此$M_L$和$M_S$对能量没有影响。
这样在只考虑剩余相互作用时,能级简并度为
\begin{equation}
    g_{\{n_i l_i\}LS} = (2L+1)(2S+1).
\end{equation}
从经典图景的角度说,这是因为剩余库伦能和剩余交换能分别由电子的相对角分布和自旋角动量的夹角决定,至于它们绝对地指向什么方向,并不重要。
在经典图景下各个$\{\vb*{l}_i\}$绕着$\vb*{L}$进动,为了把所有量确定下来要知道$\{\vb*{l}_i\}$的大小、$\vb*{L}$的大小和方向;自旋角动量同理。

现在在L-S耦合的系统中再引入$\hat{H}_2$,此时$L_z$和$S_z$也不再是守恒的了,但由于$\hat{H}_2$只是让角动量在自旋和轨道两种形式之间转换,总角动量在$z$轴的分量$J_z$还是守恒的,且$J^2$也是守恒的。同样$L^2$和$S^2$也是守恒的,这样好量子数是$\hat{H}_0, \hat{L}^2, \hat{S}^2, \hat{J}^2, \hat{J}_z$。
这样需要把$\{n_i l_i\} L M_L S M_S$表象再次变换成$\{n_i l_i\} L S J M_J$表象(这组表象由于$L,S$确定,\eqref{eq:ls-1}还是适用的),$\hat{H}_2$引入的修正在这一组表象下是对角的,为
\begin{equation}
    \begin{aligned}
        E^{\text{soi}, (1)}_{\{n_i l_i\} LSJ} &= \sum_{s_{z1}, s_{z2}, \ldots, s_{zN}} \int \prod \dd[3]{\vb*{r}_i} \Psi^{(0) *}_{\{n_i l_i\} L S J M_J}(\{\vb*{r}_i\}) \hat{H}_2 \Psi^{(0)}_{\{n_i l_i\} L S J M_J}(\{\vb*{r}_i\}) \\
        &= \frac{\zeta_{\{n_i l_i\}LS}}{2} \hbar^2 (J(J+1) - L(L+1) - S(S+1)).
        \label{eq:ls-2}
    \end{aligned}
\end{equation}
第二个等号看起来很奇怪不过下面马上会解释它。
由于空间旋转对称性,上式和$M_J$无关。这样,完整地考虑了两种相互作用带来的修正之后,能量修正为
\begin{equation}
    E^{\text{LS}, (1)}_{\{n_i l_i\} LSJ} = E^{\text{rem}, (1)}_{\{n_i l_i\} LS} + E^{\text{soi}, (1)}_{\{n_i l_i\} LSJ},
\end{equation}
它只和$L,S,J$有关,和$M_J$无关,于是能级简并度为
\begin{equation}
    g_{\{n_i l_i\}LSJ} = 2J + 1.
\end{equation}
从经典图景看,各个电子的$\vb*{l}_i$绕着$\vb*{L}$进动,$\vb*{s}_i$绕着$\vb*{S}$进动,而$\vb*{L}$和$\vb*{S}$又绕着$\vb*{J}$进动,由于$\hat{H}_1$远大于$\hat{H}_2$,前者远远快于后者。
这个经典图景有助于理解\eqref{eq:ls-2}的形式:由于$\vb*{l}_i$和$\vb*{s}_i$的进动非常快,可以做近似
\[
    H_2 \approx \bar{H}_2,
\]
从而
\[
    H_2 = \expval{\sum_{i=1}^N \left(\vb*{l}_i \cdot \frac{\vb*{L}}{L} \right) \frac{\vb*{L}}{L} \cdot \left(\vb*{s}_i \cdot \frac{\vb*{S}}{S} \right) \frac{\vb*{S}}{S} } = \zeta_{\{n_i l_i\}LS} \vb*{L} \cdot \vb*{S} = \frac{\zeta_{\{n_i l_i\}LS}}{2} \hbar^2 (J^2 - L^2 - S^2),
\]
这就导致了\eqref{eq:ls-2}的形式。$\zeta_{\{n_i l_i\}LS}$是$\vb*{l}_i$和$\vb*{s}_i$导致的比例系数,它在尚未半满的情况下大于零,否则小于零。

总之,L-S耦合中,能级分裂情况如下面的列表所示:
\begin{enumerate}
    \item 忽略电子间库仑相互作用,得到类氢原子近似,能量完全由主量子数$n$确定;
    \item 引入中心场近似,能量由$n, l$决定,出现能级交错,即具有较低的主量子数的能级可以高于具有较高的主量子数的能级,可以解释稀有气体、碱金属和卤素的性质;
    \item 考虑剩余相互作用,相同组态的能量发生分裂,不同的$L$和$S$之间能量不同,例如各个电子自旋同向的状态能量会低一些,于是出现前两个洪特规则:$S$越大越稳定,即电子倾向于尽可能地取朝向一致的自旋(或者说$S$较大的状态较稳定),这又意味着半满或全满的轨道比较稳定;在$S$确定的情况下,$L$越大越稳定,因为这意味着不同电子的电子云重叠得更加少;
    \item 考虑轨道自旋耦合,不同的$J$会带来不同能量,这就是类氢原子的精细结构在多电子原子中的对应,也是最后一个洪特规则,具体来说,在$S$和$L$也相同的情况下,对尚未半满的情况,$J$越大越稳定;否则$J$越小越稳定。
\end{enumerate}
这样,对一个组态已知的原子系统,$L, S, J$完全决定其能量,记这样的能级(或者说\concept{谱项})为$^{2S+1} L_J$,$2S+1$称为\concept{自旋多重度},因为它给出了自旋量子数$M_S$的取值个数。
将所有的$\{l_i\}$合成可以求出$L$的取值范围,将所有的$\{s_i\}$合成可以求出$S$的取值范围,将$L$和$S$合成又可以得到$J$的取值范围。
需要注意的是不是所有的$\{s_i\}$都是可能的,因为必须保持多电子波函数反对称。
精细结构的能级差为
\begin{equation}
    E_{\{n_i l_i\} LSJ} - E_{\{n_i l_i\} LS(J-1)} = J \zeta_{\{n_i l_i\}} \hbar^2.
    \label{eq:lande-gapping}
\end{equation}
这就是\concept{朗德间隔定则}:如果原子遵循L-S耦合则它成立,如果原子不遵循L-S耦合,它通常会被违反。

使用所有$\{l_i\}$合成出$L$是非常繁琐的。
在合成时,我们总是可以首先把一个满支壳层中的所有电子的角动量代数合成,而单个电子无论是磁量子数还是自旋量子数都可以跑遍$\pm l$或$\pm \frac{1}{2}$中的全部整数或半整数值,它们求和会得到零,因此一个满支壳层中所有电子的角动量合成之后,$M_L$和$M_S$一定是零,从而一个满支壳层的$L$和$S$只能是零。因此实际上满支壳层对整个原子的$L$和$S$是没有贡献的。
这个结论还意味着,当要耦合的电子数较大时,可以把它们看成“从满的支壳层拿掉少数几个电子”,由于空穴的角动量代数和电子完全一样,这样得到的角动量代数也是正确的。
换而言之,电子组态$nl^x$和$nl^{2(2l+1)-x}$的谱项是完全一样的;但需要注意这两种电子组态中,固定$L,S$并变动$J$时各个谱项的能级顺序是相反的,因为$\zeta_{\{n_i l_i\}LS}$会变号。

\subsubsection{j-j耦合}\label{sec:jj-coupling}

如果自旋-轨道相互作用的修正远大于剩余相互作用,即
\begin{equation}
    H_1 \ll H_2 \ll H_0,
\end{equation}
就需要首先考虑$\hat{H}_2$,此时每个电子的自旋角动量和轨道角动量有很强的耦合,但是不同电子的总角动量之间并没有很强的耦合,这称为\concept{j-j耦合}。
在加入$\hat{H}_2$之后,电子之间还是没有相互作用,因此我们可以只讨论单个电子的运动情况。
单个电子的哈密顿量包括两项,其一是中心场近似下的
\[
    \hat{h}_0 = \frac{\hat{p}^2}{2m_\text{e}} - \frac{Z e^2}{4\pi \epsilon_0 r} + S(\vb*{r}),
\]
其二是\eqref{eq:spin-orbital-many}导致的
\[
    \hat{h}_2 = \xi(\vb*{r}) \hat{\vb*{l}} \cdot \hat{\vb*{s}}.
\]
和上一节中讨论$\hat{H}_2$的作用时类似,一组好量子数由$\hat{h}_0, \hat{l}^2, \hat{s}^2, \hat{j}^2, \hat{j}_z$给出,于是从$nl m_l m_s$表象切换到$nlj m_j$表象,且记径向部分为$R_{nl}(\vb*{r})$(在中心场近似中,给定组态,各个电子的径向分布就给定了),则$\hat{h}_2$引入的能量修正为
\begin{equation}
    \begin{aligned}
        \epsilon_{nlj}^{\text{soi}, (1)} &= \sum_{s_z} \int \dd[3]{\vb*{r}} \psi^{(0)*}_{nlj m_j} \hat{h}_2 \psi^{(0)}_{nlj m_j} \\
        &= \frac{\xi_{nl}}{2} \hbar^2 (j(j+1) - l(l+1) - s(s+1)),
    \end{aligned}
\end{equation}
其中
\begin{equation}
    \xi_{nl} = \int_0^\infty \dd{r} (r R_{nl}(r))^2 \xi(r).
\end{equation}
由于电子间没有相互作用,这就意味着原子总能量为
\[
    E_{\{n_i l_i j_i\}} = \sum_{n, l, j} N_{nlj} \epsilon_{nlj},
\]
其中$N_{nlj}$为处于状态$(n, l, j)$的电子数目,由电子组态$\{n_i l_i\}$给定。这样$\hat{H}_2$对原子总能量的一阶修正就是
\begin{equation}
    E^{\text{soi}, (1)}_{\{n_i l_i j_i\}} = \sum_{n, l, j} N_{nlj} \epsilon_{nlj}^{\text{soi}, (1)}.
\end{equation}
能量和$m_j$没有任何关系,因此有能级简并。能级简并数为
\begin{equation}
    g_{\{n_i l_i j_i\}} = \prod_{n, l, j} C_{2j+1}^{N_{nlj}}.
\end{equation}
从经典图景看,每个电子的$\vb*{l}$和$\vb*{s}$绕着$\vb*{j}$进动。

现在再讨论剩余相互作用带来的修正。加入剩余相互作用之后,诸$\{m_{ji}\}$不再是好量子数,因为不同电子的角动量会有耦合,不过$\vb*{J}$还是守恒的,且$j^2$也还是守恒的,于是好量子数由$\hat{H}_0, \hat{J}^2, \hat{J}_z$给出。
从$\{n_i l_i j_i m_{ji}\}$表象(每个$i$对应一个单电子的$nl j m_j$表象)切换到$\{n_i l_i j_i\} J M_J$表象,$\hat{H}_1$带来的一阶修正为
\begin{equation}
    E^{\text{rem}, (1)}_{\{n_i l_i j_i\} J} = \sum_{s_{z1}, s_{z2}, \ldots, s_{zN}} \int \prod \dd[3]{\vb*{r}_i} \Psi^{(0) *}_{\{n_i l_i j_i\} J M_J}(\{\vb*{r}_i\}) \hat{H}_1 \Psi^{(0)}_{\{n_i l_i j_i\} J M_J}(\{\vb*{r}_i\}).
\end{equation}
同样空间旋转对称性意味着$M_J$对能量修正没有影响。于是同时考虑了两种相互作用带来的修正,我们有
\begin{equation}
    E^{\text{jj}, (1)}_{\{n_i l_i j_i\} J} = E^{\text{soi}, (1)}_{\{n_i l_i j_i\}} + E^{\text{rem}, (1)}_{\{n_i l_i j_i\} J},
\end{equation}
能量简并度为
\begin{equation}
    g_{\{n_i l_i j_i\}J} = 2J+1.
\end{equation}
从经典图景看,此时每个电子的$\vb*{l}$和$\vb*{s}$绕着$\vb*{j}$进动,各个$\vb*{j}_i$绕着$\vb*{J}$进动。前者明显快于后者。

j-j耦合中,能级分裂情况由以下列表所示:
\begin{enumerate}
    \item 忽略电子间库仑相互作用,得到类氢原子近似,能量完全由主量子数$n$确定;
    \item 引入中心场近似,能量由$n, l$决定,出现能级交错,可以解释稀有气体、碱金属和卤素的性质;
    \item 引入轨道自旋耦合,能量由$\{n_i l_i j_i\}$确定,这是类氢原子的剩余相互作用在多电子原子中的对应,这一步造成的能级分裂的能量大小顺序由$\{j_i\}$决定,$j_i$越大,说明方向相同的自旋角动量和轨道角动量越多,因此能量越高;
    \item 引入剩余相互作用,能量由$\{n_i l_i j_i\} J$确定,这一步中的能级分裂次序没有特别的规律。
\end{enumerate}

j-j耦合中原子组态已知时谱项表示为$(j_1, j_2, \ldots, j_N)_J$。
获得谱线的方式是先将$l$和$s$合成出$j$,然后再把各个$\{j_i\}$合成成$J$。
第一步是非常显然的,第二步则比较繁琐。实际上,此时满支壳层对$J$同样没有贡献。与L-S耦合类似,我们将一个满支壳层中的电子首先来做合成,则
\[
    M = \sum_{j=\abs{l-1/2}}^{l+1/2} \sum_{m_j=-j}^j m_j = 0,
\]
因此满支壳层的角动量代数的磁量子数唯一的取值是零,因此$J=0$,即这个角动量代数对整个原子的总角动量代数没有贡献。
实际上,使用类似的方法,可以证明满支壳层带来的自旋-轨道修正也是零,因为能量修正有的为正有的为负,加起来等于没修正。

\subsection{多电子原子的偶极辐射}

\subsubsection{选择定则}

设电偶极跃迁光子的总角动量为$\vb*{j}_\gamma$,显然有
\[
    \vb*{J} + \vb*{j}_\gamma = \vb*{J}'.
\]
对光子$j_\gamma=1$,则
\[
    \Delta J = \pm 1, 0, 
\]
而且$J$和$J'$不能同时为零,否则角动量不可能守恒。
同样,磁量子数的变化为
\[
    \Delta M_J = \pm 1, 0.
\]
同时我们还有宇称守恒,而光子具有奇宇称,而原子的宇称为
\[
    \Pi = (-1)^{\sum_{j} l_j},
\]
因此
\[
    \Delta L = \pm 1, \pm 3, \ldots,
\]
对低激发态,这意味着%
\footnote{通常只考虑低激发态的原因是,如果一份能量足够让原子的多个电子被激发,那也足够让单个电子被电离。
后者是更为常见的现象。}%
\[
    \Delta l_\text{trans} = \pm 1, \quad \Delta l_\text{other} = 0.
\]
无论如何,总角量子数发生变化意味着电偶极跃迁发生在不同组态的谱项之间。
同一组态中的谱项之间不能发生电偶极跃迁,因为轨道角动量没有发生变化,从而不满足宇称守恒。
这样就得到了任何一个多电子原子的偶极跃迁应遵循的选择定则:
\begin{equation}
    \Delta J = 0, \pm 1, \quad \Delta L = \pm 1, \pm 3, \ldots, \quad \Delta M_J = \pm 1, 0,
    \label{eq:many-electron-selective}
\end{equation}
且$J$不能从0变化到0。

除了以上规则,通过计算跃迁矩阵元,还可以发现一些选择定则(称为附加定则)。对L-S耦合,有
\begin{equation}
    \Delta S = 0, \quad \Delta L = \pm 1.
    \label{eq:l-s-selective}
\end{equation}
对j-j耦合,有
\begin{equation}
    \Delta j_\text{trans} = \pm 1, 0, \quad \Delta j_\text{other} = 0.
    \label{eq:j-j-selective}
\end{equation}

\subsubsection{类氢光谱}

考虑一个类氢原子(即氢原子或者碱金属原子),其基态价壳组态为$n$s。
考虑低激发态,即只有价电子跃迁。这样,角动量——无论是自旋还是轨道——完全来自价电子。

首先采用L-S耦合,则
\[
    L = l, \quad S = s = \frac{1}{2}, 
\]
而
\[
    J = j = \begin{cases}
        l \pm 1/2, &\quad l \neq 0, \\
        1/2, &\quad l = 0.
    \end{cases}
\]
应用L-S耦合的选择定则\eqref{eq:l-s-selective}和\eqref{eq:many-electron-selective},我们有
\[
    \Delta j = 0, \pm 1, \quad \Delta m_j = 0, \pm 1, \quad \Delta s = 0, 
\]
容易看出这正是单电子的选择定则。使用j-j耦合也可以得到同样的结果。

如果是L-S耦合,我们可以根据$nl$分析会有哪些谱项。
\begin{enumerate}
    \item 如果$l=0$,即价电子占据支壳层$n$s,那么$J=1/2$,于是谱项为\lsterm{2}{S}{1/2};
    \item 如果$l=1$,即价电子占据支壳层$n$l,那么$J=1/2, 3/2$,谱项为\lsterm{2}{P}{1/2}和\lsterm{2}{P}{3/2},后者自旋和轨道角动量平行,由\eqref{eq:spin-ortibal-coupling}可以看出后者能量高于前者,这和单电子的精细结构来自同样的物理机制;
    \item $l=3$,按照以上步骤可以得到两个谱项\lsterm{2}{D}{3/2}和\lsterm{2}{D}{5/2},后者能量高于前者;
    \item $l=4$,得到\lsterm{2}{F}{5/2}和\lsterm{2}{F}{7/2}。更高能量的谱项暂不考虑。
\end{enumerate}

现在以钠为例分析可以有哪些跃迁。首先同一$l$的谱项肯定不能相互跃迁。
下面列举了一些常见的跃迁,同一类型的跃迁导致的谱线称为一个\concept{线系}。

\begin{itemize}
    \item 会跃迁到基态3s上的只有$l=1$的谱项,即从3p,4p,5p等跃迁到3s上,这一组谱线都是双线(因为$l=1$有精细结构谱项分裂),其中3p到3s就是钠双黄线(有时也称为\concept{D双线},虽然它并不是漫线系的)。
    这一线系称为\concept{主线系(principal series)}。随着波长变短,精细结构导致的波长分裂也会变小。
    这就是将主线系的出发态$l=1$命名为p态的原因。
    \item 4s,5s,6s等谱线可以跃迁到3p,从一个单能级跳到双能级,从而导致一组双线,这组双线不同波长的两条谱线距离相等(就是\lsterm{2}{P}{1/2}和\lsterm{2}{P}{3/2}的差距),非常清晰,称为\concept{锐线系(sharp series)}。这就是锐线系的出发态$l=0$名为s态的原因。
    \item d态也可以跃迁到p态。d和p都有精细结构分裂,但由于$\Delta j$最大取到1,这实际上是一个三线系,称为\concept{漫线系(diffuse series)}。这就是漫线系的出发态$l=2$称为d态的原因。
    \item 与漫线系类似,f态可以跃迁到d态,产生一个三线系,称为\concept{基线系(fundamental series)}。这就是基线系的出发态$l=3$称为f态的原因。
\end{itemize}

其余的谱线都发生在比较高的能级之间,不容易观察到。

以上提到的都是发射光谱,但显然也可以把产生它们的过程倒转过来而得到吸收光谱。
由于基态为3s,吸收光谱中通常只能看到主线系。

对每个确定的$L$,都最多只有两个$J$值,因为$S$固定为$1/2$。可以计算出这两个$J$值之间的能级差为
\begin{equation}
    \Delta U = \frac{(\alpha Z)^4}{2 n^3 l(l+1)} E_0.
\end{equation}

\subsubsection{类氦光谱}

下面讨论类氦原子,即基态价壳组态为$n$s$^2$的原子,包括氦原子和碱土金属原子。
低激发态只有一个价电子发生跃迁。

使用L-S耦合。设跃迁价电子的角量子数为$l$,则$L=l$。两个电子给出$S=0, 1$,即有自旋单态和三重态。
附加选择定则\eqref{eq:l-s-selective}要求$\Delta S=0$,因此三重态和单态之间的跃迁是禁戒的——这样一来,如果只考虑偶极辐射,三重态的原子永远处于三重态,单态的原子永远处于单态。
我们将三重态的原子称为\concept{正氦},将单态的原子称为\concept{仲氦}。
三重态原子的自旋波函数是对称的,因此轨道波函数必须反对称,因此两个价电子不能出现在同一个轨道上。
由于仅考虑低激发态即只有一个价电子,两个价电子出现在同一轨道上只可能意味着电子组态为基态组态,即$n$s$^2$。
由于自旋平行会让能量降低,同一电子组态的正氦能量低于仲氦。
总之,正氦没有$n$s$^2$价电子组态,且同一电子组态的能量低于仲氦。

基于L-S耦合,仲氦的谱项列举如下:(以下$n=1$指的是价壳层)
\begin{enumerate}
    \item \lsterm{1}{S}{0},$n=1, 2, ,\ldots$;
    \item \lsterm{1}{P}{1},$n=2, 3, \ldots$;
    \item \lsterm{1}{D}{2},$n=3, 4, \ldots$;
    \item \lsterm{1}{F}{3},$n=4, 5, ,\ldots$。
\end{enumerate}
而正氦的谱项列举如下:
\begin{enumerate}
    \item \lsterm{3}{S}{1},$n=2, 3,\ldots$;
    \item \lsterm{3}{P}{0,1,2},$n=2, 3, \ldots$;
    \item \lsterm{3}{D}{1,2,3},$n=3, 4, \ldots$;
    \item \lsterm{3}{F}{2,3,4},$n=4, 5, ,\ldots$。
\end{enumerate}
请注意正氦没有$n=1$的态;此外除了S态以外,正氦的每个电子组态均存在关于$J$的能级三重分裂,$J$越大能量越高。

下面分别分析正氦和仲氦的光谱,由选择定则还是会有主线系、漫线系、锐线系、基线系。
对仲氦,所有这些线系都是单线系。
对正氦,主线系、锐线系是三线系,基线系、漫线系是三线系。

正氦和仲氦还有亚稳态。对仲氦,\lsterm{1}{S}{0}是亚稳态,因为低于它的能级只有基态,但从它到基态的过程$\Delta L = 0$;同样对正氦,\lsterm{3}{S}{1}是亚稳态,它要跃迁到基态必须发生正氦到仲氦的转变。
换而言之,1s2s电子组态是亚稳态,无论是仲氦还是正氦。它要跃迁到基态只能通过原子碰撞、双光子过程、电四极子跃迁、磁偶极子跃迁等微弱得多的过程。

\subsection{磁场中的多电子原子}

现在将\eqref{eq:magnetic-hamiltonian}引入。外加磁场破坏了空间各向同性,从而让磁量子数不再造成简并。
记由此产生的哈密顿量为
\begin{equation}
    \hat{H}_3 = - \hat{\vb*{\mu}} \cdot \vb*{B}.
\end{equation}

\subsubsection{弱磁场近似}

首先假定磁场很弱,比剩余相互作用和自旋-轨道耦合都弱,即
\begin{equation}
    \hat{H}_3 \ll \hat{H}_1, \hat{H}_2 \ll \hat{H}_0,
\end{equation}
从而可以在L-S耦合或者j-j耦合的基础上,计算$\hat{H}_3$引入的一阶微扰。计算结果是,
\begin{equation}
    E^{\text{mag}, (1)}_{JM_J} = \frac{J_z}{\hbar} g_J \mu_\text{B} B = m_J g_J \mu_\text{B} B = - \vb*{\mu}_J \cdot \vb*{B},
\end{equation}
其中
\begin{equation}
    \vb*{\mu}_J = - g_J \frac{\mu_\text{B}}{\hbar} \vb*{J}
    \label{eq:atom-mean-magnetic}
\end{equation}
称为\concept{原子平均磁矩}。这个形式和单电子自旋角动量或轨道角动量对磁矩的贡献非常相似:要乘以一个无量纲修正因子$g_J$(称为\concept{朗德g因子})。
无论如何,能量关于$M_J$的简并就解出了。

可以计算出对L-S耦合在$J$不为零时有
\begin{equation}
    g_J = \frac{3}{2} + \frac{S(S+1) - L(L+1)}{2J(J+1)},
    \label{eq:g-factor-ls}
\end{equation}
$J$为零时没有磁矩。
对j-j耦合$g_J$还和$\{j_i\}$有关。特别的,对L-S耦合下的双电子组态,我们有
\begin{equation}
    g_{j} = \frac{3}{2} + \frac{s(s+1) - l(l+1)}{2j(j+1)},
\end{equation}
这就是所谓的\concept{单电子朗德g因子};而对j-j耦合下的双电子组态,有
\begin{equation}
    g_{J j_1 j_2} = \frac{g_{j_1} + g_{j_2}}{2} + \frac{(g_{j_2} - g_{j_1})(j_2(j_2+1) - j_1(j_1+1))}{2J(J+1)},
\end{equation}
其中$g_{j_1}$和$g_{j_2}$指的是单电子的磁矩和总角动量相差的因子。

\subsubsection{半经典图像}

上面的表达式的导出需要使用严格的量子力学计算,不过实际上它们还是可以使用角动量矢量的经典图像推导并解释。

对L-S耦合,由强到弱的三种微扰分别造成以下影响:

\begin{enumerate}
    \item 剩余相互作用$\hat{H}_1$使得不同电子的轨道磁矩绕总轨道角动量$\vb*{L}$快速进动,自旋磁矩绕总自旋角动量$\vb*{S}$快速进动;
    \item 自旋-轨道耦合$\hat{H}_2$让磁矩$\vb*{\mu}_L$,$\vb*{\mu}_S$和总磁矩$\vb*{\mu}$绕着$\vb*{J}$缓慢进动(由于自旋和轨道角动量和对应磁矩之间的比例关系差一个因子2,$\vb*{\mu}$和$\vb*{J}$并不平行);
    \item 磁场作用$\hat{H}_3$让$\vb*{J}$围绕$\vb*{B}$做拉莫尔进动,这个进动的幅度又远小于上述两种进动。
\end{enumerate}

这就意味着,在拉莫尔进动的时间尺度上,总磁矩平均而言只在$\vb*{J}$的方向上有分量,它垂直于$\vb*{J}$的分量一直在不停地变化,无法产生明显效应。%
\footnote{系统不能对过快的外界扰动产生反应,正如简谐振子展示的那样。}%
因此有效的$\vb*{\mu}$为
\[
    \bar{\vb*{\mu}} = \vb*{\mu}_J = \left(\vb*{\mu} \cdot \frac{\vb*{J}}{\abs{\vb*{J}}}\right) \frac{\vb*{J}}{\abs{\vb*{J}}} = - \underbrace{\frac{(\vb*{L} + 2\vb*{S}) \cdot (\vb*{L} + \vb*{S})}{J^2}}_{g_J} \frac{\mu_\text{B}}{\hbar} \vb*{J}.
\]
这就找到了$J$不为零时的$g_J$的表达式。通过
\[
    \vb*{J} = \vb*{L} + \vb*{S},
\]
就得到了\eqref{eq:g-factor-ls}。
在$J$为零时,虽然一时看不出$g_J$应该取什么值,但由于$M_J$只能够取$0$,总磁矩也应该是零。

j-j耦合由于有大量的$j$,处理起来稍微复杂一些。三种由强到弱的微扰分别造成以下影响:
\begin{enumerate}
    \item 自旋-轨道耦合$\hat{H}_2$让每个电子的角动量合成成$\vb*{j}$,单电子磁矩$\vb*{\mu}$绕着$\vb*{j}$快速进动;
    \item 剩余相互作用$\hat{H}_1$让每个电子的单电子磁矩绕着$\vb*{J}$缓慢进动;
    \item 磁场作用$\hat{H}_3$让$\vb*{J}$绕$\vb*{B}$做最慢的拉莫尔进动。
\end{enumerate}
$\vb*{\mu}_i$绕$\vb*{j}_i$的进动在拉莫尔进动的时间尺度下非常快,因此真正有效的只有$\vb*{\mu}_i$在$\vb*{j}_i$方向上的投影$\vb*{\mu}_{j i}$。
相应的,原子总磁矩$\vb*{\mu}$绕$\vb*{J}$进动也很快,有效的只有$\vb*{\mu}$在$\vb*{J}$方向上的投影,从而
\[
    \bar{\vb*{\mu}} = \vb*{\mu}_J = \sum_i \left( \vb*{\mu}_{j i} \cdot \frac{\vb*{J}}{\abs{\vb*{J}}} \right) \frac{\vb*{J}}{\abs{\vb*{J}}}.
\]
使用和L-S耦合非常相似的方法,我们有
\begin{equation}
    \vb*{\mu}_{j i} = - g_{j i} \frac{\mu_\text{B}}{\hbar} \vb*{j}_i, \quad g_{j i} = \frac{3}{2} + \frac{s_i(s_i + 1) - l_i (l_i + 1)}{2 j_i (j_1 + 1)},
\end{equation}
代入上式即可。

以上两个推导都建立在拉莫尔进动非常慢这一假设上。
可以计算出,与$\vb*{J}$的变化对应的力矩为$\vb*{\mu}_J \times \vb*{B}$,则$\vb*{J}$的运动方程为
\[
    \dv{\vb*{J}}{t} = \vb*{\mu}_J \times \vb*{B},
\]
当然也可以直接从哈密顿量得到这个公式。将\eqref{eq:atom-mean-magnetic}代入上式,就得到
\[
    \dv{\vb*{J}}{t} = \frac{g_J \mu_\text{B} \vb*{B}}{\hbar} \times \vb*{J},
\]
从而推导出\concept{拉莫尔进动角速度}
\begin{equation}
    \vb*{\omega}_\text{L} = \frac{g_J \mu_\text{B} \vb*{B}}{\hbar}.
\end{equation}
这就是总角动量绕着磁场进动的角速度。相应的,可以计算出
\begin{equation}
    E^{\text{mag}, (1)}_{JM_J} = M_J g_J \mu_\text{B} B = M_J \hbar \omega_\text{L},
    \label{eq:small-magnetic-gapping}
\end{equation}
和量子力学计算出的结果一致。在拉莫尔进动中总角动量不守恒,但是磁量子数和角量子数——总角动量的进动锥体的母线和高度——都还是好量子数。
可以看到,磁矩和磁场的夹角越大,能量越高。

\subsubsection{塞曼效应}

既然磁场在能量中引入了$M_J$的依赖,原本L-S耦合和j-j耦合的每一个谱项$2J+1$重简并会发生等间距的分裂。
$M_J$越大意味着角动量和磁场的夹角越小(磁场在$z$方向上)。
分裂产生的$2J+1$个子能级称为\concept{塞曼能级},它会导致弱磁场中的原子光谱出现分裂,即\concept{塞曼效应}。
磁场的加入彻底地消除了所有简并。

塞曼效应可以分成两种,一种是\concept{正常塞曼效应},谱线等间距分裂成三根,另一种是\concept{反常塞曼效应},即不满足以上条件的光谱分裂。
塞曼效应发现时量子力学尚未建立,但是塞曼的老师——洛伦兹——通过经典电偶极振子模型计算出来正常塞曼效应。这也就是“正常”和“反常”这两个概念的来源。

正常塞曼效应的模型大体上是这样的:原子价电子受到正离子线性回复力,做简谐运动,与此同时受到一个磁场力,从而运动方程可以写成
\begin{equation}
    m_\text{e} \dv[2]{\vb*{r}}{t} = - m_\text{e} \omega_0^2 \vb*{r} + \left( - e \dv{\vb*{r}}{t} \times \vb*{B} \right),
\end{equation}
照惯例取$\vb*{B}$的方向指向$z$轴,则得到
\[
    \begin{aligned}
        m_\text{e} \dv[2]{x}{t} &= - m_\text{e} \omega_0^2 x - e \dot{y} B, \\
        m_\text{e} \dv[2]{y}{t} &= - m_\text{e} \omega_0^2 y + e \dot{x} B, \\
        m_\text{e} \dv[2]{z}{t} &= - m_\text{e} \omega_0^2 z.
    \end{aligned}
\]
考虑时谐解,并假定磁场非常小,则可以得到三个振动模式,圆频率分别是
\begin{equation}
    \omega = \omega_0, \quad \omega_0 + \frac{eB}{2m}, \quad \omega_0 - \frac{eB}{2m}.
\end{equation}
因此,原本的一个能级一分为三,三个能级间隔相等,相邻能级之间相差
\begin{equation}
    \Delta E = \frac{\hbar eB}{2m} = \mu_\text{B} B.
\end{equation}

但实际上,实验表明虽然的确会有能级分裂,且能级分裂的数量级的的确确也在$\mu_\text{B} B$的量级上,但前面的系数却并不总是$1$。
使用量子力学中的弱磁场近似\eqref{eq:small-magnetic-gapping}%
\footnote{
    需要注意的是自旋-轨道耦合本身是涉及一个磁场的,这个磁场大小大约是\SI{12}{T},已经是非常强的磁场了。外加磁场是弱还是强需要和这个做比较。
}%
可以发现我们有
\begin{equation}
    \Delta E = g_J \mu_\text{B} B,
\end{equation}
可见实际的能级分裂虽然也是均匀的分裂,但$\mu_\text{B} B$前面的系数的确未必是1,并且能级也未必分裂成三条。
在总自旋为零时,
\[
    g_J = \frac{3}{2} + \frac{-L(L+1)}{2J(J+1)} = 1,
\]
因此能级分裂和通过经典谐振子模型计算出来的相同。
总之,总自旋不为零的能级必然发生反常塞曼效应,总自旋为零且$J=1$的能级发生正常塞曼效应。

\subsubsection{强磁场和帕邢-巴克效应}

当磁场比较强时,若原子在没有磁场时为L-S耦合,则
\begin{equation}
    {H}_2 \ll {H}_3 \ll {H}_1 \ll {H}_0,
\end{equation}
此时做完剩余相互作用的微扰之后就应该考虑$\hat{H}_3$,相比之下自旋-轨道耦合则是非常小的。我们知道$\hat{H}_2$引入的能级分裂满足朗德间隔定则\eqref{eq:lande-gapping},即%
\footnote{这里我们使用某部分哈密顿量引入的能级差来衡量其影响。哈密顿量本身的绝对大小并无意义,因为总是可以加上或者减去任意大小的常数。系统中的量子数发生变化,对应的能级会发生什么变化才真正体现这一部分相互作用是否重要。}%
\[
    H_2 \sim J \zeta_{LS} \hbar^2.
\]
$\hat{H}_3$引入的能级分裂满足\eqref{eq:small-magnetic-gapping},即能级间隙为$\hbar \omega_\text{L}$,于是
\[
    H_3 \sim \hbar \omega_\text{L}.
\]
当磁场强到
\begin{equation}
    \omega_\text{L} \gg \zeta_{LS} \hbar,
\end{equation}
即拉莫尔进动的速度远快于自旋-轨道耦合时,先前的弱磁场近似就不再适用了。
我们在做完$\hat{H}_1$的微扰之后需要直接做$\hat{H}_3$的微扰。
在$\hat{H}_3$的扰动之后,$L, M_L, S, M_S$仍然是好量子数。注意到$\hat{H}_3$形如
\[
    \hat{H}_3 = (\hat{L}_z + 2 \hat{S}_z) \frac{\mu_\text{B} B}{\hbar}, 
\]
它在$L M_L S M_S$的表象下的本征值为
\begin{equation}
    E^{\text{mag}, (1)}_{\{n_i l_i\} L M_L S M_S} = (M_L + 2 M_S) \mu_\text{B} B,
\end{equation}
则先考虑$\hat{H}_1$的微扰,再考虑$\hat{H}_3$的微扰之后,有
\[
    E^{\text{rem,mag}, (0)}_{\{n_i l_i\} L M_L S M_S} = E^{\text{rem}, (1)}_{\{n_i l_i\} LS} + E^{\text{mag}, (1)}_{\{n_i l_i\} L M_L S M_S}.
\]
然后再考虑$\hat{H}_2$的微扰,此时$\vb*{L}$和$\vb*{S}$都在绕着$\vb*{B}$进动,$\vb*{J}$实际上并不守恒,已经失去意义了。
选择定则为
\begin{equation}
    \Delta L = 0, \pm 1, \quad \Delta S = 0, \quad \Delta M_S = 0, \quad \Delta M_L = 0, \pm 1.
\end{equation}
可以看到所有关于$J$的定则都被移除了。若原子放出光子,则
\begin{equation}
    \omega = \omega_0 + \frac{\mu_\text{B} B}{\hbar} \Delta M_L.
\end{equation}
$\Delta M_L$取$1$和$-1$代表光子是$\sigma$光,取$0$则代表光子是$\pi$光。

\subsubsection{磁共振}

外加磁场导致能级分裂,由于选择定则,分裂的能级之间不存在电偶极跃迁。但实际上,这些能级之间可以有磁偶极跃迁。
实际上,这就是\concept{磁共振}的原理。本节首先从一个半经典图像来分析磁共振,然后再考虑严格的量子理论。

由于$\vb*{J}$和$\vb*{\mu}$平行,我们有
\[
    \dv{\vb*{\mu}_J}{t} = \vb*{\omega}_J \times \vb*{\mu}_J,
\]
其中
\[
    \vb*{\omega}_J = \frac{g_J \mu_\text{B} \vb*{B}}{\hbar}.
\]
没有使用下标L是因为我们将要考虑方向随着时间变动的磁场。
现在如果施加两个磁场,一个磁场是恒定的$\vb*{B}_0$,还有一个是以平行于$z$轴的角速度$\vb*{\omega}$左旋的弱磁场
\begin{equation}
    \vb*{B}_1 = B_1 (\cos(\omega t) \vb*{e}_x + \sin(\omega t) \vb*{e}_y),
\end{equation}
则
\begin{equation}
    \vb*{\omega}_J = \vb*{\omega}_\text{L} + \vb*{\omega}_{1}, \quad \vb*{\omega}_\text{L} = \frac{g_J \mu_\text{B} B_0}{\hbar} \vb*{e}_z, \quad \vb*{\omega}_1 = \frac{g_J \mu_\text{B} B_1}{\hbar} (\cos(\omega t) \vb*{e}_x + \sin (\omega t) \vb*{e}_y).
\end{equation}
将实验室坐标系记为$S$,将相对实验室坐标系以角速度$\vb*{\omega}$旋转的参考系记为$S'$,则
\[
    \dv{t} = \vb*{\omega} \times + \dv'{t},
\]
于是
\[
    \dv'{t} \vb*{\mu}_J = (\vb*{\omega}_\text{L} - \vb*{\omega} + \vb*{\omega}_1) \times \vb*{\mu}_J.
\]
由于在$S'$系中$\vb*{B}_1$是恒定的,上式意味着在$S'$系中$\vb*{\mu}_J$以\concept{Rabi角速度}
\begin{equation}
    \vb*{\omega}_\text{R} = \vb*{\omega}_\text{L} - \vb*{\omega} + \vb*{\omega}_1, \quad \omega_\text{R} = \sqrt{(\omega_\text{L} - \omega)^2 + \omega_1^2}
\end{equation}
进动。我们首先考虑一个简化的情况:初始时,$\vb*{J}$和$\vb*{B}_0$平行。这样,以$\vb*{\omega}_\text{R}$在$S'$系中的指向为$z'$轴,并要求$t=0$时$S'$系的$x'$轴和$S$系的$x$轴重合,则在$S'$系中有
\begin{equation}
    \mu_{Jx}' = \mu_J \frac{\omega_1}{\omega_\text{R}} \sin (\omega_\text{R} t), \quad \mu_{Jy}' = - \mu_J \frac{\omega_1}{\omega_\text{R}} \cos (\omega_\text{R} t), \quad \mu_{Jz}' = \mu_J \frac{\omega_\text{L} - \omega}{\omega_\text{R}}.
\end{equation}
切换回$S$系,就有
\begin{equation}
    \mu_{Jz} = \mu_J \left( \frac{(\omega_\text{L} - \omega)^2}{\omega_\text{R}^2} + \frac{\omega_1^2}{\omega_\text{R}^2} \cos (\omega_\text{R} t) \right).
\end{equation}
在没有时变磁场时$\mu_{Jz}$不会有时间变化,因为显然此时$\omega_1=0$。因此,$\mu_{Jz}$的波动范围越大,时变磁场带来的影响越明显。
当$\omega_\text{L}=\omega$时,$\mu_{Jz}$的波动最为明显,即磁矩与外场有效地交换能量,达到共振状态。此时$\vb*{\mu}_J$在$S'$系中完全位于$x'$-$y'$平面内。
此时
\begin{equation}
    \begin{aligned}
        \mu_{Jx} &= \mu_J \sin(\omega_1 t) \cos(\omega_\text{L} t - \pi / 2), \\
        \mu_{Jy} &= \mu_J \sin(\omega_1 t) \sin(\omega_\text{L} t - \pi / 2), \\
        \mu_{Jz} &= \mu_J \cos(\omega_t t).
    \end{aligned}
    \label{eq:rotating-spin}
\end{equation}
$\mu_{Jx}$和$\mu_{Jy}$是水平旋转磁场$\vb*{B}_1$带来的感生磁矩,它有一个相位落后$\pi / 2$。
由于$B_1$比$B_0$小得多,有$\omega_1 \ll \omega_\text{L}$,因此感生磁矩不停地绕着$z$轴以角速度$\omega_\text{L}$转动,而转动平面的经度以$\omega_1$升高。

使用类似的方法也可以得到系统对右旋磁场或者沿着$z$方向线偏振磁场的响应,不过这两种磁场实际上并不能够导致磁共振。
总之,对系统施加一个周期性磁场,只有其左旋成分会导致磁共振。

从更加量子的角度看,磁共振实际上是电子磁矩导致的电子和外部磁场的耦合。
哈密顿量$\hat{H}_3$和$-\vb*{d} \cdot \vb*{E}$形式一致,它会导致\concept{磁偶极辐射}。
在这里由于外界磁场是完全给定的,不需要像考虑电偶极辐射那样将所有可能的电场模式造成的跃迁做非相干叠加。
仿照电偶极辐射的推导(见\autoref{sec:electro-dipole-hopping}),我们有% 下面的说法是错的
\[
    \braket{j}{\psi(t)} = \frac{\ii}{\hbar} \ee^{-\ii E_n t / \hbar} \mel{j}{\hat{\mu}_z}{i} B_1 \frac{\ee^{\ii(\omega_{ji}-\omega) t} - 1}{\ii (\omega_{ji} - \omega)},
\]
因此是否能够发生跃迁取决于两点:首先,外加磁场的频率必须要和$\omega_{ji}$匹配;其次,磁矩的矩阵元不应该是零。
后一个要求就是\concept{磁偶极跃迁的选择定则}。通过计算矩阵元可以发现,有
\begin{equation}
    \Delta J = 0, \pm 1, \quad \Delta M_J = 0, \pm 1, \quad \Delta l = 0, \Delta n = 0,
\end{equation}
且$J$不能从$0$变化到$0$。这说明原子组态不能够发生任何变化。对L-S耦合,还有以下附加定则:
\begin{equation}
    \Delta L = \Delta S = 0.
\end{equation}

例如,因为外加磁场而分裂出来的塞曼能级之间肯定能够发生跃迁。
由于$\Delta M_J=\pm 1$(不能够取$0$否则就没有发生任何跃迁),有
\[
    \hbar \omega_{ij} = g_J \mu_\text{B} B M_J,
\]
于是能够发生的跃迁对应的圆频率就是$\omega_\text{L}$。
这对应微波频段。
特别的,如果只有一个价电子,且没有轨道角动量,那么
\[
    L = 0, \quad S = \frac{1}{2}, \quad J = \frac{1}{2},
\]
此时
\begin{equation}
    \omega = \omega_\text{L} = \frac{2 \mu_\text{B} B}{\hbar},
\end{equation}
即\concept{电子自旋共振}。

计算磁矩的期望值,会发现只要把\eqref{eq:rotating-spin}中的磁矩替换为期望值,结果就是正确的。

实际上,只要有磁场和带磁矩的粒子,就能够发生磁共振,因此完全可以做\concept{核磁共振},通过共振情况推断不同成分的原子的含量。

\section{核物理}

以上讨论的都是原子核以外的物理,而下面将要讨论原子核内部的物理。原子核内部的物理涉及强力和弱力,完整地处理这些相互作用需要QCD,本文将不详细讨论这个理论。

\subsection{原子核的成分}

\subsubsection{质子和中子的发现}

质子的发现来自1919年卢瑟福使用$\alpha$粒子轰击轻元素而得到的一种“共有成分”。现在我们知道,他大致是发现了这么一个过程:
\[
    \nuclear{He}{4}{2} + \nuclear{X}{A}{Z} \longrightarrow \nuclear{Y}{A+3}{Z+1} + \nuclear{H}{1}{1}.
\]
在此过程中质量数(左上标)和电荷数(左下标)守恒。
那么,质子——事后发现这就是氢原子核——应当是原子核的一个组成部分。

当时还发现了另一个过程:
\[
    \nuclear{X}{A}{Z} \longrightarrow \nuclear{Y}{A}{Z+1} + \text{e},
\]
即\concept{$\beta$衰变}。乍一看似乎非常自然地可以认为原子核同时由质子和电子组成。但实际上这个假设给出了错误的核半径、核自旋、磁矩。
就核半径而言,电子和质子组成的体系可以看成电子嵌入在均匀的正电荷“凝胶”中,则电子实际上是三维简谐振子。计算发现要能够稳定地将电子约束在原子核内,原子核的尺度要达到原子尺度——显然这是错误的。
就核自旋而言,质子和电子都是自旋$1/2$,那么,质量数为偶数,电荷数为奇数的原子核的自旋应该是半整数,而实验结果为整数。
从磁矩的角度,质子磁矩远小于电子,则核磁矩应该是玻尔磁子的数量级,且质子的磁矩可以略去,但实际测量得到的磁矩全部都是质子的量级。
因此,这个核模型是错误的。

1920年,卢瑟福猜测,原子核内存在质量和质子接近的一种粒子。1930年,用$\alpha$轰击Be,得到一种穿透力很强的中性射线,当时假定为$\gamma$射线;1931年,用这种射线轰击石蜡,得到\SI{4.8}{MeV}的质子流,当时假定为$\gamma$射线和质子发生康普顿散射,按此假设,$\gamma$射线的能量要达到\SI{50}{MeV}。
1932年,用氮原子核代替石蜡,得到\SI{1.2}{MeV}的质子流,则对应的$\gamma$射线能量要达到\SI{89}{MeV}。
这么高的光子质量实在是离谱,且使用同一放射源产生这么大能量范围的射线完全解释不通。
查德威克于是断言,此射线不可能是$\gamma$射线,而应该是一种电中性粒子组成的,$\alpha$射线轰击原子核会发生以下过程:
\[
    \nuclear{He}{4}{2} + \nuclear{X}{A}{Z} \longrightarrow \nuclear{Y}{A+3}{Z+2} + \nuclear{n}{1}{0}.
\]
这种质量数为1(和质子相同),不带电的粒子就是中子。据使用此中性射线轰击不同原子量的物质得到的质子流的能量,在非相对论近似下反推可以发现中子质量和质子质量基本一致。

以质子和中子为基本组成的原子核模型与核物理中的实验结果一致。因此,这两种粒子是核物理层次上的基本粒子,即\concept{核子}。
核子实际上仍然可分,可以分解成夸克。

\subsubsection{自旋和磁矩}

质子和中子也有自旋,也有磁矩。设
\begin{equation}
    \mu_\text{N} = \frac{e \hbar}{2 m_\text{p}}
\end{equation}
为\concept{核磁子},它是把波尔磁子中的电子质量替换成质子质量之后得到的结果。质子和中子的轨道角动量和自旋角动量决定了其磁矩,即
\begin{equation}
    \vb*{\mu}_\text{p} = \frac{\mu_\text{N}}{\hbar} (g_{\text{p} l} \vb*{l} + g_{\text{p} s} \vb*{s}), \quad 
    \vb*{\mu}_\text{n} = \frac{\mu_\text{N}}{\hbar} (g_{\text{n} l} \vb*{l} + g_{\text{n} s} \vb*{s}),
\end{equation}
质子携带一个电荷,中子不携带电荷,则
\begin{equation}
    g_{\text{p}l} = 1, \quad g_{\text{n} l} = 0.
\end{equation}
质子和中子的自旋角动量对磁矩的贡献和电子非常不同,实际上
\begin{equation}
    g_{\text{p} s} \approx 5.6, \quad g_{\text{n} s} \approx -3.8.
\end{equation}
会有这种奇怪的行为,是因为核子的“自旋”(即所谓\concept{核自旋})是核子的总角动量扣除核子整体的轨道角动量之后剩下的部分,实际上是在核子静止的情况下,组成核子的各个夸克的自旋加上它们之间的轨道角动量得到的结果。
通常用$\hat{\vb*{I}}$表示核自旋。与之前将一个原子内部的所有电子磁矩合成成一个等效磁矩相似,我们有
\begin{equation}
    \vb*{\mu}_I = \underbrace{g_I \frac{\mu_\text{N}}{\hbar}}_\gamma \vb*{I}, \quad \mu_{Iz} = g_I \mu_\text{N} m_I.
\end{equation}

\subsection{核子的结合}

\subsubsection{结合能}

可以通过电子散射来测量原子核的半径。实验数据表明,设$A$为质量数,我们有这样的公式:
\begin{equation}
    R = r_0 A^{1/3},
\end{equation}
从而
\[
    V = \frac{4}{3} \pi r_0^3 A \propto A.
\]
这表明核几乎不可压缩,核子是非常紧密地挨在一起的。

当核子聚集,受核力力约束而成为原子核时,会放出能量,从而发生\concept{质量亏损}
\begin{equation}
    \Delta m = Z m_\text{p} + (A - Z) m_\text{n} - m_\text{N}.
\end{equation}
\concept{结合能}为
\begin{equation}
    E_\text{B} = \Delta m c^2 = - \left( \sum_i T_i + \sum_{i < j} V_{ij} \right), 
\end{equation}
显然动能为正,上式意味着为了形成稳定的原子核,势能必须为负并且绝对值很大。
结合能当然在任何过程中都存在,但是在化学反应中能量变化在eV级别,而在核反应中能量变化在MeV级别,后者要明显得多。

定义
\begin{equation}
    \epsilon = \frac{E_\text{B}}{A} 
\end{equation}
为\concept{比结合能},比结合能越大说明系统越稳定。随着原子序数增大,比结合能先上升后下降,因此轻核聚变,重核裂变。

\subsubsection{核力}

让带有同种电荷的核子能够聚集在一起的是一种非常强的相互作用:\concept{核力}。其强度比电磁相互作用大两个数量级(预料之中,否则无法把核子聚集起来),且力程非常短,因此几乎只有相邻的核子之间才能够有核力,或者说具有\concept{短程饱和性}。
确定这一点的方式是,如果只有相邻的核子之间才有核力,则应有
\[
    E_\text{B} \sim A,
\]
而如果任意两个核子之间都可以有核力,则应该有
\[
    E_\text{B} \sim A(A-1).
\]
实验支持前者而不是后者。
核力和电荷无关,因此核子数量相同的原子核的结合能的变化仅仅和电荷之间的库伦排斥有关,如$\nuclear{He}{3}{2}$的结合能小于$\nuclear{H}{3}{1}$,因为前者有两个正电荷,库伦排斥更强。

这样,质子和中子在核力上具有完全一样的性质。它们仅有的不同在于电荷这个标签。实际上,这可以对应到一个称为\concept{同位旋}的标签上。
对核子,同位旋为$I = 1/ 2$,
\begin{equation}
    I_z = \begin{cases}
        \frac{1}{2}, \quad &\text{p}, \\
        -\frac{1}{2}, \quad &\text{n},
    \end{cases}
\end{equation}
电荷为
\begin{equation}
    q = \left( I_z + \frac{1}{2} \right) e.
\end{equation}
同位旋的合成和角动量的合成类似。

核力还和自旋的相对取向有关。此外,在\SI{0.8}{fm}内,核力实际上是排斥力,而在\SI{0.8}{fm}到\SI{2}{fm}内,核力才是吸引力。距离超过\SI{10}{fm}时核力基本上观察不到了。

电磁相互作用对应的玻色子是光子,核力也应该有对应的玻色子和玻色场。
在相对论性量子场论的框架下,最容易想到的方案是,取一个带有质量的克莱因-高登场,其稳态解形如
\[
    \phi \propto \frac{\ee^{- r / a}}{r},
\]
从而这是一个短程相互作用,特征尺度为$a$,且可以根据强力的力程\SI{2}{fm}估算出其质量在电子和质子之间。
实验中果然探测到了这样的粒子,即$\pi^{0, \pm 1}$介子,它们寿命有限,有自旋,有电荷。
$\pi$介子可以看成取不同同位旋($I=1,I_z=0, \pm 1$)的同一种粒子。
对介子,有
\begin{equation}
    q = I_z e.
\end{equation}
所有反应都需要保持各个守恒量相同,这样就有
\[
    \text{p} \longleftrightarrow \text{p} + \pi^0, \quad \text{n} \longleftrightarrow \text{n} + \pi^0, \quad \text{p} \longleftrightarrow \text{n} + \pi^+, \quad \text{n} \longleftrightarrow \text{p} + \pi^-. 
\]
以上四个过程各自对应着费曼图的一种“核子-核子-介子”角点。

实际上,以上过程并不是最基本的。核子和介子由夸克组成,将夸克粘在一起的是胶子,上面所说的核子-介子相互作用——也即核力——实际上是类似于范德华力这样的剩余作用力。

\subsection{衰变}

\subsubsection{半衰期和活度}

一个原子核的中子数为
\begin{equation}
    N = Z - A.
\end{equation}
一个原子核中的中子数和质子数显然不能够相差太多,否则或者质子过多而自发溢出,或者中子过多而自发溢出,因为原子核也遵循量子力学,满足泡利不相容原理,有核能级,所以质子和中子都是从低到高填充这些能级,而如果中子数过多,那么中子能级上被占据了太多的状态,和质子能级产生很大的能量差,因此不稳定;质子数过多同理。
对比较重的核,中子通常比质子多一些,因为这样可以稀释质子之间的库伦排斥。

这样,在$N$-$Z$图上就有两条线,一条是质子溢出线,一条是中子溢出线,这两条线夹成的区域以外不能形成束缚态的核;在这两条线中间的区域分成“核心”和“周边”,前者是稳定核素,后者是不稳定核素。

通过费米黄金法则可以计算衰变的跃迁率,它和时间基本无关,因此对一个原子核,设$p(t)$是$t$时刻还不发生衰变的概率,则有以下状态转移方程:
\[
    p(t+\dd{t}) = p(t) (1 - \lambda \dd{t}),
\]
于是
\[
    \dv{p}{t} = - \lambda p.
\]
如果同时考虑大量核,设$N(t)$为$t$时刻还未衰变的核的数目,则
\[
    \dv{N}{t} = - \lambda N,
\]
从而
\begin{equation}
    N(t) = N(0) \ee^{- \lambda t}.
    \label{eq:nuclear-decay}
\end{equation}
这样,虽然对一个单独的核,我们不能预测其什么时候衰变,但大量核服从明确的统计规律。
\eqref{eq:nuclear-decay}意味着核的衰变有固定的半衰期
\begin{equation}
    T_{1/2} = \frac{\ln 2}{\lambda}.    
\end{equation}
平均寿命为
\begin{equation}
    \tau = \int t \cdot P(\text{decay between $t$ and $t+\dd{t}$}) = \int t \cdot p(t) \cdot \lambda \dd{t} = \frac{1}{\lambda} = \frac{T_{1/2}}{\ln 2}.
\end{equation}
单位时间内发生衰变的核的数目给出了一块放射性材料的放射性活性,称为\concept{活度},它是
\begin{equation}
    R = \lambda N = R(0) \ee^{-\lambda t}.
\end{equation}
活度的单位是\si{Bq}或者\si{Ci}(贝克勒尔或者居里),定义为
\[
    \SI{1}{Bq} = \SI{1}{s^{-1}}, \quad \SI{1}{Ci} = \SI{3.7e10}{Bq}.
\]

\subsubsection{常见衰变类型}

\concept{$\alpha$衰变}是指原子核放出一个$\alpha$粒子的衰变,即
\[
    \nuclear{X}{A}{Z} \longrightarrow \nuclear{Y}{A-4}{Z-2} + \nuclear{He}{4}{2}.
\]
可以通过测量衰变之后释放的$\alpha$粒子的动能来测定衰变能,两者的关系为
\begin{equation}
    Q = \frac{A}{A-4} T_\alpha.
\end{equation}
实际上,可以测定到几种分立的$T_\alpha$,这表明原子核的确具有离散的能级。

另一种衰变是$\beta$衰变,即释放电子的衰变。可以尝试将其通式写成
\[
    \nuclear{X}{A}{Z} \longrightarrow \nuclear{Y}{A}{Z+1} + \text{e}^-.
\]
使用和之前类似的方法,可以发现
\begin{equation}
    Q \approx T_\text{e}.
\end{equation}
不过,实验观察到释放出的电子的能量连续分布,这和原子核能级离散矛盾;由于核子数在反应前后不变,$X$和$Y$要么同时是玻色子要么同时是费米子,而电子是费米子,因此反应前后的角动量必定一个是整数一个是半整数,不可能相等。
唯一的可能是,在$\beta$衰变中产生了一种不参与强相互作用、不参与电磁相互作用、质量非常小的粒子。这就是\concept{中微子}。
整个过程如下:
\[
    \text{n} \longrightarrow \text{p} + \text{e}^- + \bar{\nu}_\text{e}.
\]
中微子实际上是一种自旋$1/2$的费米子,考虑它之后$\beta$衰变就是
\[
    \nuclear{X}{A}{Z} \longrightarrow \nuclear{Y}{A}{Z+1} + \text{e}^- + \bar{\nu}_\text{e}.
\]
会产生中微子是因为还有一种相互作用:\concept{弱相互作用},它会让中子衰变成质子和电子型反中微子。
另一种弱相互作用导致的衰变是质子衰变成中子、正电子和电子型中微子,如下:
\[
    \text{p} \longrightarrow \text{n} + \text{e}^+ + \nu_\text{e}.
\]
此外也可以发生\concept{核外电子俘获},即质子俘获一个电子,而形成中子和一个电子型中微子,即
\[
    \text{p} + \text{e}^- \longrightarrow \text{n} + \nu_\text{e}.
\]
以上三种过程发生的几率都在同一数量级上。

弱相互作用也是短程力,它比强相互作用的强度小九个数量级,其力程比强相互作用小三个数量级。
与弱相互作用有关的媒介玻色子为$\text{W}^\pm$和$Z^0$,称为\concept{中间玻色子}。
前两种互为反粒子,带有电荷$\pm 1$,第三种则不带电。

还有一种所谓的$\gamma$衰变,它指的实际上是高核能级的原子核释放$\gamma$光子的过程。

\end{document}

\end{document}