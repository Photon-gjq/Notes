\documentclass[hyperref, a4paper]{article}

\usepackage{geometry}
\usepackage{titling}
\usepackage{titlesec}
% No longer needed, since we will use enumitem package
% \usepackage{paralist}
\usepackage{enumitem}
\usepackage{footnote}
\usepackage{enumerate}
\usepackage{amsmath, amssymb, amsthm}
\usepackage{mathtools}
\usepackage{bbm}
\usepackage{cite}
\usepackage{graphicx}
\usepackage{subfigure}
\usepackage{physics}
\usepackage{tensor}
\usepackage{siunitx}
\usepackage{booktabs}
\usepackage[version=4]{mhchem}
\usepackage{tikz}
\usepackage{xcolor}
\usepackage{listings}
\usepackage{autobreak}
\usepackage[ruled, vlined, linesnumbered]{algorithm2e}
\usepackage{nameref,zref-xr}
\zxrsetup{toltxlabel}
\zexternaldocument*[optics-]{../optics/optics}[optics.pdf]
\zexternaldocument*[solid-]{../solid/solid}[solid.pdf]
\zexternaldocument*[113-]{lecture-11-3}[lecture-11-3.pdf]
\usepackage[colorlinks,unicode]{hyperref} % , linkcolor=black, anchorcolor=black, citecolor=black, urlcolor=black, filecolor=black
\usepackage[most]{tcolorbox}
\usepackage{prettyref}

% Page style
\geometry{left=3.18cm,right=3.18cm,top=2.54cm,bottom=2.54cm}
\titlespacing{\paragraph}{0pt}{1pt}{10pt}[20pt]
\setlength{\droptitle}{-5em}
\preauthor{\vspace{-10pt}\begin{center}}
\postauthor{\par\end{center}}

% More compact lists 
%\setlist[itemize]{
%    itemindent=17pt, 
%    leftmargin=1pt,
%    listparindent=\parindent,
%    parsep=0pt,
%}

% Math operators
\DeclareMathOperator{\timeorder}{\mathcal{T}}
\DeclareMathOperator{\diag}{diag}
\DeclareMathOperator{\legpoly}{P}
\DeclareMathOperator{\primevalue}{P}
\DeclareMathOperator{\sgn}{sgn}
\newcommand*{\ii}{\mathrm{i}}
\newcommand*{\ee}{\mathrm{e}}
\newcommand*{\const}{\mathrm{const}}
\newcommand*{\suchthat}{\quad \text{s.t.} \quad}
\newcommand*{\argmin}{\arg\min}
\newcommand*{\argmax}{\arg\max}
\newcommand*{\normalorder}[1]{: #1 :}
\newcommand*{\pair}[1]{\langle #1 \rangle}
\newcommand*{\fd}[1]{\mathcal{D} #1}
\DeclareMathOperator{\bigO}{\mathcal{O}}

% TikZ setting
\usetikzlibrary{arrows,shapes,positioning}
\usetikzlibrary{arrows.meta}
\usetikzlibrary{decorations.markings}
\tikzstyle arrowstyle=[scale=1]
\tikzstyle directed=[postaction={decorate,decoration={markings,
    mark=at position .5 with {\arrow[arrowstyle]{stealth}}}}]
\tikzstyle ray=[directed, thick]
\tikzstyle dot=[anchor=base,fill,circle,inner sep=1pt]

% Algorithm setting
% Julia-style code
\SetKwIF{If}{ElseIf}{Else}{if}{}{elseif}{else}{end}
\SetKwFor{For}{for}{}{end}
\SetKwFor{While}{while}{}{end}
\SetKwProg{Function}{function}{}{end}
\SetArgSty{textnormal}

\newcommand*{\concept}[1]{{\textbf{#1}}}

% Support for tensor double arrows.
\renewcommand{\tensor}[1]{ \stackrel{\leftrightarrow}{\vb*{#1}}}

% Embedded codes
\lstset{basicstyle=\ttfamily,
  showstringspaces=false,
  commentstyle=\color{gray},
  keywordstyle=\color{blue}
}

% Reference formatting
\newrefformat{fig}{Figure~\ref{#1} on page~\pageref{#1}}

% Color boxes
\tcbuselibrary{skins, breakable, theorems}
\newtcbtheorem[number within=section]{warning}{Warning}%
  {colback=orange!5,colframe=orange!65,fonttitle=\bfseries, breakable}{warn}
\newtcbtheorem[number within=section]{note}{Note}%
  {colback=green!5,colframe=green!65,fonttitle=\bfseries, breakable}{note}
\newtcbtheorem[number within=section]{info}{Info}%
  {colback=blue!5,colframe=blue!65,fonttitle=\bfseries, breakable}{info}

\newcommand{\opticsdoc}{\href{../optics/optics.pdf}{this optics note}}

\title{Advanced Electrodynamics by Prof. Kun Din}
\author{Jinyuan Wu}

\begin{document}

\maketitle

Some contents of the first several lectures are covered in \opticsdoc,
but some technical details are worth a separate note.

\section{Guiding vectors and eigenmodes}

We use the guiding vector method to find eigenmodes in 
several systems. Here we summarize the procedure: what we want to do is to solve the equation 
\begin{equation}
    (\laplacian + k^2) \vb*{M} = 0, \quad \div{\vb*{M}} = 0.
    \label{eq:m-problem}
\end{equation}
By doing so we can make the ansatz 
\begin{equation}
    \vb*{M} = \curl(\psi \vb*{c}), \quad \div{\vb*{c}} = \const, \quad \curl{\vb*{c}} = 0, 
\end{equation}
which is always possible. (see \eqref{optics-sec:guiding-vector} in \opticsdoc) Now \eqref{eq:m-problem} 
becomes 
\[
    \laplacian \curl{(\psi \vb*{c})} + k^2 \curl{(\psi \vb*{c})} = 0.
\]
The LHS is 
\[
    \begin{aligned}
        &\quad \laplacian \curl{(\psi \vb*{c})} + k^2 \curl{(\psi \vb*{c})} \\
        &= - \curl{ \curl \curl (\psi \vb*{c})} + k^2 \curl{(\psi \vb*{c})} \\
        &= - \curl{(\grad{\div{(\psi \vb*{c})}} - \laplacian (\psi \vb*{c}))} + k^2 \curl{(\psi \vb*{c})} \\
        &= \curl(\laplacian(\psi \vb*{c})) + k^2 \curl{(\psi \vb*{c})} \\
        &= (\laplacian + k^2) \curl(\psi \vb*{c}) \\
        &= (\laplacian + k^2) 
    \end{aligned}
\]
TODO
So what should be done is to find a certain $\vb*{c}$ and then solve 
\begin{equation}
    \laplacian \psi + k^2 \psi = 0.
    \label{eq:scalar-helmholtz}
\end{equation}

Note that we are sure that $\vb*{c}$ is perpendicular to $\vb*{M}$, but not necessarily $\vb*{N}$.
A good example is reflection: $\vb*{c}$ is the normal vector of the surface, $\grad{\psi}$ is the 
wave vector, and it can be easily found that only one of the electric field and the magnetic field 
can be perpendicular to $\vb*{c}$.
Therefore, we call the solution $\vb*{E} = \vb*{M}, \vb*{B} = \vb*{N}$ the \concept{TE mode}, 
the solution $\vb*{B} = \vb*{M}, \vb*{E} = \vb*{N}$ the \concept{TM mode}. There is yet a third mode 
\begin{equation}
    \vb*{L} = \grad{\psi},
\end{equation} 
which is a longitude mode.

Note that it is possible that $\grad {\psi} \parallel \vb*{c}$ everywhere, and in this case we need to 
find another vector field $\vb*{a}$ such that TODO: always scalar?

\begin{info*}{Bessel functions}{}
    We consider the Bessel equation 
    \begin{equation}
        x^{2} \frac{\mathrm{d}^{2} y}{\mathrm{~d} x^{2}}+x \frac{\mathrm{d} y}{\mathrm{~d} x}+\left(x^{2}-\alpha^{2}\right) y=0.
        \label{eq:bessel}
    \end{equation}
    Here is a list of its solutions:
    
    \begin{center}
        \begin{tabular}{cccc}
            \toprule
            $\alpha$ & type & first type & second type \\
            \midrule
            integer & Bessel functions & $J_\alpha$ & $Y_\alpha$ or $N_\alpha$ \\
            integer & Hankel functions & $H_{\alpha}^{(1)}=J_{\alpha}+\mathrm{i} Y_{\alpha}$ & $H_{\alpha}^{(2)}=J_{\alpha}-\mathrm{i} Y_{\alpha}$ \\
            \bottomrule
        \end{tabular}
    \end{center}

    $J_\alpha$ is well defined everywhere. $Y_\alpha$ diverges when $x \to 0$.

    Another interesting equation is
    \begin{equation}
        x^{2} \frac{\mathrm{d}^{2} y}{\mathrm{~d} x^{2}}+ 2x \frac{\mathrm{d} y}{\mathrm{~d} x}+\left(x^{2}-n(n+1)\right) y=0.
    \end{equation}
    The solutions are
    \begin{center}
        \begin{tabular}{cccc}
            \toprule
            $n$ & type & first type & second type \\
            \midrule
            integer & spherical Bessel functions & $j_n$ & $h_n$  \\
            integer & spherical Bessel functions & $h^{(1)}_n = j_n + \ii y_n$ & $h^{(2)}_n = j_n - \ii y_n$  \\
            \bottomrule
        \end{tabular}
    \end{center}
    
    We also need to work with modified Bessel equation 
    \begin{equation}
        x^{2} \frac{\mathrm{d}^{2} y}{\mathrm{~d} x^{2}}+x \frac{\mathrm{d} y}{\mathrm{~d} x}-\left(x^{2}+\alpha^{2}\right) y=0,
    \end{equation}
    which can be derived from \eqref{eq:bessel} with $x \to \ii x$. The solutions are 
    \begin{center}
        \begin{tabular}{cccc}
            \toprule
            $\alpha$ & type & first type & second type \\
            \midrule
            integer & Modified Bessel functions & $I_\alpha$ & $K_\alpha$ \\
            \bottomrule
        \end{tabular}
    \end{center}

    $I_\alpha$ is divergent as $x \to \infty$, and $K_\alpha$ is divergent when $x \to 0$.
\end{info*}

We first consider the case of a cylinder. We set 
\begin{equation}
    \vb*{c} = \vb*{\rho} = (x,y,0), \quad \div{\vb*{c}} = 2. 
\end{equation}
We need to solve \eqref{eq:scalar-helmholtz} for a cylinder system. By separation of variables we have 

Now we consider the case of a sphere. The equation is 
\begin{equation}
    \frac{1}{r^{2}} \frac{\partial}{\partial r}\left(r^{2} \frac{\partial \psi}{\partial r}\right)+\frac{1}{r^{2} \sin \theta} \frac{\partial}{\partial \theta}\left(\sin \theta \frac{\partial \psi}{\partial \theta}\right)+\frac{1}{r^{2} \sin ^{2} \theta} \frac{\partial^{2} \psi}{\partial \phi^{2}}+k^{2} \psi=0.
    \label{eq:spherical-wave}
\end{equation}
If we are dealing with a specific spherical cavity or something, $k$ is quantized: The boundary condition 
will force $k$ to take discrete values, just as is the case in the Hydrogen atom. We are, however, interested 
in a more generalized case, and therefore we just fix $k$, and solve the angular part of the problem.
In the following derivation, we use $n$ for $l$ in quantum mechanics. The radius part is 
\begin{equation}
    \frac{1}{r^{2}} \frac{\partial}{\partial r}\left(r^{2} \frac{\partial \psi}{\partial r}\right) + \left( k^2 - \frac{n(n+1)}{r^2} \right) \psi = 0,
\end{equation}
the solutions of which are the linear combination of 
\begin{equation}
    z_n(r) = j_n(r), j_n(r), h^{(1)}_n(r), h^{(2)}_n(r).
\end{equation}
The solution of the angular part is just the spherical harmonics. Since we are not dealing with a 
specific system, we ignore the normalization and the solutions of \eqref{eq:spherical-wave} are 
\begin{equation}
    \begin{aligned}
        &\psi_{e m n}(r, \theta, \phi)=z_{n}(k r) P_{n}^{m}(\cos \theta) \cos (m \phi), \\
        &\psi_{o m n}(r, \theta, \phi)=z_{n}(k r) P_{n}^{m}(\cos \theta) \sin (m \phi).
        \end{aligned}
\end{equation}
Here the subscripts $e$ and $o$ represent the even and odd properties in $\phi$.
We called $\vb*{M}$ and $\vb*{N}$ functions \concept{vector harmonics}, $\vb*{M}$ \concept{magnetic harmonics}, 
and $\vb*{N}$ \concept{electric harmonics}, because we usually make the following expansion
\begin{equation}
    E=\sum_{m=0}^{\infty} \sum_{n=m}^{\infty}\left(B_{e m n} M_{e m n}+B_{o m n} M_{o m n}+A_{e m n} N_{e m n}+A_{o m n} N_{o m n}\right).
\end{equation}

Note that when $n=0$, we have non-trivial solution of $\psi$, but both $\vb*{M}$ and $\vb*{N}$ are zero.
We can show this by a contradiction. Suppose $\vb*{M}_0$ is obtained by $\psi_{e00}$ or $\psi_{o00}$ or a 
linear combination. Since $\vb*{M}_0$ is perpendicular to $\vb*{c}$, we can see its values on a sphere 
a non-zero smooth vector field on $S^2$, without singularities (or otherwise due to the perfect rotational 
symmetry of the solution, singularities must be everywhere). The \emph{Hairy ball theorem} tells us 
this is impossible. 

Another noteworthy feature of the vector spherical functions are that they form a unambiguous definition 
of the electric and magnetic multipole expansion. The Taylor expansion definition, when the expansion order is high,
has ambiguity whether the moments are defined as ``primitive'' tensors found in Taylor coefficients 
or traceless tensors. This gives rise to some ``hidden'' toroidal moments \cite{toroidal}. 
There are not toroidal moments in the multipole expansion defined with vector harmonics. 
For example, a magnetic field where the magnetic field lines form circles will be classified as a toroidal 
dipole term, but in the multipole expansion defined with vector harmonics, it is just $\vb*{M}_{em1}$ or $\vb*{M}_{om1}$.

\section{Effective continuous medium}

Suppose $f(\vb*{r})$ is a filter or ``smoother''. Suppose the electric charge is 
\begin{equation}
    \eta(\vb*{r}) = \sum_n \sum_{i \in C_n} q_i \delta(\vb*{r} - \vb*{r}_i),
\end{equation}
where $n$ is the index of clusters of electrons, most frequently moleculars. The smoothened version is 
\[
    \begin{aligned}
        \expval{\eta}(\vb*{r}) &= \int \dd[3]{\vb*{r}'} \eta(\vb*{r} - \vb*{r}') f(\vb*{r}') \\
        &= \int \dd[3]{\vb*{r}'} \sum_n \sum_{i \in C_n} q_i \delta(\vb*{r} - \vb*{r}' - \vb*{r}_i) 
        \int \frac{\dd[3]{\vb*{k}}}{(2\pi)^3} f(\vb*{k}) \ee^{\ii \vb*{k} \cdot \vb*{r}'} \\
        &= \sum_n \int \frac{\dd[3]{\vb*{k}}}{(2\pi)^3} f(\vb*{k}) \sum_{i \in C_n} \ee^{\ii \vb*{k} \cdot (\vb*{r} - \vb*{r}_i)} \\
        &= \sum_n \int \frac{\dd[3]{\vb*{k}}}{(2\pi)^3} f(\vb*{k}) \ee^{\ii \vb*{k} \cdot (\vb*{r} - \vb*{r}_n)} \sum_{i \in C_n} \ee^{\ii \vb*{k} \cdot (\vb*{r}_n - \vb*{r}_i)} \\
        &= \sum_n \int \frac{\dd[3]{\vb*{k}}}{(2\pi)^3} f(\vb*{k}) \ee^{\ii \vb*{k} \cdot (\vb*{r} - \vb*{r}_n)} \sum_{i \in C_n} (1 - \ii \vb*{k} \cdot (\vb*{r}_i - \vb*{r}_n) + \cdots),
    \end{aligned}
\]
and if the perturbation of the charges is not strong, we can make a cutoff and get 
\[
    \begin{aligned}
        &\quad \sum_n \int \frac{\dd[3]{\vb*{k}}}{(2\pi)^3} f(\vb*{k}) \ee^{\ii \vb*{k} \cdot (\vb*{r} - \vb*{r}_n)} \sum_{i \in C_n} (1 - \ii \vb*{k} \cdot (\vb*{r}_i - \vb*{r}_n)) \\
        &= \sum_n \int \frac{\dd[3]{\vb*{k}}}{(2\pi)^3} f(\vb*{k}) \ee^{\ii \vb*{k} \cdot (\vb*{r} - \vb*{r}_n)} (q_n - \ii \vb*{k} \cdot \vb*{p}_n) \\
        &= \sum_n \left( q_n \int \frac{\dd[3]{\vb*{k}}}{(2\pi)^3} f(\vb*{k}) \ee^{\ii \vb*{k} \cdot (\vb*{r} - \vb*{r}_n)} - \vb*{p}_n \cdot \grad \int \frac{\dd[3]{\vb*{k}}}{(2\pi)^3} f(\vb*{k}) \ee^{\ii \vb*{k} \cdot (\vb*{r} - \vb*{r}_n)}  \right) \\
        &= \sum_n (q_n f(\vb*{r} - \vb*{r}_n) - \vb*{p}_n \cdot \grad f(\vb*{r} - \vb*{r}_n)) \\
        &= \sum_n (q_n f(\vb*{r} - \vb*{r}_n) - \div(\vb*{p}_n f(\vb*{r} - \vb*{r}_n))) .
    \end{aligned}
\]
Now we note that 
\[
    \expval{ \delta(\vb*{r} - \vb*{r}_n)} = f(\vb*{r} - \vb*{r}_n), 
\]
and we get 
\begin{equation}
    \expval{\eta}(\vb*{r}) = \rho(\vb*{r}) - \div{\vb*{P}(\vb*{r})},
\end{equation}
where 
\begin{equation}
    \rho(\vb*{r}) = \expval{\sum_n q_n \delta(\vb*{r} - \vb*{r}_n)}, \quad 
    \vb*{P}(\vb*{r}) = \expval{\sum_n \vb*{p}_n \delta(\vb*{r} - \vb*{r}_n)}   .
\end{equation}

TODO: magnetic average

\section{Constitutive relations}

\begin{equation}
    \vb*{D} = \epsilon_0 \vb*{E} + \vb*{P} = \epsilon_0 (1 + \chi_\text{e}) \vb*{E}, \quad 
    \vb*{H} = \frac{\vb*{B}}{\mu_0} - \vb*{M},
\end{equation}
\begin{equation}
    \div{\vb*{P}} = - \rho,
\end{equation}


\section{Frequency dispersion and the Clausius-Mossotti equation}

Now we work on a frequent mechanism of frequency dispersion, i.e. the response of the material depending on 
$\omega$. We consider a oscillator model, where the electrons can be seen as oscillators, 
the EOM of each of which is
\begin{equation}
    m \ddot{\vb*{r}} = - m \omega_0^2 \vb*{r} - m \gamma \vb*{r} + e \vb*{E}_\text{driving}.
\end{equation}
Note that though the electrons has no interaction, but first, since the electrons are immersed in the 
radiation created by themselves, $m$, $\gamma$, and $\omega_0$ will be modified, and second, an electron 
can feel the collective electric field of other electrons, and there must be a term for this in 
$\vb*{E}_\text{drive}$. Since $\vb*{E}_\text{other electrons} + \vb*{E}_\text{this electron}$ has no
long scale effects and only modifies $m$, $\gamma$, and $\omega_0$ (because they it modifies the lattice potential
and therefore change $\omega_0$, and it also modifies the phonon spectrum, thus modifying $\gamma$, and 
since we know the radiation of a moving charge has a term proportion to $\ddot{\vb*{d}}$, $m$ is also modified),
we can say that 
\[
    \vb*{E}_\text{driving} = \vb*{E}_\text{external} - \vb*{E}_\text{self}.
\]
Here and after we write $\vb*{E}_\text{external}$ as $\vb*{E}$, and we have 
\begin{equation}
    m \ddot{\vb*{r}} = - m \omega_0^2 \vb*{r} - m \gamma \vb*{r} + e (\vb*{E} - \vb*{E}_\text{self}).
\end{equation}
Note that this can also be seen as ``using $\vb*{E}_\text{self}$ to modify $m$, $\omega_0$ and $\gamma$ and
subtracting $\vb*{E}_\text{sef}$ from $\vb*{E}$''. A more generalized discussion of the modifications 
can be found in discussions around \eqref{optics-eq:local-electric-field-enhancement} in \opticsdoc.
What we are doing is actually a self-energy correction.

Suppose $\vb*{r} \propto \ee^{- \ii \omega t}$, and we have 
\begin{equation}
    e\vb*{r} \eqqcolon \vb*{d} = \alpha \vb*{E}, \quad \alpha = \frac{e^2}{m (\omega_0^2 - \omega^2 - \ii \gamma \omega)}.
    \label{eq:alpha-oscillators}
\end{equation}
The long range radiation is summarized as 
\begin{equation}
    \vb*{P} = N \vb*{d} = N \alpha (\vb*{E} - \vb*{E}_\text{self}),
\end{equation}
where $N$ is the number of electrons per volume unit, and if we approximate the self field as 
\begin{equation}
    \vb*{E}_\text{sef} = - \frac{\vb*{P}}{3 \epsilon_0},
\end{equation}
which is the field strength in a empty cavity in a homogeneously polarized insulator, and more generally, is the 
self field of a large number of dipoles (see \eqref{113-eq:self-field-dipole} in 
\href{lecture-11-3.pdf}{this note}), then we get 
\begin{equation}
    \vb*{P} = \frac{N \alpha}{1 - \frac{N \alpha}{3 \epsilon_0}} \vb*{E} \eqqcolon \epsilon_0 \chi_\text{e} \vb*{E},
\end{equation}
\begin{equation}
    \chi_\text{e} = \frac{N \alpha / \epsilon_0}{1 - N \alpha / (3 \epsilon_0)}, \quad \frac{\epsilon_\text{r} - 1}{\epsilon_\text{r} + 2} = \frac{N \alpha}{3 \epsilon_0}.
    \label{eq:clausius-mossotti}
\end{equation}
This is the so-called \concept{Clausius-Mossotti equations}.

Now we put \eqref{eq:alpha-oscillators} into \eqref{eq:clausius-mossotti}, and get 
\[
    \epsilon_\text{r} = 1 + \chi_\text{e} = 1 + \frac{1}{\left(\frac{N e^2}{m \epsilon_0}\right)^{-1} (\omega_0^2 - \omega^2 - \ii \gamma \omega) - \frac{1}{3}}, 
\]
and therefore after introducing the plasma frequency we get 
\begin{equation}
    \epsilon_\text{r} = 1 + \chi_\text{e} = 1 + \frac{\omega_\text{p}^2}{\tilde{\omega}_0^2 - \omega^2 - \ii \gamma \omega}, \quad \tilde{\omega}_0^2 = \omega_0^2 - \frac{1}{3} \omega_\text{p}^2, \quad \omega_\text{p}^2 = \frac{N e^2}{m \epsilon_0}.
    \label{eq:clausius-epsilon}
\end{equation}
This equation means a non-trivial photon band structure. Now we assume $\gamma = 0$, and because we have 
\begin{equation}
    k^2 = \left(\frac{\omega}{c}\right)^2 \epsilon_\text{r}(\omega),
\end{equation}
it is possible that $\epsilon_\text{r} < 0$ in some frequencies. The photon band structure, therefore, has 
a forbidden band. This is visualize in \prettyref{fig:clausius-epsilon}. It is easy to verify that 
\begin{equation}
    \omega_\text{L} = \sqrt{\tilde{\omega}_0^2 + \omega_\text{p}^2}, \quad \omega_\text{T} = \tilde{\omega}_0.
\end{equation}
The subscripts have meanings. When $\omega = \omega_\text{L}$, $\epsilon_\text{r} = 0$, and the equation
\[
    0 = \div{\vb*{D}} = \epsilon_\text{r} \div{\vb*{E}}
\]
is trivial, and it is possible that $\div{\vb*{E}} \neq 0$. So we see the subscript L means ``longitude mode''.
When $\omega = \omega_\text{T} = \tilde{\omega}_0$, the oscillators oscillate in its eigen mode and are coupled 
strongly to the electromagnetic field. Both of them vibrate transversely. Since there is no interaction between 
oscillators, the dispersion relation of density wave modes of these oscillators is a flat line, independent of 
$\vb*{k}$, with the frequency $\omega_\text{T}$. That is the meaning of the subscript T: the ``transverse density mode''.

\begin{figure}
    \centering
    

\tikzset{every picture/.style={line width=0.75pt}} %set default line width to 0.75pt        

\begin{tikzpicture}[x=0.75pt,y=0.75pt,yscale=-1,xscale=1]
%uncomment if require: \path (0,335); %set diagram left start at 0, and has height of 335

%Shape: Rectangle [id:dp07379263609898756] 
\draw  [draw opacity=0][fill={rgb, 255:red, 0; green, 0; blue, 0 }  ,fill opacity=0.13 ] (45.5,157.5) -- (584,157.5) -- (584,197.5) -- (45.5,197.5) -- cycle ;
%Straight Lines [id:da1768875834879553] 
\draw [color={rgb, 255:red, 0; green, 0; blue, 0 }  ,draw opacity=1 ]   (45.5,275.58) -- (260.5,275.58) ;
\draw [shift={(262.5,275.58)}, rotate = 180] [fill={rgb, 255:red, 0; green, 0; blue, 0 }  ,fill opacity=1 ][line width=0.08]  [draw opacity=0] (12,-3) -- (0,0) -- (12,3) -- cycle    ;
%Straight Lines [id:da8310117956241199] 
\draw [color={rgb, 255:red, 0; green, 0; blue, 0 }  ,draw opacity=1 ]   (108.5,275.58) -- (108.5,91.08) ;
\draw [shift={(108.5,89.08)}, rotate = 90] [fill={rgb, 255:red, 0; green, 0; blue, 0 }  ,fill opacity=1 ][line width=0.08]  [draw opacity=0] (12,-3) -- (0,0) -- (12,3) -- cycle    ;
%Straight Lines [id:da9608764760297932] 
\draw [color={rgb, 255:red, 0; green, 0; blue, 0 }  ,draw opacity=1 ]   (378.5,276) -- (593.5,276) ;
\draw [shift={(595.5,276)}, rotate = 180] [fill={rgb, 255:red, 0; green, 0; blue, 0 }  ,fill opacity=1 ][line width=0.08]  [draw opacity=0] (12,-3) -- (0,0) -- (12,3) -- cycle    ;
%Straight Lines [id:da575210054978492] 
\draw [color={rgb, 255:red, 0; green, 0; blue, 0 }  ,draw opacity=1 ]   (378.5,276) -- (378.5,91.5) ;
\draw [shift={(378.5,89.5)}, rotate = 90] [fill={rgb, 255:red, 0; green, 0; blue, 0 }  ,fill opacity=1 ][line width=0.08]  [draw opacity=0] (12,-3) -- (0,0) -- (12,3) -- cycle    ;
%Straight Lines [id:da6187444008907346] 
\draw  [dash pattern={on 0.84pt off 2.51pt}]  (45.5,197.5) -- (584,197.5) ;
%Straight Lines [id:da06319760130839658] 
\draw  [dash pattern={on 0.84pt off 2.51pt}]  (45.5,157.5) -- (584,157.5) ;
%Straight Lines [id:da4895790172476384] 
\draw [color={rgb, 255:red, 0; green, 0; blue, 0 }  ,draw opacity=1 ] [dash pattern={on 0.84pt off 2.51pt}]  (148.5,275.58) -- (148.5,89.08) ;
%Curve Lines [id:da09047241116285987] 
\draw [color={rgb, 255:red, 80; green, 227; blue, 194 }  ,draw opacity=1 ]   (378.5,276) .. controls (418.5,246) and (468,200.06) .. (579,200.06) ;
%Straight Lines [id:da8526784873506212] 
\draw [color={rgb, 255:red, 80; green, 227; blue, 194 }  ,draw opacity=1 ] [dash pattern={on 4.5pt off 4.5pt}]  (378.5,276) -- (581,120.06) ;
%Curve Lines [id:da12354963233094818] 
\draw [color={rgb, 255:red, 80; green, 227; blue, 194 }  ,draw opacity=1 ]   (148.5,275.58) .. controls (154,258.06) and (193,204.06) .. (258,201.06) ;
%Curve Lines [id:da38900509121065796] 
\draw [color={rgb, 255:red, 74; green, 144; blue, 226 }  ,draw opacity=1 ]   (45.5,194.5) .. controls (96,194.06) and (137,124.06) .. (145.5,89.08) ;
%Curve Lines [id:da4544972239093672] 
\draw [color={rgb, 255:red, 74; green, 144; blue, 226 }  ,draw opacity=1 ]   (378.5,157) .. controls (392,154.06) and (456,136.06) .. (483,85.06) ;
%Straight Lines [id:da8551657808511666] 
\draw [color={rgb, 255:red, 74; green, 144; blue, 226 }  ,draw opacity=1 ] [dash pattern={on 4.5pt off 4.5pt}]  (378.5,276) -- (486,86.06) ;
%Curve Lines [id:da12654594371430283] 
\draw    (541,72.06) .. controls (530.22,85.78) and (510.8,99.5) .. (481.79,107.58) ;
\draw [shift={(480,108.06)}, rotate = 345.07] [fill={rgb, 255:red, 0; green, 0; blue, 0 }  ][line width=0.08]  [draw opacity=0] (12,-3) -- (0,0) -- (12,3) -- cycle    ;
%Curve Lines [id:da17190727595919086] 
\draw    (622,106.06) .. controls (611.22,119.78) and (591.8,133.5) .. (562.79,141.58) ;
\draw [shift={(561,142.06)}, rotate = 345.07] [fill={rgb, 255:red, 0; green, 0; blue, 0 }  ][line width=0.08]  [draw opacity=0] (12,-3) -- (0,0) -- (12,3) -- cycle    ;

% Text Node
\draw (264.5,275.58) node [anchor=west] [inner sep=0.75pt]    {$\Re \epsilon _{\text{r}}$};
% Text Node
\draw (597.5,276) node [anchor=west] [inner sep=0.75pt]    {$k$};
% Text Node
\draw (106.5,85.68) node [anchor=south east] [inner sep=0.75pt]    {$\omega $};
% Text Node
\draw (376.5,86.1) node [anchor=south east] [inner sep=0.75pt]    {$\omega $};
% Text Node
\draw (148.5,278.58) node [anchor=north] [inner sep=0.75pt]   [align=left] {1};
% Text Node
\draw (43.5,157.5) node [anchor=east] [inner sep=0.75pt]    {$\omega _{\text{L}}$};
% Text Node
\draw (43.5,197.5) node [anchor=east] [inner sep=0.75pt]    {$\omega _{\text{T}}$};
% Text Node
\draw (506,43.4) node [anchor=north west][inner sep=0.75pt]  [color={rgb, 255:red, 0; green, 0; blue, 0 }  ,opacity=1 ]  {$ck/\sqrt{\epsilon ( \infty )}$};
% Text Node
\draw (574,80.4) node [anchor=north west][inner sep=0.75pt]  [color={rgb, 255:red, 0; green, 0; blue, 0 }  ,opacity=1 ]  {$ck/\sqrt{\epsilon ( 0)}$};
% Text Node
\draw (183,166.4) node [anchor=north west][inner sep=0.75pt]    {$\epsilon _{\text{r}} < 0$};


\end{tikzpicture}

    \caption{The photon band structure of \eqref{eq:clausius-epsilon}}
    \label{fig:clausius-epsilon}
\end{figure}

\bibliographystyle{plain}
\bibliography{ref} 

\end{document}