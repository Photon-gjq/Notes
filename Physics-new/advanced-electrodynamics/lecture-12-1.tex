\documentclass[hyperref, a4paper]{article}

\usepackage{geometry}
\usepackage{titling}
\usepackage{titlesec}
% No longer needed, since we will use enumitem package
% \usepackage{paralist}
\usepackage{enumitem}
\usepackage{footnote}
\usepackage{enumerate}
\usepackage{amsmath, amssymb, amsthm}
\usepackage{mathtools}
\usepackage{bbm}
\usepackage{cite}
\usepackage{graphicx}
\usepackage{subfigure}
\usepackage{physics}
\usepackage{tensor}
\usepackage{siunitx}
\usepackage[version=4]{mhchem}
\usepackage{tikz}
\usepackage{xcolor}
\usepackage{listings}
\usepackage{autobreak}
\usepackage[ruled, vlined, linesnumbered]{algorithm2e}
\usepackage{nameref,zref-xr}
\zxrsetup{toltxlabel}
\zexternaldocument*[optics-]{../optics/optics}[optics.pdf]
\zexternaldocument*[solid-]{../solid/solid}[solid.pdf]
\usepackage[colorlinks,unicode]{hyperref} % , linkcolor=black, anchorcolor=black, citecolor=black, urlcolor=black, filecolor=black
\usepackage[most]{tcolorbox}
\usepackage{prettyref}

% Page style
\geometry{left=3.18cm,right=3.18cm,top=2.54cm,bottom=2.54cm}
\titlespacing{\paragraph}{0pt}{1pt}{10pt}[20pt]
\setlength{\droptitle}{-5em}
\preauthor{\vspace{-10pt}\begin{center}}
\postauthor{\par\end{center}}

% More compact lists 
\setlist[itemize]{
    itemindent=17pt, 
    leftmargin=1pt,
    listparindent=\parindent,
    parsep=0pt,
}

% Math operators
\DeclareMathOperator{\timeorder}{\mathcal{T}}
\DeclareMathOperator{\diag}{diag}
\DeclareMathOperator{\legpoly}{P}
\DeclareMathOperator{\primevalue}{P}
\DeclareMathOperator{\sgn}{sgn}
\newcommand*{\ii}{\mathrm{i}}
\newcommand*{\ee}{\mathrm{e}}
\newcommand*{\const}{\mathrm{const}}
\newcommand*{\suchthat}{\quad \text{s.t.} \quad}
\newcommand*{\argmin}{\arg\min}
\newcommand*{\argmax}{\arg\max}
\newcommand*{\normalorder}[1]{: #1 :}
\newcommand*{\pair}[1]{\langle #1 \rangle}
\newcommand*{\fd}[1]{\mathcal{D} #1}
\DeclareMathOperator{\bigO}{\mathcal{O}}

% TikZ setting
\usetikzlibrary{arrows,shapes,positioning}
\usetikzlibrary{arrows.meta}
\usetikzlibrary{decorations.markings}
\tikzstyle arrowstyle=[scale=1]
\tikzstyle directed=[postaction={decorate,decoration={markings,
    mark=at position .5 with {\arrow[arrowstyle]{stealth}}}}]
\tikzstyle ray=[directed, thick]
\tikzstyle dot=[anchor=base,fill,circle,inner sep=1pt]

% Algorithm setting
% Julia-style code
\SetKwIF{If}{ElseIf}{Else}{if}{}{elseif}{else}{end}
\SetKwFor{For}{for}{}{end}
\SetKwFor{While}{while}{}{end}
\SetKwProg{Function}{function}{}{end}
\SetArgSty{textnormal}

\newcommand*{\concept}[1]{{\textbf{#1}}}

% Embedded codes
\lstset{basicstyle=\ttfamily,
  showstringspaces=false,
  commentstyle=\color{gray},
  keywordstyle=\color{blue}
}

% Reference formatting
\newrefformat{fig}{Figure~\ref{#1} on page~\pageref{#1}}

% Color boxes
\tcbuselibrary{skins, breakable, theorems}
\newtcbtheorem[number within=section]{warning}{Warning}%
  {colback=orange!5,colframe=orange!65,fonttitle=\bfseries, breakable}{warn}
\newtcbtheorem[number within=section]{note}{Note}%
  {colback=green!5,colframe=green!65,fonttitle=\bfseries, breakable}{note}

\title{Diffraction and Scattering in Electrodynamics by Prof. Kun Din}
\author{Jinyuan Wu}
\date{December 1, 2021}

\begin{document}

\maketitle

As was said in \href{lecture-11-24.pdf}{the previous lecture}, there is no rigid distinction between  
scattering and diffraction. We can say scattering reflects the particle nature of light, while 
diffraction reflects the wave nature of light, but scattering can be derived from the ``medium in 
light'' picture, which utilizes Maxwell equations about \emph{waves}, while diffraction involves 
things like aperture, i.e. boundary conditions, which may be seen as scattering. 

In \href{lecture-11-24.pdf}{the previous lecture} we derived scattering, absorption and extinction 
cross sections, the optical theorem from the conservation of energy, and we also discussed scattering 
and absorption efficiency. We discussed the geometrical optics.

\section{Mie scattering}

\begin{figure}
    \centering
    

\tikzset{every picture/.style={line width=0.75pt}} %set default line width to 0.75pt        

\begin{tikzpicture}[x=0.75pt,y=0.75pt,yscale=-1,xscale=1]
%uncomment if require: \path (0,300); %set diagram left start at 0, and has height of 300

%Straight Lines [id:da20104584041056972] 
\draw    (232.66,178.01) -- (162.74,247.93) ;
\draw [shift={(161.33,249.34)}, rotate = 315] [fill={rgb, 255:red, 0; green, 0; blue, 0 }  ][line width=0.08]  [draw opacity=0] (12,-3) -- (0,0) -- (12,3) -- cycle    ;
%Straight Lines [id:da7424478890669008] 
\draw    (232.66,178.01) -- (387.02,178.01) ;
\draw [shift={(389.02,178.01)}, rotate = 180] [fill={rgb, 255:red, 0; green, 0; blue, 0 }  ][line width=0.08]  [draw opacity=0] (12,-3) -- (0,0) -- (12,3) -- cycle    ;
%Straight Lines [id:da7190301587069228] 
\draw    (232.66,178.01) -- (232.66,49.34) ;
\draw [shift={(232.66,47.34)}, rotate = 90] [fill={rgb, 255:red, 0; green, 0; blue, 0 }  ][line width=0.08]  [draw opacity=0] (12,-3) -- (0,0) -- (12,3) -- cycle    ;
%Shape: Circle [id:dp02220990844122195] 
\draw  [fill={rgb, 255:red, 255; green, 0; blue, 0 }  ,fill opacity=0.24 ] (188.06,178.01) .. controls (188.06,153.38) and (208.03,133.41) .. (232.66,133.41) .. controls (257.29,133.41) and (277.26,153.38) .. (277.26,178.01) .. controls (277.26,202.64) and (257.29,222.61) .. (232.66,222.61) .. controls (208.03,222.61) and (188.06,202.64) .. (188.06,178.01) -- cycle ;
%Shape: Ellipse [id:dp31691702644238307] 
\draw  [dash pattern={on 4.5pt off 4.5pt}] (188.06,178.01) .. controls (188.06,167.15) and (208.03,158.34) .. (232.66,158.34) .. controls (257.29,158.34) and (277.26,167.15) .. (277.26,178.01) .. controls (277.26,188.87) and (257.29,197.67) .. (232.66,197.67) .. controls (208.03,197.67) and (188.06,188.87) .. (188.06,178.01) -- cycle ;
%Straight Lines [id:da25526492267747836] 
\draw [color={rgb, 255:red, 0; green, 0; blue, 0 }  ,draw opacity=0.41 ]   (232.66,178.01) -- (251.78,140.13) ;
\draw [shift={(252.68,138.34)}, rotate = 116.78] [color={rgb, 255:red, 0; green, 0; blue, 0 }  ,draw opacity=0.41 ][line width=0.75]    (10.93,-3.29) .. controls (6.95,-1.4) and (3.31,-0.3) .. (0,0) .. controls (3.31,0.3) and (6.95,1.4) .. (10.93,3.29)   ;
%Straight Lines [id:da00970045585010526] 
\draw    (20.67,146) -- (116.68,146) ;
\draw [shift={(68.67,146)}, rotate = 180] [fill={rgb, 255:red, 0; green, 0; blue, 0 }  ][line width=0.08]  [draw opacity=0] (12,-3) -- (0,0) -- (12,3) -- cycle    ;
%Straight Lines [id:da22731336451330875] 
\draw    (20.67,166) -- (119.68,166) ;
\draw [shift={(70.17,166)}, rotate = 180] [fill={rgb, 255:red, 0; green, 0; blue, 0 }  ][line width=0.08]  [draw opacity=0] (12,-3) -- (0,0) -- (12,3) -- cycle    ;
%Straight Lines [id:da9547860380958502] 
\draw    (21.67,184) -- (117.68,184) ;
\draw [shift={(69.67,184)}, rotate = 180] [fill={rgb, 255:red, 0; green, 0; blue, 0 }  ][line width=0.08]  [draw opacity=0] (12,-3) -- (0,0) -- (12,3) -- cycle    ;
%Straight Lines [id:da1868394895374128] 
\draw    (100,119) -- (121.02,119) ;
\draw [shift={(123.02,119)}, rotate = 180] [color={rgb, 255:red, 0; green, 0; blue, 0 }  ][line width=0.75]    (10.93,-3.29) .. controls (6.95,-1.4) and (3.31,-0.3) .. (0,0) .. controls (3.31,0.3) and (6.95,1.4) .. (10.93,3.29)   ;
%Shape: Wave [id:dp053768706962373125] 
\draw  [color={rgb, 255:red, 248; green, 231; blue, 28 }  ,draw opacity=1 ] (11.33,164.5) .. controls (15.41,175.52) and (19.31,186.01) .. (23.83,186.01) .. controls (28.36,186.01) and (32.26,175.52) .. (36.33,164.5) .. controls (40.41,153.49) and (44.31,143) .. (48.83,143) .. controls (53.36,143) and (57.26,153.49) .. (61.33,164.5) .. controls (65.41,175.52) and (69.31,186.01) .. (73.83,186.01) .. controls (78.36,186.01) and (82.26,175.52) .. (86.33,164.5) .. controls (90.41,153.49) and (94.31,143) .. (98.83,143) .. controls (102.99,143) and (106.62,151.86) .. (110.35,161.85) ;

% Text Node
\draw (232.66,43.94) node [anchor=south] [inner sep=0.75pt]    {$x$};
% Text Node
\draw (159.33,252.74) node [anchor=north east] [inner sep=0.75pt]    {$y$};
% Text Node
\draw (391.02,178.01) node [anchor=west] [inner sep=0.75pt]    {$z$};
% Text Node
\draw (250,135) node [anchor=south west] [inner sep=0.75pt]    {$a$};
% Text Node
\draw (108,88.07) node [anchor=north west][inner sep=0.75pt]    {$\hat{\boldsymbol{k}}$};


\end{tikzpicture}

    \caption{Mie scattering}
\end{figure}

We continue the discussion on Mie scattering.
\concept{Mie scattering} is among few examples that can be solved exactly. It studies a sphere made of 
dielectric. The scattered fields at infinity are expanded using spherical functions:
\begin{equation}
    \vb*{E}_\text{s} = \sum_{n=1}^\infty E_n (\ii a_n \vb*{N}^{(3)}_{e1n} - b_n \vb*{M}^{(3)}_{o1n}), \quad 
    \vb*{H}_\text{s} = \frac{k}{\omega \mu} \sum_{n=1}^\infty E_n (\ii b_n \vb*{N}^{(3)}_{o1n} + a_n \vb*{M}^{(3)}_{e1n}).
    \label{eq:scattered-beam}
\end{equation}
The coefficients $a_n$ and $b_n$ are called \concept{Mie coefficients}. We will find that the $a_n$ coefficients
give the response of electric $n$-poles while the $b_n$ coefficients give the response of magnetic $n$-poles.
The input light beam is described by 
\begin{equation}
    \vb*{E}_\text{i} = E_0 \sum_{n=1}^\infty \ii^n \frac{2n+1}{n(n+1)} (\vb*{M}^{(1)}).
    \label{eq:input-beam}
\end{equation}
We label the electric and magnetic fields inside the sphere as $\vb*{E}_1$ and $\vb*{B}_1$, respectively.

We introduce several notations. First we introduce the Legendre polynomials
\begin{equation}
    \legpoly^m_n(x) = (1 - x)^{m/2} \dv[m]{\legpoly_n(x)}{x}.
\end{equation} 
Note that Jackson, Zangwill and Mathematica add an additional $(-1)^m$ factor to the definition.
Furthermore we define 
\begin{equation}
    \pi_n(\cos \theta) = \frac{\legpoly_n(\cos \theta)}{\sin \theta}, \quad 
    \tau_n(\cos \theta) = \dv{\legpoly_n (\cos \theta)}{\theta}.
\end{equation}
We have several useful formulae about these functions. First are recurrence formulae:
\begin{equation}
    \pi_n(\nu) = \frac{2n-1}{n-1} \nu \pi_{n-1}(\nu) - \frac{n}{n-1} \pi_{n-2}(\nu),
\end{equation}
\begin{equation}
    \tau_n(\nu) = n \nu \pi_n(\nu) - (n+1) \pi_{n-1}(\nu),
\end{equation}
and 
\begin{equation}
    \pi_0 = 0, \quad \pi_1 = 1.
\end{equation}
We also have 
\begin{equation}
    \pi_n(-\nu) = (-1)^{n-1} \pi_n(\nu), \quad \tau_n(-\nu) = (-1)^n \tau_n(\nu).
\end{equation}
The orthogonality relations are 
\begin{equation}
    \int_0^\pi \sin \theta \dd{\theta} (\tau_n + \pi_n) (\tau_m + \pi_m) = 0, \quad 
    \int_0^\pi \sin \theta \dd{\theta}
\end{equation}

We compare \eqref{eq:input-beam} and \eqref{eq:scattered-beam} on the $r=a$ surface. 
The boundary conditions are 
\[
    (\vb*{E}_\text{i} + \vb*{E}_\text{s} - \vb*{E}_1) \times \vu*{e}_r = 0, \quad 
    (\vb*{H}_\text{i} + \vb*{H}_\text{s} - \vb*{H}_1) \times \vu*{e}_r = 0.
\]
We have 
% long and boring coefficient calculation
We get the final results 
\begin{equation}
    \begin{aligned}
        a_n &= \frac{\mu m^2 j_n(mx) (x j_n(x))' - \mu_1 j_n(x) (mx j_n(mx))'}{
        \mu m^2 j_n(mx) (x h^{(1)}_n(x))' - \mu_1 h_n^{(1)}(x) (mx j_n(mx))'}, \\
        b_n &= \frac{\mu_1 j_n(mx) (x j(x))' - \mu j_n(x) (mx j_n(mx))'}{
        \mu_1 j_n(mx) (x h^{(1)}_n(x))' - \mu h_n^{(1)}(x) (mx j_n(mx))' }.
    \end{aligned}
\end{equation}
We find that the denominator of $a_n$ may be zero, which gives the eigenmodes of the system.
Finding these modes is extremely hard. The behavior around poles is highly nonlinear, and ordinary 
gradient descent methods have severe divergence problems. Sometimes the poles are close to each other 
and it is almost impossible to distinguish them. Even when these problems are solved, whether we 
have already found a complete set of eigenmodes is still a question hard to answer. 
This topic - finding the poles of a scattering matrix - is still a frontline nowadays.

In the $ka \ll 1$ limit, $x \ll 1$, and the denominator of $a_n$ is 
\begin{equation}
    \begin{aligned}
        f_E(\omega, n) &= \frac{\epsilon_1 \mu_1}{\epsilon} \frac{(mx)^n}{(2n+1) !!} \ii \frac{(2n-1)!!}{x^{n+1}} n
        + \mu_1 \ii \frac{(2n-1)!!}{x^{n+1}} \frac{(mx)^n}{(2n+1)!!} (1+n) \\
        &= \frac{\mu_1}{\epsilon} \frac{\ii m^n}{(2n+1) x} (\epsilon_1 n + \epsilon (n+1)),
    \end{aligned}
\end{equation}
and its zero point is given by 
\begin{equation}
    \frac{\epsilon_1(\omega, n)}{\epsilon} = - \frac{n+1}{n}.
\end{equation}
This equation gives the ``electric'' eigenmodes. For example, for a metal sphere, we have 
\begin{equation}
    \omega_1 = \frac{\omega_\text{p}}{\sqrt{3}}. 
\end{equation}
This is called \concept{local surface plasmon polariton}. The term ``plasmon'' comes from the fact that 
this mode involves charge fluctuation in the metal, while the term ``polariton' comes from the fact that 
this mode involves electromagnetic waves in the air (though Mie scattering also works in vacuum where 
there is no polarization).

Now we derive the energy flow in Mie scattering. Since we need to handle the fields directly, 
we define \concept{Riccati-Bessel functions} as follows:
\begin{equation}
    \begin{aligned}
        \psi_n(\rho) &= \rho j_n(\rho) = S_n(\rho), \\
        \chi_n(\rho) &= - \rho y_n(\rho) = C_n(\rho), \\
        \xi_n(\rho) &= \rho h_n^{(1)}(\rho) = \psi_n - \ii \chi_n, \\
        \zeta_n(\rho) &= \rho h_n^{(2)}(\rho) = \psi_n + \ii \chi_n,
    \end{aligned}
\end{equation}
and now the scattering field can be written in one line as 
\begin{equation}
    \begin{aligned}
        E_{\text{s}, \theta} &= \frac{\cos \phi}{\rho} \sum_n E_n (\ii a_n \xi_n' \tau_n - b_n \chi_n \pi_n) , 
    \end{aligned}
\end{equation}
where the $^\prime$ superscript means the argument is $r$ and the special functions without prime superscript
have $\theta$ as arguments. 
\begin{equation}
    C_\text{sca} = \frac{2\pi}{k^2} \sum_{n=1}^\infty (2n+1) (\abs*{a_n}^2 + \abs*{b_n}^2),
\end{equation}
\begin{equation}
    C_\text{ext} = \frac{2\pi}{k^2} \sum_{n=1}^\infty (2n+1) \Re (a_n + b_n).
\end{equation}

After calculating everything numerically, we find that Mie scattering agrees with Rayleigh scattering when 
$k a \ll 1$, which, in the language of Mie scattering, can be described as throwing away all $b_n$'s and only 
keep $a_1$, or in other words treating the sphere as an electric dipole. In this case we have 
\begin{equation}
    Q_\text{sca} = \frac{8}{3} (ka)^4 \left(\frac{\epsilon_1 - \epsilon}{\epsilon+1 + 2 \epsilon_2}\right)^2.
\end{equation}
The $(ka)^4$ dependence is the typical feature of Rayleigh scattering. This can be seen as 
\begin{equation}
    \alpha = 4 \pi \epsilon_0 a^3 \frac{\epsilon_1 - \epsilon}{\epsilon_1 + 2 \epsilon}.
\end{equation}
On the other limit we have geometrical optics. 

The cross sections can also be evaluated via \concept{partial-wave expansion}, which is often used in 
quantum mechanical scattering problems. 

\begin{equation}
    2 \delta_k = \Delta \sin \alpha.
\end{equation}

\begin{equation}
    f(0) = 2 \pi a^2 \left( 1 - \frac{2}{\Delta} \sin \Delta + \frac{2}{\Delta^2} (1 - \cos \Delta) \right).
\end{equation}
We find that the large oscillating period can be easily captured if we consider the phase of each partial 
wave. In other words, the large oscillating period is almost simply the interference of the components in 
the scattering beam. On the other hand, the small oscillating period comes from the details of the sphere, 
or to be exact from the eigenmodes of the system. If we change $a$, there will not be significant change 
on the large oscillating period, because we will only see change on $\Delta$. 

\end{document}