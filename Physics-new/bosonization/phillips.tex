\documentclass[hyperref, a4paper]{article}

\usepackage{geometry}
\usepackage{marginnote}
\usepackage{titling}
\usepackage{titlesec}
% No longer needed, since we will use enumitem package
% \usepackage{paralist}
\usepackage{enumitem}
\usepackage{footnote}
\usepackage{enumerate}
\usepackage{amsmath, amssymb, amsthm}
\usepackage{mathtools}
\usepackage{bbm}
\usepackage{cite}
\usepackage{graphicx}
\usepackage{subfigure}
\usepackage{physics}
\usepackage{tensor}
\usepackage{siunitx}
\usepackage{slashed}
\usepackage{centernot}
\usepackage[version=4]{mhchem}
\usepackage{tikz}
\usepackage{xcolor}
\usepackage{listings}
\usepackage{autobreak}
\usepackage[ruled, vlined, linesnumbered]{algorithm2e}
\usepackage{nameref,zref-xr}
\zxrsetup{toltxlabel}
\zexternaldocument*[solid-]{../solid/solid}[solid.pdf]
\zexternaldocument*[electrongas-]{../band-metal-insulator/electron-gas}[electron-gas.pdf]
\usepackage[colorlinks,unicode]{hyperref} % , linkcolor=black, anchorcolor=black, citecolor=black, urlcolor=black, filecolor=black
\usepackage[most]{tcolorbox}
\usepackage{prettyref}

% Page style
\geometry{left=3.18cm,right=3.18cm,top=2.54cm,bottom=2.54cm}
\titlespacing{\paragraph}{0pt}{1pt}{10pt}[20pt]
\setlength{\droptitle}{-5em}
\preauthor{\vspace{-10pt}\begin{center}}
\postauthor{\par\end{center}}

% More compact lists 
%\setlist[itemize]{
%    itemindent=17pt, 
%    leftmargin=1pt,
%    listparindent=\parindent,
%    parsep=0pt,
%}

% Math operators
\DeclareMathOperator{\timeorder}{\mathcal{T}}
\DeclareMathOperator{\diag}{diag}
\DeclareMathOperator{\legpoly}{P}
\DeclareMathOperator{\primevalue}{P}
\DeclareMathOperator{\sgn}{sgn}
\newcommand*{\ii}{\mathrm{i}}
\newcommand*{\ee}{\mathrm{e}}
\newcommand*{\const}{\mathrm{const}}
\newcommand*{\suchthat}{\quad \text{s.t.} \quad}
\newcommand*{\argmin}{\arg\min}
\newcommand*{\argmax}{\arg\max}
\newcommand*{\normalorder}[1]{: #1 :}
\newcommand*{\pair}[1]{\langle #1 \rangle}
\newcommand*{\fd}[1]{\mathcal{D} #1}
\DeclareMathOperator{\bigO}{\mathcal{O}}

% Feynman slash
\newcommand{\fsl}[1]{{\centernot{#1}}}

% TikZ setting
\usetikzlibrary{arrows,shapes,positioning}
\usetikzlibrary{arrows.meta}
\usetikzlibrary{decorations.markings}
\tikzstyle arrowstyle=[scale=1]
\tikzstyle directed=[postaction={decorate,decoration={markings,
    mark=at position .5 with {\arrow[arrowstyle]{stealth}}}}]
\tikzstyle ray=[directed, thick]
\tikzstyle dot=[anchor=base,fill,circle,inner sep=1pt]

% Algorithm setting
% Julia-style code
\SetKwIF{If}{ElseIf}{Else}{if}{}{elseif}{else}{end}
\SetKwFor{For}{for}{}{end}
\SetKwFor{While}{while}{}{end}
\SetKwProg{Function}{function}{}{end}
\SetArgSty{textnormal}

\newcommand*{\concept}[1]{{\textbf{#1}}}

% Embedded codes
\lstset{basicstyle=\ttfamily,
  showstringspaces=false,
  commentstyle=\color{gray},
  keywordstyle=\color{blue}
}

% Reference formatting
\newrefformat{fig}{Figure~\ref{#1} on page~\pageref{#1}}

% Color boxes
\tcbuselibrary{skins, breakable, theorems}
\newtcbtheorem[number within=section]{warning}{Warning}%
  {colback=orange!5,colframe=orange!65,fonttitle=\bfseries, breakable}{warn}
\newtcbtheorem[number within=section]{note}{Note}%
  {colback=green!5,colframe=green!65,fonttitle=\bfseries, breakable}{note}
\newtcbtheorem[number within=section]{info}{Info}%
  {colback=blue!5,colframe=blue!65,fonttitle=\bfseries, breakable}{info}

\title{Bosonization in Electronic Systems}
\author{Jinyuan Wu}

\begin{document}
    
\maketitle

We have already discussed bosonization in Section.~\ref{solid-sec:one-dim-free} in 
\href{../solid/solid.pdf}{this solid state physics note}.
This article discusses some details in bosonization, summarizing important facts in famous textbooks and papers.
Bosonization can be viewed as one way to exactly solve a system, which we will see in Luttinger liquid and
bosonization of spin systems. A more ``physical'' motivation is to capture the bosonic modes in the system,
i.e. density and current fluctuations, or since we usually only deal with low-energy perturbations in 
condensed matter systems, to capture \emph{sound waves} \cite{Tomonaga1950RemarksOB,10.1143/PTPS.170.185}. \marginnote{Philip Phillips Chapter 10}
This is why the fields after bosonization are all density modes: We really do not know what to do other than
this. Sometimes, bosonization are known as the \emph{hydrodynamical approach} (for example in Wen's famous 
textbook), which reveals its physical nature. This name is slightly misleading, as \emph{hydrodynamical approaches}
often denote more ``kinetic'' approaches, which is more about deriving EOMs of expectation values instead of 
\marginnote{Wen Section 5.1.3, 5.3.3 and 5.3.4}
quantum field operators. This physical picture can also be justified using the following line of thinking: 
one motivation to study the electronic structure of a condensed matter system is to find its electromagnetic 
response, and $A_\mu$ is always coupled to a bilinear of the fermionic fields. What can be revealed directly 
by electromagnetic responses, therefore, are only bosonic modes (for example see Section~\ref{electrongas-sec:boson-modes} \href{../band-metal-insulator/electron-gas.pdf}{here}).

\section{Luttinger liquid from a Hubbard model}

\marginnote{Philip Phillips Section 10.1}

\section{The Luttinger model}

\bibliographystyle{plain}
\bibliography{../formalism/fluid-quantum,bosonization,../solid/fermi-liquid.bib} 

\end{document}