\documentclass[hyperref, UTF8, a4paper]{ctexart}

\usepackage{geometry}
\usepackage{titling}
\usepackage{titlesec}
\usepackage{paralist}
\usepackage{footnote}
\usepackage{enumerate}
\usepackage{autobreak}
\usepackage{amsmath, amssymb, amsthm}
\usepackage{mathtools}
\usepackage{bbm}
\usepackage[superscript]{cite}
\usepackage{graphicx}
\usepackage{subfigure}
\usepackage{physics}
\usepackage{siunitx}
\usepackage{tikz}
\usepackage[compat=1.1.0]{tikz-feynhand}
\usepackage[ruled, vlined, linesnumbered, noend]{algorithm2e}
\usepackage{xr-hyper}
\usepackage[colorlinks, linkcolor=black, anchorcolor=black, citecolor=black, filecolor=black]{hyperref}
\usepackage[most]{tcolorbox}
\usepackage{caption}
\usepackage{prettyref}

% Cite: superscript, [1]
\makeatletter
\renewcommand\@citess[1]{\textsuperscript{[#1]}}
\makeatother

\externaldocument[optics-]{../optics/optics}[optics.pdf]
\externaldocument[vasp-]{../cond-comp/vasp/vasp}[vasp.pdf]
\externaldocument[qft-]{../relativistic-qft/relativistic-qft}[relativistic-qft.pdf]
\externaldocument[soft-]{../soft/soft}[soft.pdf]

\geometry{left=3.18cm,right=3.18cm,top=2.54cm,bottom=2.54cm}
\titlespacing{\paragraph}{0pt}{1pt}{10pt}[20pt]
\setlength{\droptitle}{-5em}
\preauthor{\vspace{-10pt}\begin{center}}
\postauthor{\par\end{center}}

\DeclareMathOperator{\timeorder}{\mathcal{T}}
\DeclareMathOperator{\diag}{diag}
\DeclareMathOperator{\legpoly}{P}
\DeclareMathOperator{\primevalue}{P}
\DeclareMathOperator{\sgn}{sgn}
\newcommand*{\ii}{\mathrm{i}}
\newcommand*{\ee}{\mathrm{e}}
\newcommand*{\const}{\mathrm{const}}
\newcommand*{\suchthat}{\quad \text{s.t.} \quad}
\newcommand*{\argmin}{\arg\min}
\newcommand*{\argmax}{\arg\max}
\newcommand*{\normalorder}[1]{: #1 :}
\newcommand*{\pair}[1]{\langle #1 \rangle}
\newcommand*{\fd}[1]{\mathcal{D} #1}

\newrefformat{chap}{第\ref{#1}章}
\newrefformat{sec}{第\ref{#1}节}
\newrefformat{note}{注\ref{#1}}
\newrefformat{fig}{图\ref{#1}}
\newrefformat{alg}{算法\ref{#1}}
\renewcommand{\autoref}{\prettyref}

\usetikzlibrary{arrows,shapes,positioning}
\usetikzlibrary{arrows.meta}
\usetikzlibrary{decorations.markings}
\tikzstyle arrowstyle=[scale=1]
\tikzstyle directed=[postaction={decorate,decoration={markings,
    mark=at position .5 with {\arrow[arrowstyle]{stealth}}}}]
\tikzstyle ray=[directed, thick]
\tikzstyle dot=[anchor=base,fill,circle,inner sep=1pt]

% Algorithm setting
\renewcommand{\algorithmcfname}{算法}
% Python-style code
\SetKwIF{If}{ElseIf}{Else}{if}{:}{elif:}{else:}{}
\SetKwFor{For}{for}{:}{}
\SetKwFor{While}{while}{:}{}
\SetKwInput{KwData}{输入}
\SetKwInput{KwResult}{输出}
\SetArgSty{textnormal}

\tcbuselibrary{skins, breakable, theorems}

\renewcommand{\emph}[1]{\textbf{#1}}
\newcommand*{\concept}[1]{\underline{\textbf{#1}}}

\newcommand{\hmn}[1]{% Hermann-Maguin notation
  \ensuremath{\begingroup\setupHMN #1\endgroup}%
}

\newcommand{\setupHMN}{%
  \doHMN{-}{\HMNoverline}%
  \doHMN{*}{\HMNminverse}%
  \doHMN{i}{\infty}
}

\newcommand{\doHMN}[2]{%
  \begingroup\lccode`~=`#1
  \lowercase{\endgroup\let~}#2%
  \mathcode`#1="8000
}

\newcommand{\HMNminverse}[1]{\frac{#1}{m}}
\newcommand{\HMNoverline}[1]{\mkern1mu\overline{\mkern-1mu#1\mkern-1mu}\mkern1mu}

\newcommand{\Ztwo}{$\mathbb{Z}_2$}

\newcommand{\bigO}[1]{\mathcal{O}(#1)}

\title{准晶}
\author{吴晋渊 18307110155}
\date{}

\begin{document}

\maketitle

准晶的发现可以追溯到Shechtman等报道的快速冷却的Al-Mn合金的衍射图样中观察到的正二十面体对称性\cite{PhysRevLett.53.1951}。
正二十面体对称性包含一个$C_5$对称轴,而不可能有一个晶格具有这种类型的对称性——它不在晶体允许的32种点群中\cite{Johnston_1960}。

\section{准晶的几何结构}



\section{准晶相变的金斯堡-朗道理论}

本节将以文献\cite{PhysRevB.32.5764}为例,介绍准晶相变的金斯堡-朗道理论。
我们采用金斯堡-朗道理论的标准处理方法,假定系统的状态可以使用一个空间中的连续、平滑的序参量描述,系统的行为可以使用一个仅仅关于序参量的自由能完整描述,通过对称性写下自由能的形式,并分析自由能中各参数变动时系统是否发生对称性自发破缺,以及发生后系统基态的性质。
虽然金斯堡-朗道理论通常是用于处理二级相变的,但是如果序参量在两相交界处变化足够平缓,从而能够保证系统在相变点附近的行为仍然可以使用。使用金斯堡-朗道理论处理固液相变已经成为常见的方法\cite{fabrizio2008,PhysRevB.90.104101}。
事实上,对有明确、不连续的两相交界的一级相变,基于金斯堡-朗道理论的相场方法\cite{provatas2011phase}也常常在数值模拟中被使用,以避免显式追踪相边界\cite{boettinger2002phase}。

考虑一个具有平移不变性和(连续)旋转不变性的液体。液体结晶属于结构相变,故序参量大体上是密度。
对一个最一般的系统,序参量选取是否正确、系统自由能是否还依赖于序参量以外的(无法直接从系统的哈密顿量出发获得序参量),但对液体,自由能总是可以写成密度的一个泛函\cite{Evans_2016,cdft2020}。
通常的液体的低能状态是均匀的,而无论是晶体还是准晶依照定义密度分布都不是完全均匀的,如果特定条件下能够形成准晶,那么准晶态必定相较其它状态在某种意义上更加稳定,即系统自由能最低的状态将不再是密度处处为常数的状态。
因此可以将密度的$\vb*{q} \neq 0$的傅里叶分量$\rho(\vb*{q})$视作结晶的序参量。根据涉及的波矢的个数,液体的自由能的展开式子形如下式:
\begin{equation}
    \begin{aligned}
        F[\rho] &= \sum_\text{all $\vb*{q}$'s} r \rho_{\vb*{q}} \rho_{- \vb*{q}} + u (\rho_{\vb*{q}} \rho_{- \vb*{q}})^2 
        + w \rho_{\vb*{q}} \rho_{- \vb*{q}} \rho_{\vb*{p}} \rho_{- \vb*{p}} 
        + v_3 \rho_{\vb*{q}_1} \rho_{\vb*{q}_2} \rho_{\vb*{q}_3} \delta^3(\vb*{q}_1 + \vb*{q}_2 + \vb*{q}_3) \\
        &\quad \quad + v_4 \rho_{\vb*{q}_1} \rho_{\vb*{q}_2} \rho_{\vb*{q}_3} \rho_{\vb*{q}_4} \delta^3(\sum_i \vb*{q}_i) 
        + v_5 \rho_{\vb*{q}_1} \rho_{\vb*{q}_2} \rho_{\vb*{q}_3} \rho_{\vb*{q}_4} \rho_{\vb*{q}_5}
        \delta^3(\sum_i \vb*{q}_i) + \cdots,
    \end{aligned}
\end{equation}
其中的$\delta$函数保证了理论的空间平移不变性;空间旋转不变性保证了系数仅仅依赖于$\vb*{q}_i$的模长。
据此自由能可以计算$\rho(\vb*{r})$的期望值。如果发现出现非零的$\expval{\rho(\vb*{q})}$意味着出现对称性自发破缺,有某种序形成。

不同种类的序会贡献不同形式的项到$\rho(\vb*{r})$中。

为简便起见,我们考虑一个二维液体,在其中$v_4$不重要,

\section{准晶构型的生长}

在说明了理论上确实可以存在稳定的准晶相之后,我们讨论

\section{准晶背景下的新奇物理现象}

正如传统固体物理中,晶格为各种电子态提供了周期性背景,

\bibliographystyle{plain}
\bibliography{main} 

\end{document}