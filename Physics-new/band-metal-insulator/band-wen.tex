\documentclass[hyperref, a4paper]{article}

\usepackage{geometry}
\usepackage{titling}
\usepackage{titlesec}
% No longer needed, since we will use enumitem package
% \usepackage{paralist}
\usepackage{enumitem}
\usepackage{footnote}
\usepackage{marginnote}
\usepackage{enumerate}
\usepackage{amsmath, amssymb, amsthm}
\usepackage{mathtools}
\usepackage{bbm}
\usepackage{cite}
\usepackage{graphicx}
\usepackage{subfigure}
\usepackage{physics}
\usepackage{tensor}
\usepackage{siunitx}
\usepackage[version=4]{mhchem}
\usepackage{tikz}
\usepackage{xcolor}
\usepackage{listings}
\usepackage{autobreak}
\usepackage[ruled, vlined, linesnumbered]{algorithm2e}
\usepackage{nameref,zref-xr}
\zxrsetup{toltxlabel}
\zexternaldocument*[solid-]{../solid/solid}[solid.pdf]
\zexternaldocument*[optics-]{../optics/optics}[optics.pdf]
\usepackage[colorlinks,unicode]{hyperref} % , linkcolor=black, anchorcolor=black, citecolor=black, urlcolor=black, filecolor=black
\usepackage[most]{tcolorbox}
\usepackage{prettyref}

% Page style
\geometry{left=3.18cm,right=3.18cm,top=2.54cm,bottom=2.54cm}
\titlespacing{\paragraph}{0pt}{1pt}{10pt}[20pt]
\setlength{\droptitle}{-5em}
\preauthor{\vspace{-10pt}\begin{center}}
\postauthor{\par\end{center}}

% More compact lists 
\setlist[itemize]{
    itemindent=17pt, 
    leftmargin=1pt,
    listparindent=\parindent,
    parsep=0pt,
}

% Math operators
\DeclareMathOperator{\timeorder}{\mathcal{T}}
\DeclareMathOperator{\diag}{diag}
\DeclareMathOperator{\legpoly}{P}
\DeclareMathOperator{\primevalue}{P}
\DeclareMathOperator{\sgn}{sgn}
\newcommand*{\ii}{\mathrm{i}}
\newcommand*{\ee}{\mathrm{e}}
\newcommand*{\const}{\mathrm{const}}
\newcommand*{\suchthat}{\quad \text{s.t.} \quad}
\newcommand*{\argmin}{\arg\min}
\newcommand*{\argmax}{\arg\max}
\newcommand*{\normalorder}[1]{: #1 :}
\newcommand*{\pair}[1]{\langle #1 \rangle}
\newcommand*{\fd}[1]{\mathcal{D} #1}
\DeclareMathOperator{\bigO}{\mathcal{O}}

% TikZ setting
\usetikzlibrary{arrows,shapes,positioning}
\usetikzlibrary{arrows.meta}
\usetikzlibrary{decorations.markings}
\tikzstyle arrowstyle=[scale=1]
\tikzstyle directed=[postaction={decorate,decoration={markings,
    mark=at position .5 with {\arrow[arrowstyle]{stealth}}}}]
\tikzstyle ray=[directed, thick]
\tikzstyle dot=[anchor=base,fill,circle,inner sep=1pt]

% Algorithm setting
% Julia-style code
\SetKwIF{If}{ElseIf}{Else}{if}{}{elseif}{else}{end}
\SetKwFor{For}{for}{}{end}
\SetKwFor{While}{while}{}{end}
\SetKwProg{Function}{function}{}{end}
\SetArgSty{textnormal}

\newcommand*{\concept}[1]{{\textbf{#1}}}

% Embedded codes
\lstset{basicstyle=\ttfamily,
  showstringspaces=false,
  commentstyle=\color{gray},
  keywordstyle=\color{blue}
}

% Reference formatting
\newrefformat{fig}{Figure~\ref{#1} on page~\pageref{#1}}

% Color boxes
\tcbuselibrary{skins, breakable, theorems}
\newtcbtheorem[number within=section]{warning}{Warning}%
  {colback=orange!5,colframe=orange!65,fonttitle=\bfseries, breakable}{warn}
\newtcbtheorem[number within=section]{note}{Note}%
  {colback=green!5,colframe=green!65,fonttitle=\bfseries, breakable}{note}
\newtcbtheorem[number within=section]{info}{Info}%
  {colback=blue!5,colframe=blue!65,fonttitle=\bfseries, breakable}{info}

\newcommand{\soliddoc}{\href{../solid/solid.pdf}{this solid state physics note}}
\newcommand{\opticsdoc}{\href{../optics/optics.pdf}{this optics note}}

\title{Phenomena That Can Be Explained Solely by Band Theory}
\author{Jinyuan Wu}

\begin{document}

\maketitle

This article is a reading note of Xiaogang Wen's Quantum Field Theories of Many-body Systems, Chapter~4.

\section{The shape of the Fermi surface and equal-time Green function}

In this section we explicitly evaluate the equal-time Green function.  \marginnote{Sec.~4.2.4}
An important fact is that it is highly affected by the shape of the Fermi surface. When $T=0$, we have 
(when not explicitly mentioned, when there is no spin polarization mentioned, we are working with only 
one spin polarization)
\begin{equation} 
    \begin{aligned}
        \ii G(-0^+, \vb*{x}) &= \timeorder \expval*{c(\vb*{x}, -0^+) c^\dagger(0, 0)} = - \expval*{c^\dagger(0, 0) c(\vb*{x}, 0)} \\
        &= - \int \frac{\dd[d]{\vb*{k}}}{(2\pi)^d} n_\text{F}(\xi_{\vb*{k}}) \ee^{\ii \vb*{k} \cdot \vb*{x}}
        = - \int \frac{\dd[d]{\vb*{k}}}{(2\pi)^d} \Theta(-\xi_{\vb*{k}}) \ee^{\ii \vb*{k} \cdot \vb*{x}}.
    \end{aligned}
\end{equation}
We define 
\begin{equation}
    \tilde{N}(k, \vu*{x}) = \int \frac{\dd[d]{\vb*{k}}}{(2\pi)^d} \Theta(-\xi_{\vb*{k}}) \delta(k - \vb*{k} \cdot \vu*{x}),
    \label{eq:n-tilde}
\end{equation}
and since when $k = \vb*{k} \cdot \vu*{x}$, we have $k \abs*{\vb*{x}} = \vb*{k} \cdot \vb*{x}$, we have 
\begin{equation}
    \ii G(-0^+, \vb*{x}) = - \int_{-\infty}^\infty \dd{k} \tilde{N}(k, \vu*{x}) \ee^{\ii k \abs*{\vb*{x}}}.
\end{equation}
Now the most important task is to evaluate \eqref{eq:n-tilde}. The $\delta$-function is non-zero on the 
plane $\vb*{k} \cdot \vu*{x} = k$ in the momentum space. 

\begin{figure}
    \centering
    

\tikzset{every picture/.style={line width=0.75pt}} %set default line width to 0.75pt        

\begin{tikzpicture}[x=0.75pt,y=0.75pt,yscale=-1,xscale=1]
%uncomment if require: \path (0,369); %set diagram left start at 0, and has height of 369

%Shape: Rectangle [id:dp7890850960485711] 
\draw  [draw opacity=0][fill={rgb, 255:red, 155; green, 155; blue, 155 }  ,fill opacity=0.08 ] (150.46,216.46) -- (273,41) -- (371.39,109.71) -- (248.85,285.17) -- cycle ;
%Shape: Polygon Curved [id:ds16933255681931936] 
\draw  [fill={rgb, 255:red, 80; green, 227; blue, 194 }  ,fill opacity=0.26 ] (228,112) .. controls (248,102) and (279,79) .. (318,112) .. controls (357,145) and (298,142) .. (318,172) .. controls (338,202) and (248,202) .. (228,172) .. controls (208,142) and (208,122) .. (228,112) -- cycle ;
%Straight Lines [id:da8172830438322201] 
\draw [color={rgb, 255:red, 155; green, 155; blue, 155 }  ,draw opacity=1 ]   (137,145) -- (439,145) ;
%Straight Lines [id:da8464582519121175] 
\draw [color={rgb, 255:red, 155; green, 155; blue, 155 }  ,draw opacity=1 ]   (288,251.25) -- (288,43) ;
%Straight Lines [id:da6926385458902968] 
\draw  [dash pattern={on 4.5pt off 4.5pt}]  (440,259) -- (139,35) ;
%Straight Lines [id:da01759015854493362] 
\draw [color={rgb, 255:red, 74; green, 144; blue, 226 }  ,draw opacity=1 ][line width=1.5]    (273,41) -- (167,192) ;
%Straight Lines [id:da14995718249679646] 
\draw [color={rgb, 255:red, 74; green, 144; blue, 226 }  ,draw opacity=1 ]   (302.5,63.5) -- (196.5,214.5) ;
%Straight Lines [id:da4192220598532561] 
\draw [color={rgb, 255:red, 74; green, 144; blue, 226 }  ,draw opacity=1 ]   (330.5,85.5) -- (224.5,236.5) ;
%Straight Lines [id:da6663112276637151] 
\draw [color={rgb, 255:red, 74; green, 144; blue, 226 }  ,draw opacity=1 ][line width=1.5]    (372.5,109.5) -- (266.5,260.5) ;
%Straight Lines [id:da48007887551929507] 
\draw    (537,43) -- (558,43) ;
%Straight Lines [id:da8390776098943569] 
\draw    (537,43) -- (537,21) ;
%Straight Lines [id:da41530809059346585] 
\draw    (288,145) -- (318.26,183.43) ;
\draw [shift={(319.5,185)}, rotate = 231.78] [fill={rgb, 255:red, 0; green, 0; blue, 0 }  ][line width=0.08]  [draw opacity=0] (12,-3) -- (0,0) -- (12,3) -- cycle    ;
%Straight Lines [id:da7596590644475403] 
\draw [color={rgb, 255:red, 248; green, 231; blue, 28 }  ,draw opacity=1 ][line width=1.5]    (227,171) -- (280,95) ;
%Straight Lines [id:da33448048650418194] 
\draw    (288,145) -- (249.64,118.15) ;
\draw [shift={(248,117)}, rotate = 34.99] [fill={rgb, 255:red, 0; green, 0; blue, 0 }  ][line width=0.08]  [draw opacity=0] (12,-3) -- (0,0) -- (12,3) -- cycle    ;
%Straight Lines [id:da08297341295296312] 
\draw    (288,145) -- (216.91,123.58) ;
\draw [shift={(215,123)}, rotate = 16.77] [fill={rgb, 255:red, 0; green, 0; blue, 0 }  ][line width=0.08]  [draw opacity=0] (12,-3) -- (0,0) -- (12,3) -- cycle    ;
%Straight Lines [id:da2776423007341342] 
\draw [color={rgb, 255:red, 248; green, 231; blue, 28 }  ,draw opacity=1 ][line width=1.5]    (258,189) -- (315,108) ;

% Text Node
\draw (250,113.6) node [anchor=south west] [inner sep=0.75pt]    {$\hat{\boldsymbol{x}}$};
% Text Node
\draw (211,122) node [anchor=east] [inner sep=0.75pt]    {$\boldsymbol{k}_{\text{F}}(\hat{\boldsymbol{x}})$};
% Text Node
\draw (254,17.4) node [anchor=north west][inner sep=0.75pt]    {$k=\boldsymbol{k}_{\text{F}}(\hat{\boldsymbol{x}}) \cdot \hat{\boldsymbol{x}}$};
% Text Node
\draw (539,39.6) node [anchor=south west] [inner sep=0.75pt]    {$\boldsymbol{k}$};
% Text Node
\draw (380,83.4) node [anchor=north west][inner sep=0.75pt]    {$k=\boldsymbol{k}_{\text{F}}( -\hat{\boldsymbol{x}}) \cdot \hat{\boldsymbol{x}}$};
% Text Node
\draw (319.5,188.4) node [anchor=north] [inner sep=0.75pt]    {$\boldsymbol{k}_{\text{F}}( -\hat{\boldsymbol{x}})$};
% Text Node
\draw (87,285.4) node [anchor=north west][inner sep=0.75pt]    {$\boldsymbol{k}_{\text{F}}( -\hat{\boldsymbol{x}}) \cdot \hat{\boldsymbol{x}} < k< \boldsymbol{k}_{\text{F}}(\hat{\boldsymbol{x}}) \cdot \hat{\boldsymbol{x}}$};


\end{tikzpicture}

    \caption{The shape of the Fermi surface and \eqref{eq:n-tilde}}
\end{figure}

\section{Density-density correlation function}

\section{Linear response and effective theory}

Chern-Simons

\end{document}