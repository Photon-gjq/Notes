\documentclass[hyperref, UTF8, a4paper, oneside]{ctexbook}

\usepackage{geometry}
\usepackage{titling}
\usepackage{titlesec}
\usepackage{paralist}
\usepackage{footnote}
\usepackage{enumerate}
\usepackage{autobreak}
\usepackage{amsmath, amssymb, amsthm}
\usepackage{mathtools}
\usepackage{bbm}
\usepackage{cite}
\usepackage{graphicx}
\usepackage{subfigure}
\usepackage{physics}
\usepackage{siunitx}
\usepackage{tikz}
\usepackage{tikz-feynhand}
\usepackage[ruled, vlined, linesnumbered, noend]{algorithm2e}
\usepackage[colorlinks, linkcolor=black, anchorcolor=black, citecolor=black, filecolor=black]{hyperref}
\usepackage[most]{tcolorbox}
\usepackage{caption}
\usepackage{prettyref}

\geometry{left=3.18cm,right=3.18cm,top=2.54cm,bottom=2.54cm}
\titlespacing{\paragraph}{0pt}{1pt}{10pt}[20pt]
\setlength{\droptitle}{-5em}
\preauthor{\vspace{-10pt}\begin{center}}
\postauthor{\par\end{center}}

\DeclareMathOperator{\timeorder}{\mathcal{T}}
\DeclareMathOperator{\diag}{diag}
\DeclareMathOperator{\legpoly}{P}
\DeclareMathOperator{\primevalue}{P}
\DeclareMathOperator{\sgn}{sgn}
\newcommand*{\ii}{\mathrm{i}}
\newcommand*{\ee}{\mathrm{e}}
\newcommand*{\const}{\mathrm{const}}
\newcommand*{\suchthat}{\quad \text{s.t.} \quad}
\newcommand*{\argmin}{\arg\min}
\newcommand*{\argmax}{\arg\max}
\newcommand*{\normalorder}[1]{: #1 :}
\newcommand*{\pair}[1]{\langle #1 \rangle}
\newcommand*{\fd}[1]{\mathcal{D} #1}

\newrefformat{chap}{第\ref{#1}章}
\newrefformat{sec}{第\ref{#1}节}
\newrefformat{note}{注\ref{#1}}
\newrefformat{fig}{图\ref{#1}}
\newrefformat{alg}{算法\ref{#1}}
\newrefformat{back}{背景知识\ref{#1}}
\newrefformat{info}{资料框\ref{#1}}
\newrefformat{warn}{注意事项\ref{#1}}
\renewcommand{\autoref}{\prettyref}

\usetikzlibrary{arrows,shapes,positioning}
\usetikzlibrary{arrows.meta}
\usetikzlibrary{decorations.markings}
\tikzstyle arrowstyle=[scale=1]
\tikzstyle directed=[postaction={decorate,decoration={markings,
    mark=at position .5 with {\arrow[arrowstyle]{stealth}}}}]
\tikzstyle ray=[directed, thick]
\tikzstyle dot=[anchor=base,fill,circle,inner sep=1pt]

% Algorithm setting
\renewcommand{\algorithmcfname}{算法}
% Python-style code
\SetKwIF{If}{ElseIf}{Else}{if}{:}{elif:}{else:}{}
\SetKwFor{For}{for}{:}{}
\SetKwFor{While}{while}{:}{}
\SetKwInput{KwData}{输入}
\SetKwInput{KwResult}{输出}
\SetArgSty{textnormal}

\tcbuselibrary{skins, breakable, theorems}

\newtcbtheorem[number within=chapter]{back}{背景知识}%
  {colback=blue!5,colframe=blue!65,fonttitle=\bfseries, breakable}{back}
\newtcbtheorem[number within=chapter]{info}{资料框}%
  {colback=blue!5,colframe=blue!65,fonttitle=\bfseries, breakable}{info}
\newtcbtheorem[number within=chapter]{warning}{注意事项}%
  {colback=orange!5,colframe=orange!65,fonttitle=\bfseries, breakable}{warn}

\renewcommand{\emph}[1]{\textbf{#1}}
\newcommand*{\concept}[1]{\underline{\textbf{#1}}}

\numberwithin{equation}{chapter}

\newcommand{\hmn}[1]{% Hermann-Maguin notation
  \ensuremath{\begingroup\setupHMN #1\endgroup}%
}

\newcommand{\setupHMN}{%
  \doHMN{-}{\HMNoverline}%
  \doHMN{*}{\HMNminverse}%
  \doHMN{i}{\infty}
}

\newcommand{\doHMN}[2]{%
  \begingroup\lccode`~=`#1
  \lowercase{\endgroup\let~}#2%
  \mathcode`#1="8000
}

\newcommand{\HMNminverse}[1]{\frac{#1}{m}}
\newcommand{\HMNoverline}[1]{\mkern1mu\overline{\mkern-1mu#1\mkern-1mu}\mkern1mu}

\newcommand{\Ztwo}{$\mathbb{Z}_2$}

\newcommand{\bigO}[1]{\mathcal{O}(#1)}

\numberwithin{equation}{chapter}

\newcommand{\soliddoc}{\href{../solid/solid.pdf}{固体物理笔记}}

\title{液体物理}
\author{吴晋渊}

\begin{document}

\maketitle

\part{基本原理}

\chapter{连续介质力学的宏观理论框架}

以下在没有明确指定空间维数时,记维数为$d$。记号$\phi(x)$表示“在讨论$\phi$时用坐标系$x$标记空间中各点”,从而$\phi(x')$\emph{不是}简单地将$x'$代入$x \mapsto \phi(x)$得到的场。

\section{运动学}

本节以一种几何的方式,辅以对“连续介质”的概念的一些直觉性的假设,来讨论连续介质的运动学,或者说怎样描述“形变”。
在某种意义上,连续介质的动力学和广义相对论有些相似,因为形变场——一个动力学自由度——本身也可以被用作坐标。这就容易造成一些概念的混淆。

\subsection{形变和物质导数}

连续介质的基本自由度是平滑化的粒子位置。一个连续介质可以用一个空间体积$\Omega$表示。
设连续介质在某个基准时间点(比如说$t=0$)占据空间$\Omega_\text{s}$,经过形变之后占据空间$\Omega_\text{d}$。用$\vb*{r}_\text{s}$表示$\Omega_\text{s}$中的点,用$\vb*{r}_\text{d}$表示$\Omega_\text{d}$中的点,则映射
\[
    f: \Omega_\text{s} \longrightarrow \Omega_\text{d}, \; \vb*{r}_\text{s} \mapsto \vb*{r}_\text{d}
\]
就是\concept{形变映射}。我们通常认为$f$是处处连续的。矢量
\begin{equation}
    \vb*{u} = \vb*{r}_\text{d} - \vb*{r}_\text{s}
\end{equation}
就是\concept{形变}。形变可以定义成$\vb*{r}_\text{d}$的函数,也可以定义成$\vb*{r}_\text{s}$的函数。
$\vb*{r}_\text{s}$是一个不随时间变化的构型;它的作用仅仅是用于扣除一个“背景”。
这两个矢量场都可以构成连续介质中的点的坐标,即它们可以是动力学自由度,也可以是坐标。
为了避免混乱,我们通常用$\vb*{u}$做动力学自由度,而用$\vb*{r}_\text{d}$做坐标,虽然这并没有一定之规。
换句话说,我们通常让$\vb*{u}$出现在被求导的地方,而让$\vb*{r}_\text{d}$出现在求导时的自变量的位置上。

直观地看,$\vb*{r}_\text{s}$相当于微观理论中的粒子编号的一个连续化(因为可以认为在某个特定的时间点,不同编号的粒子非常不可能正好位于完全同样的位置),从而从基于粒子的微观理论做粗粒化时,用$\vb*{r}_\text{s}$是非常方便的。
然而,实验中只能确定形变场中在某个实验室坐标系下某一点的物理量(流速、密度等),而确定不了各个微团一开始在哪里,即实际可以测量的物理量都是$\vb*{r}_\text{d}$的函数;同样,对连续介质施加外力也是施加在一个实验室坐标系中固定不动的一点上。
因此做实际计算时通常应该使用$\vb*{r}_\text{d}$做坐标。

连续介质在某一点的运动速度$\vb*{v}$的形式略微复杂,因为“某一点”本身是一个含糊不清的概念。
我们可以将$\vb*{v}$写成$\vb*{r}_\text{s}$和$t$的函数,即追踪形变开始前的每一个质点在形变之后流动到了哪里,那么显然有
\begin{equation}
    \vb*{v}(\vb*{r}_\text{s}, t) = \dv{\vb*{u}}{t} = \left( \pdv{\vb*{u}}{t} \right)_{\vb*{r}_\text{s}} = \dv{\vb*{r}_\text{d}}{t} = \left( \pdv{\vb*{r}_\text{d}}{t} \right)_{\vb*{r}_\text{s}}.
\end{equation}
第三个等号是因为$\vb*{r}_\text{d}$可以写成$\vb*{r}_\text{s}$和$t$的函数,而考虑到$\vb*{r}_\text{s}$是始终静止的,时间导数就是对$t$的偏导数。
我们也可以将$\vb*{v}$写成$\vb*{r}_\text{d}$和$t$的函数,速度的表达式为
\begin{equation}
    \begin{aligned}
        \vb*{v}(\vb*{r}_\text{d}, t) &= \left( \pdv{\vb*{u}}{t} \right)_{\vb*{r}_\text{s}} = \left( \pdv{\vb*{u}}{t} \right)_{\vb*{r}_\text{d}} + \left( \pdv{\vb*{r}_\text{d}}{t} \right)_{\vb*{r}_\text{s}} \cdot \left( \pdv{\vb*{u}}{\vb*{r}_\text{d}} \right)_t \\
        &= \left( \pdv{\vb*{u}}{t} \right)_{\vb*{r}_\text{d}} + \vb*{v} \cdot \left(\pdv{\vb*{u}}{\vb*{r}_\text{d}}\right)_t.
    \end{aligned}
\end{equation}
上式中多出来了一个非线性项,这是因为某一个空间点上这一时刻和下一时刻的是不一样的,从而会有一个漂移项。

上面的定义可以推广到一般的场变量上。场变量$\varphi$的\concept{物质导数}是$\varphi(\vb*{r}_\text{d}, t)$的时间全导数,于是
\begin{equation}
    \begin{aligned}
        \dv{\varphi}{t} &= \left( \pdv{\varphi}{t} \right)_{\vb*{r}_\text{d}} + \left( \pdv{\vb*{r}_\text{d}}{t} \right)_{\vb*{r}_\text{d}} \cdot \pdv{\varphi}{\vb*{r}_\text{d}} \\
        &= \left( \pdv{\varphi}{t} \right)_{\vb*{r}_\text{d}} + \vb*{v} \cdot \pdv{\varphi}{\vb*{r}_\text{d}}.
    \end{aligned}
\end{equation}

我们通常将用$\vb*{r}_\text{s}$标记所有场量的方法称为\concept{拉格朗日法}而将用$\vb*{r}_\text{d}$标记所有场量的方法称为\concept{欧拉法}。
前者可以看成是追踪每个粒子的位置变化,后者可以看成固定观察实验室坐标系中的固定点。
一旦确定了使用哪种方法,就可以去掉下标d或是s。
例如,在欧拉法中,我们将所有公式中的$\vb*{r}_\text{d}$替换为$\vb*{r}$,从而物质导数就是
\begin{equation}
    \dv{t} = \pdv{t} + \vb*{v} \cdot \grad.
\end{equation}
这样,加速度就是
\begin{equation}
    \dv{\vb*{v}}{t} = \pdv{\vb*{v}}{t} + \vb*{v} \cdot \grad{\vb*{v}}.
\end{equation}

\subsection{几何体的形变}

应变和扭转:后者保度规前者不保

\begin{equation}
    \dv{t} \dd{V} = \vb*{v} \cdot \dd{\vb*{S}}
    \label{eq:volumn-time-change}
\end{equation}

\begin{equation}
    \dv{t} \dd{\vb*{S}} = \dd{\vb*{S}} \div{\vb*{v}} - \dd{\vb*{S}} \cdot \vb*{v} \grad
\end{equation}

\subsection{密度和输运}

设$\rho$是某个随流量的密度,也即,它是微观下某个流体粒子携带的荷的密度的粗粒化。
应当注意虽然$\rho$看起来像是一个标量,在不同坐标系中定义的密度是不能通过简单的坐标变换相转化的,因为我们有
\[
    \rho_1(\vb*{r}_1) \dd[d]{\vb*{r}_1} = \rho_2(\vb*{r}_2) \dd[d]{\vb*{r}_2}.
\]
在坐标系变换涉及时间时,事情稍微复杂一些,因为此时可能存在随流量的产生和消灭。

我们以$\vb*{r}_\text{d}$为坐标,即使用欧拉法。设$f(\vb*{r}_\text{d}, t)$是流量的源或者汇的分布函数,则守恒性就是
\[
    \rho_\text{d}(\vb*{r}_\text{d}) \dd[d]{\vb*{r}_\text{d}} + f_\text{d} (\vb*{r}_\text{d}) \dd{t} \dd[d]{\vb*{r}_\text{d}} = \rho'_\text{d} (\vb*{r}_\text{d}') \dd[d]{\vb*{r}'_\text{d}},
\]
考虑到
\[
    \frac{\dd[d]{\vb*{r}_\text{d}'}}{\dd[d]{\vb*{r}_\text{d}}} = \det( 1 + \pdv{(\vb*{r}_\text{d} + \vb*{v} \dd{t})}{\vb*{r}_\text{d}} ) = 1 + \dd{t} \div{\vb*{v}},
\]
以及
\[
    \rho'(\vb*{r}_\text{d}') - \rho(\vb*{r}_\text{d}) = \dv{\rho}{t} \dd{t},
\]
我们就得到
\begin{equation}
    \dv{\rho}{t} + \rho \div{\vb*{v}} = \pdv{\rho}{t} + \div{(\rho \vb*{v})} = f.
    \label{eq:transportation-eq}
\end{equation}
这就是\concept{连续性方程}。

也可以从积分形式推导此方程。设$V$是一团可能随时间演化而变动的空间体积,由于\eqref{eq:volumn-time-change},我们有
\begin{equation}
    \begin{aligned}
        \dv{t} \int_V \dd[d]{\vb*{r}} \rho &= \int_V \rho \dv{t} \dd[d]{\vb*{r}} + \int_V \dd[d]{\vb*{r}} \pdv{\rho}{t} \\
        &= \int_{\partial V} \rho \vb*{v}_\text{boundary} \cdot \dd{\vb*{S}} + \int_V \dd[d]{\vb*{r}} \pdv{\rho}{t}
    \end{aligned}
    \label{eq:reynolds-transportation}
\end{equation}
注意在这里
\[
    \dv{\rho}{t} = \pdv{\rho}{t},
\]
因为我们\emph{没有}在一条流线上计算全导数;$\vb*{v}_\text{boundary}$是$\partial V$上各点的运动速度。
\eqref{eq:reynolds-transportation}称为\concept{雷诺输运定理}。
现在我们将$V$取为一个连续介质体系,也就是说,让$\vb*{v}_\text{boundary}$取为该连续介质体系的边界的运动速度,那么就有
\[
    \begin{aligned}
        0 &= \dv{t} \int_V \dd[d]{\vb*{r}} \rho \\
        &= \int_{\partial V} \rho \vb*{v} \cdot \dd{\vb*{S}} + \int_V \dd[d]{\vb*{r}} \pdv{\rho}{t} \\
        &= \int_V \dd[d]{\vb*{r}} \left(\div{(\rho \vb*{v})} + \pdv{\rho}{t} \right),
    \end{aligned}
\]
由于$V$是可以任取的,我们有
\begin{equation}
    \div{(\rho \vb*{v})} + \pdv{\rho}{t} = 0.
\end{equation}
因此我们就再一次推导出了连续性方程。显然我们也可以反过来使用以上推导,从连续性方程推导雷诺输运定理,以绕过体元性变速率\eqref{eq:volumn-time-change}。

\section{连续介质的拉格朗日动力学}

\subsection{动能和无外场的作用量}

本节我们开始讨论连续介质的动力学。在拉格朗日法中动能项对应的作用量就是简单地将多粒子的情况连续化一下,即
\begin{equation}
    S_T = \int \dd{t} \int \dd[d]{\vb*{r}_\text{s}} \frac{1}{2} \rho_\text{s} \left( \pdv{\vb*{u}}{t} \right)_{\vb*{r}_\text{s}}^2.
    \label{eq:kinetic-action}
\end{equation}
这里我们始终有质量守恒成立,即
\begin{equation}
    \rho_\text{d} \dd[d]{\vb*{r}_\text{d}} = \rho_\text{s} \dd[d]{\vb*{r}_\text{s}},
    \label{eq:density-eq}
\end{equation}
或者在欧拉法中就是
\begin{equation}
    \pdv{\rho}{t} + \div{(\rho \vb*{v})} = 0.
    \label{eq:mass-conservation}
\end{equation}

对\eqref{eq:kinetic-action}做变分,由于动力学自由度是$\vb*{u}$,而$\vb*{r}_\text{s}$本身从来不会发生变化,我们可以非常轻松地得到
\begin{equation}
    \begin{aligned}
        \var{S_T} &= \int \dd{t} \int \dd[d]{\vb*{r}_\text{s}}  \rho_\text{s} \left( \pdv{\vb*{u}}{t} \right)_{\vb*{r}_\text{s}} \cdot \var{\left( \pdv{\vb*{u}}{t} \right)_{\vb*{r}_\text{s}}} \\
        &= - \int \dd{t} \int \dd[d]{\vb*{r}_\text{s}}  \rho_\text{s} \left( \pdv[2]{\vb*{u}}{t} \right)_{\vb*{r}_\text{s}} \cdot \var{\vb*{u}}.
    \end{aligned}
\end{equation}
我们没有对$\rho_\text{s}$做任何操作,因为它是给定的、不变的。当然,上式给出的就是牛顿第二定律$F=ma$中的$ma$。

在欧拉法中事情要稍微复杂一些。由动量守恒条件和\eqref{eq:kinetic-action}我们有
\begin{equation}
    S_T = \int \dd{t} \int \dd[d]{\vb*{r}_\text{d}} \frac{1}{2} \rho_\text{d} \vb*{v}^2.
\end{equation}
首先要注意这里的$\rho_\text{d}$不是独立的自由度,因为它可以通过$\rho_\text{s}$和$\vb*{r}_\text{s}$,按照\eqref{eq:density-eq}直接计算出来。
其次,这里最基本的变分是$\var{\vb*{r}_\text{d}}$,$\var{\vb*{v}}$中会同时包含$\var{\vb*{r}_\text{d}}$及其一阶导数。
由于\eqref{eq:density-eq}我们有
\[
    \var{S_T} = \int \dd{t} \int \dd[d]{\vb*{r}_\text{d}} \rho_\text{d} \vb*{v} \cdot \var{\vb*{v}}.
\]
我们注意到
\[
    \var{\vb*{v}} = \left( \pdv{\var{\vb*{r}_\text{d}}}{t} \right)_{\vb*{r}_\text{s}} = \left( \pdv{\var{\vb*{r}_\text{d}}}{t} \right)_{\vb*{r}_\text{d}} + \vb*{v} \cdot \left(\pdv{\var{\vb*{r}_\text{d}}}{\vb*{r}_\text{d}}\right)_t,
\]
做分部积分,就得到
\[
    \var{S_T} = - \int \dd{t} \int \dd[d]{\vb*{r}_\text{d}} \left( \left(\pdv{(\rho_\text{d} \vb*{v})}{t}\right)_{\vb*{r}_\text{d}} \cdot \var{\vb*{r}_\text{d}} + \grad_{\vb*{r}_\text{d}} \cdot (\rho_\text{d} \vb*{v} \vb*{v}) \cdot \var{\vb*{r}_\text{d}} \right),
\]
展开并代入\eqref{eq:mass-conservation},就得到最终的
\begin{equation}
    \var{S_T} = - \int \dd{t} \int \dd[d]{\vb*{r}_\text{d}}  \rho_\text{d} \left( \left(\pdv{\vb*{v}}{t}\right)_{\vb*{r}_\text{d}} + \vb*{v} \cdot \grad_{\vb*{r}_\text{d}} \vb*{v} \right) \cdot \var{\vb*{r}_\text{d}}.
\end{equation}
上式中出现了熟悉的输运项。

\subsection{外力和势能}

外场$\vb*{f}$

\subsection{耗散}

\subsection{作为低能有效理论的连续介质力学}

连续介质力学实际上可以当成一种低能有效理论。%
\footnote{
    纯经典的理论也可以讨论低能有效理论,见\cite{reall2021effective}。这是好理解的,因为一个经典的理论$\mathcal{L}$总是可以看成一个量子理论的鞍点近似,那么这个量子理论的低能有效理论的鞍点近似就给出了$\mathcal{L}$的低能有效理论。
    主要的困难在于纯经典的理论做这样的操作时可能出现无穷振荡之类的问题,通常需要某种平均来消除它们,然而本文讨论的系统都充分热化,这一点一般不成问题。
}%
只要一个系统的最低能自由度可以被赋予位移场的意义,并且这些位移场给出的能量的形式和牛顿力学中的动能和势能一样,这个系统就可以被连续介质力学描述。
严格来说这需要在一个非平衡态统计场论中做,但是如果这个“更高级的”形式理论里面能够用拉氏量的概念,下面的所有讨论都是正确的。
应注意由于同样的经典运动方程可以通过不同的拉氏量获得,可能不能直接将通过对称性等方法论证出来的经典拉氏量量子化来得到考虑了量子效应的连续介质力学。
例如,文献\cite{eft-fluid-rel}描述了如何用有效场论的观点得到流体力学,但是非常狡猾地没有提我们如何量子化流体力学。

实际上,“量子化流体力学”这件事本身就有一定的微妙之处。
一个量子液体系统中似乎很难给位移场一个算符的对应:为多粒子系统定义位移场需要用到“某个粒子一开始在某个位置,然后位移到了某个位置”这样的说法,实际上隐含地为粒子编号了(“这个粒子”)。但是由全同粒子假设,这是办不到的。
粒子运动范围高度定域的系统可以定义位移场,因为此时各个粒子的运动范围足够作为近似的区分不同粒子的标签或者说编号,但是此时,输运项$\vb*{v} \cdot (\div{\vb*{v}})$基本上毫无用处。
得到粒子运动范围高度定域的系统的位移场之后我们立刻可以求解其波动,得到声子,然后后面所有的工作都可以用声子完成。
话又说回来,虽然无法直接给量子液体一个位移场的定义,但是量子液体的确可以使用动理学方程描述,而动理学方程一般来说总是能够给出某种“流体动力学”,虽然这种“流体”的行为和普通的分子流体可能非常不同(例如具有零声)。
对温度较高的系统,这种系统中的粒子,无论是费米子还是玻色子,看起来似乎都服从玻尔兹曼分布,和经典的、没有交换对称性或反对称性的粒子的行为完全一致,因此给粒子编号后做计算得到的结果是可靠的。
如果一个多电子系统能够用这种模型描述,那基本上就是等离子体,因为普通的凝聚态系统中电子在不同能级上的分布情况基本上可以用费米球描述,远远没有到服从经典玻尔兹曼分布的程度。

总之,对一个粒子运动范围非常不局域的凝聚态系统,我们有两个极端:在温度较低时它是“量子液体”,基本的自由度不确定,如在固体物理中基质是电子,则其低能有效理论的基本自由度可以就是重整化之后的电子,即费米液体中的准粒子,也可以是Luttinger液体等,这样的系统中“位移场”或是“流速场”对应着什么算符是不很确切的,但是它们的确可以有动理学理论;在温度很高时它是等离子体,可以使用完全经典的理论描述,基本的自由度是位移场——实际上是流速场,此时它是通常的、直观意义上的流体。
介于这两者之间的情况一般来说是比较难以分析的。
从上述说法可以看到“液体”一词被用于形容粒子间有微弱相互作用的系统的原因:一个这样的系统的运动方程的形式和普通的液体有很强的相似性,虽然它们的行为可能很不同。

\section{宏观受力分析和能流分析}

\subsection{牛顿运动方程作用于介质微团}

\begin{equation}
    \dv{t} \int_V \dd[d]{\vb*{r}} \rho \vb*{v} = \int_{\partial V} \dd{\vb*{S}} \cdot \vb*{\sigma} + \int_V \dd[d]{\vb*{r}} \rho \vb*{f}, 
\end{equation}
由于
\[
    \dv{t} \int_V \dd[d]{\vb*{r}} \rho \vb*{v} = \int_V \dd[d]{\vb*{r}} \dv{(\rho \vb*{v})}{t} = \int_V \dd[d]{\vb*{r}} \left( \pdv{\vb*{v}}{t} + \vb*{v} \cdot \grad \vb*{v} \right),
\]
并作用高斯散度定理
\[
    \int_{\partial V} \dd{\vb*{S}} \cdot \vb*{\sigma} = \int_V \dd[d]{\vb*{r}} \div{\vb*{\sigma}},
\]
就得到
\begin{equation}
    \rho \pdv{\vb*{u}}{t} + \rho \vb*{u} \cdot \grad{\vb*{u}} = \div{\vb*{\sigma}} + \rho \vb*{f},
    \label{eq:newton-continuum}
\end{equation}

\section{材料性能和准热力学平衡}

我们经常%
\footnote{
    但不总是——尤其是涉及到量子液体时。如费米液体的零声模式就不是热力学平衡的。见\soliddoc中的第\ref{solid-sec:zero-sound-fermi-liquid}节。
}%
假定连续介质运动过程中几乎总是保持准热力学平衡——这就是说,虽然介质整体显然没有达到热力学平衡,但是介质中每一个宏观小微观大的微团都可以近似认为达到了热力学平衡。
这样就可以使用诸如“单位体积的熵变”、“微团的热力学”等概念推导其行为。

如果我们确信连续介质是准热力学平衡的,那么将热力学方程中的$\dd$替换为$\grad$即可得到关于介质中各点的热力学量的一个方程,因为我们总是可以想象一个充分小以至于不会扰动介质的其它部分,同时热化足够快的流体微团在介质中做了一段速度恰到好处,相对于他自身是热力学过程而相对于介质来说很慢,以至于介质可以看成静态的背景的空间位移,则这段空间位移中该流体微观的各个参数的(满足热力学定律的)变化可以完全归结为其空间位置的变化,从而将$\dd$替换为$\grad$是合理的。

\chapter{动理学}

\section{二粒子相互作用和BBGKY序列}

\subsection{BBGKY序列}\label{sec:bbgky}

考虑一个一般的仅包含二粒子相互作用的系统,系统中共有$N$个粒子(注意此处的$N$不是晶胞个数),且相互作用力场具有各向同性:
\begin{equation}
    H = \sum_i \frac{p_i^2}{2m} + \sum_{i \neq j} U(\abs*{\vb*{r}_i - \vb*{r}_j}).
    \label{eq:general-gas-hamiltonian}
\end{equation}
\eqref{eq:general-gas-hamiltonian}会导致如下的刘维尔方程:
\[
    \dv{P_N}{t} + \sum_i \left( \vb*{v}_i \cdot \pdv{P_N}{\vb*{r}_i} + \frac{\vb*{F}_i}{m} \cdot \pdv{P_N}{\vb*{v}_i} \right), \quad \vb*{F}_i = - \sum_{j \neq i} \pdv{U_{ij}}{\vb*{r}_i}, 
\]
其中$P_N$是$\{\vb*{q}_i, \vb*{p}_i\}$的函数,且随意交换两个粒子,$P_N$不变。考虑如下的$s$粒子边缘分布:
\begin{equation}
    P_N^{(s)} = \int \prod_{i \geq s+1} \dd{\vb*{r}_i} \dd{\vb*{v}_i} P_N,
\end{equation}
使用$P_N$的对称性以及积分边界项为零的事实,可以推导出
\begin{equation}
    \pdv{P_N^{(1)}}{t} + \vb*{v}_1 \cdot \pdv{P_N^{(1)}}{\vb*{r}_1} = \frac{N-1}{m} \int \dd{\vb*{r}_2} \dd{\vb*{v}_2} \pdv{U_{12}}{\vb*{r}_1} \cdot \pdv{P_N^{(2)}}{\vb*{v}_1},  
    \label{eq:from-p2-to-p1}
\end{equation}
以及类似的从$P^{(3)}_N$推导出$P^{(2)}_N$的方程
\begin{align}
    \begin{autobreak}
        \pdv{P_N^{(2)}}{t} + \vb*{v}_1 \cdot \pdv{P_N^{(2)}}{\vb*{r}_1} 
        + \vb*{v}_2 \cdot \pdv{P_N^{(2)}}{\vb*{r}_2} 
        - \frac{1}{m} \pdv{U_{12}}{\vb*{r}_1} \cdot \pdv{P_N^{(2)}}{\vb*{v}_1} 
        - \frac{1}{m} \pdv{U_{12}}{\vb*{r}_2} \cdot \pdv{P_N^{(2)}}{\vb*{v}_2} 
        = \frac{N-2}{m} \int \dd{\vb*{r}_3} \dd{\vb*{v}_3} \left( \pdv{P_N^{(3)}}{\vb*{v}_1} \cdot \pdv{U_{13}}{\vb*{r}_1} + \pdv{P_N^{(3)}}{\vb*{v}_2} \cdot \pdv{U_{23}}{\vb*{r}_2} \right),
    \end{autobreak}
    \label{eq:from-p3-to-p2}
\end{align}
还有从$P_N^{(4)}$推导出$P_N^{(3)}$等等的方程。这就是\concept{BBGKY序列}。
将某个高阶$P_N^{(s)}$取为零,就可以做一个截断,从而得到一组自洽方程,可以从$P_N^{(s)}$计算$P_N^{(s-1)}$,最后计算出$P_N^{(1)}$。

\subsection{一阶近似:玻尔兹曼方程}

在只关心单粒子边缘分布,假定系统近平衡(或者说充分热化了),且比较稀薄(从而基本上是一种\emph{气体}),以至于气体分子间距常常在相互作用力程之外时,我们可以推导出所谓\concept{玻尔兹曼方程}。
\eqref{eq:from-p3-to-p2}右边的空间积分只有在两个粒子间距在相互作用力程$d$中,即满足$\abs*{\vb*{r}_i - \vb*{r}_j} \lesssim d$时才有非零值,因此该积分应该和$d^3$同阶。另一方面,设分子间距的数量级为$\delta$,则系统体积满足
\[
    V \sim N \delta^3.
\]
最后,注意到对整个系统体积和动量空间积分,有
\[
    \int \dd{\vb*{r}_3} \dd{\vb*{v}_3} \pdv{P_N^{(3)}}{\vb*{v}_1} \cdot \pdv{U_{13}}{\vb*{r}_1} \sim \pdv{P_N^{(2)}}{\vb*{v}_1} \cdot \pdv{U_{13}}{\vb*{r}_1},
\]
于是合起来就有
\[
    \frac{N-2}{m} \int \dd{\vb*{r}_3} \dd{\vb*{v}_3} \pdv{P_N^{(3)}}{\vb*{v}_1} \cdot \pdv{U_{13}}{\vb*{r}_1} \sim \frac{1}{m} \pdv{P_N^{(2)}}{\vb*{v}_1} \cdot \pdv{U_{13}}{\vb*{r}_1} \frac{d^3}{\delta^3}.
\]
如果气体非常稀薄,那么$d/\delta$就是小量,于是\eqref{eq:from-p3-to-p2}右边可以略去,得到
\begin{equation}
    \pdv{P_N^{(2)}}{t} 
    + \vb*{v}_1 \cdot \pdv{P_N^{(2)}}{\vb*{r}_1} 
    + \vb*{v}_2 \cdot \pdv{P_N^{(2)}}{\vb*{r}_2} 
    - \frac{1}{m} \pdv{U_{12}}{\vb*{r}_1} \cdot \pdv{P_N^{(2)}}{\vb*{v}_1} 
    - \frac{1}{m} \pdv{U_{12}}{\vb*{r}_2} \cdot \pdv{P_N^{(2)}}{\vb*{v}_2} = 0.
    \label{eq:effective-2-particle}
\end{equation}
上式实际上可以写成一个全微分的形式:
\begin{equation}
    \dv{P_N^{(2)}}{t}=0,
    \label{eq:pn2-constant}
\end{equation}
这个全微分沿着哈密顿量
\begin{equation}
    H_\text{eff} = \frac{p_1^2}{2m} + \frac{p_2^2}{2m} + U(\abs*{\vb*{r}_1 - \vb*{r}_2})
    \label{eq:2-particle-hamiltonian}
\end{equation}
描写的相轨道。这当然是正确的,因为近似\eqref{eq:effective-2-particle}只使用了两对坐标-动量对,因此描述了一个近似只有两个粒子的系统,那么它显然应该服从二粒子系统的刘维尔定律。

既然系统已经充分热化,不应该有除了单粒子分布函数以外更多的信息,我们做\concept{分子混沌性假设}
\begin{equation}
    P^{(2)}_N(\vb*{r}_1, \vb*{v}_1, \vb*{r}_2, \vb*{v}_2, t) = P^{(1)}_N(\vb*{r}_1, \vb*{v}_1, t) P^{(1)}_N(\vb*{r}_2, \vb*{v}_2, t).
\end{equation}
设$t_0$是某个固定的“计时起点”,并用下标0表示两个粒子在$t_0$时的各种物理量,由于\eqref{eq:pn2-constant},我们有
\[
    P^{(2)}_N(\vb*{r}_1, \vb*{v}_1, \vb*{r}_2, \vb*{v}_2, t) = P^{(1)}_N(\vb*{r}_{10}, \vb*{v}_{10}, t_0) P^{(1)}_N(\vb*{r}_{20}, \vb*{v}_{20}, t_0).
\]
注意这里的$\vb*{r}_{10}, \vb*{r}_{20}$是$t-t_0,\vb*{r}_1,\vb*{r}_2,\vb*{v}_1, \vb*{v}_2$的函数,而$\vb*{v}_{10}, \vb*{v}_{20}$是$\vb*{r}_1,\vb*{r}_2,\vb*{v}_1, \vb*{v}_2$的函数。(这是时间反演不变性的结果,因为我们总是可以把时间倒着演化回去,从$t$演化到$t_0$)
将上式代入\eqref{eq:from-p2-to-p1},并将$N-1$近似为$N$,定义(当然这里的记号和费米子分布函数又冲突了)
\begin{equation}
    f=NP_N^{(1)},
\end{equation}
得到
\[
    \pdv{f}{t} + \vb*{v}_1 \cdot \pdv{f}{\vb*{r}_1} = \frac{1}{m} \int \dd{\vb*{r}_2} \dd{\vb*{v}_2} \pdv{U_{12}}{\vb*{r}_1} \cdot \pdv{f(\vb*{r}_{10}, \vb*{v}_{10}, t_0) f(\vb*{r}_{20}, \vb*{v}_{20}, t_0)}{\vb*{v}_1}.
\]
由于$f$要出现显著的变化需要在平均自由程的尺度上,而上式所述的积分在该尺度上是高度定域的($d$远小于平均自由程),我们有
\[
    \pdv{\vb*{r}}{t} \cdot \pdv{f}{\vb*{r}} \ll \pdv{U}{\vb*{r}} \cdot \pdv{f}{\vb*{v}},
\]
这样可以将\eqref{eq:effective-2-particle}中的时间偏导数项去掉,而对$\vb*{r}_2$的偏导数对$\dd{\vb*{r}_2}$积分之后得到表面项,为零,于是
\[
    \begin{aligned}
        \pdv{f}{t} + \vb*{v}_1 \cdot \pdv{f}{\vb*{r}_1} &= \int \dd{\vb*{r}_2} \dd{\vb*{v}_2} \left( \vb*{v}_1 \cdot \pdv{f(\vb*{r}_{10}, \vb*{v}_{10}, t_0) f(\vb*{r}_{20}, \vb*{v}_{20}, t_0)}{\vb*{r}_1} + \vb*{v}_2 \cdot \pdv{f(\vb*{r}_{10}, \vb*{v}_{10}, t_0) f(\vb*{r}_{20}, \vb*{v}_{20}, t_0)}{\vb*{r}_2} \right) \\
        &= \int \dd{\vb*{r}} \dd{\vb*{v}_2} \vb*{u} \cdot \pdv{f(\vb*{r}_{10}, \vb*{v}_{10}, t_0) f(\vb*{r}_{20}, \vb*{v}_{20}, t_0)}{\vb*{r}},
    \end{aligned}
\]
其中$\vb*{r}$和$\vb*{u}$分别定义为
\begin{equation}
    \vb*{r} = \vb*{r}_1 - \vb*{r}_2, \quad \vb*{u} = \vb*{v}_1 - \vb*{v}_2.
\end{equation}
以$\vb*{u}$的方向为$z$轴建立柱坐标系,设$\vb*{r}$的三个坐标是$\rho, \varphi, z$,并注意到可以将$\vb*{r}_{10}, \vb*{r}_{20}, \vb*{v}_{10}, \vb*{v}_{20}$表示成$\vb*{r}$的函数(因为$\vb*{r}$和$t-t_0$一一对应),有
\[
    \pdv{f}{t} + \vb*{v}_1 \cdot \pdv{f}{\vb*{r}_1} = \int \rho \dd{\rho} \dd{\varphi} \dd{\vb*{v}_2} u (f(\vb*{r}_{10}, \vb*{v}_{10}, t_0) f(\vb*{r}_{20}, \vb*{v}_{20}, t_0))\big|_{z=-\infty}^\infty.
\]
请注意$\vb*{r}$是两个发生碰撞的粒子的相对位移,$\vb*{u}$指向碰撞发生的方向,则$z$趋于$\infty$意味着碰撞结束,而$z$趋于$-\infty$意味着碰撞开始。
我们假定碰撞在时间和空间上都是高度定域的,从而,碰撞所需时间忽略不计,将$f$中出现的所有$t_0$替换为$t$不会造成什么影响,碰撞前后粒子移动可忽略不计,从而碰撞前后$f$中出现的位矢可认为基本不变,均位于$\vb*{r}_1$附近。
于是就得到
\[
    \begin{aligned}
        \pdv{f}{t} + \vb*{v}_1 \cdot \pdv{f}{\vb*{r}_1} &= \int \rho \dd{\rho} \dd{\varphi} \dd{\vb*{v}_2} u (f(\vb*{r}_{1}, \vb*{v}_{1}, t) f(\vb*{r}_{2}, \vb*{v}_{2}, t))\big|_{z=-\infty}^\infty \\
        &= \int \rho \dd{\rho} \dd{\varphi} \dd{\vb*{v}_2} u (f(\vb*{r}_{1}, \vb*{v}'_{1}, t) f(\vb*{r}_{1}, \vb*{v}'_{2}, t) - f(\vb*{r}_{1}, \vb*{v}_{1}, t) f(\vb*{r}_{1}, \vb*{v}_{2}, t)),
    \end{aligned}
\]
$\vb*{r}(t)$曲线组成一束束流管,微分形式$\rho \dd{\rho} \dd{\varphi}$实际上就是散射截面:
\begin{equation}
    \rho \dd{\rho} \dd{\varphi} = \sigma \dd{\Omega},
\end{equation}
其中$\dd{\Omega}$可以取为矢量$\vb*{r}_{10} - \vb*{r}_{20}$在$z \to \infty$时的方向对应的立体角元,实际上就是$\vb*{v}' - \vb*{v}'_2$的方向对应的立体角元。
于是把$\vb*{v}_1$和$\vb*{r}_1$的下标去掉,最终得到
\begin{equation}
    \pdv{f}{t} + \vb*{v} \cdot \pdv{f}{\vb*{r}} = \int \dd{\vb*{v}_2} \dd{\Omega} \sigma \abs*{\vb*{v}-\vb*{v}_2} (f(\vb*{r}, \vb*{v}', t) f(\vb*{r}, \vb*{v}'_{2}, t) - f(\vb*{r}, \vb*{v}, t) f(\vb*{r}, \vb*{v}_{2}, t)),
\end{equation}
其中$\vb*{v}'$和$\vb*{v}_2'$是以$\vb*{v}$和$\vb*{v}_2$为入射速度而得到的出射速度,$\dd{\Omega}$是$\vb*{v}' - \vb*{v}_2'$的指向对应的立体角元。
仅仅通过$\vb*{v}$和$\vb*{v}_2$是无法确定$\vb*{v}'$和$\vb*{v}_2'$的,必须知道$\vb*{v}' - \vb*{v}_2'$的指向才行。
虽然我们是从经典牛顿力学出发推导出的这个结果,但是量子力学中的散射理论同样是这样的,这是靠数自由度可以得到的结论,和理论的细节无关。

我们这就得到了无外力时的玻尔兹曼方程,它要求系统中粒子数密度稀薄、分子混沌性假设成立、碰撞在时间和空间上是局域的(前者要求分子平均自由时间相对于碰撞的时间尺度很大,后者要求分子之间的间距相对于碰撞的空间尺度很大)。
此外\eqref{eq:general-gas-hamiltonian}是对系统足够好的描述也意味着粒子除了整体的坐标自由度以外的自由度是不重要的,且没有非弹性过程(比如说化学反应)。
有外场作用时可以重复以上推导,不过在外场的空间尺度和时间尺度都很大时(通常的情况,比如说外加电场磁场,都是这样的),基本上任何一小块空间中的粒子都可以认为只是受到了一个恒定外力作用,于是就有一般的玻尔兹曼方程
\begin{equation}
    \pdv{f}{t} + \vb*{v} \cdot \pdv{f}{\vb*{r}} + \frac{\vb*{F}}{m} \cdot \pdv{f}{\vb*{v}} = \int \dd{\vb*{v}_2} \dd{\Omega} \sigma \abs*{\vb*{v}-\vb*{v}_2} (f(\vb*{r}, \vb*{v}', t) f(\vb*{r}, \vb*{v}'_{2}, t) - f(\vb*{r}, \vb*{v}, t) f(\vb*{r}, \vb*{v}_{2}, t)).
    \label{eq:boltzmann-eq-with-force}
\end{equation}
通常将等号右边的部分称为\concept{碰撞项}或者说\concept{碰撞积分},记作$C[f]$。

\subsection{BBGKY序列和玻尔兹曼方程的量子版本}

虽然玻尔兹曼方程是通过经典方式推导出来的,量子多体理论中也有BBGKY序列,在这里,边缘$s$粒子分布被“等效$s$粒子密度矩阵”取代,或者说被特定的格林函数取代。
玻尔兹曼方程中的$f$实际上在量子多体理论中就是单粒子密度矩阵(的Wigner函数)。
严格来说,通过非平衡量子场论得到的玻尔兹曼方程中,应当做替换
\begin{equation}
    f_1 f_2 \longrightarrow f_1 f_2 (1 \pm f_1') (1 \pm f_2'), \quad f_1' f_2' \longrightarrow f_1' f_2' (1 \pm f_1) (1 \pm f_2),
\end{equation}
玻色子取正,费米子取负。

在凝聚态物理中通常不会涉及特别偏离平衡的状态,因此$f$偏离基态$f_0$的幅度是并不大的,即$f - f_0$比较小。
由于基态不会出现任何时间演化,$C[f_0] = 0$,因此可以做展开
\begin{equation}
    C[f] = \frac{f - f_0}{\tau}.
    \label{eq:relaxation-approx}
\end{equation}
我们将比例系数命名为$1 / \tau$,这不是偶然的,因为马上可以看出$\tau$实际上给出了$f$弛豫回到$f_0$的时间尺度。
我们将\eqref{eq:relaxation-approx}称为\concept{弛豫时间近似}。
在系统中存在多种散射机理时。% TODO,以及非弹性散射?

\subsection{玻尔兹曼方程到单组分的连续介质运动方程}\label{sec:boltmann-to-continuum}

完全经典的体系似乎无法“积掉”变量,但是正如\cite{reall2021effective}所示,在满足一定的条件之后,对完全经典的系统也可以定义低能有效理论,并且通过向系统中引入小幅的随机扰动能够扩大可以定义低能有效理论的经典体系的范围。
按照这种思路,玻尔兹曼方程应该也可以有低能有效理论。

我们需要首先指定在玻尔兹曼方程的低能有效理论中被保留的变量。一个非常合理的粗粒化定义是
\begin{equation}
    \expval{Q} = \frac{1}{\rho} \int \dd[3]{\vb*{v}} m Q(\vb*{r}, \vb*{v}) f(\vb*{r}, \vb*{v}, t),
    \label{eq:corse-grain-in-boltzman}
\end{equation}
其中$\rho$定义为
\begin{equation}
    \rho = \int \dd[3]{\vb*{v}} m f(\vb*{r}, \vb*{v}, t).
\end{equation}
这里期望值只是将$\vb*{v}$平均了,保留了$\vb*{r}$。
粒子数、动量、能量是任何情况下都能够保证有的守恒量,玻尔兹曼方程的宏观理论的基本自由度中应该是有它们的。
于是我们尝试列写它们做了\eqref{eq:corse-grain-in-boltzman}粗粒化之后的运动方程,如果能够得到封闭的方程组,那么就得到了一个玻尔兹曼方程的低能有效理论。
这里的具体计算可以在著名的朗道物理学教程的第十卷\cite{lifsic_physical_2008}的第一章中找到。

先证明一个有用的结论。我们可以通过加入适当的$\delta$函数因子,将\eqref{eq:boltzmann-eq-with-force}右边的碰撞积分写成
\begin{equation}
    C[f] = \int \dd[3]{\vb*{v}_2} \int \dd[3]{\vb*{v}'} \int \dd[3]{\vb*{v}_2'} w(\vb*{p}, \vb*{p}_2, \vb*{p}', \vb*{p}_2') (f(\vb*{r}, \vb*{v}', t) f(\vb*{r}, \vb*{v}'_{2}, t) - f(\vb*{r}, \vb*{v}, t) f(\vb*{r}, \vb*{v}_{2}, t)),
\end{equation}
其中$w$是$\sigma \abs{\vb*{v} - \vb*{v}_2}$乘以适当的保证能量守恒和动量守恒的$\delta$函数因子。
可以验证,由于时间反演不变性,交换$(\vb*{p}, \vb*{p}_2)$并且同时交换$(\vb*{p}', \vb*{p}_2')$之后$w$不变。
将\eqref{eq:boltzmann-eq-with-force}右边的碰撞积分中关于入射速度$\vb*{v}$的一个任意的函数为$\varphi(\vb*{p})$,简记为$\varphi$,则
\[
    \int \dd[3]{\vb*{v}} \varphi(\vb*{p}) C[f] = \int \dd[3]{\vb*{v}} \int \dd[3]{\vb*{v}_2} \int \dd[3]{\vb*{v}'} \int \dd[3]{\vb*{v}_2'} \varphi(\vb*{p}) w(\vb*{p}, \vb*{p}_2, \vb*{p}', \vb*{p}_2') (f' f_2' - f f_2).
\]
上式的积分测度现在是高度对称的,不妨记作$\dd{\Gamma}$。
在上式的第二项中交换$(\vb*{p}, \vb*{p}_2)$并且同时交换$(\vb*{p}', \vb*{p}_2')$之后,得到
\[
    \int \dd[3]{\vb*{v}} \varphi(\vb*{p}) C[f] = \int \dd{\Gamma} (\varphi - \varphi') w f' f_2'.
\]
上式在交换$(\vb*{p}, \vb*{p}_2)$同时交换$(\vb*{p}', \vb*{p}_2')$的变换下不变,则
\[
    \begin{aligned}
        \int \dd[3]{\vb*{v}} \varphi(\vb*{p}) C[f] &= \int \dd{\Gamma} ((\varphi - \varphi') w f' f_2' + (\varphi_2 - \varphi_2') w f'_2 f') / 2 \\
        &= \int \dd{\Gamma} (\varphi + \varphi_2 - \varphi' - \varphi'_2) w f'_2 f' / 2.
    \end{aligned}
\]
因此,如果$\varphi(\vb*{p})$对$\vb*{p}$求和是某种守恒量,那么相应的就有
\begin{equation}
    \int \dd[3]{\vb*{v}} \varphi(\vb*{p}) C[f] = 0.
    \label{eq:collision-conservation}
\end{equation}

将\eqref{eq:boltzmann-eq-with-force}写成简约形式
\begin{equation}
    \pdv{f}{t} + \vb*{v} \cdot \pdv{f}{\vb*{r}} + \frac{\vb*{F}}{m} \cdot \pdv{f}{\vb*{v}} = C[f],
    \label{eq:boltmann-eq-simple-form}
\end{equation}
设单粒子质量为$m$,在上式两边乘上$m$并对$\vb*{v}$积分。
第一项显然就是$\pdv*{\rho}{t}$。第二项根据\eqref{eq:corse-grain-in-boltzman},是
\[
    \int \dd[3]{\vb*{v}} m \vb*{v} \cdot \pdv{f}{\vb*{r}} = \pdv{\vb*{r}} (\rho \expval{\vb*{v}}),
\]
第三项是典型的对散度积分的项,由于无穷远处粒子分布为零而为零。
方程右边根据\eqref{eq:collision-conservation}为零。
于是我们有
\begin{equation}
    \pdv{\rho}{t} + \div{(\rho \vb*{u})} = 0,
    \label{eq:continue-particle}
\end{equation}
即粒子数守恒方程;这里我们定义
\begin{equation}
    \vb*{u} = \expval{\vb*{v}}, \quad \vb*{w} = \vb*{v} - \vb*{u}.
    \label{eq:v-decomposite-u-w}
\end{equation}

类似的,我们在玻尔兹曼方程\eqref{eq:boltmann-eq-simple-form}两边乘上$m v_i$,并对$\vb*{v}$积分,由于动量守恒,根据\eqref{eq:collision-conservation},方程右边为零,方程左边则是
\[
    \pdv{t} \int \dd[3]{\vb*{v}} m v_i f + \int \dd[3]{\vb*{v}} m v_i v_j \pdv{f}{r_j} + \int \dd[3]{\vb*{v}} v_i F_j \pdv{f}{v_j} = 0.
\]
第一项是$\pdv*{(\rho u_i)}{t}$,第二项是
\[
    \pdv{r_j} \int \dd[3]{\vb*{v}} mf v_j v_i = \pdv{r_j} (\rho \expval{v_j v_i}) = \pdv{r_j} (\rho u_j u_i + \rho \expval{w_j w_i}).
\]
第三项通过分部积分法可以化简为
\[
    \int \dd[3]{\vb*{v}} v_i F_j \pdv{f}{v_j} = - \int \dd[3]{\vb*{v}} \pdv*{v_i}{v_j} F_j f = - \int \dd[3]{\vb*{v}} m f \frac{F_j}{m} = - \rho f_j,
\]
其中我们定义
\begin{equation}
    \vb*{f} = \expval{\frac{\vb*{F}}{m}}.
\end{equation}
这样就有
\begin{equation}
    \pdv{(\rho u_i)}{t} + \pdv{(\rho u_j u_i)}{r_j} + \pdv{\rho \expval*{w_j w_i}}{r_j} = \rho f_i.
    \label{eq:continue-momentum-in-components}
\end{equation}
定义
\begin{equation}
    \sigma_{ij} = - \rho \expval*{w_i w_j},
\end{equation}
\eqref{eq:continue-momentum-in-components}就成为
\begin{equation}
    \pdv{(\rho \vb*{u})}{t} + \div{(\rho \vb*{u} \vb*{u})} = \div{\vb*{\sigma}} + \rho \vb*{f}.
    \label{eq:continue-momentum}
\end{equation}
我们由于没有计算$\vb*{\sigma}$是什么,似乎得到了\eqref{eq:continue-momentum}也没有什么用,但是无论如何,$\vb*{\sigma}$一定是$\vb*{r}, \vb*{v}, \rho, e$的函数,

同样,在\eqref{eq:boltmann-eq-simple-form}两边乘上$m \vb*{v}^2 / 2$,根据\eqref{eq:collision-conservation},方程右边为零,方程左边则是
\[
    \pdv{t} \int \dd[3]{\vb*{v}} \frac{1}{2} m \vb*{v}^2 f + \pdv{r_i} \int \dd[3]{\vb*{v}} \frac{1}{2} m \vb*{v}^2 v_i f - \int \dd[3]{\vb*{v}} \pdv{v_i} \left( \frac{F_i}{m} \frac{1}{2} m \vb*{v}^2 \right) f = 0.
\]
将分解\eqref{eq:v-decomposite-u-w}代入上式,并注意到
\[
    \pdv{v_i} (\rho f_i \vb*{v}^2 / 2) = \rho f_i v_i,
\]
得到
\[
    0 = \pdv{t} (\rho (\vb*{u}^2 + \expval*{\vb*{w}^2}) / 2) + \pdv{r_i} (\rho (\vb*{u}^2 u_i / 2 + u_i \expval*{\vb*{w}^2} /2 + \expval*{\vb*{w}^2 w_i} + \expval*{(\vb*{u} \cdot \vb*{w}) w_i})) - \rho \vb*{u} \cdot \vb*{f}. 
\]
引入内能的质量密度
\begin{equation}
    e = \dv{E_\text{internal}}{m} = \frac{1}{\rho} \dv{E_\text{internal}}{V} = \frac{1}{2} \expval*{\vb*{w}^2},
\end{equation}
以及
\begin{equation}
    J_i = \rho \expval*{\vb*{w}^2 w_i},
\end{equation}
得到
\begin{equation}
    \pdv{t} (\rho e + \rho \vb*{u}^2 / 2) + \div{((\rho e + \rho \vb*{u}^2 / 2) \vb*{u})} + \div{(\rho \vb*{J})} = \div{(\vb*{\sigma} \cdot \vb*{u})} + \rho \vb*{u} \cdot \vb*{f}.
    \label{eq:continue-energy}
\end{equation}

以上含有$w$的量或是待求解变量,或是可以通过状态方程写成待求解变量的函数,从而实际上方程\eqref{eq:continue-particle},\eqref{eq:continue-momentum}和\eqref{eq:continue-energy}联立,并给定$\vb*{\sigma}, \vb*{J}$和$\vb*{u}, \rho, \vb*{r}, e$的关系(当然,这就是连续介质的\concept{状态方程},它的导出涉及连续介质的微观细节;工程文献常常称之为\concept{本构方程})就给出了封闭求解$\rho, \vb*{u}, e$需要的全部方程。
这里的过程和常规的场论计算中计算被积掉的变量的关联函数而得到低能有效理论中的系数的步骤是完全一样的。
将\eqref{eq:continue-particle}代入\eqref{eq:continue-momentum},可以将\eqref{eq:continue-momentum}转化为
\begin{equation}
    \rho \pdv{\vb*{u}}{t} + \rho \vb*{u} \cdot \grad{\vb*{u}} = \div{\vb*{\sigma}} + \rho \vb*{f},
    \label{eq:continue-newton}
\end{equation}
上式就是牛顿定律作用于连续介质微团上得到的运动方程\eqref{eq:newton-continuum},将它和\eqref{eq:continue-particle}以及状态方程联立求解同样能够封闭求解$\rho, \vb*{u}, e$。
类似的能量方程也可以被修改,让动能消失掉。在\eqref{eq:continue-newton}两边点乘上$\vb*{u}$并做矢量不定积分,得到
\[
    \rho \pdv{t}(\vb*{u}^2 / 2) + \rho \vb*{u} \cdot (\grad{\vb*{u}}) \cdot \vb*{u} = \vb*{u} \cdot (\div{\vb*{\sigma}}) + \rho \vb*{u} \cdot \vb*{f}.
\]
另一方面我们有
\[
    \begin{aligned}
        &\quad \pdv{t} (\rho \vb*{u}^2 / 2) + \div{(\rho \vb*{u}^2 \vb*{u} / 2)} \\
        &= \rho \pdv{t}(\vb*{u}^2 / 2) + \frac{\vb*{u}^2}{2} \pdv{\rho}{t} + \rho \vb*{u} \cdot \grad \frac{\vb*{u}^2}{2} + \frac{\vb*{u}^2}{2} \div{(\rho \vb*{u})} \\
        &= \rho \pdv{t}(\vb*{u}^2 / 2) + \rho \vb*{u} \cdot (\vb*{u} \cdot \grad \vb*{u}),
    \end{aligned}
\]
于是就得到
\begin{equation}
    \quad \pdv{t} (\rho \vb*{u}^2 / 2) + \div{(\rho \vb*{u}^2 \vb*{u} / 2)} = \vb*{u} \cdot (\div{\vb*{\sigma}}) + \rho \vb*{u} \cdot \vb*{f}.
\end{equation}
这就是动能的运动方程,将它代入\eqref{eq:continue-energy}就得到
\begin{equation}
    \pdv{(\rho e)}{t} + \div{(\rho e \vb*{u})} = - \div(\rho \vb*{J}) + \sigma_{ij} \partial_i u_j.
    \label{eq:continue-internal-energy}
\end{equation}
这就是内能的体积密度的运动方程。上式左边也等于
\[
    \rho \pdv{e}{t} + \rho \vb*{u} \cdot \grad e,
\]
于是得到
\begin{equation}
    \rho \pdv{\rho}{t} + \rho \vb*{u} \cdot \grad e = - \div(\rho \vb*{J}) + \sigma_{ij} \partial_i u_j.
\end{equation}
这就是内能的质量密度的运动方程。可以看到,实际上这里的$\vb*{\sigma}$和$\vb*{J}$就是应力和热流,这和直接通过牛顿定律和能量守恒推导出来的完全一样。

本节讨论的都是单组分的连续介质;如果有多种组分,那么需要多个速度场,方程形式也就不那么简单了。

\section{分子动力学模拟}

\part{流体}

\cite{eft-fluid-rel}中非常狡猾地没有提我们如何量子化流体力学。

\chapter{牛顿流体的流体动力学}

设距离为$y$,相对速度为$v$,平行移动的两块板之间充斥着流体,如果流体对移动的板的作用力满足
\begin{equation}
    F = A \eta \frac{u}{y},
    \label{eq:shear-force-experiment}
\end{equation}
其中$\eta$是常数,就说这是\concept{牛顿流体}。$\eta$称为粘度;我们也可以将\eqref{eq:shear-force-experiment}作为粘度的定义,粘度是常数的流体称为牛顿流体。
非牛顿流体常常是软物质的特征,见\autoref{chap:non-newtonian}。

对流体,我们有\concept{纳维-斯托克斯方程}
\begin{equation}
    \rho \left( \pdv{\vb*{v}}{t} + \vb*{v} \cdot \grad{\vb*{v}} \right) = - \grad{P} + \vb*{f}.
\end{equation}

\subsection{声波}

当速度的时间变化相比于空间输运非常大时,即
\[
    \pdv{t} \gg \vb*{v} \cdot \grad
\]
时,近似有
\begin{equation}
    \rho \pdv{\vb*{v}}{t} = - \grad{p},
    \label{eq:ns-eq-small-v}
\end{equation}
两边计算散度,并利用输运方程\eqref{eq:transportation-eq},得到
\[
    \laplacian{p} = \pdv[2]{\rho}{t},
\]
再假定压强变化不大,有
\[
    \rho = \eval{\pdv{\rho}{P}}_{P_0} (P - P_0) = \eval{\pdv{\rho}{P}}_{P_0} p,
\]
于是就得到波动方程
\begin{equation}
    \frac{1}{c^2} \pdv[2]{p}{t} = \laplacian{p},
    \label{eq:sound-wave-fluid}
\end{equation}
其中
\begin{equation}
    \frac{1}{c^2} = \eval{\pdv{\rho}{P}}_{P_0}.
\end{equation}
这就是说,快速振动而振幅不大的流体中会有线性机械波,这就是\concept{声波}。

声波一定是横波,因为\eqref{eq:ns-eq-small-v}两边同取旋度,就有
\[
    \pdv{t} \curl{\vb*{v}} = 0, 
\]
即$\curl{\vb*{v}}$不随时间变化。由于\eqref{eq:sound-wave-fluid}是线性的,我们可以只取其小幅振动的解,即从一个一般的解中剥离整体平移运动的部分。这种小幅振动的解的基底不妨取为平面波,而对每个平面波,都有$\curl{\vb*{v}}=0$,于是我们就得出结论:声波是无旋的。
实际上可以直接从压强的性质出发得到这个结论,因为横波要求剪力而压强不能提供剪力。

\chapter{液滴}

与大片的流体不同,液滴的行为和它周围的环境和表面的性质有很强的联系。

\section{飞溅}

在低气压环境中,液体落在表面上不会出现飞溅现象。这意味着飞溅来自液体和周围气体的接触:如果这种接触导致液体表面不稳定,那么飞溅就能够发生。
在气压不变时,将液体落到的表面替换成多孔板,从而液体和表面之间的气体能够随着液体下降被快速导走,同样不会出现飞溅现象。
因此,液体和表面之间的气体膜决定了飞溅是否出现。

\section{浸润}

设接触角为$\theta$,则
\[
    \dd{F} = (\gamma_\text{SL} - \gamma_\text{SV}) 2 \pi r \dd{r} + \gamma_\text{VL} \cos \theta \times 2 \pi r \dd{r},
\]
因此
\begin{equation}
    \cos \theta = \frac{\gamma_\text{SL} - \text{SV}}{\gamma_\text{VL}}.
\end{equation}

\subsection{荷叶效应}

固体表面上的坑坑洼洼能够改变液滴和固体表面的接触面积。可以想象这里有两种可能的接触状态。
\concept{Wenzel状态}为液滴下部充分填充坑坑洼洼的状态。此时液滴与界面接触的有效面积大于表观面积,液体与气体接触的有效面积不变。
我们有
\[
    \dd{F} = r (\gamma_\text{SL} - \gamma_\text{SV}) 2 \pi r \dd{r} + \gamma_\text{VL} \cos \theta^* \times 2 \pi r \dd{r},
\]
即
\begin{equation}
    \cos \theta^* = r \cos \theta.
\end{equation}
\concept{Cassie状态}为液滴下部不充分填满坑坑洼洼的状态。这时液滴与界面接触的有效面积比表观面积\emph{小},但是相应的,液体与气体接触的有效面积比表观面积大。
我们有
\[
    \dd{F} = \phi_\text{s} (\gamma_\text{SL} - \gamma_\text{SV}) 2 \pi r \dd{r} + (1 - \phi_\text{s}) \gamma \times 2 \pi r \dd{r} + \gamma \times 2 \pi r \dd{r} \cos \theta^*,
\]
从而
\begin{equation}
    \cos\theta^* = - 1 + \phi_\text{s} (\cos \theta + 1).
\end{equation}

高压下Cassie状态会转化为Wenzel状态。Cassie状态能够稳定存在是因为。

\part{软凝聚态物理}

软凝聚态物理顾名思义仍然研究凝聚态物质,也就是由原子、分子通过各种化学键或是分子间作用力结合得到的物质,不过其注意力并不在通常的固体物理或者说“硬”凝聚态物理上面(这些话题见\soliddoc),而是在\concept{软物质}上。
固体物理中原子是定死在晶格上的,其运动通过声子体现,其对晶格结构的偏离用杂质、无序等体现,低能自由度是声子、电子。
流体中的原子和分子则是可以自由移动的,通常量子效应比较微弱,经典动理学可以描述它们。
流体中最为简单的大概就是牛顿流体了,这包括水、空气、各种油等等,描述它们的理论是纳维-斯托克斯方程。
而软物质中则有大量尺度在纳米级乃至微米级的\concept{基元}:如果我们往水中放入一些大一点的基元呢?糨糊是水和一些粉末的混合物,但是剪切增稠效应就出现了。
胶体、泡沫、泡在液体中的颗粒物质(沙粒,石块,甚至冰)都是软物质的例子。向液体中加入足量的颗粒几乎总是会产生软物质体系。
二氧化硅纳米颗粒的介观基元自组织能够形成光子晶体,从而产生结构色,其剪切增稠效应能够用于制造防弹织物。
高分子溶液从细管中涌出时会膨胀(想想挤牙膏),当一根棒子在其中搅动时能够爬上棒子。
正如硬凝聚态物理中,四处移动的电子能够形成新奇的物态、各种集体模式,简单流体(气体和液体)的流速场能够形成各种有趣的模式一样,软物质中的基元能够自组织、产生非牛顿效应,以及其它奇特现象。
从上面的说法可以看出软物质说白了就是\concept{复杂流体}:是流体,但是不是普通的液体和气体,其中有大量介观尺寸、响应时间能慢到\SI{0.1}{s}量级的基元,存在大量比化学键弱的相互作用,其量级在$k_\text{B} T$,和熵效应相当。

复杂流体是多尺度的:我们在\autoref{part:simple-continuum}中讨论的问题都是基本上同质的基元组成的体系,我们可以讨论其微观理论——基本上就是单种或是少数几种的粒子的动理学——也可以讨论其宏观理论——基本上就是简单的宏观连续介质力学。
复杂流体中介观尺度上的行为是非平凡的:液体基质仍然可以视为连续介质,而颗粒基元则是离散的粒子;将尺度再拉大,似乎可以积掉液体基质而系统的主要行为来自颗粒基元。

本文中一些其它的主题——比如说普通液体的液滴等——也常常被纳入软物质的范畴。

\chapter{液体中的颗粒以及非牛顿流体}\label{chap:non-newtonian}

\section{悬浊液}

\subsection{一些唯象的考虑和经验公式}

考虑一个\concept{悬浊液},或者说浸泡在液体中的颗粒系统。设其体积分数为$\phi$。简单的几何考虑告诉我们$\phi$必然有一个小于1的上限,因为颗粒的堆积不可能完全占据每一片空间。
一般来说液体中最大的体积分数为$\phi_\text{m} \approx 0.64$。
$\phi_\text{m}$显然不会达到最密堆积的极限,因为最密堆积虽然是全局最优值,但是并不容易达到。
在$\phi \approx \phi_\text{m}$时体系和普通固体没有什么差别,而$\phi \approx 0$时体系和普通液体没什么区别。

在体积分数相同时,一般颗粒体积小的悬浊液粘度更大,因为颗粒更多,也更容易散射。

\subsection{粘度的电场调控}

理论上通过施加电场来改变流体性质,不过对普通液体,可能需要\SI{1e8}{V/m}量级的电场,并不现实。
悬浊液中的粒子可以使用小得多的电场来调控,因此可以使用电场来调控悬浊液的各种性能。
设想一个悬浊液被放置在电场当中,而其中的颗粒能够容易地极化。电偶极子之间存在相互作用,因此,将悬浊液放置在电场中可以让颗粒吸附在一起。
此时$\phi$不变,但是颗粒体积增大,数目减少,从而粘度大大下降。
不过如果电场施加时间过长,那么会形成宏观的链条,粘度反而会增大。

这种技术可能可以用来改善高血脂患者的健康状况:让他的血管被放置在强场中,脂肪吸附在一起,就能够让血液变稀。

\subsection{颗粒的相互作用}

\subsubsection{电偶极相互作用}

颗粒间的相互作用的微观机制主要为结构力,电双层相互作用和范德华力,在范德华力中,色散力是不重要的,因为复杂流体中量子效应通常不重要。

幂律通常能够从微观一路影响到宏观。考虑颗粒的
Hamaker常数

\subsubsection{熵力}

液体中的较大体积的物体靠得更近时,小颗粒能够占据的体积更大。这意味着大的物体靠近时系统熵增大,从而小颗粒的存在会产生浸泡在液体中的较大物体的等效吸引力。这是一种熵力,并且相当有趣,熵效应反而导致系统的一部分变得有序。
随着较大体积的物体的间距$d$减小,等效吸引力会有振荡,因为在$d$和小颗粒形成的层状结构的层间距可公度时,大的物体之间的小颗粒能够发生自组织,形成层状结构,此时自由能减小,从而熵力减小。

\subsubsection{表面力}

如果液体-固体界面的表面能很大——即,液体和颗粒之间不浸润——颗粒之间将会有吸引相互作用。
反之,如果液体和颗粒之间浸润,那么颗粒之间会有排斥相互作用。

\chapter{液晶}

一个液晶系统中的熵可以写成\concept{转动熵}和\concept{平移熵}之和。

\chapter{颗粒系统}

颗粒可以浸泡在液体当中得到悬浊液,此时颗粒之间碰撞不明显,可以认为是比较自由地移动;颗粒也可以自己堆积成谷堆之类的东西。

\section{自堆积效应}

将颗粒置于重力场中,并附加幅度为$A$圆频率为$\omega$的振动,在
\begin{equation}
    A \omega^2 > g
\end{equation}
时,颗粒将自发形成一个斜坡。设空气压强为$P$则斜坡高度有经验公式
\begin{equation}
    L = b \tanh(KP) + c.
\end{equation}
这表明空气在自堆积效应中起了很大作用。

\chapter{表面}

\section{泡沫}

各种密堆积

\bibliographystyle{plain}
\bibliography{soft} 

\end{document}