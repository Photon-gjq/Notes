\chapter{液滴}

与大片的流体不同,液滴的行为和它周围的环境和表面的性质有很强的联系。

\section{飞溅}

在低气压环境中,液体落在表面上不会出现飞溅现象。这意味着飞溅来自液体和周围气体的接触:如果这种接触导致液体表面不稳定,那么飞溅就能够发生。
在气压不变时,将液体落到的表面替换成多孔板,从而液体和表面之间的气体能够随着液体下降被快速导走,同样不会出现飞溅现象。
因此,液体和表面之间的气体膜决定了飞溅是否出现。

