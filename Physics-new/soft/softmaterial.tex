\part{软凝聚态物理}

软凝聚态物理顾名思义仍然研究凝聚态物质,也就是由原子、分子通过各种化学键或是分子间作用力结合得到的物质,不过其注意力并不在通常的固体物理或者说“硬”凝聚态物理上面(这些话题见\soliddoc),而是在\concept{软物质}上。
固体物理中原子是定死在晶格上的,其运动通过声子体现,其对晶格结构的偏离用杂质、无序等体现,低能自由度是声子、电子。
流体中的原子和分子则是可以自由移动的,通常量子效应比较微弱,经典动理学可以描述它们。
流体中最为简单的大概就是牛顿流体了,这包括水、空气、各种油等等,描述它们的理论是纳维-斯托克斯方程。
而软物质中则有大量尺度在纳米级乃至微米级的\concept{基元}:如果我们往水中放入一些大一点的基元呢?糨糊是水和一些粉末的混合物,但是剪切增稠效应就出现了。
二氧化硅纳米颗粒的介观基元自组织能够形成光子晶体,从而产生结构色,其剪切增稠效应能够用于制造防弹织物。
高分子溶液从细管中涌出时会膨胀(想想挤牙膏),当一根棒子在其中搅动时能够爬上棒子。
正如硬凝聚态物理中,四处移动的电子能够形成新奇的物态、各种集体模式,简单流体(气体和液体)的流速场能够形成各种有趣的模式一样,软物质中的基元能够自组织、产生非牛顿效应,以及其它奇特现象。
从上面的说法可以看出软物质说白了就是\concept{复杂流体}:是流体,但是不是普通的液体和气体,其中有大量介观尺寸、响应时间能慢到\SI{0.1}{s}量级的基元,存在大量比化学键弱的相互作用,其量级在$k_\text{B} T$,和熵效应相当。

胶体、泡沫、泡在液体中的颗粒物质(沙粒,石块,甚至冰)都是软物质的例子。向液体中加入足量的颗粒几乎总是会产生软物质体系。

本文中一些其它的主题——比如说普通液体的液滴等——也常常被纳入软物质的范畴。

\chapter{液体中的颗粒以及非牛顿流体}\label{chap:non-newtonian}

\section{悬浊液}

\subsection{一些唯象的考虑和经验公式}

考虑一个\concept{悬浊液},或者说浸泡在液体中的颗粒系统。设其体积分数为$\phi$。简单的几何考虑告诉我们$\phi$必然有一个小于1的上限,因为颗粒的堆积不可能完全占据每一片空间。
一般来说液体中最大的体积分数为$\phi_\text{m} \approx 0.64$。
$\phi_\text{m}$显然不会达到最密堆积的极限,因为最密堆积虽然是全局最优值,但是并不容易达到。
在$\phi \approx \phi_\text{m}$时体系和普通固体没有什么差别,而$\phi \approx 0$时体系和普通液体没什么区别。

在体积分数相同时,一般颗粒体积小的悬浊液粘度更大,因为颗粒更多,也更容易散射。

\subsection{粘度的电场调控}

理论上通过施加电场来改变流体性质,不过对普通液体,可能需要\SI{1e8}{V/m}量级的电场,并不现实。
悬浊液中的粒子可以使用小得多的电场来调控,因此可以使用电场来调控悬浊液的各种性能。
设想一个悬浊液被放置在电场当中,而其中的颗粒能够容易地极化。电偶极子之间存在相互作用,因此,将悬浊液放置在电场中可以让颗粒吸附在一起。
此时$\phi$不变,但是颗粒体积增大,数目减少,从而粘度大大下降。
不过如果电场施加时间过长,那么会形成宏观的链条,粘度反而会增大。

这种技术可能可以用来改善高血脂患者的健康状况:让他的血管被放置在强场中,脂肪吸附在一起,就能够让血液变稀。

\section{泡沫}

各种密堆积