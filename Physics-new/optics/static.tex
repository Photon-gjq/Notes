\chapter{静电学}

\section{静电系统的基本方程}

本节讨论仅含有导体和线性电介质的静电系统。所谓\concept{静电}指的是系统中没有任何电流的情况,此时我们有$\div{\vb*{j}}=0$,从而电荷密度分布没有变化。
注意我们没有使用“电荷密度没有变化”作为定义,因为恒稳电路也具有这样的性质,但是与此同时的确有电荷流动。
在静电的条件下我们有
\[
    \curl{\vb*{E}} = - \pdv{\vb*{B}}{t}, \quad \curl{\vb*{B}} = \mu_0 \epsilon_0 \pdv{\vb*{E}}{t},
\]
然后我们会发现磁场$\vb*{B}$满足一个波动方程。看起来这非常奇怪,因为在根本没有电流的时候磁场怎么会存在呢?
这实际上是来自边界条件的不清晰。在处理电磁波时我们并不要求无穷远处场强衰减至零,因为我们实际上是认为电磁波由非常远的地方的一个源产生的,没完没了地传播到其它地方,从而处理电磁波时“无穷远处”有源是完全可以的。
在静电学中我们要求电荷约束在一个有限的范围内,从而无穷远处场强快速衰减,那么磁场满足的波动方程如果要有非平凡解,只能取类似球面波的形式,但是这样一来$\div{\vb*{B}}$会不为零,和磁场无源的条件违背。
总之,在静电学情况下$\vb*{B}=0$。于是我们就有静电学方程
\begin{equation}
    \div{\vb*{E}} = \frac{\rho}{\epsilon_0}, \quad \curl{\vb*{E}} = 0,
    \label{eq:static-e-field}
\end{equation}
或者
\begin{equation}
    \vb*{E} = - \grad{\varphi}, \quad \laplacian{\varphi} = - \frac{\rho_0}{\epsilon_0}.
    \label{eq:static-phi-field}
\end{equation}
这是拉普拉斯方程,有现成的通解,即
\begin{equation}
    \varphi(\vb*{r}) = \int \dd[3]{\vb*{r}'} \frac{1}{4\pi \epsilon_0} \frac{\rho(\vb*{r}')}{\abs*{\vb*{r} - \vb*{r}'}}.
    \label{eq:from-q-to-phi}
\end{equation}

静电学中能量可以看成是电荷携带的而不是电场携带的。这是因为一个区域内的能量为
\[
    E = \int \dd[3]{\vb*{r}} \frac{1}{2} \epsilon_0 \vb*{E}^2 = \frac{1}{2} \epsilon_0 \int \dd[3]{\vb*{r}} (\grad{\varphi})^2,
\]
做分部积分,并使用无穷远处场强为零这一条件,得到
\begin{equation}
    E = - \frac{\epsilon_0}{2} \int \dd[3]{\vb*{r}} \varphi \laplacian{\varphi} = \int \dd[3]{\vb*{r}} \rho \varphi.
\end{equation}
因此在静电学的情况下能量可以认为是定域在电荷周围的。

总之,求解
\begin{equation}
    \varphi|_\text{surface} = \const,
\end{equation}
\begin{equation}
    \pdv{\varphi}{\vb*{n}} = - \frac{\sigma}{\epsilon}
\end{equation}

真空中或者均匀线性电介质中不能有电势极大值或者极小值,因为在这样的区域内$\varphi$是调和函数,而调和函数在它调和的区域内部不能有极大值、极小值。
物理上这很好理解,如极大值出现意味着从这一点向它周围的各个方向都有电场,因此这一点上应该有电荷,矛盾。

在计算静电系统中导体的受力时,不能简单地将无导体的空间内的电场外推到导体表面,然后使用$\vb*{f}=\sigma\vb*{E}$,因为导体表面电场是不连续的。
更加物理地看,这是因为导体表面实际上是非常复杂的一个系统:电场在微观层面快速衰减,表面上的电荷之间有相互作用力,数量级估计可以发现这些电荷之间的相互作用力和电荷受到的电场力是同阶的,因此简单的$\vb*{f}=\sigma\vb*{E}$会漏掉一部分作用力。
最为可靠的方法是使用麦克斯韦张量来计算,因为动量守恒是总是成立的,则在静电情况下动量流是连续的,所以直接计算导体外的麦克斯韦张量然后外推到导体表面即可。%
\footnote{
    这里还有一个可能的疑难:麦克斯韦张量计算的是电场对自由电荷的作用力,但是首先导体上的电荷并不是自由的,其次我们要计算的也是导体受到的作用力。
    但是,电磁场本身对导体并没有任何作用,而由受力平衡,导体对电荷施加的作用力应该和电场对电荷施加的作用力平衡,于是电场对电荷的作用力就传递给了导体。
}%

\section{唯一性定理}

唯一性定理成立的条件是$\vb*{D}$和$\vb*{E}$之间的关系应该是一一对应的。
反之,在两者之间的关系实际上不一一对应的时候,唯一性定理就被破坏了。例如,如果$\vb*{D}-\vb*{E}$关系实际上构成了一条电滞回线,那就没有唯一性定理。

% TODO: 有限大小的体系内电荷总量应该为零:这是高斯定理的推论

\[
    \int \dd[3]{\vb*{r}} (\varphi_2 \laplacian{\varphi_1} - \varphi_1 \laplacian{\varphi_2}) = \int \dd{\vb*{S}} (\varphi_2 \grad{\varphi_1} - \varphi_1 \grad{\varphi_2}),
\]
右边是
\begin{equation}
    \int \dd[3]{\vb*{r}} \varphi_1 \rho_2 = \int \dd[3]{\vb*{r}} \varphi_2 \rho_1.
\end{equation}
在导体系统中电荷仅仅分布在导体表面上,而且同一个导体表面电势处处相同,于是
\begin{equation}
    \sum_i \varphi_i^{(1)} q_i^{(2)} = \sum_i \varphi_i^{(2)} q_i^{(1)}.
\end{equation}

\section{电多极子}

设空间中的电荷密度为$\rho$,可能还要算上面密度,电势为
\[
    \varphi(\vb*{r}) = \frac{1}{4\pi \epsilon_0} \int \dd[3]{\vb*{r}'} \frac{1}{\abs*{\vb*{r} - \vb*{r}'}} \rho(\vb*{r}'),
\]
对$1/\abs*{\vb*{r}-\vb*{r}'}$做多极展开:
\[
    \frac{1}{\abs*{\vb*{r}-\vb*{r}'}} = \frac{1}{\abs*{\vb*{r}}} - \vb*{r}' \cdot \grad{\frac{1}{\abs*{\vb*{r}}}} + \frac{1}{2} \vb*{r}' \vb*{r}' : \grad{\grad{\frac{1}{\abs*{\vb*{r}}}}} + \cdots,
\]
就得到一个$\varphi$的展开式,即所谓\concept{多极展开},其中
\begin{equation}
    \varphi^{(0)}(\vb*{r}) = \frac{1}{4\pi \epsilon_0} \frac{1}{\abs*{\vb*{r}}} \underbrace{\int \dd[3]{\vb*{r}'} \rho(\vb*{r}')}_{Q}
\end{equation}
就是将整个体系当成一个点电荷计算得到的电势,
\begin{equation}
    \begin{aligned}
        \varphi^{(1)}(\vb*{r}) &= - \frac{1}{4\pi \epsilon_0} \grad{\frac{1}{\abs*{\vb*{r}}}} \cdot \int \dd[3]{\vb*{r}'} \rho(\vb*{r}') \vb*{r}' \\
        &= \frac{1}{4\pi \epsilon_0} \frac{\vb*{r}}{\abs*{\vb*{r}}^3} \cdot \underbrace{\int \dd[3]{\vb*{r}'} \rho(\vb*{r}') \vb*{r}'}_{\vb*{p}}
    \end{aligned}
\end{equation}
是电偶极电势,电四极矩是
\begin{equation}
    \varphi^{(2)}(\vb*{r}) = \frac{1}{4 \pi \epsilon_0} \frac{1}{6} \grad{\grad{\frac{1}{\abs*{\vb*{r}}}}} : \underbrace{3 \int \dd[3]{\vb*{r}'} \rho(\vb*{r}') \vb*{r}' \vb*{r}' }_{\vb*{D}}.
\end{equation}
在电荷分布相对于坐标系原点空间反演对称时,电偶极矩是零,而当电荷分布相对于坐标系原点空间反演反对称时,电四极矩是零。

容易看出,电四极矩$D_{ij}$是对称的,因此有6个独立分量。实际上这些独立分量并不都是有用的,注意到$\vb*{r} \neq 0$时
\[
    \laplacian{\frac{1}{\abs*{\vb*{r}}}} = 0,
\]
我们发现
\[
    \grad{\grad{\frac{1}{\abs*{\vb*{r}}}}} : \vb*{I} = 0,
\]
即我们可以任意地在$\vb*{D}$中加上单位张量的倍数,而不改变电势分布。因此我们可以手动加入一个约束:定义\concept{约化电四极矩}
\begin{equation}
    \tilde{\vb*{D}} = \vb*{D} - \frac{1}{3} \trace(\vb*{D}) \vb*{I} = \int \dd[3]{\vb*{r}'} (3 \vb*{r}' \vb*{r}' - \abs*{\vb*{r}'}^2 \vb*{I}) \rho(\vb*{r}') ,
\end{equation}
将$\vb*{D}$的迹消除掉,然后用$\tilde{\vb*{D}}$代替$\vb*{D}$同样可以得到正确的电四极矩;$\tilde{\vb*{D}}$独立的分量有5个,因为读多了一个无迹的条件。

电四极矩造成的电势衰减得比电偶极矩造成的电势快,电偶极矩造成的电势的衰减又比点电荷快。随着场点越来越接近源点,越来越复杂的电多极矩结构开始展现出来。

在$\vb*{r}$很大,即电场源离我们很远,其长度尺度趋于零时,我们需要将电场源替换成某种“点源”。我们不能朴素地将$\rho(\vb*{r})$替换成一个$\delta$函数,因为这样无法拿到电多极子的信息。正确的做法是
\begin{equation}
    \rho(\vb*{r}) = \delta(\vb*{r} - \vb*{r}_0) - \vb*{p} \cdot \grad \delta(\vb*{r} - \vb*{r}_0) + \cdots,
\end{equation}
即通过$\delta$函数的导数来引入“电场源的内部不均匀分布”的信息;通过分部积分法可以将导数转移到$1 / \abs*{\vb*{r}}$上,我们就能够拿到电多极子了。

\chapter{静磁学}

和静电学类似,我们可以考虑恒定电流的情况,即虽然有电流但是没有任何电荷变化,电流强度也不变的情况,则由输运方程有$\div{\vb*{j}}=0$。
我们可以直接引用\eqref{eq:wave-eq-general},得到
\[
    \frac{1}{c^2} \pdv[2]{\vb*{E}}{t} - \laplacian{\vb*{E}} = - \frac{1}{\epsilon_0} \grad{\rho} , \quad \frac{1}{c^2} \pdv[2]{\vb*{B}}{t} - \laplacian{\vb*{B}} = \mu_0 \curl{\vb*{j}},
\]
由于$\rho$和$\vb*{j}$都不随时间变化,如果我们像在静电学中一样,要求无穷远处场强衰减足够快,那么以上两式可以直接化为静态的拉普拉斯方程
\[
    \laplacian{\vb*{E}} = \frac{1}{\epsilon_0} \grad{\rho}, \quad \laplacian{\vb*{B}} = - \mu_0 \curl{\vb*{j}}.
\]
由于$\vb*{B}$不会变化,我们直接得到\eqref{eq:static-e-field},于是就可以求解出电场。
至于$\vb*{B}$,引入磁矢势,就得到
\[
    \curl{\laplacian{\vb*{B}}} = - \mu_0 \curl{\vb*{j}},
\]
那么只需要取库伦规范$\div{A}=0$就可以有
\[
    \div{\laplacian{\vb*{B}}} = - \mu_0 \div{\vb*{j}},
\]
于是就得到
\begin{equation}
    \vb*{B} = \curl{\vb*{A}}, \quad \laplacian{\vb*{A}} = - \mu_0 \vb*{j}.
\end{equation}
这个方程的形式和\eqref{eq:static-phi-field}非常相似,也是拉普拉斯方程,从而直接可以写出
\begin{equation}
    \vb*{A}(\vb*{r}) = \int \dd[3]{\vb*{r}'} \frac{\mu_0}{4\pi} \frac{\vb*{j}(\vb*{r}')}{\abs*{\vb*{r} - \vb*{r}'}}.
    \label{eq:from-j-to-a}
\end{equation}

总之,在电荷分布不变、电流分布不变的情况下,电场可以用$\rho$表示出来,并且是无旋场;磁场可以用$\vb*{j}$表示出来。在存在$\vb*{j}$的情况下,电荷分布不变,电场的形式和静电场完全一样,但磁场的存在会导致和静电学不同的一些物理现象,因此此时的电场可以称为\concept{恒定电场}。

静磁学指的是存在电荷流动,但是各个物理量的分布都恒稳的情况。要保持电流存在必须有一个外部的驱动力(\concept{非静电力}),这意味着此时的电磁场%
\footnote{
    当然,这个外部驱动力通常归根到底也是电磁力;但是我们将与它有关的那部分场自由度积掉了。
}%
不再是一个孤立体系。这可能让一些使用能量做的推导不再成立。%
\footnote{
    一种可能的诘难是,维持静电场的稳定也需要外部力(恩肖定理),为什么我们从来将静电场当成孤立系统看待?
    原因是,单纯从理论上说,要维持静电场稳定我们只需要将各个导体、电荷的动力学“关掉”即可(如认为点电荷受力不运动),等价的,维持静电场稳定的外力并不做功。
    另一方面,我们不能对电流做同样的事情:我们必须引入电流和电场之间的本构关系,从而自然地产生一个能量耗散项。
    静磁学理论中不可能不考虑这个能量耗散项。
}%

\begin{equation}
    \vb*{A}(\vb*{r}) = \frac{\mu_0}{4\pi} \frac{I \vb*{S} \times \vb*{r}}{\abs*{\vb*{r}}^3} = \frac{\mu_0}{4\pi} \frac{\vb*{m} \times \vb*{r}}{\abs*{\vb*{r}}^3}.
\end{equation}

\begin{equation}
    \vb*{B}(\vb*{r}) = - \frac{\mu_0}{4\pi} \left( \frac{\vb*{m} - 3 (\vb*{m} \cdot \vb*{e}_r) \vb*{e}_r}{r^3} \right).
\end{equation}

由于在边界上$\vb*{B}$有限大,应有
\begin{equation}
    \vb*{n} \times (\vb*{A}_2 - \vb*{A}_1) = 0.
\end{equation}
对库伦规范,

讨论静磁学系统的能量时需要把电源考虑进去,因为系统构型的小的变化会带来一个感生电动势,从而改变一些分布?

磁场的多极展开从$1$开始编号。