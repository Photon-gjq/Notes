\chapter{偶极辐射的量子理论}

\section{二能级系统的偶极辐射}

考虑一个二能级系统——比如说氢原子——它正处在状态
\begin{equation}
    \ket*{\psi(t=0)} = c_\text{g} \ket*{\text{g}} + c_\text{e} \ket*{\text{e}}
\end{equation}
上。如果它不受到外界扰动,其时间演化是显然的:
\begin{equation}
    \ket*{\psi} = c_\text{g} \ee^{- \ii \omega_\text{g} t} \ket*{\text{g}} + c_\text{e} \ee^{- \ii \omega_\text{e} t} \ket*{\text{e}}.
\end{equation}
电偶极矩可能在基态和激发态上都为零,但是这不意味着$\mel{\text{e}}{\vb*{d}}{\text{d}} \eqqcolon \vb*{d}_\text{eg}$也是零。
倘若它不是零,那么电偶极矩期望值就是
\begin{equation}
    \expval*{\vb*{d}}(t) = c^*_\text{e} c_\text{g} \vb*{d}_{\text{eg}} \ee^{- \ii \omega_\text{eg} t} + \text{c.c.},
\end{equation}
其中$\omega_\text{eg}$
这意味着二能级系统如果处在基态和激发态的叠加态上,它有能力辐射出频率为$\omega_\text{eg}$的电磁波。

\section{调制光束}

偶极辐射产生的光能够传向四面八方。我们需要想出办法来将偶极辐射转化为平面波,再聚焦到需要的位置。

\chapter{激光}

本节首先介绍强光如何能够产生。

如果我们能够有一个长期保持粒子数反转的系统(显然需要持续的能量输入),那么向这个系统入射一束光将会产生更强的出射光,因为会有受激发射,且受激发射出的光和入射光是非常相干的。
因此我们向粒子数反转的系统注入的能量可以用于增强入射光,并且如果入射光相干性非常好,那么出射光也保持非常好的相干性,并且比入射光更亮,这就是\concept{激光}。

最简单的方案——使用一个二能级系统,直接通过一次入射来得到激光——是不现实的,因为此时大量的能量会消耗在保持粒子数反转上,而为了 TODO
两个可能的改进:将粒子数反转的系统放在一个四壁强反射的腔体内,从而光束可以来回走,不断被增强,并且使用一个三能级系统,其中间能级是一个亚稳态,从而很容易制造粒子数反转。
可以在腔体的一个地方“开洞”——比如说让反射率稍微低一些——让一些光子泄露出来,激光就从这里被导出。

多光子过程:可以避免光学屏障,以及排除荧光本底