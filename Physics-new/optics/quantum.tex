\chapter{单光子现象}

\section{单光子器件}

分束器

\section{证明单光子量子性的实验}

\subsection{Aspect实验}

要证明单光子具有量子性,最好的办法是使用一些这样的实验:它的一种版本能够证明光子的粒子性,它的另一个仅仅做了少许修正的版本(比如说探测器被移动到别的位置)能够证明光子的波动性。
两个版本区别很小这件事能够排除实验装置和光的复杂相互作用显著地改变了光的行为这样的说法,而粒子性和波动性同时出现则强烈暗示需要量子理论描述光。
1986年的Aspect实验是这种实验的一个典范。

\subsection{Zeilinger实验}

A.Zeilinger
一种更加简明的实验是这样的:同样使用分束器和反射镜,构造这样的光路:

\subsection{如果狄拉克是错的……}

\subsubsection{两个激光器产生的光束的干涉}

两个激光器产生的光似乎是相干的?

实际上激光器产生的是相干态光而不是光子数确定的多光子玻色波函数。

\subsubsection{玻色-爱因斯坦凝聚态中的干涉}

\subsection{Hanbury Brown和Twiss效应}

\subsubsection{天狼星上的光的疑似量子性}

\subsubsection{经典电动力学解释}

\chapter{纠缠光}