\documentclass[hyperref, UTF8, a4paper, oneside]{ctexbook}

\usepackage{geometry}
\usepackage{titling}
\usepackage{titlesec}
\usepackage{paralist}
\usepackage{footnote}
\usepackage{enumerate}
\usepackage{amsmath, amssymb, amsthm}
\usepackage{mathtools}
\usepackage{cite}
\usepackage{booktabs}
\usepackage{multirow}
\usepackage{graphicx}
\usepackage{subfigure}
\usepackage{physics}
\usepackage{siunitx}
\usepackage[compat=1.1.0]{tikz-feynhand}
\usepackage{xr-hyper}
\usepackage[colorlinks, linkcolor=black, anchorcolor=black, citecolor=black, filecolor=black]{hyperref}
\usepackage[most]{tcolorbox}
\usepackage{caption}
\usepackage{prettyref}

\externaldocument[qft-]{../relativistic-qft/relativistic-qft}[relativistic-qft.pdf]
\externaldocument[solid-]{../solid/solid}[solid.pdf]

\geometry{left=3.28cm,right=3.28cm,top=2.54cm,bottom=2.54cm}
\titlespacing{\paragraph}{0pt}{1pt}{10pt}[20pt]
\setlength{\droptitle}{-5em}
\preauthor{\vspace{-10pt}\begin{center}}
\postauthor{\par\end{center}}

\newcommand*{\ee}{\mathrm{e}}
\newcommand*{\ii}{\mathrm{i}}
\newcommand*{\const}{\mathrm{const}}
\newcommand*{\natnums}{\mathbb{N}}
\newcommand*{\reals}{\mathbb{R}}
\newcommand*{\complexes}{\mathbb{C}}
\DeclareMathOperator{\timeorder}{T}
\newcommand*{\ogroup}[1]{\mathrm{O}(#1)}
\newcommand*{\sogroup}[1]{\mathrm{SO}(#1)}
\DeclareMathOperator{\legpoly}{P}

\newrefformat{sec}{第\ref{#1}节}
\newrefformat{note}{注\ref{#1}}
\newrefformat{fig}{图\ref{#1}}
\newrefformat{part}{第\ref{#1}部分}
\newrefformat{tbl}{表\ref{#1}}
\newrefformat{back}{背景知识\ref{#1}}
\newrefformat{info}{资料框\ref{#1}}
\newrefformat{warn}{注意事项\ref{#1}}
\renewcommand{\autoref}{\prettyref}

\newcommand{\concept}[1]{\underline{#1}}
\renewcommand{\emph}{\textbf}

\usetikzlibrary{arrows,shapes,positioning}
\usetikzlibrary{arrows.meta}
\usetikzlibrary{decorations.markings}
\tikzstyle arrowstyle=[scale=1]
\tikzstyle directed=[postaction={decorate,decoration={markings,
    mark=at position .5 with {\arrow[arrowstyle]{stealth}}}}]
\tikzstyle ray=[directed, thick]

\tikzfeynhandset{
    every boldphoton@@/.style={
    /tikz/draw=none,
    /tikz/postaction={
            /tikz/draw,
            /tikz/double,
            /tikz/line width = \feynhandlinesize,
            /tikz/decoration={
                complete sines,
                amplitude=3\feynhandlinesize,
                segment length=7.5\feynhandlinesize,
            },
            /tikz/decorate=true,
        },
    },
    every boldphoton/.style={/tikzfeynhand/every boldphoton@@/.append style={#1}},
    boldphoton/.style={
    /tikzfeynhand/every boldphoton@@,
    },
}

\tikzfeynhandset{
    every extphoton@@/.style={
        /tikz/draw=none,
        /tikz/decoration={name=none},
        /tikz/postaction={
          /tikz/draw,
          /tikz/line width = \feynhandlinesize,
          /tikzfeynhand/with arrow=0.9,
        }
    },
    every extphoton/.style={/tikzfeynhand/every extphoton@@/.append style={#1}},
    extphoton/.style={
    /tikzfeynhand/every extphoton@@,
    }
}

\tikzfeynhandset{
    every outphoton@@/.style={
    /tikz/draw=none,
    /tikzfeynhand/with arrow=0.9,
    /tikz/postaction={
            /tikz/draw,
            /tikz/line width = \feynhandlinesize,
            /tikz/decoration={
                complete sines,
                amplitude=3\feynhandlinesize,
                segment length=7.5\feynhandlinesize,
            },
            /tikz/decorate=true,
        },
    },
    every outphoton/.style={/tikzfeynhand/every outphoton@@/.append style={#1}},
    outphoton/.style={
    /tikzfeynhand/every outphoton@@,
    },
}

\tcbuselibrary{skins, breakable, theorems}

\newtcbtheorem[number within=chapter]{back}{背景知识}%
  {colback=blue!5,colframe=blue!65,fonttitle=\bfseries, breakable}{back}
\newtcbtheorem[number within=chapter]{info}{资料框}%
  {colback=blue!5,colframe=blue!65,fonttitle=\bfseries, breakable}{info}
\newtcbtheorem[number within=chapter]{warning}{注意事项}%
  {colback=orange!5,colframe=orange!65,fonttitle=\bfseries, breakable}{warn}

\newenvironment{bigcase}{\left\{\quad\begin{aligned}}{\end{aligned}\right.}

\numberwithin{equation}{chapter}

\newcommand{\qftdoc}{\href{../relativistic-qft/relativistic-qft.pdf}{相对论性量子场论笔记}}
\newcommand{\soliddoc}{\href{../solid/solid.pdf}{固体物理笔记}}

\title{电磁场,电磁波,光学}
\author{吴晋渊}

\begin{document}

\maketitle

\part{介质的电动力学性能}

\chapter{晶体}\label{chap:components}

\section{原子物质的哈密顿量}

普通的固体、液体、气体由一系列原子组成。通过实验和计算可以发现,原子的最外层电子在各种过程中容易发生重新排列,称为\concept{价电子};内层电子和原子核(合称为\concept{离子实})则通常保持为一个整体,也即,其内部状态发生变化的物理过程的描述需要使用QCD,其涉及的能标远高于价电子发生变化涉及的能标。

本文基本上只分析涉及价电子低能运动的物理过程,即只讨论非相对论极限下的电荷-电磁场耦合系统,而忽略强相互作用、弱相互作用和引力。
此时,带电粒子由薛定谔场完全描述,系统具有$U(1)$规范对称性且无粒子数生灭,有确定的粒子数,于是可以原则上实物粒子部分可以直接用单粒子量子力学描述。
进一步,我们假定系统中没有变化特别快的电磁场,这意味着实际上我们可以积掉电磁场并且得到一个延迟不明显的相互作用,那就是说,我们可以把所有电磁相互作用都用静电学和静磁学处理。由于无论是电子还是离子实都是非相对论性的,电子-电子相互作用、电子-离子实相互作用、离子实-离子实相互作用几乎完全是库伦相互作用。

\subsection{动能和库伦能}

设有$N_e$个价电子,$N_i$个离子实(i表示离子),不考虑外界扰动的一次量子化哈密顿量为
\begin{equation}
    {H} = {H}_\text{e} + {H}_\text{i} + {H}_\text{ei},
    \label{eq:many-body-hamiltonian}
\end{equation}
其中${H}_\text{e}$表示仅涉及价电子的哈密顿量,${H}_\text{i}$表示仅涉及离子实的哈密顿量,最后一项则是两者的相互作用,所有相互作用是库仑相互作用。
诸价电子组成的系统就好像由电子组成的气体,称为\concept{相互作用电子气}。单体哈密顿量为电子的动能项加上单体势能项。在物质不受外界作用时当然不应该有单体势能项,于是
\[
    {H}_\text{e1} = \frac{{\vb*{p}}^2}{2m},
\]
在坐标表象下它就是
\[
    {H}_\text{e1} = - \frac{\laplacian}{2m}.
\]
二体哈密顿量为电子两两作用而产生的库伦势能是
\[
    {H}_\text{e2} = \frac{e^2}{\abs{\vb*{r}_1 - \vb*{r}_2}},
\]
从而价电子本身的能量以及它们之间发生库伦相互作用的能量就是
\begin{equation}
    {H}_\text{e} = \sum_{i=1}^{N_\text{e}} \frac{{p}_i^2}{2m_\text{e}} + \frac{1}{2} \sum_{i\neq j} \frac{e^2}{\abs{\vb*{r}_i - \vb*{r}_j}}.
\end{equation}
使用类似的方法,离子实的组成的系统(如果是晶体那就是晶格)的哈密顿量为
\begin{equation}
    {H}_\text{i} = \sum_{\alpha=1}^{N_\text{i}} \frac{{p}_\alpha^2}{2m_i} + \frac{1}{2} \sum_{\alpha\neq\beta} V(\vb*{R}_\alpha-\vb*{R}_\beta).
\end{equation}
由于离子实中的内层电子结构复杂,离子实之间的相互作用能写不出特别简单的表达式(我们相当于把内层电子的自由度也积掉了)。请注意这个相互作用能是平移不变的,这是当然的,因为QED是平移不变的;但是实际的固体在短距离上并不是平移不变的,因为在低能下有对称性自发破缺。
离子实和价电子的相互作用则是
\begin{equation}
    {H}_\text{ei} = \sum_{\alpha, i} V_\text{ei}(\vb*{r}_i-\vb*{R}_\alpha). 
\end{equation}
分别使用$i$表示价电子,用$\alpha$表示离子实;由于价电子和离子实不全同,不需要加上$1/2$系数。
同样我们还是假定了相互作用本身的平移不变性。
本文仅仅讨论固体(实际上主要是晶体)的物理,因此我们将假定离子实的位移始终局限在非常小的范围内。

\begin{back}{量子涨落}{quantum-fluctuation}
    “量子涨落”一词的含义其实有一些不清楚之处。

    由于量子理论中的哈密顿量可以有彼此不对易的项,计算该哈密顿量的配分函数必然会得到一个路径积分,每条路径上,不同虚时间点的物理量可以任意取值,这个现象就可以称为量子涨落。
    这个说法的经典极限此时是有一定的微妙之处的,就平衡态统计中的配分函数而言,取$\hbar \to 0$似乎会让配分函数变成简单的$\sum \ee^{-\beta H}$,每条路径都没有虚时间演化;但是另一方面,经典动力学中$\hbar \to 0$时虽然哈密顿量中没有不对易的项,但是物理量的时间演化还是有的。
    这个微妙之处是必然会出现的,因为量子力学中我们依靠算符的非平庸对易关系($[H, A]$)来获得时间演化,求经典极限时,直接关于哈密顿量的配分函数在$\hbar \to 0$极限下退化为$\sum \ee^{-\beta H}$,但是海森堡运动方程由于有$1/\hbar$的前置因子,会给出经典的哈密顿运动方程。
    这也体现出了“经典统计物理”的内在张力:经典统计物理相比量子统计物理更加难以良定义。

    在求解薛定谔方程时,可以将发生得较慢的过程涉及到的物理量当成不变的,或者将量子效应不明显的那部分系统用经典动力学处理,这就称为“忽略量子涨落”。

    经典物理的微扰求解实际上也是可以用费曼图表示的,只不过这种费曼图中的粒子线的方向必须严格体现“事件发生的顺序”。因此,量子场论中超出经典场论的费曼图的那部分费曼图也可以称为量子涨落。
\end{back}

在大部分过程中,由于原子核的质量比电子的质量大至少三个数量级,涉及价电子的过程通常比涉及离子实的过程发生得快很多,从而在价电子的时间尺度上,诸离子实的位置可以看成是给定的。
从而,在分析价电子时我们可以将${H}_i$项直接略去,并忽略离子实位置的量子涨落,而将${H}_\text{ei}(\vb*{r}_i-\vb*{R}_\alpha)$项对$\vb*{R}_\alpha$求和得到$V_\text{ion}(\vb*{r}_i)$。这个近似称为\concept{玻恩–奥本海默近似}。这样一来相互作用电子气的一次量子化哈密顿量在坐标表象下就是
\begin{equation}
    {H} = \sum_{i=1}^{N_\text{e}} \left( - \frac{\laplacian}{2m_\text{e}} + V_\text{ion}(\vb*{r}_i)\right) + \frac{1}{2} \sum_{i\neq j} \frac{e^2}{\abs{\vb*{r}_i - \vb*{r}_j}},
    \label{eq:electron-gas-hamiltonian}
\end{equation}
从而二次量子化哈密顿量为
\begin{equation}
    \begin{aligned}
        {H} = &\sum_{\sigma} \int \dd[3]{\vb*{r}} {\psi}_\sigma^\dagger(\vb*{r}) \left( - \frac{\laplacian}{2m} + V_\text{ion}(\vb*{r}) \right) {\psi}_\sigma(\vb*{r}) \\
        &+ \frac{1}{2} \sum_{\alpha, \beta} \int \dd[3]{\vb*{r}_1} \int \dd[3]{\vb*{r}_2} 
        {\psi}_\alpha^\dagger (\vb*{r}_1) {\psi}_\beta^\dagger (\vb*{r}_2) \frac{e^2}{\abs{\vb*{r}_1 - \vb*{r}_2}} {\psi}_\beta (\vb*{r}_2) {\psi}_\alpha (\vb*{r}_1). 
    \end{aligned}
    \label{eq:electron-gas-hamiltonian-sq}
\end{equation}
其中${\psi}^\dagger(\vb*{r})$是薛定谔场的场算符,它也是在位置为$\vb*{r}$的位置产生一个电子的产生算符。这个哈密顿量当然也可以通过QED的低能近似得到,但并没有必要这么做。请注意电子是费米子。
\eqref{eq:electron-gas-hamiltonian-sq}实际上不是对角的,因为它的单粒子项涉及一个梯度算符。
另一方面,离子实的相互作用可以认为遵从由电子总能量确定的等效势,%
\footnote{
    注意,先计算电子-电子库伦散射导致的能量然后用它计算晶格中离子实的相互作用最后计算离子实振动对电子的影响,这个过程中\emph{没有}双重计数,虽然我们表面上似乎“积掉了电子而计算等效的原子间相互作用”。
}%
可以列写晶格的薛定谔方程求解其位置。
然而由于离子实通常比较大,我们很多时候会彻底忽略离子实位置的量子涨落而使用分子动力学方法处理它。

最后我们指出,由于凝聚态介质始终可以和外界交换电子——外界的电子可以进入系统,系统中的电子可以溢出——系统中电子气本身的哈密顿量\eqref{eq:electron-gas-hamiltonian-sq}不足以充分描述系统。
本文将只研究近平衡系统,因此这种与外界的相互作用可以使用化学势描述,即我们需要在\eqref{eq:electron-gas-hamiltonian-sq}中加入一项$-\mu {\psi}^\dagger(\vb*{r}) {\psi}(\vb*{r})$,这样得到的哈密顿量才是完整的。
换而言之,电子气完整的哈密顿量形如
\begin{equation}
    \begin{aligned}
        {H} &= \sum_{\vb*{k}, \sigma} \Big( \underbrace{\frac{\vb*{k}^2}{2m}}_{\epsilon_{\vb*{k}}} - \mu \Big) {c}^\dagger_{\vb*{k} \sigma} {c}_{\vb*{k} \sigma} 
        + \int \dd[3]{\vb*{r}} V_\text{ion}(\vb*{r}) {\psi}^\dagger(\vb*{r}) {\psi}(\vb*{r}) \\ 
        &+ \frac{1}{2} \sum_{\alpha, \beta} \int \dd[3]{\vb*{r}_1} \int \dd[3]{\vb*{r}_2} 
        {\psi}_\alpha^\dagger (\vb*{r}_1) {\psi}_\beta^\dagger (\vb*{r}_2) \frac{e^2}{\abs{\vb*{r}_1 - \vb*{r}_2}} {\psi}_\beta (\vb*{r}_2) {\psi}_\alpha (\vb*{r}_1).
    \end{aligned}
    \label{eq:full-electron-gas-hamiltonian}
\end{equation}
在有些模型中有时也将外势场并入$\epsilon_{\vb*{k}}$项。
为了简便起见,通常用$\xi$表示扣除了化学势的单电子能量,即
\begin{equation}
    \xi_{\vb*{k}} = \epsilon_{\vb*{k}} - \mu.
\end{equation}

\subsection{哈密顿量的一般形式}

以上我们都是将库伦相互作用加入薛定谔场中,可以说是给出了第一性原理计算需要的哈密顿量(虽然实际上从高能物理的角度这远非第一性原理,但对凝聚态理论来说通常已经够用了)。
但实际上还有以下机制没有考虑:
\begin{itemize}
    \item 作用在单体上的外场的束缚,离子实的束缚已经被计入考虑了,但是还有其它外场,比如说或许会有一个磁场,然后哈密顿量中将会有一项$-\vb*{\mu} \cdot \vb*{B}$;无论如何这会是一个二体算符。
    \item 电子-声子相互作用会引入一个二电子和一个声子发生相互作用的顶角,积掉声子自由度之后会留下一个等效的电子-电子相互作用,这会让\eqref{eq:full-electron-gas-hamiltonian}中电子-电子相互作用的系数发生变化,不再是严格的库伦排斥。
\end{itemize}
于是我们写出一般形式的相互作用电子气的二次量子化哈密顿量:
\begin{equation}
    {H} = \sum_{\vb*{k}, \sigma} (T_{\vb*{k}} - \mu) {c}^\dagger_{\vb*{k} \sigma} {c}_{\vb*{k} \sigma} 
    + \sum_{\vb*{k}_1, \vb*{k}_2, \sigma} V_{\vb*{k}_1 \vb*{k}_2 \sigma} {c}^\dagger_{\vb*{k}_1 \sigma} {c}_{\vb*{k}_2 \sigma}
    + \sum_{\vb*{k}_1, \vb*{k}_2, \vb*{q}, \alpha, \beta} {c}^\dagger_{\vb*{k}_1+\vb*{q}, \alpha} {c}^\dagger_{\vb*{k}_2-\vb*{q}, \beta} V_{\vb*{q}} {c}_{\vb*{k}_2 \beta} {c}_{\vb*{k}_1 \alpha}. 
\end{equation}
相应的,热力学作用量为
\begin{equation}
    S = \sum_n \left( 
        \sum_{\vb*{k}, \sigma} (-\ii \omega_n + T_{\vb*{k}} - \mu) \bar{c}_{\vb*{k} \sigma} c_{\vb*{k} \sigma} 
        + \sum_{\vb*{k}_1, \vb*{k}_2, \sigma} V_{\vb*{k}_1 \vb*{k}_2 \sigma} \bar{c}_{\vb*{k}_1 \sigma} c_{\vb*{k}_2 \sigma} 
        + \frac{1}{2V} \sum_{\vb*{k}_1, \vb*{k}_2, \vb*{q}, \sigma} \bar{c}_{\vb*{k}_1+\vb*{q}, \sigma} \bar{c}_{\vb*{k}_2-\vb*{q}, \sigma} V_{\vb*{q}} c_{\vb*{k}_2 \sigma} c_{\vb*{k}_1 \sigma} \right). 
        \label{eq:general-electron-electron-interaction}
\end{equation}
上式中的动能项和相互作用项的形式由对称性确定,具体系数可以暂时不设置具体值(因为直接从\eqref{eq:full-electron-gas-hamiltonian}出发做重整化群计算显然是非常困难的)。
我们经常写出一些简化了的模型以复现某个现象,如忽略一部分动能(即忽略一部分电子跃迁方式),或者简化相互作用形式。

\begin{back}{量子多体系统中的理论与数值计算方法}{manybody-theory}
    \begin{itemize}
        \item 费曼图微扰计算(需要注意由于库伦相互作用是瞬时而超距的,通常将库仑相互作用顶角写成用虚线连接的两个顶角,每个顶角有一个电子入射和一个电子出射,虚线可以携带动量),由于相互作用是二体的,在相互作用较弱时计算一到二圈图就可以得到很好的效果,不过实际上相互作用并不总是那么弱,此时需要一些其它近似手段,如RPA近似等各种重求和方法。
        \item 做平均场计算并与实验或数值计算作比较(可以对原来的模型做平均场也可以对Hubbard-Stratonovich变换之后的辅助场做平均场,两者等价)。
        考虑平均场之上的涨落,获得一个关于平均值附近涨落的理论,具体方法通常是这样的:由于相互作用项是四阶的,可以使用Hubbard-Stratonovich变换引入一个辅助场,通过适当选取辅助场(通常要和某个有趣的序参量具有同样的对称性)并积掉电子自由度,则接近临界点时,辅助场满足的场论就给出了长程自由度。
        \item 密度泛函理论方法:凝聚态系统彼此不同的地方其实就在于晶格势场,我们只需要找到系统基态的一些量,如电子数密度等,能够和晶格势场建立一一对应关系即可,然后通过一些办法把系统总能量写成系统基态的这些量的泛函,通过优化该泛函,就得到了不少关于基态的性质,并且在适当的近似下能够计算出关于系统的全部信息。
        \item 量子蒙特卡洛方法、张量网络方法等数值方法。
    \end{itemize}

    量子多体系统通常都足够复杂,且可能有很强的相互作用,以至于分析方法多种多样,但是没有哪一种能够占有支配地位。
\end{back}

\subsection{外加电磁场}

现在讨论外加电磁场导致的哈密顿量变化,或者说“辐射和物质相互作用”导致的哈密顿量变化。
一般的,设系统被放置在电磁场$(\varphi, \vb*{A})$中,则一次量子化哈密顿量(使用一次量子化哈密顿量是为了和经典的“一群电子定向移动”的物理图像对应上)为
\begin{equation}
    {H} = \frac{1}{2m} \sum_i ({\vb*{p}_i} - q_i \vb*{A}({\vb*{r}_i}))^2 + \sum_i q_i \varphi({\vb*{r}}_i) + {H}_\text{int},
    \label{eq:hamiltonian-with-eb-original}
\end{equation}
其中$\vb*{p}$是正则动量,${H}_\text{int}$表示粒子间相互作用。
本文仅考虑外加电磁场产生的线性响应,于是考虑辐射场不很强以至于$\vb*{A}^2$可以忽略的情况,也即,仅保留单光子过程,或者说做偶极辐射近似。%
\eqref{eq:hamiltonian-with-eb-original}的导出见\opticsdoc,其中第\ref{optics-sec:particle-hamiltonian}节解释了将辐射场和库伦相互作用分开的方式,第\ref{optics-sec:multipole}节解释了偶极辐射近似。

对电子系统,$q=-e$,那么就有
\begin{equation}
    \begin{aligned}
        {H} &= \frac{1}{2m} \sum_i ({\vb*{p}_i} + e \vb*{A}({\vb*{r}_i}))^2 - e \sum_i \varphi({\vb*{r}}_i) + {H}_\text{int} \\ 
        &= - \frac{1}{2m} \sum_i (\grad + \ii e \vb*{A}({\vb*{r}_i}))^2 - e \sum_i \varphi({\vb*{r}}_i) + {H}_\text{int}.
    \end{aligned}
\end{equation}
设$\Omega$是某个空间区域,电流密度为$\vb*{J}$,则
\begin{equation}
    \int_\Omega \dd[3]{\vb*{r}} {\vb*{J}} = - \sum_i e {\vb*{v}}_i,
\end{equation}
其中$\vb*{v}_i$是电子移动的速度,满足
\begin{equation}
    {\vb*{v}}_i = \pdv{H}{\vb*{r}_i} = \frac{{\vb*{p}}_i + e \vb*{A}({\vb*{r}}_i)}{m}.
\end{equation}
考虑到$\Omega$的任意性,我们就有以下近似表达式:
\begin{equation}
    {\vb*{J}} = \underbrace{- \frac{e}{m} \sum_i ( \delta(\vb*{r} - {\vb*{r}}_i) {\vb*{p}} + {\vb*{p}} \delta(\vb*{r} - {\vb*{r}}_i) )}_{{\vb*{j}}} \underbrace{- \frac{e^2}{m} {n}_\text{e}}_{{\vb*{j}}_\text{D}} \vb*{A}.
\end{equation}
${\vb*{j}}$项特意被写成了厄米的形式;${\vb*{j}}_\text{D}$已经做了一遍粗粒化了,将诸$\vb*{A}_i$平均了一遍。
通常这是合理的,因为电磁波的波长通常不会特别小,从而不会有很大的空间起伏。(而如果有很大的空间起伏,我们就会使用cQED而不是经典电动力学讨论问题了)

现在写出略去高阶项的哈密顿量的形式。选取库伦规范,并认为$\varphi=0$,此时会发现,实际上我们有
\[
    {H} = \frac{1}{2m} \sum_i {\vb*{p}}_i^2 + {H}_\text{int} + \frac{e}{m} \sum_i {\vb*{p}}_i \cdot \vb*{A}({\vb*{r}}_i) - e \sum_i \varphi({\vb*{r}_i}),
\]
$\vb*{A}({\vb*{r}}_i)$和${\vb*{p}}_i$本来是不对易的,但是库伦规范下它们对易。
代入${\vb*{J}}$的表达式并再次略去高阶项,就有
\[
    {H} = \frac{1}{2m} \sum_i {\vb*{p}}_i^2 + {H}_\text{int} - \int \dd[3]{\vb*{r}} {\vb*{J}} \cdot \vb*{A} - e \sum_i \varphi({\vb*{r}_i}).
\]
至于含有电势的那一项,注意到电荷密度为
\[
    {\rho} = - e \sum_{i} \delta({\vb*{r}}_i - \vb*{r}),
\]
于是最后得到
\begin{equation}
    {H} = \frac{1}{2m} \sum_i {\vb*{p}}_i^2 + {H}_\text{int} + \int \dd[3]{\vb*{r}} \varphi {\rho} - \int \dd[3]{\vb*{r}} {\vb*{J}} \cdot \vb*{A}.
\end{equation}
虽然物质本身的哈密顿量不是洛伦兹协变的(因为取了非相对论近似),但是物质和辐射的相互作用项却是洛伦兹协变的——对电磁场的描述一般都是如此。
以上哈密顿量实际上仅仅讨论了轨道部分,电子还有自旋磁矩
\[
    {H}_{\text{spin}} = \sum_i {\vb*{\mu}}_i \cdot \vb*{B}({\vb*{r}}_i),
\]
我们可以如法炮制地将它写成
\[
    {H}_{\text{spin}} = \int \dd[3]{\vb*{r}} {\vb*{\mu}} \cdot \vb*{B}.
\]
因此完整的哈密顿量实际上是
\begin{equation}
    {H} = - \frac{1}{2m} \sum_i {\vb*{p}}_i^2 + {H}_\text{int} + \int \dd[3]{\vb*{r}} \varphi {\rho} - \int \dd[3]{\vb*{r}} {\vb*{J}} \cdot \vb*{A} + \int \dd[3]{\vb*{r}} {\vb*{\mu}} \cdot \vb*{B}.
    \label{eq:hamiltonian-with-eb}
\end{equation}
所有和单粒子携带的电荷数量有关的量全部被藏在电荷密度和电流密度中了,上式在电荷正反变换下不变。

\eqref{eq:hamiltonian-with-eb}当然也可以非常容易地写成二次量子化的形式。
薛定谔场满足$U(1)$对称性,因此通过诺特定理可以得到守恒荷(当然就是电荷)
\begin{equation}
    \rho = - e \sum_\sigma {\psi}^\dagger_\sigma {\psi}_\sigma = - e {n}_\text{e},
\end{equation}
守恒流(也即电流密度)
\begin{equation}
    {\vb*{j}} = - \frac{\ii e}{2m} \sum_\sigma ({\psi}_\sigma^\dagger(\vb*{r}) \grad{{\psi}_\sigma}(\vb*{r}) - (\grad{{\psi}_\sigma^\dagger}(\vb*{r})) {\psi}_\sigma(\vb*{r})) - \frac{e^2}{m} \vb*{A}(\vb*{r}) \sum_\sigma {\psi}^\dagger_\sigma(\vb*{r}) {\psi}_\sigma(\vb*{r}),
\end{equation}
而另一方面自旋磁矩为
\begin{equation}
    {\vb*{\mu}} = {\psi}^\dagger_\alpha \vb*{\sigma}_{\alpha \beta} {\psi}_\beta.
\end{equation}
这样就把三个对外加电磁场的线性响应全部写成二次量子化的形式了;我们可以用推迟格林函数计算出有关的响应大小。

由线性响应理论,我们有
\[
    \begin{aligned}
        \expval*{{J}_i}_A (t) &= \expval*{{J}_i}_0 + \ii \int \dd{t} \int \dd[3]{\vb*{r}'} \theta(t-t') \expval*{\comm*{{J}_i(\vb*{r}, t)}{{J}_j(\vb*{r}', t')}} A_j(\vb*{r}', t') \\
        &= - \frac{e^2}{m} \expval*{{n}_\text{e}} A_i + \ii \int \dd{t} \int \dd[3]{\vb*{r}'} \theta(t-t') \expval*{\comm*{{j}_i(\vb*{r}, t)}{{j}_j(\vb*{r}', t')}} A_j(\vb*{r}', t').
    \end{aligned}
\]
这里的$i, j$为维度脚标,并不表示粒子编号,且使用爱因斯坦求和约定;下标$A$表示有外场$\vb*{A}$时的期望值;${j}_i$的无外场期望是零,这是对称性的结果。

实际上,电阻率定义为%
\footnote{$\vb*{A}(\vb*{r}, t)$完全可以不是时间、空间平移不变的,但是既然我们将$\vb*{A}$当成扰动,只需要无扰动的系统的动力学时间、空间平移不变即可。}%
\begin{equation}
    J_i(\vb*{r}, t) = \int \dd[3]{\vb*{r}'} \int_{-\infty}^t \sigma_{ij}(\vb*{r}-\vb*{r}', t - t') E_j(\vb*{r}', t'),
\end{equation}
于是为了避免繁琐的时间上的微积分我们切换到频域上,有
% TODO:确认记号问题

现在我们回到二次量子化的框架下,考虑怎么计算有关的推迟格林函数。

在虚时间路径积分中不能简单地将\eqref{eq:hamiltonian-with-eb-original}(从而,\eqref{eq:hamiltonian-with-eb})做勒让德变换。
在这两个哈密顿量中有一个$U(1)$规范场$(\varphi, \vb*{A})$,由于规范对称性的存在,实际上只有$\vb*{A}$是独立的自由度,因此不能保证在Wick转动中含有$\varphi$的项是不变的。
最简单的从\eqref{eq:hamiltonian-with-eb-original}推导出虚时间路径积分的方法是最小耦合。我们知道$\varphi \rho$项会出现本质上是因为加入$U(1)$规范场之后需要将导数替换为协变导数,而可以验证以下替换
\begin{equation}
    \partial_\tau \longrightarrow \partial_\tau - \ii e \varphi, \quad - \ii \grad \longrightarrow - \ii \grad + e \vb*{A} 
    \label{eq:imaginary-em-covariant-derivative}
\end{equation}
给出了虚时间场论中的协变导数,于是在自由理论中做这个替换就得到了与电磁场发生相互作用的电子场的虚时间配分函数。

这样会带来一个疑难,就是Wick转动后电势前面多出来了一个负号;但是凝聚态理论中电子可以被放在一个势场中,显然Wick转动后势场前面不需要多出来一个负号。
这个疑难的解答是,含有$\ii e \varphi \rho$项的理论描述了一个电子场和一个电磁场的耦合,而“电子置于势场中”的模型中电磁场已经被积掉了。
如果将电磁场积掉,按照高斯积分的原理,$e \varphi \rho$项前面会多一个负号,并且要乘以系数$\ii$的平方,于是我们发现Wick转动后前面不需要多出来一个负号的势场出现了。

在以上的推导中,我们均取$e>0$为正的元电荷;在分析电子系统时,还有一种记号是取$e<0$,令$\abs*{e}$为元电荷。
这样可以让电子系统的配分函数看起来更像是直接从$U(1)$规范不变性得到的(即$\vb*{p}$替换成$\vb*{p} - e \vb*{A}$),从而看起来更加接近高能物理中的记号。
此处我们采用第一种记号,认为$e > 0$。

\section{晶格形成}

所谓晶体指的是一种在三个独立的空间方向上具有离散的平移不变性且并没有连续平移不变性的物体。\eqref{eq:many-body-hamiltonian}显然具有连续的平移不变性,因此晶体的形成必然经历了对称性自发破缺,且在较高的能量下原本的晶体一定会相变成某种更加均匀的东西。
本章将展示组成晶体的离子实是具体如何组成晶体的。原则上可以有哪些晶体见\autoref{chap:lattice-structure}。

将固体分解为组成它的组件所需的能量称为\concept{内聚能}。造成内聚的相互作用方式包括:
\begin{itemize}
    \item 离子键,离子之间的库伦引力。
    \item 共价键,相对局域的电子云的重叠导致的等效原子间吸引力。
    \item 金属键,离域电子组成大范围的电子气,导致等效的原子间吸引力。此时静电屏蔽非常强,从而正电荷可以看成均匀的“凝胶”。很直观地,晶格中原子排列的具体形式没有特别的要求。
    \item 范德华力,一种分子之间较弱的相互作用力,键能很低,大体上可以分为取向力、诱导力、色散力。
    \item 氢键,几乎裸露的氢原子和其它原子的吸引力,比范德华力强但比真正的化学键弱。
\end{itemize}
这些吸引相互作用不能让原子无限制地靠近。这包括两个原因。
首先,当然是因为原子核带正电,彼此之间存在排斥,正如氢分子的形成时,电子成键能够降低能量,而原子核靠近会增高能量。
然而对稍微大一些的原子,内层电子的屏蔽就很明显,此时阻止原子靠近的主要因素是电子距离足够近时的\concept{泡利排斥能}。
举一个极端的例子:设两个氦原子核距离非常接近,此时1s轨道上似乎应该有四个电子。
如果电子不遵循费米统计,多电子态就是单电子波函数的直积,那么的确电子-电子库伦排斥会增强,但是电子介导的原子核之间的吸引相互作用也会增强。
然而,由于费米统计,实际上两个电子要出现在高于1s轨道的轨道上,因此能量大大增大,并且没有什么能够弥补这种能量增大。
即使没有电子-电子库伦排斥,费米统计也会导致原子核接近时体系能量增大,这就是泡利排斥能。

可以使用\concept{分子轨道理论}近似地分析共价键:设有两个(可以不一样的)原子A和B。我们在两个原子周围各自取一个单原子电子波函数(即所谓原子轨道),然后将来自另一个原子的库伦吸引力当成微扰。
如果作为出发点的两个原子轨道能量差别不大,那么考虑了来自另一个原子的相互作用之后,将会出现显著的能级劈裂。
较低的那个能级就是\concept{成键轨道},而较高的那个能级就是\concept{反键轨道},它们是彼此正交的、原来两个原子轨道的线性组合。
最后,引入电子-电子相互作用,但是仍然假定多电子波函数近似为两个单电子波函数的直积(这就是\concept{分子轨道}一说的来历:我们假定系统中仍然有定义良好的单电子轨道,尽管此时这个轨道遍布整个分子)。
此时电子-电子相互作用导致的一阶能量修正包括一项密度-密度排斥力和一项交换相互作用。前者会让成键轨道上的电子相互排斥,然而无论怎么放置这两个电子,都会有互相排斥;后一项在成键轨道上有两个自旋相反的电子时却会让两个电子相互吸引。
如果总的电子-电子相互作用导致的修正没有大过成键轨道能量降低的量,那么一个共价键就形成了:因为电子放在成键轨道上能够降低总能量。
直观地看,此时成键轨道上的两个电子的电子云互相重叠,由于没有特别紧密地重叠,库伦排斥并不是特别大,而两个原子中间的这团电子云同时受到两边的原子的库伦吸引,因此就在两个原子之间建立了明显的等效吸引力。
反之,如果一开始的两个原子轨道能量相差很大,那么成键轨道和反键轨道和原来的两个原子轨道没有什么差别,电子放在成键轨道上不能降低多少能量,此时也不会有等效的强烈的原子-原子相互作用,也就没有共价键。

金属键弥散到整个金属晶体中,原子-原子间的等效吸引力没有特别明显的方向性之类,原子排列得越密集,库伦能越低。
这就是金属中密排结构特别常见的原因。
但是另外一方面,原子密排又会导致电子动量增大,以及泡利不相容原理导致的排斥。

\section{固体系统的表征}

表征一个固体系统通常可以使用的方法,或者说固体系统常见的可测量量(这里不是指可观察量算符,而是指真的做量子测量能够测出来的物理量,比如说期望,等等),包括导电性、磁性、对电磁波的响应、力学和热学性质。

\subsection{热学和固体力学}

热学性质——从而固体力学性质——经常可以使用自由粒子的模型来估计。
一种无相互作用的粒子的哈密顿量是完全对角化的:
\[
    H = \sum_{\vb*{k}, \sigma} \omega_{\vb*{k}\sigma} \left(n_{\vb*{k} \sigma} + \frac{1}{2}\right),
\]
于是对配分函数的贡献为
\begin{equation}
    Z = \sum_{E_i} \ee^{- \beta E_i} = \ee^{-\beta E_{\text{eq}}} \prod_{\vb*{k}, \sigma} \ee^{- \beta \frac{\omega_{\vb*{k}\sigma}}{2}} \sum_{n_{\vb*{k}, \sigma}} \ee^{-\beta \omega_{\vb*{k}\sigma} n_{\vb*{k} \sigma}},
\end{equation}
因此这种粒子对内能的贡献为
\begin{equation}
    U = E_{\text{eq}} + \sum_{\vb*{k}, \sigma} \left( \frac{1}{2} + \frac{1}{\ee^{\beta \omega_{\vb*{k}\sigma}} - 1} \right) \omega_{\vb*{k}\sigma}.
\end{equation}
这样,等容热容为
\begin{equation}
    C_V = \left(\pdv{U}{V}\right)_V = \sum_{\vb*{k}, \sigma} \left(\frac{\omega_{\vb*{k}\sigma}}{T}\right)^2 \frac{\ee^{\omega_{\vb*{k}\sigma} / T}}{(\ee^{\omega_{\vb*{k}\sigma} / T} - 1)^2}.
\end{equation}
每种粒子都会对热容有一个贡献,相互作用会修正这个贡献。

\subsection{电磁响应和输运}

如果一种粒子携带某种荷,那么还可以观察到\concept{输运}。大体上输运过程可以分成两种,一种是粒子平均自由程远小于系统尺寸的过程,此时系统内部足以出现荷的梯度,输运流量和这种梯度有关,可以称为\concept{扩散};另一种是粒子平均自由程大于系统尺寸的过程,此时系统内部的粒子基本上不受到任何阻碍,输运流量由系统边界上的性质确定,即所谓\concept{弹道输运}。

输运经常是依靠外加电磁场而产生的。
由于$\vb*{A}$耦合在$\vb*{j}$上,而能够决定极化电场的也无非是电子数密度,讨论电子气的电磁响应基本上就是要计算密度-密度关联函数。
如果不考虑电子之间的库仑相互作用,那么这是平凡的:诸如“谐振子受到电场策动,然后出现极化”的图像就能工作得很好。
在绝缘体中这大体上是正确的,在金属中则不见得是这样。
因此,比较精确地计算电子气的电磁响应,实际上就是需要将电子-电子库伦相互作用纳入计算,而这就很自然地要求我们使用一种完整的凝聚态场论。

\part{静态和准静态情况}

\chapter{静电学}

\section{静电系统的基本方程}

本节讨论仅含有导体和线性电介质的静电系统。所谓\concept{静电}指的是系统中没有任何电流的情况,此时我们有$\div{\vb*{j}}=0$,从而电荷密度分布没有变化。
注意我们没有使用“电荷密度没有变化”作为定义,因为恒稳电路也具有这样的性质,但是与此同时的确有电荷流动。
在静电的条件下我们有
\[
    \curl{\vb*{E}} = - \pdv{\vb*{B}}{t}, \quad \curl{\vb*{B}} = \mu_0 \epsilon_0 \pdv{\vb*{E}}{t},
\]
然后我们会发现磁场$\vb*{B}$满足一个波动方程。看起来这非常奇怪,因为在根本没有电流的时候磁场怎么会存在呢?
这实际上是来自边界条件的不清晰。在处理电磁波时我们并不要求无穷远处场强衰减至零,因为我们实际上是认为电磁波由非常远的地方的一个源产生的,没完没了地传播到其它地方,从而处理电磁波时“无穷远处”有源是完全可以的。
在静电学中我们要求电荷约束在一个有限的范围内,从而无穷远处场强快速衰减,那么磁场满足的波动方程如果要有非平凡解,只能取类似球面波的形式,但是这样一来$\div{\vb*{B}}$会不为零,和磁场无源的条件违背。
总之,在静电学情况下$\vb*{B}=0$。于是我们就有静电学方程
\begin{equation}
    \div{\vb*{E}} = \frac{\rho}{\epsilon_0}, \quad \curl{\vb*{E}} = 0,
    \label{eq:static-e-field}
\end{equation}
或者
\begin{equation}
    \vb*{E} = - \grad{\varphi}, \quad \laplacian{\varphi} = - \frac{\rho_0}{\epsilon_0}.
    \label{eq:static-phi-field}
\end{equation}
这是拉普拉斯方程,有现成的通解,即
\begin{equation}
    \varphi(\vb*{r}) = \int \dd[3]{\vb*{r}'} \frac{1}{4\pi \epsilon_0} \frac{\rho(\vb*{r}')}{\abs*{\vb*{r} - \vb*{r}'}}.
    \label{eq:from-q-to-phi}
\end{equation}

静电学中能量可以看成是电荷携带的而不是电场携带的。这是因为一个区域内的能量为
\[
    E = \int \dd[3]{\vb*{r}} \frac{1}{2} \epsilon_0 \vb*{E}^2 = \frac{1}{2} \epsilon_0 \int \dd[3]{\vb*{r}} (\grad{\varphi})^2,
\]
做分部积分,并使用无穷远处场强为零这一条件,得到
\begin{equation}
    E = - \frac{\epsilon_0}{2} \int \dd[3]{\vb*{r}} \varphi \laplacian{\varphi} = \int \dd[3]{\vb*{r}} \rho \varphi.
\end{equation}
因此在静电学的情况下能量可以认为是定域在电荷周围的。

总之,求解
\begin{equation}
    \varphi|_\text{surface} = \const,
\end{equation}
\begin{equation}
    \pdv{\varphi}{\vb*{n}} = - \frac{\sigma}{\epsilon}
\end{equation}

真空中或者均匀线性电介质中不能有电势极大值或者极小值,因为在这样的区域内$\varphi$是调和函数,而调和函数在它调和的区域内部不能有极大值、极小值。
物理上这很好理解,如极大值出现意味着从这一点向它周围的各个方向都有电场,因此这一点上应该有电荷,矛盾。

在计算静电系统中导体的受力时,不能简单地将无导体的空间内的电场外推到导体表面,然后使用$\vb*{f}=\sigma\vb*{E}$,因为导体表面电场是不连续的。
更加物理地看,这是因为导体表面实际上是非常复杂的一个系统:电场在微观层面快速衰减,表面上的电荷之间有相互作用力,数量级估计可以发现这些电荷之间的相互作用力和电荷受到的电场力是同阶的,因此简单的$\vb*{f}=\sigma\vb*{E}$会漏掉一部分作用力。
最为可靠的方法是使用麦克斯韦张量来计算,因为动量守恒是总是成立的,则在静电情况下动量流是连续的,所以直接计算导体外的麦克斯韦张量然后外推到导体表面即可。%
\footnote{
    这里还有一个可能的疑难:麦克斯韦张量计算的是电场对自由电荷的作用力,但是首先导体上的电荷并不是自由的,其次我们要计算的也是导体受到的作用力。
    但是,电磁场本身对导体并没有任何作用,而由受力平衡,导体对电荷施加的作用力应该和电场对电荷施加的作用力平衡,于是电场对电荷的作用力就传递给了导体。
}%

\section{唯一性定理}

唯一性定理成立的条件是$\vb*{D}$和$\vb*{E}$之间的关系应该是一一对应的。
反之,在两者之间的关系实际上不一一对应的时候,唯一性定理就被破坏了。例如,如果$\vb*{D}-\vb*{E}$关系实际上构成了一条电滞回线,那就没有唯一性定理。

% TODO: 有限大小的体系内电荷总量应该为零:这是高斯定理的推论

\[
    \int \dd[3]{\vb*{r}} (\varphi_2 \laplacian{\varphi_1} - \varphi_1 \laplacian{\varphi_2}) = \int \dd{\vb*{S}} (\varphi_2 \grad{\varphi_1} - \varphi_1 \grad{\varphi_2}),
\]
右边是
\begin{equation}
    \int \dd[3]{\vb*{r}} \varphi_1 \rho_2 = \int \dd[3]{\vb*{r}} \varphi_2 \rho_1.
\end{equation}
在导体系统中电荷仅仅分布在导体表面上,而且同一个导体表面电势处处相同,于是
\begin{equation}
    \sum_i \varphi_i^{(1)} q_i^{(2)} = \sum_i \varphi_i^{(2)} q_i^{(1)}.
\end{equation}

\section{电多极子}

设空间中的电荷密度为$\rho$,可能还要算上面密度,电势为
\[
    \varphi(\vb*{r}) = \frac{1}{4\pi \epsilon_0} \int \dd[3]{\vb*{r}'} \frac{1}{\abs*{\vb*{r} - \vb*{r}'}} \rho(\vb*{r}'),
\]
对$1/\abs*{\vb*{r}-\vb*{r}'}$做多极展开:
\[
    \frac{1}{\abs*{\vb*{r}-\vb*{r}'}} = \frac{1}{\abs*{\vb*{r}}} - \vb*{r}' \cdot \grad{\frac{1}{\abs*{\vb*{r}}}} + \frac{1}{2} \vb*{r}' \vb*{r}' : \grad{\grad{\frac{1}{\abs*{\vb*{r}}}}} + \cdots,
\]
就得到一个$\varphi$的展开式,即所谓\concept{多极展开},其中
\begin{equation}
    \varphi^{(0)}(\vb*{r}) = \frac{1}{4\pi \epsilon_0} \frac{1}{\abs*{\vb*{r}}} \underbrace{\int \dd[3]{\vb*{r}'} \rho(\vb*{r}')}_{Q}
\end{equation}
就是将整个体系当成一个点电荷计算得到的电势,
\begin{equation}
    \begin{aligned}
        \varphi^{(1)}(\vb*{r}) &= - \frac{1}{4\pi \epsilon_0} \grad{\frac{1}{\abs*{\vb*{r}}}} \cdot \int \dd[3]{\vb*{r}'} \rho(\vb*{r}') \vb*{r}' \\
        &= \frac{1}{4\pi \epsilon_0} \frac{\vb*{r}}{\abs*{\vb*{r}}^3} \cdot \underbrace{\int \dd[3]{\vb*{r}'} \rho(\vb*{r}') \vb*{r}'}_{\vb*{p}}
    \end{aligned}
\end{equation}
是电偶极电势,电四极矩是
\begin{equation}
    \varphi^{(2)}(\vb*{r}) = \frac{1}{4 \pi \epsilon_0} \frac{1}{6} \grad{\grad{\frac{1}{\abs*{\vb*{r}}}}} : \underbrace{3 \int \dd[3]{\vb*{r}'} \rho(\vb*{r}') \vb*{r}' \vb*{r}' }_{\vb*{D}}.
\end{equation}
在电荷分布相对于坐标系原点空间反演对称时,电偶极矩是零,而当电荷分布相对于坐标系原点空间反演反对称时,电四极矩是零。

容易看出,电四极矩$D_{ij}$是对称的,因此有6个独立分量。实际上这些独立分量并不都是有用的,注意到$\vb*{r} \neq 0$时
\[
    \laplacian{\frac{1}{\abs*{\vb*{r}}}} = 0,
\]
我们发现
\[
    \grad{\grad{\frac{1}{\abs*{\vb*{r}}}}} : \vb*{I} = 0,
\]
即我们可以任意地在$\vb*{D}$中加上单位张量的倍数,而不改变电势分布。因此我们可以手动加入一个约束:定义\concept{约化电四极矩}
\begin{equation}
    \tilde{\vb*{D}} = \vb*{D} - \frac{1}{3} \trace(\vb*{D}) \vb*{I} = \int \dd[3]{\vb*{r}'} (3 \vb*{r}' \vb*{r}' - \abs*{\vb*{r}'}^2 \vb*{I}) \rho(\vb*{r}') ,
\end{equation}
将$\vb*{D}$的迹消除掉,然后用$\tilde{\vb*{D}}$代替$\vb*{D}$同样可以得到正确的电四极矩;$\tilde{\vb*{D}}$独立的分量有5个,因为读多了一个无迹的条件。

电四极矩造成的电势衰减得比电偶极矩造成的电势快,电偶极矩造成的电势的衰减又比点电荷快。随着场点越来越接近源点,越来越复杂的电多极矩结构开始展现出来。

在$\vb*{r}$很大,即电场源离我们很远,其长度尺度趋于零时,我们需要将电场源替换成某种“点源”。我们不能朴素地将$\rho(\vb*{r})$替换成一个$\delta$函数,因为这样无法拿到电多极子的信息。正确的做法是
\begin{equation}
    \rho(\vb*{r}) = \delta(\vb*{r} - \vb*{r}_0) - \vb*{p} \cdot \grad \delta(\vb*{r} - \vb*{r}_0) + \cdots,
\end{equation}
即通过$\delta$函数的导数来引入“电场源的内部不均匀分布”的信息;通过分部积分法可以将导数转移到$1 / \abs*{\vb*{r}}$上,我们就能够拿到电多极子了。

\chapter{静磁学}

和静电学类似,我们可以考虑恒定电流的情况,即虽然有电流但是没有任何电荷变化,电流强度也不变的情况,则由输运方程有$\div{\vb*{j}}=0$。
我们可以直接引用\eqref{eq:wave-eq-general},得到
\[
    \frac{1}{c^2} \pdv[2]{\vb*{E}}{t} - \laplacian{\vb*{E}} = - \frac{1}{\epsilon_0} \grad{\rho} , \quad \frac{1}{c^2} \pdv[2]{\vb*{B}}{t} - \laplacian{\vb*{B}} = \mu_0 \curl{\vb*{j}},
\]
由于$\rho$和$\vb*{j}$都不随时间变化,如果我们像在静电学中一样,要求无穷远处场强衰减足够快,那么以上两式可以直接化为静态的拉普拉斯方程
\[
    \laplacian{\vb*{E}} = \frac{1}{\epsilon_0} \grad{\rho}, \quad \laplacian{\vb*{B}} = - \mu_0 \curl{\vb*{j}}.
\]
由于$\vb*{B}$不会变化,我们直接得到\eqref{eq:static-e-field},于是就可以求解出电场。
至于$\vb*{B}$,引入磁矢势,就得到
\[
    \curl{\laplacian{\vb*{B}}} = - \mu_0 \curl{\vb*{j}},
\]
那么只需要取库伦规范$\div{A}=0$就可以有
\[
    \div{\laplacian{\vb*{B}}} = - \mu_0 \div{\vb*{j}},
\]
于是就得到
\begin{equation}
    \vb*{B} = \curl{\vb*{A}}, \quad \laplacian{\vb*{A}} = - \mu_0 \vb*{j}.
\end{equation}
这个方程的形式和\eqref{eq:static-phi-field}非常相似,也是拉普拉斯方程,从而直接可以写出
\begin{equation}
    \vb*{A}(\vb*{r}) = \int \dd[3]{\vb*{r}'} \frac{\mu_0}{4\pi} \frac{\vb*{j}(\vb*{r}')}{\abs*{\vb*{r} - \vb*{r}'}}.
    \label{eq:from-j-to-a}
\end{equation}

总之,在电荷分布不变、电流分布不变的情况下,电场可以用$\rho$表示出来,并且是无旋场;磁场可以用$\vb*{j}$表示出来。在存在$\vb*{j}$的情况下,电荷分布不变,电场的形式和静电场完全一样,但磁场的存在会导致和静电学不同的一些物理现象,因此此时的电场可以称为\concept{恒定电场}。

静磁学指的是存在电荷流动,但是各个物理量的分布都恒稳的情况。要保持电流存在必须有一个外部的驱动力(\concept{非静电力}),这意味着此时的电磁场%
\footnote{
    当然,这个外部驱动力通常归根到底也是电磁力;但是我们将与它有关的那部分场自由度积掉了。
}%
不再是一个孤立体系。这可能让一些使用能量做的推导不再成立。%
\footnote{
    一种可能的诘难是,维持静电场的稳定也需要外部力(恩肖定理),为什么我们从来将静电场当成孤立系统看待?
    原因是,单纯从理论上说,要维持静电场稳定我们只需要将各个导体、电荷的动力学“关掉”即可(如认为点电荷受力不运动),等价的,维持静电场稳定的外力并不做功。
    另一方面,我们不能对电流做同样的事情:我们必须引入电流和电场之间的本构关系,从而自然地产生一个能量耗散项。
    静磁学理论中不可能不考虑这个能量耗散项。
}%

\begin{equation}
    \vb*{A}(\vb*{r}) = \frac{\mu_0}{4\pi} \frac{I \vb*{S} \times \vb*{r}}{\abs*{\vb*{r}}^3} = \frac{\mu_0}{4\pi} \frac{\vb*{m} \times \vb*{r}}{\abs*{\vb*{r}}^3}.
\end{equation}

\begin{equation}
    \vb*{B}(\vb*{r}) = - \frac{\mu_0}{4\pi} \left( \frac{\vb*{m} - 3 (\vb*{m} \cdot \vb*{e}_r) \vb*{e}_r}{r^3} \right).
\end{equation}

由于在边界上$\vb*{B}$有限大,应有
\begin{equation}
    \vb*{n} \times (\vb*{A}_2 - \vb*{A}_1) = 0.
\end{equation}
对库伦规范,

讨论静磁学系统的能量时需要把电源考虑进去,因为系统构型的小的变化会带来一个感生电动势,从而改变一些分布?

磁场的多极展开从$1$开始编号。

\chapter{似稳场和电路}

\section{基本方程}

\subsection{似稳条件}

本节开始我们讨论随时间发生变化的系统。
介质中的麦克斯韦方程看起来是时间平移不变的,当然,本构关系可以显含时间,因此它也许并不真的是时间平移不变的,但这种情况非常少见。
为方便起见我们经常求解\concept{时谐场},即假定$\vb*{E} \propto \ee^{- \ii \omega t}$,这相当于将场的时间部分切换到频域。
傅里叶变换意味着这当然不会丢失任何一般性。

当然,完整地解麦克斯韦方程是最精确的,但是很多情况下我们发现这类系统并没有特别明显的电磁辐射。
只要电场生磁场、磁场生电场,就可以有电磁辐射,因此电磁辐射不明显的系统中要么基本上没有电场产生的磁场,要么没有磁场产生的电场。 % TODO:,要么两者都有但是可以在感生电场和感生磁场之间建立直接的关系从而简化
这就是\concept{似稳场}或者\concept{准静态场}。体系中的电流被束缚在一些体积相对于电磁波波长不大的导体中的情况,也\concept{电路},经常可以用似稳场处理。
本节将主要讨论没有电场产生的磁场的情况,即忽略了位移电流的情况。
如果特殊需求,假定系统中的各个本构关系都是线性的。
此时麦克斯韦方程为
\begin{equation}
    \begin{bigcase}
        \div{\vb*{D}} &= \rho, \\
        \curl{\vb*{E}} &= - \pdv{\vb*{B}}{t}, \\
        \div{\vb*{B}} &= 0, \\
        \curl{\vb*{H}} &= \vb*{j}.
    \end{bigcase}
    \label{eq:quasi-stable-field}
\end{equation}

何时能够使用似稳场近似?对良导体,位移电流肯定要充分小,即
\[
    \pdv*{\vb*{D}}{t} \ll \sigma \vb*{E},
\]
即
\begin{equation}
    \omega \ll \omega_\sigma = \frac{\sigma}{\epsilon}.
    \label{eq:quasi-stable-field-cond-1}
\end{equation}
表面上看有这个条件就够了,但实际上这里有一个微妙的地方。在电场频率很大时,很多材料中电子在外场作用下不断“折返跑”,不会有宏观上的定向移动。此时的电流更像是束缚电流而不是传导电流,有
\[
    \curl{\vb*{H}} = \vb*{j} \sim \pdv{\vb*{E}}{t},
\]
从而又有了位移电流,因此电场实际的行为更像电磁波而不是似稳场。在频域上看,在$\omega$增大时,$\sigma$会有较大的、随着$\omega$变化的虚部,从而\eqref{eq:quasi-stable-field}的解和我们马上要看到的场的扩散方程(在其中$\sigma$就是一个常数)非常不同。
因此如果我们将直流电阻代入\eqref{eq:quasi-stable-field-cond-1}那这个判据太弱了。

对绝缘体,即在导电区域以外,显然只有$\omega=0$即完全静态的情况下才有\eqref{eq:quasi-stable-field-cond-1}成立。
然而,这仅仅意味着我们没有全空间的似稳场近似,并不意味着在系统的空间尺度较小时没有似稳场近似。
在绝缘体条件下,以$\epsilon$和$\mu$代替真空中的李纳-维谢尔势中的$\epsilon_0$和$\mu_0$,于是
\[
    \begin{aligned}
        \vb*{B}(\vb*{r}, t) &\approx \frac{\mu}{4\pi} \int \dd[3]{\vb*{r}'} \frac{\vb*{j}(\vb*{r}', t - R / c) \times \vb*{R}}{R^3} \\
        &= \frac{\mu}{4\pi} \int \dd[3]{\vb*{r}'} \frac{\vb*{j}(\vb*{r}', t) \ee^{- \ii \omega (t - R / c)} \times \vb*{R}}{R^3},
    \end{aligned}
\]
如果某一点的电流变化要瞬间传递到系统的各处,应有
\begin{equation}
    R \ll \frac{c}{\omega}.
\end{equation}
这当然是非常合理的:扰动基本上以光速传递,因此如果系统足够小,那么系统内一点的扰动总是可以快速传遍整个系统。

如果一个系统能够用似稳场分析,这通常意味着我们可以把系统看成某种电路:磁场可以写成电流的函数,电场可以分成两部分,一部分是磁场的变化率的函数,一部分是电荷的函数,而电流又正比于总电场,因此可以写出一个类似于“电流乘以电导率=由磁场变化导致的电场+电荷导致的电场+外加电场”这样的方程,这正好是电路的方程,其中考虑了四种效应:电阻、电感、电容、电源。
反之,则需要将系统看成某种传导电磁波的介质。

\subsection{场的扩散方程}

% TODO:似乎涉及电荷的重新分布等问题的情况不能用似稳场近似,因为在似稳场情况下电流散度为零

在系统中各处的本构关系都是空间均匀的情况下,经过大约为$\epsilon/\sigma$量级的时间,电荷密度为零(请注意这个结论和是否有外加场、外加场是否变化无关),因此大部分时候我们只需要求解
\[
    \begin{bigcase}
        \div{\vb*{D}} &= 0, \\
        \curl{\vb*{E}} &= - \mu \pdv{\vb*{H}}{t}, \\
        \div{\vb*{H}} &= 0, \\
        \curl{\vb*{H}} &= \sigma \vb*{E},
    \end{bigcase}
\]
从而
\begin{equation}
    \laplacian{\vb*{E}} = \mu \sigma \pdv{\vb*{E}}{t}, \quad \laplacian{\vb*{H}} = \mu \sigma \pdv{\vb*{H}}{t}.
\end{equation}
这就是\concept{场的扩散方程}。可以发现,对良好的导体,场的扩散反而是非常慢的,这是正确的,因为静电场中导体内部不应该有电场,因此在似稳场下导体内部的电场应该很弱,正好说明场的扩散很差。

在频域下,我们有
\begin{equation}
    \laplacian{\vb*{E}} = - \ii \omega \mu \sigma \vb*{E}, \quad \laplacian{\vb*{H}} = - \ii \omega \mu \sigma \vb*{H}.
    \label{eq:semi-stable-omega}
\end{equation}
如果$\sigma$有很大的虚部,以上方程的行为看起来就更像亥姆霍兹方程,从而在时域给出传递的波动。
在$\omega$很大时$\sigma$通常会有很大的虚部,因此此时似稳场不适用。

在似稳场确实适用的情况下,导体内部基本上没有场强分布,即出现\concept{趋肤效应}。
这可以通过在导体表面求解\eqref{eq:semi-stable-omega}看出。在一个无穷大平面边界上,设
\begin{equation}
    \vb*{E} = \vb*{E}_0 \ee^{-\alpha z},
    \label{eq:damping-surface-field}
\end{equation}
% TODO
趋肤深度为
\begin{equation}
    \delta = \sqrt{\frac{2}{\mu \omega \sigma}}.
\end{equation}
对理想导体,$\sigma \to \infty$,因此任何频率下电场都不会进入导体内部。
对实际导体,频率越高,趋肤效应越明显,但是当$\omega$继续增大以至于似稳场不再适用时,趋肤效应就消失了,此时的导体是透明的。

这里有一个看起来的佯谬:对有限大小的电导率,$\omega$很小时似乎有$\delta \to \infty$,也即,静电场可以直接穿透导体!
但是应当注意到一点:似稳场近似不仅包括了静电场的那些模式,也包括了\concept{恒定电场}即导体有稳定的、不随时间变化的电流输入的那些模式。
后者的确不存在任何趋肤效应:很容易验证,$\vb*{E}$在导体内处处均匀分布,指向同一个方向的模式是存在的,这里没有任何趋肤效应。
实际上从麦克斯韦方程可以看出,要产生场的扩散方程,感生电场是必须的(注意$\mu$出现在了场的扩散方程中),因此趋肤效应实际上是因为导体内部感应出的涡旋电场抵消了外加电场。这个机制和静电屏蔽是不一样的。

静电屏蔽实际上来自边界条件。设电场从绝缘体被打到导体上,用1标记绝缘体,2标记导体,有
\[
    \vb*{n} \cdot (\vb*{D}_2 - \vb*{D}_1) = \sigma, \quad \pdv{\rho}{t} = \vb*{n} \cdot \vb*{j},
\]
在频域下就有
\[
    \vb*{n} \cdot (\vb*{D}_2 - \vb*{D}_1) = \sigma, \quad - \ii \omega \sigma = \vb*{n} \cdot \vb*{j},
\]
于是
\[
    \vb*{n} \cdot \vb*{E}_2 = \frac{\vb*{n} \cdot \vb*{D}_1}{\epsilon-0 + \frac{\sigma}{\ii \omega}},
\]
在$\omega \to 0$时$\vb*{n} \cdot \vb*{E}_2$趋于零,即使趋肤深度很大,电场也不能进入导体。因此与静电屏蔽相关的电场衰减的尺度是微观尺度上的:在导体和绝缘体的交接层上电场已经衰减了;另一方面,趋肤效应的尺度虽然很小,但仍然是宏观的,它发生在导体体块内部,而不是导体和绝缘体的交接层上。
我们在这里只讨论了垂直于表面的电场,因为平行于表面的电场不会出现在静电场中,但是可以出现在恒定电场中;如果导体表面附近有平行于表面的电场,那么由$\vb*{n} \times (\vb*{E}_2 - \vb*{E}_1)=0$导体内部肯定也有平行于表面的电场,因此会有电流,就和“静电场”的条件冲突了。
总之,无论外界的静电场如何分布,导体内部根本不会有静电场,因此导体内部$\omega=0$的电场全部都是恒定电场。

还有另一个可能造成疑难的地方:我们知道导体可以远距离传输电流,从而必然可以远距离传输电场,但是上面的论证似乎是说,导体中的电场一定会快速衰减!
我们来分析一下远距离传输的电场可能出现在哪里,也即,持续的、远距离的稳定电流可能出现在哪里。
趋肤效应的推导是非常一般的,因此这样的电流只能出现在导体边界附近,并且在没有外加电流流入的地方一定平行于导体边界。
这些电流一定最后会撞上另一个导体边界,因为电流不可能在缺乏非静电力的导体内部环流。
因此后一个导体边界的边界条件中,电流主要分布在这个边界的边缘上,如对柱状导体,主要分布在柱子的上下底面的棱上。
这样的场构型让\eqref{eq:damping-surface-field}在这些表面上失效,因为此时电磁场在$z$方向和$x, y$方向上都有很大变化。
在静电学中只有静电力,我们不会产生源源不断的电流,也不会将这样的电流从某处导入导体中,因此从来不会激发这样的模式。
在导体和绝缘体交界的界面上根本无所谓电流输入,同样不会激发这样的模式。

\chapter{辐射}\label{chap:radiation}

我们已经讨论了电磁波的传播,本节则讨论电磁波是怎么产生的。

\section{单粒子的辐射}

\subsection{李纳-维谢尔势}

本节求解空间中有单个运动电荷时的电势和矢势分布情况,所得结果称为\concept{李纳-维谢尔势}。
实际上这就是在求解真空中电磁场的格林函数,但是只有在讨论辐射时这才有意义,因为只有此时我们需要确切地知道“电荷运动方式给定后电磁场的分布”。
其它时候,或是根本不需要让电荷动起来(静电学),或是电荷运动恒稳,磁场可以直接计算出来(静磁学),或是电荷的存在根本可以归入有效介电常数中(电磁波的传播)。
当然由于李纳-维谢尔势就是麦克斯韦方程的通解,对它做近似和多麦克斯韦方程做近似是一样的,所以我们其实也可以通过将李纳-维谢尔势做不同的近似,恢复出电磁波传播、似稳场近似、静电学等。

在取洛伦兹规范之后,我们需要求解
\begin{equation}
    \begin{bigcase}
        \laplacian{\varphi} - \frac{1}{c^2} \pdv[2]{\varphi}{t} &= - \frac{\rho(\vb*{r})}{\epsilon_0}, \\
        \laplacian{\vb*{A}} - \frac{1}{c^2} \pdv[2]{\vb*{A}}{t} &= - \mu_0 \vb*{j}(\vb*{r}).
    \end{bigcase}
\end{equation}
使用格林函数法,有(暂时先不引入无穷小虚部)
\[
    \begin{aligned}
        \varphi(\vb*{r}, t) &= \int \frac{\dd{\omega}}{2\pi} \int \frac{\dd[3]{\vb*{k}}}{(2\pi)^3} \frac{\ee^{\ii (\vb*{k} \cdot \vb*{r} - \omega t)}}{- \vb*{k}^2 + \omega^2/c^2} \left( - \frac{\rho(\vb*{k}, \omega)}{\epsilon_0} \right) \\
        &= \frac{1}{\epsilon_0} \int \frac{\dd{\omega}}{2\pi} \int \frac{\dd[3]{\vb*{k}}}{(2\pi)^3} \frac{\ee^{\ii (\vb*{k} \cdot \vb*{r} - \omega t)}}{\vb*{k}^2 - \omega^2/c^2} \int \dd[3]{\vb*{r}'} \int \dd{t'} \ee^{\ii (\omega t' - \vb*{k} \cdot \vb*{r}')} \rho(\vb*{r}', t') \\
        &= \frac{1}{\epsilon_0} \int \dd[3]{\vb*{r}'} \int \dd{t'} \rho(\vb*{r}', t') \int \frac{\dd{\omega}}{2\pi} \int \frac{\dd[3]{\vb*{k}}}{(2\pi)^3} \frac{\ee^{\ii (\vb*{k} \cdot (\vb*{r} - \vb*{r}') - \omega (t - t'))}}{\vb*{k}^2 - \omega^2/c^2}.
    \end{aligned}
\]
首先计算$\vb*{k}$部分的积分,有
\[
    \begin{aligned}
        \int \dd[3]{\vb*{k}} \frac{\ee^{\ii \vb*{k} \cdot \vb*{R}}}{\vb*{k}^2 - \omega^2/c^2} &= \int k^2 \sin \theta \dd{k} \dd{\theta} \dd{\varphi} \frac{\ee^{\ii k R \cos \theta}}{k^2 - \omega^2 / c^2} \\
        &= 2\pi \int_0^\infty \frac{k^2 \dd{k}}{k^2 - \omega^2 / c^2} \frac{1}{\ii k R} (\ee^{\ii k R} - \ee^{-\ii k R}) \\
        &= \frac{\pi}{\ii R} \int_{-\infty}^\infty \frac{k \dd{k}}{k^2 - \omega^2 / c^2} (\ee^{\ii k R} - \ee^{-\ii k R}) \\
        &= \frac{\pi}{2 \ii R} \int_{-\infty}^\infty \dd{k} \left( \frac{1}{k + \omega / c} + \frac{1}{k - \omega / c} \right) (\ee^{\ii k R} - \ee^{-\ii k R}).
    \end{aligned}
\]
此时必须在分母上加入无穷小虚部。按照关于$\omega$的零点必须在下半平面以保证因果性的原则,我们将$\omega$替换为$\omega + \ii 0^+$,并使用留数定理(注意$\ee^{\ii k R}$项应取上半平面极点而$\ee^{- \ii k R}$项应取下半平面极点)就得到
\[
    \int \dd[3]{\vb*{k}} \frac{\ee^{\ii \vb*{k} \cdot \vb*{R}}}{\vb*{k}^2 - \omega^2/c^2} = \frac{2 \pi^2}{R} \ee^{\ii \omega R / c}, 
\]
于是
\begin{equation}
    \begin{aligned}
        \varphi(\vb*{r}, t) &= \int \dd[3]{\vb*{r}'} \int \dd{t'} \int \frac{\dd{\omega}}{2\pi} \ee^{-\ii \omega (t-t')} \rho(\vb*{r}', t') \frac{1}{4\pi \epsilon_0} \frac{\ee^{\ii \omega R / c}}{R} \\
        &= \int \dd[3]{\vb*{r}'} \int \frac{\dd{\omega}}{2\pi} \ee^{- \ii \omega t} \rho(\vb*{r}', \omega) \frac{1}{4\pi \epsilon_0} \frac{\ee^{\ii \omega R / c}}{R}.
    \end{aligned}
\end{equation}
这个结果展示了一个出射波:从$\rho(\vb*{r}', t')$出发的向外传播的球面波,而不是向内聚集的波。
现在我们再做关于$\omega$的积分,会直接得到一个$\delta$函数:
\[
    \begin{aligned}
        \varphi(\vb*{r}, t) &= \int \dd[3]{\vb*{r}'} \int \dd{t'} \int \frac{\dd{\omega}}{2\pi} \ee^{-\ii \omega (t-t')} \rho(\vb*{r}', t') \frac{1}{4\pi \epsilon_0} \frac{\ee^{\ii \omega R / c}}{R} \\
        &= \int \dd[3]{\vb*{r}'} \int \dd{t'} \delta(R/c + t' - t) \rho(\vb*{r}', t') \frac{1}{4\pi \epsilon_0 R} \\
        &= \int \dd[3]{\vb*{r}'} \frac{1}{4\pi \epsilon_0} \frac{\rho(\vb*{r}', t - R / c)}{R}.
    \end{aligned}
\]
同样的操作也可以对$\vb*{A}$和$\vb*{j}$做,最终得到
\begin{equation}
    \begin{bigcase}
        \varphi(\vb*{r}, t) &= \int \dd[3]{\vb*{r}'} \frac{1}{4\pi \epsilon_0} \frac{\rho(\vb*{r}', t - R / c)}{R}, \\
        \vb*{A}(\vb*{r}, t) &= \int \dd[3]{\vb*{r}'} \frac{\mu_0}{4\pi} \frac{\vb*{j}(\vb*{r}', t - R / c)}{R}.
    \end{bigcase}
    \label{eq:general-solution-wave}
\end{equation}
标势的形式和静电场一致,矢势的形式和静磁场一致,只不过出现了一个时间推迟。
我们经常把这样有时间推迟的量放在中括号里,即
\[
    \rho(\vb*{r}, t) = \int \dd[3]{\vb*{r}'} \frac{1}{4\pi \epsilon_0} \frac{[\rho]}{R},
\]
等等。

当空间中只有一个电荷时,有
\[
    \rho(\vb*{r}, t) = q \delta(\vb*{r} - \vb*{r}_0(t)), \quad \vb*{j}(\vb*{r}, t) = q \dot{\vb*{r}}_0(t) \delta(\vb*{r} - \vb*{r}_0(t)),
\]
其中$\vb*{r}_0 = \vb*{r}_0(t)$是该电荷的运动轨迹。代入\eqref{eq:general-solution-wave},有
\[
    \varphi(\vb*{r}, t) = \int \dd[3]{\vb*{r}'} \frac{1}{4\pi \epsilon_0} \frac{q \delta(\vb*{r}' - \vb*{r}_0(t - R / c))}{R},
\]
因此只有满足
\begin{equation}
    \vb*{r}' = \vb*{r}_0(t - R/c)
    \label{eq:retarded-position-original}
\end{equation}
的部分才有贡献。但是要注意,$\vb*{r}'$同时也出现在$R$中,因此积分时不能仅仅将$\vb*{r}'$替换为$\vb*{r}_0(t-R/c)$,还需要做一个积分测度的变换。
我们有
\[
    \grad_{\vb*{r}'} {(\vb*{r}' - \vb*{r}_0(t - R/c))} = \vb*{I} - \frac{\vb*{R}}{cR} \dot{\vb*{r}_0}(t-R/c) ,
\]
于是
\[
    \det(\grad_{\vb*{r}'} {(\vb*{r}' - \vb*{r}_0(t - R/c))}) = 1 - \frac{\vb*{R}}{cR} \cdot \dot{\vb*{r}}_0(t-R/c),
\]
从而
\[
    \begin{aligned}
        \varphi(\vb*{r}, t) &= \int \dd[3]{\vb*{r}'} \frac{1}{4\pi \epsilon_0} \frac{q \delta(\vb*{r}' - \vb*{r}_0(t - R / c))}{R} \\
        &= \frac{1}{4\pi \epsilon_0} \eval{\frac{1}{\det(\grad_{\vb*{r}'} {(\vb*{r}' - \vb*{r}_0(t - R/c))})} \frac{q}{R}}_{\vb*{r}' = \vb*{r}_0(t - R/c)} \\
        &= \frac{1}{4\pi \epsilon_0} \eval{\frac{q}{R - \frac{\vb*{R} \cdot \dot{\vb*{r}}_0(t-R/c)}{c}}}_{\vb*{r}' = \vb*{r}_0(t - R/c)}.
    \end{aligned}
\]
用$\vb*{v}$表示粒子的运动速度,就有
\begin{equation}
    \varphi(\vb*{r}, t) = \frac{1}{4\pi \epsilon_0} \frac{q}{R' - \frac{\vb*{R}' \cdot \vb*{v}'}{c}},
    \label{eq:retarded-phi}
\end{equation}
类似的
\begin{equation}
    \vb*{A}(\vb*{r}, t) = \frac{\mu_0}{4\pi} \frac{q \vb*{v}'}{R' - \frac{\vb*{R}' \cdot \vb*{v}'}{c}},
    \label{eq:retarded-a}
\end{equation}
其中$R'$和$\vb*{v}'$均为$t'$时刻的$R$和$\vb*{v}$而$t'$由
\begin{equation}
    R(t') = \abs*{\vb*{r} - \vb*{r}_0(t')} = c(t-t')
    \label{eq:retarded-time}
\end{equation}
确定。这个方程看起来非常合理,我们将$\rho$有速度地出现在某个地方当成一个事件,它传递到$\vb*{r}$必然存在时间延迟,事件传播的速度就是光速,在$t$时刻,$\vb*{r}$点看到的$\vb*{r}_0$处的情况是$t'$时刻的,两者之差为
\[
    t - t' = \frac{\abs*{\vb*{r} - \vb*{r}_0(t')}}{c},
\]
就得到\eqref{eq:retarded-time}。

前面$\delta$函数的积分改变了积分测度,让它比通常的要大一些。这看起来似乎有些奇怪,因为狭义相对论中似乎应该有尺缩效应,积分测度应该缩小。
这里的关键在于运动电荷对某一点的电场的贡献涉及的空间积分应该体现的是“在这一点看到的运动物体的长度”(在静止参考系看到的物体两端传来的信号可能来自不同时刻)而不是“试图在静止参考系中测量得到的运动物体的长度”(测量时物体两端到达观察点的用时是一样的)。

\subsection{辐射的多极展开}

\eqref{eq:general-solution-wave}可以做多极展开,所得结果和静电场、静磁场完全一致,仅有的区别在于$\varphi$和$\vb*{j}$是推迟的。

\subsubsection{电偶极辐射}

设系统中的电荷分布主要体现为偶极子,对$\varphi$,零阶项为
\[
    \varphi^{(0)}(\vb*{r}, t) = \frac{1}{4\pi \epsilon_0} \frac{1}{\abs*{\vb*{r}}} \int \dd[3]{\vb*{r}'} \rho(\vb*{r}', t - R / c) = \frac{1}{4\pi \epsilon_0} \frac{Q}{r},
\]
不随时间变化,没有辐射。一阶项
\[
    \begin{aligned}
        \varphi^{(1)}(\vb*{r}, t) &= - \frac{1}{4\pi \epsilon_0} \grad{\frac{1}{\abs*{\vb*{r}}}} \cdot \int \dd[3]{\vb*{r}'} \rho(\vb*{r}') \vb*{r}' \\
        &= - \frac{1}{4\pi \epsilon_0} \div{\int \dd[3]{\vb*{r}'} \frac{\rho(\vb*{r}', t-R/c) \vb*{r}'}{\abs*{\vb*{r}}}},
    \end{aligned}
\]
设$\vb*{p}$为总偶极矩,则
\begin{equation}
    \varphi^{(1)}(\vb*{r}, t) = - \div{\frac{[\vb*{p}]}{4\pi \epsilon_0 r}}. 
\end{equation}
对磁矢势,有
\[
    \begin{aligned}
        \vb*{A}^{(1)}(\vb*{r}, t) &= \frac{\mu_0}{4\pi} \int \dd[3]{\vb*{r}'} \frac{\vb*{j}(\vb*{r}', t - R/c)}{\abs*{\vb*{r}}} \\
        &= \frac{\mu_0}{4\pi} \int \dd[3]{\vb*{r}'} \frac{[\rho][\vb*{v}]}{\abs*{\vb*{r}}} \\
        &= \frac{\mu_0}{4\pi} \dv{(t-R/c)} \int \dd[3]{\vb*{r}'} \frac{[\rho][\vb*{r}']}{\abs*{\vb*{r}}}.
    \end{aligned}
\]
最后一个等号需要解释一下。我们可以将电荷分布离散化,从而
\[
    \int \dd[3]{\vb*{r}'} \rho(\vb*{r}') \vb*{v}(\vb*{r}') = \sum_i e \vb*{v}_i = \dv{t} \sum_i e \vb*{r}_i = \dv{t} \int \dd[3]{\vb*{r}'} \rho(\vb*{r}') \vb*{r}',
\]
然后将$t$换成$t-R/c$即可。实际上,设$\phi$为某个守恒量的密度,那么一定有
\[
    \dv{t} \int \dd[3]{\vb*{r}} \phi \psi = \int \dd[3]{\vb*{r}} \phi \dv{\psi}{t},
\]
这个方程可以直接从连续性方程推得。于是我们有
\begin{equation}
    \vb*{A}^{(1)}(\vb*{r}, t) = \frac{\mu_0}{4\pi} \frac{[\dot{\vb*{p}}]}{r}.
\end{equation}
这里$[\dot{f}]$表示先让$f$对$t$求导,然后用$t-R/c$代替$t$,或者说让$f(t-R/c)$对$t-R/c$求导。

现在我们还是可以一如既往地开始讨论时谐场,此时只需要认为电偶极子在做周期性振动即可。
这种振动当然会消耗能量,但是我们暂时先假定有某些外加能量输入让电偶极子持续振荡。

先计算出磁场,然后计算出电场比较方便。
\begin{equation}
    \vb*{B} = \frac{\mu_0 \omega^2}{4\pi c r} \vb*{e}_r \times [\vb*{p}].
\end{equation}

\section{电路的辐射}


\part{线性介质中光的传播}

\chapter{线性介质与经典麦克斯韦方程}

\section{线性介质假设到光学方程}

\subsection{单色波解}

现在我们认为不存在自由电流和自由电荷,从而\eqref{eq:maxwell-material}成为
\begin{equation}
    \begin{bigcase}
        \div{\vb*{D}} &= 0, \\
        \curl{\vb*{E}} &= - \pdv{\vb*{B}}{t}, \\
        \div{\vb*{B}} &= 0, \\
        \curl{\vb*{H}} &= \pdv{\vb*{D}}{t}
    \end{bigcase}
    \label{eq:no-free-charge}
\end{equation}
并且假定所有本构关系都是线性的(这里由于自由电荷被认为是零,不再需要考虑关于$\vb*{j}_\text{f}$的本构关系),
且$\vb*{D}$只和$\vb*{E}$有关,$\vb*{H}$只和$\vb*{B}$有关。
这样就可以写出一般的本构关系
\[
    \begin{aligned}
        \vb*{D}(\vb*{r}, t) &= \int K_E(\vb*{r} - \vb*{r}', t - t') \vb*{E}(\vb*{r}', t') \dd^3 \vb*{r}' \dd t', \\
        \vb*{H}(\vb*{r}, t) &= \int K_B(\vb*{r} - \vb*{r}', t - t') \vb*{B}(\vb*{r}', t') \dd^3 \vb*{r}' \dd t'
    \end{aligned}
\]
其中的$K_E$和$K_B$都是张量。我们再要求本构关系是局部的,则应有%
\footnote{在本节中$K_E$和$K_B$代表不同的常(张)量,其值在不同地方可能不同}
\[
    \begin{aligned}
        \vb*{D}(\vb*{r}, t) &= \int K_E(\vb*{r}, t-t') \vb*{E}(\vb*{r}, t') \dd t', \\
        \vb*{H}(\vb*{r}, t) &= \int K_B(\vb*{r}, t-t') \vb*{B}(\vb*{r}, t') \dd t'
    \end{aligned}
\]
这是一个时间上的卷积运算。由傅里叶变换,我们只需要讨论平面波的本构关系就可以了,因为其它所有形式的场都能够写成平面波的叠加,既然\eqref{eq:no-free-charge}是线性齐次的(这就是假定没有自由电荷的好处!)。于是,不失一般性的,假设所有的场都取
\[
    \vb*{A}(\vb*{r}) \ee^{-\ii \omega t}
\]
的形式,其中$\vb*{A}(\vb*{r})$可以有虚部,但是一定能够写成某个实矢量乘以$\ee^{\ii \phi}$的形式%
\footnote{在这个条件下,可能的$\vb*{A}$并不能覆盖整个$\complexes^3$,但附加这个条件并不会失去一般性,因为允许$\vb*{A}$有虚部是为了表示$\vb*{A}_0 \cos (\omega t + \phi)$这样的函数,而这一点在限制$\vb*{A}(\vb*{r})$取某个实矢量乘以$\ee^{\ii \phi}$的形式时足以满足。这是三维下的实函数的傅里叶变换的一个特点。}
,则只需要形如
\begin{equation}
    \vb*{D}(\vb*{r}) = K_E(\vb*{r}, \omega) \vb*{E}(\vb*{r}), \quad \vb*{H}(\vb*{r}) = K_B(\vb*{r}, \omega) \vb*{B}(\vb*{r})
    \label{eq:linear-constitutive}
\end{equation}
的本构关系就可以了。\eqref{eq:linear-constitutive}中的所有变量可以取它们完整的表达式,也可以取它们的振幅。注意这些量可以是复数(用以表示相位)。
此时的麦克斯韦方程组成为
\begin{equation}
    \begin{bigcase}
        \div{\vb*{D}} &= 0, \\
        \curl{\vb*{E}} &= \ii \omega \vb*{B}, \\
        \div{\vb*{B}} &= 0, \\
        \curl{\vb*{H}} &= - \ii \omega \vb*{D}
    \end{bigcase}
    \label{eq:sin-wave-eqs}
\end{equation}

现在转而讨论有自由电流和自由电荷的情况。我们认为自由电流正比于电场,即服从\concept{欧姆定律}。
同样不失一般性地设所有的场都可以写成一个复振幅乘以$\ee^{- \ii \omega t}$的形式。
使用导出$\vb*{D}$和$\vb*{E}$的关系、$\vb*{H}$和$\vb*{B}$的关系同样的方法,有
\begin{equation}
    \vb*{j}_\text{f}(\vb*{r}) = \sigma(\vb*{r}, \omega) \vb*{E}(\vb*{r})
    \label{eq:ohm-law-sin-wave}
\end{equation}
那么从\eqref{eq:maxwell-material}和\eqref{eq:transportation}可以得到
\[
    \begin{bigcase}
        \div{\vb*{D}} &= \rho_\text{f}, \\
        \curl{\vb*{E}} &= \ii \omega \vb*{B}, \\
        \div{\vb*{B}} &= 0, \\
        \curl{\vb*{H}} &= \vb*{j}_\text{f} - \ii \omega \vb*{D}, \\
        - \ii \omega \rho_\text{f} + \div{\vb*{j}_\text{f}} &= 0,
    \end{bigcase}
\]
我们希望能够消去$\rho_\text{f}$和$\vb*{j}_\text{f}$,为此反复使用输运方程得到
\[
    \begin{bigcase}
        \div{\left( \vb*{D} - \frac{\vb*{j}_\text{f}}{\ii \omega} \right)   } &= 0, \\
        \curl{\vb*{E}} &= \ii \omega \vb*{B}, \\
        \curl{\vb*{H}} &=  - \ii \omega \left( \vb*{D} - \frac{\vb*{j}_\text{f}}{\ii \omega} \right)
    \end{bigcase}
\]
代入本构关系\eqref{eq:ohm-law-sin-wave}得到
\[
    \begin{bigcase}
        \div{\left( \vb*{D} - \frac{\sigma \vb*{E}}{\ii \omega} \right)   } &= 0, \\
        \curl{\vb*{E}} &= \ii \omega \vb*{B}, \\
        \curl{\vb*{H}} &=  - \ii \omega \left( \vb*{D} - \frac{\sigma \vb*{E}}{\ii \omega} \right)
    \end{bigcase}
\]
注意到$\vb*{D}$和$\vb*{E}$、$\vb*{H}$和$\vb*{B}$之间的关系(见\eqref{eq:linear-constitutive}),做下面的替换
\[
    \vb*{D} - \frac{\sigma \vb*{E}}{\ii \omega} \longrightarrow \vb*{D}, \quad K_E - \frac{\sigma}{\ii \omega} \longrightarrow K_E
\]
得到的方程和\eqref{eq:sin-wave-eqs}形式完全一致,本构关系的形式也还是\eqref{eq:linear-constitutive}。
唯一的不同是,$K_E$和$K_B$都是复数,并且
\begin{equation}
    \Im K_E = \frac{\sigma}{\omega}
\end{equation}

\subsection{各向同性介质中的亥姆霍兹方程}

我们特别感兴趣的是\eqref{eq:linear-constitutive}中$K_E$和$K_B$都是标量(不一定是实数)的情况,
也就是说,\eqref{eq:linear-constitutive}中给出的响应是\concept{各向同性}的。
此时有
\begin{equation}
    \vb*{D}(\vb*{r}) = \epsilon(\vb*{r}, \omega) \vb*{E}(\vb*{r}), \quad \vb*{H}(\vb*{r}) = \frac{1}{\mu(\vb*{r}, \omega)} \vb*{B}(\vb*{r})
    \label{eq:scalar-constitutive}
\end{equation}
第二个公式的比例系数特意被放到了分母中,以达到和\eqref{eq:original-maxwell}相同的形式。
将\eqref{eq:scalar-constitutive}代入\eqref{eq:sin-wave-eqs}中,得到
\begin{equation}
    \begin{bigcase}
        \div{(\epsilon \vb*{E})} &= 0, \\
        \curl{\vb*{E}} &= \ii \omega \vb*{B}, \\
        \div{\vb*{B}} &= 0, \\
        \curl{\left(\frac{\vb*{B}}{\mu}\right)} &= - \ii \omega \epsilon \vb*{E}
    \end{bigcase}
    \label{eq:scalar-cons-maxwell-e-and-b}
\end{equation}
上式中第三式可以通过第二式推导而来;将第二式表示出的$\vb*{B}$代入第四式可以消去$\vb*{B}$。于是将\eqref{eq:scalar-cons-maxwell-e-and-b}简化为
\begin{equation}
    \begin{bigcase}
        \div{(\epsilon \vb*{E})} &= 0, \\
        \curl{ \left(\frac{1}{\mu} \curl{\vb*{E}} \right) } &= \omega^2 \epsilon \vb*{E}, \\
        \curl{\vb*{E}} &= \ii \omega \vb*{B}
    \end{bigcase}
    \label{eq:e-in-material}
\end{equation}
这个方程关于电场的部分等价于
\begin{equation}
    \begin{bigcase}
        \div{(\epsilon \vb*{E})} &= 0, \\
        \laplacian \vb*{E} + \mu \epsilon \omega^2 \vb*{E} &= - \frac{1}{\epsilon} \grad{(\vb*{E} \cdot \grad{\epsilon})} - \frac{1}{\mu} \grad{\mu} \cross (\curl{\vb*{E}})
    \end{bigcase}
    \label{eq:e-in-material-e-only}
\end{equation}
以上方程是基于$\vb*{E}$的,也可以写出基于$\vb*{H}$的类似的方程,遵从同样的步骤,可以得到
\begin{equation}
    \begin{bigcase}
        \div{\mu \vb*{H}} &= 0, \\
        \curl{\left( \frac{1}{\epsilon} \curl{\vb*{H}} \right)} &=  \mu \omega^2 \vb*{H}, \\
        \curl{\vb*{E}} &= \ii \omega \mu \vb*{H}.
    \end{bigcase}
    \label{eq:h-in-material}
\end{equation}
\eqref{eq:e-in-material}或\eqref{eq:h-in-material}称为\concept{主方程}。
\eqref{eq:e-in-material-e-only}的第二条方程有三组解,两组横波解,一组纵波解。
由于我们讨论的介质中没有自由电荷和自由电流,纵波解是不应该出现的,它被\eqref{eq:e-in-material-e-only}的第一条方程禁止。
纵波解是必然要出现的,因为涉及$\vb*{E}$的单个波动方程有三个偏振方向;横波条件需要另一个方程引入。

特别的,在$\epsilon, \mu$在空间中处处相等或者变化得比较缓慢时,有
\begin{equation}
    \laplacian \vb*{E} + \frac{\omega^2}{c^2} \vb*{E} = 0, \quad c^2 = \frac{1}{\epsilon \mu}
    \label{eq:halmholtz-eq}
\end{equation}
\eqref{eq:halmholtz-eq}就是所谓的\concept{亥姆霍兹方程},求解出它就求解出了整个\eqref{eq:e-in-material}。
需要注意的是并不是所有\eqref{eq:halmholtz-eq}的解都是\eqref{eq:e-in-material}的解,因为\eqref{eq:halmholtz-eq}没有包含\eqref{eq:e-in-material}的第一式。当然这个信息可以在求解\eqref{eq:halmholtz-eq}的时候使用一些边界条件加上去。
这里占用了$c$表示介质中光速,相对应地,设真空中光速为$c_0$,
其值为$1/\sqrt{\epsilon_0 \mu_0}$,因为简单地令$\epsilon=\epsilon_0$,$\mu = \mu_0$,
\eqref{eq:scalar-cons-maxwell-e-and-b}就退化到了真空的情况。

需要注意的是由于$\mu$和$\epsilon$可能有虚部(对应着吸收等情况),$c^2$不一定是实数。这就产生了一个问题:开根号在复平面上是多值的,那么$c$应该取哪一个值呢?

当$c$在某区域中基本上可以看成常的实数时,它就对应着平面波传播的速度。\eqref{eq:halmholtz-eq}有下面的平面波解
\begin{equation}
    \vb*{E} = \vb*{E}_0 \ee^{\ii(\vb*{k} \cdot \vb*{r} - \omega t)}, \quad k = \frac{\omega}{c}
    \label{eq:plane-wave}
\end{equation}
其中$\vb*{k}$是实矢量。
由空间中的傅里叶变换,\eqref{eq:halmholtz-eq}的所有解都可以写成不同$\vb*{k}$的平面波的叠加。

以上完成了关于介质中的控制方程的探讨。下面考虑界面上的衔接条件。最自然的想法,由\eqref{eq:scalar-cons-maxwell-e-and-b},可以写出自然边界条件
\[
    \begin{bigcase}
        \vb*{n} \cdot (\epsilon_i \vb*{E}_1 - \epsilon_t \vb*{E}_2) &= 0, \\
        \vb*{n} \times (\vb*{E}_1 - \vb*{E}_2) &= 0, \\
        \vb*{n} \cdot (\vb*{B}_1 - \vb*{B}_2) &= 0, \\
        \vb*{n} \times \left( \frac{\vb*{B}_1}{\mu_1} - \frac{\vb*{B}_2}{\mu_2} \right) &= 0
    \end{bigcase}
\]
但是正如\eqref{eq:scalar-cons-maxwell-e-and-b}实际上有冗余一样,上式也有方程是多余的。实际上,由\eqref{eq:e-in-material}可以得出的关于电场的相互独立的边界条件为
\begin{equation}
    \left\{\quad
        \begin{aligned}
            \epsilon_1 \vb*{n} \cdot \vb*{E}_1 = \epsilon_2 \vb*{n} \cdot \vb*{E}_2, \\
            \vb*{n} \times \vb*{E}_1 = \vb*{n} \times \vb*{E}_2, \\
            \vb*{n} \times \left( \frac{1}{\mu_1} \curl{\vb*{E}_1} \right) = \vb*{n} \times \left(\frac{1}{\mu_2} \curl{\vb*{E}_2} \right)
        \end{aligned}
    \right.
    \label{eq:e-bound-condition}
\end{equation}
只需要这些边界条件结合\eqref{eq:e-in-material}就能够定解。

\subsection{电各向异性光学介质}

现在考虑一种稍为推广的情况。很多介质——比如晶体——都具有空间上的各向异性,这是因为从不同的方向施加电场可以导致不同强度的极化。
在几乎所有常见的情况中,各向异性仅限于$\vb*{D}$和$\vb*{E}$的关系中,于是\eqref{eq:scalar-constitutive}修正为
\begin{equation}
    \vb*{D}(\vb*{r}) = \vb*{\epsilon}(\vb*{r}, \omega) \vb*{E}(\vb*{r}), \quad \vb*{H}(\vb*{r}) = \frac{1}{\mu(\vb*{r}, \omega)} \vb*{B}(\vb*{r})
    \label{eq:e-tensor-constitutive}
\end{equation}
相应的,\eqref{eq:e-in-material}修改为
\begin{equation}
    \begin{bigcase}
        \div{(\vb*{\epsilon} \cdot \vb*{E})} &= 0, \\
        \curl{ \left(\frac{1}{\mu} \curl{\vb*{E}} \right) } &= \omega^2 \vb*{\epsilon} \cdot \vb*{E}, \\
        \curl{\vb*{E}} &= \ii \omega \vb*{B}
    \end{bigcase}
    \label{eq:e-in-tensor-material}
\end{equation}

我们要研究$\mu$和$\vb*{\epsilon}$变化不大时\eqref{eq:e-in-tensor-material}的平面波解。
此时关于电场的全部方程为
\[
    \begin{aligned}
        \div{\vb*{\epsilon} \cdot \vb*{E}} = 0, \\
        \grad{(\div{\vb*{E}})} - \laplacian{\vb*{E}} = \omega^2 \mu \vb*{\epsilon} \cdot \vb*{E}
    \end{aligned}
\]
不过不难注意到第一个方程可以从第二个方程推导出来,因此方程
\begin{equation}
    \grad{(\div{\vb*{E}})} - \laplacian{\vb*{E}} = \omega^2 \mu \vb*{\epsilon} \cdot \vb*{E}
    \label{eq:anisotropy}
\end{equation}
如果$\vb*{E}$是一个波矢为$\vb*{k}$,频率为$\omega$的单色波,那么这两个方程就推出
\[
    \begin{aligned}
        - (\vb*{k} \cdot \vb*{E}) \vb*{k} + k^2 \vb*{E} = \omega^2 \mu \vb*{\epsilon} \cdot \vb*{E}, \\
    (\vb*{k} \cdot \vb*{\epsilon}) \cdot \vb*{E} = 0
    \end{aligned}
\]
但是这两个方程不是相互独立的——在第一个方程两边点乘$\vb*{k}$就可以推出第二个。
顺带提一句:第二个方程表明,在介质是电各向异性的情况下,波矢$\vb*{k}$并不垂直于电场,而是垂直于$\vb*{D}$。

那么电各向异性光学介质中的平面波就完全地被下式描写:
\[
    k^2 \vb*{E} - (\vb*{k} \cdot \vb*{E}) \vb*{k} - \omega^2 \mu \vb*{\epsilon} \cdot \vb*{E} = 0
\]
也就是说
\begin{equation}
    \left( k^2 \vb*{\delta} - \vb*{k} \vb*{k} - \omega^2 \mu \vb*{\epsilon} \right) \cdot \vb*{E} = 0
    \label{eq:plain-wave-in-anistrophy}
\end{equation}
其中$\vb*{\delta}$为单位张量。

\eqref{eq:plain-wave-in-anistrophy}将电场的方向和波矢的方向联系了起来。要观察波矢自己要满足什么条件,只需要求解
\begin{equation}
    \det \left( k^2 \vb*{\delta} - \vb*{k} \vb*{k} - \omega^2 \mu \vb*{\epsilon} \right) = 0
    \label{eq:k-det}
\end{equation}
即可。

直接处理矢量形式的方程会比较棘手。但是注意到由于能量守恒的原因,$\vb*{D}$不可能和$\vb*{E}$反向,也就是
\[
    \vb*{D} \cdot \vb*{E} = \vb*{E} \cdot \vb*{\epsilon} \cdot \vb*{E} > 0
\]
那么我们可以确定$\vb*{\epsilon}$是正定的(TODO:有虚部时怎么办?)另一方面,晶体的空间对称性意味着$\vb*{\epsilon}$一定可以对角化,因此我们总是可以找到一个坐标系(未必是正交坐标系),在其中$\vb*{\epsilon}$被对角化,且对角元均为正数。
这个坐标系就是所谓的\concept{主轴系}。

\section{介质中的能流}

为了讨论能量,这一节我们再次引入自由电荷的概念,用于讨论场中有实物粒子的情况。(如果实物粒子不带电荷,那么它就不参与电磁相互作用,此时不方便从牛顿力学出发讨论能量)
我们只追踪自由电荷的能量,而不追踪所有电荷的能量。
于是,使用\eqref{eq:maxwell-material}以及洛伦兹力公式
\[
    \vb*{f} = q \vb*{E} + q \vb*{v} \times \vb*{B}
\]
可以得到
\[
    \vb*{f} \cdot \vb*{v} = - \div{(\vb*{E} \times \vb*{H})} - \vb*{H} \cdot \pdv{\vb*{B}}{t} - \vb*{E} \cdot \pdv{\vb*{D}}{t}
\]
则可以选取最简单的电磁能量密度和电磁能流为
\begin{equation}
    \begin{bigcase}
        \vb*{S} &= \vb*{E} \times \vb*{H}, \\
        \pdv{w}{t} &= \vb*{E} \cdot \pdv{\vb*{D}}{t} + \vb*{H} \cdot \pdv{\vb*{B}}{t}
    \end{bigcase}
    \label{eq:energy-in-material}
\end{equation}
其中的$\vb*{S}$就是\concept{坡印廷矢量}。公式\eqref{eq:energy-in-material}的导出没有使用线性介质假设。
在已知$\vb*{D}$和$\vb*{E}$的关系线性、$\vb*{H}$和$\vb*{B}$的关系也是线性的情况下,有
\begin{equation}
    w = \frac{1}{2} (\vb*{E} \cdot \vb*{D} + \vb*{H} \cdot \vb*{B})
    \label{eq:energy-density-linear-material}
\end{equation}
需要注意的是\eqref{eq:energy-in-material}和\eqref{eq:energy-density-linear-material}本身都是二次型,因此在这些公式中的场必须是完整的电场和磁场,而不能够是(可能带有虚部的)傅里叶分量或者诸如此类的东西。

在稳态时各个场做正弦变化,这也是本文关注的主要场景之一。首先讨论介质中只有一个单色波的情况。不失一般性地认为$\vb*{E}$的相位零点在$t=0$处,则有
\[
    \vb*{S} = \vb*{E}_0 \times \vb*{H}_0 \cos(\omega t) \cos(\omega t + \phi)
\]
其中$\phi$是两者的相位差,$\vb*{E}_0, \vb*{H}_0$分别是振幅,这三者均只有实部。我们更加关心一段时间内的平均能流,则有
\[
    \expval*{\vb*{S}} = \frac{1}{2} \vb*{E}_0 \times \vb*{H}_0 \cos \phi
\]
现在考虑含有虚部的$\vb*{E}, \vb*{H}$表达式
\[
    \vb*{E} = \vb*{E}_0 \ee^{\ii \omega t}, \vb*{H} = \vb*{H}_0 \ee^{\ii \phi} \ee^{\ii \omega t}
\]
我们会发现一个有趣的结果:
\[
    \Re \vb*{E}^* \times \vb*{H} = \expval*{\vb*{S}}
\]
直观地看,可以将$\vb*{E}^* \times \vb*{H}$的实部看成“有功分量”,虚部看作“无功分量”。
这样可以定义\concept{实数版本的辐照度}
\begin{equation}
    \vb*{I} = \expval*{\vb*{S}} = \frac{1}{2} \vb*{E}_0 \times \vb*{H}_0 \cos \phi
    \label{eq:radiation-real}
\end{equation}
也可以定义\concept{复数版本的辐照度}
\begin{equation}
    \vb*{I} = \frac{1}{2} \Re \vb*{E}^* \times \vb*{H}
    \label{eq:radiation-complex}
\end{equation}
\eqref{eq:radiation-complex}的实部就是\eqref{eq:radiation-real}。

接着再讨论平面波的能量密度。由于
% \[
%     \vb*{k} \cdot \vb*{B} = 0, \quad \vb*{D} = - \frac{\vb*{k}}{\omega} \cross \vb*{H},
% \]
% 我们有
% \[
%     D^2 = \frac{k^2}{\mu^2 \omega^2} B^2
% \]
% 则
% \[
%     \frac{1}{2} \vb*{H} \cdot \vb*{B} = \frac{\mu \omega^2}{2 k^2} D^2, 
% \]
% \[
%     \frac{1}{2} \vb*{E} \cdot \vb*{D} = \frac{1}{2} \vb*{E} \cdot \vb*{\epsilon} \cdot \vb*{E}
% \]
% 于是
% \begin{equation}
%     w = w_E + w_B = \frac{1}{2} \vb*{E} \cdot \vb*{\epsilon} \cdot \vb*{E} + \frac{\mu \omega^2}{2 k^2} D^2
%     \label{eq:energy-density}
% \end{equation}
\[
    w = w_E + w_B,
\]
分别化简两项,有
\[
    w_E = \frac{1}{2} \vb*{E} \cdot \vb*{D} 
        = \frac{1}{2} \vb*{E} \cdot \left( - \frac{\vb*{k}}{\omega} \times \vb*{H} \right) 
        = \frac{1}{2} \frac{\vb*{k}}{\omega} \cdot (\vb*{E} \times \vb*{H}),
\]
且
\[
    w_B = \frac{1}{2} \vb*{H} \cdot \vb*{B} 
        = \frac{1}{2} \vb*{H} \cdot \left( \frac{\vb*{k}}{\omega} \cdot \vb*{E} \right) 
        = \frac{1}{2} \frac{\vb*{k}}{\omega} \cdot (\vb*{E} \times \vb*{H}),
\]
因此电场能和磁场能各占总能量的一半,且
\begin{equation}
    w = \frac{\vb*{k}}{\omega} \cdot (\vb*{E} \times \vb*{H}) = \frac{k S}{\omega} \cos \alpha
    \label{eq:energy-density-and-s}
\end{equation}
其中$\alpha$是$\vb*{k}$和$\vb*{S}$的夹角。
相应的,能量传递的速度为
\begin{equation}
    \vb*{v} = \frac{\vb*{S}}{w} = \frac{\omega}{k}  \cos \alpha \vu*{S}
\end{equation}


\section{线性介质的光学性能的微观模型}

\subsection{经典谐振子模型}

\subsubsection{洛伦兹模型}

我们现在转而考虑存在阻尼的谐振子,有
\begin{equation}
    m \ddot{\vb*{r}} = - m \gamma \dot {\vb*{r}} + q (\vb*{E} + \vb*{v} \times \vb*{B}),
\end{equation}
以$\vb*{B}$的指向为$z$轴,则$\vb*{E}$一定在$xy$平面内,则上式成为% TODO
\begin{equation}
    v_x = \frac{q}{m} \tau E_x + \omega_\text{L} \tau v_y,
\end{equation}
其中
\begin{equation}
    \omega_\text{L} = \frac{qB}{m}
\end{equation}
为\concept{拉莫频率},而
\begin{equation}
    \tau = \frac{1}{\gamma}.
\end{equation}

\subsection{能带理论的线性响应}

关于能带理论的详细介绍应当参考\soliddoc中的有关章节。

\chapter{均匀线性介质中的电磁波}

\section{各向同性线性介质中的平面波}\label{sec:light-propagate}

光无非是成束的电磁场,因此接下来可以通过求解\eqref{eq:e-in-material}获得光的传播情况。
在光学中常见的介质或者是性能变化比较均匀的,或者是性能变化非常剧烈的(也就是介质界面附近,例如水和空间交界处)。
前者中的

\subsection{均匀介质内部光的传播}\label{sec:in-interior-uniform}

在均匀介质内部光的传播情况由\eqref{eq:halmholtz-eq}控制。
当介质内部的$\epsilon$和$\mu$都是实数时可以直接使用平面波解\eqref{eq:plane-wave}。而当$\epsilon$和$\mu$含有虚部——也就是说,介质含有吸收等性质——那么根据解析延拓的原理,\eqref{eq:plane-wave}的形式应该得到保持,但是$\vb*{k}$需要有虚部。%
\footnote{从积分变换的角度来看,$\epsilon\mu$含有虚部意味着有关的方程难以直接通过傅里叶展开化简,因为傅里叶展开在实函数上比较简单。为此我们将$\vb*{k}$推广到复数的情况,实际上就是从空间傅里叶变换推广到了空间拉普拉斯变换。}
实际上,即使是$\epsilon$和$\mu$都取实数值的时候,\eqref{eq:plane-wave}中的$\vb*{k}$也可以有虚部。当然,此时的电场不再是一个基本解了,因为它可以使用若干个实数$\vb*{k}$的真正的平面波叠加出来。

在最一般的、$\epsilon$和$\mu$是否有虚部不知道的情况下,将\eqref{eq:plane-wave}代入\eqref{eq:halmholtz-eq}得到
\[
    - (\Re \vb*{k} + \Im \vb*{k})^2 + \omega^2 \epsilon\mu = 0
\]
化简,由于交叉项是纯虚数而其他项都是纯实数,可得
\[
    \begin{bigcase}
        (\Re \vb*{k})^2 - (\Im \vb*{k})^2 &= \omega^2\Re \epsilon\mu, \\
        2 (\Re \vb*{k}) \cdot (\Im \vb*{\vb*{k}}) &= \omega^2\Im \epsilon\mu.
    \end{bigcase}
\]
因此,在一个$\epsilon$和$\mu$都没有虚部的介质中,有
\[
    (\Re \vb*{k}) \cdot (\Im \vb*{\vb*{k}}) = 0
\]
还有\eqref{eq:e-in-material}的第一式没有使用。它意味着
\[
    \vb*{k} \cdot \vb*{E} = (\Re \vb*{k} + \ii \Im \vb*{k}) \cdot \vb*{E} = 0
\]

这样一来我们得出结论:在均匀的、可能有吸收等因素,从而使$c^2$有虚部的介质中,有形如下式的解:
\begin{equation}
    \left\{
        \begin{aligned}
            &\vb*{E} = \vb*{E}_0 \ee^{\ii (\vb*{k} \cdot \vb*{r} - \omega t)}, \\
            &(\Re \vb*{k})^2 - (\Im \vb*{k})^2 = \omega^2 \Re \epsilon\mu, \\
            &2 (\Re \vb*{k}) \cdot (\Im \vb*{\vb*{k}}) = \omega^2 \Im \epsilon\mu, \\
            &\vb*{k} \cdot \vb*{E} = 0
        \end{aligned}
    \right.
    \label{eq:uniform-wave}
\end{equation}

在$\epsilon\mu$完全就是实数的时候\eqref{eq:uniform-wave}意味着$\vb*{k}$的实部、虚部相互正交(电场的方向和$\vb*{k}$的实部、虚部没有特别明显的关系)。
这种情况只有可能发生在界面附近,并且要求$\Re \vb*{k}$平行于界面而$\Im \vb*{k}$要垂直于界面且方向从界面指向介质内部,
否则电场会在远处发散到无穷大,不是物理解。%
\footnote{关于物理解有必要说明这一点:实际上,平面波本身也不是真正具有物理意义的波,因为不可能让电场充满整个空间,且同时只具有一个频率。
然而,实际的光的分布满足的特定边界条件意味着我们确实没有必要讨论所有可能的平面波——只需要讨论满足这些边界条件的平面波即可。
例如,光通常是从一个光源打出来的,这就暗示了两个边界条件:
首先,某个平面(也就是光源所在的平面)上的场强是给定的,需要计算的是这个平面某一侧的场强,另一侧的场强没有意义(因为它在光源之后);
其次,光源两侧无穷远处的场强应趋于零。
这就意味着,组合成这个光的平面波在垂直于$\Re \vb*{k}$的方向上不能发散,在与$\Re\vb*{k}$相反的方向上可以发散,在与$\Re\vb*{k}$同向的方向上不能发散。
这就是我们使用的“物理解”的条件。
}
我们将在\ref{sec:total-reflect}节看到这种解的一个例子。
而当$\epsilon\mu$含有虚部时,可以做下面的分解:
\[
    \Im \vb*{k} = \vb*{k}_\parallel + \vb*{k}_\bot
\]
使$\vb*{k}_\parallel$平行于$\Re\vb*{k}$,$\vb*{k}_\bot$垂直于$\Re\vb*{k}$。
这两个分量当然都会让电场在远处发散到无穷大,但是物理解的要求意味着$\vb*{k}_\parallel$与$\Re \vb*{k}$同向,
而如果$\vb*{k}_\bot$非零,那么必定有$\Re \vb*{k}$(从而$\vb*{k}_\parallel$)平行于界面而$\vb*{k}_\bot$要垂直于界面且方向从界面指向介质内部。

既然$\vb*{k}_\parallel$和$\Re \vb*{k}$总是同向,我们有理由认为这两者可以看成同一个对象的实部和虚部。
因此,在\ref{sec:light-propagate}节的剩下部分,我们用$\vb*{k}$代替原本的$\Re \vb*{k} + \ii \vb*{k}_\parallel$,
而使用$\vb*{\beta}$代替原本的$\vb*{k}_\bot$,这样\eqref{eq:uniform-wave}就需要改写为
\begin{equation}
    \left\{\quad
        \begin{aligned}
            \vb*{E} = \vb*{E}_0 \ee^{- \vb*{\beta} \cdot \vb*{r}} \ee^{\ii (\vb*{k} \cdot \vb*{r} - \omega t)}, \\
            \vb*{k}^2 - \vb*{\beta}^2 = \mu\epsilon\omega^2, \quad \vb*{k} \cdot \vb*{\beta} = 0, \\
            (\vb*{k} + \ii \vb*{\beta}) \cdot \vb*{E} = 0
        \end{aligned}
    \right.
    \label{eq:beta-k-uniform-wave}
\end{equation}
其中$\vb*{k}$与波前传播方向同向。$\vb*{k}$,$\vb*{\beta}$,$\vb*{E}$三者相互垂直,$\vb*{k}$可以有虚部而$\vb*{\beta}$没有。
无论$\vb*{k}$有没有虚部,它都可以写成一个复数乘以一个实矢量的形式,因此可以非常良好地定义$\vb*{k}$方向上的单位矢量$\hat{\vb*{k}}$。
反之,在\eqref{eq:uniform-wave}中,由于没有将$\vb*{k}$做适当的分解,可能难以良定义一个单位矢量,因为此时不同基向量上的分量可能具有不成比例的实部和虚部。

同样,为了让解是物理解,应当有$\vb*{\beta}$垂直于某个界面且方向从界面指向介质内部,否则解会发散。

特别的,在$\vb*{\beta}$为零的时候,我们有
\begin{equation}
    \left\{
        \begin{aligned}
            \vb*{E} = \vb*{E}_0 \ee^{\ii(\vb*{k} \cdot \vb*{r} - \omega t)}, \\
            \vb*{k}^2 = \mu\epsilon \omega^2, \\
            \vb*{k} \cdot \vb*{E} = 0.
        \end{aligned}
    \right.
    \label{eq:beta-zero-uniform-wave}
\end{equation}

关于$\vb*{k}$的取值需要额外的注记。我们设$\vb*{k}=k \hat{\vb*{k}}$,其中$\hat{\vb*{k}}$正是\eqref{eq:uniform-wave}中的那个$\Re\vb*{k}$(和\eqref{eq:beta-k-uniform-wave}中的$\vb*{k}$同向但是不相同)的单位向量。
那么从\eqref{eq:beta-k-uniform-wave}就可以得出
\begin{equation}
    k^2 - \beta^2 = \mu \epsilon \omega^2
\end{equation}
其中$\beta$的取值通常和边界条件有关,确定了$\beta$就能够得到$k$。
但是还有一个额外的问题:$\mu \epsilon$含有虚部,因此下面的表达式
\[
    k = \sqrt{\beta^2 + \mu \epsilon \omega^2}
\]
就是多值的。具体取哪一个值需要根据物理条件确定。在能够确定具体取那个值的时候,我们规定\concept{折射率}为
\begin{equation}
    n(\omega) = c_0 \sqrt{\epsilon(\omega) \mu(\omega)}, 
    \label{eq:refractivity}
\end{equation}
这时就有
\begin{equation}
    \quad \frac{\omega}{k} = \frac{c_0}{n} \equiv c
    \label{eq:k-and-omega-and-n}
\end{equation}
在$n$没有虚部时$c$就是介质中\eqref{eq:beta-k-uniform-wave}的波前传播速度。

\subsection{两种透明均匀介质界面上的折射和反射}\label{sec:two-isotrophy-surface}

所谓透明介质指的是折射率完全为实数的介质。在这种介质中,
如图\ref{fig:ray-onto-flat-surface}所示,我们假定有两个几乎无限大的介质以一个完全平坦的界面隔开。
\begin{figure}
    \centering
    \begin{tikzpicture}
        % 两种介质
        \node[above] at (-4, 2) {$n_i$};
        \node[above] at (-4, -3) {$n_t$};
        % 介质界面
        \draw (-4,0) -- (4,0);
        \node[above] at (-3.5,0) {\small 介质界面};
        % 法线
        \draw[dash pattern=on5pt off3pt] (0,4) -- (0,-4);
        % 入射光线
        \draw[ray] (130:5.2) -- (0,0);
        \node[] at (-2,3) {$\vb*{k}_i$};
        \draw (0,1) arc (90:130:1);
        \node[] at (110:1.4) {$\theta_i$};
    \end{tikzpicture}
    \caption{光入射平整表面}
    \label{fig:ray-onto-flat-surface}
\end{figure}
现在让一束光照射到这个界面上。为了方便起见,假定在我们感兴趣的尺度内光强\concept{处处相同},也就是入射光形式为
\[
    \vb*{E}_i = \vb*{E}_{i0} \ee^{\ii(\vb*{k}_i \cdot \vb*{r} - \omega t)}
\]
(图\ref{fig:ray-onto-flat-surface}中的实线只是代表波的传播方向,不代表光束。)

\begin{figure}
    \centering
    \begin{tikzpicture}
        \draw [-{Stealth}] (-1, 1) -- (1, -1);
        \node[above] at (0,-1) {$\vb*{k}$};
        \draw [-{Stealth}, thick] (0.5,0.5) -- (1.5,1.5);
        \node at (0.4,1) {$\vb*{E}_p$};
        \node at (1.5, 0) {$\odot$};
        \node at (2, 0) {$\vb*{E}_s$};
    \end{tikzpicture}
    \caption{电场的分解}
    \label{fig:decomposition-of-e}
\end{figure}

同时为了将矢量方程\eqref{eq:e-bound-condition}标量化,将电场按照图\ref{fig:decomposition-of-e}的方式做分解。
我们没有给出平行于$\vb*{k}$的电场分量,因为按照\eqref{eq:beta-k-uniform-wave},这个电场分量不存在。

\subsubsection{有透射光的情况}

\begin{figure}
    \centering
    \begin{tikzpicture}
        % 两种介质
        \node[above] at (-4, 2) {$n_i$};
        \node[above] at (-4, -3) {$n_t$};
        % 介质界面
        \draw (-4,0) -- (4,0);
        \node[above] at (-3.5,0) {\small 介质界面};
        % 法线
        \draw[dash pattern=on5pt off3pt] (0,4) -- (0,-4);
        \draw [-{Stealth}, thick] (0,0) -- (0,2);
        \node [above] at (0.3, 2.1) {$\vb*{n}$};
        % 入射光线
        \draw[ray] (130:5.2) -- (0,0);
        \node[] at (-2,3) {$\vb*{k}_i$};
        \draw (0,1) arc (90:130:1);
        \node[] at (110:1.4) {$\theta_i$};
        % 反射光线
        \draw[ray] (0,0) -- (50:5.2) ;
        \node[] at (2,3) {$\vb*{k}_r$};
        \draw (0,1.2) arc (90:50:1.2);
        \node[] at (70:1.4) {$\theta_r$};
        % 折射光线
        \draw[ray] (0,0) -- (-70:4.5);
        \node[] at (1.5,-3) {$\vb*{k}_t$};
        \draw (0,-1.1) arc (-90:-70:1.1);
        \node[] at (-80:1.4) {$\theta_t$};
    \end{tikzpicture}
    \caption{光入射平整表面之后发生反射和折射}
    \label{fig:ray-refraction}
\end{figure}

凭借经验,我们会认为入射波$\vb*{E}_i$会导致两个平面波,分别称为\concept{反射波}和\concept{折射波},图示为图\ref{fig:ray-refraction},形式如下:
\[
    \begin{aligned}
        \vb*{E}_r &= \vb*{E}_{r0} \ee^{\ii(\vb*{k}_r \cdot \vb*{r} - \omega t)}, \\
        \vb*{E}_t &= \vb*{E}_{t0} \ee^{\ii(\vb*{k}_t \cdot \vb*{r} - \omega t)}
    \end{aligned}
\]

这样在介质1中和在介质2中分别有
\[
    \begin{aligned}
        \vb*{E}_1 = \vb*{E}_{i0} \ee^{\ii(\vb*{k}_i \cdot \vb*{r} - \omega t)} +  \vb*{E}_{r0} \ee^{\ii(\vb*{k}_r \cdot \vb*{r} - \omega t)}, \\
        \vb*{E}_2 = \vb*{E}_{t0} \ee^{\ii(\vb*{k}_t \cdot \vb*{r} - \omega t)}
    \end{aligned}
\]
使用\eqref{eq:e-bound-condition},可以得到形如下式的结果:
\[
    \text{something } \ee^{\ii(\vb*{k}_i \cdot \vb*{r} - \omega t)} + \text{something } \ee^{\ii(\vb*{k}_r \cdot \vb*{r} - \omega t)} = \text{something } \ee^{\ii(\vb*{k}_t \cdot \vb*{r} - \omega t)}
\]
从而必须有
\[
    (\vb*{k}_i - \vb*{k}_r) \cdot \vb*{r} = \const, \quad (\vb*{k}_i - \vb*{k}_t) \cdot \vb*{r} = \const
\]
考虑到这些方程描写了一个平面,而它们都在界面(一个平面)上恒成立,有%
\footnote{能够导出这个结果是因为$\ee^{\ii \vb*{k} \cdot \vb*{r}}$是函数基。上面的方程在界面上恒成立,因此在界面上$\vb*{k}_i$,$\vb*{k}_r$,$\vb*{k}_t$的投影,也就是$\vb*{n} \times \vb*{k}$都一样。}
\begin{equation}
    \vb*{n} \times (\vb*{k}_i - \vb*{k}_r) = \vb*{n} \times (\vb*{k}_i - \vb*{k}_t) = 0
    \label{eq:k-and-n}
\end{equation}

假设入射波、反射波、折射波\concept{它们自己}都能够成为\eqref{eq:e-in-material}的解。
考虑到介质被认为是均匀的,假设这三个波都能够写成\eqref{eq:uniform-wave}的形式。
进一步,假设这三个波均没有\eqref{eq:beta-k-uniform-wave}中的$\vb*{\beta}$。
这些假设是太多了,我们将发现它们只能够在一部分场景下成立。

在这种情况下,三个波的$\vb*{k}$都可以写成一个复数乘以一个实矢量的形式,因此可以良好地定义“波前前进的方向”。
于是就有
\[
    \vb*{k}_i = k_i \hat{\vb*{k}}_i, \; \vb*{k}_r = k_r \hat{\vb*{k}}_r, \; \vb*{k}_t = k_t \hat{\vb*{k}}_t
\]
由于入射波和反射波都位于介质1中,我们确定$k_i=k_r$。

按照图\ref{fig:ray-refraction}中展示的方式标记角度。
(注意$\vb*{k}_i$和法向量的夹角为$\pi-\theta_i$)
\eqref{eq:k-and-n}可以写作
\[
    k_i \vb*{n} \times \vu*{k}_i = k_r \vb*{n} \times \vu*{k}_r, \quad k_i \vb*{n} \times \vu*{k}_i = k_t \vb*{n} \times \vu*{k}_t
\]

\begin{equation}
    \theta_i = \theta_r, \quad \frac{\sin \theta_i}{\sin \theta_t} = \frac{n_t}{n_i}
    \label{eq:snell}
\end{equation}

最终得到\concept{菲涅尔公式}
\begin{equation}
    \begin{bigcase}
        r_\bot &= \frac{E_{rs}}{E_{is}} = 
        \frac{\frac{n_i}{\mu_i} \cos \theta_i - \frac{n_t}{\mu_t} \cos \theta_t}{\frac{n_i}{\mu_i} \cos \theta_i + \frac{n_t}{\mu_t} \cos \theta_t}, \\
        t_\bot &= \frac{E_{ts}}{E_{is}} = 
        \frac{2 \frac{n_i}{\mu_i} \cos \theta_i}{\frac{n_i}{\mu_i} \cos \theta_i + \frac{n_t}{\mu_t} \cos \theta_t}, \\
        r_\parallel &= \frac{E_{rp}}{E_{ip}} = 
        \frac{\frac{n_t}{\mu_t} \cos \theta_i - \frac{n_i}{\mu_i} \cos \theta_t}{\frac{n_i}{\mu_i} \cos \theta_t + \frac{n_t}{\mu_t} \cos \theta_i}, \\
        t_\parallel &= \frac{E_{tp}}{E_{ip}} =
        \frac{2 \frac{n_i}{\mu_i} \cos \theta_i}{\frac{n_i}{\mu_i} \cos \theta_t + \frac{n_t}{\mu_t} \cos \theta_i}
    \end{bigcase}
    \label{eq:fresnel-formulas}
\end{equation}

当$\theta_i \to 0$,也就是入射光垂直于界面时,我们有
\begin{equation}
    r_\bot = \frac{n_i / \mu_i - n_t / \mu_t}{n_i / \mu_i + n_t / \mu_t}, \quad 
    r_\parallel = \frac{n_t / \mu_t - n_i / \mu_i}{n_i / \mu_i + n_t / \mu_t}
\end{equation}
看起来这样很奇怪,因为此时不能区分p光和s光,而两者的反射系数却差了一个负号。
但实际上这是错觉。回顾图\ref{fig:decomposition-of-e},我们会发现入射和反射的p光的基矢量在$\theta \to 0$时是反向的,而入射和反射s光的基矢量在$\theta \to 0$时同向,因此反射系数就应该差一个负号——最后无论将入射光当成p光还是当成s光,都能够得到同样的反射光矢量,于是可以写出
\begin{equation}
    \vb*{E}_r = \frac{n_i / \mu_i - n_t / \mu_t}{n_i / \mu_i + n_t / \mu_t} \vb*{E}_i
\end{equation}

关于能量有下面的等式:
\begin{equation}
    T = \left( \frac{n_t \cos \theta_t / \mu_t}{n_i \cos \theta_i / \mu_i} \right) t^2, \quad R = r^2
\end{equation}

\subsubsection{全反射}\label{sec:total-reflect}

TODO:隐逝波的方向不完全

当$n_i > n_t$而
\[
    \theta_i > \arcsin \frac{n_t}{n_i}
\]
时,先前做的“入射波产生反射波和折射波,这三个波都是平面波”的假设就失效了,因为这个假设导致方程组\eqref{eq:e-bound-condition}无解。
为了在这种情况下求解\eqref{eq:e-bound-condition},尝试放松一个假设。
实验上,在全反射发生的时候没有观察到折射波,因此假定折射波在边界处快速衰减了。因此此时我们假定反射波、折射波采取下面的形式:
\[
    \begin{aligned}
        \vb*{E}_r &= \vb*{E}_{r0} \ee^{\ii (\vb*{k}_r \cdot \vb*{r} - \omega t)}, \\
        \vb*{E}_t &= \vb*{E}_{t0} \ee^{-\vb*{\beta} \cdot \vb*{r}} \ee^{\ii (\vb*{k}_t \cdot \vb*{r} - \omega t)}
    \end{aligned}
\]
其中$\vb*{\beta}$垂直于界面。所有的参量都是实的。

此时在边界上会有类似于这样的表达式:
\[
    \mathrm{something} \; \ee^{\ii (\vb*{k}_r \cdot \vb*{r} - \omega t)} + \mathrm{something} \; \ee^{\ii (\vb*{k}_r \cdot \vb*{r} - \omega t)} = \mathrm{something} \; \ee^{-\vb*{\beta} \cdot \vb*{r}} \ee^{\ii (\vb*{k}_t \cdot \vb*{r} - \omega t)}
\]

\subsubsection{公式形式的统一处理}

\subsection{各向同性介质中平面波的能量}

由于在各向同性介质中

\section{各向异性线性介质中的平面波}

以上讨论的都是各向同性介质中光的传播,也就是说,$\vb*{D}$和$\vb*{E}$、$\vb*{H}$和$\vb*{B}$之间的联系都是标量倍数。
现在稍微放松这个假设,假定$\vb*{E}$和之间的联系是一个张量,也就是说,
\[
    \vb*{D} = \vb*{\epsilon} \cdot \vb*{E}
\]
其中$\vb*{\epsilon}$是一个张量,或者也可以写成
\[
    D_i = \epsilon_{ij} E_j
\]

\subsection{单光轴透明介质内部的平面波}\label{sec:one-axis-transparent}

首先讨论一种最简单的情况。此时介质有一个对称轴,绕着这个对称轴有旋转不变性。
这就意味着$\vb*{\epsilon}$有两个特征值,其中一个对应着一个唯一的特征向量,另一个对应着两个特征向量,
且后面两个特征向量垂直于前一个特征向量(否则不能保证旋转不变性)。
% TODO:群论
此时可以将$\vb*{\epsilon}$正交对角化。
于是,可以找到一个直角坐标系$x,y,z$,使$z$方向对应着前一个特征向量,$x,y$方向对应着后两个特征向量(需要对它们做正交化)
此时没有必要区分逆变协变,可以直接写出矩阵形式
\begin{equation}
    [\mu \epsilon_{ij}]_{ij} = \frac{1}{c_0^2} \bmqty{\dmat{n_o^2,n_o^2,n_e^2}}
    \label{eq:one-axis-matrix}
\end{equation}
其中$n_e, n_o > 0$。我们下这个断言是因为介质对外加电场产生的响应不可能使总电场和外加电场方向相反,也就是说,如果$\vb*{E}$取某个适当的方向使$\vb*{E}$和$\vb*{D}$之间只差一个标量倍数,那么这个标量倍数一定大于零,因此这个倍数一定有正的平方根,从而,$\vb*{\epsilon}$的特征值一定大于零。

此时\eqref{eq:k-det}等价于
\[
    \mdet{
        k_y^2 + k_z^2 - \frac{\omega^2}{c_0^2} n_o^2 & -k_x k_y & - k_x k_z \\
        - k_x k_y & k_x^2 + k_z^2 - \frac{\omega^2}{c_0^2} n_o^2 & -k_y k_z \\
        -k_x k_z & -k_y k_z & k_x^2 + k_y^2 - \frac{\omega^2}{c_0^2} n_e^2
    } = 0
\]
化简得到
\begin{equation}
    \left( \frac{k_x^2}{n_o^2} + \frac{k_y^2}{n_o^2} + \frac{k_z^2}{n_o^2} - \frac{\omega^2}{c_0^2} \right) \left( \frac{k_x^2}{n_e^2} + \frac{k_y^2}{n_e^2} + \frac{k_z^2}{n_o^2} - \frac{\omega^2}{c_0^2} \right) = 0
    \label{eq:uniaxial-crystal}
\end{equation}
因此,$\vb*{k}$只需要让其中的一个因式为零,就是可能的解。

下面讨论\eqref{eq:uniaxial-crystal}的两个解。第一种情况是
\[
    \frac{k_x^2}{n_o^2} + \frac{k_y^2}{n_o^2} + \frac{k_z^2}{n_o^2} - \frac{\omega^2}{c_0^2} = 0
\]
我们称此时的波为\concept{o光}。o光的波矢需要且只需要满足
\[
    \abs{\vb*{k}_o} = \frac{n_o \omega}{c_0}
\]
o光的电场方向需要满足什么条件?可以将\eqref{eq:plain-wave-in-anistrophy}写成
\[
    \bmqty{
        k_y^2 + k_z^2 - \frac{\omega^2}{c_0^2} n_e^2 & -k_x k_y & - k_x k_z \\
        - k_x k_y & k_x^2 + k_z^2 - \frac{\omega^2}{c_0^2} n_e^2 & -k_y k_z \\
        -k_x k_z & -k_y k_z & k_x^2 + k_y^2 - \frac{\omega^2}{c_0^2} n_o^2
    }
    \bmqty{
        E_x \\ E_y \\ E_z
    } = 0
\]
在球坐标系中写出
\[
    k_x = \frac{n_o \omega}{c_0} \sin \varphi \cos \theta, 
    \quad k_y = \frac{n_o \omega}{c_0} \sin \varphi \sin \theta, \quad k_z = \frac{n_o \omega}{c_0} \cos \varphi
\]
然后代入上面的矩阵表达式,经过一系列初等变换得到
\[
    \bmqty{
        \sin \varphi \cos \theta & \sin \varphi \sin \theta & \cos \varphi \\
        0 & 0 & 1
    }
    \bmqty{E_x \\ E_y \\ E_z} = 0
\]
于是得到了o光需要(且只需要)满足的条件:
\begin{equation}
    \abs{\vb*{k}_o} = \frac{n_o \omega}{c_0}, \quad\vb*{e}_z \cdot \vb*{E}_o = \vb*{k}_o \cdot \vb*{E}_o = 0
    \label{eq:o-light}
\end{equation}
因此,除了o光永远不会有平行于光轴的分量以外,o光在介质中的传播方式和各向同性介质中的光完全一样:$\vb*{k}$和$\vb*{E}, \vb*{D}$均垂直(并且可以验证$\vb*{D}$和$\vb*{E}$平行),且$k$和$\omega$之间的关系就是普通的折射率确定的关系。

第二种情况是
\[
    \frac{k_x^2}{n_e^2} + \frac{k_y^2}{n_e^2} + \frac{k_z^2}{n_o^2} - \frac{\omega^2}{c_0^2} = 0
\]
称此时的波为\concept{e光}。使用和上面相同的方法,写出椭球坐标之下的$\vb*{k}_e$表达式
\[
    k_x = \frac{n_e \omega}{c_0} \sin \theta \cos \varphi, \quad k_y = \frac{n_e \omega}{c_0} \sin \theta \sin \varphi, \quad k_z = \frac{n_o \omega}{c_0} \cos \theta
\]
将\eqref{eq:plain-wave-in-anistrophy}写成矩阵形式之后代入椭球坐标下的$\vb*{k}_e$表达式,然后做初等变换,得到
\[
    \bmqty{
        n_0 \sin \theta \cos \varphi & n_0 \sin \theta \sin \varphi & n_e \cos \theta \\
        -\sin \varphi & \cos \varphi & 0
    }
    \bmqty{E_x \\ E_y \\ E_z} = 0
\]
这是一个有两个方程组成的方程组。
容易验证,在本坐标系中$\bmqty{n_0 \sin \theta \cos \varphi & n_0 \sin \theta \sin \varphi & n_e \cos \theta}$
和$\vb*{k} \cdot \vb*{\epsilon}$共线,$\bmqty{-\sin \varphi & \cos \varphi & 0}$与$\vb*{e}_z \times \vb*{k}$共线。
因此e光需要且只需要满足的方程为
\begin{equation}
    \frac{k_{ex}^2}{n_e^2} + \frac{k_{ey}^2}{n_e^2} + \frac{k_{ez}^2}{n_o^2} =
    \frac{\omega^2}{c_0^2}, \quad \vb*{k}_e \cdot \vb*{\epsilon} \cdot \vb*{E}_e = 0, \quad (\vb*{e}_z \times \vb*{k}_e) \cdot \vb*{E}_e = 0
    \label{eq:e-light}
\end{equation}
因此,e光的振动方向被限制在了光轴和波矢确定的平面上。
此时通过$k$和$\omega$之间的关系仍然可以定义等效的折射率,它是
\begin{equation}
    n = \frac{c_0 \abs*{\vb*{k}_e}}{\omega} , \quad \frac{1}{n^2} = \frac{\sin^2 \theta}{n_e^2} + \frac{\cos^2 \theta}{n_o^2}.
    \label{eq:e-light-effective-index}
\end{equation}

e光被认为是“反常的”,因为它具有许多各向同性介质中的光完全不显示的性质。

注意到\eqref{eq:o-light}和\eqref{eq:e-light}中关于$\vb*{k}$的方程无论$\vb*{k}$的方向是什么样都是有解的,而且有唯一解。
因此,一旦$\vb*{k}$的方向确定了,e光和o光的$\vb*{k}$以及可能的振动方向也就完全确定了。

总之,各向异性线性介质中的光一般来说是满足\eqref{eq:o-light}的o光和满足\eqref{eq:e-light}的e光的叠加。

% TODO:走移角

\subsection{双光轴透明介质内部的平面波}

\subsubsection{菲涅尔法线方程}

当我们试图将\ref{sec:one-axis-transparent}节中的方法原封不动地推广到一般的各向异性介质中时,会遇到一个严重的困难:无法得到像\eqref{eq:uniaxial-crystal}这样的因式分解好了的关于$\vb*{k}$的方程。

设介质中有一列平面波,记\concept{这一列波的折射率}为
\begin{equation}
    n = \frac{k c_0}{\omega} = \frac{k}{\omega} \frac{1}{\sqrt{\mu_0 \epsilon_0}}
\end{equation}
那么就有
\[
    \vb*{k} = n \omega \sqrt{\mu_o \epsilon_0} \vu*{k}
\]
由\eqref{eq:k-det},可以得到
\[
    \det \left( n^2 \omega^2 \epsilon_0 \mu_0 (\vb*{\delta} - \vu*{k} \vu*{k}) - \omega^2 \mu \vb*{\epsilon} \right) = 0
\]
\[
    \det \left( \vb*{\delta} - \vu*{k} \vu*{k} - \frac{\mu_r \vb*{\epsilon}_r}{n^2} \right) = 0
\]
然后求解这个方程。由于此时$\vu*{k}$已经是给定的了,我们将使用$\vu*{k}, \mu_r, \vb*{\epsilon}_r$表示出$n$。
在主轴坐标系中表示$\vb*{k}$和其它矢量,并且设对角化之后%
\footnote{
    所谓对角化,在这里指的是写成
    \[
        \vb*{\epsilon}_r = \epsilon_{r \underline{i}} \vb*{g}_i \vb*{g}^i
    \]
    的形式。
}%
的$\vb*{\epsilon}_r$的三个元素为$\epsilon_{rx}, \epsilon_{ry}, \epsilon_{rz}$
则有
\begin{equation}
    \frac{k_x k^x}{\frac{1}{n^2} - \frac{1}{\mu_r \epsilon_{rx}}} + \frac{k_y k^y}{\frac{1}{n^2} - \frac{1}{\mu_r \epsilon_{ry}}} + \frac{k_z k^z}{\frac{1}{n^2} - \frac{1}{\mu_r \epsilon_{rz}}} = 0
    \label{eq:fresnel-k-n}
\end{equation}
将\eqref{eq:fresnel-k-n}展开为$1/n$的多项式之后会发现这是一个关于$1/n^2$的二次多项式,且在实数域内有解,因此\eqref{eq:fresnel-k-n}有两个正根两个负根。
仅考虑物理解,能够得到两个正根。
这表明了任意的各向异性介质的特点:给定一个$\vb*{k}$,可以有两种相位传播速度不同的波。

若设$\vb*{\epsilon}$在主轴系中被对角化为
\begin{equation}
    c_0^2 [\mu \epsilon_{ij}]_{ij} = [\mu_r \epsilon_{r\;ij}]_{ij} = \bmqty{\dmat{n_x^2, n_y^2, n_z^2}}
    \label{eq:diag-two-axis}
\end{equation}
则有
\begin{equation}
    \frac{k_x k^x}{\frac{1}{n^2} - \frac{1}{n_x^2}} + \frac{k_y k^y}{\frac{1}{n^2} - \frac{1}{n_y^2}} + \frac{k_z k^z}{\frac{1}{n^2} - \frac{1}{n_z^2}} = 0
\end{equation}
这个方程称为\concept{菲涅尔法线方程}。虽然我们使用$\vb*{k}$的各个分量写出了它,由于其齐次性,完全可以将所有$\vb*{k}$的分量替换为$\vu*{k}$的分量。

需要注意的是对应于$n$的两个根的波的振动方向并不是任意的。使用本节的记号,可以将\eqref{eq:plain-wave-in-anistrophy}写成
\begin{equation}
    \left(\vb*{\delta} - \vu*{k}\vu*{k} - \frac{\mu_r \vb*{\epsilon}_r}{n^2} \right) \cdot \vb*{E} = 0
\end{equation}
它意味着:首先,$n$应当被适当地选定,让方程左边的张量的行列式为零,这等价于\eqref{eq:fresnel-k-n};其次,$n$被确定后,$\vb*{E}$可能的方向也被确定了下来。$\vb*{E}$可能的取值就是$\vb*{\delta} - \vu*{k}\vu*{k} - \mu_r \vb*{\epsilon}_r / n^2$的零空间。

一个可能的问题:在已经选定了$\vu*{k}$之后,我们能够得到两个$n$,从而两个$\vb*{E}$振动的方向,那么为什么不是三个方向?
原因在于我们有约束$\vb*{k} \cdot \vb*{D} = \vb*{k} \cdot \vb*{\epsilon} \cdot \vb*{E} = 0$,因此实际能够取的$\vb*{E}$(或者$\vb*{D}$)分布在一个二维的空间中,而不是三维的空间。

此外,注意到相速度$v_\text{p}$就是$c_0 / n$,因此在波矢方向已经给定的情况下从\eqref{eq:fresnel-k-n}可以解出两个相速度。当然,这就是对应于两个$n$的平面波的传播速度。

\subsubsection{折射率椭球}

% TODO:各向异性是不是还是电场能和磁场能各占一半?
现在考虑区域内电场能量密度
\[
    w_E = \frac{1}{2} \vb*{E} \cdot \vb*{D} = \frac{1}{2} (E^x D_x + E^y D_y + E^z D_z),
\]
同样在主轴坐标系当中工作,由于
\[
    D^x = \epsilon_0 \epsilon_{rx} E^x, \quad D^y = \epsilon_0 \epsilon_{ry} E^y, \quad D^z = \epsilon_0 \epsilon_{rz} E^z
\]
我们有
\begin{equation}
    \frac{2 \epsilon_0 w_E}{\mu_r} = \frac{D_x D^x}{n_x^2} + \frac{D_y D^y}{n_y^2} + \frac{D_z D^z}{n_z^2}
\end{equation}
因此如果固定电磁能密度不变,那么这一点的$\vb*{D}$扫过一个椭球面。这个椭球面称为\concept{折射率椭球}。

\subsubsection{各矢量方向的分析}

首先,下面三个方程还是成立的,正如在各向同性介质中一样:
\[
    \vb*{k} \cdot \vb*{D} = 0, \quad \vb*{k} \cdot \vb*{H} = 0, \quad \vb*{k} \times \vb*{H} = - \omega \vb*{D}
\]
这意味着$\vb*{D}, \vb*{H}, \vb*{k}$构成一组右手系。$\vb*{B}$的方向和$\vb*{H}$完全一致,因此无需单独讨论其方向。

反之,由于$\vb*{\epsilon}$的各向异性,$\vb*{E}$的方向需要特别注意。
由于
\[
    \vb*{k} \times \vb*{E} = \omega \vb*{B}, \quad \vb*{k} \times \vb*{H} = - \omega \vb*{D}
\]
可以导出下面的方程
\begin{equation}
    \vb*{D} = \mu_r \epsilon_0 n^2 (\vb*{E} - (\vu*{k} \cdot \vb*{E}) \vu*{k}) = \mu_r \epsilon_0 n^2 \vb*{E}_{\bot}
    \label{eq:first-crystal-eq}
\end{equation}
即所谓\concept{晶体光学第一方程},其中$\bot$表示在垂直于$\vb*{k}$的方向上做投影。实际上,通过将$\vb*{D} = \vb*{\epsilon} \cdot \vb*{E}$代入上式,也能够导出菲涅尔法线方程\eqref{eq:fresnel-k-n}。

% TODO:画图
\eqref{eq:first-crystal-eq}使用$\vb*{E}$表示了$\vb*{D}$;我们也可以反过来尝试使用$\vb*{D}$表示$\vb*{E}$。注意到$\vb*{E}$虽然和$\vb*{D}$未必重合,但是它一定落在垂直于$\vb*{H}$的平面内;而$\vu*{k} \propto \vb*{D} \times \vb*{H}$,$\vu*{S} \propto \vb*{E} \times \vb*{H}$,于是几何观察告诉我们,将$\vu*{D},\vu*{k}$做一个旋转角为$\alpha$(它正是\eqref{eq:energy-density-and-s}中的那个$\alpha$),在垂直于$\vb*{H}$的平面上的旋转就得到了$\vu*{E}, \vu*{S}$。
于是记
\begin{equation}
    n_r = n \cos \alpha,
\end{equation}
就得到了
\begin{equation}
    \vb*{E} = \frac{1}{\mu_r \epsilon_0 n_r^2} (\vb*{D} - (\vu*{S} \cdot \vb*{D}) \vu*{S})
    \label{eq:second-crystal-eq}
\end{equation}
即所谓的\concept{晶体光学第二方程}。

联立这两个方程\eqref{eq:first-crystal-eq}和\eqref{eq:second-crystal-eq}中的其中一个和本构关系$\vb*{D} = \vb*{\epsilon} \vb*{E}$,
可以得到仅仅关于$\vb*{E}$或仅仅关于$\vb*{D}$的方程。
仅仅关于$\vb*{E}$的方程已经被建立了,它就是\eqref{eq:plain-wave-in-anistrophy},它有非零解的条件就是\eqref{eq:fresnel-k-n}。
联立晶体光学第二方程\eqref{eq:second-crystal-eq}和本构关系,尝试得到仅仅关于$\vb*{D}$的方程。
在主轴系下进行计算,此时本构关系为\eqref{eq:diag-two-axis},
就得到
\begin{equation}
    \frac{S_x S^x}{n_x^2 - \mu_r n_r^2} + \frac{S_y S^y}{n_y^2 - \mu_r n_r^2} + \frac{S_z S^z}{n_z^2 - \mu_r n_r^2} = 0
\end{equation}
它是使用$\vb*{S}$和$n_r$表示的\eqref{eq:fresnel-k-n}的对应物。
定义

\subsubsection{能量和能流}



\section{光强,成像和相干性}

\subsection{几何光学中的光强传输}

实际上,我们也可以将\eqref{eq:optical-distance}当成$L$的\emph{定义},此时无论几何光学是不是适用,都能够定义光线等。在此基础上,\eqref{eq:expanded-wave-eq}化为
\begin{equation}
    \laplacian \vb*{A} + 2 \frac{\ii \omega}{c_0} \grad{\vb*{A}} \cdot \grad{L} + \frac{\ii \omega}{c_0} \vb*{A} = 0.
\end{equation}
如果我们做\concept{慢变振幅近似},则上式可以写成
\begin{equation}
    \div{(\vb*{A}^2 \grad{L})} = 0.
\end{equation}
因此实际上$\vb*{A}^2 \grad{L}$可以看成一个静态的“光强传输的流”,即光线实际上给出了光强传输的流管。
物理地看,由于是各向同性介质,$\vb*{S}$和$\vb*{k}$平行,这意味着$\vu*{k}$的方向连缀而成的“光线”实际上就是$\vb*{S}$的流量线,因此若干光线包围出的“管路”实际上就是能量流动的流管。

当两束光相交时,光线可以交叉,因此“流管”的概念实际上并没有良好定义——实际上,这正是衍射、干涉等现象的起源。% TODO

因此,在慢变振幅近似成立时,使用几何光学确定光线,并以光线为振幅传输的“流线”就足够给出可靠的结果了。
我们通常称\emph{此时几何光学适用}。理论上我们对任何体系都可以通过光线方程计算光线,但是最终我们关系的是空间中各点的亮度分布,因此如果光线的概念无助于计算亮度分布,则几何光学没有什么意义,这就好像对任何一个量子理论我们都可以计算其经典版本,但是有时候计算经典版本并不能提供什么信息。

假定光学系统在每个瞬时都可以认为是处在某个稳态上,从而,虽然我们在分析动态问题,沿用亥姆霍兹方程的解足以给出精确的结果。
然而,假定光学系统中的光源会以一种随机的方式发射电磁波,从而空间中某一点的光强实际上是一系列具有不同概率权重的亥姆霍兹方程的解在这一点给出的光强的期望值。
写成公式,设某一点的电场包含从各个方向传来的电场(注意此时我们已经切换到了亥姆霍兹方程下,所谓“电场传播”实际上是电场在空间上的联系,虽然它和时域下波包的传播是直接相关的)之和
\begin{equation}
    \vb*{E}(\vb*{r}) = \sum_i \vb*{E}_i \ee^{\ii k_i \abs*{\vb*{r} - \vb*{r}_i}},
\end{equation}
展开电场平方的期望值,有
\begin{equation}
    \begin{aligned}
        \expval*{\vb*{E}(\vb*{r})^2} &= \sum_P P(\vb*{E}_1, \vb*{E}_2, \cdots, \vb*{E}_n) \sum_{i, j} \vb*{E}_i \ee^{\ii k_i \abs*{\vb*{r} - \vb*{r}_i}} \vb*{E}_j^* \ee^{- \ii k_j \abs*{\vb*{r} - \vb*{r}_j}}  \\
        &= \sum_{i, j} \expval*{\vb*{E}_i \vb*{E}_j^*} \ee^{\ii (k_i \abs*{\vb*{r} - \vb*{r}_i} - k_j \abs*{\vb*{r} - \vb*{r}_j})}.
    \end{aligned}
\end{equation}
在组成$\vb*{E}$的各个组分极度非相干的情况下,$\expval*{\vb*{E}_i \vb*{E}_i^*}$项占据压倒性优势,从而
\begin{equation}
    \expval*{\vb*{E}^2} = \sum_i \expval*{\vb*{E}_i^2}.
\end{equation}
这就是说,对高度非相干的情况,不同来源的光强可以直接相加,而无需考虑衍射等问题。
这实际上说明对高度非相干的光几何光学通常都是适用的。但这并不是说以高度非相干的光为光源就产生不了干涉和衍射,例如我们将一束非相干光分束,得到的两束光中,一束光的一个分量和另一束光中的一个分量相干,从而仍然可能产生干涉。
必须每两个$\vb*{E}_i$和$\vb*{E}_j$之间都不相干才能够保证几何光学总是成立。

何时衍射不明显:波长非常短的时候肯定不明显;相干性差的时候也是;这和“量子效应什么时候不明显”是类似的:

\chapter{光学谐振腔}

前几章的电磁波传播都是没有任何边界条件约束的,此时如果介质均匀,那么电磁波可以以平面波的形式稳定传播。
本章则讨论束缚在有限区域内的电磁波模式。

原则上当然可以通过求解关于标势和矢势的方程来分析谐振腔的行为,例如解\eqref{eq:wave-eq}即可。
通常谐振腔内部没有源,于是\eqref{eq:wave-eq}中的各个方程都是齐次的,且各个矢势分量之间没有关系,从而似乎求解标量亥姆霍兹方程即可获知谐振腔的行为。
然而,边界条件实际上还是会让各个矢势分量和标势等都产生关系。
因此使用标势和矢势并不能简化问题。于是,本章还是直接求解\eqref{eq:e-in-tensor-material}。
特别的,各向同性体态中有
\begin{equation}
    \laplacian \vb*{E} + k_0^2 \vb*{E} = 0, \quad \div{\vb*{E}} = 0, \quad k_0 = \frac{\omega}{c},
    \label{eq:isotropic-cavity-problem-origin}
\end{equation}
这里$\vb*{E} = \vb*{E}(\vb*{r}, \omega)$。
通常我们会定义
\begin{equation}
    \tilde{\vb*{E}} = \sqrt{\epsilon} \vb*{E}(\vb*{r}, \omega), \quad \tilde{\vb*{B}} = \frac{\ii}{\sqrt{\mu}} \vb*{B}(\vb*{r}, \omega)
\end{equation}
来让$\vb*{E}$和$\vb*{B}$相差不要太大,此时
\begin{equation}
    \tilde{\vb*{B}} = \frac{1}{k_0} \curl{\tilde{\vb*{E}}}.
    \label{eq:tilde-b-tilde-e-cavity}
\end{equation}
将\eqref{eq:isotropic-cavity-problem-origin}用$\tilde{\vb*{E}}$和$\tilde{\vb*{B}}$写出来,就是
\begin{equation}
    \begin{bigcase}
        &\laplacian \tilde{\vb*{E}} + k_0^2 \tilde{\vb*{E}} = 0, \quad \laplacian \tilde{\vb*{B}} + k_0^2 \tilde{\vb*{B}} = 0, \\
        &\tilde{\vb*{B}} = \frac{1}{k_0} \curl{\tilde{\vb*{E}}}, \quad \tilde{\vb*{E}} = \frac{1}{k_0} \curl{\tilde{\vb*{B}}}.
    \end{bigcase}
    \label{eq:isotropic-cavity-problem}
\end{equation}
最后一个方程只需要在倒数第二个方程两边作用散度,通过一些计算即可看出。

\section{方形谐振腔}

\subsection{立方体谐振腔}

\subsection{方形波导}

\concept{波导}是指一个长条状的、电磁波可以在其中传递的装置。我们讨论一个柱状的波导,它是一个形状任意的闭合曲线沿着垂直于截面的方向平移而形成的直导管,其壁为导体,内部填充了某种均匀介质。
基本上,能够称为电磁波的电磁场构型都有很强的趋肤效应,因此接下来如无特殊说明我们认为波导的壁是理想导体,即认为导体内部没有任何场分布,即认为边界条件为
\begin{equation}
    \vb*{n} \cdot \vb*{B} = 0, \quad \vb*{n} \times \vb*{E} = 0.
\end{equation}
表面上,从边界条件$\vb*{n} \cdot (\vb*{D}_1 - \vb*{D}_2) = 0$出发,并利用导体内部没有电场分布这一条件,似乎可以得到$\vb*{n} \cdot \vb*{D}_1=0$,但这是错误的:如果我们要求$\vb*{n} \cdot (\vb*{D}_1 - \vb*{D}_2) = 0$成立,即将导体内部的电流归入$\epsilon$中,即给$\epsilon$一个虚部,那么随着导体电导率的上升,导体内部的$\vb*{E}$的确会下降,但是$\epsilon$会上升,最后在边界上会留下一个不为零的$\vb*{n} \cdot \vb*{D}_2$。
而如果我们不将电流归入$\epsilon$,那么就有$\vb*{n} \cdot (\vb*{D}_2 - \vb*{D}_1) = \sigma$,而$\vb*{D}_2$正常地衰减,于是边界条件就是$\vb*{n} \cdot \vb*{D}_1 = \sigma$。
我们并不知道$\sigma$到底是什么,因此这个边界条件实际上是用来在$\vb*{E}$已知后返回来求解$\sigma$的。
$\vb*{n} \times (\vb*{H}_2 - \vb*{H}_1) = \vb*{j}$同理。

考虑时谐场。由于$z$方向上的平移不变性,我们可以认为
\[
    \vb*{E}, \vb*{B} \propto \ee^{\ii (k_z z - \omega t)},
\]
在导管内部,波动方程为
\begin{equation}
    \left( \pdv[2]{x} + \pdv[2]{y} + k_0^2 - k_z^2 \right) \pmqty{\vb*{E} \\ \vb*{B}} = 0, \quad k_0 = \frac{\omega}{c}.
\end{equation}
以上方程并不能定解,但是实际上通过使用方程$\curl{\vb*{E}}=-\partial_t \vb*{B}$以及$\curl{\vb*{H}} = \partial_t \vb*{E}$,$x, y$方向上的场可以写成$z$方向上的场及其导数的线性函数,因此我们只需要求解
\begin{equation}
    \left( \pdv[2]{x} + \pdv[2]{y} + k_\text{c}^2 \right) \pmqty{E_z \\ B_z} = 0, \quad k_\text{c}^2 = k_0^2 - k_z^2.
\end{equation}
我们不能指望$E_z$和$B_z$都是零,因为此时没有非平庸解,即波导的约束意味着严格的横波是不可能的。
可能的偏振模式可以分成$B_z=0$的\concept{横磁波}(TM)和$E_z=0$的\concept{横电波}(TE)两类。

简单的计算表明对横电波我们有(本段中所有的$\grad$都是二维平面上的,我们暂时忽略电磁场在$z$方向上的周期性波动)
\begin{equation}
    \vb*{B}_\text{t} = \frac{\ii k_z}{k_\text{c}^2} \grad{B_z}, \quad \vb*{E}_\text{t} = - \ii \frac{c k_0}{k_\text{c}^2} \vb*{e}_z \times \grad{B_z},
\end{equation}
于是从$\vb*{n} \cdot \vb*{B} = 0$得到$B_z$满足的边界条件
\begin{equation}
    \pdv{B_z}{n} = 0,
\end{equation}
并且这个条件也能够让$\vb*{n} \times \vb*{E} = 0$成立,于是据此条件求解$B_z$满足的亥姆霍兹方程就确定了一切。对横磁波类似的有
\begin{equation}
    \vb*{E}_\text{t} = \frac{\ii k_z}{k_\text{c}^2} \grad{E_z}, \quad \vb*{B}_\text{t} = \ii \frac{k_0}{ck_\text{c}^2} \vb*{e}_z \times \grad{E_z},
\end{equation}
边界条件为
\begin{equation}
    E_z = 0.
\end{equation}
这个边界条件是$\vb*{n} \times \vb*{E}=0$的直接推论,但是由于它让$\grad{E_z}$在边界上一定沿着$\vb*{n}$,可以验证$\vb*{n} \cdot \vb*{B}=0$也是成立的。

在$xy$平面上求解可能的TE或是TM模式,得到的是离散谱,而电磁场在$z$方向的传播却是散射态,即$\omega$和$k_z$都可以连续取值,于是波导内的模式的能谱形如
\begin{equation}
    \omega = c \sqrt{k_z^2 + k_{\text{c}, mn}^2},
\end{equation}
其中$m, n$为标记$xy$平面上的模式的整数编号。可以看到这个能谱是有能隙的,能量低于
\begin{equation}
    \omega_\text{c} = \min (c k_{\text{c}, mn})
\end{equation}
的电磁波入射波导之后会快速衰减。

\section{导引矢量法}\label{sec:guiding-vector}

注意到问题\eqref{eq:isotropic-cavity-problem}中$\tilde{\vb*{E}}$和$\tilde{\vb*{B}}$的形式高度对称,我们可以尝试通过一个特殊的构造产生它的解。

由于\eqref{eq:isotropic-cavity-problem}中的电场和磁场均满足横波条件,它们总是可以写成某个东西的旋度。
设某个矢量场$\vb*{M}$是某个东西的旋度,即满足
\begin{equation}
    {\vb*{M}} = \curl{(\psi \vb*{c})},
    \label{eq:guiding-vector-construction}
\end{equation}
这样关于$\vb*{M}$的亥姆霍兹方程就变为
\begin{equation}
    0 = \laplacian {\vb*{M}} + k_0^2 {\vb*{M}} = \curl{((\laplacian \psi + k_0^2 \psi) \vb*{c} + \psi \laplacian \vb*{c})}.
    \label{eq:m-helmholtz-original}
\end{equation}
注意此处$\vb*{c}$和$\psi$的定义不唯一。我们总是能够找到(虽然一般都不容易解析地找到)一个$\vb*{c}$满足
\begin{equation}
    \div{\vb*{c}} = \text{const}, \quad \curl{\vb*{c}} = 0.
\end{equation}
这是因为,设
\[
    \vb*{M} = \curl{\vb*{M}'},
\]
我们注意到关于某个标量场$\lambda$的方程
\[
    \div{(\lambda \vb*{M}')} = \text{const}, \quad \curl{(\lambda \vb*{M}')} = 0
\]
一定有解,因为第二个方程的三个分量方程实际上只有两个独立。
因此,我们设
\[
    \vb*{M}' = \psi \vb*{c}, \quad \psi = \frac{1}{\lambda}
\]
即可得到\eqref{eq:guiding-vector-construction},即\eqref{eq:guiding-vector-construction}中的$\vb*{c}$和$\psi$总是可以构造出来的。
我们称$\vb*{c}$为\concept{导引矢量},因为它大体上描绘了$\vb*{M}$的“指向”。

对$\vb*{c}$我们有
\[
    \laplacian \vb*{c} = \grad(\div{\vb*{c}}) - \curl{(\curl{\vb*{c}})} = 0,
\]
于是
\begin{equation}
    \vb*{M} = \grad{\psi} \times \vb*{c}, \quad \vb*{M} \bot \vb*{c},
\end{equation}
关于$\vb*{M}$的亥姆霍兹方程\eqref{eq:m-helmholtz-original}等价于
\begin{equation}
    \laplacian \psi + k_0^2 \psi = 0.
    \label{eq:scalar-cavity-eq}
\end{equation}
在解出$\vb*{M}$之后我们会注意到矢量场
\begin{equation}
    \vb*{N} = \frac{1}{k_0} \curl{\vb*{M}}
    \label{eq:cavity-n-def}
\end{equation}
满足
\begin{equation}
    \vb*{M} = \frac{1}{k_0} \curl{\vb*{N}}, 
\end{equation}
并且它也满足和$\vb*{M}$满足的亥姆霍兹方程完全一样的方程
\begin{equation}
    \laplacian \vb*{N} + k_0^2 \vb*{N} = 0.
\end{equation}

对比$\vb*{M}$和$\vb*{N}$矢量场满足的各个方程和\eqref{eq:isotropic-cavity-problem},我们发现可以将$\vb*{M}$看成$\tilde{\vb*{E}}$,将$\vb*{N}$看成$\tilde{\vb*{B}}$,也可以反过来将$\vb*{M}$看成$\tilde{\vb*{B}}$,将$\vb*{N}$看成$\tilde{\vb*{E}}$。
因此我们得到了一种原则上一般的求解光学谐振腔中的模式的方法:求解标量方程\eqref{eq:scalar-cavity-eq},然后将各个模式代入\eqref{eq:guiding-vector-construction}和\eqref{eq:cavity-n-def},这样就得到了全部的电磁波模式。

\subsection{平凡的例子:平面波}

平面波的经验给出了一种挑选$\vb*{c}$的方法:尽可能让$\vb*{c}$垂直于主要的界面方向。

\subsection{圆柱波导中的柱面波}

在圆柱波导中可以验证将$\vb*{c}$选择为
\begin{equation}
    \vb*{c} = \vb*{e}_\rho
\end{equation}
是可行的,此时需要求解
\begin{equation}
    \frac{1}{\rho} \pdv{\rho} \left( \rho \pdv{\psi}{\rho} \right) + \frac{1}{\rho^2} \pdv[2]{\psi}{\phi} + \pdv[2]{\psi}{z} + k_0^2 \psi = 0.
\end{equation}
这个方程的求解是已知的:它最终转化为柱贝塞尔方程的求解。


我们分析几种极限情况。$k \rho \to 0$的情况对应于在我们关心的距离尺度内电磁波的传播速度可以忽略的情况(或者$\rho$很小,或者$c$很大以至于$k$很小),而$x \to 0$时 % TODO:贝塞尔函数的渐近行为
此时的解就是静电势的通解。

圆柱波导中的电磁波模式本质上还是标量的。

\subsection{球面波}

选取
\begin{equation}
    \vb*{c} = r \vb*{e}_r.
\end{equation}
关于$\psi$的亥姆霍兹方程为
\begin{equation}
    \frac{1}{r^2} \pdv{r}\left( r^2 \pdv{\psi}{r} \right) + \frac{1}{r^2 \sin \theta} \pdv{\theta} \left( \sin \theta \pdv{\psi}{\theta} \right) + \frac{1}{r^2 \sin \theta} \pdv[2]{\psi}{\phi} + k_0^2 \psi = 0.
\end{equation}

以上求解过程说明三维球腔中不存在s波。
从数学上看这来自所谓\emph{毛球定理}:$S^1$上可以有一个连续而处处不为零的切向量场,但是$S^2$上不可能有这样的切向量场。

我们将球面波推导出的$\vb*{N}$和$\vb*{M}$称为\concept{球波函数}。
球波函数显然可以用于做多极矩展开。实际上,它比我们前面通过泰勒级数得到的多极矩展开更加优越,因为后者只在远场情况下能够毫无疑难地定义,在近场时会有一定的模糊性。
一个重要的例子就是\concept{环形磁偶极矩}。

\chapter{非均匀折射率导致的散射}

\part{光源}\label{part:source}

\chapter{偶极辐射的量子理论}

\section{二能级系统的偶极辐射}

考虑一个二能级系统——比如说氢原子——它正处在状态
\begin{equation}
    \ket*{\psi(t=0)} = c_\text{g} \ket*{\text{g}} + c_\text{e} \ket*{\text{e}}
\end{equation}
上。如果它不受到外界扰动,其时间演化是显然的:
\begin{equation}
    \ket*{\psi} = c_\text{g} \ee^{- \ii \omega_\text{g} t} \ket*{\text{g}} + c_\text{e} \ee^{- \ii \omega_\text{e} t} \ket*{\text{e}}.
\end{equation}
电偶极矩可能在基态和激发态上都为零,但是这不意味着$\mel{\text{e}}{\vb*{d}}{\text{d}} \eqqcolon \vb*{d}_\text{eg}$也是零。
倘若它不是零,那么电偶极矩期望值就是
\begin{equation}
    \expval*{\vb*{d}}(t) = c^*_\text{e} c_\text{g} \vb*{d}_{\text{eg}} \ee^{- \ii \omega_\text{eg} t} + \text{c.c.},
\end{equation}
其中$\omega_\text{eg}$
这意味着二能级系统如果处在基态和激发态的叠加态上,它有能力辐射出频率为$\omega_\text{eg}$的电磁波。

\section{调制光束}

偶极辐射产生的光能够传向四面八方。我们需要想出办法来将偶极辐射转化为平面波,再聚焦到需要的位置。

\chapter{激光}

本节首先介绍强光如何能够产生。

如果我们能够有一个长期保持粒子数反转的系统(显然需要持续的能量输入),那么向这个系统入射一束光将会产生更强的出射光,因为会有受激发射,且受激发射出的光和入射光是非常相干的。
因此我们向粒子数反转的系统注入的能量可以用于增强入射光,并且如果入射光相干性非常好,那么出射光也保持非常好的相干性,并且比入射光更亮,这就是\concept{激光}。

最简单的方案——使用一个二能级系统,直接通过一次入射来得到激光——是不现实的,因为此时大量的能量会消耗在保持粒子数反转上,而为了 TODO
两个可能的改进:将粒子数反转的系统放在一个四壁强反射的腔体内,从而光束可以来回走,不断被增强,并且使用一个三能级系统,其中间能级是一个亚稳态,从而很容易制造粒子数反转。
可以在腔体的一个地方“开洞”——比如说让反射率稍微低一些——让一些光子泄露出来,激光就从这里被导出。

多光子过程:可以避免光学屏障,以及排除荧光本底

\part{非线性光学}

\section{非线性光学过程的经典模型}\label{sec:classical-models}

\subsection{非线性谐振子模型}\label{sec:classical-oscillator}

本节将材料当成一系列谐振子的组合,并且暂时不考虑谐振子之间的相互作用。
这是相对合理的,因为能够长距离传输光的介质一般不是金属,从而电子是相对定域的。
然而,这并不意味着我们的理论是自由的。
我们知道一个标准的经典谐振子可以用
\begin{equation}
    m \dv[2]{x}{t} + m \gamma \dv{x}{t} + m \omega_0^2 x = q E
\end{equation}
来描述,而如果我们加上诸如$x^3$这样的项,即让谐振子的回复力为非线性的,就可以造成谐振子模式发生自相互作用。
我们讨论的问题的能量都并不高,谐振子运动不会特别快,因此可以认为谐振子只产生电场,并且其方式为“电偶极子产生库伦场”。
用作用量表示,就是% TODO:有误,这里的关键在于qxE可以给出电场对电荷的作用,但是是否能够给出电荷对电场的作用??或者说,电场和电荷的相互作用拉氏量或是哈密顿量要怎么写??
\[
    S = \int \dd{t} \left( \frac{1}{2} m \dot{x}^2 - \frac{1}{2} k x^2 + \text{higher order $x^n$} + qxE \right).
\]
现在积掉谐振子,就能够得到非线性的光子-光子过程,即多个光子和一或是一个光子分裂为多个光子。

\subsubsection{二阶非线性极化}

我们现在在谐振子能量中加入一个三次项,即在运动方程中加入一个二次回复力:
\begin{equation}
    m \dv[2]{x}{t} + m \gamma \dv{x}{t} + m \omega_0^2 x + m a x^2 = q E.
    \label{eq:x3-eq}
\end{equation}
这相当于在能量中加入了一个$\frac{1}{3} m a x^3$项。这个项破坏了系统的中心反演对称性。
我们实际上是要从电场计算$x$(从$x$计算响应电场的公式是显然的,就是大量偶极子加总为$\vb*{P}$然后从$\vb*{P}$出发算电场)。

对这个问题的标准的处理是微扰求解微分方程,但是实际上可以使用费曼图分析这个问题。由于只考虑经典情况,无需计算圈图。
“经典情况”到底指的是什么需要进一步说明:在经典情况下我们没有二次量子化,粒子图景的经典理论和场的图景的经典理论还不一样。
粒子图景下运动方程是关于各个粒子的位置和动量的,入射和出射外线没有任何限制。
场的图景下,我们求解系统的基本自由度(在这里是电场和谐振子坐标)的运动方程,得到场变量随时间的变化情况,即实际上在求解$\expval*{\phi}$,因此只能有一条出射外线,入射外线应当被当成外源。
在经典极限下这两种图景不会造成太大差别:同一张图的外线数目是固定的,在场的图景下,出射外线多了外源就少,由于我们要求场强满足$\phi / \hbar \ll 1$(但与此同时能标又没有高到多顶角图非常重要,从而圈图修正有必要计算),外源较少的过程是非常不重要的。

我们采用后一种图景,因为我们实际上就是在微扰求解\eqref{eq:x3-eq}。
将电磁场和带有$x^3$形势能的非线性振子耦合,则费曼图中应该有\autoref{fig:x3-vertex}和\autoref{fig:light-osci-couple}两种基本元件。
应当注意这里的短直线代表的是$x$的某个频率的分量,如果做量子化,就是一个谐振子模式。
这里的传播子并不代表谐振子的状态本身。这也就是\autoref{fig:light-osci-couple}顶角中只有一条短直线而不是两条的原因:它代表一个入射光子激发出一个谐振子模式,而不是谐振子整体吸收一个光子之后变成另一个状态。
由于电场和谐振子的耦合是完全线性的(并且由于谐振子是一个没有空间分布的点,实际上耦合项就是电偶极子能量),且我们关心的是“非线性介质中有哪些光学过程”,可以将电场暂时当成背景场,于是\autoref{fig:light-osci-couple}应该被\autoref{fig:external-field}取代。
例如,我们只需要分析一阶过程\autoref{fig:first-order-x3-external}就能知道\autoref{fig:first-order-x3-photon}的来源——如果$E$让$x$产生非线性响应,那就有光子分裂和合并的过程。

费曼规则可以很容易地写出:(我们认为频率为$\omega$的成分携带$\ee^{- \ii \omega t}$因子)
\begin{itemize}
    \item 传播子为
    \[
        \begin{tikzpicture}
            \begin{feynhand}
                \vertex (a) at (0, 0);
                \vertex (b) at (1, 0);
                \propag [plain, mom={$\omega$}] (a) to (b); 
            \end{feynhand}
        \end{tikzpicture} = \frac{\ii}{m (\omega^2 + \ii \gamma \omega - \omega_0^2)}.
    \]
    \item 顶角为
    \[
        \begin{tikzpicture}
            \begin{feynhand}
                \vertex (a) at (-1,-1); \vertex (b) at (1,-1); \vertex (c) at (0,1);
                \vertex (o) at (0,0); 
                \propag [plain] (a) to (o);
                \propag [plain] (b) to (o); 
                \propag [plain] (c) to (o);    
            \end{feynhand}
        \end{tikzpicture} = - \ii 2 m a \cdot 2\pi \delta(\sum \omega).
    \]
    注意正常情况下$x^3$相互作用要配一个$1/3!$的因子但是这里只有$1/3$,因此顶角实际上是$2ma$而不是$ma$。
    \item 外源为
    \[
        \begin{tikzpicture}
            \begin{feynhand}
                \vertex [crossdot] (a) at (0, 0){};
                \vertex (b) at (1, 0);
                \propag [plain, mom={$\omega$}] (a) to (b); 
            \end{feynhand}
        \end{tikzpicture} = \ii q E(\omega).
    \]
    请注意这里没有负号,而$x^3$是负号的,这是因为均匀电场会倾向于把谐振子拉向无穷远处而回复力则会将谐振子拉回来。
    本节采取的傅里叶变换约定为
    \[
        E(t) = \int \dd{\omega} E(\omega) \ee^{- \ii \omega t},
    \]
    没有加入$2\pi$是因为很多时候入射光并不是连续谱,而是离散的几个频域分量加起来。
\end{itemize}

\begin{figure}
    \centering
    \subfigure[$x^3$自相互作用顶角]{
        \begin{tikzpicture}
            \begin{feynhand}
                \vertex (a) at (-1,-1); \vertex (b) at (1,-1); \vertex (c) at (0,1);
                \vertex [dot] (o) at (0,0) {}; 
                \propag [plain] (a) to (o);
                \propag [plain] (b) to (o); 
                \propag [plain] (c) to (o);    
            \end{feynhand}
        \end{tikzpicture}
        \label{fig:x3-vertex}
    }
    \subfigure[光子激发出一个谐振子模式]{
        \begin{tikzpicture}
            \begin{feynhand}
                \vertex (a) at (-1, 1.5);
                \vertex (b) at (0, 1.5);
                \vertex (c) at (1, 1.5);
                \propag [photon] (a) to (b);
                \propag [plain] (b) to (c);
            \end{feynhand}
        \end{tikzpicture}
        \label{fig:light-osci-couple}
    }
    \subfigure[外源驱动谐振子,即\autoref{fig:light-osci-couple}中的光子被当成无动力学的外场后得到的图形]{
        \begin{tikzpicture}
            \begin{feynhand}
                \vertex [crossdot] (a) at (0, 0) {};
                \vertex (b) at (1, 0);
                \propag [plain] (a) to (b);
            \end{feynhand}
        \end{tikzpicture}
        \label{fig:external-field}
    }
    \caption{加入$\frac{1}{3} m a x^3$势能之后的费曼图元件}
\end{figure}

\begin{figure}
    \centering
    \subfigure[外场导致的响应的一阶近似]{
        \begin{tikzpicture}
            \begin{feynhand}
                \vertex [crossdot] (a) at (-1,-1) {};
                \vertex [crossdot] (b) at (1,-1) {}; 
                \vertex (c) at (0,1);
                \vertex (o) at (0,0) ; 
                \propag [plain, mom={$\omega_1$}] (a) to (o);
                \propag [plain, mom={$\omega_2$}] (b) to (o); 
                \propag [plain, mom={$\omega_1 + \omega_2$}] (o) to (c);
            \end{feynhand}
        \end{tikzpicture}
        \label{fig:first-order-x3-external}
    }
    \subfigure[\autoref{fig:first-order-x3-external}导致的非线性光学过程]{
        \begin{tikzpicture}
            \begin{feynhand}
                \vertex (a0) at (-1.5, -1.5);
                \vertex (a) at (-0.5,-0.5);
                \vertex (b0) at (1.5, -1.5);
                \vertex (b) at (0.5,-0.5); 
                \vertex (c) at (0,0.5);
                \vertex (c0) at (0, 1.5);
                \vertex (o) at (0,0) ; 
                \propag [photon, mom={$\omega_1$}] (a0) to (a);
                \propag [plain] (a) to (o);
                \propag [photon, mom={$\omega_2$}] (b0) to (b);
                \propag [plain] (b) to (o); 
                \propag [plain] (o) to (c);
                \propag [photon, mom={$\omega_1 + \omega_2$}] (c) to (c0);
            \end{feynhand}
        \end{tikzpicture}
        \label{fig:first-order-x3-photon}
    }
    \caption{一阶过程}
    \label{fig:x3-first-order}
\end{figure}

据此,线性响应(零阶,没有发生任何非线性效应)为
\begin{equation}
    x_1(\omega) = \ii q E(\omega) \frac{\ii}{m (\omega^2 + \ii \gamma \omega - \omega_0^2)} = \frac{q / m}{\omega_0^2 - \omega^2 - \ii \gamma \omega} E(\omega).
\end{equation}
一阶过程(即\autoref{fig:first-order-x3-external})给出如下修正:
\begin{equation}
    \begin{aligned}
        x_2(\omega) &= \frac{1}{2} \int \omega_1 \int \omega_2 (\ii q E(\omega_1)) (\ii q E(\omega_2)) \frac{\ii}{m(\omega_1^2 + \ii \gamma \omega_1 - \omega_0^2)} \frac{\ii}{m(\omega_2^2 + \ii \gamma \omega_2 - \omega_0^2)} \\ 
        &\quad \quad \times (-\ii 2 m a) \frac{\ii}{(m(\omega_1 + \omega_2)^2 + \ii \gamma (\omega_1 + \omega_2) - \omega_0^2)} 2\pi \delta(\omega_1 + \omega_2 - \omega) \\
        &= \int \dd{\omega_1} \int \dd{\omega_2} \frac{a (q / m)^2}{(\omega_1^2 + \ii \gamma \omega_1 - \omega_0^2) (\omega_2^2 + \ii \gamma \omega_2 - \omega_0^2) (\omega^2 + \ii \gamma \omega - \omega_0^2)}  \\
        &\quad \quad \times 2\pi \delta(\omega_1 + \omega_2 - \omega) \times E(\omega_1) E(\omega_2).
    \end{aligned} 
    \label{eq:continuous-x3-first-order}
\end{equation}
这里需要注意一点:\autoref{fig:first-order-x3-external}中外场出现了两次,而
\[
    E(t)^2 = \int \dd{\omega_1} \int \dd{\omega_2} E(\omega_1) E(\omega_2) \ee^{-\ii (\omega_1 + \omega_2) t},
\]
如果$\omega_1 \neq \omega_2$那么$E(\omega_1) E(\omega_2)$项实际上会被求和两次;同样,此时\eqref{eq:continuous-x3-first-order}中的$E(\omega_1) E(\omega_2)$项也会被求和两次。
直观地看,外场是给定的而不能随意交换,所以\autoref{fig:first-order-x3-external}中的两个外场从左到右为$\omega_1$和$\omega_2$的图和从左到右为$\omega_2$和$\omega_1$的图虽然给出一样的结果,但是是两张图,不能看成一张图,它们加起来会导致因子$2$出现。
如果我们令$E(t)$实际上只有两个频率分量,这一点会显得尤其明显。
$\omega_1 = \omega_2$的情况中没有因子$2$,我们常将这样的过程称为\emph{简并的}。

现在我们采取更加常规的,微扰求解微分方程的做法。费曼图计算已经告诉我们主要的光学过程来自\autoref{fig:first-order-x3-external}。
因此,我们将输入的$E$设置为
\begin{equation}
    E = E_1 (\ee^{\ii \omega_1 t} + \ee^{-\ii \omega_1 t}) + E_2 (\ee^{\ii \omega_2 t} + \ee^{-\ii \omega_2 t}), 
    \label{eq:input-two-freq-e}
\end{equation}
做展开
\begin{equation}
    x = x_1 + x_2 + \ldots, \quad x_n \sim E^{n },
\end{equation}
并记
\begin{equation}
    x_i = \sum_n x_i(\omega_n) + \text{c.c.}.
\end{equation}
线性项$x_1$由
\[
    m \ddot{x}_1 + m \gamma \dot{x_1} +  m \omega_0^2 x_1 = q E
\]
给出,为
\begin{equation}
    x_1(\omega_1) = \frac{(q/m) E_i}{\omega_0^2 } \ee^{-\ii \omega_n t},
\end{equation}
二阶项由
% TODO:懒得写了
这些项分别称为:
\begin{itemize}
    \item \concept{和频(SFG, sum frequency generation)},
    \item \concept{差频(DFG, difference frequency generation)},
    \item \concept{倍频(SHG, second harmonic generation)},
    \item \concept{光学整流(OR, optic rectification)}(因为输入交流波而得到直流波,那当然是整流了)。
\end{itemize}
倍频是和频的特殊情况,光学整流是差频的特殊情况。当然,直接将\eqref{eq:input-two-freq-e}代入\eqref{eq:continuous-x3-first-order}也能够得到这些过程。

乍一看,费曼图方法不仅能够得到两个光子合并为一个光子的过程,也能够得到一个光子分裂成两个光子的过程(所谓的\concept{SPDC过程(Spontaneous parametric down-conversion)},也称为\concept{OPG过程(Optical parametric generation)}),但是我们后面将看到,微分方程方法似乎只能给出两个光子合并为一个光子的过程——如果我们在$E$中放入只有一个频率$\omega_0$的波,那么非线性效应似乎只会给出$\omega=0$和$\omega=2\omega_0$两种波。
但是其实这里并没有矛盾:SPDC过程需要两条出射外线;如前所述,我们采用场的图景,由于我们采用微分方程的写法,即从$E$求解$x$(其实是$\expval*{x}$),然后用“谐振子位置的偏离导致极化电场产生”计算总电场的变化,对$x$也要采用场的图景,所以的确只应该考虑只有一条外线的费曼图。
这暗示着SPDC过程实际上是非常弱的(本该如此,和频过程的振幅正比于$E^2$而SPDC过程的振幅正比于$E$),因此在经典图景下这个过程根本就不会出现。%
\footnote{
    从这里也可以看出光学中量子理论的重要性,即使我们讨论的能标自始至终都没有高到让只有QED才有的过程(如四光子等效相互作用)出现。
    经典理论对电磁波的描述是非常粗糙的:如果我们要描述一个物理状态中有两种不同频率的光子,应该怎么做?
    在经典理论中只有一种方法:设
    \[
        \vb*{E} = \vb*{E}_1 \ee^{\ii \omega_1 t} + \vb*{E}_2 \ee^{\ii \omega_2 t} + \text{c.c.}.
    \]
    现在如果要将从一个单频波到以上状态的过程画成费曼图,由于只能画一条外线的图,势必只能画出$\omega_0 \to \omega_1$和$\omega_0 \to \omega_2$两个图,然后能量就不守恒了。
}%

这并不是说SPDC过程——或者说OPG过程——在适用经典近似的体系中完全看不到,因为我们可以在OPG过程后面再放一个DFG过程。
DFG过程也可以称为\concept{OPA过程(Optical parametric amplification)},因为它让入射的两束光的一束变弱而另一束变强。
例如,设我们希望将一束频率为$\omega_1$的光分裂成两束光,频率分别是$\omega_2$和$\omega_3$。
我们可以将一个有二阶非线性极化的光学晶体放在一个内壁对频率为$\omega_2$和$\omega_3$的反射率很高的谐振腔中。
按照后面会提到的\eqref{eq:sfg-intensity},如果有相位匹配条件成立,那么$\omega_1 \to \omega_2 + \omega_3$的OPG过程转化效率很高(在那里是SFG过程效率很高,这里就是OPG过程转化效率很高),于是产生足够强的$\omega_2$光束和$\omega_3$光束,这些光束被谐振腔反射回来,回到非线性晶体内部,于是发生很强的OPA过程。
因此,我们仅仅通过一束单频入射光就得到了两束不同频率的出射光。
% TODO:怎么定量算?
在经典理论中OPA过程是允许的,因此时间反演对称性并没有丧失:的的确确可以有光子的分裂。
但是,经典理论中所有电磁波模式上的光子都是足够多的,因此从“完全没有光子”到“有一个光子”的过程在经典理论中无法被描述。这就是OPG过程看不到的原因。
换而言之,经典理论中的光子分裂,即OPA,不仅需要入射的泵浦光,还需要一个(直观上看,引导泵浦光分裂成哪些频率的光的)\concept{种子光}。
一旦种子光入射了,随着光的传播它会增强。

\begin{figure}
    \centering
    \subfigure[外场导致的响应的二阶近似]{
        \begin{tikzpicture}
            \begin{feynhand}
                \vertex [crossdot] (a) at (-1,-1) {};
                \vertex [crossdot] (b) at (1,-1) {}; 
                \vertex (c) at (0,1);
                \vertex (o) at (0,0) ;
                \vertex [crossdot] (e) at (1, 2) {};
                \vertex  (f) at (-1, 2) ; 
                \propag [plain, mom={$\omega_1$}] (a) to (o);
                \propag [plain, mom={$\omega_2$}] (b) to (o); 
                \propag [plain, mom={$\omega_1 + \omega_2$}] (o) to (c);
                \propag [plain, mom={$\omega_3$}] (e) to (c);
                \propag [plain, mom={$\omega_1 + \omega_2 + \omega_3$}] (c) to (f);
            \end{feynhand}
        \end{tikzpicture}
        \label{fig:second-order-x3-external}
    }
    \subfigure[\autoref{fig:second-order-x3-external}导致的非线性光学过程]{
        \begin{tikzpicture}
            \begin{feynhand}
                \vertex (a0) at (-1.5, -1.5);
                \vertex (a) at (-0.5,-0.5);
                \vertex (b0) at (1.5, -1.5);
                \vertex (b) at (0.5,-0.5); 
                \vertex (c) at (0,0.5);
                \vertex (o) at (0,0) ; 
                \vertex (d) at (0.5, 1);
                \vertex (e) at (-0.5, 1);
                \vertex (d0) at (1.5, 2);
                \vertex (e0) at (-1.5, 2);
                \propag [photon, mom={$\omega_1$}] (a0) to (a);
                \propag [plain] (a) to (o);
                \propag [photon, mom={$\omega_2$}] (b0) to (b);
                \propag [plain] (b) to (o); 
                \propag [plain] (o) to (c);
                \propag [plain] (c) to (e);
                \propag [plain] (c) to (d);
                \propag [photon, mom={$\omega_3$}] (d0) to (d);
                \propag [photon, mom={$\omega_1 + \omega_2 + \omega_3$}] (e) to (e0);
            \end{feynhand}
        \end{tikzpicture}
        \label{fig:second-order-x3-photon}
    }
    \caption{二阶过程}
    \label{fig:x3-second-order}
\end{figure}

还可以进一步往上计算微扰。例如,二阶微扰将给出\autoref{fig:x3-second-order}。
这里给出的四光子相互作用和\autoref{fig:first-order-x3-photon}产生的等效四光子相互作用不同,后者需要两个光子先合并,产生的光子传播一会,然后再和另一个光子合并。
\autoref{fig:second-order-x3-photon}给出的四光子相互作用是直截了当的。

我们来对各阶微扰的量级做一个估计。如果$\omega$和$\omega_0$比较接近,那么微扰论根本就不适用:此时共振发生,$x$是非常大的,可能高阶修正比低阶修正还大。
此时需要从头做光和物质耦合的计算而不能使用加入微弱非线性因素的振子模型。
如果$\omega$远大于$\omega_0$,我们将得到等离子体,此时彼此无关的、振幅不大的振子的图像更加失效了,可能晶格都已经被破坏了,电子的运动状况主要受电场控制。
在这两种情况下本节给出的非线性振子模型都不适用。(等离子体情况下有另一个非线性来源,即协变导数的输运项;见后文)
对$\omega \ll \omega_0$的情况,线性响应的振幅的量级为
\[
    x_1 \sim \frac{(q/m) E}{\omega_0^2},
\]
而
\[
    x_2 \sim \frac{a (q/m)^2 E^2}{\omega_0^6},
\]
因此
\begin{equation}
    \frac{x_2}{x_1} \sim \frac{a q E}{m \omega_0^4}.
\end{equation}
设原子对电子的束缚电场的量级为$E_\text{atom}$,则总位移$x$的振幅可以估计为
\[
    q E_\text{atom} \sim m \omega_0^2 x .
\]
$x$的量级具体有多大是不确定的,它包括没有外加电场时由$q E_\text{atom}$做回复力的内禀振荡,线性响应$x_1$和非线性响应$x_2$。
我们不妨采取一个非常极端的假设,认为线性回复力和非线性回复力已经一样大了(如果非线性回复力很小,那么当然只需要计算一阶图),此时
\[
    m \omega_0^2 x \sim m a x^2,
\]
于是
\[
    q E_\text{atom} \sim m \omega_0^2 \frac{\omega_0^2}{a},
\]
最后
\begin{equation}
    \frac{x_2}{x_1} \sim \frac{E}{E_\text{atom}}.
\end{equation}
通常原子内部电场的数量级为\SI{3e8}{V/m},因此即使认为非线性回复力和线性回复力一样大,一般来说$x_2$也远小于$x_1$,即此时非线性极化相对于线性极化来说还是不大的。
介质的光学性能由极化给出,和回复力没有直接关系,因此非线性极化一般来说总是比非线性极化小得多的。
类似地实际上可以证明
\begin{equation}
    \frac{x_{n+1}}{x_n} \sim \frac{E}{E_\text{atom}}.
\end{equation}

\subsubsection{三阶非线性极化}

\subsection{自由电子气的输运项}

此时的非线性效应来自$(\vb*{v} \cdot \grad) \vb*{v}$。
我们写下电荷的运动方程:
\begin{equation}
    \pdv{\vb*{v}}{t} + (\vb*{v} \cdot \grad) \vb*{v} = q \vb*{E},
\end{equation}

金属表面也能够产生二次谐波

\section{非线性极化的量子理论}

\autoref{sec:classical-oscillator}中我们以光子和经典谐振子的振动模式为基本自由度,通过为谐振子引入一个非二次型的势来得到非线性效应。
这种做法在量子理论中当然也是成立的,并且此时谐振子的振动模式真的就像一个个粒子一样。
然而,应当注意,这种“光子和谐振子振动模式相互作用,谐振子的振动模式通过非简谐的势能相互作用”的理论并不是最方便的,因为无论为振子——在这里实际上就是原子中的电子——引入怎样的非线性势,我们有的都是一个单体问题,因此最终电子自身的能谱都可以被一系列能级完整描述,并且这些能级是比较容易算出来的。
光子与电子碰撞会让电子从一个能级跳到另一个能级,并且能量守恒条件——其中电子的能量由已经经过非二次型势能修正的能级给出——必须成立。
这意味着,首先考虑非简谐的势能的作用,计算出电子能级,然后考虑光子让电子在这些能级之间跃迁,是更加方便的。
这个图景在经典理论中无法使用,因为此时的光子吸收相互作用顶角有一条入射线,两条出射线,从而一个有光子出射的过程一定有多条出射线(至少一条光子线,以及一条雷打不动的电子出射线),从而无法在经典理论中表达。

在基于电子能级的图景中,非线性光学效应来自高阶微扰论,因为相互作用顶角上连接了两条电子线和一条光子线,从而,一张费曼图中可以有数量任意的入射和出射光子线,另一方面,电子线除了和光子相互作用以外,没有别的相互作用。
这和基于非线性振子的图景非常不同,在后者中一个光子只能连接到一条代表振子振动模式(而不是电子本身)的内线上,即光子到振子振动模式的转换始终是线性的,但是振子振动模式之间可以碰撞,从而有非线性过程。
遮去电子线,将电子的非线性响应表示为宏观的“极化”,以上两种图景统一变成了\autoref{sec:non-linear-maxwell}中的图景。
电子与光场通过电偶极跃迁耦合,在耦合哈密顿量中电场是线性的,从而,电子电偶极矩的$n$次方就对应一个$n$光子顶角。

\subsection{纯态单电子系统的高阶微扰论}

\subsubsection{电子状态的含时微扰论}

本节考虑一个受到经典电磁场扰动的单电子系统。设系统一开始位于某个态$\ket*{g}$上,电磁场扰动会让它在各个瞬时的状态变得不确定起来。
我们用$m, n$等标记电子能级,用$p, q$等标记光子模式,并做傅里叶分解
\begin{equation}
    \vb*{E}(t) = \sum_p \vb*{E}(\omega_p) \ee^{- \ii \omega_p t}.
\end{equation}

我们首先计算电场扰动下的电子波函数$\ket*{\psi}$。单光子吸收过程$\braket*{m}{\psi^{(1)}}$为
\begin{equation}
    \begin{aligned}
        \begin{gathered}
            \begin{tikzpicture}
                \begin{feynhand}
                    \vertex (a) at (-1.3, 0) {$g$};
                    \vertex (o) at (0, 0);
                    \vertex (b) at (1.3, 0) {$m$};
                    \vertex (c) at (-0.5, 0.87) ;
                    
                    \propag[fermion] (a) to (o);
                    \propag[fermion] (o) to (b);
                    \propag[photon, mom={$p$}] (c) to (o);
                \end{feynhand}
            \end{tikzpicture}
        \end{gathered} &= \frac{1}{\hbar} \sum_p \frac{1}{\omega_g + \omega_p - \omega_m} \mel{m}{- \vb*{d} \cdot \vb*{E}(\omega_p)}{g} \ee^{- \ii (\omega_g + \omega_p - \omega_m) t} \\
        &= \frac{1}{\hbar} \sum_p \frac{\vb*{d}_{mg} \cdot \vb*{E}(\omega_p)}{\omega_{mg} - \omega_p} \ee^{\ii (\omega_{mg} - \omega_p) t},
    \end{aligned}
\end{equation}
双光子吸收过程$\braket*{n}{\psi^{(2)}}$为%
\footnote{
    这里使用的实际上是time ordered perturbation theory, 但是其效果和covariant perturbation theory基本上是一样的;传播子中的$\omega_g + \omega_p$可以看成$m$模式通过能量守恒定律计算出的能量,而$\omega_m$可以看成从能谱中读出的$m$模式的能量,整个传播子和$\omega - \vb*{p}^2 / 2m$是差不多的。
    我们可以认为$g$是在壳的而其它所有模式——包括$n$——都是离壳的。
}%
\begin{equation}
    \begin{aligned}
        \begin{gathered}
            \begin{tikzpicture}
                \begin{feynhand}
                    \vertex (a) at (-1.3, 0) {$g$};
                    \vertex (o1) at (0, 0);
                    \vertex (o2) at (1.7, 0);
                    \vertex (b) at (3.0, 0) {$n$};
                    \vertex (c) at (-0.5, 0.87) ;
                    \vertex (d) at (1.2, 0.87);
                    
                    \propag[fermion] (a) to (o1);
                    \propag[fermion] (o1) to[edge label={$m$}] (o2);
                    \propag[fermion] (o2) to (b);
                    \propag[photon, mom={$p$}] (c) to (o1);
                    \propag[photon, mom={$q$}] (d) to (o2);
                \end{feynhand}
            \end{tikzpicture}
        \end{gathered} &= \frac{1}{\hbar^2} \sum_{p, q} \sum_m \frac{1}{\omega_g + \omega_p + \omega_q - \omega_n} \mel{n}{- \vb*{d} \cdot \vb*{E}(\omega_p)}{m} \\ 
        &\quad \quad \times \frac{1}{\omega_g + \omega_p - \omega_m} \mel{m}{- \vb*{d} \cdot \vb*{E}(\omega_p)}{g} \ee^{- \ii (\omega_g + \omega_p + \omega_q - \omega_n) t} \\
        &= \frac{1}{\hbar^2} \sum_{p, q} \sum_m \frac{(\vb*{d}_{nm} \cdot \vb*{E}(\omega_q)) (\vb*{d}_{mg} \cdot \vb*{E}(\omega_p))}{(\omega_{ng} - \omega_q - \omega_p) (\omega_{mg} - \omega_p)} \ee^{\ii (\omega_{ng} - \omega_p - \omega_q) t},
    \end{aligned}
\end{equation}
同理还能够得到三光子吸收过程$\braket*{l}{\psi}$为
\begin{equation}
    \begin{aligned}
        &\quad \begin{gathered}
            \begin{tikzpicture}
                \begin{feynhand}
                    \vertex (a) at (-1.3, 0) {$g$};
                    \vertex (o1) at (0, 0);
                    \vertex (o2) at (1.7, 0);
                    \vertex (o3) at (3.4, 0);
                    \vertex (b) at (4.7, 0) {$l$};
                    \vertex (c) at (-0.5, 0.87) ;
                    \vertex (d) at (1.2, 0.87);
                    \vertex (e) at (2.9, 0.87);
                    
                    \propag[fermion] (a) to (o1);
                    \propag[fermion] (o1) to[edge label={$m$}] (o2);
                    \propag[fermion] (o2) to[edge label={$n$}] (o3);
                    \propag[fermion] (o3) to (b);
                    \propag[photon, mom={$p$}] (c) to (o1);
                    \propag[photon, mom={$q$}] (d) to (o2);
                    \propag[photon, mom={$r$}] (e) to (o3);
                \end{feynhand}
            \end{tikzpicture}
        \end{gathered} \\
        &= \frac{1}{\hbar^3} \sum_{p, q, r} \sum_{m, n} \frac{(\vb*{d}_{ln} \cdot \vb*{E}(\omega_r)) (\vb*{d}_{nm} \cdot \vb*{E}(\omega_q)) (\vb*{d}_{mg} \cdot \vb*{E}(\omega_p))}{(\omega_{lg} - \omega_p - \omega_q - \omega_r) (\omega_{ng} - \omega_q - \omega_p) (\omega_{mg} - \omega_p)} \ee^{\ii (\omega_{lg} - \omega_p - \omega_q - \omega_r) t}.
    \end{aligned}
\end{equation}

以上三个过程都是严格按照电偶极辐射哈密顿量计算出来的;实际的系统中除了电偶极辐射以外,还有各种各样的噪声扰动。
我们采取一种唯象的做法,认为能级之间的跃迁存在一个弛豫,即传播子中需要加入$\ii \gamma_{mn}$,为了简化书写,令
\begin{equation}
    \omega_{mn} = \omega_{m} - \omega_{n} - \ii \gamma_{mn},
\end{equation}
从而考虑弛豫这个非幺正因素之后,以上三个过程的表达式仍然是正确的,但是此时$\omega_{mn}$具有虚部,其复共轭不等于它本身。
这在计算期望值$\mel*{\psi}{\cdot}{\psi}$时非常重要。
我们还假定
\begin{equation}
    \gamma_{mn} = \gamma_{nm},
\end{equation}
这个假设的合理性需要在\autoref{sec:electron-density-matrix}中看到。

在经典理论中是画不出以上三个过程的,原因是显然的:有多个输出线。
我们此处用实线表示电子而不是电子的振动模式,从而,虽然电场和电子的耦合中电场是线性的,电子线却有两条。
二次量子化之后电场和电子的耦合实际上形如$A_\mu \bar{\psi} \psi$,因此的确有两条电子线。

\subsubsection{极化矢量}

耦合项$- \vb*{d} \cdot \vb*{E}$会导致电子受到电场影响,自然也会导致电场被电子激发出来。
本节讨论经典电磁场,从而不能真的用光子入射散射等概念计算等效光子-光子顶角。
经典电磁场中介质极化是新的波源,而极化矢量为
\begin{equation}
    \vb*{P} = N \expval*{\vb*{d}} = N \mel*{\psi}{\vb*{d}}{\psi},
\end{equation}
将$\vb*{P}$代入介质中的麦克斯韦方程,即可得到介质中光的行为。
我们这里直接计算$\vb*{d}$的期望值,以得到极化矢量,这个做法的合理性在于我们本质上还是在积掉电子,即在配分函数中保留电磁场不动,积掉电子场,计算(两边是基态的)关联函数。

这样,极化矢量对电场的一阶响应为
\begin{equation}
   \begin{aligned}
    &\quad \mel*{\psi^{(0)}}{\vb*{d}}{\psi^{(1)}} + \mel*{\psi^{(1)}}{\vb*{d}}{\psi^{(0)}} \\
    &= \sum_m \ee^{- \ii \omega_m t} \vb*{d}_{gm} \ee^{\ii \omega_g t} \frac{1}{\hbar} \sum_p \frac{\vb*{d}_{mg} \cdot \vb*{E}(\omega_p)}{\omega_{mg} - \omega_p} \ee^{\ii (\omega_{mg} - \omega_p) t} + \text{h.c.} \\
    &= \frac{1}{\hbar} \sum_{m, p} \left( \frac{ \vb*{d}_{gm} (\vb*{d}_{mg} \cdot \vb*{E}(\omega_p))}{\omega_{mg} - \omega_p} \ee^{- \ii \omega_p t} + \frac{ \vb*{d}_{mg} (\vb*{d}_{gm} \cdot \vb*{E}(\omega_p)^* )}{\omega_{mg}^* - \omega_p} \ee^{\ii \omega_p t} \right) \\
    &= \frac{1}{\hbar} \sum_{m, p} \left( \frac{ \vb*{d}_{gm} (\vb*{d}_{mg} \cdot \vb*{E}(\omega_p))}{\omega_{mg} - \omega_p} \ee^{- \ii \omega_p t} + \frac{ \vb*{d}_{mg} (\vb*{d}_{gm} \cdot \vb*{E}(\omega_p) )}{\omega_{mg}^* + \omega_p} \ee^{- \ii \omega_p t} \right), 
   \end{aligned} 
   \label{eq:dipole-first-perturbation}
\end{equation}
其中第三个等号将第二项中的$\omega_p$换成了$-\omega_p$。更高阶的响应也可以用类似的方式获得。

直接展开计算是非常繁琐的,不过我们会发现\eqref{eq:dipole-first-perturbation}实际上可以用费曼图表示:
\begin{equation}
    \begin{gathered}
        \begin{tikzpicture}
            \begin{feynhand}
                \vertex (g1) at (-0.25, 0) {$g$};
                \vertex (g2) at (0.25, 0) {$g$};
                \vertex (t1) at (-0.25, 3);
                \vertex (t2) at (0.25, 3);
                \propag[plain] (t1) to[out=90, in=90] (t2);

                \vertex (v1) at (-0.25, 1) ;
                \vertex (l1) at (-1.55, 0.5) {$\omega_p$};
                \propag[extphoton] (l1) to (v1);
                \propag[plain] (g1) to (v1) ;

                \vertex (o) at (-0.25, 2);
                \vertex (e) at (-1.55, 2.5) {$\omega_p$};
                \propag[outphoton] (o) to (e);
                \propag[plain] (v1) to[edge label={$m$}] (o);

                \propag[plain] (o) to (t1);

                \propag[plain] (t2) to (g2);
            \end{feynhand}
        \end{tikzpicture}
    \end{gathered} = \frac{1}{\hbar} \sum_{m, p} \frac{ \vb*{d}_{gm} (\vb*{d}_{mg} \cdot \vb*{E}(\omega_p))}{\omega_{mg} - \omega_p} \ee^{- \ii \omega_p t} ,
    \label{eq:left-in-one-order-perturbation}
\end{equation}
以及
\begin{equation}
    \begin{gathered}
        \begin{tikzpicture}
            \begin{feynhand}
                \vertex (g1) at (-0.25, 0) {$g$};
                \vertex (g2) at (0.25, 0) {$g$};
                \vertex (t1) at (-0.25, 3);
                \vertex (t2) at (0.25, 3);
                \propag[plain] (t1) to[out=90, in=90] (t2);

                \vertex (v1) at (0.25, 1) ;
                \vertex (l1) at (1.55, 0.5) {$\omega_p$};
                \propag[extphoton] (l1) to (v1);
                \propag[plain] (g2) to (v1) ;

                \propag[plain] (v1) to[edge label'={$m$}] (t2);
                \propag[plain] (t1) to (o);

                \vertex (o) at (-0.25, 2);
                \vertex (e) at (-1.55, 2.5) {$\omega_p$};
                \propag[outphoton] (o) to (e);

                \propag[plain] (o) to (g1);
            \end{feynhand}
        \end{tikzpicture}
    \end{gathered} = \frac{1}{\hbar} \sum_{m, p} \frac{ \vb*{d}_{mg} (\vb*{d}_{gm} \cdot \vb*{E}(\omega_p) )}{\omega_{mg}^* + \omega_p} \ee^{- \ii \omega_p t}.
    \label{eq:right-in-one-order-perturbation}
\end{equation}
更高阶的响应也可以用类似的方式用费曼图计算。可以总结出如下规则:
\begin{itemize}
    \item 电子线包括从下而上的左侧线和从上而下的右侧线,即所谓\emph{double sided Feynman diagram};
    \item 在左侧线最顶端放置$\vb*{\mu}$算符,使用一根波浪线代表它产生电磁场;
    \item 由于顶角总是有两条电子线,传播子和顶角可以合并成一个组件。在左侧线上,从下到上第$i$个顶角给出
    \begin{equation}
        \begin{gathered}
            \begin{tikzpicture}
                \begin{feynhand}
                    \vertex (g) at (0, 0);
                    \vertex (t) at (0, 2);
                    \vertex (l) at (-1, 0.5) {$\omega_i$};
                    \vertex (v) at (0, 1);
                    
                    \propag[plain] (g) to[edge label={$m$}] (v) ;
                    \propag[plain] (v) to[edge label={$n$}] (t);
                    \propag[extphoton] (l) to (v);
                \end{feynhand}
            \end{tikzpicture}
        \end{gathered} = \frac{\vb*{d}_{nm} \cdot \vb*{E}(\omega_p) }{\omega_{ng} - \sum_{j=1}^i \omega_j},
        \label{eq:feynman-diagram-left}
    \end{equation}
    而在右侧线上,从下到上第$i$个顶角给出
    \begin{equation}
        \begin{gathered}
            \begin{tikzpicture}
                \begin{feynhand}
                    \vertex (g) at (0, 0);
                    \vertex (t) at (0, 2);
                    \vertex (l) at (1, 0.5) {$\omega_i$};
                    \vertex (v) at (0, 1);
                    
                    \propag[plain] (g) to[edge label={$m$}] (v) ;
                    \propag[plain] (v) to[edge label={$n$}] (t);
                    \propag[extphoton] (l) to (v);
                \end{feynhand}
            \end{tikzpicture}
        \end{gathered} = \frac{\vb*{d}_{mn} \cdot \vb*{E}(\omega_p) }{\omega_{ng}^* + \sum_{j=1}^i \omega_j}.
        \label{eq:feynman-diagram-right}
    \end{equation}
\end{itemize}
为了区分外场和$\mel*{\psi}{\vb*{d}}{\psi}$,我们用直线表示前者而用波浪线代表后者,虽然在凝聚态场论中我们通常用带有$\otimes$的波浪线代表前者。

\subsubsection{双侧费曼图和凝聚态场论中的费曼图}

以上费曼图的规则看起来非常奇特,但是实际上可以将它理解为凝聚态场论中的图。
它和一般的凝聚态场论的不同之处在于,本节讨论的带有弛豫的模型破缺时间反演不变性,从而传播子的书写有一些需要注意的地方。
大体上说我们是要计算
\[
    \expval*{{c}^\dagger_m(t) \vb*{\mu}_{mn} c_n(t) }, \quad \ket*{\Omega} = c^\dagger_g(-\infty) \ket*{0}.
\]
我们有两种做法,其中之一是直接计算$\mel*{0}{c_g(-\infty) {c}^\dagger_m(t) \vb*{\mu}_{mn} c_n(t) c^\dagger_g(-\infty)}{0}$,对应的费曼图形如
\begin{equation}
    \sum_{m, n} \vb*{\mu}_{mn} \begin{gathered}
        \begin{tikzpicture}
            \begin{feynhand}
                \vertex [grayblob] (o) at (0, 0) {};
                \vertex (a) at (-1, 1) {$g, -\infty$};
                \vertex (b) at (-1, -1) {$g, -\infty$};
                \vertex (c) at (1, 1) {$m, t$};
                \vertex (d) at (1, -1) {$n, t$};
    
                \propag[fermion] (a) to (o);
                \propag[fermion] (o) to (b);
                \propag[fermion] (o) to (c);
                \propag[fermion] (d) to (o);
            \end{feynhand}
        \end{tikzpicture}
    \end{gathered},
    \label{eq:double-sided-feynman-diagram-in-condensed}
\end{equation}
如果电磁场也是量子化的,上图实际上就是
\[
    \begin{gathered}
        \begin{tikzpicture}
            \begin{feynhand}
                \vertex [grayblob] (o) at (0, 0) {};
                \vertex (a) at (-1, 1) {$g, -\infty$};
                \vertex (b) at (-1, -1) {$g, -\infty$};
                \vertex (e) at (1.75, 0);
                \vertex (f) at (2.75, 0);
    
                \propag[fermion] (a) to (o);
                \propag[fermion] (o) to (b);
                \propag[fermion] (o) to[out=45, in=135] (e);
                \propag[fermion] (e) to[in=315, out=225] (o);
                \propag[photon] (e) to (f);
            \end{feynhand}
        \end{tikzpicture}
    \end{gathered}.
\]
在本节中我们无需计算上图最右边的产生光子的顶角,光子产生是在非线性麦克斯韦方程中通过极化矢量引入的。

\eqref{eq:double-sided-feynman-diagram-in-condensed}中的$g$外线实际上是系统基态的一部分,因此我们不能将两个$g$外线连接起来,否则得到的实际上是真空气泡图。因此,我们只需要考虑下图
\begin{equation}
    \sum_{m, n} \vb*{\mu}_{mn} \begin{gathered}
        \begin{tikzpicture}
            \begin{feynhand}
                \vertex (a) at (-2, 1) {$g, -\infty$};
                \vertex (b) at (-2, 0) {$g, -\infty$};
                \vertex (c) at (2, 1) {$m, t$};
                \vertex (d) at (2, 0) {$n, t$};
                \vertex[grayblob] (o1) at (0, 1) {};
                \vertex[grayblob] (o2) at (0, 0) {};
                
                \propag[fermion] (a) to (o1);
                \propag[fermion] (o1) to (c);
                \propag[fermion] (o2) to (b);
                \propag[fermion] (d) to (o2);
            \end{feynhand}
        \end{tikzpicture}
    \end{gathered}
\end{equation}
即可,其中圆圈内是一个或多个光子吸收或发射过程。由于本节的系统时间反演对称性破缺,因为存在弛豫项,图
\[
    \begin{gathered}
        \begin{tikzpicture}
            \begin{feynhand}
                \vertex (a) at (-2, 1) {$g, -\infty$};
                \vertex (c) at (2, 1) {$m, t$};
                \vertex[grayblob] (o1) at (0, 1) {};
                
                \propag[fermion] (a) to (o1);
                \propag[fermion] (o1) to (c);
            \end{feynhand}
        \end{tikzpicture}
    \end{gathered}
\]
中的频域电子传播子是

另一种理解方式是将系统看成一个多体系统,此时

总之,$\mel{\psi}{\vb*{d}}{\psi}$和通常意义上的格林函数是不同的:后者是做了编时乘积的,各个算符所在的时间从$-\infty$演化到$\infty$,而前者中,各个算符从右往左位于的时间先从$-\infty$演化到我们要计算的时间点,然后再演化回$-\infty$,这导致一个重要的结果:从我们要计算的时间点演化回$-\infty$时,由于做了时间反演变换,弛豫参量需要加上一个负号,因此如果我们将$-\infty$演化到我们要计算的时间点的算符序列画在左边的电子线上,而将从我们要计算的时间点演化到$-\infty$的算符序列画在右边的电子线上,那么右边的传播子中的弛豫参量要差一个负号。
从\eqref{eq:dipole-first-perturbation},我们也将弛豫参量的负号理解为是$\bra{\psi}$相比于$\ket*{\psi}$取了复共轭而产生的。
对弛豫参量的负号的两种理解是一致的,因为时间反演和复共轭紧密相关。
基于以上考虑,把这些费曼图当成凝聚态场论的图理解,可以写出
\begin{equation}
    \begin{gathered}
        \begin{tikzpicture}
            \begin{feynhand}
                \vertex (g1) at (-0.25, 0) {$g$};
                \vertex (g2) at (0.25, 0) {$g$};
                \vertex (t1) at (-0.25, 3);
                \vertex (t2) at (0.25, 3);
                \propag[plain] (t1) to[out=90, in=90] (t2);

                \vertex (v1) at (-0.25, 1) ;
                \vertex (l1) at (-1.55, 0.5) {$\omega_p$};
                \propag[extphoton] (l1) to (v1);
                \propag[plain] (g1) to (v1) ;

                \vertex (o) at (-0.25, 2);
                \vertex (e) at (-1.55, 2.5) {$\omega_p$};
                \propag[outphoton] (o) to (e);
                \propag[plain] (v1) to[edge label={$m$}] (o);

                \propag[plain] (o) to (t1);

                \propag[plain] (t2) to (g2);
            \end{feynhand}
        \end{tikzpicture}
    \end{gathered} = \sum_p \vb*{d}_{gm} \ee^{-\ii \omega_p t} \sum_{m} \frac{\ii}{\omega_p + \omega_g - \omega_m + \ii \gamma_{mg}} \frac{(\ii \vb*{d}_{mg} \cdot \vb*{E}(\omega_p))}{\hbar},
\end{equation}
和\eqref{eq:left-in-one-order-perturbation}是一致的。同样我们使用凝聚态场论的理解方式计算入射光子在右侧电子线的图,有
\begin{equation}
    \begin{gathered}
        \begin{tikzpicture}
            \begin{feynhand}
                \vertex (g1) at (-0.25, 0) {$g$};
                \vertex (g2) at (0.25, 0) {$g$};
                \vertex (t1) at (-0.25, 3);
                \vertex (t2) at (0.25, 3);
                \propag[plain] (t1) to[out=90, in=90] (t2);

                \vertex (v1) at (0.25, 1) ;
                \vertex (l1) at (1.55, 0.5) {$\omega_p$};
                \propag[extphoton] (l1) to (v1);
                \propag[plain] (g2) to (v1) ;

                \propag[plain] (v1) to[edge label'={$m$}] (t2);
                \propag[plain] (t1) to (o);

                \vertex (o) at (-0.25, 2);
                \vertex (e) at (-1.55, 2.5) {$\omega_p$};
                \propag[outphoton] (o) to (e);

                \propag[plain] (o) to (g1);
            \end{feynhand}
        \end{tikzpicture}
    \end{gathered} = \sum_p \vb*{d}_{mg} \ee^{- \ii \omega_p t} \sum_m \frac{\ii \vb*{d}_{gm} \cdot \vb*{E}(\omega_p)}{\hbar} \frac{\ii}{\omega_g - \omega_p - \omega_m + \ii \gamma_{mg}} ,
\end{equation}
和\eqref{eq:right-in-one-order-perturbation}一致。
类似的可以验证,\eqref{eq:feynman-diagram-left}和\eqref{eq:feynman-diagram-right}两个费曼规则也是可以通过凝聚态场论导出的,其中诸如$\sum \omega_i$这样奇怪的表达式实际上是在使用能量守恒条件。

我们也可以将以上费曼图理解为关于密度矩阵的费曼图,在每个时间点,有\emph{两个}状态,分别对应密度矩阵的左矢和右矢,因此电子线包括左右两部分。
我们将在\autoref{sec:electron-density-matrix}中显式地分析基于密度矩阵的计算。

\subsection{密度矩阵}\label{sec:electron-density-matrix}

\begin{equation}
    \dot{\rho} = 
\end{equation}
\begin{equation}
    \rho^\text{eq}_{mn} = \rho^\text{eq}_{mm} \delta_{mn}.
\end{equation}

\subsection{空间相位}

TODO:多个分子之间的间距如果恰到好处地让它们的辐射相位正好差了$\pi$,有可能在远处观察不到辐射

\section{非线性极化的麦克斯韦方程}\label{sec:non-linear-maxwell}

本节将从一个比较唯象的角度描述经典非线性光学——我们忽略所有微观细节,只是去解麦克斯韦方程组,不过在这里,$\vb*{P}$和$\vb*{E}$之间的关系不再是线性的。
一般来说$E \sim \SI{1}{kV/cm}$时非线性效应变得重要起来。

\subsection{非线性波动方程的建立和微扰求解}

\subsubsection{均匀介质中的传播}

本节讨论均匀非线性介质中的光的传播。在大部分情况下介质对电磁波中的磁场的响应都是可以忽略的,因此介质对光的影响可以概括为极化。
以下我们用短直线代表极化$P$,用波浪线代表电场;后者的意义一目了然,前者和前面几节的意义其实也是一样的,因为无非有
\[
    \vb*{P} = N q \vb*{x},
\]
其中$N$为单位体积的谐振子数目。我们知道极化强度矢量可以产生一个电场。我们有波动方程
\[
    \curl{(\curl{\vb*{E}})} + \frac{1}{\epsilon_0 c^2} \pdv[2]{\vb*{D}}{t} = 0,
\]
代入$\vb*{D}$的定义,就得到
\begin{equation}
    \curl{(\curl{\vb*{E}})} + \frac{1}{c^2} \pdv[2]{\vb*{E}}{t} = - \frac{1}{\epsilon_0 c^2} \pdv[2]{\vb*{P}}{t}.
\end{equation}
实际上,这就是“电流作为一个天线,发出电磁波”,只不过此时天线是由总电场驱动的,从而上式左边和右边都有$\vb*{E}$,并且由于$\vb*{P}$和$\vb*{E}$之间的关心是非线性的,上式关于$\vb*{E}$也是非线性的。
不过,从$\vb*{P}$到$\vb*{E}$的关系的确是完全线性的,因此可以使用
\begin{equation}
    \begin{gathered}
        \begin{tikzpicture}
            \begin{feynhand}
                \vertex (a) at (0, 0);
                \vertex (b) at (1, 0);
                \vertex (c) at (2, 0);

                \propag[plain] (a) to (b);
                \propag[photon] (b) to (c);
            \end{feynhand}
        \end{tikzpicture}
    \end{gathered} = - \left( \curl{\curl{}} + \frac{1}{c^2} \pdv[2]{t} \right)^{-1} \frac{1}{\epsilon_0 c^2} \pdv[2]{\vb*{P}}{t}
\end{equation}
表示。这是$\vb*{P}$消灭而光子产生的过程,看起来非常直观:一个天线上的激发态消失了,取而代之的是天线发出来的光。
本节讨论的是均匀介质,我们可以认为$\grad{\rho_\text{p}}=0$,从而$\grad{\div{\vb*{P}}}=0$,从而就得到$\div{\vb*{E}} = 0$。这样一来我们要求解的就是
\begin{equation}
    - \laplacian \vb*{E} + \frac{1}{c^2} \pdv[2]{\vb*{E}}{t} = -\frac{1}{\epsilon_0 c^2} \pdv[2]{\vb*{P}}{t},
\end{equation}
而
\begin{equation}
    \begin{gathered}
        \begin{tikzpicture}
            \begin{feynhand}
                \vertex (a) at (0, 0);
                \vertex (b) at (1, 0);
                \vertex (c) at (2, 0);

                \propag[plain] (a) to (b);
                \propag[photon] (b) to (c);
            \end{feynhand}
        \end{tikzpicture}
    \end{gathered} = \left( \laplacian - \frac{1}{c^2} \pdv[2]{t} \right)^{-1} \frac{1}{\epsilon_0 c^2} \pdv[2]{\vb*{P}}{t}.
\end{equation}

此外我们还有任意的形如
\begin{equation}
    \begin{gathered}
        \begin{tikzpicture}
            \begin{feynhand}
                \vertex (a) at (-0.8, 0.25) ;
                \vertex (b) at (-0.8, -0.25) ;
                \vertex (d) at (0, -1) {$\ldots$};
                \vertex (c) at (0, 0);
                \vertex (e) at (1, 0);

                \propag[photon] (a) to (c);
                \propag[photon] (b) to (c);
                \propag[photon] (d) to (c);
                \propag[plain] (c) to (e);
            \end{feynhand}
        \end{tikzpicture}
    \end{gathered} = \epsilon_0 \chi^{(n)}_{i_1 i_2 \ldots i_n}
\end{equation}
的产生$\vb*{P}$的顶角,代表$n$阶非线性极化(一阶极化是线性的),使得
\[
    \begin{gathered}
        \begin{tikzpicture}
            \begin{feynhand}
                \vertex[crossdot] (a) at (-0.8, 0.25) {};
                \vertex[crossdot] (b) at (-0.8, -0.25) {};
                \vertex (d) at (0, -1) {$\ldots$};
                \vertex (c) at (0, 0);
                \vertex (e) at (1, 0);

                \propag[photon] (a) to (c);
                \propag[photon] (b) to (c);
                \propag[photon] (d) to (c);
                \propag[plain] (c) to (e);
            \end{feynhand}
        \end{tikzpicture}
    \end{gathered} = \epsilon_0 \vb*{\chi}^{(n)} : \vb*{E} \vb*{E} \cdots \vb*{E}.
\]
线性光学中只有一条入射线,非线性光学中可以有多条。

这就是非线性波动方程中的所有顶角了。我们马上可以发现一件事,就是虽然看起来$\vb*{P}$场能够有自能修正
\[
    \begin{gathered}
        \begin{tikzpicture}
            \begin{feynhand}
                \vertex (a) at (0, 0);
                \vertex (b) at (1, 0);
                \propag[double] (a) to (b);
            \end{feynhand}
        \end{tikzpicture}
    \end{gathered} = \begin{gathered}
        \begin{tikzpicture}
            \begin{feynhand}
                \vertex (a) at (0, 0);
                \vertex (b) at (1, 0);
                \propag[plain] (a) to (b);
            \end{feynhand}
        \end{tikzpicture}
    \end{gathered} + 
    \begin{gathered}
        \begin{tikzpicture}
            \begin{feynhand}
                \vertex (a) at (0, 0);
                \vertex (b) at (1, 0);
                \vertex (c) at (2, 0);
                \vertex (d) at (3, 0);
                \propag[plain] (a) to (b);
                \propag[photon] (b) to (c);
                \propag[plain] (c) to (d);
            \end{feynhand}
        \end{tikzpicture}
    \end{gathered} + \cdots,
\]
但是,因为我们讨论的所有过程都以光子开始以光子结束,加上入射和出射线之后上式实际上是光的自能修正。因此实际上$\vb*{P}$只是一个辅助量,我们可以将
\[
    \begin{gathered}
        \begin{tikzpicture}
            \begin{feynhand}
                \vertex (a) at (0, 0);
                \vertex (b) at (1, 0);
                \vertex (c) at (2, 0);
                \vertex (d) at (3, 0);
                \propag[photon] (a) to (b);
                \propag[plain] (b) to (c);
                \propag[photon] (c) to (d);
            \end{feynhand}
        \end{tikzpicture}
    \end{gathered}
\]
看成一个整体。当然,本该如此。

我们要求解外加电场下的总场。这就是说,要计算
\[
    \begin{aligned}
        \begin{gathered}
            \begin{tikzpicture}
                \begin{feynhand}
                    \vertex[grayblob] (a) at (0, 0) {};
                    \vertex (b) at (1.25, 0);
                    \propag[boldphoton] (a) to (b);
                \end{feynhand}
            \end{tikzpicture}
        \end{gathered} &= 
        \begin{gathered}
            \begin{tikzpicture}
                \begin{feynhand}
                    \vertex[crossdot] (a) at (0, 0) {};
                    \vertex (b) at (1, 0);
                    \propag[photon] (a) to (b);
                \end{feynhand}
            \end{tikzpicture} 
        \end{gathered} + 
        \begin{gathered}
            \begin{tikzpicture}
                \begin{feynhand}
                    \vertex[crossdot] (a) at (0, 0) {};
                    \vertex (b) at (1, 0);
                    \vertex (c) at (2, 0);
                    \vertex (d) at (3, 0);
                    \propag[photon] (a) to (b);
                    \propag[plain] (b) to (c);
                    \propag[photon] (c) to (d);
                \end{feynhand}
            \end{tikzpicture}
        \end{gathered} + \cdots \\
        &+ \begin{gathered}
            \begin{tikzpicture}
                \begin{feynhand}
                    \vertex[crossdot] (a) at (-0.8, 0.25) {};
                    \vertex[crossdot] (b) at (-0.8, -0.25) {};
                    \vertex (d) at (0, -1) {$\ldots$};
                    \vertex (c) at (0, 0);
                    \vertex (e) at (1, 0);
                    \vertex (f) at (2, 0);
    
                    \propag[photon] (a) to (c);
                    \propag[photon] (b) to (c);
                    \propag[photon] (d) to (c);
                    \propag[plain] (c) to (e);
                    \propag[photon] (e) to (f);
                \end{feynhand}
            \end{tikzpicture}
        \end{gathered}
    \end{aligned}
\]
请注意上式左边的源不是$\otimes$而是一个灰色圆圈,因为非线性效应的存在让我们不知道外源被用到了多少次。
上式直接计算当然是非常复杂的,我们做重求和,根据最后一个顶角是什么,可以将上式写成
\begin{equation}
    \begin{aligned}
        \begin{gathered}
            \begin{tikzpicture}
                \begin{feynhand}
                    \vertex[grayblob] (a) at (0, 0) {};
                    \vertex (b) at (1.25, 0);
                    \propag[boldphoton] (a) to (b);
                \end{feynhand}
            \end{tikzpicture}
        \end{gathered} &= \begin{gathered}
            \begin{tikzpicture}
                \begin{feynhand}
                    \vertex[crossdot] (a) at (0, 0) {};
                    \vertex (b) at (1, 0);
                    \propag[photon] (a) to (b);
                \end{feynhand}
            \end{tikzpicture} 
        \end{gathered} +
        \begin{gathered}
            \begin{tikzpicture}
                \begin{feynhand}
                    \vertex[grayblob] (a) at (0, 0) {};
                    \vertex (b) at (1, 0);
                    \vertex (c) at (1.6, 0);
                    \vertex (d) at (2.2, 0);
                    \propag[boldphoton] (a) to (b);
                    \propag[plain] (b) to (c);
                    \propag[photon] (c) to (d);
                \end{feynhand}
            \end{tikzpicture}
        \end{gathered} +
        \sum \begin{gathered}
            \begin{tikzpicture}
                \begin{feynhand}
                    \vertex[grayblob] (a) at (-1.2, 0.5) {};
                    \vertex[grayblob] (b) at (-1.2, -0.5) {};
                    \vertex (f) at (0, -1) {$\ldots$};
                    \vertex (c) at (0, 0);
                    \vertex (d) at (0.75, 0);
                    \vertex (e) at (1.5, 0);
    
                    \propag[boldphoton] (a) to (c);
                    \propag[boldphoton] (b) to (c);
                    \propag[boldphoton] (f) to (c);
                    \propag[plain] (c) to (d);
                    \propag[photon] (d) to (e);
                \end{feynhand}
            \end{tikzpicture}
        \end{gathered} \\
        &= \begin{gathered}
            \begin{tikzpicture}
                \begin{feynhand}
                    \vertex[crossdot] (a) at (0, 0) {};
                    \vertex (b) at (1, 0);
                    \propag[photon] (a) to (b);
                \end{feynhand}
            \end{tikzpicture} 
        \end{gathered} +
        \begin{gathered}
            \begin{tikzpicture}
                \begin{feynhand}
                    \vertex[grayblob] (a) at (0, 0) {};
                    \vertex (b) at (1, 0);
                    \vertex (c) at (2, 0);
                    \propag[plain] (a) to (b);
                    \propag[photon] (b) to (c);
                \end{feynhand}
            \end{tikzpicture}
        \end{gathered} \ .
    \end{aligned}
\end{equation}
当然,这就是
\[
    \vb*{E} = \vb*{E}_\text{ext} + \left( \laplacian - \frac{1}{c^2} \pdv[2]{t} \right)^{-1} \frac{1}{\epsilon_0 c^2} \pdv[2]{\vb*{P}}{t} ,
\]
稍加变形就会发现实际上这就是
\[
    \left( \laplacian - \frac{1}{c^2} \pdv[2]{t} \right) (\vb*{E} - \vb*{E}_\text{ext}) = \frac{1}{\epsilon_0 c^2} \pdv[2]{\vb*{P}}{t}.
\]
求解出上式中的$\vb*{E}$就完全确定了外加电场下的总场。“外加电场”当然是一个不良定义的概念,但这是因为我们将$\otimes$的来源省去了;它可能来自介质中的电荷或是介质两边的极板(归根到底,也是一些电荷),无论如何不来自$\vb*{P}$,即它是“没有经过相互作用修正的”,从而
\[
    \left( \laplacian - \frac{1}{c^2} \pdv[2]{t} \right) \vb*{E}_\text{ext} = 0.
\]
如果将理论量子化,它就是入射的近乎自由的粒子。我们经常称外加电场为\concept{泵浦光},因为它将材料中的电子激发。%
\footnote{
    只要满足这个形式的电场分量都有机会被称为泵浦光。因此这个概念的确有定义模糊的地方。它可以指代从空气射向介质的入射光,可以指代将空气-介质界面当成线性介质界面而获得的折射光,甚至我们不能排除它指代经过局域场修正之后的折射光。
    本节讨论均匀介质内部的宏观问题,因此泵浦光不是从空气到介质的入射光也不需要做局域场修正。
}%

不过,具体什么是“相互作用修正”仍然有选择的余地,如我们可以将线性极化(即$\vb*{P}$关于$\vb*{E}$是线性的那部分;从$\vb*{P}$到$\vb*{E}$的转换则完全是线性的)部分作为自能修正重求和,然后将经过线性极化修正的光束(也就是服从关于不等于$\epsilon_0$的$\epsilon$的线性麦克斯韦方程组的光束)重新定义为\begin{tikzpicture}
    \begin{feynhand}
        \vertex (a) at (0, 0);
        \vertex (b) at (1, 0);
        \propag[photon] (a) to (b);
    \end{feynhand}
\end{tikzpicture},此时\begin{tikzpicture}
    \begin{feynhand}
        \vertex (a) at (0, 0);
        \vertex (b) at (1, 0);
        \vertex (c) at (2, 0);

        \propag[plain] (a) to (b);
        \propag[photon] (b) to (c);
    \end{feynhand}
\end{tikzpicture}也需要调整(原本的意思是总的$\vb*{P}$产生真空中电场的响应函数,现在的意思是(非线性的)$\vb*{P}$产生线性介质中电场的响应函数),引入$\epsilon_{ij}$张量,然后顶角只剩下一种,就有
\begin{equation}
    \begin{gathered}
        \begin{tikzpicture}
            \begin{feynhand}
                \vertex[grayblob] (a) at (0, 0) {};
                \vertex (b) at (1.25, 0);
                \propag[boldphoton] (a) to (b);
            \end{feynhand}
        \end{tikzpicture}
    \end{gathered} = \begin{gathered}
        \begin{tikzpicture}
            \begin{feynhand}
                \vertex[crossdot] (a) at (0, 0) {};
                \vertex (b) at (1, 0);
                \propag[photon] (a) to (b);
            \end{feynhand}
        \end{tikzpicture} 
    \end{gathered} +
    \sum \begin{gathered}
        \begin{tikzpicture}
            \begin{feynhand}
                \vertex[grayblob] (a) at (-1.2, 0.5) {};
                \vertex[grayblob] (b) at (-1.2, -0.5) {};
                \vertex (f) at (0, -1) {$\ldots$};
                \vertex (c) at (0, 0);
                \vertex (d) at (0.75, 0);
                \vertex (e) at (1.5, 0);

                \propag[boldphoton] (a) to (c);
                \propag[boldphoton] (b) to (c);
                \propag[boldphoton] (f) to (c);
                \propag[plain] (c) to (d);
                \propag[photon] (d) to (e);
            \end{feynhand}
        \end{tikzpicture}
    \end{gathered} \ ,
\end{equation}
即
\begin{equation}
    \left( \laplacian - \frac{\vb*{\epsilon}_\text{r} \cdot}{c^2} \pdv[2]{t} \right) (\vb*{E} - \vb*{E}_\text{ext}) = \left( \laplacian - \frac{\vb*{\epsilon}_\text{r} \cdot}{c^2} \pdv[2]{t} \right) \vb*{E} = \frac{1}{\epsilon_0 c^2} \pdv[2]{\vb*{P}_\text{NL}}{t},
    \label{eq:nonlinear-maxwell-eq}
\end{equation}
其中NL表示非线性。这样做的好处一目了然:线性极化相当于对光速做了一个修正,是可以严格处理的,从而可以将主要精力集中在非线性极化上。

一个在线性光学中也会出现的现象是\concept{局域场强化}。这个现象来自于,从$\vb*{P}$产生的电场在空间上可能有比较大的变化,我们关心的是空间平均过之后的电场,即只是从$\vb*{P}$产生的电场的低频傅里叶分量。
因此我们做分解
\begin{equation}
    \begin{gathered}
        \begin{tikzpicture}
            \begin{feynhand}
                \vertex (a) at (0, 0);
                \vertex (b) at (1, 0);
                \vertex (c) at (2, 0);

                \propag[plain] (a) to (b);
                \propag[photon] (b) to (c);
            \end{feynhand}
        \end{tikzpicture}
    \end{gathered} = 
    \begin{gathered}
        \begin{tikzpicture}
            \begin{feynhand}
                \vertex (a) at (0, 0);
                \vertex (b) at (1, 0);
                \vertex (c) at (2, 0);

                \propag[plain] (a) to (b);
                \propag[photon, mom={$\text{low $k$}$}] (b) to (c);
            \end{feynhand}
        \end{tikzpicture}
    \end{gathered} + 
    \begin{gathered}
        \begin{tikzpicture}
            \begin{feynhand}
                \vertex (a) at (0, 0);
                \vertex (b) at (1, 0);
                \vertex (c) at (2, 0);

                \propag[plain] (a) to (b);
                \propag[photon, mom={$\text{high $k$}$}] (b) to (c);
            \end{feynhand}
        \end{tikzpicture}
    \end{gathered},
\end{equation}
那么就有
\begin{equation}
    \begin{aligned}
        \begin{gathered}
            \begin{tikzpicture}
                \begin{feynhand}
                    \vertex[grayblob] (a) at (0, 0) {};
                    \vertex (b) at (1.25, 0);
                    \propag[boldphoton, mom={$\text{low $k$}$}] (a) to (b);
                \end{feynhand}
            \end{tikzpicture}
        \end{gathered} &= \begin{gathered}
            \begin{tikzpicture}
                \begin{feynhand}
                    \vertex[crossdot] (a) at (0, 0) {};
                    \vertex (b) at (1, 0);
                    \propag[photon, mom={$\text{low $k$}$}] (a) to (b);
                \end{feynhand}
            \end{tikzpicture} 
        \end{gathered} +
        \begin{gathered}
            \begin{tikzpicture}
                \begin{feynhand}
                    \vertex[grayblob] (a) at (0, 0) {};
                    \vertex (b) at (1, 0);
                    \vertex (c) at (1.6, 0);
                    \vertex (d) at (2.2, 0);
                    \propag[boldphoton] (a) to (b);
                    \propag[plain] (b) to (c);
                    \propag[photon, mom={$\text{low $k$}$}] (c) to (d);
                \end{feynhand}
            \end{tikzpicture}
        \end{gathered} +
        \sum \begin{gathered}
            \begin{tikzpicture}
                \begin{feynhand}
                    \vertex[grayblob] (a) at (-1.2, 0.5) {};
                    \vertex[grayblob] (b) at (-1.2, -0.5) {};
                    \vertex (f) at (0, -1) {$\ldots$};
                    \vertex (c) at (0, 0);
                    \vertex (d) at (0.75, 0);
                    \vertex (e) at (1.5, 0);
    
                    \propag[boldphoton] (a) to (c);
                    \propag[boldphoton] (b) to (c);
                    \propag[boldphoton] (f) to (c);
                    \propag[plain] (c) to (d);
                    \propag[photon, mom={$\text{low $k$}$}] (d) to (e);
                \end{feynhand}
            \end{tikzpicture}
        \end{gathered} \\
        &= \begin{gathered}
            \begin{tikzpicture}
                \begin{feynhand}
                    \vertex[crossdot] (a) at (0, 0) {};
                    \vertex (b) at (1, 0);
                    \propag[photon, mom={$\text{low $k$}$}] (a) to (b);
                \end{feynhand}
            \end{tikzpicture} 
        \end{gathered} +
        \begin{gathered}
            \begin{tikzpicture}
                \begin{feynhand}
                    \vertex[grayblob] (a) at (0, 0) {};
                    \vertex (b) at (1, 0);
                    \vertex (c) at (2, 0);
                    \propag[plain] (a) to (b);
                    \propag[photon, mom={$\text{low $k$}$}] (b) to (c);
                \end{feynhand}
            \end{tikzpicture}
        \end{gathered} \ .
    \end{aligned}
    \label{eq:non-linear-smooth-p-e}
\end{equation}
然而,在上式中与$\vb*{P}$有关的图中的入射电场却没有做这个动量截断。
其结果是我们无法单纯从上式计算经过动量截断的出射电场,即我们需要手动提供没做动量截断的入射电场。
由于入射电场通常只具有低频分量,即
\[
    \begin{gathered}
        \begin{tikzpicture}
            \begin{feynhand}
                \vertex[crossdot] (a) at (0, 0) {};
                \vertex (b) at (1, 0);
                \propag[photon, mom={$\text{low $k$}$}] (a) to (b);
            \end{feynhand}
        \end{tikzpicture} 
    \end{gathered} = \begin{gathered}
        \begin{tikzpicture}
            \begin{feynhand}
                \vertex[crossdot] (a) at (0, 0) {};
                \vertex (b) at (1, 0);
                \propag[photon] (a) to (b);
            \end{feynhand}
        \end{tikzpicture} 
    \end{gathered},
\]
我们有
\[
    \begin{aligned}
        \begin{gathered}
            \begin{tikzpicture}
                \begin{feynhand}
                    \vertex[grayblob] (a) at (0, 0) {};
                    \vertex (b) at (1.25, 0);
                    \propag[boldphoton, mom={$\text{high $k$}$}] (a) to (b);
                \end{feynhand}
            \end{tikzpicture}
        \end{gathered} &= 
        \begin{gathered}
            \begin{tikzpicture}
                \begin{feynhand}
                    \vertex[grayblob] (a) at (0, 0) {};
                    \vertex (b) at (1, 0);
                    \vertex (c) at (1.6, 0);
                    \vertex (d) at (2.5, 0);
                    \propag[boldphoton] (a) to (b);
                    \propag[plain] (b) to (c);
                    \propag[photon, mom={$\text{high $k$}$}] (c) to (d);
                \end{feynhand}
            \end{tikzpicture}
        \end{gathered} +
        \sum \begin{gathered}
            \begin{tikzpicture}
                \begin{feynhand}
                    \vertex[grayblob] (a) at (-1.2, 0.5) {};
                    \vertex[grayblob] (b) at (-1.2, -0.5) {};
                    \vertex (f) at (0, -1) {$\ldots$};
                    \vertex (c) at (0, 0);
                    \vertex (d) at (0.75, 0);
                    \vertex (e) at (1.5, 0);
    
                    \propag[boldphoton] (a) to (c);
                    \propag[boldphoton] (b) to (c);
                    \propag[boldphoton] (f) to (c);
                    \propag[plain] (c) to (d);
                    \propag[photon, mom={$\text{high $k$}$}] (d) to (e);
                \end{feynhand}
            \end{tikzpicture}
        \end{gathered} \\
        &= \begin{gathered}
            \begin{tikzpicture}
                \begin{feynhand}
                    \vertex[grayblob] (a) at (0, 0) {};
                    \vertex (b) at (1, 0);
                    \vertex (c) at (2, 0);
                    \propag[plain] (a) to (b);
                    \propag[photon, mom={$\text{high $k$}$}] (b) to (c);
                \end{feynhand}
            \end{tikzpicture}
        \end{gathered} \ ,
    \end{aligned}
\]
于是就有
\begin{equation}
    \begin{gathered}
        \begin{tikzpicture}
            \begin{feynhand}
                \vertex[grayblob] (a) at (0, 0) {};
                \vertex (b) at (1.25, 0);
                \propag[boldphoton] (a) to (b);
            \end{feynhand}
        \end{tikzpicture}
    \end{gathered} = \begin{gathered}
        \begin{tikzpicture}
            \begin{feynhand}
                \vertex[grayblob] (a) at (0, 0) {};
                \vertex (b) at (1.25, 0);
                \propag[boldphoton, mom={$\text{low $k$}$}] (a) to (b);
            \end{feynhand}
        \end{tikzpicture}
    \end{gathered} + 
    \begin{gathered}
        \begin{tikzpicture}
            \begin{feynhand}
                \vertex[grayblob] (a) at (0, 0) {};
                \vertex (b) at (1, 0);
                \vertex (c) at (2, 0);
                \propag[plain] (a) to (b);
                \propag[photon, mom={$\text{high $k$}$}] (b) to (c);
            \end{feynhand}
        \end{tikzpicture}
    \end{gathered}.
    \label{eq:local-field-enhancement-diagram}
\end{equation}
联立计算\eqref{eq:non-linear-smooth-p-e}和\eqref{eq:local-field-enhancement-diagram}(求解过程中很容易看到,虽然$\vb*{P}$是辅助量但是确实能够简化书写),就得到外场作用下的做了动量截断的$\vb*{E}$——实际上,由于宏观下我们讨论的“介质中电场”一般都是均一化的电场,我们将做了动量截断的$\vb*{E}$就称为$\vb*{E}$。%
\footnote{
    电动力学教科书上通常会泛泛地说这是“做了宏观平均的电场”,实际上这就是动量截断。
    要看出这是为什么,注意到“做了宏观平均的电场”总是可以通过一个体积为$V$的移动平均给出:
    \[
        \bar{\vb*{E}}(\vb*{r}) = \frac{1}{V} \int \dd[3]{\vb*{r}'} \vb*{E}(\vb*{r}'),
    \]
    做傅里叶展开,有
    \[
        \begin{aligned}
            \bar{\vb*{E}}(\vb*{r}) &= \frac{1}{V} \int \dd[3]{\vb*{r}'} \int \frac{\dd[3]{\vb*{k}}}{(2\pi)^3} \vb*{E}(\vb*{k}) \ee^{\ii \vb*{k} \cdot \vb*{r}'} \\
            &= \int \frac{\dd[3]{\vb*{k}}}{(2\pi)^3} \vb*{E}(\vb*{k}) \frac{1}{V} \int \dd[3]{\vb*{r}'} \ee^{\ii \vb*{k} \cdot \vb*{r}'},
        \end{aligned}
    \]
    显然对那些较大的$\vb*{k}$,对$\vb*{r}'$的积分会由于快速振荡而变成零,于是我们就得到了一个动量截断。
}%
此时\eqref{eq:local-field-enhancement-diagram}应当理解为“靠近介质粒子时电场的局域增强”。
在这种视角下,有
\begin{equation}
    \begin{aligned}
        \begin{gathered}
            \begin{tikzpicture}
                \begin{feynhand}
                    \vertex[grayblob] (a) at (0, 0) {};
                    \vertex (b) at (1.25, 0);
                    \propag[boldphoton] (a) to (b);
                \end{feynhand}
            \end{tikzpicture}
        \end{gathered} &= \begin{gathered}
            \begin{tikzpicture}
                \begin{feynhand}
                    \vertex[crossdot] (a) at (0, 0) {};
                    \vertex (b) at (1, 0);
                    \propag[photon] (a) to (b);
                \end{feynhand}
            \end{tikzpicture} 
        \end{gathered} +
        \begin{gathered}
            \begin{tikzpicture}
                \begin{feynhand}
                    \vertex[grayblob] (a) at (0, 0) {};
                    \vertex (b) at (1, 0);
                    \vertex (c) at (1.6, 0);
                    \vertex (d) at (2.2, 0);
                    \propag[boldphoton, mom={$\text{enhanced}$}] (a) to (b);
                    \propag[plain] (b) to (c);
                    \propag[photon] (c) to (d);
                \end{feynhand}
            \end{tikzpicture}
        \end{gathered} +
        \sum \begin{gathered}
            \begin{tikzpicture}
                \begin{feynhand}
                    \vertex[grayblob] (a) at (-1.2, 0.5) {};
                    \vertex[grayblob] (b) at (-1.2, -0.5) {};
                    \vertex (f) at (0, -1) {$\ldots$};
                    \vertex (c) at (0, 0);
                    \vertex (d) at (0.75, 0);
                    \vertex (e) at (1.5, 0);
    
                    \propag[boldphoton, mom={$\text{enhanced}$}] (a) to (c);
                    \propag[boldphoton, mom={$\text{enhanced}$}] (b) to (c);
                    \propag[boldphoton, mom={$\text{enhanced}$}] (f) to (c);
                    \propag[plain] (c) to (d);
                    \propag[photon] (d) to (e);
                \end{feynhand}
            \end{tikzpicture}
        \end{gathered} \\
        &= \begin{gathered}
            \begin{tikzpicture}
                \begin{feynhand}
                    \vertex[crossdot] (a) at (0, 0) {};
                    \vertex (b) at (1, 0);
                    \propag[photon] (a) to (b);
                \end{feynhand}
            \end{tikzpicture} 
        \end{gathered} +
        \begin{gathered}
            \begin{tikzpicture}
                \begin{feynhand}
                    \vertex[grayblob] (a) at (0, 0) {};
                    \vertex (b) at (1, 0);
                    \vertex (c) at (2, 0);
                    \propag[plain] (a) to (b);
                    \propag[photon] (b) to (c);
                \end{feynhand}
            \end{tikzpicture}
        \end{gathered} , \\
        \begin{gathered}
            \begin{tikzpicture}
                \begin{feynhand}
                    \vertex[grayblob] (a) at (0, 0) {};
                    \vertex (b) at (1.25, 0);
                    \propag[boldphoton, mom={$\text{enhanced}$}] (a) to (b);
                \end{feynhand}
            \end{tikzpicture}
        \end{gathered} &= \begin{gathered}
            \begin{tikzpicture}
                \begin{feynhand}
                    \vertex[grayblob] (a) at (0, 0) {};
                    \vertex (b) at (1.25, 0);
                    \propag[boldphoton] (a) to (b);
                \end{feynhand}
            \end{tikzpicture}
        \end{gathered} + 
        \begin{gathered}
            \begin{tikzpicture}
                \begin{feynhand}
                    \vertex[grayblob] (a) at (0, 0) {};
                    \vertex (b) at (1, 0);
                    \vertex (c) at (2, 0);
                    \propag[plain] (a) to (b);
                    \propag[photon, mom={$\text{high $k$}$}] (b) to (c);
                \end{feynhand}
            \end{tikzpicture}
        \end{gathered}.
    \end{aligned}
\end{equation}

\subsubsection{边界条件}

边界,反射,折射:在线性的折射、反射上引入三个相互作用,一个是介质内部的光子-光子和频和差频,一个是入射端,介质内部两个光子打在边界上然后返回形成一个光子,一个是出射端,两个光子打在边界上然后一起出射。

\subsubsection{哪些解是重要的?}

非线性光学和线性光学非常不同的地方是,在线性光学中,给定一个时谐的(并且通常是平面波的,因为介质一般都很大,近似空间平移不变)泵浦光,计算入射导致的响应以及总电场,我们就得到了一个\emph{稳定的}模式;一个任意的入射波传入介质后的行为可以直接根据入射波在各个模式上的分量计算出来。
这意味着在线性光学中我们无需为任意的泵浦光求解麦克斯韦方程:只需要得到线性波动方程的所有振动模式就可以,各个振动模式的形状决定了这个光学系统的全部信息;场论上,这是自由场的情况。
线性性也意味着一个光束可以被分割成许多波包,计算每个波包的时间演化,加起来就得到光束的时间演化——实际上,我们经常反过来用线性性,即先求解一个系统中的电磁波模式(其时间演化是平凡的)然后诠释说“这意味着光先走到这里再走到那里”。
非线性光学则不同,输出和激励并没有简单的线性关系。不过实际上,这并不意味着我们真的需要为所有的泵浦场独立地从头求解非线性波动方程。

首先,我们看到\eqref{eq:nonlinear-maxwell-eq}相对于时间是线性的。这意味着我们可以在时间的频域求解\eqref{eq:nonlinear-maxwell-eq}。
也就是说,\eqref{eq:nonlinear-maxwell-eq}的全体时谐解已经包含了诸如“两个略有错开的波包打进介质将会如何反应”之类的瞬态问题的答案——只要入射波包不过于极端,以至于非线性麦克斯韦方程不再适用(如非常快导致介质来不及弛豫或是过强导致介质损坏等)。
在线性光学中这是常用的手段:例如,我们求解了时谐场在金属中的衰减,就可以不重新求解麦克斯韦方程就知道一个波包在金属中如何衰减,做一个傅里叶变换就行了:稳态问题提供了瞬态问题的答案,而前者往往更加容易做具体计算。
非线性光学中这个手段也是适用的。
因此,我们会在频域求解\eqref{eq:nonlinear-maxwell-eq},然后看着光强随着光束前进的\emph{距离}发生的变化,评论说“波包随着传播\emph{时间}发生的变化”,因为两者是可以互相转化的。

其次,似乎存在这样一个疑难:在后文中我们将会求解诸如“泵浦光含有两个频率成分,它们都是平面波”这样的问题,但是实际上我们能够制备的光都是含有很多混杂的成分的,那么有什么理由保证后者发生的现象和前者一致?
例如,设泵浦光为
\[
    \vb*{E}^\text{ext} = \vb*{E}_1 \ee^{-\ii \omega_1 t} + \vb*{E}_2 \ee^{-\ii \omega_2 t} + \text{c.c.},
\]
微扰求解得到输出光为
\[
    \vb*{E} = \alpha^{(1)}(\omega_1) \vb*{E}_1 \ee^{-\ii \omega_1 t} + \alpha^{(1)}(\omega_2) \vb*{E}_2 \ee^{-\ii \omega_2 t} + \alpha^{(2)}(\omega_3 = \omega_1 + \omega_2) : \vb*{E}_1 \vb*{E}_2 \ee^{-\ii (\omega_1 + \omega_2) t} + \cdots + \text{c.c.}.
\]
现在如果我们要将泵浦光取为
\[
    \vb*{E}^\text{ext}(t) = \int \dd{\omega} \ee^{-\ii \omega t} \vb*{E}^\text{ext}(\omega),
\]
那么输出光$E(t)$不能通过简单地做一个傅里叶变换获得。
但其实我们可以效仿量子场论中的做法来解决这个问题。
我们要做的事情是通过计算\eqref{eq:nonlinear-maxwell-eq}的一些特解来获得通用的“看到泵浦光-写出输出光”的方法,这相当于我们在积掉整个光学系统,直接得到泵浦光的各个频率分量和输出的非线性谐波之间的关系。
所以其实可以采用量子场论中计算等效相互作用顶角的标准方法,即只计算连通图:设泵浦光包含$n$个(频率不同的)单频成分,微扰求解\eqref{eq:nonlinear-maxwell-eq},只计算那些完全连通且正好包含$n$个\begin{tikzpicture}
    \begin{feynhand}
        \vertex[crossdot] (a) at (0, 0) {};
        \vertex (b) at (1, 0);
        \propag[photon] (a) to (b);
    \end{feynhand}
\end{tikzpicture}的图(后一个条件是为了保证我们计算的是$n$阶非线性过程,前一个条件是为了保证低于$n$阶的非线性过程没有被重复计数),这样可以得到形如$\alpha^{(n)} \vb*{E} \vb*{E} \cdots \vb*{E}$的响应,输出光就是
\[
    \vb*{E} = \int \dd{\omega} \alpha^{(1)}(\omega) \vb*{E}^\text{ext}(\omega) \ee^{-\ii \omega t} + \int \dd{\omega_1} \int \dd{\omega_2} \alpha^{(2)}(\omega_1 + \omega_2) : \vb*{E}^\text{ext}(\omega_1) \vb*{E}^\text{ext}(\omega_2) \ee^{-\ii (\omega_1 + \omega_2) t} + \cdots.
\]
请注意由于不存在圈图,每个$\alpha^{(n)}$中的图的数目是有限的,从而可以精确计算而不需要对圈数做截断等。

总之,虽然非线性光学问题比线性光学复杂上不少,我们仍然可以只计算泵浦光为若干个单频光叠加的情况,并且计算到给定的微扰阶数,这样足够提取到足够多的关于非线性光学体系的信息。

\subsection{近似求解方法}

由前述讨论,我们只要拿到非线性极化$\vb*{P}_\text{NL}$,就能够完全确定一个介质的非线性光学性能,只需要计算向介质中输入若干束单频光得到的输出就够了。
此时对\eqref{eq:nonlinear-maxwell-eq}做时域傅里叶变换,就得到不同频率的电场的空间分布方程,$\vb*{P}_\text{NL}$中如果有电场的$n-1$次方,这就是\concept{$n$波混频方程}。

求解$n$波混频方程时会用到一系列近似,列举如下:
\begin{itemize}
    \item \concept{泵浦光无衰减近似},在非线性介质非常厚,或者转换效率非常高以至于泵浦光衰减很明显时失效。
    \item \concept{平面波近似},即认为介质中电磁波的所有频率分量都可以看成(振幅可能会变化的)平面波。
    \item \concept{慢变振幅近似},和几何光学中的那种类似。在分析超短脉冲时失效。
\end{itemize}

我们仍然可以定义折射率的概念。
TODO:$n^2 = 1 + \chi$

\section{二阶非线性极化的波动光学}

之前已经讨论过一个问题:具有两个频率分量的泵浦光被打入具有二阶非线性极化的材料中,计算和频光的强度。本节用非线性波动方程重新计算它。
二阶非线性极化意味着材料的中心反演对称性破缺。

\subsection{SFG过程的一阶微扰}

\subsubsection{输入光为两束泵浦光}

我们考虑一个最为简单的情况:各向同性介质,泵浦光足够强,以至于在介质中泵浦光几乎没有衰减,即做了\concept{泵浦波无衰减近似}。
假定泵浦光中有两个频率分量的光,它们的偏振方向均相同。取其波矢指向为$z$轴。
我们用$E_i$指代第$i$种傅里叶分量,用$A_i$表示其大小,即
\[
    E_i = \ee^{\ii k_i z - \ii \omega_i t} A_i + \ee^{- \ii k_i z + \ii \omega_i t} A_i^*.
\]
在线性光学中只取正频率或是负频率的分量不会造成任何问题,并且通常会让问题更加容易,但是我们现在在讨论非线性过程,因此$E$必须是实际的入射光,显然必须是实数。
这样,非线性极化导致的一阶微扰中%
\footnote{
    “阶”在本文中有两个意思,一个是非线性光学过程的阶数,$n$阶非线性光学过程就是$n$束光变成一束光;还有一个意思是微扰计算的阶数。
    前者描述单个顶角的粒子线数目,后者描述一张图中顶角的总数。
}%
,SFG过程给出修正(我们为何只考虑SFG过程,而且只考虑$\omega_2$和$\omega_3$指定的SFG过程马上就可以看到;相位匹配条件决定了SFG过程和DFG过程等不太可能同时很重要)
\[
    \left( \pdv[2]{z} - \frac{\epsilon_\text{r}}{c^2} \pdv[2]{t} \right) E_3 = \frac{1}{\epsilon_0 c^2} \pdv[2]{t} \left( 2 \epsilon_0  \chi^{(2)} A_1 A_2 \ee^{\ii (k_1 + k_2) z} \ee^{-\ii \omega_3 t} \right) + \text{h.c.},
\]
其中我们设$\omega_3 = \omega_1 + \omega_2$,$\chi^{(2)}$指的是$\chi^{(2)}(\omega_3=\omega_1+\omega_2)$。
请注意没有什么能够保证线性极化是频率无关的——没有理由认为$\omega=\omega(k)$是线性的。
$E_3$的频率当然还是$\omega_3$。我们假定$E_3$大体上仍然是一个平面波(即做了\concept{平面波近似}),但是随着$z$增大,$A_3$会有变化。这样就有
\[
    \ee^{\ii k_3 z - \ii \omega_3 t} \left( \pdv[2]{z} + 2 \ii k_3 \pdv{z} - k_3^2 + \frac{\epsilon_\text{r}}{c^2} \omega^2 \right) A_3 = - \frac{\omega_3^2}{\epsilon_0 c^2} 2 \epsilon_0  \chi^{(2)} A_1 A_2 \ee^{\ii (k_1 + k_2) z} \ee^{-\ii \omega_3 t} .
\]
由于色散关系可以不是线性的,从而$k_3$和$k_1 + k_2$完全可以不同。
上式左边括号中最后两项抵消了。再做\emph{慢变振幅近似}(几何光学中就做了这一近似,也可以称为旁轴近似),就有
\[
    \pdv{z} A_3 = \frac{\ii \omega_3^2}{k_3 c^2} \chi^{(2)} A_1 A_2 \ee^{\ii (k_1 + k_2 - k_3) z}.
\]
求解这一方程,考虑到泵浦光中没有$E_3$分量,从而$z=0$处$A_3=0$,以及晶体厚度为$L$,就得到
\begin{equation}
    A_3(L) = \frac{\ii \chi^{(2)} \omega_3^2 A_1 A_2}{k_3 c^2} \left( \frac{\ee^{\ii \Delta k L} - 1}{\ii \Delta k} \right),
    \label{eq:two-pump-a3}
\end{equation}
其中
\begin{equation}
    \Delta k = k_1 + k_2 - k_3.
\end{equation}
$E_3$分量的强度是
\[
    I_3 = \expval*{\epsilon_0 n c E_3^2} = 2 \epsilon_0 n c \abs*{A_3}^2,
\]
于是就得到
\begin{equation}
    I_3 = \frac{8 \epsilon_0 (\chi^{(2)})^2 \abs*{A_1}^2 \abs*{A_2}^2 \omega_3^2}{nc} \frac{\sin^2(\Delta k L / 2)}{(\Delta k)^2} = \frac{(\chi^{(2)})^2 \omega_3^2 I_1 I_2}{2 \epsilon_0 c^3 n_1 n_2 n_3} \mathrm{sinc}^2 \left( \frac{\Delta k L}{2} \right) L^2.
    \label{eq:sfg-intensity}
\end{equation}

如果\concept{相位匹配条件}——即$k_3 = k_1 + k_2$——的确成立,那么$I_3$就稳定地随着$L^2$增大而线性增大,此时的SFG转换效率是非常高的。
当然,$L$大到一定程度时,$I_1$和$I_2$就开始随着$L$增大而下降,并且吸收也开始明显了。
大部分情况下相位匹配条件是没法成立的,因为色散非线性。此时随着$L$的增大,能量先是从$E_1$和$E_2$移向$E_3$模式,然后再从$E_3$模式移向$E_1$和$E_2$模式。
大约需要
\begin{equation}
    L_\text{coh} = \frac{1}{\Delta k}
\end{equation}
的距离可以看到明显的能量转移,这个长度称为\concept{相干长度}。

相位匹配条件叫做这个名字并非没有原因。在某个位置$z$处新产生的$E_3$光大体上形如
\[
    E_3 \sim E_1 E_2 \sim A_1 A_2 \ee^{\ii (k_1 + k_2) z - \ii (\omega_1 + \omega_2) t},
\]
如果$\Delta k \neq 0$,那么新产生的$E_3$光和原来已有的$E_3$光就会有一个不断变化的相位差,在一些地方它们会发生相消干涉。
这就是$\Delta k \neq 0$时$E_3$光强度发生振荡的原因。

相位匹配时才有较高效的能量转换这一事实实际上大大简化了我们的分析。
原则上,我们将频率为$\omega_1$和$\omega_2$的光输入一个非线性晶体,所有过程——和频,差频,倍频,光学整流——都会发生,因此我们\emph{不能}只取一个过程并按照它写出振幅分布方程。
例如,种子光可以因为DFG过程而指数上升,也可以因为SFG过程而振荡,这两个效应按理说应该叠加,然后反过来影响SFG和DFG产生的光的强度。
然而,很多时候,入射光给定之后SFG和DFG过程或者其它一些过程不能同时满足相位匹配条件,从而\eqref{eq:sfg-intensity}分母中的$\Delta k$会压低$I_3$能够达到的最大值。
因此,其实可以分开计算每个过程。
在有广谱的输入光——如制作光谱仪时——这个近似可能不再适用,此时需要做完整的三波混频方程求解。

\subsubsection{输入光为泵浦光和种子光}

本节考虑一个有些不同的情况:输入光还是有两个频率分量,但是其中一个频率分量是很弱的,从而SFG过程产生的光的幅度振荡的同时,那个比较弱的光的幅度也会振荡。
用1表示泵浦光,2表示那个频率分量较弱的输入光(即所谓种子光),3表示SFG过程产生的光。
此时的非线性麦克斯韦方程为
\[
    \begin{aligned}
        \left( \pdv[2]{z} - \frac{\epsilon_\text{r}}{c^2} \pdv[2]{t} \right) E_3 &= \frac{1}{\epsilon_0 c^2} \pdv[2]{t} \left( 2 \epsilon_0  \chi^{(2)} A_1 A_2 \ee^{\ii (k_1 + k_2) z} \ee^{-\ii \omega_3 t} \right) + \text{h.c.}, \\
        \left( \pdv[2]{z} - \frac{\epsilon_\text{r}}{c^2} \pdv[2]{t} \right) E_2 &= \frac{1}{\epsilon_0 c^2} \pdv[2]{t} \left( 2 \epsilon_0  \chi^{(2)} A_1^* A_3 \ee^{\ii (- k_1 + k_3) z} \ee^{-\ii \omega_2 t} \right) + \text{h.c.}.
    \end{aligned}
\]
使用和上一节类似的慢变振幅近似,我们有
\begin{equation}
    \begin{aligned}
        \pdv{A_3}{z} &= \frac{\ii \omega_3^2}{k_3 c^2} \chi^{(2)} A_1 A_2 \ee^{\ii \Delta k z}, \\
        \pdv{A_2}{z} &= \frac{\ii \omega_2^2}{k_2 c^2} \chi^{(2)} A_1^* A_3 \ee^{- \ii \Delta k z}.
    \end{aligned}
    \label{eq:pump-and-seed-sfg}
\end{equation}
我们需要尝试将$A_2$和$A_3$的方程解耦。在相位匹配条件成立的情况下,只需要让以上两个方程两边各自对$z$再求一次导数,即可得到分别关于$A_2$和$A_3$的两个二阶微分方程
\[
    \pdv[2]{A_3}{z} = - \frac{\omega_3^2 \omega_2^2}{k_2 k_3 c^4} \abs*{A_1}^2 (\chi^{(2)})^2 A_3, \quad \pdv[2]{A_2}{z} = - \frac{\omega_3^2 \omega_2^2}{k_2 k_3 c^4} \abs*{A_1}^2 (\chi^{(2)})^2 A_2.
\]
在相位匹配条件不成立的时候,对\eqref{eq:pump-and-seed-sfg}中每个方程求导时等式右边会多出来一个$A_2$或$A_3$项,让求解变得困难。
然而,根据之前“相位不匹配会导致新产生的光和已有的光在一些地方发生相消干涉”的物理图像,我们仍然可以确信,相位不匹配会导致能量转化效率下降。

在相位匹配条件成立时,代入边界条件$A_3(0) = 0$,解得
\begin{equation}
    A_2(z) = A_2(0) \cos(k z), \quad A_3(z) = \ii \sqrt{\frac{\omega_3 n_2}{\omega_2 n_3}} \frac{A_1}{\abs*{A_1}} A_2(0) \sin(k z),
\end{equation}
其中
\begin{equation}
    k^2 = \frac{\omega_3^2 \omega_2^2}{k_2 k_3 c^4} \abs*{A_1}^2 (\chi^{(2)})^2. 
\end{equation}
我们发现一开始,能量从$\omega_2$模式转向$\omega_3$模式,但是一段时间之后能量又从$\omega_3$模式转向$\omega_2$模式。
两个耦合的模式如此来回振荡。存在能量从$E_3$模式流向$E_2$模式的过程意味着在完全经典的计算中,SFG过程的逆过程也是可行的,虽然由于没有量子涨落,必须有种子光这个过程才能发生。
在本节计算的例子中,泵浦光$E_1$起到了$\omega_3$转化为$\omega_1$和$\omega_2$的种子光的作用。

\subsection{DFG过程的一阶微扰}

我们现在转而考虑DFG过程。前面提到过,经典理论中DFG过程不能从一束光凭空产生两束光。
于是我们考虑一个这样的过程:$E_3$是基本上不衰减的泵浦光,它将要产生$E_1$和$E_2$两束光;$E_1$光也是输入光,即$E_1$模式上有种子光;$E_2$在$z=0$处为零。
使用慢变振幅近似,有以下方程:
\[
    \begin{aligned}
        \pdv{A_1}{z} &= \frac{\ii \chi^{(2)}(\omega_1 = \omega_3 - \omega_2) \omega_1^2}{k_1 c^2} A_3 A_2^* \ee^{-\ii \Delta k z}, \\
        \pdv{A_2}{z} &= \frac{\ii \chi^{(2)}(\omega_2 = \omega_3 - \omega_1) \omega_2^2}{k_2 c^2} A_3 A_1^* \ee^{-\ii \Delta k z}.
    \end{aligned}
\]
显然我们可以指定
\begin{equation}
    \chi^{(2)}(\omega_1 = \omega_3 - \omega_2) = \chi^{(2)}(\omega_2 = \omega_3 - \omega_1) = \chi^{(2)}.
\end{equation}
求解以上方程,得到
\begin{equation}
    A_1(z) = A_1(0) \cosh(\kappa z), \quad A_2(z) = \ii \sqrt{\frac{n_1 \omega_2}{n_2 \omega_1}} \frac{A_3}{\abs*{A_3}} A_1^*(0) \sinh(\kappa z),
\end{equation}
其中
\begin{equation}
    \kappa^2 = \frac{\omega_1^2 \omega_2^2}{k_1 k_2 c^4} \abs*{A_3}^2 (\chi^{(2)})^2 .
\end{equation}
这里,$\omega_1$和$\omega_2$光都指数增长。这里的关键点在于关于$A_1$的方程右边的$A_2$取了复共轭。
物理图像上,单泵浦光的DFG过程中,$\omega_1$光子的出现能够刺激$\omega_2$光子的出现,反之亦然,因此一旦有$\omega_1$光和泵浦光同时出现,$\omega_1$光和$\omega_2$光就会不断扩增,从而指数增长。
反之,在单泵浦光输入的SFG过程中,产生$\omega_3$光子会消耗$\omega_2$光子,但是$\omega_3$光子不能诱发更多$\omega_1$光子转化为$\omega_2$光子,从而$\omega_2$光子和$\omega_3$光子存在竞争关系。

从以上解可以看到一个有趣的现象,就是种子光$\omega_1$的相位其实是自行决定的,而$\omega_2$光的相位同时由泵浦光$\omega_3$和和种子光$\omega_1$的相位决定。
这意味着非线性晶体不仅可以用来从一个频率的光源产生另一个频率的光源,还可以用于产生相位特定的新光源。

\subsection{相位匹配条件的实现}

如前所述,只有相位匹配时SFG过程或是DFG过程才足够明显。本节讨论给定三个任意频率的光,如何让它们能够满足相位匹配条件。
由于介质色散的存在,相位匹配条件不总是能够完成的,因为联立方程
\[
    \omega_3 = \omega_1 + \omega_2, \quad \omega_3 n(\omega_3) = \omega_2 n(\omega_2) + \omega_1 n(\omega_1)
\]
未必有解。实际上,在所谓的正常折射率的情况下——即在$n$随着$\omega$增大而增大的情况下——这个方程就是无解的,因为显然
\[
    \omega_3 > \omega_1, \omega_3 > \omega_2,
\]
从而
\[
    \omega_3 n(\omega_3) > \omega_1 n(\omega_3) + \omega_2 n(\omega_3) > \omega_1 n(\omega_1) + \omega_2 n(\omega_2).
\]
当然,为了实现相位匹配,我们可以去寻找一种特殊的材料,它在$\omega_3$附近有反常折射率,从而让相位匹配条件能够成立。但这样的材料显然不那么好找。

\subsubsection{双折射}

一种获得相位匹配的方式是使用双折射晶体。
Type I

\begin{equation}
    n^\text{e}_3 \omega_3 = n^\text{o}_1 \omega_1 + n^\text{o}_2 \omega_2
\end{equation}

\begin{equation}
    n^\text{e}_3 \omega_3 = n^\text{o}_1 \omega_1 + n^\text{e}_2 \omega_2
\end{equation}
这样的好处在于可以通过偏振将两束光分开。

问题:Walk off angle,走移角,

如果关心长距离

\subsubsection{准相位匹配}

在$\Delta k$非零但不大——即所谓\concept{准相位匹配}——时,可以通过这样的方法获得高效率的SFG转换:将一系列$\chi^{(2)}$指向周期性倒转的二阶非线性晶体贴在一起,让第一块晶体的厚度是$\pi L_\text{coh} / 2$,后面所有的晶体的厚度都是$\pi L_\text{coh}$。
这样,根据\eqref{eq:sfg-intensity},走过第一块晶体时,$I_3$强度打到最大值,此时能和$E_3$光发生相长干涉的光的相位是$\pi$;随后进入第二块晶体,$\chi^{(2)}$倒转,根据\eqref{eq:two-pump-a3},这意味着新产生的$E_3$光获得一个$\pi$的相位,于是第二块晶体内仍然发生了相长干涉;在第二块晶体和第三块晶体的交界处,能和$E_3$发生相长干涉的光的相位是$0$,$\chi^{(2)}$再次倒转,根据\eqref{eq:two-pump-a3},在第三块晶体中产生的$E_3$光的相位是$0$,于是还是相长干涉……
如此重复即可获得持续增加的$\omega_3$光,虽然其增速不如相位完全匹配时的情况。

准相位匹配已经是成熟的技术。PPLN装置等铁电体阵列是实现准相位匹配的常用装置。

准相位匹配实际上说明了一点,就是晶体的性质有空间起伏时,之前的诸如相位匹配条件的东西都是不能直接适用的。
这就提示我们,晶体内部的元激发——声子或者别的什么——可以借此和光耦合。

\subsection{二阶非线性效应的各种应用}

光纤:?

设我们手头上只有一个频率的光——比如说\SI{800}{nm}的光——而需要得到一束频率低一些的光——比如说\SI{1500}{nm}的光。没有别的光源可用。
这时候可以这么做:首先将\SI{800}{nm}光尽可能聚焦到一个材料中,让super continue generation发生,从而得到一个非常宽的频谱(并且很弱),然后将纯净的\SI{800}{nm}光和这个宽谱光入射到一个二阶非线性晶体当中,这样这两者就分别起到了泵浦光和种子光的作用。
当然,DFG过程产生的光的频率也是宽谱的,但是请注意只有满足相位匹配条件的过程才是最可能发生的。
因此我们可以通过选用适当的非线性晶体,调整温度、入射角度等,让\SI{800}{nm}光变成\SI{1500}{nm}光的过程正好符合相位匹配条件,于是就得到了相当纯净的\SI{1500}{nm}输出光。


OPO装置:入射光

\subsubsection{荧光信号的时间分辨}

荧光持续的时间很长,但是很弱,我们可以在需要仔细分析的时间段制备一个脉冲,将荧光和这个脉冲同时输入一个非线性晶体,就可以把我们需要仔细观察的那一段分离出来。

\subsubsection{光谱学}

\section{三阶非线性极化的波动光学}

光学Korre效应,受激拉曼效应,受激光栅(让光自己产生干涉,然后非线性效应让折射率发生周期性变化,再来一束光,就发生了衍射),CARS

将一个波包压缩成一个阿秒级别的脉冲

\subsection{自相互作用,或者说简并四波混频}

三阶非线性极化允许这样的过程发生:$\omega = \omega - \omega + \omega$,即一束光可以同时提供三个光子,产生同一频率的一个光子。
也即,三阶非线性极化允许单频光自相互作用。在二阶非线性极化中没有这种现象。

\subsubsection{自聚焦}

波列很长的波进入一个各向同性的、中心反演对称的非线性晶体,我们会发现它的折射率大体上是
\begin{equation}
    n = n_0 + \bar{n}_2 \expval*{\vb*{E}(t)^2} = n_0 + 2 \bar{n}_2(\omega) \abs*{\vb*{E}(\omega)}^2,
    \label{eq:ref-index-change}
\end{equation}
其中$n_0$是弱光的折射率而$n_2$是一个二阶折射系数。

如果我们只考虑三阶非线性极化,就有
\[
    \vb*{P}_\text{NL} = 3 \epsilon_0 \chi^{(3)}(\omega=\omega + \omega - \omega) : \vb*{E}(\omega) \vb*{E}(\omega) \vb*{E}(\omega)^*,
\]
在介质各向同性的情况下可以将$\chi^{(3)}$用一个标量代替,它后面的三个电场的乘积和单位张量缩并,于是
\[
    \begin{aligned}
        \vb*{P} &= \epsilon_0 \chi^{(1)}(\omega) \vb*{E}(\omega) +  3 \chi^{(3)}(\omega=\omega + \omega - \omega) : \vb*{E}(\omega) \vb*{E}(\omega) \vb*{E}(\omega)^* \\
        &= \epsilon_0 (\chi^{(1)}(\omega) + 3 \chi^{(3)}(\omega=\omega+\omega-\omega) \abs*{\vb*{E}(\omega)}^2) \vb*{E}(\omega).
    \end{aligned}
\]
折射率的定义为
\[
    n^2 = 1 + \chi,
\]
取小量近似,就得到
\begin{equation}
    n_0(\omega)^2 = 1 + \chi^{(1)}(\omega), \quad \bar{n}_2(\omega) = \frac{3}{4n_0} \chi^{(3)}(\omega=\omega-\omega+\omega).
\end{equation}

我们还可以将折射率写成光的强度
\begin{equation}
    I = \frac{\epsilon_0 c n_0}{2} \abs*{\vb*{E}(\omega)}^2
\end{equation}
的函数,即
\begin{equation}
    n = n_0 + n_2 I,
\end{equation}
其中
\begin{equation}
    n_2 = \frac{4}{\epsilon_0 c n_0} \bar{n}_2.
\end{equation}

折射率会随着入射光强而变化这件事意味着进入三阶非线性晶体的光会\concept{自聚焦}。
如果$\bar{n}_2 > 0$,那么光束中心的光被偏折得更厉害,光束相当于经过了一个凸面镜;反之光束相当于经过了一个凹面镜。
这个等效“透镜”的行为和频率相关,因此自聚焦可以用来设计一个频率筛选装置,即可以用于锁频。
例如,可以在一个谐振腔内部放置两个自聚焦晶体,则只有频率适当的光能够在谐振腔中稳定地来回传播,频率不适当的光经过多次成像,会散得越来越开。

\subsubsection{自相位调节}

光波经过介质之后(相比另一束没有经过介质的光)会有相位变化,而由于$\bar{n}_2$的存在,光波经过一个三阶非线性晶体之后会有一个额外的相位变化
\begin{equation}
    \phi_\text{NL}(t) = \frac{\omega}{c} L n_2 I(t).
    \label{eq:self-phase-adjustement}
\end{equation}
仪器把一束光大体上当成单频光而测它的频率(即所谓\concept{即时频率},对单色光它就是光的频率,对有多频率光它大概是波包的中心频率),就得到
\[
    \omega_\text{temp} = \pdv{\phi}{t},
\]
于是从三阶非线性晶体出来的光的即时频率会因为非线性效应而变化
\begin{equation}
    \Delta \omega_\text{temp} = - \frac{n_2 \omega L}{c} \pdv{I}{t},
    \label{eq:temp-phase-change}
\end{equation}
即一束多频率的光经过时前面的会看起来更红,后面的会看起来更蓝。
把这束光做频谱分析会发现频谱变宽了。频谱变宽了,时域的波包尺度就会变窄——因此可以用三阶非线性晶体做一个超快激光。
例如可以用三阶非线性晶体做一系列薄片,波包每经过一个薄片,在时域的展宽就窄一些。
我们这里不用一个非常厚的晶体,因为随着频谱变宽,波包的强度会下降,从而$\pdv*{I}{t}$也会下降。因此一个特别长的晶体除了让光束散开来以外并没有什么用处。
同样,表面上,虽然弱光的$I$并不大,但我们可以增大$L$去调节它的相位,但其实这是不现实的。

设脉冲在时间上持续了$\tau$,则频谱宽度为
\[
    \Delta \omega \sim \frac{2\pi}{\tau},
\]
而即时频率的变化为$\Delta \phi_\text{NL} / \tau$,因此为了让频率展宽足够明显,应当有
\begin{equation}
    \Delta \phi_\text{NL} \sim 2\pi.
\end{equation}

\subsubsection{光学孤子}

自相位调节其实提醒我们一点:可能可以使用一些特殊的非线性效应反过来补偿色散导致的不同频率的光的相位差,从而让介质中能够产生孤子,这是一个波包,它能够稳定地在介质中传播,而不发生波包展宽。
在光纤中这已经有了应用,一些时候可以制造一个展宽为微米级的孤子。

我们设有一个波包
\begin{equation}
    E(z, t) = A(z, t) \ee^{\ii (k_0 - \omega_0 t)},
    \label{eq:wave-package}
\end{equation}
其中$k_0$和$\omega_0$是中心波矢和频率,$A(z, t)$的时间和频率依赖给出波包的(随时间变化的)包络线。我们通常认为中心波矢和频率之间的关系是遵从线性折射率的,即
\begin{equation}
    k_0 = n_0(\omega) \frac{\omega_0}{c}.
\end{equation}
我们不区分线性和非线性效应,统一地求解非线性波动方程
\[
    \pdv[2]{E}{z} - \frac{1}{\epsilon_0 c^2} \pdv[2]{D}{t} = 0.
\]
做傅里叶变换
\[
    E(z, t) = \int \frac{\dd{\omega}}{2\pi} \ee^{- \ii \omega t} E(z, \omega), \quad D(z, t) = \int \frac{\dd{\omega}}{2\pi} \ee^{- \ii \omega t} D(z, \omega),
\]
并且
\[
    D(z, \omega) = \epsilon(\omega) E(z, \omega),
\]
其中$\epsilon(\omega)$是(带有非线性效应,依赖于$E$的)介电常数。
我们于是得到非线性版本的亥姆霍兹方程
\begin{equation}
    \pdv[2]{E(z, \omega)}{z} + \epsilon(\omega) \frac{\omega^2}{c^2} E(z, \omega) = 0.
    \label{eq:nonlinear-freq-domain-eq}
\end{equation}
我们能有幸得到形式这么好的方程当然归功于材料本身没有时间演化,否则在频域中$D$和$E$之间的关系就不是简单的“乘以一个系数”了。
我们根据\eqref{eq:wave-package}以及$k_0$和$\omega_0$之间的关系是线性色散关系,得到
\begin{equation}
    E(z, \omega) = A(z, \omega - \omega_0) \ee^{\ii k_0 z } + \text{c.c.}.
\end{equation}

我们考虑慢变振幅近似。需要注意的是,此时这个近似是可能会失效的,几百飞秒的波包仍然满足这个近似,再小一些可能就失效了。
我们进一步假定波包的频谱展宽相比于$\omega_0$是很小的。
这样,\eqref{eq:nonlinear-freq-domain-eq}就变成
\begin{equation}
    2 k_0 \pdv{A(z, \omega - \omega_0)}{z} + (k^2 - k_0^2) A(z, \omega - \omega_0) = 0.
    \label{eq:wave-package-amplitude}
\end{equation}
做泰勒展开
\[
    k = k_0 + \Delta k_\text{NL} + k_1 (\omega - \omega_0) + \frac{1}{2} k_2 (\omega - \omega_0)^2 + \cdots,
\]
其中$k_\text{NL}$为非线性效应导致的自相位调节。这样,\eqref{eq:wave-package-amplitude}就成为
\begin{equation}
    \pdv{A}{z} - \ii \Delta k_\text{NL} A - \ii k_1 (\omega - \omega_0) A - \frac{1}{2} \ii k_2 (\omega - \omega_0)^2 A = 0.
\end{equation}
做傅里叶反变换
\[
    A(z, t) = \int \frac{\dd{\omega}}{2\pi} A(z, \omega - \omega_0) \ee^{- \ii (\omega - \omega_0) t},
\]
得到
\begin{equation}
    \pdv{A}{z} + k_1 \pdv{A}{t} + \frac{1}{2} \ii k_2 \pdv[2]{A}{t} = \ii \Delta k_\text{NL} A.
    \label{eq:wave-package-evolve}
\end{equation}
这其中,$k_1$和$k_2$是线性色散的一阶和二阶泰勒展开系数,它们是
\begin{equation}
    k_1 = \left(\pdv{k}{\omega}\right)_{\omega = \omega_0} = \left( \frac{1}{v_\text{g}} \right)_{\omega = \omega_0},
\end{equation}
以及
\begin{equation}
    k_2 = \left( \pdv[2]{k}{\omega} \right)_{\omega = \omega_0} = - \left( \frac{1}{v_\text{g}} \dv{v_\text{g}}{\omega} \right)_{\omega = \omega_0}.
\end{equation}
自相位调节为(把\eqref{eq:self-phase-adjustement}右边除以$L$就得到)
\begin{equation}
    \Delta k_\text{NL} = n_2 I \frac{\omega_0}{c}.
\end{equation}

如果\eqref{eq:wave-package-evolve}中完全没有非线性光学效应,并且$k$和$\omega$之间的关系不是线性的,那么就会出现波包展宽,因为此时$k_2 \neq 0$,做代换
\[
    \tau = t - \frac{z}{v_\text{g}},
\]
得到% TODO:衰减
同理,自相位调节也会导致波包展宽。如果我们要求
\begin{equation}
    \frac{1}{2} k_2 \pdv[2]{A}{t} = \Delta k_\text{NL} A,
\end{equation}
那么就不会有任何波包展宽。一个例子是
\begin{equation}
    A(z, \tau) = A_0 \sech(\tau / \tau_0) \ee^{\ii k z},
\end{equation}
虽然以上求解过程似乎要求波包要具有特定的形状,这样才能形成孤子,但是实际上,一些形状不那么好的波包输入材料之后其实也能形成孤子,因为不符合波包形状要求的那些频率成分由于色散都各自跑远了,只留下一个孤子波包。

\subsubsection{自陷}

想象一束已经被聚焦过了,然后被输入一个三阶非线性晶体。例如,可以将三阶非线性晶体放在一个高斯光的光腰上。
自聚焦现象如期发生,让光变得更强,然后自聚焦进一步增强……如果几何光学总是适用,那么最终光束将终结到一个点上。
当然,在此之前衍射已经变得明显了。
这里发生的事情就好像“外压和量子涨落的竞争”(我们会看到“量子涨落”并不只是比喻),最终形成一个光束尺寸相对稳定的\concept{光丝}。这就是所谓的\concept{自陷}。

光丝并不是一个非常稳定的状态,因为如果介质中有什么东西散射了一下光丝,它的直径就会增大,光强变小,于是自聚焦的逆过程开始发生,最后光束又四散开去。
但如果光丝足够强,它可能已经将介质内部打出一个等离子体通道了,这个时候支配光丝所在区域的光学性质的不是三阶非线性极化,而是等离子体的光学,光丝也就这样一直传播下去了。

我们来估算什么时候自聚焦会发生。假定自聚焦区域内的折射率大体上是均一的(从而自聚焦区域内的$I$是常数,一旦出了自聚焦区域,就快速衰减为零),则自聚焦区域的边界上的临界角为
\[
    \cos \theta_0 = \frac{n_0}{n_0 + \var{n}}, \quad \var{n} = n_2 I,
\]
射向边界而入射角大于这个角的光将被反射回去,即不会溢出光丝。
另一方面,光丝会有衍射,即在偏离入射光的波矢的地方仍然有光传播,本质上这是因为长得像平面波的光束不可能具有有限直径——光丝内部类似于平面波,但是它有有限大小的直径,我们在偏离入射光的波矢的方向上计算总电场,是能够得到非零结果的。
我们也可以说这是位置和动量的不确定性:光丝的位置是比较确定的,从而“传播方向”是不完全确定的。
我们借用孔径衍射的公式(因为这可以算是一个孔径衍射),衍射角为
\[
    \theta_d = \frac{0.61 \lambda_0}{n_0 d}.
\]
如果很多衍射光的衍射角小于全反射临界角,那么衍射会破坏光丝,而反之光丝可以进一步聚焦。
因此,平衡时,$\theta_0 \sim \theta_d$。做近似
\[
    \cos\theta = 1 - \frac{\theta^2}{2},
\]
并且注意到$n_2 I$无论如何相比$n_0$都是非常小的,我们能够得到
\begin{equation}
    d \sim \frac{0.61 \lambda_0}{\sqrt{2 n_0 n_2 I}} .
\end{equation}
这给出了指定波长、线性折射率和自聚焦效应之后,形成稳定光丝的直径。
我们马上可以,这对应一个功率
\begin{equation}
    P_\text{cr} = \frac{\pi d^2}{4} I = \frac{\pi 0.61^2 \lambda_0^2}{8 n_0 n_2}.
\end{equation}
这是一个完全确定的功率值,不多也不少。这表明光丝形成的条件是非常苛刻的。如果入射光功率非常小,那么自聚焦根本不足以让光丝形成,而如果入射光功率很大,那么

入射光仍然需要走过一段距离才能够形成光丝。

\subsection{光学共轭和非线性光栅}

全息??

干涉条纹的空间频率是波矢差;干涉条纹因为自相位调节导致折射率光栅

无损光强测量和光开关

谐振腔的光强双稳

\section{光和介质中模式的散射}

布里渊散射,拉曼散射;

\subsection{经典电磁波}

Raman散射可以用一些经典图景分析。我们知道介质中的电子在振动,于是将$\alpha$泰勒展开到一阶,得到
\begin{equation}
    \vb*{P} = \alpha \vb*{E}, \quad \alpha(t) = \alpha_0 + \pdv{\alpha}{Q} Q(t),
\end{equation}
设介质中谐振子正在以$\omega_q$振荡,泵浦光频率为$\omega_l$,则
\begin{equation}
    \vb*{P}(t) = \alpha_0 \vb*{E}_0 \cos(\omega_l t) + \frac{1}{2} \pdv{\alpha}{Q} \vb*{E}_0 Q_0 (\cos(\omega_l + \omega_q) t + \cos(\omega_l - \omega_q)t).
\end{equation}
可以看到我们有三个过程:一个是普通的$\omega_l$光的传播,一个是\concept{Stokes过程},即输出光频率为$\omega_l - \omega_q$,还有一个是\concept{反Stokes过程},即输出频率为$\omega_l + \omega_q$。

以上经典图景无法给出一些需要“能级布居数”才能解释的东西。例如,基本上低能级上的电子数目要远大于高能级,因此Stokes过程总是比反Stokes过程容易发生。
实际上这个原理可以用来测定一个已知能谱的系统的温度,因为通过比较Stokes过程和反Stokes过程的发生几率来确定能级布居数,从而推算出温度。
例如,通过声子和光子的耦合,我们可以测量出固体晶格的温度。
分析Raman过程中的$\omega_q$也可以用于确定系统中的各个能级的相对能量差。

在有磁场存在,且两个能级的间距相比磁场并不大的时候,Stokes

现在我们改用量子力学做计算。由于并没有对光场做量子化,只需要做含时微扰即可。
\[
    \dv{W}{\omega} = \frac{2\pi}{\hbar} g(\omega) \abs{\sum_n \left(
        \frac{\mel*{f}{e \vb*{r} \cdot \vb*{E}}{n} \mel*{n}{e \vb*{r} \cdot \vb*{E}}{g}}{\omega_l - }
    \right)}^2
\]

我们在这里遇到了之前遇到过的类似的问题:如果没有频率为$\omega_l + \omega_q$的种子光,似乎拉曼光不能产生。
当然,这是因为我们没有考虑电磁场的量子涨落。我们现在将光场也视为量子化的,即

\begin{equation}
    \dv]{W}{\omega_s} = \frac{8\pi^3 N \omega_l \omega_s}{n_s^2 n_l^2 V} \abs*{\mel{f}{M}{g}}^2 g(\omega) \abs*{\mel*{\alpha_f}{a_s^\dagger a_l}{\alpha_s}}^2
\end{equation}

受激Raman效应中Stokes光似乎可以以$\ee$指数增加。这个机制称为\concept{Stokes激光},是一种产生激光的机制。
Stokes光的相位和
重点:是否能够保证$\omega_s$是相干的;如果一个材料中的元激发(光学声子、自旋波等)、泵浦光和Stokes光能够持续耦合,“滚着往前跑”,那么就会有比较棘手的东西出现。

受激Raman效应可以用于展宽光谱。

在计算完单次光子-材料中模式的Raman散射之后,我们可以把有关计算结果用于确定宏观参数。我们有
\begin{equation}
    \abs{\vb*{E}_s}^2 = \ee^{G_R z - \alpha_s z} \abs{\vb*{E}_s(0)}^2,
\end{equation}
其中
\begin{equation}
    G_R = \gamma m_l \propto \frac{N}{V} \dv{\sigma}{\Omega} \frac{m_l}{\Gamma},
\end{equation}
其中
\[
    fuck%\frac{N}{V} \dv{\sigma}{\Omega} \sim \SI{10^{-8}}{cm^{-1}}, \quad \Gamma \sim \SI{1}{cm^{-1}}, \quad \gamma \sim %\SI{10^{-3}}{\cm/MW}
\]

使用拉曼散射还可以做高精度测量。其测定精度

拉曼散射

\subsection{自发发射}



\subsection{广义的三波混频}

传播相位和天线相位不同导致相位差。

\subsection{基于拉曼散射的激光冷却}

由于斯托克斯过程和反斯托克斯过程完全是等价的,可以设想,既然能够将激光照在介质上向介质提供能量,当然也可以设法将介质的能量转移到激光中。
实际上,我们可以制备一个腔共振线正好落在反斯托克斯过程上的谐振腔,然后将一些分子放在腔中,并入射激光。
其结果是反斯托克斯过程受到激励,分子不仅没有从激光吸收能量,反而被冷却了。

\subsection{应用}

\subsubsection{生物成像}

生物体中不同组分的吸收波长的重合不大。
自发拉曼散射足够用于做静态成像。如果要做动态成像,自发拉曼散射发生的速率就太慢了,必须要通过受激拉曼散射完成测定。

\section{布里渊散射}

设介质中有声波,则介质性能出现空间起伏,有
\begin{equation}
    \chi(\vb*{r}) = \chi_0 \cos(\vb*{G} \cdot (\vb*{r} - \vb*{v}_{\vb*{G}} t)), \quad \omega_\text{S} = \vb*{G} \cdot \vb*{v}_{\vb*{G}}.
\end{equation}
毫无疑问这会导致一定的光学效应:我们想象一系列折射率不同的介质被贴在一起,一束光被打入其中,则每一层界面都会有微弱的反射,在条件适当的情况下这些反射光相长干涉,产生明显的总的反射。%
\footnote{
    实际上,这样可以获得非常好的反射镜,比镀银的好得多。通过这种方法可以获得5到6个9的反射率。
}%
$\chi$出现周期性起伏的介质就是这样一种系统。
另一个值得注意的地方是反射光会有小的频率变化,因为反射面在动,会有多普勒效应。

在有了以上直觉性的考虑之后我们开始解方程。做通常的拟设

受激布里渊散射会限制光线通讯的激光功率。光纤存在热涨落,时不时就会自发出现声子,

TODO:Kerr effect

\section{超快脉冲}

\subsection{离子化}

direct ionization, multiphoton ionization, tunnel ionization

\part{光的量子性质}

\chapter{量子化的光场}

\section{线性介质中的光场量子化}

电磁场足够强以至于难以看到单光子效应,而又足够弱以至于能量不至于强到需要考虑量子电动力学的圈图修正,这样就可以使用经典电动力学描述整个系统。
为了讨论电磁波的量子涨落(在分析诸如腔内辐射场,或是非线性光学中的DFG过程时非常重要,在做高精度测量时有时也要考虑),即使没有圈图效应,我们也要做光场的量子化。
这个做法的必要性将在后续的章节中多次体现出来,我们这里只是讨论量子化技术本身,暂时不考虑光的量子性在哪些情境下最为明显。

\subsection{真空}\label{sec:quantization-in-vacuum}

我们首先考虑真空中的光场的量子化,此时我们无非是在重复QED中的运算,实际上是在重复无质量矢量场的量子化(见\qftdoc中的\ref{qft-sec:massless-vector-quantize}节)。
QED中矢量场展开为
\begin{equation}
    A_\mu(\vb*{x}, t) = (\frac{\varphi}{c}, - \vb*{A}) = \int \frac{\dd[3]{\vb*{k}}}{(2\pi)^3} \sqrt{\frac{\hbar}{2\omega_{\vb*{k}} \epsilon_0}} \sum_{\sigma=1}^2 \left( a_{\vb*{k} \sigma} \epsilon_\mu^\sigma(\vb*{k}) \ee^{\ii \vb*{k} \cdot \vb*{x} - \ii \omega_{\vb*{k}} t} + a_{\vb*{k} \sigma}^\dagger \epsilon_\mu^\sigma(\vb*{k})^* \ee^{- \ii \vb*{k} \cdot \vb*{x} + \ii \omega_{\vb*{k}} t} \right),
    \label{eq:vector-field-components}
\end{equation}
其中电磁场模式为平面波。
取费曼规范,做一些分部积分并去掉表面项,得到
\begin{equation}
    \mathcal{L} = - \frac{1}{2 \mu_0} \partial_\mu A_\nu \partial^\mu A^\nu,
\end{equation}
从而正则动量为
\begin{equation}
    \pi^\mu = \pdv{\mathcal{L}}{\partial_0 A_\mu} = - \partial^0 A^\mu,
\end{equation}
可以据此写出正则量子化条件,即时间相同时,$A^\mu$同$A^\nu$对易,而
\begin{equation}
    [A^\mu(\vb*{x}, t), \pi^\mu(\vb*{y}, t)] = \ii \eta^{\mu \nu} \delta^{(3)}(\vb*{x} - \vb*{y}).
\end{equation}
哈密顿量为
\[
    \begin{aligned}
        H &= \int \dd[3]{\vb*{r}} (\pi^\mu \partial_0 A_\mu - \mathcal{L}) \\
        &= \int \dd[3]{\vb*{r}} \left( - \frac{1}{c^2} (\partial_t A^\mu)^2 + \frac{1}{2} \partial_\mu A_\nu \partial^\mu A^\nu \right),
    \end{aligned}
\]
这里要注意$x^0 = c t$。代入$A_\mu$的展开式计算得到
\begin{equation}
    H = \sum_{\sigma=1}^2 \int \frac{\dd[2]{\vb*{k}}}{(2\pi)^3} \hbar \omega_{\vb*{k}} \left( a^\dagger_{\vb*{k} \sigma} a_{\vb*{k} \sigma} + \frac{1}{2} \right), \quad \omega_{\vb*{k}} = c \abs*{\vb*{k}}.
\end{equation}
这就得到了量子化的能量。
在量子化过程中我们已经通过限制$\sigma$而施加了规范,不过这个规范并不是辐射规范,而是洛伦兹规范。横波条件通过$\epsilon$矢量和波矢垂直而保证。
不过,既然我们只关心电偶极辐射而有关的相互作用哈密顿量可以完全写成$\vb*{E}$,这也不重要。

从四维矢量计算电场,得到
\begin{equation}
    \vb*{E}(\vb*{r}, t) = \int \frac{\dd[3]{\vb*{k}}}{(2\pi)^3} \sqrt{\frac{\hbar}{2\omega_{\vb*{k}} \epsilon_0}} \sum_{\sigma=1}^2 \left( (- \ii \vb*{k} \epsilon_0^\sigma(\vb*{k}) + \ii \omega_{\vb*{k}} \vb*{\epsilon}^\sigma(\vb*{k})) a_{\vb*{k} \sigma} \ee^{\ii \vb*{k} \cdot \vb*{r} - \ii \omega_{\vb*{k}} t} + \text{h.c.} \right),
\end{equation}
以及
\begin{equation}
    \vb*{B}(\vb*{r}, t) = \int \frac{\dd[3]{\vb*{k}}}{(2\pi)^3} \sqrt{\frac{\hbar}{2\omega_{\vb*{k}} \epsilon_0}} \sum_{\sigma=1}^2 \left( \ii \vb*{k} \times \vb*{\epsilon}_\sigma a_{\vb*{k} \sigma} \ee^{\ii \vb*{k} \cdot \vb*{r} - \ii \omega_{\vb*{k}} t } + \text{h.c.} \right).
\end{equation}
其中
\begin{equation}
    \epsilon^\mu_\sigma = (\epsilon_\sigma^0, \vb*{\epsilon}_\sigma), \quad \frac{\omega_{\vb*{k}}}{c} \epsilon_0^\sigma - \vb*{k} \cdot \vb*{\epsilon}_\sigma = 0, \quad \abs*{\epsilon_\sigma^0}^2 - \abs*{\vb*{\epsilon}_\sigma}^2 = 1.
\end{equation}
通过以上公式,能够验证以下哈密顿量形式:
\begin{equation}
    H = \int \dd[3]{\vb*{r}} \left( \frac{\epsilon_0}{2} \vb*{E}^2 + \frac{1}{2\mu_0} \vb*{B}^2 \right).
    \label{eq:e-and-b-hamiltonian}
\end{equation}
这正是电动力学中常见的形式。因此实际上我们也可以直接用\eqref{eq:vector-field-components}写出$\vb*{E}$和$\vb*{B}$并代入\eqref{eq:e-and-b-hamiltonian}。

在本文讨论的光学问题中,我们可以使用一种对具体计算更加友好的形式,即采用\concept{辐射规范}。
在辐射规范之下,我们有
\begin{equation}
    \begin{aligned}
        \mathcal{L} &= - \frac{1}{4 \mu_0} (\partial_\mu A_\nu - \partial_\nu A_\mu) (\partial^\mu A^\nu - \partial^\nu A^\mu) \\
        &= \frac{1}{2 \mu_0} \frac{1}{c^2} (\partial_t \vb*{A})^2 - \frac{1}{4\mu_0} (\partial_i A_j - \partial_j A_i) (\partial^i A^j - \partial^j A^i) \\
        &= \frac{\epsilon_0}{2} ((\dot{\vb*{A}})^2 - c^2 (\curl{\vb*{A}})^2).
    \end{aligned}
\end{equation}
以这个拉氏量为出发点做正则量子化。做展开
\begin{equation}
    \vb*{A}(\vb*{r}, t) = \int \frac{\dd[3]{\vb*{k}}}{(2\pi)^3} \sqrt{\frac{\hbar}{2 \omega_{\vb*{k}} \epsilon_0}} \sum_{\sigma=1}^2 (a_{\vb*{k} \sigma} \vu*{e}^\sigma \ee^{\ii \vb*{k} \cdot \vb*{r} - \ii \omega_{\vb*{k}} t} + a^\dagger_{\vb*{k} \sigma} (\vu*{e}^\sigma)^* \ee^{- \ii \vb*{k} \cdot \vb*{r} + \ii \omega_{\vb*{k}} t}),
\end{equation}
从而电场和磁场分别为
\begin{equation}
    \vb*{E}(\vb*{r}, t) = \ii \int \frac{\dd[3]{\vb*{k}}}{(2\pi)^3} \sqrt{\frac{\hbar \omega_{\vb*{k}}}{2 \epsilon_0}} \sum_{\sigma=1}^2 (a_{\vb*{k} \sigma} \vu*{e}^\sigma \ee^{\ii \vb*{k} \cdot \vb*{r} - \ii \omega_{\vb*{k}} t} - a^\dagger_{\vb*{k} \sigma} (\vu*{e}^\sigma)^* \ee^{- \ii \vb*{k} \cdot \vb*{r} + \ii \omega_{\vb*{k}} t})
    \label{eq:vacuum-e-field}
\end{equation}
和
\begin{equation}
    \vb*{B}(\vb*{r}, t) = \ii \int \frac{\dd[3]{\vb*{k}}}{(2\pi)^3} \sqrt{\frac{\hbar}{2 \omega_{\vb*{k}} \epsilon_0}} \sum_{\sigma=1}^2 (a_{\vb*{k} \sigma} \vb*{k} \times \vu*{e}_\sigma \ee^{\ii \vb*{k} \cdot \vb*{r} - \ii \omega_{\vb*{k}} t} - a^\dagger_{\vb*{k} \sigma} \vb*{k} \times \vu*{e}_\sigma^* \ee^{- \ii \vb*{k} \cdot \vb*{r} + \ii \omega_{\vb*{k}} t}).
    \label{eq:vacuum-b-field}
\end{equation}
正则动量为
\begin{equation}
    \vb*{\pi} = \epsilon_0 \dot{\vb*{A}},
\end{equation}
施加正则对易关系,会得到正确的
\begin{equation}
    \comm*{a_{\vb*{k} \sigma}}{a_{\vb*{k}' \sigma'}} = (2\pi)^3 \delta(\vb*{k} - \vb*{k}') \delta_{\sigma \sigma'},
\end{equation}
而哈密顿量为
\begin{equation}
    \begin{aligned}
        H &= \int \dd[3]{\vb*{r}} \left(\vb*{\pi} \cdot \pdv{\vb*{A}}{t} - \mathcal{L} \right) = \int \dd[3]{\vb*{r}} \left( \frac{\epsilon_0}{2} \left(\pdv{\vb*{A}}{t}\right)^2 + \frac{\epsilon_0}{2} c^2 (\curl{\vb*{A}})^2 \right) \\
        &= \int \dd[3]{\vb*{r}} \left( \frac{\epsilon_0}{2} \vb*{E}^2 + \frac{1}{2\mu_0} \vb*{B}^2 \right) \\
        &= \sum_{\sigma=1}^2 \int \frac{\dd[3]{\vb*{k}}}{(2\pi)^3} \hbar \omega_{\vb*{k}} \left(a^\dagger_{\vb*{k} \sigma} a_{\vb*{k} \sigma} + \frac{1}{2} \right).
    \end{aligned}
\end{equation}
因此,辐射规范给出的结果和完整的QED计算是完全一致的。
在辐射规范中我们还可以证明一个在横场条件成立时也成立,并且在一般的QED中很难计算的公式:
\begin{equation}
    \comm*{E^i(\vb*{r}, t)}{B^j(\vb*{r}', t)} = - \frac{\ii \hbar}{\epsilon_0} \pdv{x^k} \delta(\vb*{r} - \vb*{r}')
\end{equation}
其中$i, j, k$是$x, y, z$的轮换排列;其它情况下对易子为零。
还能够发现电场和自己的对易子始终为零,磁场亦然。因此电场的三个分量可以同时确定地被测量,磁场亦然。
但是不能同时准确测出$\vb*{E}$和$\vb*{B}$。
由于$(\vb*{E}, \vb*{B})$,$\vb*{A}$和$a_{\vb*{k} \sigma}$之间的关系是线性的,$a$的产生湮灭算符对易关系、$\vb*{A}$和$\vb*{\pi}$的正则对易关系以及$\vb*{E}$和$\vb*{B}$的对易关系是彼此等价的。

\begin{back}{粒子数表象和升降算符}{ladder-operator-particle-number}
    如果
    \begin{equation}
        n = a^\dagger a,
    \end{equation}
    且
    \begin{equation}
        \comm*{a}{a^\dagger} = 1,
    \end{equation}
    那么
    \begin{equation}
        a = \sum_{n} \sqrt{n} \ket*{n-1} \bra{n}, \quad a^\dagger = \sum_{n} \sqrt{n} \ket*{n} \bra*{n-1},
    \end{equation}
    或者说
\end{back}

\subsection{长波光子和介质中的麦克斯韦方程}\label{sec:long-wavelength-photon-maxwell-general}

下面我们讨论和\eqref{eq:material-hamiltonian}匹配的对易关系,以及它对角化之后将给出什么样的能谱。
应当指出,此时真空中的那些对易关系——$\vb*{A}$和$\epsilon_0 \dot{\vb*{A}}$之间的对易关系,$\vb*{E}$和$\vb*{B}$之间的对易关系——可能不能够直接适用。
这是因为正则量子化中,积掉自由度会导致哈密顿量的本征态的意义发生变化,从而算符的意义会发生变化。在高能物理中这导致场强重整化,在本文讨论的量子光学中则还会让对易关系发生变化。
同样这也会让横波条件的形式发生变化——介质中横波条件是$\div{\vb*{\epsilon} \cdot \vb*{E}} = 0$。

我们需要直接从\eqref{eq:material-hamiltonian}计算正则动量。同样取辐射规范,以$\vb*{A}$为基本自由度,则\eqref{eq:material-hamiltonian}就是
\begin{equation}
    H = \int \dd[3]{\vb*{r}} \left( \frac{1}{2} \dot{\vb*{A}} \cdot \vb*{\epsilon} \cdot \dot{\vb*{A}} + \frac{1}{2} (\curl{\vb*{A}}) \cdot (\vb*{\mu}^{-1}) \cdot (\curl{\vb*{A}}) \right),
\end{equation}
于是正则动量为
\begin{equation}
    \vb*{\pi} = \vb*{\epsilon} \cdot \dot{\vb*{\vb*{A}}}.
\end{equation}
我们现在需要展开$\vb*{A}$。此时空间平移不变性不能保持,我们不能使用动量来标记电场的振动模式,
我们将\eqref{eq:photon-in-material}右边的$\vb*{j}$取为零——我们此处在对线性介质做正则量子化,暂时不考虑电流——那就得到了一个广义本征值问题。
这就意味着,我们可以求解出一整套本征函数,它们由下式
\begin{equation}
    \curl{(\mu^{-1} \cdot \curl{\vb*{u}_n})} - \omega_n^2 \epsilon \cdot \vb*{u}_n = 0
\end{equation}
确定,其中$\omega_n$对应着能够在系统中稳定传播的电磁波模式的频率,且有正交归一关系
\begin{equation}
    \int_V \dd[3]{\vb*{r}} \vb*{u}^*_m \cdot \vb*{\epsilon} \cdot \vb*{u}_n = \delta_{mn}, 
\end{equation}
请注意由于$\vb*{A}$的厄米性,$\vb*{u}_m^*$一般是另一个$\vb*{u}_n$。
正交归一关系又意味着
\begin{equation}
    \omega_n^2 \delta_{mn} = \int \dd[3]{\vb*{r}} (\curl{\vb*{u}_m^*}) \cdot (\vb*{\mu}^{-1}) \cdot (\curl{\vb*{u}_n}) + \int \dd{\vb*{S}} \cdot (\vb*{u}_m^* \times ((\vb*{\mu}^{-1}) \cdot (\curl{\vb*{u}_n}))),
\end{equation}
在自由空间中等式右边第二项可以略去,在一个反射性能尚可的反射腔体(如果我们只讨论有限空间中的问题,那么基本上这个问题需要放在一个腔体中,否则无法忽视外界影响)中可以把第二项当成微扰。
本节仅仅给出最为简单的理论,暂时不考虑第二项。
用这组基底$\{\vb*{u}_n\}$做展开
\begin{equation}
    \vb*{A}(\vb*{r}, t) = \sum_n \ii \sqrt{\frac{\hbar}{2\omega_{n}}} \vb*{u}_n(\vb*{r}) a_n \ee^{- \ii \omega_n t} + \text{h.c.},
\end{equation}
得到
\begin{equation}
    - \vb*{E} = \dot{\vb*{A}} = \sum_n \sqrt{\frac{\hbar \omega_n}{2}} \vb*{u}_n(\vb*{r}) a_n \ee^{-\ii \omega_n t} + \text{h.c.},
\end{equation}
以及
\begin{equation}
    \vb*{B} = \curl{\vb*{A}} = \sum_n \ii \sqrt{\frac{\hbar}{2\omega_n}} \curl{\vb*{u}_n(\vb*{r})} a_n \ee^{-\ii \omega_n t} + \text{h.c.}.
\end{equation}
施加正则对易关系
\begin{equation}
    \comm*{A^i(\vb*{r}, t)}{\pi^j(\vb*{r}', t)} = \ii \hbar \delta(\vb*{r} - \vb*{r}') \delta^{ij},
\end{equation}
我们发现我们能够得到我们想要的产生湮灭算符对易关系
\begin{equation}
    \comm*{a_n}{a_m^\dagger} = \delta_{mn}.
\end{equation}
然后,可以计算出哈密顿量为
\begin{equation}
    H = \sum_n \hbar \omega_n \left( a^\dagger_n a_n + \frac{1}{2} \right).
\end{equation}
这个哈密顿量的形式和真空中完全一样,不同的地方在于$\omega_{\vb*{k}}$被$\omega_n$取代,色散关系可能变得非常不一样。

既然$\epsilon$和$\mu$的概念对长波光子在量子情况下仍然适用,反射、折射等概念对长波光子仍然有意义,且和经典情况非常类似。
特别的,光场可能被约束在一个四面都是反射镜的腔体中,此时的光场被所谓的\concept{cavity QED}或者简写为\concept{cQED}描述。

一个介质系统中的量子化光场的自由哈密顿量就是普通的谐振子哈密顿量。
使用本质上是经典的方程\eqref{eq:photon-in-material},得到一系列振动模式,其频率即为这个介质系统中的量子化光场中的模式的频率,振动模式的场强分布就是\eqref{eq:photon-in-material}给出的本征模式。

\subsection{归一化和单光子电场}

现在设想我们在一个有限大小的空间中做光场量子化,不过该空间中还是能够良定义波矢。
这样我们只需要在\autoref{sec:quantization-in-vacuum}中做代换
\[
    \int \frac{\dd[3]{\vb*{k}}}{(2\pi)^3} \longrightarrow \frac{1}{V} \sum_{\vb*{k}},
\]
于是电场和磁场分别为
\begin{equation}
    \vb*{E}(\vb*{r}, t) = \frac{\ii}{V} \sum_{\vb*{k}} \sqrt{\frac{\hbar \omega_{\vb*{k}}}{2 \epsilon_0}} \sum_{\sigma=1}^2 (a_{\vb*{k} \sigma} \vu*{e}^\sigma \ee^{\ii \vb*{k} \cdot \vb*{r} - \ii \omega_{\vb*{k}} t} - a^\dagger_{\vb*{k} \sigma} (\vu*{e}^\sigma)^* \ee^{- \ii \vb*{k} \cdot \vb*{r} + \ii \omega_{\vb*{k}} t})
    \label{eq:vacuum-e-field-cavity-origin}
\end{equation}
和
\begin{equation}
    \vb*{B}(\vb*{r}, t) = \frac{\ii}{V} \sum_{\vb*{k}} \sqrt{\frac{\hbar}{2 \omega_{\vb*{k}} \epsilon_0}} \sum_{\sigma=1}^2 (a_{\vb*{k} \sigma} \vb*{k} \times \vu*{e}_\sigma \ee^{\ii \vb*{k} \cdot \vb*{r} - \ii \omega_{\vb*{k}} t} - a^\dagger_{\vb*{k} \sigma} \vb*{k} \times \vu*{e}_\sigma^* \ee^{- \ii \vb*{k} \cdot \vb*{r} + \ii \omega_{\vb*{k}} t}).
    \label{eq:vacuum-b-field-cavity-origin}
\end{equation}
不过我们注意到,此时哈密顿量为
\[
    H = \frac{1}{V} \sum_{\vb*{k}, \sigma} \hbar \omega_{\vb*{k}} \left( a^\dagger_{\vb*{k} \sigma} a_{\vb*{k} \sigma} + \frac{1}{2} \right),
\]
多出来一个不太美观的因子$V$;此外对易关系也是
\[
    \comm*{a_{\vb*{k} \sigma}}{a_{\vb*{k}' \sigma'}^\dagger} = (2\pi)^3 \delta(\vb*{k} - \vb*{k}') \delta_{\sigma \sigma'} \longrightarrow \frac{1}{V} \delta_{\vb*{k} \vb*{k}'} \delta_{\sigma \sigma'}.
\]
为此我们如下重新定义产生湮灭算符:
\begin{equation}
    a_{\vb*{k} \sigma} \longrightarrow \sqrt{V} a_{\vb*{k} \sigma},
\end{equation}
于是电场和磁场分别为
\begin{equation}
    \vb*{E}(\vb*{r}, t) = \ii \sum_{\vb*{k}} \sqrt{\frac{\hbar \omega_{\vb*{k}}}{2 \epsilon_0 V}} \sum_{\sigma=1}^2 (a_{\vb*{k} \sigma} \vu*{e}^\sigma \ee^{\ii \vb*{k} \cdot \vb*{r} - \ii \omega_{\vb*{k}} t} - a^\dagger_{\vb*{k} \sigma} (\vu*{e}^\sigma)^* \ee^{- \ii \vb*{k} \cdot \vb*{r} + \ii \omega_{\vb*{k}} t})
    \label{eq:vacuum-e-field-cavity-1}
\end{equation}
和
\begin{equation}
    \vb*{B}(\vb*{r}, t) = \ii \sum_{\vb*{k}} \sqrt{\frac{\hbar}{2 \omega_{\vb*{k}} \epsilon_0 V}} \sum_{\sigma=1}^2 (a_{\vb*{k} \sigma} \vb*{k} \times \vu*{e}_\sigma \ee^{\ii \vb*{k} \cdot \vb*{r} - \ii \omega_{\vb*{k}} t} - a^\dagger_{\vb*{k} \sigma} \vb*{k} \times \vu*{e}_\sigma^* \ee^{- \ii \vb*{k} \cdot \vb*{r} + \ii \omega_{\vb*{k}} t}),
    \label{eq:vacuum-b-field-cavity-1}
\end{equation}
哈密顿量为
\begin{equation}
    H = \sum_{\vb*{k}, \sigma} \hbar \omega_{\vb*{k}} \left( a^\dagger_{\vb*{k} \sigma} a_{\vb*{k} \sigma} + \frac{1}{2} \right),
\end{equation}
对易关系为
\begin{equation}
    \comm*{a_{\vb*{k} \sigma}}{a_{\vb*{k}' \sigma'}^\dagger} = \delta_{\vb*{k} \vb*{k}'} \delta_{\sigma \sigma'}.
\end{equation}
我们定义
\begin{equation}
    \mathcal{E}_{\vb*{k} \sigma} = \sqrt{\frac{\hbar \omega_{\vb*{k}}}{2 \epsilon_0 V}},
\end{equation}
则
\begin{equation}
    \vb*{E} = \sum_{\vb*{k}, \sigma} \mathcal{E}_{\vb*{k} \sigma} \vb*{f}_{\vb*{k} \sigma} a_{\vb*{k} \sigma} \ee^{- \ii \omega_{\vb*{k}} t} + \text{h.c.},
\end{equation}
其中
\begin{equation}
    \vb*{f}_{\vb*{k} \sigma} = \ii \ee^{\ii \vb*{k} \cdot \vb*{r}} \vu*{e}^\sigma,
\end{equation}
满足
\begin{equation}
    \frac{1}{V} \int \dd[3]{\vb*{r}} \vb*{f}^*_{\vb*{k} \sigma} \cdot \vb*{f}_{\vb*{k}' \sigma'} = \delta_{\vb*{k} \vb*{k}'} \delta_{\sigma \sigma'}.
\end{equation}
我们称$\mathcal{E}_{\vb*{k} \sigma}$为\concept{单光子电场},因为它给出了增多一个电子,电场大体上增多的幅度,而$\vb*{f}$则称为\concept{模式函数},它给出了光场的稳定振动方式,且不显含体积(体积出现在了积分前的归一化常数中)。

对一个一般的体系,我们会定义
\begin{equation}
    \mathcal{E}_{n} = \sqrt{\frac{\hbar \omega_n}{2 \epsilon_0 V}},
\end{equation}
从而电场为
\begin{equation}
    \vb*{E} = \sum_n \mathcal{E}_n \vb*{f}_{n} a_n \ee^{- \ii \omega_n t} + \text{h.c.},
    \label{eq:general-optical-field}
\end{equation}
其中
\begin{equation}
    \vb*{f}_n = \sqrt{\epsilon_0 V} \vb*{u}_n,
\end{equation}
且
\begin{equation}
    \frac{1}{V} \int \dd[3]{\vb*{r}} \vb*{f}_m^* \cdot \frac{\vb*{\epsilon}}{\epsilon_0} \cdot \vb*{f}_n = \delta_{mn}.
\end{equation}

实际上从这里我们可以看出,经典的麦克斯韦方程本身已经是一个具有一定量子特性的理论了——“单光子波函数”(虽然没有良定义的单光子量子力学,但是我们不妨这么指代$\mel*{0}{A^i(\vb*{r})}{\psi}$)服从的方程就是麦克斯韦方程。
也可以从另一个角度看这件事:在麦克斯韦方程两边乘上$\hbar$,由于$E \sim \hbar \omega$,得到的理论看上去就是一个量子理论。
将光场量子化引入的新物理只有两件事:光束由分立的光子构成,以及存在光子数的量子涨落,但是,在没有非线性光学效应的情况下,光子数目守恒,第一件事完全可以通过手动引入“光子”的概念并指派其波函数为(经过适当归一化的)经典电磁场来做到。
光的量子性只有在下面的地方才会变得重要:
\begin{itemize}
    \item 光子生灭明显,一些光子模式上原本没有电子而一段时间后有光子产生时,即处理非线性光学时,因为此时会有一些原本完全没有光子分布的模式上出现了光子。经典处理只能在有种子光的时候处理光子的产生——并不奇怪,因为光子从零到一产生的过程涉及一个极为弱的场强,弱到经典场论不再使用。
    \item 纠缠重要时。经典电动力学面对“光子增多了”的描述方法是更大的场强,而没有直积的希尔伯特空间这样的概念,从而无法捕捉到纠缠。
\end{itemize}
可以看到这些光的量子性变得明显的情况都涉及多光子Fock态。
我们其实可以在这里看到一个相当有趣的情况:当所研究的问题中涉及非常弱的光场(如特定频率的光子一开始没有,但是一段时间后被产生)时,经典电动力学就失效了,然而经典电动力学的形式却又很像是在处理“单光子波函数”。
当然,这两者并没有矛盾,本质上是因为经典电动力学无法正确处理“多光子形成的多体波函数”:“单光子波函数”不涉及多体波函数,它给出的所有物理就是一个麦克斯韦方程,正好和经典电动力学一致;
相干态下电场的标准差相比于电场期望值本身比较小,从而电场能够被看成经典的量,同样可以被经典电动力学处理。
经典电动力学无法正确处理“多光子形成的多体波函数”,因为没有Fock空间;但是这并不是说经典电动力学就缺乏(相比于经典质点动力学的)一切量子性,如坐标和动量的不确定性等。
实际上,对缺乏纠缠、缺乏粒子生灭和碰撞、粒子数大的有质量粒子系统,单粒子波函数乘以适当的因子也可以诠释为“粒子数的平方根”。在这个意义上它和电磁场的地位是类似的。
当然,有质量粒子系统中有大量的碰撞,其宏观理论通常是动理学方程,且“经典费米场”不是一个物理意义特别明确的东西,因此我们很少看到“电子的宏观场”。
然而在电动力学以外的系统中,由于系统非常接近相干态,量子场能够被看成经典场的例子也是有的,如超流和超导中的序参量。

从本节的计算中也可以看到为什么很多时候经典电动力学已经够用了,因为一般的偶极辐射产生的就是相干光。要产生明显偏离相干光的辐射场实际上是很不容易的。

在本文中“电磁场”可能代表量子化的场算符,也可能代表经典电磁场,也可能代表“单光子波函数”。在不考虑非线性效应时这三者的时间演化是相同的。
在$\vb*{E}$被认为是经典场时,$\vb*{E}^2$——从而$I$——在形式上对应于“光子出现的概率”。
光场中被传输的不是$I$,电磁场的相位信息是很重要的,正如量子力学中叠加的不是概率而是概率振幅一样。
对非相干光(后文将讨论),可以直接将$I$相加,正如高度混合态的系统可以直接使用经典概率论处理一样。

\section{相干态和Wigner波函数}

如果我们直接计算电场在光子的Fock态下的期望值,无疑会得到零,因为$\vb*{E}$算符正比于单个产生算符和湮灭算符之和,从而,设$\ket*{\psi}$是一个光子Fock态,$\mel{\psi}{E_n}{\psi}$中,$E_n$作用到右边后模式$n$上的光子数目发生变化,因此电场期望值为零。

这看起来真是匪夷所思,因为这似乎说明Fock态中没有光子,没有能量,而这当然不正确。
例如,计算电场平方的期望值却又会得到非零结果。电场平方正比于该点的“电场能”,因此Fock态是有能量的。

问题的核心在于Fock态\emph{不是}电场的本征态。我们现在要找到一个和经典的电场接近的量子态,并且考虑如何用一种经典意义明显的方式表示一个任意的量子光学中的光场波函数。

\subsection{相干态}

\begin{back}{相干态和相干态路径积分}{coherent-state}
    设动力学变量$x$和它的正则动量满足正则对易关系
    \begin{equation}
        \comm{x}{p} = \ii,
    \end{equation}
    引入产生湮灭算符
    \begin{equation}
        a = \frac{1}{\sqrt{2}} (x + \ii p), \quad a^\dagger = \frac{1}{\sqrt{2}} (x - \ii p),
    \end{equation}
    则有正确的对易关系
    \begin{equation}
        \comm{a}{a^\dagger} = 1.
    \end{equation}
    我们现在考虑怎么将一个密度矩阵写成某种“准概率分布”的形式,即能够将它和一个函数$W(x, p)$或者$W(a, a^*)$建立线性关系。

    考虑用复参数定义的相干态
    \begin{equation}
        \ket*{\alpha} = \ee^{- \abs*{\alpha}^2 / 2} \sum_{n = 0}^\infty \frac{\alpha^n}{\sqrt{n!}} \ket*{0},
    \end{equation}
    其中$\ket*{n}$是谐振子哈密顿量
    \begin{equation}
        H = \sum_{n \geq 0} \hbar \omega \left( a^\dagger a + \frac{1}{2} \right) 
    \end{equation}
    的第$n$激发态。
    能够证明完备条件
    \begin{equation}
        \frac{1}{\pi} \int \dd[2]{\alpha} \dyad{\alpha} = 1
    \end{equation}
    成立,这里积分测度为
    \begin{equation}
        \dd[2]{\alpha} = \frac{\dd{\alpha^*} \wedge \dd{\alpha}}{2 \ii} = R \dd{R} \wedge \dd{\theta},
    \end{equation}
    其中$R$是$\alpha$的长度而$\theta$是相角,$\dd[2]{\alpha}$是复平面上的积分测度。
    由于不同$\alpha$的相干态彼此不正交——实际上,我们有
    \begin{equation}
        \braket*{\alpha}{\beta} = \ee^{- (\abs{\alpha}^2 + \abs{\beta}^2) / 2 + \alpha^* \beta},
    \end{equation}
    全体相干态实际上是超完备的。

\end{back}

我们在构造路径积分时遇到过形式最为一般的相干态。在量子光学中相干态还有特殊的意义。

设想空间中有一个振动频率给定的电偶极子
\begin{equation}
    \vb*{d}(t) = \vb*{d} \ee^{- \ii \omega t } + \text{h.c.},
\end{equation}
我们考虑空间中的光场的演化情况。设光场波函数为$\ket*{\psi}$,取偶极辐射近似。
在相互作用绘景中有
\begin{equation}
    \ii \pdv{t} \ket*{\psi} = - \vb*{d}(t) \cdot \vb*{E} \ket*{\psi}.
\end{equation}
假定$t=0$时没有任何光子,因此,等价地我们可以认为从$t=0$开始电偶极子才开始振动,在此之前系统中一直没有任何光子。
我们有形式解
\[
    \begin{aligned}
        \ket*{\psi(t)} &= \exp(\frac{\ii}{\hbar} \int_0^t \dd{t'} \sum_n (\vb*{d} \ee^{- \ii \omega t } + \text{h.c.}) \cdot (\mathcal{E}_n \vb*{f}_n a_n \ee^{- \ii \omega_n t'} + \text{h.c.}) ) \ket*{0} \\
        &= \exp(\sum_n (\alpha_n a_n^\dagger - \alpha_n^* a_n) ) \ket*{0},
    \end{aligned}
\]
这里我们定义
\begin{equation}
    \alpha_n \coloneqq \frac{1}{\hbar} \int_0^t \dd{t'} \mathcal{E}_n \vb*{f}_n^* \ee^{-\ii \omega_n t'} \cdot (\vb*{d} \ee^{- \ii \omega t} + \text{h.c.}).
\end{equation}
由于
\[
    \comm*{a_n}{\comm*{a_n}{a^\dagger_n}} = \comm*{a^\dagger_n}{\comm*{a_n}{a^\dagger_n}} = 0,
\]
我们有
\[
    \begin{aligned}
        \exp(\alpha_n a_n^\dagger - \alpha_n^* a_n) &= \ee^{\alpha_n a_n^\dagger} \ee^{- \alpha_n^* a_n} \ee^{\frac{1}{2} \comm*{\alpha_n a_n^\dagger}{\alpha_n^* a_n}} \\
        &= \ee^{\alpha_n a_n^\dagger} \ee^{- \alpha_n^* a_n} \ee^{\frac{1}{2} \abs{\alpha_n}},
    \end{aligned}
\]
定义其为\concept{位移算符}
\begin{equation}
    D(\{\alpha_n\}) \coloneqq \prod_n \ee^{\alpha_n a_n^\dagger} \ee^{- \alpha_n^* a_n} \ee^{\frac{1}{2} \abs{\alpha_n}},
\end{equation}
它作用在真空态上给出
\begin{equation}
    \ket*{\{\alpha_n\}} \coloneqq D(\{\alpha_n\}) \ket*{0} = \exp(\sum_n (\alpha_n a_n^\dagger - \alpha_n^* a_n) ) \ket*{0} = \prod_n \ee^{- \frac{\abs{\alpha_n}^2}{2}} \ee^{\alpha_n a^\dagger_n} \ket*{0_n}.
\end{equation}
将上式中的$\ee^{\alpha_n a^\dagger_n}$展开,我们发现
\begin{equation}
    \ket*{\{\alpha_n\}} = \prod_n \ee^{- \frac{\abs{\alpha_n}^2}{2}} \sum_{i_n=0}^\infty \frac{\alpha_n^{i_n}}{\sqrt{i_n!}} \ket*{0_n},
\end{equation}
因此它的确就是路径积分量子化中定义的相干态。

现在我们回来看看$\{\alpha_n\}$到底是什么,或者说相干态到底是什么。
实际上在路径积分量子化中我们已经知道系统处在相干态上意味着系统大体上服从经典场论了,不过还是做一些具体计算来更清楚地展示这一点。

为了更加简便我们考虑一个\concept{单模光场},即只考虑\eqref{eq:general-optical-field}中的一个模式。
这是合理的,因为一个线性体系中的所有模式之间不存在任何关系。
对某个特定的模式对应的产生湮灭算符$(a, a^\dagger)$定义
\begin{equation}
    X_1 = \frac{1}{2} (a + a^\dagger), \quad X_2 = \frac{1}{2\ii} (a - a^\dagger),
\end{equation}
则
\begin{equation}
    \vb*{E} = 2 \mathcal{E} (\vb*{f} X_1 \cos(\omega t) + \vb*{f}^* X_2 \sin(\omega t)).
\end{equation}
经典电动力学中系统总是在相干态附近,将$a$替换成$\alpha$就能够从量子光学过渡到经典光学。
在做了这个过渡之后,$X_1$大体上是$\Re{\alpha}$而$X_2$大体上就是$\Im{\alpha}$。
于是我们就看到了相干态$\ket*{\alpha}$的经典意义:一个相干态$\ket*{\alpha}$对应一个正弦振动的经典电场,其幅度为
\begin{equation}
    A = 2 \mathcal{E} \abs{\vb*{f}} \sqrt{X_1^2 + X_2^2} = 2 \mathcal{E} \abs{\vb*{f}} \abs{\alpha},
\end{equation}
其相位则是
\begin{equation}
    \varphi = \arg \alpha = \arctan \frac{X_2}{X_1}.
\end{equation}

$X_1$和$X_2$当然是不对易的,因此即使是相干态下(当然也包括其它相干态下),实际上电场
\begin{equation}
    E = A \ee^{\ii \varphi} \ee^{- \omega t} + \text{h.c.}
\end{equation}
是不能够完全确定的。或者也可以说,$A$(实际上就是粒子数开平方)和$\varphi$是不能同时完全确定的。

相干态下的光子数服从泊松分布。
因此,相干态的平均值和经典电场类似,但是存在涨落,其涨落行为和真空类似。

\subsection{准概率分布函数}

\begin{back}{量子力学中的相空间和准概率分布函数}{phase-wigner}
    虽然在量子力学中坐标和动量不能同时确定,从而看起来似乎不能够有良定义的相空间,不过注意到,算符$O$的矩阵元一般来说形如$\mel{x}{O}{x'}$,即需要两个标签——$x$和$x'$——标记一个矩阵元,那么我们将$x$和$x'$线性组合一下,将其中一个坐标做傅里叶变换切换到动量空间,似乎还是能够写出$O(x, p)$这样的式子,从而有一个等效的相空间。
    当然,这样得到的“相空间中的分布函数”,即满足下述条件的函数$f(x, p)$:
    \[
        \expval*{O} = \int \dd{x} \dd{p} O(x, p) f(x, p)
    \]
    未必能够赋予概率的意义,因为它可以取负值甚至虚数值。

    我们现在考虑几种$f(x, p)$的定义。最为知名的可能是Wigner函数,它基本上是经典力学中的粒子分布函数的量子推广。
    设$\rho$是一个密度矩阵,\concept{Wigner函数}定义为
    \begin{equation}
        W(x, p) = \frac{1}{2\pi \hbar} \int \dd{y} \mel{x - \frac{y}{2}}{\rho}{x + \frac{y}{2}} \ee^{\ii p y / \hbar}.
    \end{equation}
    当然也可以用$\alpha$和$\alpha^*$——或者说,用二维相平面上的复数$\alpha$——做Wigner函数的宗量。
    将算符$O$写成关于$a$和$a^\dagger$的\emph{对称排序}$O_\text{S}$之后,我们有
    \begin{equation}
        \expval*{O} = \int \dd[2]{\alpha} W(\alpha, \alpha^*) O_\text{S}(\alpha, \alpha^*).
    \end{equation}
    例如,
    \begin{equation}
        \frac{1}{2} \expval*{a a^\dagger + a^\dagger a} = \int \dd[2]{\alpha} W(\alpha, \alpha^*) \alpha \alpha^*.
    \end{equation}
    
    Wigner分布函数可以验证是实数,不过有正有负。能够证明Wigner函数取负值的区域不会太大——在几个$\hbar$以内——这直观地展示了从量子过渡到经典的过程。
    与路径积分类似,Wigner函数也并非只能在坐标-动量构成的相空间中定义——任何有广义坐标和广义动量的能够用哈密顿动力学描述的系统中的密度矩阵都能够用Wigner函数等价地给出。
    因此,Wigner函数实际上可以构成量子力学的另一种形式理论:所谓量子力学,就是将普通的概率分布拓展为Wigner函数的物理理论。

    我们当然也可以寻找一种函数$P(\alpha, \alpha^*)$使得
    \begin{equation}
        \expval*{O} = \int \dd[2]{\alpha} P(\alpha, \alpha^*) O_\text{N}(\alpha, \alpha^*),
    \end{equation}
    这里$O_\text{N}$表示算符$O$的\emph{正则排序},即将湮灭算符排在右边,产生算符排在左边。
    我们有
    \[
        \begin{aligned}
            \expval*{O} &= \sum_{m, n} c_{mn} \trace((a^\dagger)^m a^n \rho) \\
            &= \sum_{m, n} c_{mn} \int \dd[2]{\alpha} \trace((\alpha^*)^m \alpha^n \rho \delta(\alpha^* - a^\dagger) \delta(\alpha - a)) \\
            &= \int \dd[2]{\alpha} \sum_{m,n} c_{mn} (\alpha^*)^m \alpha^n \trace(\rho \delta(\alpha^* - a^\dagger) \delta(\alpha - a)),
        \end{aligned}
    \]
    上式最后一行的形式正是$O_\text{N}$乘以某个分布函数,于是我们得到
    \begin{equation}
        P(\alpha, \alpha^*) = \trace(\rho \delta(\alpha^* - a^\dagger) \delta(\alpha - a)).
    \end{equation}
    简单地验证会发现
    \begin{equation}
        \rho=\int \dd^{2} \alpha P\left(\alpha, \alpha^{*}\right)|\alpha\rangle\langle\alpha| .
    \end{equation}

    我们当然还可以让寻找一种函数$Q(\alpha, \alpha^*)$使得
    \begin{equation}
        \expval*{O} = \int \dd[2]{\alpha} Q(\alpha, \alpha^*) O_\text{A}(\alpha, \alpha^*),
    \end{equation}
    这里$O_\text{A}$表示算符$O$的\emph{反正则排序},即湮灭算符排在左边而产生算符排在右边。和$P$函数一样如法炮制,能够得到
    \begin{equation}
        Q\left(\alpha, \alpha^{*}\right)=\operatorname{tr}\left[\rho \delta(\alpha-a) \delta\left(\alpha^{*}-a^{\dagger}\right)\right].
    \end{equation}
    由于
    \[
        \begin{aligned}
            Q\left(\alpha, \alpha^{*}\right) &=\frac{1}{\pi} \operatorname{tr} \int d^{2} \alpha^{\prime}\left[\rho \delta(\alpha-a)\left|\alpha^{\prime}\right\rangle\left\langle\alpha^{\prime}\right| \delta\left(\alpha^{*}-a^{\dagger}\right)\right] \\
            &=\frac{1}{\pi} \operatorname{tr} \int d^{2} \alpha^{\prime}\left\{\rho \delta\left(\alpha-\alpha^{\prime}\right)\left|\alpha^{\prime}\right\rangle\left\langle\alpha^{\prime}\right| \delta\left[\alpha^{*}-\left(\alpha^{\prime}\right)^{*}\right]\right\} \\
            &=\frac{1}{\pi} \operatorname{tr}(\rho|\alpha\rangle\langle\alpha|) \\
            &=\frac{1}{\pi}\langle\alpha|\rho| \alpha\rangle,
            \end{aligned}
    \]
    我们有
    \begin{equation}
        Q(\alpha, \alpha^*) = \frac{1}{\pi} \mel{\alpha}{\rho}{\alpha},
    \end{equation}

    $P$函数,$W$函数,$Q$函数都能够写成某些算符的期望值的某种积分变换。
    我们有
    \begin{equation}
        \delta\left(\alpha^{*}-a^{\dagger}\right) \delta(\alpha-a) =\frac{1}{\pi^{2}} \int \dd^{2} \beta \exp \left[-\beta\left(\alpha^{*}-a^{\dagger}\right)\right] \exp \left[\beta^{*}(\alpha-a)\right],
    \end{equation}
    于是
    \begin{equation}
        P\left(\alpha, \alpha^{*}\right)=\frac{1}{\pi^{2}} \int \dd^{2} \beta \ee^{- \ii \beta \alpha^{*}- \ii \beta^{*} \alpha} C_\text{N}\left(\beta, \beta^{*}\right),
    \end{equation}
    其中
    \begin{equation}
        C_\text{N}\left(\beta, \beta^{*}\right)=\operatorname{tr}\left(\ee^{\ii \beta a^{\dagger}} \ee^{\ii \beta^{*} a} \rho\right).
    \end{equation}
    类似的
    \begin{equation}
        Q\left(\alpha, \alpha^{*}\right)=\frac{1}{\pi^{2}} \int \dd^{2} \beta \ee^{- \ii \beta \alpha^{*}- \ii \beta^{\cdot} \alpha} C_\text{A}\left(\beta, \beta^{*}\right),
    \end{equation}
    其中
    \begin{equation}
        C_\text{A}\left(\beta, \beta^{*}\right)=\operatorname{tr}\left(\ee^{\ii \beta^{*} a} \ee^{\ii \beta a^{\dagger}} \rho\right).
    \end{equation}
    对Wigner函数也有类似的公式:
    \begin{equation}
        W\left(\alpha, \alpha^{*}\right)=\frac{1}{\pi^{2}} \int \dd^{2} \beta \ee^{-\ii \beta \alpha^{*}- \ii \beta^{*} \alpha} C_\text{S}\left(\beta, \beta^{*}\right),
    \end{equation}
    其中
    \begin{equation}
        C_\text{S}\left(\beta, \beta^{*}\right)=\operatorname{tr}\left(\ee^{\ii \beta a^{\dagger}+ \ii \beta^{*} a} \rho\right).
    \end{equation}

    作为一个例子,相干态的三种准
    $Q$函数相比Wigner函数要宽一些;$P$函数是最窄的。
\end{back}

\section{光的探测}

实用的光探测器基本上都是根据偶极相互作用哈密顿量$- \vb*{d} \cdot \vb*{E}$工作的,即一个凝聚态系统——这里是光传感器——吸收了一个光子,将电子激发出来,然后探测由此产生的光电流。
由于光传感器中电子的能量通常很低,不会激发出多少光子,实验中置于光场中的探测器探测到光信号的概率几乎完全由“光场中一个光子湮灭,光场掉落到一个能量更低的状态”这样的过程贡献。
因此,能够对实验中探测器探测到光信号的概率有贡献的关联函数基本上都是正规序的,即它们均形如
\[
    \mel{\psi}{a^\dagger_{\vb*{q}_1} a^\dagger_{\vb*{q}_2} \cdots a_{\vb*{p}_2 a_{\vb*{p}_2}}}{\psi}.
\]
从而,% TODO
其中
\begin{equation}
    \vb*{E}^{+} (\vb*{r}, t) = \sum_n \mathcal{E}_n \vb*{f}_n(\vb*{r}) a_n \ee^{-\ii \omega_n t} , \quad \vb*{E}^{-} (\vb*{r}, t) = \sum_n \mathcal{E}_n \vb*{f}^*_n(\vb*{r}) a^\dagger_n \ee^{\ii \omega_n t}, \quad \vb*{E} = \vb*{E}^+ + \vb*{E}^-.
\end{equation}

\chapter{单光子现象}

\section{单光子器件}

分束器

\section{证明单光子量子性的实验}

\subsection{Aspect实验}

要证明单光子具有量子性,最好的办法是使用一些这样的实验:它的一种版本能够证明光子的粒子性,它的另一个仅仅做了少许修正的版本(比如说探测器被移动到别的位置)能够证明光子的波动性。
两个版本区别很小这件事能够排除实验装置和光的复杂相互作用显著地改变了光的行为这样的说法,而粒子性和波动性同时出现则强烈暗示需要量子理论描述光。
1986年的Aspect实验是这种实验的一个典范。

\subsection{Zeilinger实验}

A.Zeilinger
一种更加简明的实验是这样的:同样使用分束器和反射镜,构造这样的光路:

\subsection{如果狄拉克是错的……}

\subsubsection{两个激光器产生的光束的干涉}

两个激光器产生的光似乎是相干的?

实际上激光器产生的是相干态光而不是光子数确定的多光子玻色波函数。

\subsubsection{玻色-爱因斯坦凝聚态中的干涉}

\subsection{Hanbury Brown和Twiss效应}

\subsubsection{天狼星上的光的疑似量子性}

\subsubsection{经典电动力学解释}

\chapter{纠缠光}

\bibliographystyle{plain}
\bibliography{optics,../formalism/classical-feyn,quantum-state} 

\end{document}