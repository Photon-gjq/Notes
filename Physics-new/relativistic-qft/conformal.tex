\chapter{二维共形场论的基本结构}

提出共形场论的一个动机是相变现象。凝聚态系统在相变点附近关联长度发散,从而可以预期,能够使用一个无能隙的理论描述它,并且很大一类系统的这种理论将会具有非常好的性质。
如果这个理论正好就是一个自由理论,那么对任何的关联函数的计算都是显然的:我们总是可以找到一组场,记之为$\phi(x)$,任何一个局域的算符$A(x)$均可以展开为$\phi^n$的正规序的线性组合:
\begin{equation}
    A(x) = \sum_{n \geq 0} a_n \normord{\phi(x)^n},
\end{equation}
其中
\[
    \normord{\phi(x)^n} = \phi(x)^n - \expval*{\phi(x)^n},
\]
在这么定义会产生疑难的时候,只需要让各个$x$略微差一些,然后计算完成后让它们相等即可。
由于自由理论中场的量纲就是工程量纲,无需做任何特殊的考虑,就能够得到
\begin{equation}
    \expval*{\phi(x) \phi(y)} \sim \frac{1}{r^{2 d_\phi}}, \quad r = \abs*{x - y}.
\end{equation}
于是我们可以据此估计出任何一个关联函数的衰减趋势,并实际上真的计算出它。

一个非常自然的问题是,能否将这种做法——仅仅通过一个理论的\emph{尺度不变性}就确定重要的关联函数的形式而无需使用常规的费曼图等技术,甚至通过形式由对称性确定的若干关联函数\emph{定义}一个理论,完全绕开拉氏量(这称为\concept{bootstrap})——推广到一些不那么平凡、带有相互作用的量子场论中。
为了尽可能减少理论中其它的结构,从而让bootstrap可以通过非常清晰的方式实现(例如只通过对称性信息是无法确定下来QED的),我们下面将要研究一大类具有非常完美的对称性的场论——\concept{共形场论}。

\section{共形对称性和共形群}

\subsection{任意维度的共形群}

简单地说,\concept{共形群}为空间尺度变换和洛伦兹变换共同组成的群(允许局域的操作)。具有共形不变性的场论就是\concept{共形场论}。
更加清楚的定义是这样的:如果一个坐标变换$x \longrightarrow x'$会导致
\begin{equation}
    g'(x') = \Lambda(x) g(x),
    \label{eq:conformal-def}
\end{equation}
那么这就是一个共形变换;全体共形变换组成共形群。
洛伦兹群显然是共形群的一个子群;局域的坐标缩放——让原本均匀的坐标网格变得一些区域大一些区域小——也是共形变换。
显然,共形场论是相对论性量子场论的一个特例。

我们现在分析共形群里面有什么操作。无穷小坐标变换
\[
    x^\mu \longrightarrow x^\mu + \epsilon^\mu
\]
下度规变化为
\begin{equation}
    g_{\mu \nu} \longrightarrow g_{\mu \nu} - (\partial_\mu \epsilon_\nu + \partial_\nu \epsilon_\mu),
\end{equation}
共形变换的定义\eqref{eq:conformal-def}等价于能够找到函数$f(x)$使得
\begin{equation}
    \partial_\mu \epsilon_\nu + \partial_\nu \epsilon_\mu = g_{\mu \nu} f(x).
    \label{eq:small-conformal-f-def}
\end{equation}
实际上我们还能够将$f(x)$关于$\epsilon$的形式确定下来——只需要在上式两边取迹即可,我们有
\[
    2 \partial_\mu \epsilon^\mu = d f(x),
\]
即
\begin{equation}
    f(x) = \frac{2}{d} \partial_\mu \epsilon^\mu,
\end{equation}
其中$d$为时空总维数。这里要注意$g\indices{_\mu^\nu}$实际上就是$\delta_\mu^\nu$,从而求迹之后一定是$d$,而不是$\eta_{\mu \nu}$矩阵求迹后的$2-d$。

下面为了方便起见我们只考虑$g_{\mu \nu}$是欧氏空间度规的情况;也即,我们做一个Wick转动,将所有的相对论性量子场论切换到虚时间下。
我们接着可以尝试确定可能的$\epsilon$的形式。
对\eqref{eq:small-conformal-f-def}两边求导数,我们有


维数大于2的共形场论是一个有限维的普通李群,有有限个生成元。高维共性不变性是整体的,没有局域不变性,从而也不是规范理论。
然而,对二维系统,共形变换是局域变换群,共性不变的场论也是一种规范理论。此时的共性群的生成元有无限多个,且量子化时会出现反常。

\subsection{二维共形群和共形场论}

二维共形场论的重要性一方面来自它自身的奇特性质——对称群有无穷多个生成元,具有很多非常有趣的性质——一方面也有很强的物理意义。
很多一维系统——比如一维电子气演生出来的Luttinger液体(见\soliddoc的第\ref{solid-chap:luttinger-liquid}章)——的低能有效理论看起来非常“简单”,且时间和空间对应得非常好,使得做完Wick转动之后我们几乎就得到了一个定义在\emph{复平面上的}且具有共形对称性的场论。

二维系统的$\epsilon_1$和$\epsilon_2$满足柯西-黎曼条件,从而二维的共性变换就是一个解析函数。

\begin{equation}
    \partial_z = \frac{1}{2} (\partial_1 - \ii \partial_2), \quad \partial_{\bar{z}} = \frac{1}{2} (\partial_1 + \ii \partial_2),
\end{equation}
\begin{equation}
    \partial_1 = \partial_z + \partial_{\bar{z}}, \quad \partial_2 = \ii (\partial_z - \partial_{\bar{z}}).
\end{equation}

\begin{equation}
    \dd{s^2} = \dd{x^2} + \dd{y^2} = \dd{z} \dd{\bar{z}}.
\end{equation}

\begin{equation}
    g_{zz} = g_{\bar{z} \bar{z}} = 0, \quad g_{z \bar{z}} = g_{\bar{z} z} = \frac{1}{2}.
\end{equation}

严格的共形不变性会对$T_{\mu \nu}$做出非常强的限制:旋转对称性意味着
\begin{equation}
    T_{\mu \nu} = T_{\nu \mu},
\end{equation}
标度不变性意味着
\begin{equation}
    T_{\mu}^\nu = 0.
\end{equation}

$x^\mu = (x^1, x^2)$, $x^{\mu'} = (z, \bar{z})$,则
\[
    \left[\pdv{x^\mu}{x^{\mu'}}\right]_{\mu \mu'} = \pmqty{\frac{1}{2} & \frac{1}{2} \\ - \frac{\ii}{2} & \frac{\ii}{2}},
\]
由于
\[
    T_{\mu' \nu'} = \pdv{x^\mu}{x^{\mu'}} T_{\mu \nu} \pdv{x^\nu}{x^{\nu'}},
\]
有
\[
    \left[T_{\mu' \nu'}\right]_{\mu' \nu'} = \left[\pdv{x^\mu}{x^{\mu'}}\right]_{\mu \mu'}^\top \left[T_{\mu \nu}\right]_{\mu \nu} \left[\pdv{x^\mu}{x^{\mu'}}\right]_{\mu \mu'},
\]
计算得到
\begin{equation}
    \begin{aligned}
        T_{zz} &= \frac{1}{4} (T_{11} - T_{22} + 2\ii T_{12}) \eqqcolon T(z, \bar{z}) , \\
        T_{\bar{z} \bar{z}} &= \frac{1}{4} (T_{11} - T_{22} - 2\ii T_{12}) \eqqcolon \bar{T}(z, \bar{z}) \\
        T_{z \bar{z}} &= T_{\bar{z} z} = \frac{1}{4} (T_{11} + T_{22}) = \frac{1}{4} T_\mu^\mu \eqqcolon \frac{1}{4} \Theta(z, \bar{z}).
    \end{aligned}
\end{equation}
使用这些记号,守恒律$\partial_\mu T^{\mu \nu} = 0$变成
% TODO
在共形对称性严格成立的时候,或者说在临界点上,有
\begin{equation}
    \partial_z \bar{T}(z, \bar{z}) = \partial_{\bar{z}} T(z, \bar{z}) = 0.
\end{equation}

\subsection{二维无穷小共性变换}

既然二维共形变换实际上就是一个解析函数,

无穷小生成元:
\begin{equation}
    l_n = - z^{n+1} \partial, \quad \bar{l}_n = - \bar{z}^{n+1} \bar{\partial}.
\end{equation}


\section{共形场论中的关联函数和算符代数}

\subsection{局域场的算符代数}

现在我们考虑相互作用系统。此时随意选择一个$\varphi(x)$,一般来说是不能有以上操作了,因为此时$\phi$未必有完全确定的反常量纲,例如场论中$\phi^2$项的$\beta$函数可能同时显含其它很多项的参数。
然而,\emph{假定}我们确实找到了一组可数的场$\{\varphi_i(x)\}$,使得$\varphi_i^2$项真的就是临界点附近重整化群流的本征方向,从而
\begin{equation}
    \varphi_i(x) = \lambda^{d_i} \varphi_i(\lambda x),
    \label{eq:primary-field-scaling}
\end{equation}
并且,进一步,任何一个局域算符都可以写成这组场的线性组合(或者至少,在关联函数的括号$\expval*{\cdot}$中可以这么做——这种线性展开可能并不一般地成立;后文中很多类似的线性展开也需要如此理解),即
\begin{equation}
    A(x) = \sum_{i} a_i \varphi_i(x).
\end{equation}
对自由理论,显然
\begin{equation}
    \varphi_i(x) = \normord{\varphi(x)^i},
\end{equation}
对相互作用体系我们尚不清楚$\varphi_i(x)$是什么。
无论如何,在\eqref{eq:primary-field-scaling}严格成立时,我们有
\begin{equation}
    \expval*{\varphi_n(x)} = 0.
\end{equation}

现在我们得到了任何局域的算符的代数:就是一个线性代数。$\varphi_n(x)$是局域的算符的基底。
现在考虑两个相隔了有限距离的点上的局域算符的乘积,即$A(x_1) B(x_2)$。
当$x_2 \to x_1$时,这个形式会变成一个单一的局域算符,从而可以做算符展开,
% TODO:适用条件
\begin{equation}
    A(x_1) B(x_2) = \sum_{k} \beta(x_1, x_2) \varphi_k(x_2).
\end{equation}
特别的,如果$A(x)$和$B(x)$实际上都是$\phi(x)$场,
\begin{equation}
    \varphi_p(x_1) \varphi_q(x_2) = \sum_{r} C_{pq}^r(x_1, x_2) \varphi_r(x_2). 
\end{equation}

\begin{equation}
    C_{pq}^r(x_1, x_2) = c_{pq}^r \frac{1}{\abs*{x_1 - x_2}^{d_p + d_q - d_r}}.
\end{equation}
其中$c_{pq}^r$是\concept{算符代数的结构常数}。

似乎我们有两种方法定义共形场论,其一是通过局部的、在尺度变换下协变的算符代数,其二是假定度规在局域尺度变换下只差一个常数。

\begin{equation}
    x' = x + \epsilon,
\end{equation}
\begin{equation}
    \partial_\mu \epsilon_\nu + \partial_\ni \epsilon_\mu = \rho(x) g_{\mu \nu},
\end{equation}
对上式两边求迹,得到
\[
    2 \partial_\mu \epsilon^\mu = D \rho(x),
\]
于是
\begin{equation}
    \partial_\mu \epsilon_\nu + \partial_\nu \epsilon_\mu = \frac{2}{D} g_{\mu \nu} \partial \cdot \epsilon
\end{equation}

% TODO: g_\mu^\mu = D这件事


描述它们的代数不再是普通的李代数,而是Virasora代数。