\chapter{量子色动力学}

本节给出非阿贝尔杨-米尔斯理论的一个例子,著名的描述了强相互作用的量子理论,\concept{量子色动力学}或者简称\concept{QCD}。

QCD是通过$SU(3)$规范对称性得到的理论。$SU(3)$的李代数的结构常数为
\begin{equation}
    f^{123}=1\ ,\quad f^{147}=f^{165}=f^{246}=f^{257}=f^{345}=f^{376}={\frac {1}{2}}\ ,\quad f^{458}=f^{678}={\frac {\sqrt {3}}{2}}.
\end{equation}
我们取$SU(3)$的维数最小的幺正表示,此时李代数的表示的基底是
\begin{equation}
    T^a = \frac{1}{2} \lambda^a,
\end{equation}
$\lambda^a$是所谓的\concept{盖尔曼矩阵},它是泡利矩阵在$SU(3)$中的对应物,定义为
\begin{equation}
    \lambda^1 = 
\end{equation}
这是一个三维表示,从而我们需要三个放在一起的旋量场做表示空间,记作
\begin{equation}
    \psi = \pmqty{\psi_1 \\ \psi_2 \\ \psi_3},
\end{equation}
称为\concept{三重态}。

我们将$\psi$场给出的粒子称为\concept{夸克}。QCD中夸克的标签包括旋量场一定有的动量、自旋、手性,以及$i=1, 2, 3$提供的\concept{色指标}。
在标准模型中还有更多夸克的标签,包括三代夸克,每代有两种。
$A^\mu$场给出的粒子称为\concept{胶子},QCD中胶子的标签包括无质量矢量场一定有的动量和偏振,以及$a=1$到$8$,这里$a$也称为\concept{色指标}。
