\documentclass[hyperref, a4paper]{article}

\usepackage{geometry}
\usepackage{titling}
\usepackage{titlesec}
% No longer needed, since we will use enumitem package
% \usepackage{paralist}
% \usepackage{enumitem}
\usepackage{footnote}
\usepackage{enumerate}
\usepackage{amsmath, amssymb, amsthm}
\usepackage{mathtools}
\usepackage{bbm}
\usepackage{cite}
\usepackage{graphicx}
\usepackage{subfigure}
\usepackage{physics}
\usepackage{tensor}
\usepackage{siunitx}
\usepackage[version=4]{mhchem}
\usepackage{tikz}
\usepackage{xcolor}
\usepackage{listings}
\usepackage{autobreak}
\usepackage[ruled, vlined, linesnumbered]{algorithm2e}
\usepackage{nameref,zref-xr}
\zxrsetup{toltxlabel}
\zexternaldocument*[optics-]{../optics/optics}[optics.pdf]
\zexternaldocument*[solid-]{../solid/solid}[solid.pdf]
\zexternaldocument*[info-]{../information/quantum-circuit}[quantum-circuit.pdf]
\usepackage[colorlinks,unicode]{hyperref} % , linkcolor=black, anchorcolor=black, citecolor=black, urlcolor=black, filecolor=black
\usepackage{prettyref}

% Page style
\geometry{left=3.18cm,right=3.18cm,top=2.54cm,bottom=2.54cm}
\titlespacing{\paragraph}{0pt}{1pt}{10pt}[20pt]
\setlength{\droptitle}{-5em}
\preauthor{\vspace{-10pt}\begin{center}}
\postauthor{\par\end{center}}

% More compact lists 
%\setlist[itemize]{
%    itemindent=17pt, 
%    leftmargin=1pt,
%    listparindent=\parindent,
%    parsep=0pt,
%}

% Math operators
\DeclareMathOperator{\timeorder}{\mathcal{T}}
\DeclareMathOperator{\diag}{diag}
\DeclareMathOperator{\legpoly}{P}
\DeclareMathOperator{\primevalue}{P}
\DeclareMathOperator{\sgn}{sgn}
\newcommand*{\ii}{\mathrm{i}}
\newcommand*{\ee}{\mathrm{e}}
\newcommand*{\const}{\mathrm{const}}
\newcommand*{\suchthat}{\quad \text{s.t.} \quad}
\newcommand*{\argmin}{\arg\min}
\newcommand*{\argmax}{\arg\max}
\newcommand*{\normalorder}[1]{: #1 :}
\newcommand*{\pair}[1]{\langle #1 \rangle}
\newcommand*{\fd}[1]{\mathcal{D} #1}
\DeclareMathOperator{\bigO}{\mathcal{O}}

% TikZ setting
\usetikzlibrary{arrows,shapes,positioning}
\usetikzlibrary{arrows.meta}
\usetikzlibrary{decorations.markings}
\tikzstyle arrowstyle=[scale=1]
\tikzstyle directed=[postaction={decorate,decoration={markings,
    mark=at position .5 with {\arrow[arrowstyle]{stealth}}}}]
\tikzstyle ray=[directed, thick]
\tikzstyle dot=[anchor=base,fill,circle,inner sep=1pt]

% Algorithm setting
% Julia-style code
\SetKwIF{If}{ElseIf}{Else}{if}{}{elseif}{else}{end}
\SetKwFor{For}{for}{}{end}
\SetKwFor{While}{while}{}{end}
\SetKwProg{Function}{function}{}{end}
\SetArgSty{textnormal}

\newcommand*{\concept}[1]{{\textbf{#1}}}

% Embedded codes
\lstset{basicstyle=\ttfamily,
  showstringspaces=false,
  commentstyle=\color{gray},
  keywordstyle=\color{blue}
}

\newcommand{\opticsdoc}{\href{../optics/optics}{the optics note}}
\newcommand{\soliddoc}{\href{../solid/solid}{the solid state physics note}}
\newcommand{\infodoc}{\href{../information/quantum-circuit}{the quantum information note}}

\newrefformat{fig}{Figure~\ref{#1} on page~\pageref{#1}}
\newrefformat{sec}{Section~\ref{#1}}

\title{Quantum Optics by Prof. Saijun Wu}
\author{Jinyuan Wu}
\date{November 4, 2021}

\begin{document}

\maketitle

\section{The rotational wave approximation and the Bloch sphere}

We can also think of the RWT in an intuitive way illustrated in \prettyref{fig:energy-diagram},
where we start from a $\ket*{g}$ state, and then a photon comes in, and the system is brought to the $\ket*{g, n=1}$
state (shown as a dotted line), and then an absorption process happens, and this $\ket*{g, n=1}$ state 
is turned into an $\ket*{e}$ state, and the whole process can be summarized as a $\ket*{g}$ state being 
turned into an $\ket*{e}$ state (the process is illustrated using a blue arrow). RWT can be thought as 
redefining $\ket*{g}$ into $\ket*{g, n=1}$, the energy of which is $\omega_g + \omega$.

\begin{figure}
    \centering
    

\tikzset{every picture/.style={line width=0.75pt}} %set default line width to 0.75pt        

\begin{tikzpicture}[x=0.75pt,y=0.75pt,yscale=-1,xscale=1]
%uncomment if require: \path (0,300); %set diagram left start at 0, and has height of 300

%Straight Lines [id:da18819191009761904] 
\draw    (359,113) -- (450,113) ;
%Straight Lines [id:da1376948784583416] 
\draw    (359,200) -- (450,200) ;
%Straight Lines [id:da21818736070050204] 
\draw [color={rgb, 255:red, 248; green, 231; blue, 28 }  ,draw opacity=1 ]   (414,146) -- (414,154) .. controls (415.67,155.67) and (415.67,157.33) .. (414,159) .. controls (412.33,160.67) and (412.33,162.33) .. (414,164) .. controls (415.67,165.67) and (415.67,167.33) .. (414,169) .. controls (412.33,170.67) and (412.33,172.33) .. (414,174) .. controls (415.67,175.67) and (415.67,177.33) .. (414,179) .. controls (412.33,180.67) and (412.33,182.33) .. (414,184) .. controls (415.67,185.67) and (415.67,187.33) .. (414,189) .. controls (412.33,190.67) and (412.33,192.33) .. (414,194) .. controls (415.67,195.67) and (415.67,197.33) .. (414,199) -- (414,200) -- (414,200) ;
\draw [shift={(414,144)}, rotate = 90] [fill={rgb, 255:red, 248; green, 231; blue, 28 }  ,fill opacity=1 ][line width=0.08]  [draw opacity=0] (12,-3) -- (0,0) -- (12,3) -- cycle    ;
%Straight Lines [id:da6990718420713145] 
\draw  [dash pattern={on 4.5pt off 4.5pt}]  (378,144) -- (450,144) ;
%Straight Lines [id:da6995865544982842] 
\draw    (392,115) -- (392,142) ;
\draw [shift={(392,144)}, rotate = 270] [fill={rgb, 255:red, 0; green, 0; blue, 0 }  ][line width=0.08]  [draw opacity=0] (12,-3) -- (0,0) -- (12,3) -- cycle    ;
\draw [shift={(392,113)}, rotate = 90] [fill={rgb, 255:red, 0; green, 0; blue, 0 }  ][line width=0.08]  [draw opacity=0] (12,-3) -- (0,0) -- (12,3) -- cycle    ;
%Straight Lines [id:da8678920245446373] 
\draw [color={rgb, 255:red, 74; green, 144; blue, 226 }  ,draw opacity=1 ]   (366,115) -- (366,200) ;
\draw [shift={(366,113)}, rotate = 90] [fill={rgb, 255:red, 74; green, 144; blue, 226 }  ,fill opacity=1 ][line width=0.08]  [draw opacity=0] (12,-3) -- (0,0) -- (12,3) -- cycle    ;
%Straight Lines [id:da46707020520155873] 
\draw    (147,113) -- (238,113) ;
%Straight Lines [id:da15216197072886306] 
\draw    (147,200) -- (238,200) ;
%Straight Lines [id:da7329038078465273] 
\draw [color={rgb, 255:red, 248; green, 231; blue, 28 }  ,draw opacity=1 ]   (202,85) -- (202,93) .. controls (203.67,94.67) and (203.67,96.33) .. (202,98) .. controls (200.33,99.67) and (200.33,101.33) .. (202,103) .. controls (203.67,104.67) and (203.67,106.33) .. (202,108) .. controls (200.33,109.67) and (200.33,111.33) .. (202,113) .. controls (203.67,114.67) and (203.67,116.33) .. (202,118) .. controls (200.33,119.67) and (200.33,121.33) .. (202,123) .. controls (203.67,124.67) and (203.67,126.33) .. (202,128) .. controls (200.33,129.67) and (200.33,131.33) .. (202,133) .. controls (203.67,134.67) and (203.67,136.33) .. (202,138) .. controls (200.33,139.67) and (200.33,141.33) .. (202,143) .. controls (203.67,144.67) and (203.67,146.33) .. (202,148) .. controls (200.33,149.67) and (200.33,151.33) .. (202,153) .. controls (203.67,154.67) and (203.67,156.33) .. (202,158) .. controls (200.33,159.67) and (200.33,161.33) .. (202,163) .. controls (203.67,164.67) and (203.67,166.33) .. (202,168) .. controls (200.33,169.67) and (200.33,171.33) .. (202,173) .. controls (203.67,174.67) and (203.67,176.33) .. (202,178) .. controls (200.33,179.67) and (200.33,181.33) .. (202,183) .. controls (203.67,184.67) and (203.67,186.33) .. (202,188) .. controls (200.33,189.67) and (200.33,191.33) .. (202,193) .. controls (203.67,194.67) and (203.67,196.33) .. (202,198) -- (202,200) -- (202,200) ;
\draw [shift={(202,83)}, rotate = 90] [fill={rgb, 255:red, 248; green, 231; blue, 28 }  ,fill opacity=1 ][line width=0.08]  [draw opacity=0] (12,-3) -- (0,0) -- (12,3) -- cycle    ;
%Straight Lines [id:da5384879208606175] 
\draw  [dash pattern={on 4.5pt off 4.5pt}]  (166,83) -- (238,83) ;
%Straight Lines [id:da9870657692472646] 
\draw    (180,85) -- (180,112) ;
\draw [shift={(180,114)}, rotate = 270] [fill={rgb, 255:red, 0; green, 0; blue, 0 }  ][line width=0.08]  [draw opacity=0] (12,-3) -- (0,0) -- (12,3) -- cycle    ;
\draw [shift={(180,83)}, rotate = 90] [fill={rgb, 255:red, 0; green, 0; blue, 0 }  ][line width=0.08]  [draw opacity=0] (12,-3) -- (0,0) -- (12,3) -- cycle    ;
%Straight Lines [id:da34754409527991026] 
\draw [color={rgb, 255:red, 74; green, 144; blue, 226 }  ,draw opacity=1 ]   (154,115) -- (154,200) ;
\draw [shift={(154,113)}, rotate = 90] [fill={rgb, 255:red, 74; green, 144; blue, 226 }  ,fill opacity=1 ][line width=0.08]  [draw opacity=0] (12,-3) -- (0,0) -- (12,3) -- cycle    ;

% Text Node
\draw (452,200) node [anchor=west] [inner sep=0.75pt]    {$\omega _{g}$};
% Text Node
\draw (452,113) node [anchor=west] [inner sep=0.75pt]    {$\omega _{e}$};
% Text Node
\draw (400,128.5) node [anchor=west] [inner sep=0.75pt]    {$|\Delta |$};
% Text Node
\draw (411,172) node [anchor=east] [inner sep=0.75pt]    {$\omega $};
% Text Node
\draw (364,156.5) node [anchor=east] [inner sep=0.75pt]    {$\Omega $};
% Text Node
\draw (240,200) node [anchor=west] [inner sep=0.75pt]    {$\omega _{g}$};
% Text Node
\draw (240,113) node [anchor=west] [inner sep=0.75pt]    {$\omega _{e}$};
% Text Node
\draw (150,97.5) node [anchor=west] [inner sep=0.75pt]    {$|\Delta |$};
% Text Node
\draw (199,172) node [anchor=east] [inner sep=0.75pt]    {$\omega $};
% Text Node
\draw (152,156.5) node [anchor=east] [inner sep=0.75pt]    {$\Omega $};
% Text Node
\draw (184,234) node [anchor=north west][inner sep=0.75pt]   [align=left] {(a)};
% Text Node
\draw (393,234) node [anchor=north west][inner sep=0.75pt]   [align=left] {(b)};


\end{tikzpicture}

    \caption{The energy diagram of a two-level system in an external optical field. A yellow wavy arrow 
    is an incoming photon, and the blue lines labeled with $\Omega$ denote to the coupling between 
    the $\ket*{g}$ state and the $\ket*{e}$ state.
    (a) $\omega - \omega_{eg} > 0$, $\Delta > 0$. (b) $\omega - \omega_{eg} < 0$, $\Delta < 0$.}
    \label{fig:energy-diagram}
\end{figure}

Under the rotational wave approximation and the corresponding RWA transformation, the Hamiltonian of a two-level system is 
\begin{equation}
    H = \frac{\hbar}{2} \vb*{\Omega} \cdot \vb*{\sigma},
    \label{eq:rwa-hamiltonian}
\end{equation}
where the norm of the \concept{Rabi vector} $\vb*{\Omega} = (\Omega_\text{r}, \Omega_\text{i} , \Delta)$ is
\begin{equation}
    \abs*{\vb*{\Omega}} = \sqrt{\abs{\Omega}^2 + \Delta^2},
\end{equation}
\begin{equation}
    \Omega = \frac{-2 E_0^+ \vu*{\epsilon} \cdot \vb*{d}_{eg}}{\hbar}
    \label{eq:rabi-freq}
\end{equation}
is the \concept{Rabi frequency} and $\Delta$ the \concept{detuning}.
The wave function is always in the form of 
\begin{equation}
    \ket*{\psi} = \cos \frac{\theta}{2} \ee^{\ii \varphi /2 } \ket*{g} + \sin \frac{\theta}{2} \ee^{- \ii \varphi / 2} \ket*{e}, 
\end{equation}
and the density matrix is 
\begin{equation}
    \rho = \frac{1}{2} (1 + \vb*{n} \cdot \vb*{\sigma}),
\end{equation}
where 
\begin{equation}
    \vb*{n} = (\sin \theta \cos \varphi, \sin \theta \sin \varphi, \cos \theta).
\end{equation}
It is natural to put $\vb*{n}$ on a sphere, which is known as \concept{Bloch sphere}.
The constructions are standard for a qubit and can be found in Section~\ref{info-sec:single-qubit} in \infodoc.
The equation of motion is 
\[
    \frac{\ii \hbar}{2} \dot{\vb*{n}} \cdot \vb*{\sigma} = \ii \hbar \dot{\rho} = \comm*{H}{\rho} = \comm*{\frac{\hbar}{2} \vb*{\Omega} \cdot \vb*{\sigma}}{\frac{1}{2} \vb*{n} \cdot \vb*{\sigma}} = \frac{\hbar}{4} \cdot 2 \ii (\vb*{\Omega} \times \vb*{n}) \cdot \vb*{\sigma},
\]
and therefore we have 
\begin{equation}
    \dot{\vb*{n}} = \vb*{\Omega} \times \vb*{n}. 
    \label{eq:n-eom}
\end{equation}
We find that the Bloch vector rotates with the frequency of 
\begin{equation}
    \tilde{\Omega} = \abs*{\vb*{\Omega}} = \sqrt{\abs*{\Omega}^2 + \Delta^2}.
\end{equation}

We review some properties about the Bloch sphere.
First, an equation of motion of a c-number defined in terms of the wave function is likely to be an equation of motion of an operator, and this is indeed the case of \eqref{eq:n-eom}, where we may find $\vb*{n} = \expval{\vb*{\sigma}}$.
Also, it can be verified that 
\begin{equation}
    \abs*{\braket*{\psi_1}{\psi_2}}^2 = \trace(\rho_1 \rho_2) = \frac{1}{2} (1 + \vb*{n}_1 \cdot \vb*{n}_2),
\end{equation}
and therefore we find that two wave functions are the same or differ with only a phase factor if and only if they have the same Bloch vector, and they are orthogonal if and only if their Bloch vectors are exactly the opposite of each other.

Note that the direction of the Rabi vector is not determined.
Usually we use the convention
\begin{equation}
    \vb*{\Omega} = (\Omega, 0, \Delta), 
    \label{eq:omega-convention}
\end{equation} 
and the two Bloch vectors
\begin{equation}
    \vb*{n}_{\pm \Omega} = \pm (\sin \theta, 0, \cos \theta)
\end{equation}
represents two eigenstates of \eqref{eq:rwa-hamiltonian}, namely $\ket*{\tilde{g}}$ (with eigenvalue $- \hbar \abs*{\vb*{\Omega}} / 2$) and $\ket*{\tilde{e}}$ (with eigenvalue $\hbar \abs*{\vb*{\Omega}} / 2$), where 
\begin{equation}
    \theta = \arctan \frac{\Omega}{\Delta}.
\end{equation}
The whole formulation - representing the wave function of a two-level atom as a vector on the Bloch sphere, 
rephrasing the EOM as \eqref{eq:n-eom} - is together called \concept{Feynman–Vernon–Hellwarth representation (FVH)}.
\eqref{eq:n-eom} directs $\vb*{n}$ to rotate anticlockwise when $\vb*{\Omega}$ is toward us.

\section{Control a two-level system with external fields}

Note that since \eqref{eq:rabi-freq}, a two-level system's state can be controlled via changing the external field.

\begin{figure}
    \centering
    

\tikzset{every picture/.style={line width=0.75pt}} %set default line width to 0.75pt        

\begin{tikzpicture}[x=0.75pt,y=0.75pt,yscale=-1,xscale=1]
%uncomment if require: \path (0,300); %set diagram left start at 0, and has height of 300

%Straight Lines [id:da8125477960653353] 
\draw    (95,171) -- (214,171) ;
\draw [shift={(216,171)}, rotate = 180] [fill={rgb, 255:red, 0; green, 0; blue, 0 }  ][line width=0.08]  [draw opacity=0] (12,-3) -- (0,0) -- (12,3) -- cycle    ;
%Straight Lines [id:da5838070816593157] 
\draw    (95,171) -- (95,65.02) ;
\draw [shift={(95,63.02)}, rotate = 450] [fill={rgb, 255:red, 0; green, 0; blue, 0 }  ][line width=0.08]  [draw opacity=0] (12,-3) -- (0,0) -- (12,3) -- cycle    ;
%Straight Lines [id:da8598243285485796] 
\draw    (95,171) -- (188.2,125.89) ;
\draw [shift={(190,125.02)}, rotate = 514.1700000000001] [fill={rgb, 255:red, 0; green, 0; blue, 0 }  ][line width=0.08]  [draw opacity=0] (12,-3) -- (0,0) -- (12,3) -- cycle    ;
%Straight Lines [id:da43220335567629653] 
\draw    (95,171) -- (155.85,84.66) ;
\draw [shift={(157,83.02)}, rotate = 485.17] [color={rgb, 255:red, 0; green, 0; blue, 0 }  ][line width=0.75]    (10.93,-3.29) .. controls (6.95,-1.4) and (3.31,-0.3) .. (0,0) .. controls (3.31,0.3) and (6.95,1.4) .. (10.93,3.29)   ;
%Straight Lines [id:da7003435646275242] 
\draw    (95,106.02) -- (95,171) ;
\draw [shift={(95,104.02)}, rotate = 90] [color={rgb, 255:red, 0; green, 0; blue, 0 }  ][line width=0.75]    (10.93,-3.29) .. controls (6.95,-1.4) and (3.31,-0.3) .. (0,0) .. controls (3.31,0.3) and (6.95,1.4) .. (10.93,3.29)   ;
%Shape: Arc [id:dp7751197002401395] 
\draw  [draw opacity=0][dash pattern={on 0.84pt off 2.51pt}] (95,104.02) .. controls (108.77,94.84) and (129.93,97.81) .. (145.49,112.08) .. controls (155.82,121.57) and (161.4,133.95) .. (161.64,145.45) -- (121.82,137.87) -- cycle ; \draw  [color={rgb, 255:red, 208; green, 2; blue, 27 }  ,draw opacity=1 ][dash pattern={on 0.84pt off 2.51pt}] (95,104.02) .. controls (108.77,94.84) and (129.93,97.81) .. (145.49,112.08) .. controls (155.82,121.57) and (161.4,133.95) .. (161.64,145.45) ;
%Straight Lines [id:da9547032756267608] 
\draw    (159.77,146.16) -- (95,171) ;
\draw [shift={(161.64,145.45)}, rotate = 159.02] [color={rgb, 255:red, 0; green, 0; blue, 0 }  ][line width=0.75]    (10.93,-3.29) .. controls (6.95,-1.4) and (3.31,-0.3) .. (0,0) .. controls (3.31,0.3) and (6.95,1.4) .. (10.93,3.29)   ;
%Straight Lines [id:da44322203840446583] 
\draw [color={rgb, 255:red, 208; green, 2; blue, 27 }  ,draw opacity=1 ]   (160.64,140.45) -- (161.25,143.48) ;
\draw [shift={(161.64,145.45)}, rotate = 258.69] [fill={rgb, 255:red, 208; green, 2; blue, 27 }  ,fill opacity=1 ][line width=0.08]  [draw opacity=0] (12,-3) -- (0,0) -- (12,3) -- cycle    ;
%Shape: Ellipse [id:dp5189226557816553] 
\draw  [color={rgb, 255:red, 74; green, 144; blue, 226 }  ,draw opacity=1 ][dash pattern={on 0.84pt off 2.51pt}] (29,145.45) .. controls (29,138.81) and (58.69,133.43) .. (95.32,133.43) .. controls (131.95,133.43) and (161.64,138.81) .. (161.64,145.45) .. controls (161.64,152.08) and (131.95,157.46) .. (95.32,157.46) .. controls (58.69,157.46) and (29,152.08) .. (29,145.45) -- cycle ;
%Straight Lines [id:da5564344266734615] 
\draw  [dash pattern={on 4.5pt off 4.5pt}]  (30.87,146.17) -- (95,171) ;
\draw [shift={(29,145.45)}, rotate = 21.17] [color={rgb, 255:red, 0; green, 0; blue, 0 }  ][line width=0.75]    (10.93,-3.29) .. controls (6.95,-1.4) and (3.31,-0.3) .. (0,0) .. controls (3.31,0.3) and (6.95,1.4) .. (10.93,3.29)   ;

% Text Node
\draw (218,171) node [anchor=west] [inner sep=0.75pt]    {$x$};
% Text Node
\draw (95,59.62) node [anchor=south] [inner sep=0.75pt]    {$z\left(\boldsymbol{\Omega } \ \text{with} \ \boldsymbol{E} =0\right)$};
% Text Node
\draw (192,121.62) node [anchor=south west] [inner sep=0.75pt]    {$y$};
% Text Node
\draw (150,64.4) node [anchor=north west][inner sep=0.75pt]    {$\boldsymbol{\Omega } \ \text{with external field}$};
% Text Node
\draw (27,95.4) node [anchor=north west][inner sep=0.75pt]    {$\boldsymbol{n}( t=0)$};
% Text Node
\draw (164,138.4) node [anchor=north west][inner sep=0.75pt]    {$\boldsymbol{n}( t=T/2)$};


\end{tikzpicture}

    \caption{The consequence of a $\pi$-pulse illustrated with the Bloch sphere}
    \label{fig:pi-pulse-result}
\end{figure}

Consider we start from $\ket*{g}$, i.e. $\vb*{n} = \vb*{e}_z$.
We want to excite the atom to $\ket*{e}$, i.e. $\vb*{n} = - \vb*{e}_z$.
When there is no external field $\vb*{\Omega}$ is along the $z$ axis, and therefore $\vb*{n}$ stays at $\vb*{e}_z$. 
Now suppose we add an electric field $\vb*{E}_0$, the Bloch vector starts to rotate around the new Rabi vector (depicted as the red curve in \prettyref{fig:pi-pulse-result}).
If, after $t = T/2 = \pi / \tilde{\Omega}$, we remove the external field, the Bloch vector will again rotate around the $z$ axis (depicted as the blue curve in \prettyref{fig:pi-pulse-result}).
Therefore we successfully change the $\theta$ angle of $\vb*{n}$.
This process is called a \concept{$\pi$-pulse}.
With a series of $\pi$-pulse, we can make $\vb*{n}$ arbitrarily close to $- \vb*{e}_z$.

This is quite an interesting result, because it shows that the atom can be excited even with the presence of detuning.
This can also be seen as an example of the $E$-$t$ uncertainty principle, because what we are actually do is to use \emph{pulses} to create some uncertainty in the energy to make up for the gap between $\omega$ and $\omega_{eg}$.

\section{Adiabatic states}

When $\abs*{\vb*{\Omega}} \gg \abs*{\mel**{\tilde{e}}{\partial_t}{\tilde{g}}}$, or in other words the perturbation mix eigenstates very slowly, we can make the adiabatic approximation and we have 

\section{Perturbative solution}

Now we go back to 
\begin{equation}
    \ii \dot{c}_e = \Delta c_e + \frac{\Omega}{2} c_g, \quad \ii \dot{c}_g = \frac{\Omega^*}{2} c_e.
\end{equation}
In the first order perturbation, we can just ignore $c_e$ in the second equation if we start from the ground state.
We, therefore, have 
\begin{equation}
    c^{(1)}_e(T) = - \ii \int_0^T \dd{t} \frac{\Omega}{2} \ee^{\ii \Delta t} \dd{t} = - \frac{\ii}{2} \bar{\Omega}(\omega)|_{\omega = \Delta},
\end{equation}
where 
\begin{equation}
    \bar{\Omega}(\omega) = \int_{-\infty}^\infty \bar{\Omega}(t) \ee^{\ii \omega t} \dd{t}, 
\end{equation}
and 
\begin{equation}
    \bar{\Omega}(t) = \begin{cases}
        \Omega(t) , &\quad 0 < t < T, \\
        0         , &\quad \text{otherwise}.
    \end{cases}
\end{equation}
We see that the Fourier transformation of the windowed Rabi frequency is an approximation of how strongly the two-level system will be disturbed.

\section{Atomic state detection}

Detecting what quantum state is an atom on has similar mathematical structure with single photon interferometers.
Suppose we have a two-level system, of which we can measure directly $\sigma^z$, and we want to estimate, say, the $\theta$ angle.
We have
\begin{equation}
    \expval{\sigma^z} = \cos^2 \frac{\theta}{2} - \sin^2 \frac{\theta}{2} = \cos \theta,
\end{equation}
and the standard variation is 
\begin{equation}
    \Delta \sigma^z = \sqrt{\expval{(\sigma^z)^2} - \expval{\sigma^z}^2} = \sqrt{1 - \cos^2 \theta} = \sin \theta.
\end{equation}
The uncertainty of $\theta$ is therefore 
\begin{equation}
    \Delta\theta = \frac{\Delta \sigma^z}{\pdv*{\expval{\sigma^z}}{\theta}} = 1.
\end{equation}
Therefore the measurement of $\theta$ with a single atom is extremely imprecise.
If we can generate lots of atoms in the same state, we can measure the sum of $\sigma^z$ of them, and we have 
\begin{equation}
    \expval{\Sigma^z} = N \cos \theta.
\end{equation}
On the other hand, 
\begin{eqnarray}
    \Delta \Sigma^z = \sqrt{N} \sin \theta,
\end{eqnarray}
and we have 
\begin{equation}
    \Delta \theta = \frac{1}{\sqrt{N}},
\end{equation}
which is the standard quantum error.

\section{Atomic clock}

A light clock is discussed in the last problem of \href{quantum-optics-homework/3/3-discussion.pdf}{this homework},
where we claimed that an atomic clock is much more precise. Now we can use the technologies mentioned in the 
previous few sections to build an atomic clock based on Ramsey’s method of separated, oscillatory fields. 

\begin{figure}
    \centering
    

\tikzset{every picture/.style={line width=0.75pt}} %set default line width to 0.75pt        

\begin{tikzpicture}[x=0.75pt,y=0.75pt,yscale=-1,xscale=1]
%uncomment if require: \path (0,300); %set diagram left start at 0, and has height of 300

%Straight Lines [id:da51441218283759] 
\draw [color={rgb, 255:red, 248; green, 231; blue, 28 }  ,draw opacity=1 ][line width=2.25]    (107,86) .. controls (108.67,87.67) and (108.67,89.33) .. (107,91) .. controls (105.33,92.67) and (105.33,94.33) .. (107,96) .. controls (108.67,97.67) and (108.67,99.33) .. (107,101) .. controls (105.33,102.67) and (105.33,104.33) .. (107,106) .. controls (108.67,107.67) and (108.67,109.33) .. (107,111) .. controls (105.33,112.67) and (105.33,114.33) .. (107,116) .. controls (108.67,117.67) and (108.67,119.33) .. (107,121) .. controls (105.33,122.67) and (105.33,124.33) .. (107,126) .. controls (108.67,127.67) and (108.67,129.33) .. (107,131) .. controls (105.33,132.67) and (105.33,134.33) .. (107,136) -- (107,137.69) -- (107,145.69) ;
\draw [shift={(107,151.69)}, rotate = 270] [fill={rgb, 255:red, 248; green, 231; blue, 28 }  ,fill opacity=1 ][line width=0.08]  [draw opacity=0] (19.2,-4.8) -- (0,0) -- (19.2,4.8) -- cycle    ;
%Straight Lines [id:da7662175680694188] 
\draw [color={rgb, 255:red, 248; green, 231; blue, 28 }  ,draw opacity=1 ][line width=2.25]    (282,86) .. controls (283.67,87.67) and (283.67,89.33) .. (282,91) .. controls (280.33,92.67) and (280.33,94.33) .. (282,96) .. controls (283.67,97.67) and (283.67,99.33) .. (282,101) .. controls (280.33,102.67) and (280.33,104.33) .. (282,106) .. controls (283.67,107.67) and (283.67,109.33) .. (282,111) .. controls (280.33,112.67) and (280.33,114.33) .. (282,116) .. controls (283.67,117.67) and (283.67,119.33) .. (282,121) .. controls (280.33,122.67) and (280.33,124.33) .. (282,126) .. controls (283.67,127.67) and (283.67,129.33) .. (282,131) .. controls (280.33,132.67) and (280.33,134.33) .. (282,136) -- (282,137.69) -- (282,145.69) ;
\draw [shift={(282,151.69)}, rotate = 270] [fill={rgb, 255:red, 248; green, 231; blue, 28 }  ,fill opacity=1 ][line width=0.08]  [draw opacity=0] (19.2,-4.8) -- (0,0) -- (19.2,4.8) -- cycle    ;
%Straight Lines [id:da2383040495531039] 
\draw    (74.5,151.69) -- (352.5,151.69) ;
\draw [shift={(213.5,151.69)}, rotate = 180] [fill={rgb, 255:red, 0; green, 0; blue, 0 }  ][line width=0.08]  [draw opacity=0] (12,-3) -- (0,0) -- (12,3) -- cycle    ;
%Straight Lines [id:da8650147298236577] 
\draw    (105,86) -- (105,56.69) ;
%Straight Lines [id:da3799472249470688] 
\draw    (111,86) -- (111,56.69) ;
%Straight Lines [id:da7976983986692758] 
\draw    (105,71.34) -- (111,71.34) ;
%Straight Lines [id:da39476271636994875] 
\draw    (113,71.34) -- (131.5,71.34) ;
\draw [shift={(111,71.34)}, rotate = 0] [fill={rgb, 255:red, 0; green, 0; blue, 0 }  ][line width=0.08]  [draw opacity=0] (12,-3) -- (0,0) -- (12,3) -- cycle    ;
%Straight Lines [id:da05730624651790883] 
\draw    (73.5,71.34) -- (103,71.34) ;
\draw [shift={(105,71.34)}, rotate = 180] [fill={rgb, 255:red, 0; green, 0; blue, 0 }  ][line width=0.08]  [draw opacity=0] (12,-3) -- (0,0) -- (12,3) -- cycle    ;
%Shape: Rectangle [id:dp9609421835293179] 
\draw  [draw opacity=0][fill={rgb, 255:red, 0; green, 0; blue, 0 }  ,fill opacity=0.35 ] (323,129.71) -- (351.5,129.71) -- (351.5,146) -- (323,146) -- cycle ;
%Shape: Rectangle [id:dp14743106980136078] 
\draw  [draw opacity=0][fill={rgb, 255:red, 0; green, 0; blue, 0 }  ,fill opacity=0.35 ] (323,157) -- (351.5,157) -- (351.5,173.29) -- (323,173.29) -- cycle ;
%Shape: Rectangle [id:dp48739879197409763] 
\draw  [color={rgb, 255:red, 0; green, 0; blue, 0 }  ,draw opacity=1 ][fill={rgb, 255:red, 0; green, 0; blue, 0 }  ,fill opacity=1 ] (31,138.69) -- (75.3,138.69) -- (75.3,164) -- (31,164) -- cycle ;
%Straight Lines [id:da7004524085571604] 
\draw    (109,116) -- (279.5,116) ;
\draw [shift={(281.5,116)}, rotate = 180] [fill={rgb, 255:red, 0; green, 0; blue, 0 }  ][line width=0.08]  [draw opacity=0] (12,-3) -- (0,0) -- (12,3) -- cycle    ;
\draw [shift={(107,116)}, rotate = 0] [fill={rgb, 255:red, 0; green, 0; blue, 0 }  ][line width=0.08]  [draw opacity=0] (12,-3) -- (0,0) -- (12,3) -- cycle    ;
%Curve Lines [id:da12367316584764931] 
\draw [color={rgb, 255:red, 245; green, 166; blue, 35 }  ,draw opacity=1 ]   (191.5,53.69) .. controls (155.73,58.28) and (143.01,78.84) .. (121.81,94.72) ;
\draw [shift={(120.5,95.69)}, rotate = 323.97] [fill={rgb, 255:red, 245; green, 166; blue, 35 }  ,fill opacity=1 ][line width=0.08]  [draw opacity=0] (12,-3) -- (0,0) -- (12,3) -- cycle    ;
%Curve Lines [id:da3353753608603731] 
\draw [color={rgb, 255:red, 245; green, 166; blue, 35 }  ,draw opacity=1 ]   (228.5,55.69) .. controls (246.84,52.79) and (259.58,70.38) .. (271.24,84.2) ;
\draw [shift={(272.5,85.69)}, rotate = 229.4] [fill={rgb, 255:red, 245; green, 166; blue, 35 }  ,fill opacity=1 ][line width=0.08]  [draw opacity=0] (12,-3) -- (0,0) -- (12,3) -- cycle    ;

% Text Node
\draw (85,67.94) node [anchor=south] [inner sep=0.75pt]    {$l = v \tau$};
% Text Node
\draw (309,179) node [anchor=north west][inner sep=0.75pt]   [align=left] {detector};
% Text Node
\draw (194.25,112.6) node [anchor=south] [inner sep=0.75pt]    {$L = v T$};
% Text Node
\draw (90,158) node [anchor=north west][inner sep=0.75pt]   [align=left] {ground state atom jet with velocity $\displaystyle v$};
% Text Node
\draw (144,27) node [anchor=north west][inner sep=0.75pt]   [align=left] {identical coherent light beams};


\end{tikzpicture}

    \caption{A atomic clock based on Ramsey's method of separated, oscillatory fields}
    \label{fig:clock}
\end{figure}

The device is illustrated as \prettyref{fig:clock}. We use the convention \eqref{eq:omega-convention}, and assume 
$\Delta \ll \Omega$, so $\tilde{\Omega} \approx \Omega$ and $\vb*{\Omega} = \Omega \vu*{x}$. We carefully 
choose the width $l$ of the two beams so that 
\begin{equation}
    \tilde{\Omega} \frac{l}{v} = \tilde{\Omega} \tau \approx \Omega \tau = \frac{\pi}{2}.
\end{equation}
What happens on the Bloch vector of a single atom is shown in \prettyref{fig:bloch-vector}. An atom first 
goes through the first light beam and rotate around the $x$ axis \SI{90}{\degree}, then goes around the $z$ 
axis and rotate $\tilde{\Omega} T$, and finally goes through the second light beam and rotate around the $x$
axis \SI{90}{\degree} again. It is easy to find that if $T$ is chose so that $\vb*{n}(T + \tau) = - \vu*{y}$,
then $\vb*{n}(T + 2 \tau) = - \vu*{z}$, while if $\vb*{n}(T + \tau) = \vu*{y}$, then $\vb*{n}(T + 2 \tau) = \vu*{z}$.
Therefore, we find that the probability to find the final state on $\ket*{e}$ - which we denote as $P_e$ - depends
on $T$. It is straightforward to find that 
\[
    \exp(-\ii \frac{\Omega}{2} \sigma^x \tau) = \exp(- \frac{\pi}{4} \sigma^x) = \frac{1}{\sqrt{2}} - \frac{1}{\sqrt{2}} \ii \sigma^x,
\]
and 
\[
    \exp(- \ii \frac{\Delta}{2} \sigma^z T) = \cos(\frac{T \Delta}{2}) - \ii \sin(\frac{T \Delta}{2}) \sigma^z,
\]
so the total transformation matrix is 
\begin{equation}
    \begin{aligned}
        U &= \exp(-\ii \frac{\Omega}{2} \sigma^x \tau) \exp(- \ii \frac{\Delta}{2} \sigma^z T) \exp(-\ii \frac{\Omega}{2} \sigma^x \tau) \\
        &= \pmqty{ - \ii \sin \frac{T \Delta}{2} & - \ii \cos \frac{T \Delta}{2} \\ - \ii \cos \frac{T \Delta}{2} & \ii \sin \frac{T \Delta}{2} },
    \end{aligned}
\end{equation}
and therefore 
\begin{equation}
    \ket*{\psi}(T + 2\tau) = U \ket*{\psi}(0) = \pmqty{ - \ii \sin \frac{T \Delta}{2} \\ - \ii \cos \frac{T \Delta}{2} },
\end{equation}
and we have 
\begin{equation}
    P_e = \cos^2 \frac{T \Delta}{2} = \frac{1}{2} (1 + \cos(T \Delta)).
\end{equation}
These oscillations are called \concept{Ramsey fringes}. 

Here is how to use the atomic clock: we need a laser device with adjustable frequencies, and a device which 
can generate a jet of atoms on the ground state whose velocity is known, and we can adjust the frequency 
of the laser to maximize $P_e$. When this is done, we know the frequency of the laser is the same as $\omega_{eg}$.
We known $\omega_{eg}$ can hardly be changed by interaction with the external environment, so after we maximize 
$P_e$, we extract the hard-to-change frequency $\omega_{eg}$ into $\omega$, the inverse of which is a ruler of time.

\begin{figure}
    \centering
    

\tikzset{every picture/.style={line width=0.75pt}} %set default line width to 0.75pt        

\begin{tikzpicture}[x=0.75pt,y=0.75pt,yscale=-1,xscale=1]
%uncomment if require: \path (0,312); %set diagram left start at 0, and has height of 312

%Straight Lines [id:da7848980629300351] 
\draw [color={rgb, 255:red, 248; green, 231; blue, 28 }  ,draw opacity=1 ]   (200.75,171.75) -- (38.21,269.66) ;
\draw [shift={(36.5,270.69)}, rotate = 328.94] [color={rgb, 255:red, 248; green, 231; blue, 28 }  ,draw opacity=1 ][line width=0.75]    (10.93,-3.29) .. controls (6.95,-1.4) and (3.31,-0.3) .. (0,0) .. controls (3.31,0.3) and (6.95,1.4) .. (10.93,3.29)   ;
%Shape: Circle [id:dp2022100392404811] 
\draw   (113,171.75) .. controls (113,123.29) and (152.29,84) .. (200.75,84) .. controls (249.21,84) and (288.5,123.29) .. (288.5,171.75) .. controls (288.5,220.21) and (249.21,259.5) .. (200.75,259.5) .. controls (152.29,259.5) and (113,220.21) .. (113,171.75) -- cycle ;
%Shape: Ellipse [id:dp11699912573923155] 
\draw  [fill={rgb, 255:red, 80; green, 227; blue, 194 }  ,fill opacity=0.23 ] (113,171.75) .. controls (113,160.7) and (152.29,151.75) .. (200.75,151.75) .. controls (249.21,151.75) and (288.5,160.7) .. (288.5,171.75) .. controls (288.5,182.8) and (249.21,191.75) .. (200.75,191.75) .. controls (152.29,191.75) and (113,182.8) .. (113,171.75) -- cycle ;
%Straight Lines [id:da050591265379689165] 
\draw    (200.75,171.75) -- (200.75,56.69) ;
\draw [shift={(200.75,54.69)}, rotate = 90] [fill={rgb, 255:red, 0; green, 0; blue, 0 }  ][line width=0.08]  [draw opacity=0] (12,-3) -- (0,0) -- (12,3) -- cycle    ;
%Straight Lines [id:da550944948978433] 
\draw    (200.75,171.75) -- (357.5,171.69) ;
\draw [shift={(359.5,171.69)}, rotate = 179.98] [fill={rgb, 255:red, 0; green, 0; blue, 0 }  ][line width=0.08]  [draw opacity=0] (12,-3) -- (0,0) -- (12,3) -- cycle    ;
%Straight Lines [id:da7132616118498538] 
\draw    (200.75,171.75) -- (101.21,231.66) ;
\draw [shift={(99.5,232.69)}, rotate = 328.96] [fill={rgb, 255:red, 0; green, 0; blue, 0 }  ][line width=0.08]  [draw opacity=0] (12,-3) -- (0,0) -- (12,3) -- cycle    ;
%Straight Lines [id:da19326397751399682] 
\draw  [dash pattern={on 4.5pt off 4.5pt}]  (42,171.81) -- (200.75,171.75) ;
%Straight Lines [id:da5007447778090508] 
\draw [color={rgb, 255:red, 208; green, 2; blue, 27 }  ,draw opacity=1 ][line width=1.5]    (200.75,171.75) -- (200.75,87) ;
\draw [shift={(200.75,84)}, rotate = 90] [color={rgb, 255:red, 208; green, 2; blue, 27 }  ,draw opacity=1 ][line width=1.5]    (14.21,-4.28) .. controls (9.04,-1.82) and (4.3,-0.39) .. (0,0) .. controls (4.3,0.39) and (9.04,1.82) .. (14.21,4.28)   ;
%Straight Lines [id:da8090900757565247] 
\draw [color={rgb, 255:red, 74; green, 144; blue, 226 }  ,draw opacity=1 ]   (113.71,162.69) -- (113.47,167.98) ;
\draw [shift={(113.39,169.98)}, rotate = 272.53] [fill={rgb, 255:red, 74; green, 144; blue, 226 }  ,fill opacity=1 ][line width=0.08]  [draw opacity=0] (12,-3) -- (0,0) -- (12,3) -- cycle    ;
%Straight Lines [id:da59853592736301] 
\draw [color={rgb, 255:red, 80; green, 227; blue, 194 }  ,draw opacity=1 ]   (168.31,153.14) -- (167.47,153.26) ;
\draw [shift={(165.5,153.55)}, rotate = 351.68] [fill={rgb, 255:red, 80; green, 227; blue, 194 }  ,fill opacity=1 ][line width=0.08]  [draw opacity=0] (12,-3) -- (0,0) -- (12,3) -- cycle    ;
%Straight Lines [id:da8934478617364343] 
\draw  [dash pattern={on 4.5pt off 4.5pt}]  (200.75,288.81) -- (200.75,171.75) ;
%Shape: Arc [id:dp3798664572181705] 
\draw  [draw opacity=0] (178.5,245.69) .. controls (172.04,239.07) and (166.78,228.83) .. (163.69,215.56) .. controls (159.05,195.64) and (160.3,173.18) .. (166.02,154.23) -- (200.82,183.24) -- cycle ; \draw  [color={rgb, 255:red, 184; green, 233; blue, 134 }  ,draw opacity=1 ] (178.5,245.69) .. controls (172.04,239.07) and (166.78,228.83) .. (163.69,215.56) .. controls (159.05,195.64) and (160.3,173.18) .. (166.02,154.23) ;
%Shape: Arc [id:dp27150518666038126] 
\draw  [draw opacity=0] (113.39,169.98) .. controls (114.14,122.23) and (152.8,83.75) .. (200.38,83.75) .. controls (200.78,83.75) and (201.18,83.75) .. (201.58,83.76) -- (200.38,171.41) -- cycle ; \draw  [color={rgb, 255:red, 74; green, 144; blue, 226 }  ,draw opacity=1 ] (113.39,169.98) .. controls (114.14,122.23) and (152.8,83.75) .. (200.38,83.75) .. controls (200.78,83.75) and (201.18,83.75) .. (201.58,83.76) ;
%Shape: Arc [id:dp5432770976020507] 
\draw  [draw opacity=0] (168.45,153.16) .. controls (178.45,152.26) and (189.35,151.77) .. (200.75,151.77) .. controls (249.19,151.77) and (288.46,160.71) .. (288.46,171.75) .. controls (288.46,182.79) and (249.19,191.73) .. (200.75,191.73) .. controls (152.31,191.73) and (113.04,182.79) .. (113.04,171.75) .. controls (113.04,171.42) and (113.08,171.1) .. (113.14,170.77) -- (200.75,171.75) -- cycle ; \draw  [color={rgb, 255:red, 80; green, 227; blue, 194 }  ,draw opacity=1 ] (168.45,153.16) .. controls (178.45,152.26) and (189.35,151.77) .. (200.75,151.77) .. controls (249.19,151.77) and (288.46,160.71) .. (288.46,171.75) .. controls (288.46,182.79) and (249.19,191.73) .. (200.75,191.73) .. controls (152.31,191.73) and (113.04,182.79) .. (113.04,171.75) .. controls (113.04,171.42) and (113.08,171.1) .. (113.14,170.77) ;
%Straight Lines [id:da9377445119383205] 
\draw [color={rgb, 255:red, 208; green, 2; blue, 27 }  ,draw opacity=1 ][line width=1.5]    (200.75,171.75) -- (116,171.75) ;
\draw [shift={(113,171.75)}, rotate = 360] [color={rgb, 255:red, 208; green, 2; blue, 27 }  ,draw opacity=1 ][line width=1.5]    (14.21,-4.28) .. controls (9.04,-1.82) and (4.3,-0.39) .. (0,0) .. controls (4.3,0.39) and (9.04,1.82) .. (14.21,4.28)   ;
%Straight Lines [id:da34551979860290394] 
\draw [color={rgb, 255:red, 208; green, 2; blue, 27 }  ,draw opacity=1 ][line width=1.5]    (200.75,171.75) -- (170.14,155.12) ;
\draw [shift={(167.5,153.69)}, rotate = 28.51] [color={rgb, 255:red, 208; green, 2; blue, 27 }  ,draw opacity=1 ][line width=1.5]    (14.21,-4.28) .. controls (9.04,-1.82) and (4.3,-0.39) .. (0,0) .. controls (4.3,0.39) and (9.04,1.82) .. (14.21,4.28)   ;
%Straight Lines [id:da9440630200952458] 
\draw [color={rgb, 255:red, 184; green, 233; blue, 134 }  ,draw opacity=1 ]   (176.8,243.22) -- (177.37,244.04) ;
\draw [shift={(178.5,245.69)}, rotate = 235.43] [fill={rgb, 255:red, 184; green, 233; blue, 134 }  ,fill opacity=1 ][line width=0.08]  [draw opacity=0] (12,-3) -- (0,0) -- (12,3) -- cycle    ;
%Straight Lines [id:da6204836007465544] 
\draw [color={rgb, 255:red, 208; green, 2; blue, 27 }  ,draw opacity=1 ][line width=1.5]    (200.75,171.75) -- (179.36,242.81) ;
\draw [shift={(178.5,245.69)}, rotate = 286.75] [color={rgb, 255:red, 208; green, 2; blue, 27 }  ,draw opacity=1 ][line width=1.5]    (14.21,-4.28) .. controls (9.04,-1.82) and (4.3,-0.39) .. (0,0) .. controls (4.3,0.39) and (9.04,1.82) .. (14.21,4.28)   ;

% Text Node
\draw (200.75,51.29) node [anchor=south] [inner sep=0.75pt]    {$z$};
% Text Node
\draw (101.5,236.09) node [anchor=north west][inner sep=0.75pt]    {$x$};
% Text Node
\draw (361.5,171.69) node [anchor=west] [inner sep=0.75pt]    {$y$};
% Text Node
\draw (34.5,267.29) node [anchor=south east] [inner sep=0.75pt]    {$\boldsymbol{\Omega }$};
% Text Node
\draw (218,64.4) node [anchor=north west][inner sep=0.75pt]    {$t=0$};
% Text Node
\draw (101,168.35) node [anchor=south east] [inner sep=0.75pt]    {$t=\tau $};
% Text Node
\draw (152,130.4) node [anchor=north west][inner sep=0.75pt]    {$t=T+\tau $};
% Text Node
\draw (118,101) node [anchor=north west][inner sep=0.75pt]  [color={rgb, 255:red, 74; green, 144; blue, 226 }  ,opacity=1 ] [align=left] {1};
% Text Node
\draw (244.5,188.5) node [anchor=north west][inner sep=0.75pt]  [color={rgb, 255:red, 80; green, 227; blue, 194 }  ,opacity=1 ] [align=left] {2};
% Text Node
\draw (114,266.4) node [anchor=north west][inner sep=0.75pt]    {$t=T+2\tau $};
% Text Node
\draw (152.5,214.5) node [anchor=north west][inner sep=0.75pt]  [color={rgb, 255:red, 184; green, 233; blue, 134 }  ,opacity=1 ] [align=left] {3};


\end{tikzpicture}

    \caption{How the Bloch vector moves for atoms in \prettyref{fig:clock}}
    \label{fig:bloch-vector}
\end{figure}

It can be easily seen that the mathematical structure of the atomic clock is just the same as interferometers
of light. Indeed, the two states - $\ket*{g}$ and $\ket*{e}$ - are just two modes of the atoms; since there is 
coupling between the two states in the distance between two light beams, we can regard this period as two 
arms of a Michelson interferometer and regard the two light beams as two beam splitters. And finally, we find 
that both the Michelson interferometer and the atomic clock have the same mathematical structure with 
the most famous double slit experiment: all of them can be summarized into the following procedure:
\begin{enumerate}
    \item One mode is split into the composition of two modes, and
    \item The two modes then propagate separately, and when the propagation ends, there is a relative error between them, and
    \item The two modes are again fused together. 
\end{enumerate}
The only differences are how these steps are implemented. \prettyref{fig:same-structure} shows the corresponding 
parts of three interferometers. The double split and the Michelson interferometer use two beams with the same 
frequency but different propagating path, so we have a relative phase factor of the two beams, while the atomic
clock uses two beams with the same propagating time and speed but different frequencies.

\begin{figure}
    \centering
    

\tikzset{every picture/.style={line width=0.75pt}} %set default line width to 0.75pt        

\begin{tikzpicture}[x=0.75pt,y=0.75pt,yscale=-1,xscale=1]
%uncomment if require: \path (0,300); %set diagram left start at 0, and has height of 300

%Straight Lines [id:da98076427437259] 
\draw [line width=2.25]    (156.75,82.49) -- (156.75,95.52) ;
%Straight Lines [id:da22469100290166044] 
\draw [line width=2.25]    (156.75,102.52) -- (156.75,121.52) ;
%Straight Lines [id:da5189169972470085] 
\draw [line width=2.25]    (156.75,127.52) -- (156.75,143.52) ;
%Straight Lines [id:da21855229634401785] 
\draw    (126.75,111.49) -- (156.75,99.52) ;
%Straight Lines [id:da6793617153567355] 
\draw    (126.75,111.49) -- (156.75,125.52) ;
%Straight Lines [id:da2477311666347024] 
\draw    (156.75,99.52) -- (171.25,100.52) ;
%Straight Lines [id:da7940313041981606] 
\draw    (156.75,125.52) -- (171.25,124.52) ;

%Shape: Rectangle [id:dp011705092710941889] 
\draw   (158.43,156.12) -- (146.07,156.12) -- (146.07,168.47) -- (158.43,168.47) -- cycle ;
%Straight Lines [id:da46074387496410485] 
\draw    (158.43,168.47) -- (146.07,156.12) ;

%Straight Lines [id:da4764012139794458] 
\draw    (129.5,162.29) -- (152.25,162.29) ;
\draw [shift={(140.88,162.29)}, rotate = 180] [fill={rgb, 255:red, 0; green, 0; blue, 0 }  ][line width=0.08]  [draw opacity=0] (12,-3) -- (0,0) -- (12,3) -- cycle    ;
%Straight Lines [id:da9273573306665419] 
\draw    (152.25,162.29) -- (175,162.29) ;
\draw [shift={(177,162.29)}, rotate = 180] [fill={rgb, 255:red, 0; green, 0; blue, 0 }  ][line width=0.08]  [draw opacity=0] (12,-3) -- (0,0) -- (12,3) -- cycle    ;
%Straight Lines [id:da760746400021453] 
\draw    (152.25,162.29) -- (152.25,186.93) ;
\draw [shift={(152.25,188.93)}, rotate = 270] [fill={rgb, 255:red, 0; green, 0; blue, 0 }  ][line width=0.08]  [draw opacity=0] (12,-3) -- (0,0) -- (12,3) -- cycle    ;
%Straight Lines [id:da9210577925794821] 
\draw    (114.5,162.29) -- (129.5,162.29) ;
%Straight Lines [id:da5463796420184142] 
\draw    (175,162.29) -- (183.5,162.29) ;
%Straight Lines [id:da2244110523730023] 
\draw    (152.25,186.93) -- (152.25,192.93) ;

%Straight Lines [id:da28250514918500724] 
\draw    (114.25,235.01) -- (185.75,235.01) ;
\draw [shift={(150,235.01)}, rotate = 180] [fill={rgb, 255:red, 0; green, 0; blue, 0 }  ][line width=0.08]  [draw opacity=0] (12,-3) -- (0,0) -- (12,3) -- cycle    ;
%Straight Lines [id:da43397268753341933] 
\draw [color={rgb, 255:red, 248; green, 231; blue, 28 }  ,draw opacity=1 ][line width=2.25]    (150,201.02) .. controls (151.67,202.69) and (151.67,204.35) .. (150,206.02) .. controls (148.33,207.69) and (148.33,209.35) .. (150,211.02) .. controls (151.67,212.69) and (151.67,214.35) .. (150,216.02) -- (150,221.01) -- (150,229.01) ;
\draw [shift={(150,235.01)}, rotate = 270] [fill={rgb, 255:red, 248; green, 231; blue, 28 }  ,fill opacity=1 ][line width=0.08]  [draw opacity=0] (19.2,-4.8) -- (0,0) -- (19.2,4.8) -- cycle    ;
%Straight Lines [id:da9161765045264532] 
\draw    (288,86.99) -- (342,132.02) ;
%Straight Lines [id:da2910579879446922] 
\draw    (288,123.02) -- (342,139.02) ;
%Straight Lines [id:da8887259082613865] 
\draw  [dash pattern={on 0.84pt off 2.51pt}]  (288,86.99) -- (288,123.02) ;
%Straight Lines [id:da6622918527087773] 
\draw  [dash pattern={on 0.84pt off 2.51pt}]  (288,123.02) -- (305,102.02) ;

%Straight Lines [id:da4841631375499058] 
\draw    (295.71,168.51) -- (343.71,168.51) ;
%Straight Lines [id:da8827221987808762] 
\draw    (343.71,183.5) -- (343.71,153.52) ;
%Straight Lines [id:da11821094935363363] 
\draw    (343.71,177.6) -- (348,181.73) ;
%Straight Lines [id:da551644497787346] 
\draw    (343.71,172.35) -- (348,176.48) ;
%Straight Lines [id:da008261152264246396] 
\draw    (343.71,166.45) -- (348,170.57) ;
%Straight Lines [id:da6635750712698332] 
\draw    (343.71,161.2) -- (348,165.33) ;
%Straight Lines [id:da08797082187173366] 
\draw    (343.71,155.95) -- (348,160.08) ;

%Straight Lines [id:da5216390226457006] 
\draw    (295.71,168.51) -- (295.71,191.52) ;
%Straight Lines [id:da13243531034064415] 
\draw    (282,191.5) -- (310,191.5) ;
%Straight Lines [id:da9217566816427496] 
\draw    (287.51,191.5) -- (283.66,195.52) ;
%Straight Lines [id:da48517962832493833] 
\draw    (292.41,191.5) -- (288.56,195.52) ;
%Straight Lines [id:da45653258407332675] 
\draw    (297.93,191.5) -- (294.07,195.52) ;
%Straight Lines [id:da2559821947431993] 
\draw    (302.83,191.5) -- (298.97,195.52) ;
%Straight Lines [id:da09081651382535694] 
\draw    (307.73,191.5) -- (303.87,195.52) ;


%Straight Lines [id:da6552969893359823] 
\draw    (280.25,235.01) -- (351.75,235.01) ;
\draw [shift={(316,235.01)}, rotate = 180] [fill={rgb, 255:red, 0; green, 0; blue, 0 }  ][line width=0.08]  [draw opacity=0] (12,-3) -- (0,0) -- (12,3) -- cycle    ;
%Straight Lines [id:da9796758952688152] 
\draw    (509.5,94.5) -- (509.5,142.52) ;
%Straight Lines [id:da33245251694556677] 
\draw    (455.5,83.49) -- (509.5,128.51) ;
\draw [shift={(482.5,106)}, rotate = 219.82] [fill={rgb, 255:red, 0; green, 0; blue, 0 }  ][line width=0.08]  [draw opacity=0] (12,-3) -- (0,0) -- (12,3) -- cycle    ;
%Straight Lines [id:da41812655556272027] 
\draw    (455.5,112.51) -- (509.5,128.51) ;
\draw [shift={(482.5,120.51)}, rotate = 196.5] [fill={rgb, 255:red, 0; green, 0; blue, 0 }  ][line width=0.08]  [draw opacity=0] (12,-3) -- (0,0) -- (12,3) -- cycle    ;
%Shape: Rectangle [id:dp677963444786315] 
\draw  [fill={rgb, 255:red, 155; green, 155; blue, 155 }  ,fill opacity=1 ] (509.5,120.51) -- (504.5,120.51) -- (504.5,135.49) -- (509.5,135.49) -- cycle ;

%Shape: Rectangle [id:dp9432323141362153] 
\draw   (487.93,156.12) -- (475.58,156.12) -- (475.58,168.47) -- (487.93,168.47) -- cycle ;
%Straight Lines [id:da16978324180498539] 
\draw    (487.93,168.47) -- (475.58,156.12) ;

%Straight Lines [id:da02614389609097678] 
\draw    (461,162.29) -- (481.75,162.29) ;
\draw [shift={(459,162.29)}, rotate = 0] [fill={rgb, 255:red, 0; green, 0; blue, 0 }  ][line width=0.08]  [draw opacity=0] (12,-3) -- (0,0) -- (12,3) -- cycle    ;
%Straight Lines [id:da4321212645143093] 
\draw    (481.75,162.29) -- (497,162.29) ;
\draw [shift={(495,162.29)}, rotate = 0] [fill={rgb, 255:red, 0; green, 0; blue, 0 }  ][line width=0.08]  [draw opacity=0] (12,-3) -- (0,0) -- (12,3) -- cycle    ;
%Straight Lines [id:da4435677647775138] 
\draw    (481.75,162.29) -- (481.75,176.93) ;
\draw [shift={(481.75,174.93)}, rotate = 90] [fill={rgb, 255:red, 0; green, 0; blue, 0 }  ][line width=0.08]  [draw opacity=0] (12,-3) -- (0,0) -- (12,3) -- cycle    ;
%Straight Lines [id:da24206020072858903] 
\draw    (446,162.29) -- (461,162.29) ;
%Straight Lines [id:da489510320670278] 
\draw    (519,162.29) -- (497,162.29) ;
%Straight Lines [id:da8670628803389584] 
\draw    (481.75,186.93) -- (481.75,192.93) ;

%Straight Lines [id:da5841588211828297] 
\draw    (447.75,235.01) -- (519.25,235.01) ;
\draw [shift={(483.5,235.01)}, rotate = 180] [fill={rgb, 255:red, 0; green, 0; blue, 0 }  ][line width=0.08]  [draw opacity=0] (12,-3) -- (0,0) -- (12,3) -- cycle    ;
%Straight Lines [id:da4991930146409558] 
\draw [color={rgb, 255:red, 248; green, 231; blue, 28 }  ,draw opacity=1 ][line width=2.25]    (483.5,201.02) .. controls (485.17,202.69) and (485.17,204.35) .. (483.5,206.02) .. controls (481.83,207.69) and (481.83,209.35) .. (483.5,211.02) .. controls (485.17,212.69) and (485.17,214.35) .. (483.5,216.02) -- (483.5,221.01) -- (483.5,229.01) ;
\draw [shift={(483.5,235.01)}, rotate = 270] [fill={rgb, 255:red, 248; green, 231; blue, 28 }  ,fill opacity=1 ][line width=0.08]  [draw opacity=0] (19.2,-4.8) -- (0,0) -- (19.2,4.8) -- cycle    ;

% Text Node
\draw (22.5,104.51) node [anchor=north west][inner sep=0.75pt]   [align=left] {Double split};
% Text Node
\draw (29.5,166.02) node [anchor=north west][inner sep=0.75pt]   [align=left] {Michelson};
% Text Node
\draw (36.5,216.51) node [anchor=north west][inner sep=0.75pt]   [align=left] {Ramsey};


\end{tikzpicture}

    \caption{The common structure of a double slit experiment, a Michelson interferometer and an atomic clock}
    \label{fig:same-structure}
\end{figure}

Ramsey fringes can also be used to check whether two energy levels have exactly the same energy. 
If not, fringes will form. Historically, this was how Lamb found that the 2S$_{1/2}$ and the 2P$_{1/2}$
states of the hydrogen atom have different energies, though the Dirac equation predicts that they are the same.
The fact is now known as \concept{Lamb displacement} and is caused by the quantum fluctuation of the 
optical field.

\end{document}