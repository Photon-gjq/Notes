\documentclass[hyperref, a4paper]{article}

\usepackage{geometry}
\usepackage{titling}
\usepackage{titlesec}
% No longer needed, since we will use enumitem package
% \usepackage{paralist}
\usepackage{enumitem}
\usepackage{footnote}
\usepackage{enumerate}
\usepackage{amsmath, amssymb, amsthm}
\usepackage{mathtools}
\usepackage{bbm}
\usepackage{cite}
\usepackage{graphicx}
\usepackage{subfigure}
\usepackage{physics}
\usepackage{tensor}
\usepackage{siunitx}
\usepackage[version=4]{mhchem}
\usepackage{tikz}
\usepackage{xcolor}
\usepackage{listings}
\usepackage{autobreak}
\usepackage[ruled, vlined, linesnumbered]{algorithm2e}
\usepackage{nameref,zref-xr}
\zxrsetup{toltxlabel}
\zexternaldocument*[optics-]{../optics/optics}[optics.pdf]
\zexternaldocument*[solid-]{../solid/solid}[solid.pdf]
\zexternaldocument*[info-]{../information/quantum-circuit}[quantum-circuit.pdf]
\usepackage[colorlinks,unicode]{hyperref} % , linkcolor=black, anchorcolor=black, citecolor=black, urlcolor=black, filecolor=black
\usepackage{prettyref}

% Page style
\geometry{left=3.18cm,right=3.18cm,top=2.54cm,bottom=2.54cm}
\titlespacing{\paragraph}{0pt}{1pt}{10pt}[20pt]
\setlength{\droptitle}{-5em}
\preauthor{\vspace{-10pt}\begin{center}}
\postauthor{\par\end{center}}

% More compact lists 
\setlist[itemize]{
    itemindent=17pt, 
    leftmargin=1pt,
    listparindent=\parindent,
    parsep=0pt,
}

% Math operators
\DeclareMathOperator{\timeorder}{\mathcal{T}}
\DeclareMathOperator{\diag}{diag}
\DeclareMathOperator{\legpoly}{P}
\DeclareMathOperator{\primevalue}{P}
\DeclareMathOperator{\sgn}{sgn}
\newcommand*{\ii}{\mathrm{i}}
\newcommand*{\ee}{\mathrm{e}}
\newcommand*{\const}{\mathrm{const}}
\newcommand*{\suchthat}{\quad \text{s.t.} \quad}
\newcommand*{\argmin}{\arg\min}
\newcommand*{\argmax}{\arg\max}
\newcommand*{\normalorder}[1]{: #1 :}
\newcommand*{\pair}[1]{\langle #1 \rangle}
\newcommand*{\fd}[1]{\mathcal{D} #1}
\DeclareMathOperator{\bigO}{\mathcal{O}}

% TikZ setting
\usetikzlibrary{arrows,shapes,positioning}
\usetikzlibrary{arrows.meta}
\usetikzlibrary{decorations.markings}
\tikzstyle arrowstyle=[scale=1]
\tikzstyle directed=[postaction={decorate,decoration={markings,
    mark=at position .5 with {\arrow[arrowstyle]{stealth}}}}]
\tikzstyle ray=[directed, thick]
\tikzstyle dot=[anchor=base,fill,circle,inner sep=1pt]

% Algorithm setting
% Julia-style code
\SetKwIF{If}{ElseIf}{Else}{if}{}{elseif}{else}{end}
\SetKwFor{For}{for}{}{end}
\SetKwFor{While}{while}{}{end}
\SetKwProg{Function}{function}{}{end}
\SetArgSty{textnormal}

\newcommand*{\concept}[1]{{\textbf{#1}}}

% Embedded codes
\lstset{basicstyle=\ttfamily,
  showstringspaces=false,
  commentstyle=\color{gray},
  keywordstyle=\color{blue}
}

\newcommand{\opticsdoc}{\href{../optics/optics}{the optics note}}
\newcommand{\soliddoc}{\href{../solid/solid}{the solid state physics note}}
\newcommand{\infodoc}{\href{../information/quantum-circuit}{the quantum information note}}

\newrefformat{fig}{Figure~\ref{#1} on page~\pageref{#1}}
\newrefformat{sec}{Section~\ref{#1}}

\title{Quantum Optics by Prof. Saijun Wu}
\author{Jinyuan Wu}
\date{November 4, 2021}

\begin{document}

\maketitle

\section{The rotational wave approximation and the Bloch sphere}

\begin{figure}
    \centering
    

\tikzset{every picture/.style={line width=0.75pt}} %set default line width to 0.75pt        

\begin{tikzpicture}[x=0.75pt,y=0.75pt,yscale=-1,xscale=1]
%uncomment if require: \path (0,300); %set diagram left start at 0, and has height of 300

%Straight Lines [id:da18819191009761904] 
\draw    (359,113) -- (450,113) ;
%Straight Lines [id:da1376948784583416] 
\draw    (359,200) -- (450,200) ;
%Straight Lines [id:da21818736070050204] 
\draw [color={rgb, 255:red, 248; green, 231; blue, 28 }  ,draw opacity=1 ]   (414,146) -- (414,154) .. controls (415.67,155.67) and (415.67,157.33) .. (414,159) .. controls (412.33,160.67) and (412.33,162.33) .. (414,164) .. controls (415.67,165.67) and (415.67,167.33) .. (414,169) .. controls (412.33,170.67) and (412.33,172.33) .. (414,174) .. controls (415.67,175.67) and (415.67,177.33) .. (414,179) .. controls (412.33,180.67) and (412.33,182.33) .. (414,184) .. controls (415.67,185.67) and (415.67,187.33) .. (414,189) .. controls (412.33,190.67) and (412.33,192.33) .. (414,194) .. controls (415.67,195.67) and (415.67,197.33) .. (414,199) -- (414,200) -- (414,200) ;
\draw [shift={(414,144)}, rotate = 90] [fill={rgb, 255:red, 248; green, 231; blue, 28 }  ,fill opacity=1 ][line width=0.08]  [draw opacity=0] (12,-3) -- (0,0) -- (12,3) -- cycle    ;
%Straight Lines [id:da6990718420713145] 
\draw  [dash pattern={on 4.5pt off 4.5pt}]  (378,144) -- (450,144) ;
%Straight Lines [id:da6995865544982842] 
\draw    (392,115) -- (392,142) ;
\draw [shift={(392,144)}, rotate = 270] [fill={rgb, 255:red, 0; green, 0; blue, 0 }  ][line width=0.08]  [draw opacity=0] (12,-3) -- (0,0) -- (12,3) -- cycle    ;
\draw [shift={(392,113)}, rotate = 90] [fill={rgb, 255:red, 0; green, 0; blue, 0 }  ][line width=0.08]  [draw opacity=0] (12,-3) -- (0,0) -- (12,3) -- cycle    ;
%Straight Lines [id:da8678920245446373] 
\draw [color={rgb, 255:red, 74; green, 144; blue, 226 }  ,draw opacity=1 ]   (366,115) -- (366,200) ;
\draw [shift={(366,113)}, rotate = 90] [fill={rgb, 255:red, 74; green, 144; blue, 226 }  ,fill opacity=1 ][line width=0.08]  [draw opacity=0] (12,-3) -- (0,0) -- (12,3) -- cycle    ;
%Straight Lines [id:da46707020520155873] 
\draw    (147,113) -- (238,113) ;
%Straight Lines [id:da15216197072886306] 
\draw    (147,200) -- (238,200) ;
%Straight Lines [id:da7329038078465273] 
\draw [color={rgb, 255:red, 248; green, 231; blue, 28 }  ,draw opacity=1 ]   (202,85) -- (202,93) .. controls (203.67,94.67) and (203.67,96.33) .. (202,98) .. controls (200.33,99.67) and (200.33,101.33) .. (202,103) .. controls (203.67,104.67) and (203.67,106.33) .. (202,108) .. controls (200.33,109.67) and (200.33,111.33) .. (202,113) .. controls (203.67,114.67) and (203.67,116.33) .. (202,118) .. controls (200.33,119.67) and (200.33,121.33) .. (202,123) .. controls (203.67,124.67) and (203.67,126.33) .. (202,128) .. controls (200.33,129.67) and (200.33,131.33) .. (202,133) .. controls (203.67,134.67) and (203.67,136.33) .. (202,138) .. controls (200.33,139.67) and (200.33,141.33) .. (202,143) .. controls (203.67,144.67) and (203.67,146.33) .. (202,148) .. controls (200.33,149.67) and (200.33,151.33) .. (202,153) .. controls (203.67,154.67) and (203.67,156.33) .. (202,158) .. controls (200.33,159.67) and (200.33,161.33) .. (202,163) .. controls (203.67,164.67) and (203.67,166.33) .. (202,168) .. controls (200.33,169.67) and (200.33,171.33) .. (202,173) .. controls (203.67,174.67) and (203.67,176.33) .. (202,178) .. controls (200.33,179.67) and (200.33,181.33) .. (202,183) .. controls (203.67,184.67) and (203.67,186.33) .. (202,188) .. controls (200.33,189.67) and (200.33,191.33) .. (202,193) .. controls (203.67,194.67) and (203.67,196.33) .. (202,198) -- (202,200) -- (202,200) ;
\draw [shift={(202,83)}, rotate = 90] [fill={rgb, 255:red, 248; green, 231; blue, 28 }  ,fill opacity=1 ][line width=0.08]  [draw opacity=0] (12,-3) -- (0,0) -- (12,3) -- cycle    ;
%Straight Lines [id:da5384879208606175] 
\draw  [dash pattern={on 4.5pt off 4.5pt}]  (166,83) -- (238,83) ;
%Straight Lines [id:da9870657692472646] 
\draw    (180,85) -- (180,112) ;
\draw [shift={(180,114)}, rotate = 270] [fill={rgb, 255:red, 0; green, 0; blue, 0 }  ][line width=0.08]  [draw opacity=0] (12,-3) -- (0,0) -- (12,3) -- cycle    ;
\draw [shift={(180,83)}, rotate = 90] [fill={rgb, 255:red, 0; green, 0; blue, 0 }  ][line width=0.08]  [draw opacity=0] (12,-3) -- (0,0) -- (12,3) -- cycle    ;
%Straight Lines [id:da34754409527991026] 
\draw [color={rgb, 255:red, 74; green, 144; blue, 226 }  ,draw opacity=1 ]   (154,115) -- (154,200) ;
\draw [shift={(154,113)}, rotate = 90] [fill={rgb, 255:red, 74; green, 144; blue, 226 }  ,fill opacity=1 ][line width=0.08]  [draw opacity=0] (12,-3) -- (0,0) -- (12,3) -- cycle    ;

% Text Node
\draw (452,200) node [anchor=west] [inner sep=0.75pt]    {$\omega _{g}$};
% Text Node
\draw (452,113) node [anchor=west] [inner sep=0.75pt]    {$\omega _{e}$};
% Text Node
\draw (400,128.5) node [anchor=west] [inner sep=0.75pt]    {$|\Delta |$};
% Text Node
\draw (411,172) node [anchor=east] [inner sep=0.75pt]    {$\omega $};
% Text Node
\draw (364,156.5) node [anchor=east] [inner sep=0.75pt]    {$\Omega $};
% Text Node
\draw (240,200) node [anchor=west] [inner sep=0.75pt]    {$\omega _{g}$};
% Text Node
\draw (240,113) node [anchor=west] [inner sep=0.75pt]    {$\omega _{e}$};
% Text Node
\draw (150,97.5) node [anchor=west] [inner sep=0.75pt]    {$|\Delta |$};
% Text Node
\draw (199,172) node [anchor=east] [inner sep=0.75pt]    {$\omega $};
% Text Node
\draw (152,156.5) node [anchor=east] [inner sep=0.75pt]    {$\Omega $};
% Text Node
\draw (184,234) node [anchor=north west][inner sep=0.75pt]   [align=left] {(a)};
% Text Node
\draw (393,234) node [anchor=north west][inner sep=0.75pt]   [align=left] {(b)};


\end{tikzpicture}

    \caption{The energy diagram of a two-level system in an external optical field}
\end{figure}

Under the rotational wave approximation and the corresponding RWA transformation, the Hamiltonian of a two-level system is 
\begin{equation}
    H = \frac{\hbar}{2} \vb*{\Omega} \cdot \vb*{\sigma},
    \label{eq:rwa-hamiltonian}
\end{equation}
where the norm of the \concept{Rabi vector} $\vb*{\Omega} = (\Omega_r, \Omega_i , \Delta)$ is
\begin{equation}
    \abs*{\vb*{\Omega}} = \sqrt{\Omega^2 + \Delta^2},
\end{equation}
\begin{equation}
    \Omega = \frac{\vb*{E}_0 \cdot \vb*{d}_{eg}}{\hbar}
    \label{eq:rabi-freq}
\end{equation}
is the \concept{Rabi frequency} and $\Delta$ the \concept{detuning}.
The wave function is always in the form of 
\begin{equation}
    \ket*{\psi} = \cos \frac{\theta}{2} \ee^{\ii \varphi /2 } \ket*{g} + \sin \frac{\theta}{2} \ee^{- \ii \varphi / 2} \ket*{e}, 
\end{equation}
and the density matrix is 
\begin{equation}
    \rho = \frac{1}{2} (1 + \vb*{n} \cdot \vb*{\sigma}),
\end{equation}
where 
\begin{equation}
    \vb*{n} = (\sin \theta \cos \varphi, \sin \theta \sin \varphi, \cos \theta).
\end{equation}
It is natural to put $\vb*{n}$ on a sphere, which is known as \concept{Bloch sphere}.
The constructions are standard for a qubit and can be found in Section~\ref{info-sec:single-qubit} in \infodoc.
The equation of motion is 
\[
    \frac{\ii \hbar}{2} \vb*{n} \cdot \vb*{\sigma} = \ii \hbar \rho = \comm*{H}{\rho} = \comm*{\vb*{\Omega} \cdot \vb*{\sigma}}{\frac{1}{2} \vb*{n} \cdot \vb*{\sigma}} = \frac{1}{2} (\vb*{\Omega} \times \vb*{n}) \cdot \vb*{\sigma},
\]
and therefore we have 
\begin{equation}
    \dot{\vb*{n}} = \vb*{\Omega} \times \vb*{n}. 
    \label{eq:n-eom}
\end{equation}

We review some properties about the Bloch sphere.
First, an equation of motion of a c-number defined in terms of the wave function is likely to be an equation of motion of an operator, and this is indeed the case of \eqref{eq:n-eom}, where we may find $\vb*{n} = \expval{\vb*{\sigma}}$.
Also, it can be verified that 
\begin{equation}
    \abs*{\braket*{\psi_1}{\psi_2}}^2 = \trace(\rho_1 \rho_2) = \frac{1}{2} (1 + \vb*{n}_1 \cdot \vb*{n}_2),
\end{equation}
and therefore we find that two wave functions are the same or differ with only a phase factor if and only if they have the same Bloch vector, and they are orthogonal if and only if their Bloch vectors are exactly the opposite of each other.

Note that the direction of the Rabi vector is not determined.
Usually we use the convention
\begin{equation}
    \vb*{\Omega} = (\Omega, 0, \Delta), 
\end{equation} 
and the two Bloch vectors
\begin{equation}
    \vb*{n}_{\pm \Omega} = \pm (\sin \theta, 0, \cos \theta)
\end{equation}
represents two eigenstates of \eqref{eq:rwa-hamiltonian}, namely $\ket*{\tilde{g}}$ (with eigenvalue $- \hbar \abs*{\vb*{\Omega}} / 2$) and $\ket*{\tilde{e}}$ (with eigenvalue $\hbar \abs*{\vb*{\Omega}} / 2$), where 
\begin{equation}
    \theta = \arctan \frac{\Omega}{\Delta}.
\end{equation}

\section{Control a two-level system with external fields}

Note that since \eqref{eq:rabi-freq}, a two-level system's state can be controlled via changing the external field.

\begin{figure}
    \centering
    

\tikzset{every picture/.style={line width=0.75pt}} %set default line width to 0.75pt        

\begin{tikzpicture}[x=0.75pt,y=0.75pt,yscale=-1,xscale=1]
%uncomment if require: \path (0,300); %set diagram left start at 0, and has height of 300

%Straight Lines [id:da8125477960653353] 
\draw    (95,171) -- (214,171) ;
\draw [shift={(216,171)}, rotate = 180] [fill={rgb, 255:red, 0; green, 0; blue, 0 }  ][line width=0.08]  [draw opacity=0] (12,-3) -- (0,0) -- (12,3) -- cycle    ;
%Straight Lines [id:da5838070816593157] 
\draw    (95,171) -- (95,65.02) ;
\draw [shift={(95,63.02)}, rotate = 450] [fill={rgb, 255:red, 0; green, 0; blue, 0 }  ][line width=0.08]  [draw opacity=0] (12,-3) -- (0,0) -- (12,3) -- cycle    ;
%Straight Lines [id:da8598243285485796] 
\draw    (95,171) -- (188.2,125.89) ;
\draw [shift={(190,125.02)}, rotate = 514.1700000000001] [fill={rgb, 255:red, 0; green, 0; blue, 0 }  ][line width=0.08]  [draw opacity=0] (12,-3) -- (0,0) -- (12,3) -- cycle    ;
%Straight Lines [id:da43220335567629653] 
\draw    (95,171) -- (155.85,84.66) ;
\draw [shift={(157,83.02)}, rotate = 485.17] [color={rgb, 255:red, 0; green, 0; blue, 0 }  ][line width=0.75]    (10.93,-3.29) .. controls (6.95,-1.4) and (3.31,-0.3) .. (0,0) .. controls (3.31,0.3) and (6.95,1.4) .. (10.93,3.29)   ;
%Straight Lines [id:da7003435646275242] 
\draw    (95,106.02) -- (95,171) ;
\draw [shift={(95,104.02)}, rotate = 90] [color={rgb, 255:red, 0; green, 0; blue, 0 }  ][line width=0.75]    (10.93,-3.29) .. controls (6.95,-1.4) and (3.31,-0.3) .. (0,0) .. controls (3.31,0.3) and (6.95,1.4) .. (10.93,3.29)   ;
%Shape: Arc [id:dp7751197002401395] 
\draw  [draw opacity=0][dash pattern={on 0.84pt off 2.51pt}] (95,104.02) .. controls (108.77,94.84) and (129.93,97.81) .. (145.49,112.08) .. controls (155.82,121.57) and (161.4,133.95) .. (161.64,145.45) -- (121.82,137.87) -- cycle ; \draw  [color={rgb, 255:red, 208; green, 2; blue, 27 }  ,draw opacity=1 ][dash pattern={on 0.84pt off 2.51pt}] (95,104.02) .. controls (108.77,94.84) and (129.93,97.81) .. (145.49,112.08) .. controls (155.82,121.57) and (161.4,133.95) .. (161.64,145.45) ;
%Straight Lines [id:da9547032756267608] 
\draw    (159.77,146.16) -- (95,171) ;
\draw [shift={(161.64,145.45)}, rotate = 159.02] [color={rgb, 255:red, 0; green, 0; blue, 0 }  ][line width=0.75]    (10.93,-3.29) .. controls (6.95,-1.4) and (3.31,-0.3) .. (0,0) .. controls (3.31,0.3) and (6.95,1.4) .. (10.93,3.29)   ;
%Straight Lines [id:da44322203840446583] 
\draw [color={rgb, 255:red, 208; green, 2; blue, 27 }  ,draw opacity=1 ]   (160.64,140.45) -- (161.25,143.48) ;
\draw [shift={(161.64,145.45)}, rotate = 258.69] [fill={rgb, 255:red, 208; green, 2; blue, 27 }  ,fill opacity=1 ][line width=0.08]  [draw opacity=0] (12,-3) -- (0,0) -- (12,3) -- cycle    ;
%Shape: Ellipse [id:dp5189226557816553] 
\draw  [color={rgb, 255:red, 74; green, 144; blue, 226 }  ,draw opacity=1 ][dash pattern={on 0.84pt off 2.51pt}] (29,145.45) .. controls (29,138.81) and (58.69,133.43) .. (95.32,133.43) .. controls (131.95,133.43) and (161.64,138.81) .. (161.64,145.45) .. controls (161.64,152.08) and (131.95,157.46) .. (95.32,157.46) .. controls (58.69,157.46) and (29,152.08) .. (29,145.45) -- cycle ;
%Straight Lines [id:da5564344266734615] 
\draw  [dash pattern={on 4.5pt off 4.5pt}]  (30.87,146.17) -- (95,171) ;
\draw [shift={(29,145.45)}, rotate = 21.17] [color={rgb, 255:red, 0; green, 0; blue, 0 }  ][line width=0.75]    (10.93,-3.29) .. controls (6.95,-1.4) and (3.31,-0.3) .. (0,0) .. controls (3.31,0.3) and (6.95,1.4) .. (10.93,3.29)   ;

% Text Node
\draw (218,171) node [anchor=west] [inner sep=0.75pt]    {$x$};
% Text Node
\draw (95,59.62) node [anchor=south] [inner sep=0.75pt]    {$z\left(\boldsymbol{\Omega } \ \text{with} \ \boldsymbol{E} =0\right)$};
% Text Node
\draw (192,121.62) node [anchor=south west] [inner sep=0.75pt]    {$y$};
% Text Node
\draw (150,64.4) node [anchor=north west][inner sep=0.75pt]    {$\boldsymbol{\Omega } \ \text{with external field}$};
% Text Node
\draw (27,95.4) node [anchor=north west][inner sep=0.75pt]    {$\boldsymbol{n}( t=0)$};
% Text Node
\draw (164,138.4) node [anchor=north west][inner sep=0.75pt]    {$\boldsymbol{n}( t=T/2)$};


\end{tikzpicture}

    \caption{The consequence of a $\pi$-pulse illustrated with the Bloch sphere}
    \label{fig:pi-pulse-result}
\end{figure}

Consider we start from $\ket*{g}$, i.e. $\vb*{n} = \vb*{e}_z$.
We want to excite the atom to $\ket*{e}$, i.e. $\vb*{n} = - \vb*{e}_z$.
When there is no external field $\vb*{\Omega}$ is along the $z$ axis, and therefore $\vb*{n}$ stays at $\vb*{e}_z$. 
Now suppose we add an electric field $\vb*{E}_0$, the Bloch vector starts to rotate around the new Rabi vector (depicted as the red curve in \prettyref{fig:pi-pulse-result}).
If, after $t = T/2 = \pi / \abs*{\vb*{\Omega}}$, we remove the external field, the Bloch vector will again rotate around the $z$ axis (depicted as the blue curve in \prettyref{fig:pi-pulse-result}).
Therefore we successfully change the $\theta$ angle of $\vb*{n}$.
This process is called a \concept{$\pi$-pulse}.
With a series of $\pi$-pulse, we can make $\vb*{n}$ arbitrarily close to $- \vb*{e}_z$.

This is quite an interesting result, because it shows that the atom can be excited even with the presence of detuning.
This can also be seen as an example of the $E$-$t$ uncertainty principle, because what we are actually do is to use \emph{pulses} to create some uncertainty in the energy to make up for the gap between $\omega$ and $\omega_{eg}$.

\section{Adiabatic states}

When $\abs*{\vb*{\Omega}} \gg \abs*{\mel**{\tilde{e}}{\partial_t}{\tilde{g}}}$, or in other words the perturbation mix eigenstates very slowly, we can make the adiabatic approximation and we have 

\section{Perturbative solution}

Now we go back to 
\begin{equation}
    \ii \dot{c}_e = \Delta c_e + \frac{\Omega}{2} c_g, \quad \ii \dot{c}_g = \frac{\Omega^*}{2} c_e.
\end{equation}
In the first order perturbation, we can just ignore $c_e$ in the second equation if we start from the ground state.
We, therefore, have 
\begin{equation}
    c^{(1)}_e(T) = - \ii \int_0^T \dd{t} \frac{\Omega}{2} \ee^{\ii \Delta t} \dd{t} = - \frac{\ii}{2} \bar{\Omega}(\omega)|_{\omega = \Delta},
\end{equation}
where 
\begin{equation}
    \bar{\Omega}(\omega) = \int_{-\infty}^\infty \bar{\Omega}(t) \ee^{\ii \omega t} \dd{t}, 
\end{equation}
and 
\begin{equation}
    \bar{\Omega}(t) = \begin{cases}
        \Omega(t) , &\quad 0 < t < T, \\
        0         , &\quad \text{otherwise}.
    \end{cases}
\end{equation}
We see that the Fourier transformation of the windowed Rabi frequency is an approximation of how strongly the two-level system will be disturbed.

\section{Atomic state detection}

Detecting what quantum state is an atom on has similar mathematical structure with single photon interferometers.
Suppose we have a two-level system, of which we can measure directly $\sigma^z$, and we want to estimate, say, the $\theta$ angle.
We have
\begin{equation}
    \expval{\sigma^z} = \cos^2 \frac{\theta}{2} - \sin^2 \frac{\theta}{2} = \cos \theta,
\end{equation}
and the standard variation is 
\begin{equation}
    \Delta \sigma^z = \sqrt{\expval{(\sigma^z)^2} - \expval{\sigma^z}^2} = \sqrt{1 - \cos^2 \theta} = \sin \theta.
\end{equation}
The uncertainty of $\theta$ is therefore 
\begin{equation}
    \Delta\theta = \frac{\Delta \sigma^z}{\pdv*{\expval{\sigma^z}}{\theta}} = 1.
\end{equation}
Therefore the measurement of $\theta$ with a single atom is extremely imprecise.
If we can generate lots of atoms in the same state, we can measure the sum of $\sigma^z$ of them, and we have 
\begin{equation}
    \expval{\Sigma^z} = N \cos \theta.
\end{equation}
On the other hand, 
\begin{eqnarray}
    \Delta \Sigma^z = \sqrt{N} \sin \theta,
\end{eqnarray}
and we have 
\begin{equation}
    \Delta \theta = \frac{1}{\sqrt{N}},
\end{equation}
which is the standard quantum error.

\section{Atomic clock}

\end{document}