\documentclass[hyperref, a4paper]{article}

\usepackage{geometry}
\usepackage{titling}
\usepackage{titlesec}
% No longer needed, since we will use enumitem package
% \usepackage{paralist}
\usepackage{enumitem}
\usepackage{footnote}
\usepackage{enumerate}
\usepackage{amsmath, amssymb, amsthm}
\usepackage{mathtools}
\usepackage{bbm}
\usepackage{cite}
\usepackage{graphicx}
\usepackage{subfigure}
\usepackage{physics}
\usepackage{tensor}
\usepackage{siunitx}
\usepackage[version=4]{mhchem}
\usepackage{tikz}
\usepackage{xcolor}
\usepackage{listings}
\usepackage{autobreak}
\usepackage[ruled, vlined, linesnumbered]{algorithm2e}
\usepackage{nameref,zref-xr}
\zxrsetup{toltxlabel}
\zexternaldocument*[optics-]{../optics/optics}[optics.pdf]
\zexternaldocument*[solid-]{../solid/solid}[solid.pdf]
\zexternaldocument*[info-]{../information/quantum-circuit}[quantum-circuit.pdf]
\usepackage[colorlinks,unicode]{hyperref} % , linkcolor=black, anchorcolor=black, citecolor=black, urlcolor=black, filecolor=black
\usepackage{prettyref}

% Page style
\geometry{left=3.18cm,right=3.18cm,top=2.54cm,bottom=2.54cm}
\titlespacing{\paragraph}{0pt}{1pt}{10pt}[20pt]
\setlength{\droptitle}{-5em}
\preauthor{\vspace{-10pt}\begin{center}}
\postauthor{\par\end{center}}

% More compact lists 
\setlist[itemize]{
    itemindent=17pt, 
    leftmargin=1pt,
    listparindent=\parindent,
    parsep=0pt,
}

% Math operators
\DeclareMathOperator{\timeorder}{\mathcal{T}}
\DeclareMathOperator{\diag}{diag}
\DeclareMathOperator{\legpoly}{P}
\DeclareMathOperator{\primevalue}{P}
\DeclareMathOperator{\sgn}{sgn}
\newcommand*{\ii}{\mathrm{i}}
\newcommand*{\ee}{\mathrm{e}}
\newcommand*{\const}{\mathrm{const}}
\newcommand*{\suchthat}{\quad \text{s.t.} \quad}
\newcommand*{\argmin}{\arg\min}
\newcommand*{\argmax}{\arg\max}
\newcommand*{\normalorder}[1]{: #1 :}
\newcommand*{\pair}[1]{\langle #1 \rangle}
\newcommand*{\fd}[1]{\mathcal{D} #1}
\DeclareMathOperator{\bigO}{\mathcal{O}}

% TikZ setting
\usetikzlibrary{arrows,shapes,positioning}
\usetikzlibrary{arrows.meta}
\usetikzlibrary{decorations.markings}
\tikzstyle arrowstyle=[scale=1]
\tikzstyle directed=[postaction={decorate,decoration={markings,
    mark=at position .5 with {\arrow[arrowstyle]{stealth}}}}]
\tikzstyle ray=[directed, thick]
\tikzstyle dot=[anchor=base,fill,circle,inner sep=1pt]

% Algorithm setting
% Julia-style code
\SetKwIF{If}{ElseIf}{Else}{if}{}{elseif}{else}{end}
\SetKwFor{For}{for}{}{end}
\SetKwFor{While}{while}{}{end}
\SetKwProg{Function}{function}{}{end}
\SetArgSty{textnormal}

\newcommand*{\concept}[1]{{\textbf{#1}}}

% Embedded codes
\lstset{basicstyle=\ttfamily,
  showstringspaces=false,
  commentstyle=\color{gray},
  keywordstyle=\color{blue}
}

\newcommand{\opticsdoc}{\href{../optics/optics}{the optics note}}
\newcommand{\soliddoc}{\href{../solid/solid}{the solid state physics note}}

\newrefformat{fig}{Figure~\ref{#1} on page~\pageref{#1}}
\newrefformat{sec}{Section~\ref{#1}}

\title{Quantum Optics by Prof. Kun Ding}
\author{Jinyuan Wu}
\date{October 28, 2021}

\begin{document}

\maketitle

\section{Review of linear quantum optics}

\begin{figure}
    \centering
    

\tikzset{every picture/.style={line width=0.75pt}} %set default line width to 0.75pt        

\begin{tikzpicture}[x=0.75pt,y=0.75pt,yscale=-1,xscale=1]
%uncomment if require: \path (0,300); %set diagram left start at 0, and has height of 300

%Shape: Square [id:dp6821783781180006] 
\draw   (210,155) -- (184,155) -- (184,181) -- (210,181) -- cycle ;
%Straight Lines [id:da8541061473329348] 
\draw    (210,181) -- (184,155) ;

%Straight Lines [id:da438666844140605] 
\draw    (107,168) -- (197,168) ;
\draw [shift={(152,168)}, rotate = 180] [fill={rgb, 255:red, 0; green, 0; blue, 0 }  ][line width=0.08]  [draw opacity=0] (12,-3) -- (0,0) -- (12,3) -- cycle    ;
%Straight Lines [id:da9355210781699652] 
\draw    (197,168) -- (287,168) ;
\draw [shift={(242,168)}, rotate = 180] [fill={rgb, 255:red, 0; green, 0; blue, 0 }  ][line width=0.08]  [draw opacity=0] (12,-3) -- (0,0) -- (12,3) -- cycle    ;
%Straight Lines [id:da872963247433993] 
\draw    (197,92.98) -- (197,168) ;
\draw [shift={(197,130.49)}, rotate = 270] [fill={rgb, 255:red, 0; green, 0; blue, 0 }  ][line width=0.08]  [draw opacity=0] (12,-3) -- (0,0) -- (12,3) -- cycle    ;
%Straight Lines [id:da6126074128891439] 
\draw    (197,168) -- (197,243.02) ;
\draw [shift={(197,205.51)}, rotate = 270] [fill={rgb, 255:red, 0; green, 0; blue, 0 }  ][line width=0.08]  [draw opacity=0] (12,-3) -- (0,0) -- (12,3) -- cycle    ;

% Text Node
\draw (110,142.4) node [anchor=north west][inner sep=0.75pt]    {$\ket{\alpha }$};
% Text Node
\draw (197,246.42) node [anchor=north] [inner sep=0.75pt]    {$1$};
% Text Node
\draw (289,168) node [anchor=west] [inner sep=0.75pt]    {$2$};
% Text Node
\draw (197,89.58) node [anchor=south] [inner sep=0.75pt]    {$\ket{0}$};


\end{tikzpicture}

    \caption{Vacuum displacement after going through a beam splitter}
    \label{fig:vacuum-displacement}
\end{figure}

Consider an optical circuit in \prettyref{fig:vacuum-displacement}. The linear transformation matrix of the beam splitter is 
\begin{equation}
    S = \pmqty{ \cos \frac{\varphi}{2} & \sin \frac{\varphi}{2} \\ - \sin \frac{\varphi}{2} & \cos \frac{\varphi}{2} }, \quad \varphi \ll 1.
\end{equation}
It can be verified that the output port 

The idea of \emph{dark port} can be generalized to many interferometers.

Nonlinear interferometers can achieve much better precision. Consider a most simple optical circuit, which is just a beam splitter with joint measurement at the output ports. 
The output state is 
\begin{equation}
    \begin{aligned}
        \ket*{\psi} &= a^\dagger_1 a^\dagger_2 \ket*{0} \\
        &= \Big( \cos\frac{\varphi}{2} b_1^\dagger + \sin\frac{\varphi}{2} b_2^\dagger \Big) \Big( \cos\frac{\varphi}{2} b_2^\dagger - \sin\frac{\varphi}{2} b_1^\dagger \Big) \ket*{0} \\
        &= \cos \varphi b_1^\dagger b_2^\dagger + \frac{1}{2} \sin \varphi ((b_1^\dagger)^2 - (b_2^\dagger)^2) \ket*{0}. 
    \end{aligned}
\end{equation}
This is a generalization of the Hong-Ou-Mandel effect, which occurs when $\varphi = \pi / 2$.
The two-point correlation function is 
\begin{equation}
    P^{(2)}(1, 2) = \expval*{a_1^\dagger a_2^\dagger a_2 a_1}{\psi} = \cos^2 \varphi,    
\end{equation}
and the standard variance is 
\begin{equation}
    \Delta P^{(2)}(1, 2) = \sqrt{ \expval*{:(n_1 n_2)^2:}{\psi}  - P^{(2)}(1, 2)^2} = \frac{1}{2} \abs*{\sin 2 \varphi}.
\end{equation}
It can be seen that nonlinear interferometers can achieve better precision.
Nonetheless, the usage of nonlinear interferometers is still largely limited because currently, 
no detector is reliable enough to perform good joint measurement.

\concept{Quantum imaging} is a possible approach to achieve super-resolution. 
Large biological molecules, like DNAs, cannot be seen clearly by optical microscopes because of the Abbe limit of resolution.
When the wave length of the detecting radiation is small enough to see them clearly, 
they are likely to be destroyed devitalized by the radiation.
Quantum approaches can be used 
We construct the so-called \concept{noon state}
\begin{equation}
    \ket*{N 00 N} = \frac{1}{\sqrt{2}} (\ket*{N, 0} + \ket*{0, N}).
\end{equation}
% TODO

Consider a pulse 
\begin{equation}
    \ket*{\psi} = D(\alpha) \ket*{0}
\end{equation}
that last for roughly $T$. It can also be considered as a 
\begin{equation}
    \expval*{n} = 
\end{equation}

\section{Semiclassical light-atom interaction}

The Hamiltonian of an electron in a (non-relativistic) atom is 
\begin{equation}
    H_\text{atom} = \sum_{i} \frac{\vb*{p}_i^2}{2m} - \sum_{i} \frac{1}{4 \pi \epsilon_0} \frac{Z e^2}{r_i} + \sum_{i \neq j} \frac{1}{4 \pi \epsilon_0} \frac{e^2}{\abs*{\vb*{r}_i - \vb*{r}_j}}.
\end{equation}
The dipole interaction Hamiltonian is 
\begin{equation}
    H_\text{dipole} = - \vb*{d} \cdot \vb*{E}, \quad \vb*{d} = - e \vb*{r}.
    \label{eq:dipole-interaction}
\end{equation}
This is usually the only term worth considering (see Section~\ref{optics-sec:dipole-radiation} in \opticsdoc{} for details).
In principle we can just use \eqref{eq:dipole-interaction} to couple an atom and the light field together.
In practice, such an approach is far beyond existing computational resources.

In this section we discuss the case where the light field is almost always in a coherent state, 
and therefore can be described by classical electrodynamics.
The approximation, therefore, is \emph{not} quantum optics strictly speaking.
It should be noted that semi-classical models are much easier to simulate and can reveal a lot of physics.
The inverse semi-classical approximation, i.e. treating the atom as classical, can be found in \href{../quantum-optics-homework/1}{here}.

We consider the case with only one atom, which is place at $\vb*{r} = 0$.
We label the energy eigenstates as $\{\ket*{n}\}$. The atom Hamiltonian is therefore 
\begin{equation}
    H_\text{atom} = \sum_n \hbar \omega_n \dyad{n},
\end{equation}
and the dipole operator is 
\begin{equation}
    \vb*{d} = \sum_{m, n} \underbrace{\mel**{m}{\vb*{d}}{n}}_{\vb*{d}_{mn}} \dyad{m}{n}
\end{equation}
For the sake of convenience we consider $H_\text{atom}$ as the ``free'' Hamiltonian and switch to the interaction picture, and thus the dipole operator is%
\footnote{
    To switch to the interaction picture we replace all operators in the Hamiltonian with their version in the interaction picture.
    Note that if $\vb*{d}$ evolves in accordance with $H_\text{atom}$, we must replace $\ket*{m}$ with $\ket*{m} \ee^{\ii \omega_m t}$ instead of $\ket*{m} \ee^{-\ii \omega_m t}$.
}%
\begin{equation}
    \vb*{d} = \sum_{m, n} \vb*{d}_{mn} \ee^{\ii \omega_{mn} t} \dyad{m}{n}, \quad \omega_{mn} = \omega_m - \omega_n,
\end{equation}
and the dipole Hamiltonian - now the interaction Hamiltonian - is 
\begin{equation}
    H_\text{I} = - (\vb*{E}_0 \ee^{- \ii \omega t} + \text{h.c.}) \cdot \sum_{m, n} \vb*{d}_{mn} \ee^{\ii \omega_{mn} t} \dyad{m}{n}.
\end{equation}

A widely used model is the \concept{two-level system}, where we only keep two atomic states $\ket*{g}$ and $\ket*{e}$ and higher excited states are integrated out.
In a two-level system we have 
\begin{equation}
    H_\text{I} = - \vb*{E}_0 \cdot \vb*{d}_{eg} \ee^{- \ii (\omega - \omega_{eg}) t} \dyad{e}{g} - \vb*{E}_0^* \cdot \vb*{d}^*_{{eg}} \ee^{\ii (\omega + \omega_{eg})} \dyad{e}{g} + \text{h.c.}.
\end{equation}
We usually denote 
\begin{equation}
    \Omega = \frac{\vb*{E}_0 \cdot \vb*{d}_{eg}}{\hbar}
\end{equation}
as the \concept{Rabi frequency}.
Suppose the initial state is 
\begin{equation}
    \ket*{\psi(0)} = c_{g} \ket*{g} + c_{e} \ket*{e}.
\end{equation}
The first order perturbation is 
\begin{equation}
    c^{(1)}_e(g) = - \ii \int_0^t \dd{\tau} (\Omega \ee^{- \ii (\omega - \omega_{eg}) t} + \Omega^* \ee^{\ii (\omega + \omega_{eg}) t}).
\end{equation}
We immediately notice that the $\ee^{\ii (\omega + \omega_{eg}) t}$ term oscillates much faster and therefore after the integration has a $1 / (\omega + \omega_{eg})$ factor.
Therefore, it can be expected that in many occasions this term can be omitted, which is called the \concept{rotational wave approximation}.
The Hamiltonian corresponding to the rotational wave approximation is 
\begin{equation}
    H_\text{RWA} = - \hbar \Omega \ee^{- \ii \Delta t} \dyad{e}{g} + \text{h.c.}, \quad \Delta = \omega - \omega_{eg}.
    \label{eq:rwa-original}
\end{equation}
We usually call $\Delta$ the \concept{detuning}.

It can then be noticed that \eqref{eq:rwa-original}, undergoing another picture transformation, is equivalent to 
\begin{equation}
    H = - \hbar \Delta \dyad{e} + \hbar \Omega \dyad{e}{g} + \text{h.c.}.
\end{equation}
The transformation is called the \concept{rotational wave transformation}.

Now since the Hamiltonian is much more simpler we can then study it exactly instead of using perturbation theory, 
which is not that reliable with a strong external field or with a long time duration.
The Schrodinger equation reads in terms of $c_e$ and $c_g$
\begin{equation}
    \ii \dot{c}_g = \frac{1}{2} \Omega^* c_e, \quad \ii \dot{c}_e = \Delta c_e + \frac{1}{2} \Omega c_g.
\end{equation}
It can be seen that the excited component in the atomic wave function first goes up and then goes down, oscillating periodically,
which is called the \concept{Rabi oscillation}. 
Classically speaking, the atom undergoes first absorption then stimulated emission.

A two-level system is essentially a qubit. The Hamiltonian can be rephrase into 
\begin{equation}
    H = \frac{\hbar \Delta}{2} \sigma^z + \frac{\hbar \Omega_r}{2} \sigma^x + \frac{\hbar \Omega_i}{2} \sigma^y, 
\end{equation}
and if we define 
\begin{equation}
    \vb*{\Omega} = (\Omega_r, \Omega_i, \Delta),
\end{equation}
the vector representing the atomic wave function on the Bloch sphere satisfies 
\begin{equation}
    \dot{\vb*{n}} = \vb*{\Omega} \times \vb*{n}.
\end{equation}

\end{document}