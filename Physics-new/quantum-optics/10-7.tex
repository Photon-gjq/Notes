\documentclass[hyperref, a4paper]{article}

\usepackage{geometry}
\usepackage{titling}
\usepackage{titlesec}
% No longer needed, since we will use enumitem package
% \usepackage{paralist}
\usepackage{enumitem}
\usepackage{footnote}
\usepackage{enumerate}
\usepackage{amsmath, amssymb, amsthm}
\usepackage{mathtools}
\usepackage{bbm}
\usepackage{cite}
\usepackage{graphicx}
\usepackage{subfigure}
\usepackage{physics}
\usepackage{tensor}
\usepackage{siunitx}
\usepackage[version=4]{mhchem}
\usepackage{tikz}
\usepackage{xcolor}
\usepackage{listings}
\usepackage{autobreak}
\usepackage[ruled, vlined, linesnumbered]{algorithm2e}
\usepackage{nameref,zref-xr}
\zxrsetup{toltxlabel}
\zexternaldocument*[optics-]{../optics/optics}[optics.pdf]
\zexternaldocument*[solid-]{../solid/solid}[solid.pdf]
\usepackage[colorlinks,unicode]{hyperref} % , linkcolor=black, anchorcolor=black, citecolor=black, urlcolor=black, filecolor=black
\usepackage{prettyref}

% Page style
\geometry{left=3.18cm,right=3.18cm,top=2.54cm,bottom=2.54cm}
\titlespacing{\paragraph}{0pt}{1pt}{10pt}[20pt]
\setlength{\droptitle}{-5em}
\preauthor{\vspace{-10pt}\begin{center}}
\postauthor{\par\end{center}}

% More compact lists 
\setlist[itemize]{
    itemindent=17pt, 
    leftmargin=1pt,
    listparindent=\parindent,
    parsep=0pt,
}

% Math operators
\DeclareMathOperator{\timeorder}{\mathcal{T}}
\DeclareMathOperator{\diag}{diag}
\DeclareMathOperator{\legpoly}{P}
\DeclareMathOperator{\primevalue}{P}
\DeclareMathOperator{\sgn}{sgn}
\newcommand*{\ii}{\mathrm{i}}
\newcommand*{\ee}{\mathrm{e}}
\newcommand*{\const}{\mathrm{const}}
\newcommand*{\suchthat}{\quad \text{s.t.} \quad}
\newcommand*{\argmin}{\arg\min}
\newcommand*{\argmax}{\arg\max}
\newcommand*{\normalorder}[1]{: #1 :}
\newcommand*{\pair}[1]{\langle #1 \rangle}
\newcommand*{\fd}[1]{\mathcal{D} #1}
\DeclareMathOperator{\bigO}{\mathcal{O}}

% TikZ setting
\usetikzlibrary{arrows,shapes,positioning}
\usetikzlibrary{arrows.meta}
\usetikzlibrary{decorations.markings}
\tikzstyle arrowstyle=[scale=1]
\tikzstyle directed=[postaction={decorate,decoration={markings,
    mark=at position .5 with {\arrow[arrowstyle]{stealth}}}}]
\tikzstyle ray=[directed, thick]
\tikzstyle dot=[anchor=base,fill,circle,inner sep=1pt]

% Algorithm setting
% Julia-style code
\SetKwIF{If}{ElseIf}{Else}{if}{}{elseif}{else}{end}
\SetKwFor{For}{for}{}{end}
\SetKwFor{While}{while}{}{end}
\SetKwProg{Function}{function}{}{end}
\SetArgSty{textnormal}

\newcommand*{\concept}[1]{{\textbf{#1}}}

% Embedded codes
\lstset{basicstyle=\ttfamily,
  showstringspaces=false,
  commentstyle=\color{gray},
  keywordstyle=\color{blue}
}

\newcommand{\opticsdoc}{\href{../optics/optics}{the optics note}}
\newcommand{\soliddoc}{\href{../solid/solid}{the solid state physics note}}

\newrefformat{fig}{Figure~\ref{#1} on page~\pageref{#1}}
\newrefformat{sec}{Section~\ref{#1}}

\title{Quantum Optics by Prof. Saijun Wu}
\author{Jinyuan Wu}
\date{October 7, 2021}

\begin{document}

\maketitle

\section{Measurement in quantum optics}

The only thing we are confident that we can measure from an optical field is the photon number at one point.
It is just like particle physics. In particle physics all we can measure is the scattering cross section, 
and most of our time will be spent on calculating $S$-matrices or amputated Feynman diagrams, 
while in quantum optics the only thing we can measure is the photon number, and most of our attention will be 
put on the correlation functions arising from the Fermi golden rule. 

The details of photodetectors re of course beyond this course, and here we just introduce a sketch of a single photon detection process.

What a photodetector does is amplification of the electrons excited by incident photons, 
and therefore despite the fact that we know nothing about the whole Hamiltonian of the detector, we do know the part of the Hamiltonian involving photons is $- \vb*{d} \cdot \vb*{E}^+$, where 
\begin{equation}
    \vb*{E}^+ = \sum_k \mathcal{E}_k \vb*{f}_k a_k \ee^{- \ii \omega_k t},
\end{equation}
because the detector annihilates an photon instead of creating one.
The probability that ``a photon is annihilated and the whole system's final state (the tensor product of the detector's state and the optical field's state) is not its initial state but rather the state $\ket*{f}$'' is 
\[
    P_f \propto \abs*{\mel**{f}{- \vb*{d} \cdot \vb*{E}^+}{i}}^2 \tau = \mel{i}{\vb*{d} \cdot \vb*{E}^-}{f} \mel**{f}{\vb*{d} \cdot \vb*{E}^+}{i} \tau,
\]
where $\tau$ is the detection time, or the time between we open the detector to let light come in and the detector temporarily stops to see whether something has happened.
Summing all final states up, and note the completeness condition
\[
    \sum_f \dyad{f} = \mathbbm{1}_\text{detector}, 
\]
we find 
\[
    P(\text{one photon detected}) \propto \tau \mel**{i}{(\vb*{d} \cdot \vb*{E}^-) (\vb*{d} \cdot \vb*{E}^+)}{i}.
\]
It can be immediately see that we can factor the expression into the multiplication of something about the detector and something about the optical field, and by doing so we finally get 
\begin{equation}
    P(\text{one photon detected at $\vb*{r}, t$}) = \eta \expval{\vb*{E}^-(\vb*{r}, t) \vb*{E}^+(\vb*{r}, t)}, \quad \eta \propto \tau d^2,
\end{equation}
where $\eta$ can be a tensor and the expectation is taken only for the optical field.

This measurement scheme can be easily generated, and we will find that 
\begin{equation}
    \begin{aligned}
        &\quad P(\text{photon detected at ($\vb*{r}_1, t_1), \ldots, (\vb*{r}_n$}, t_n)) \\
        &= \eta^n \expval{\vb*{E}^-(\vb*{r}_1, t_1) \cdots \vb*{E}^-(\vb*{r}_n, t_n) \vb*{E}^+(\vb*{r}_n, t_n) \cdots \vb*{E}^+(\vb*{r}_1, t_1)}.
    \end{aligned}
    \label{eq:joint-detection}
\end{equation}
Note that despite the detectors are independent to each other, since they are all immersed in the same optical field, we cannot just factor \eqref{eq:joint-detection} into several factors concerning each detector.

The correlation function in \eqref{eq:joint-detection} is in \emph{normal ordering}.
Despite being derived with an optical field in a pure state, it is easy to see \eqref{eq:joint-detection} can be simply generated into mixed states.
In a word, we only need to calculate \emph{time ordered correlation function} in quantum optics.

\section{Coherence}

It is natural, then, to ask how far \eqref{eq:joint-detection} deviates from the product of several independent probabilities regarding each detector, or in other words how \concept{coherent} the optical field is.

\end{document}