\documentclass[hyperref, a4paper]{article}

\usepackage{geometry}
\usepackage{titling}
\usepackage{titlesec}
% No longer needed, since we will use enumitem package
% \usepackage{paralist}
\usepackage{enumitem}
\usepackage{footnote}
\usepackage{enumerate}
\usepackage{amsmath, amssymb, amsthm}
\usepackage{mathtools}
\usepackage{bbm}
\usepackage{cite}
\usepackage{graphicx}
\usepackage{subfigure}
\usepackage{physics}
\usepackage{tensor}
\usepackage{siunitx}
\usepackage[version=4]{mhchem}
\usepackage{tikz}
\usepackage{xcolor}
\usepackage{listings}
\usepackage{autobreak}
\usepackage[ruled, vlined, linesnumbered]{algorithm2e}
\usepackage{nameref,zref-xr}
\zxrsetup{toltxlabel}
\zexternaldocument*[optics-]{../optics/optics}[optics.pdf]
\zexternaldocument*[solid-]{../solid/solid}[solid.pdf]
\usepackage[colorlinks,unicode]{hyperref} % , linkcolor=black, anchorcolor=black, citecolor=black, urlcolor=black, filecolor=black
\usepackage{prettyref}

% Page style
\geometry{left=3.18cm,right=3.18cm,top=2.54cm,bottom=2.54cm}
\titlespacing{\paragraph}{0pt}{1pt}{10pt}[20pt]
\setlength{\droptitle}{-5em}
\preauthor{\vspace{-10pt}\begin{center}}
\postauthor{\par\end{center}}

% More compact lists 
\setlist[itemize]{
    itemindent=17pt, 
    leftmargin=1pt,
    listparindent=\parindent,
    parsep=0pt,
}

% Math operators
\DeclareMathOperator{\timeorder}{\mathcal{T}}
\DeclareMathOperator{\diag}{diag}
\DeclareMathOperator{\legpoly}{P}
\DeclareMathOperator{\primevalue}{P}
\DeclareMathOperator{\sgn}{sgn}
\newcommand*{\ii}{\mathrm{i}}
\newcommand*{\ee}{\mathrm{e}}
\newcommand*{\const}{\mathrm{const}}
\newcommand*{\suchthat}{\quad \text{s.t.} \quad}
\newcommand*{\argmin}{\arg\min}
\newcommand*{\argmax}{\arg\max}
\newcommand*{\normalorder}[1]{: #1 :}
\newcommand*{\pair}[1]{\langle #1 \rangle}
\newcommand*{\fd}[1]{\mathcal{D} #1}
\DeclareMathOperator{\bigO}{\mathcal{O}}

% TikZ setting
\usetikzlibrary{arrows,shapes,positioning}
\usetikzlibrary{arrows.meta}
\usetikzlibrary{decorations.markings}
\tikzstyle arrowstyle=[scale=1]
\tikzstyle directed=[postaction={decorate,decoration={markings,
    mark=at position .5 with {\arrow[arrowstyle]{stealth}}}}]
\tikzstyle ray=[directed, thick]
\tikzstyle dot=[anchor=base,fill,circle,inner sep=1pt]

% Algorithm setting
% Julia-style code
\SetKwIF{If}{ElseIf}{Else}{if}{}{elseif}{else}{end}
\SetKwFor{For}{for}{}{end}
\SetKwFor{While}{while}{}{end}
\SetKwProg{Function}{function}{}{end}
\SetArgSty{textnormal}

\newcommand*{\concept}[1]{{\textbf{#1}}}

% Embedded codes
\lstset{basicstyle=\ttfamily,
  showstringspaces=false,
  commentstyle=\color{gray},
  keywordstyle=\color{blue}
}

\newcommand{\opticsdoc}{\href{../optics/optics}{the optics note}}
\newcommand{\soliddoc}{\href{../solid/solid}{the solid state physics note}}

\newrefformat{fig}{Figure~\ref{#1} on page~\pageref{#1}}
\newrefformat{sec}{Section~\ref{#1}}

\title{Quantum Optics by Prof. Saijun Wu}
\author{Jinyuan Wu}
\date{October 21, 2021}

\begin{document}

\maketitle

This article is a lecture note of Prof. Saijun Wu's lecture on Quantum Optics on October 21, 2021.

\section{Measurement with reference beams}

Suppose we have a linear optics system with a few parameters, 
and we want to measure them according to the measurement results of photon numbers
- which is almost the only observable that is really observable use present tools.
Let $\bar{n}_l$ be the average observed photon number at detector $l$. 
If we do enough times of measurement, we have 
\begin{equation}
    \expval{n_l} = \bar{n}_l.
\end{equation}
Let $P$ be the collection of parameters, and $\bar{n}_l$ is a function of $P$.
If we have an explicit expression $\bar{n}_l = \bar{n}_l(P)$, we must already have a computationally efficient model of the linear optics system, which is often hard to obtain.
Anyway, if there are enough detectors, we have 
\begin{equation}
    P = n^{-1}(\bar{n}),
\end{equation}
where $\bar{n}$ is the collection of photon number operators at all detectors.
The error can then be estimated using standard error propagation techniques.

\begin{figure}
    \centering
    

\tikzset{every picture/.style={line width=0.75pt}} %set default line width to 0.75pt        

\begin{tikzpicture}[x=0.75pt,y=0.75pt,yscale=-1,xscale=1]
%uncomment if require: \path (0,300); %set diagram left start at 0, and has height of 300

%Shape: Square [id:dp7292498374829528] 
\draw   (235,140) -- (261,140) -- (261,166) -- (235,166) -- cycle ;
%Straight Lines [id:da010139268745176677] 
\draw    (235,166) -- (261,140) ;

%Straight Lines [id:da5573883310885233] 
\draw    (168,153) -- (248,153) ;
\draw [shift={(208,153)}, rotate = 180] [fill={rgb, 255:red, 0; green, 0; blue, 0 }  ][line width=0.08]  [draw opacity=0] (12,-3) -- (0,0) -- (12,3) -- cycle    ;
%Straight Lines [id:da26712641506196255] 
\draw    (248,153) -- (353.71,153) ;
\draw [shift={(300.85,153)}, rotate = 180] [fill={rgb, 255:red, 0; green, 0; blue, 0 }  ][line width=0.08]  [draw opacity=0] (12,-3) -- (0,0) -- (12,3) -- cycle    ;
%Straight Lines [id:da6641395542938842] 
\draw    (248,153) -- (248,221.67) ;
\draw [shift={(248,187.33)}, rotate = 270] [fill={rgb, 255:red, 0; green, 0; blue, 0 }  ][line width=0.08]  [draw opacity=0] (12,-3) -- (0,0) -- (12,3) -- cycle    ;
%Straight Lines [id:da6178054140461928] 
\draw    (248,85.33) -- (248,153) ;
\draw [shift={(248,119.17)}, rotate = 270] [fill={rgb, 255:red, 0; green, 0; blue, 0 }  ][line width=0.08]  [draw opacity=0] (12,-3) -- (0,0) -- (12,3) -- cycle    ;
%Straight Lines [id:da6569709591508739] 
\draw    (248,221.67) -- (353.71,221.67) ;
%Shape: Chord [id:dp4222004696727377] 
\draw   (353.82,137.75) .. controls (363.01,137.77) and (370.52,144.37) .. (370.69,152.67) .. controls (370.85,161.09) and (363.38,168.06) .. (354,168.25) -- cycle ;
%Shape: Chord [id:dp5038597778456795] 
\draw   (353.82,206.42) .. controls (363.01,206.43) and (370.52,213.04) .. (370.69,221.33) .. controls (370.85,229.76) and (363.38,236.73) .. (354,236.91) -- cycle ;

% Text Node
\draw (375.71,153) node [anchor=west] [inner sep=0.75pt]    {$n_{1}$};
% Text Node
\draw (373.71,221) node [anchor=west] [inner sep=0.75pt]    {$n_{2}$};
% Text Node
\draw (253,74.4) node [anchor=north west][inner sep=0.75pt]    {$\ket{0}$};


\end{tikzpicture}

    \caption{Measuring the reflective coefficient of a beam splitter}
    \label{fig:measure-mean-splitter}
\end{figure}

Consider a device like \prettyref{fig:measure-mean-splitter}, where we use two detectors to measure the reflective coefficient of a beam splitter.
Suppose we give an input consisting a few pulses, and the wave function of the system, under Heisenberg picture, is
\begin{equation}
    \ket*{\psi} = \prod_{j=1}^N a_j^\dagger \ket*{0} = \prod_{j=1}^N \left( \cos\frac{\theta}{2} b_{j1}^\dagger + \sin\frac{\theta}{2} b_{j2}^\dagger \right) .
\end{equation}
We have 
\begin{equation}
    \expval*{n_1} = \sum_j \expval*{b_{j1}^\dagger b_{j1}} = N \cos^2\frac{\theta}{2}, \quad \expval*{n_2} = \sum_j \expval*{b_{j2}^\dagger b_{j2}} = N \sin^2\frac{\theta}{2}.
\end{equation}
In principle we know everything about the incident light; in practice we do not, so from the equations above we have 
\begin{equation}
    \tan^2 \frac{\theta}{2} = \frac{\expval*{n_1}}{\expval*{n_2}},
\end{equation}
where parameters that we may not know are not present.
We can consider this as a tricky to get rid of classical errors, like the (non-coherent and usually thermal) fluctuation of $N$.
We can then evaluate the standard error of the two quantities, which are 
\begin{equation}
    \var{n}_1 = \var{n_2} = \frac{1}{2} \sqrt{N} \sin \theta.
\end{equation}
Thus 
\begin{equation}
    \var{\theta} = \frac{\var{\bar{n}}_1}{\pdv*{\bar{n}_1}{\theta}} = \frac{1}{2 \sqrt{N}}.
\end{equation}

We can also use a coherent light input, and the wave function of the whole system, again in the Heisenberg picture, is 
\begin{equation}
    \ket*{\psi} = \prod_{i=1}^N \ee^{\alpha_i a^\dagger_i - \alpha_i^* a_i} \ket*{0},
\end{equation}
and we have 
\begin{equation}
    \expval*{n_1} = N \abs*{\alpha}^2 \cos^2 \frac{\theta}{2}, \quad \expval*{n_2} = N \abs*{\alpha}^2 \sin^2 \frac{\theta}{2},
\end{equation}
and again, we have 
\begin{equation}
    \tan^2 \frac{\theta}{2} \frac{\expval*{n_1}}{\expval*{n_2}}.
\end{equation}
The error of $\theta$ can be estimated as 
\begin{equation}
    \var{\theta} = \frac{\var{\bar{n}}_1}{\pdv*{\bar{n}_1}{\theta}} = \frac{1}{2 \sqrt{N} \abs*{\alpha} \cos \frac{\theta}{2}}.
\end{equation}

\begin{figure}
    \centering
    

\tikzset{every picture/.style={line width=0.75pt}} %set default line width to 0.75pt        

\begin{tikzpicture}[x=0.75pt,y=0.75pt,yscale=-1,xscale=1]
%uncomment if require: \path (0,300); %set diagram left start at 0, and has height of 300

%Shape: Square [id:dp19899658350277827] 
\draw   (158,128) -- (184,128) -- (184,154) -- (158,154) -- cycle ;
%Straight Lines [id:da974109657198428] 
\draw    (158,154) -- (184,128) ;

%Straight Lines [id:da6154002143329464] 
\draw    (91,141) -- (171,141) ;
\draw [shift={(131,141)}, rotate = 180] [fill={rgb, 255:red, 0; green, 0; blue, 0 }  ][line width=0.08]  [draw opacity=0] (12,-3) -- (0,0) -- (12,3) -- cycle    ;
%Straight Lines [id:da9149325559495276] 
\draw    (171,141) -- (276.71,141) ;
\draw [shift={(223.85,141)}, rotate = 180] [fill={rgb, 255:red, 0; green, 0; blue, 0 }  ][line width=0.08]  [draw opacity=0] (12,-3) -- (0,0) -- (12,3) -- cycle    ;
%Straight Lines [id:da8196940004792728] 
\draw    (171,141) -- (171,209.67) ;
\draw [shift={(171,175.33)}, rotate = 270] [fill={rgb, 255:red, 0; green, 0; blue, 0 }  ][line width=0.08]  [draw opacity=0] (12,-3) -- (0,0) -- (12,3) -- cycle    ;
%Straight Lines [id:da6775872216534589] 
\draw    (171,73.33) -- (171,141) ;
\draw [shift={(171,107.17)}, rotate = 270] [fill={rgb, 255:red, 0; green, 0; blue, 0 }  ][line width=0.08]  [draw opacity=0] (12,-3) -- (0,0) -- (12,3) -- cycle    ;
%Straight Lines [id:da16052607713424072] 
\draw    (171,209.67) -- (376.71,209.67) ;
%Shape: Chord [id:dp9631399451726803] 
\draw   (377.82,125.75) .. controls (387.01,125.77) and (394.52,132.37) .. (394.69,140.67) .. controls (394.85,149.09) and (387.38,156.06) .. (378,156.25) -- cycle ;
%Shape: Chord [id:dp47106547985409697] 
\draw   (376.82,194.42) .. controls (386.01,194.43) and (393.52,201.04) .. (393.69,209.33) .. controls (393.85,217.76) and (386.38,224.73) .. (377,224.91) -- cycle ;
%Shape: Rectangle [id:dp1632530807820043] 
\draw   (277.71,125.12) -- (347.71,125.12) -- (347.71,156) -- (277.71,156) -- cycle ;
%Straight Lines [id:da5075004941191303] 
\draw    (348,142) -- (377.71,142) ;
%Straight Lines [id:da28099279347268435] 
\draw  [dash pattern={on 4.5pt off 4.5pt}]  (314,36) -- (314,125.12) ;
\draw [shift={(314,80.56)}, rotate = 90] [fill={rgb, 255:red, 0; green, 0; blue, 0 }  ][line width=0.08]  [draw opacity=0] (12,-3) -- (0,0) -- (12,3) -- cycle    ;

% Text Node
\draw (399.71,141) node [anchor=west] [inner sep=0.75pt]    {$n_{1}$};
% Text Node
\draw (396.71,209) node [anchor=west] [inner sep=0.75pt]    {$n_{2}$};
% Text Node
\draw (176,62.4) node [anchor=north west][inner sep=0.75pt]    {$\ket{0}$};
% Text Node
\draw (312.71,140.56) node   [align=left] {material};
% Text Node
\draw (194,114.4) node [anchor=north west][inner sep=0.75pt]    {$b_{1}^{\dagger }$};
% Text Node
\draw (355,115.4) node [anchor=north west][inner sep=0.75pt]    {$c_{1}^{\dagger }$};
% Text Node
\draw (319,32.4) node [anchor=north west][inner sep=0.75pt]    {$c_{2}^{\dagger }$};
% Text Node
\draw (194,184.4) node [anchor=north west][inner sep=0.75pt]    {$b_{2}^{\dagger }$};


\end{tikzpicture}

    \caption{Measuring light absorption}
    \label{fig:absorption-measure}
\end{figure}

The idea in \prettyref{fig:measure-mean-splitter} can be generated to some more realistic measurements.
Consider we are analyzing light absorption in a certain kind of material, and we have an optical circuit depicted in \prettyref{fig:absorption-measure}.
To make things easier we use a 50-50 beam splitter.
Though the light absorption is a non-unitary process, we can view it as another beam splitter, where a part of the incident light is scattered to somewhere else.
We can therefore derive 
\begin{equation}
    \expval*{n}_1 = \frac{N \abs*{\alpha}^2}{2} \cos^2 \frac{\theta}{2}, \quad \expval*{n}_2 = \frac{1}{2} N \abs*{\alpha}^2.
\end{equation}
Since 
\begin{equation}
    I = I_0 \ee^{- \rho \sigma L} \eqqcolon \ee^{- \mathrm{OD}},
\end{equation}
we have 
\begin{equation}
    \sigma = - \frac{1}{\rho L} \ln \frac{\expval*{n_1}}{\expval*{n_2}}.
\end{equation}
It should be noted that large absorption cannot be measured in an easy way.
We have 
\[
    n = \frac{P \tau}{\hbar \omega},
\]
and we have 
\begin{equation}
    \expval*{c^\dagger_1 c_1} \simeq \frac{P \tau}{\hbar \omega} \ee^{- \text{OD}}.
\end{equation}
When the absorption rate is large, $\expval*{c_1^\dagger c_1}$ can be quite small compared to its error, and the measurement can be extremely unreliable.
That is why absorption experiments are usually carried out with dilute solutions.
But on the other hand, $\text{OD}$ must be large to make sure 
\begin{equation}
    \frac{\var{\sigma}}{\sigma} = \frac{\var{\text{OD}}}{\text{OD}}
\end{equation}
is small enough.

\begin{figure}
    \centering
    

\tikzset{every picture/.style={line width=0.75pt}} %set default line width to 0.75pt        

\begin{tikzpicture}[x=0.75pt,y=0.75pt,yscale=-1,xscale=1]
%uncomment if require: \path (0,300); %set diagram left start at 0, and has height of 300

%Shape: Square [id:dp7887778907073675] 
\draw   (131,81) -- (105,81) -- (105,107) -- (131,107) -- cycle ;
%Straight Lines [id:da2661929402456471] 
\draw    (131,107) -- (105,81) ;

%Straight Lines [id:da9763350854839723] 
\draw    (38,94) -- (118,94) ;
\draw [shift={(78,94)}, rotate = 180] [fill={rgb, 255:red, 0; green, 0; blue, 0 }  ][line width=0.08]  [draw opacity=0] (12,-3) -- (0,0) -- (12,3) -- cycle    ;
%Straight Lines [id:da15047983448119173] 
\draw    (118,94) -- (118,161.67) ;
\draw [shift={(118,127.83)}, rotate = 270] [fill={rgb, 255:red, 0; green, 0; blue, 0 }  ][line width=0.08]  [draw opacity=0] (12,-3) -- (0,0) -- (12,3) -- cycle    ;
%Straight Lines [id:da4201522121961012] 
\draw    (118,39.62) -- (118,94) ;
\draw [shift={(118,66.81)}, rotate = 270] [fill={rgb, 255:red, 0; green, 0; blue, 0 }  ][line width=0.08]  [draw opacity=0] (12,-3) -- (0,0) -- (12,3) -- cycle    ;
%Straight Lines [id:da8860781623476384] 
\draw    (118,94) -- (198,94) ;
\draw [shift={(158,94)}, rotate = 180] [fill={rgb, 255:red, 0; green, 0; blue, 0 }  ][line width=0.08]  [draw opacity=0] (12,-3) -- (0,0) -- (12,3) -- cycle    ;
%Shape: Square [id:dp17128277645325585] 
\draw   (362,150) -- (336,150) -- (336,176) -- (362,176) -- cycle ;
%Straight Lines [id:da06631129948630177] 
\draw    (362,176) -- (336,150) ;

%Shape: Rectangle [id:dp5724236602167219] 
\draw   (197.71,78.12) -- (267.71,78.12) -- (267.71,109) -- (197.71,109) -- cycle ;
%Straight Lines [id:da5485960535161039] 
\draw    (118,162) -- (349,162) ;
\draw [shift={(233.5,162)}, rotate = 180] [fill={rgb, 255:red, 0; green, 0; blue, 0 }  ][line width=0.08]  [draw opacity=0] (12,-3) -- (0,0) -- (12,3) -- cycle    ;
%Straight Lines [id:da05386170776958932] 
\draw    (268,94) -- (348,94) ;
\draw [shift={(308,94)}, rotate = 180] [fill={rgb, 255:red, 0; green, 0; blue, 0 }  ][line width=0.08]  [draw opacity=0] (12,-3) -- (0,0) -- (12,3) -- cycle    ;
%Straight Lines [id:da2677683405854743] 
\draw    (331.98,77.98) -- (364.02,110.02) ;
%Straight Lines [id:da6791525913111485] 
\draw    (101.98,145.65) -- (134.02,177.69) ;
%Straight Lines [id:da8755162093263016] 
\draw    (349,95.33) -- (349,163) ;
\draw [shift={(349,129.17)}, rotate = 270] [fill={rgb, 255:red, 0; green, 0; blue, 0 }  ][line width=0.08]  [draw opacity=0] (12,-3) -- (0,0) -- (12,3) -- cycle    ;
%Straight Lines [id:da8526400749967091] 
\draw    (349,162) -- (429,162) ;
\draw [shift={(389,162)}, rotate = 180] [fill={rgb, 255:red, 0; green, 0; blue, 0 }  ][line width=0.08]  [draw opacity=0] (12,-3) -- (0,0) -- (12,3) -- cycle    ;
%Straight Lines [id:da8734171157543595] 
\draw    (349,163) -- (349,226.73) ;
\draw [shift={(349,194.86)}, rotate = 270] [fill={rgb, 255:red, 0; green, 0; blue, 0 }  ][line width=0.08]  [draw opacity=0] (12,-3) -- (0,0) -- (12,3) -- cycle    ;

% Text Node
\draw (220,119.4) node [anchor=north west][inner sep=0.75pt]    {$S_{\text{F}}$};
% Text Node
\draw (86,59.4) node [anchor=north west][inner sep=0.75pt]    {$S_{1}$};
% Text Node
\draw (364,179.4) node [anchor=north west][inner sep=0.75pt]    {$S_{2}$};
% Text Node
\draw (20,82.4) node [anchor=north west][inner sep=0.75pt]    {$a_{1}^{\dagger }$};
% Text Node
\draw (111,16.4) node [anchor=north west][inner sep=0.75pt]    {$a_{2}^{\dagger }$};
% Text Node
\draw (436,149.4) node [anchor=north west][inner sep=0.75pt]    {$b_{2}^{\dagger }$};
% Text Node
\draw (344,235.4) node [anchor=north west][inner sep=0.75pt]    {$b_{1}^{\dagger }$};


\end{tikzpicture}

    \caption{A Mach–Zehnder interferometer used to measure absorption}
    \label{fig:mach-zehnder}
\end{figure}

\section{Measurement with light interferometry}

One way to eschew this problem is to measure the \emph{phase}. We construct a Mach-Zehnder interferometer shown in \prettyref{fig:mach-zehnder}.
The two beam splitters are 50-50 ones, i.e. their transition matrix being 
\begin{equation}
    S_1 = S_2 = \frac{1}{\sqrt{2}} \pmqty{ 1 & -1 \\ 1 & 1 }.
\end{equation}
The spacial propagating process can be described by 
\begin{equation}
    S_\text{F} = \pmqty{\dmat{\ee^{\ii k L_1 - \varphi_1}, \ee^{\ii k L_2 + \varphi_2}}} \eqqcolon \text{unitary const} \times \pmqty{\dmat{\ee^{\ii \varphi / 2}, \ee^{- \ii \varphi / 2}}}.
\end{equation}
Ignoring unimportant factors, we have 
\begin{equation}
    \begin{aligned}
        n_1 &= \cos^2\frac{\varphi}{2} a_1^\dagger a_1 + \sin^2\frac{\varphi}{2} a_2^\dagger a_2 + \frac{1}{2} \sin \varphi (a_1^\dagger a_2 + a^\dagger_2 a_1), \\
        n_2 &= \cos^2\frac{\varphi}{2} a_2^\dagger a_2 + \sin^2\frac{\varphi}{2} a_1^\dagger a_1 - \frac{1}{2} \sin \varphi (a_1^\dagger a_2 + a^\dagger_2 a_1).
    \end{aligned}
\end{equation}
By measuring $n_1$ and $n_2$, we are able to measure $\varphi$, and hence the absorption.

\begin{figure}
    \centering
    

\tikzset{every picture/.style={line width=0.75pt}} %set default line width to 0.75pt        

\begin{tikzpicture}[x=0.75pt,y=0.75pt,yscale=-1,xscale=1]
%uncomment if require: \path (0,300); %set diagram left start at 0, and has height of 300

%Straight Lines [id:da8639906661280463] 
\draw    (103.71,146) -- (198,146) ;
\draw [shift={(150.85,146)}, rotate = 180] [fill={rgb, 255:red, 0; green, 0; blue, 0 }  ][line width=0.08]  [draw opacity=0] (12,-3) -- (0,0) -- (12,3) -- cycle    ;
%Shape: Square [id:dp04569967316026591] 
\draw   (211,133) -- (185,133) -- (185,159) -- (211,159) -- cycle ;
%Straight Lines [id:da3166689586404512] 
\draw    (211,159) -- (185,133) ;

%Straight Lines [id:da29968214881941746] 
\draw    (198,146) -- (198,242.34) ;
\draw [shift={(198,194.17)}, rotate = 270] [fill={rgb, 255:red, 0; green, 0; blue, 0 }  ][line width=0.08]  [draw opacity=0] (12,-3) -- (0,0) -- (12,3) -- cycle    ;
%Straight Lines [id:da1553972370497858] 
\draw    (175,242) -- (220.71,242) ;
%Straight Lines [id:da4087266464328636] 
\draw    (184,242) -- (177.71,248.29) ;
%Straight Lines [id:da11303237989663151] 
\draw    (192,242) -- (185.71,248.29) ;
%Straight Lines [id:da23272994898790333] 
\draw    (201,242) -- (194.71,248.29) ;
%Straight Lines [id:da565945114755471] 
\draw    (209,242) -- (202.71,248.29) ;
%Straight Lines [id:da7662144225669372] 
\draw    (217,242) -- (210.71,248.29) ;

%Shape: Chord [id:dp8830733298615043] 
\draw   (182.75,71.98) .. controls (182.78,62.79) and (189.39,55.29) .. (197.69,55.14) .. controls (206.11,54.99) and (213.08,62.47) .. (213.25,71.84) -- cycle ;
%Straight Lines [id:da9870604378048866] 
\draw    (198,72.12) -- (198,146) ;
\draw [shift={(198,109.06)}, rotate = 90] [fill={rgb, 255:red, 0; green, 0; blue, 0 }  ][line width=0.08]  [draw opacity=0] (12,-3) -- (0,0) -- (12,3) -- cycle    ;
%Straight Lines [id:da71951825473869] 
\draw    (198,146) -- (198,242.34) ;
\draw [shift={(198,194.17)}, rotate = 90] [fill={rgb, 255:red, 0; green, 0; blue, 0 }  ][line width=0.08]  [draw opacity=0] (12,-3) -- (0,0) -- (12,3) -- cycle    ;
%Straight Lines [id:da5453688188581256] 
\draw    (198,146) -- (292.29,146) ;
\draw [shift={(245.15,146)}, rotate = 180] [fill={rgb, 255:red, 0; green, 0; blue, 0 }  ][line width=0.08]  [draw opacity=0] (12,-3) -- (0,0) -- (12,3) -- cycle    ;
%Shape: Chord [id:dp8729614260140681] 
\draw   (292.39,178.39) .. controls (303.31,173.16) and (311,160.63) .. (311,146) .. controls (311,131.24) and (303.17,118.61) .. (292.09,113.47) -- cycle ;
%Shape: Circle [id:dp9771704599798701] 
\draw   (406,146) .. controls (406,143.04) and (408.4,140.65) .. (411.35,140.65) .. controls (414.31,140.65) and (416.71,143.04) .. (416.71,146) .. controls (416.71,148.96) and (414.31,151.35) .. (411.35,151.35) .. controls (408.4,151.35) and (406,148.96) .. (406,146) -- cycle ;
%Straight Lines [id:da34181641468504864] 
\draw    (292.29,146) -- (406,146) ;
\draw [shift={(349.15,146)}, rotate = 180] [fill={rgb, 255:red, 0; green, 0; blue, 0 }  ][line width=0.08]  [draw opacity=0] (12,-3) -- (0,0) -- (12,3) -- cycle    ;
%Shape: Rectangle [id:dp2845062324633054] 
\draw  [draw opacity=0][fill={rgb, 255:red, 0; green, 0; blue, 0 }  ,fill opacity=0.13 ] (422,124) -- (433.71,124) -- (433.71,170.12) -- (422,170.12) -- cycle ;
%Shape: Rectangle [id:dp9612219314295352] 
\draw  [draw opacity=0][fill={rgb, 255:red, 0; green, 0; blue, 0 }  ,fill opacity=0.13 ] (391,124) -- (402.71,124) -- (402.71,170.12) -- (391,170.12) -- cycle ;
%Straight Lines [id:da24475763951458918] 
\draw    (406,146) -- (338.71,137.12) ;
\draw [shift={(372.35,141.56)}, rotate = 367.52] [fill={rgb, 255:red, 0; green, 0; blue, 0 }  ][line width=0.08]  [draw opacity=0] (12,-3) -- (0,0) -- (12,3) -- cycle    ;




\end{tikzpicture}

    \caption{Measuring the position of a nanosphere with phase measurement}
    \label{fig:nanosphere}
\end{figure}

The phase measure technique can also be used to measure an unknown position in the light circuit.
For example, consider \prettyref{fig:nanosphere}, where a nanosphere is contained in a light trap and we want to use an light beam to determine where exactly it is.
The equation of motion is 
\begin{equation}
    m \ddot{x} = - \omega_0^2 x_1 + F,
    \label{eq:nanosphere-eom}
\end{equation}
where teh radiational force is roughly
\begin{equation}
    F = \frac{P}{c} \sim \frac{SA}{c} \sim a^\dagger a \abs*{\mathcal{E}_l}^2 A,
\end{equation}
where $P$ is the light power, $S$ the Poynting vector, and $A$ is some ``effective area'' of the nanosphere.
To reflect the fact that light is made of photons, we replace the $F$ in \eqref{eq:nanosphere-eom} with its expectation plus a classical fluctuation, i.e.
\begin{equation}
    F(t) = \expval*{F} + \var{F} = \frac{\abs*{\mathcal{E}_l}^2 \abs*{\alpha}^2 A}{c} + \var{F},
\end{equation}
where 
\begin{equation}
    \expval*{\var{F(t)} \var{F(t')}} = \hbar k \expval*{F} \delta(t - t') \eqqcolon D \delta(t - t').
\end{equation}
The errors are
\begin{equation}
    (\var{x})_\text{shot noise} = \frac{1}{k} \var{\varphi} = \frac{1}{k} \frac{1}{2 \sqrt{\frac{P \tau}{\hbar \omega}}} , \quad (\var{x})_\text{radiation force} \sim \sqrt{\hbar k \frac{P \tau}{c}} \frac{\tau}{m}, 
\end{equation}

\begin{figure}
    \centering
    

\tikzset{every picture/.style={line width=0.75pt}} %set default line width to 0.75pt        

\begin{tikzpicture}[x=0.75pt,y=0.75pt,yscale=-1,xscale=1]
%uncomment if require: \path (0,300); %set diagram left start at 0, and has height of 300

%Straight Lines [id:da8528309674715007] 
\draw    (186.71,126) -- (234,126) ;
\draw [shift={(210.35,126)}, rotate = 180] [fill={rgb, 255:red, 0; green, 0; blue, 0 }  ][line width=0.08]  [draw opacity=0] (12,-3) -- (0,0) -- (12,3) -- cycle    ;
%Shape: Square [id:dp7042724426833782] 
\draw   (247,113) -- (221,113) -- (221,139) -- (247,139) -- cycle ;
%Straight Lines [id:da3438859397932843] 
\draw    (247,139) -- (221,113) ;

%Straight Lines [id:da10191972722445564] 
\draw    (234,126) -- (375.71,126) ;
\draw [shift={(304.85,126)}, rotate = 0] [fill={rgb, 255:red, 0; green, 0; blue, 0 }  ][line width=0.08]  [draw opacity=0] (12,-3) -- (0,0) -- (12,3) -- cycle    ;
%Straight Lines [id:da9686481941943572] 
\draw    (234,243.73) -- (234,126) ;
\draw [shift={(234,184.87)}, rotate = 270] [fill={rgb, 255:red, 0; green, 0; blue, 0 }  ][line width=0.08]  [draw opacity=0] (12,-3) -- (0,0) -- (12,3) -- cycle    ;
%Straight Lines [id:da3247807690686555] 
\draw    (211,243) -- (256.71,243) ;
%Straight Lines [id:da6427297077092098] 
\draw    (220,243) -- (213.71,249.29) ;
%Straight Lines [id:da913953914462837] 
\draw    (228,243) -- (221.71,249.29) ;
%Straight Lines [id:da9179362838857856] 
\draw    (237,243) -- (230.71,249.29) ;
%Straight Lines [id:da4675313777790828] 
\draw    (245,243) -- (238.71,249.29) ;
%Straight Lines [id:da9258057401969635] 
\draw    (253,243) -- (246.71,249.29) ;

%Straight Lines [id:da47963082724356765] 
\draw    (233.85,243) -- (233.85,125.27) ;
\draw [shift={(233.85,184.13)}, rotate = 450] [fill={rgb, 255:red, 0; green, 0; blue, 0 }  ][line width=0.08]  [draw opacity=0] (12,-3) -- (0,0) -- (12,3) -- cycle    ;
%Straight Lines [id:da6376390333090904] 
\draw    (234,126) -- (375.71,126) ;
\draw [shift={(304.85,126)}, rotate = 180] [fill={rgb, 255:red, 0; green, 0; blue, 0 }  ][line width=0.08]  [draw opacity=0] (12,-3) -- (0,0) -- (12,3) -- cycle    ;
%Straight Lines [id:da4148550943984819] 
\draw    (376.71,149) -- (376.71,103.29) ;
%Straight Lines [id:da7887178597524234] 
\draw    (376.71,140) -- (383,146.29) ;
%Straight Lines [id:da8324835853553432] 
\draw    (376.71,132) -- (383,138.29) ;
%Straight Lines [id:da337768231757265] 
\draw    (376.71,123) -- (383,129.29) ;
%Straight Lines [id:da5175439152369408] 
\draw    (376.71,115) -- (383,121.29) ;
%Straight Lines [id:da80009283071686] 
\draw    (376.71,107) -- (383,113.29) ;

%Straight Lines [id:da2255036133523889] 
\draw    (234,126) -- (234,51.29) ;
\draw [shift={(234,88.64)}, rotate = 450] [fill={rgb, 255:red, 0; green, 0; blue, 0 }  ][line width=0.08]  [draw opacity=0] (12,-3) -- (0,0) -- (12,3) -- cycle    ;
%Shape: Chord [id:dp5337552916403379] 
\draw   (218.75,51.15) .. controls (218.78,41.96) and (225.39,34.46) .. (233.69,34.31) .. controls (242.11,34.16) and (249.08,41.64) .. (249.25,51.01) -- cycle ;

% Text Node
\draw (311,133.4) node [anchor=north west][inner sep=0.75pt]    {$x_{1}$};
% Text Node
\draw (241,207.4) node [anchor=north west][inner sep=0.75pt]    {$x_{2}$};


\end{tikzpicture}

    \caption{Gravitational wave measurement}
\end{figure}

Of course, the length measurement technique developed for nanospheres is definitely not limited to this usage, and actually this idea is key to \emph{gravitational wave measurement}.
When gravitational waves in certain modes go by the interferometer, we have 
\begin{equation}
    x_1 = x_1(0) + h(t) L, \quad x_2 = x_2(0) - h(t) L.
\end{equation}

The measurement of the position of a nanosphere and the gravitational wave measurement both reveal a fact that simply increase the power cannot lift the precision unboundedly. 
Some part of the measurement error arises from the quantum nature of light.

\section{Beyond the standard quantum limit}

\begin{figure}
    \centering
    

\tikzset{every picture/.style={line width=0.75pt}} %set default line width to 0.75pt        

\begin{tikzpicture}[x=0.75pt,y=0.75pt,yscale=-1,xscale=1]
%uncomment if require: \path (0,387); %set diagram left start at 0, and has height of 387

%Straight Lines [id:da3354822585289712] 
\draw    (210.71,241) -- (258,241) ;
\draw [shift={(234.35,241)}, rotate = 180] [fill={rgb, 255:red, 0; green, 0; blue, 0 }  ][line width=0.08]  [draw opacity=0] (12,-3) -- (0,0) -- (12,3) -- cycle    ;
%Shape: Square [id:dp42488105431001344] 
\draw   (271,228) -- (245,228) -- (245,254) -- (271,254) -- cycle ;
%Straight Lines [id:da5181336762608837] 
\draw    (271,254) -- (245,228) ;

%Straight Lines [id:da9086781744813357] 
\draw    (258,241) -- (399.71,241) ;
\draw [shift={(328.85,241)}, rotate = 0] [fill={rgb, 255:red, 0; green, 0; blue, 0 }  ][line width=0.08]  [draw opacity=0] (12,-3) -- (0,0) -- (12,3) -- cycle    ;
%Straight Lines [id:da9942415443908998] 
\draw    (258,358.73) -- (258,241) ;
\draw [shift={(258,299.87)}, rotate = 270] [fill={rgb, 255:red, 0; green, 0; blue, 0 }  ][line width=0.08]  [draw opacity=0] (12,-3) -- (0,0) -- (12,3) -- cycle    ;
%Straight Lines [id:da09156784979712551] 
\draw    (235,358) -- (280.71,358) ;
%Straight Lines [id:da9047481813546134] 
\draw    (244,358) -- (237.71,364.29) ;
%Straight Lines [id:da39005286396348526] 
\draw    (252,358) -- (245.71,364.29) ;
%Straight Lines [id:da03705508132976587] 
\draw    (261,358) -- (254.71,364.29) ;
%Straight Lines [id:da07114261042559344] 
\draw    (269,358) -- (262.71,364.29) ;
%Straight Lines [id:da07570684336608013] 
\draw    (277,358) -- (270.71,364.29) ;

%Straight Lines [id:da43738155750881846] 
\draw    (257.85,358) -- (257.85,240.27) ;
\draw [shift={(257.85,299.13)}, rotate = 450] [fill={rgb, 255:red, 0; green, 0; blue, 0 }  ][line width=0.08]  [draw opacity=0] (12,-3) -- (0,0) -- (12,3) -- cycle    ;
%Straight Lines [id:da867414788618375] 
\draw    (258,241) -- (399.71,241) ;
\draw [shift={(328.85,241)}, rotate = 180] [fill={rgb, 255:red, 0; green, 0; blue, 0 }  ][line width=0.08]  [draw opacity=0] (12,-3) -- (0,0) -- (12,3) -- cycle    ;
%Straight Lines [id:da5272162143458297] 
\draw    (400.71,264) -- (400.71,218.29) ;
%Straight Lines [id:da5345743231367925] 
\draw    (400.71,255) -- (407,261.29) ;
%Straight Lines [id:da9171633594183413] 
\draw    (400.71,247) -- (407,253.29) ;
%Straight Lines [id:da7743837405811953] 
\draw    (400.71,238) -- (407,244.29) ;
%Straight Lines [id:da6740379931741514] 
\draw    (400.71,230) -- (407,236.29) ;
%Straight Lines [id:da8910595843036342] 
\draw    (400.71,222) -- (407,228.29) ;

%Straight Lines [id:da7070335138446038] 
\draw    (258,241) -- (258,166.29) ;
\draw [shift={(258,203.64)}, rotate = 270] [fill={rgb, 255:red, 0; green, 0; blue, 0 }  ][line width=0.08]  [draw opacity=0] (12,-3) -- (0,0) -- (12,3) -- cycle    ;
%Shape: Square [id:dp07552782640308608] 
\draw   (270,153) -- (244,153) -- (244,179) -- (270,179) -- cycle ;
%Straight Lines [id:da4235677909356923] 
\draw    (270,179) -- (244,153) ;

%Straight Lines [id:da9566604376901069] 
\draw    (258,166.29) -- (258,80.9) ;
\draw [shift={(258,123.6)}, rotate = 270] [fill={rgb, 255:red, 0; green, 0; blue, 0 }  ][line width=0.08]  [draw opacity=0] (12,-3) -- (0,0) -- (12,3) -- cycle    ;
%Straight Lines [id:da019543940713699026] 
\draw    (175.71,166.29) -- (258,166.29) ;
\draw [shift={(216.85,166.29)}, rotate = 0] [fill={rgb, 255:red, 0; green, 0; blue, 0 }  ][line width=0.08]  [draw opacity=0] (12,-3) -- (0,0) -- (12,3) -- cycle    ;
%Shape: Chord [id:dp5191039540292677] 
\draw   (175.72,181.54) .. controls (166.53,181.6) and (158.96,175.06) .. (158.73,166.76) .. controls (158.5,158.34) and (165.91,151.31) .. (175.28,151.05) -- cycle ;
%Shape: Rectangle [id:dp2202555785202991] 
\draw  [dash pattern={on 4.5pt off 4.5pt}] (152.71,67.9) -- (337.71,67.9) -- (337.71,186.9) -- (152.71,186.9) -- cycle ;

% Text Node
\draw (335,248.4) node [anchor=north west][inner sep=0.75pt]    {$x_{1}$};
% Text Node
\draw (265,322.4) node [anchor=north west][inner sep=0.75pt]    {$x_{2}$};
% Text Node
\draw (267,81.4) node [anchor=north west][inner sep=0.75pt]    {$S( \zeta )\ket{0}$};
% Text Node
\draw (154.71,70.9) node [anchor=north west][inner sep=0.75pt]   [align=left] {dark part};


\end{tikzpicture}

    \caption{One solution beyond the standard quantum limit}
    \label{fig:beyond-sql}
\end{figure}

To go beyond the standard quantum limit we can make use of the other input port, i.e. \emph{the dark port} of the beam splitter, for example we can use a device shown in \prettyref{fig:beyond-sql}, where 
\begin{equation}
    S(\zeta) = \ee^{\abs*{\zeta} (a^\dagger a^\dagger - a a)}.
\end{equation}
Defining
\begin{equation}
    X = \frac{1}{\sqrt{2}} (a + a^\dagger), \quad P = \frac{1}{\sqrt{2}} (a - a^\dagger),
\end{equation}
we find that 
\begin{equation}
    S^\dagger(\zeta) X S(\zeta) = X(\zeta) , \quad S^\dagger(\zeta) P S = P(\zeta), 
\end{equation}
and 
\begin{equation}
    \dv{X(\zeta)}{\zeta} = - X.
\end{equation}
Measuring $n_1$, we have 
\begin{equation}
    \var{n}_1 = \abs*{\alpha} \ee^{-\zeta}, 
\end{equation}
so we find tha the noise is \emph{squeezed} by the injection of $S(\zeta) \ket*{0}$.

Nonlinear interferometry and entangled state.

\end{document}