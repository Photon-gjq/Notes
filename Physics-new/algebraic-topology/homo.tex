\documentclass[hyperref, a4paper]{article}

\usepackage{geometry}
\usepackage{float}
\usepackage{titling}
\usepackage{titlesec}
% No longer needed, since we will use enumitem package
% \usepackage{paralist}
\usepackage{enumitem}
\usepackage{footnote}
\usepackage{enumerate}
\usepackage{amsmath, amssymb, amsthm}
\usepackage{mathtools}
\usepackage{bbm}
\usepackage{cite}
\usepackage{graphicx}
\usepackage{subcaption}
\usepackage{physics}
\usepackage{tensor}
\usepackage{siunitx}
\usepackage{booktabs}
\usepackage[version=4]{mhchem}
\usepackage{tikz}
\usepackage{xcolor}
\usepackage{listings}
\usepackage{autobreak}
\usepackage[ruled, vlined, linesnumbered]{algorithm2e}
\usepackage{xr-hyper}
\usepackage[colorlinks,unicode]{hyperref} % , linkcolor=black, anchorcolor=black, citecolor=black, urlcolor=black, filecolor=black
\usepackage{prettyref}

% Page style
\geometry{left=3.18cm,right=3.18cm,top=2.54cm,bottom=2.54cm}
\titlespacing{\paragraph}{0pt}{1pt}{10pt}[20pt]
\setlength{\droptitle}{-5em}
\preauthor{\vspace{-10pt}\begin{center}}
\postauthor{\par\end{center}}

% More compact lists 
\setlist[itemize]{itemindent=17pt, leftmargin=1pt}

% Math operators
\DeclareMathOperator{\timeorder}{T}
\DeclareMathOperator{\diag}{diag}
\DeclareMathOperator{\legpoly}{P}
\DeclareMathOperator{\primevalue}{P}
\DeclareMathOperator{\sgn}{sgn}
\DeclareMathOperator{\im}{im}
\newcommand*{\ii}{\mathrm{i}}
\newcommand*{\ee}{\mathrm{e}}
\newcommand*{\const}{\mathrm{const}}
\newcommand*{\suchthat}{\quad \text{s.t.} \quad}
\newcommand*{\argmin}{\arg\min}
\newcommand*{\argmax}{\arg\max}
\newcommand*{\normalorder}[1]{: #1 :}
\newcommand*{\pair}[1]{\langle #1 \rangle}
\newcommand*{\fd}[1]{\mathcal{D} #1}
\DeclareMathOperator{\bigO}{\mathcal{O}}
\DeclareMathOperator{\object}{Ob}
\DeclareMathOperator{\morphism}{Hom}

% TikZ setting
\usetikzlibrary{arrows,shapes,positioning}
\usetikzlibrary{arrows.meta}
\usetikzlibrary{decorations.markings}
\tikzstyle arrowstyle=[scale=1]
\tikzstyle directed=[postaction={decorate,decoration={markings,
    mark=at position .5 with {\arrow[arrowstyle]{stealth}}}}]
\tikzstyle ray=[directed, thick]
\tikzstyle dot=[anchor=base,fill,circle,inner sep=1pt]

% Algorithm setting
% Julia-style code
\SetKwIF{If}{ElseIf}{Else}{if}{}{elseif}{else}{end}
\SetKwFor{For}{for}{}{end}
\SetKwFor{While}{while}{}{end}
\SetKwProg{Function}{function}{}{end}
\SetArgSty{textnormal}

\newcommand*{\concept}[1]{{\textbf{#1}}}

\newrefformat{fig}{Figure~\ref{#1}}

% Embedded codes
\lstset{basicstyle=\ttfamily,
  showstringspaces=false,
  commentstyle=\color{gray},
  keywordstyle=\color{blue}
}

% Disable unsupported commands in bookmark titles 
\pdfstringdefDisableCommands{%
  \def\\{}%
  \def\texttt#1{<#1>}%
  \def\mathbb#1{#1}%
}
\pdfstringdefDisableCommands{\def\eqref#1{(\ref{#1})}}

\makeatletter
\pdfstringdefDisableCommands{\let\HyPsd@CatcodeWarning\@gobble}
\makeatother

\title{Homology and Homotopy Groups}
\author{Jinyuan Wu}

\begin{document}

\maketitle

This article is mainly based on \cite{nakahara}.

Topological field theories often emerge from condensed matter systems, and the topological invariants they give
are often connected to \concept{homotopy groups}, which, intuitively speaking, classify possible field 
configurations into different topological sectors. 

In practice homotopy groups are hard to calculate. That is why people often seek algebraic objects that 
are more easy to calculated and then connect them to homotopy groups (or other algebraic objects that we
are interested in). This approach is called \concept{algebraic topology}, and one most frequently used 
algebraic object is the \concept{homology group}. They do not have very intuitive meaning, but they are 
easier to deal with.

\section{Some basic facts about Abelian groups}

We use $+$ to denote the group operation of an Abelian group.
The expression $x - y$ is defined as $x^{-1} \circ y$.
The unit is denoted as $0$.
The expression $n x$ where $n \in \mathbb{N}$ and $g \in G$ means $g^n$.

A map between Abelian groups $f: G_1 \to G_2$ is said
to be a \concept{homomorphism} if $f(x + y) = f(x) + f(y)$, i.e. it keeps the multiplication relations. 
An \concept{isomorphism} is a homomorphism that is also a bijection. 

Suppose $H$ is a subgroup of $G$. We have a equivalence relation $x \sim y$ if and only if $x - y \in H$.
The equivalence class of $x$ is denoted as $[x]$, and the set of all equivalence classes in $G$ is denoted 
as $G / H$, which is the \concept{the quotient space}. We can easily find that $G / H$ is also a group 
and the group operation $+$ in $G$ naturally induces the group operation $+$ in $G/H$.
We have 
\begin{equation}
    G / G = \{ [0] \} = \{ [h] \}, \quad h \in H, 
\end{equation}
and 
\begin{equation}
    G / \{0\} = G.
\end{equation}

Some examples of quotient spaces:
\begin{equation}
    \mathbb{Z} / k \mathbb{Z} \simeq \mathbb{Z}_k.
\end{equation}

If $f: G_1 \to G_2$ is a homomorphism, it can be found that $\ker f$ is a subgroup of $G_1$ and $\im f$ is a subgroup 
of $G_2$. This lemma can be proved almost directly by definition.

Now we state the \concept{fundamental theorem of homomorphism}: for a homomorphism $f: G_1 \to G_2$, we have 
\begin{equation}
    G_1 / \ker f \simeq \im f.
\end{equation}
This is Theorem~3.1 in \cite{nakahara}. 



\bibliographystyle{plain}
\bibliography{algebraic-topo} 

\end{document}