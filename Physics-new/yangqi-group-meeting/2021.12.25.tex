\documentclass[hyperref, a4paper]{article}

\usepackage{geometry}
\usepackage{titling}
\usepackage{titlesec}
% No longer needed, since we will use enumitem package
% \usepackage{paralist}
\usepackage{enumitem}
\usepackage{footnote}
% Conflicts with enumitem
%\usepackage{enumerate}
\usepackage{amsmath, amssymb, amsthm}
\usepackage{mathtools}
\usepackage{bbm}
\usepackage{cite}
\usepackage{graphicx}
\usepackage{subfigure}
\usepackage{physics}
\usepackage{tensor}
\usepackage{siunitx}
\usepackage[version=4]{mhchem}
\usepackage{tikz}
\usepackage{xcolor}
\usepackage{listings}
\usepackage{autobreak}
\usepackage[ruled, vlined, linesnumbered]{algorithm2e}
\usepackage{nameref,zref-xr}
\zxrsetup{toltxlabel}
\zexternaldocument*[optics-]{../optics/optics}[optics.pdf]
\zexternaldocument*[solid-]{../solid/solid}[solid.pdf]
\usepackage[colorlinks,unicode]{hyperref} % , linkcolor=black, anchorcolor=black, citecolor=black, urlcolor=black, filecolor=black
\usepackage[most]{tcolorbox}
\usepackage{prettyref}

% Page style
\geometry{left=3.18cm,right=3.18cm,top=2.54cm,bottom=2.54cm}
\titlespacing{\paragraph}{0pt}{1pt}{10pt}[20pt]
\setlength{\droptitle}{-5em}
\preauthor{\vspace{-10pt}\begin{center}}
\postauthor{\par\end{center}}

% More compact lists 
\setlist[itemize]{
    itemindent=17pt, 
    leftmargin=1pt,
    listparindent=\parindent,
    parsep=0pt,
}

% Math operators
\DeclareMathOperator{\timeorder}{\mathcal{T}}
\DeclareMathOperator{\diag}{diag}
\DeclareMathOperator{\legpoly}{P}
\DeclareMathOperator{\primevalue}{P}
\DeclareMathOperator{\sgn}{sgn}
\newcommand*{\ii}{\mathrm{i}}
\newcommand*{\ee}{\mathrm{e}}
\newcommand*{\const}{\mathrm{const}}
\newcommand*{\suchthat}{\quad \text{s.t.} \quad}
\newcommand*{\argmin}{\arg\min}
\newcommand*{\argmax}{\arg\max}
\newcommand*{\normalorder}[1]{: #1 :}
\newcommand*{\pair}[1]{\langle #1 \rangle}
\newcommand*{\fd}[1]{\mathcal{D} #1}
\DeclareMathOperator{\bigO}{\mathcal{O}}

% TikZ setting
\usetikzlibrary{arrows,shapes,positioning}
\usetikzlibrary{calc}
\usetikzlibrary{arrows.meta}
\usetikzlibrary{decorations.markings}
\tikzstyle arrowstyle=[scale=1]
\tikzstyle directed=[postaction={decorate,decoration={markings,
    mark=at position .5 with {\arrow[arrowstyle]{stealth}}}}]
\tikzstyle ray=[directed, thick]
\tikzstyle dot=[anchor=base,fill,circle,inner sep=1pt]

% Algorithm setting
% Julia-style code
\SetKwIF{If}{ElseIf}{Else}{if}{}{elseif}{else}{end}
\SetKwFor{For}{for}{}{end}
\SetKwFor{While}{while}{}{end}
\SetKwProg{Function}{function}{}{end}
\SetArgSty{textnormal}

\newcommand*{\concept}[1]{{\textbf{#1}}}

% Embedded codes
\lstset{basicstyle=\ttfamily,
  showstringspaces=false,
  commentstyle=\color{gray},
  keywordstyle=\color{blue}
}

% Reference formatting
\newrefformat{fig}{Figure~\ref{#1} on page~\pageref{#1}}

% Color boxes
\tcbuselibrary{skins, breakable, theorems}
\newtcbtheorem[number within=section]{warning}{Warning}%
  {colback=orange!5,colframe=orange!65,fonttitle=\bfseries, breakable}{warn}
\newtcbtheorem[number within=section]{note}{Note}%
  {colback=green!5,colframe=green!65,fonttitle=\bfseries, breakable}{note}
\newtcbtheorem[number within=section]{info}{Info}%
  {colback=blue!5,colframe=blue!65,fonttitle=\bfseries, breakable}{info}

\title{Defect String Operator in CFT and Topological Orders by Prof. Yang Qi}
\author{Jinyuan Wu}

\begin{document}

\maketitle

The idea originated in an article by Liang Kong and Kitaev, and Wen-Jie Li and Xiao-Gang Wen, PR Research 2, 033417 2020.
We consider an (1+1)-dimensional transverse field Ising model. When we adjust the transverse field $h$,
there is a well known quantum phase transition, and there is a CFT on the critical point. 
We can, then, consider the system as the boundary of a (2+1)-dimensional $\mathbb{Z}_2$ toric-code model.
The ferromagnetic phase can then be viewed as an anyon condensation phase where $\expval{e} \neq 0$,
while the paramagnetic phase is an anyon condensation phase where $\expval{m} \neq 0$. 

\begin{info*}{}{}
    Haldane conjecture says spin-$1/2$ chains are gapless, and the conjecture can be generalized into 
    higher dimensions. Now we consider a spin-$1/2$ chain as the boundary state of a (2+1)-dimensional
    spin system (just like the spin-$1/2$ edge states of a AKLT chain). This idea means 
    \concept{Lieb-Schultz-Mattis (LSM) theorem}, which states that a spin system with translation and 
    spin rotation symmetry and half-integer spin per unit cell does not admit a gapped symmetric 
    ground state lacking fractionalized excitations.

    This can even be done for Fermi surface. Now people tend to consider an arbitrary gapless system as 
    the boundary of a gapped system.
\end{info*}

\begin{figure}
    \centering
    \tikzset{every picture/.style={line width=0.75pt}} %set default line width to 0.75pt        

\begin{tikzpicture}[x=0.75pt,y=0.75pt,yscale=-1,xscale=1]
%uncomment if require: \path (0,300); %set diagram left start at 0, and has height of 300

%Shape: Square [id:dp9775774399102888] 
\draw   (169,58) -- (219,58) -- (219,108) -- (169,108) -- cycle ;
%Shape: Square [id:dp5803861188505348] 
\draw   (219,58) -- (269,58) -- (269,108) -- (219,108) -- cycle ;
%Shape: Square [id:dp7379150177319249] 
\draw   (169,108) -- (219,108) -- (219,158) -- (169,158) -- cycle ;
%Shape: Square [id:dp6254975238468878] 
\draw   (219,108) -- (269,108) -- (269,158) -- (219,158) -- cycle ;
%Shape: Square [id:dp9668425225055952] 
\draw   (269,58) -- (319,58) -- (319,108) -- (269,108) -- cycle ;
%Shape: Square [id:dp1445208133029452] 
\draw   (269,108) -- (319,108) -- (319,158) -- (269,158) -- cycle ;
%Shape: Square [id:dp06887087359110211] 
\draw   (169,158) -- (219,158) -- (219,208) -- (169,208) -- cycle ;
%Shape: Square [id:dp9361353940191639] 
\draw   (219,158) -- (269,158) -- (269,208) -- (219,208) -- cycle ;
%Shape: Square [id:dp8594572702460204] 
\draw   (269,158) -- (319,158) -- (319,208) -- (269,208) -- cycle ;
%Straight Lines [id:da3812400947940302] 
\draw [color={rgb, 255:red, 80; green, 227; blue, 194 }  ,draw opacity=1 ][line width=1.5]    (169,58) -- (169,108) ;
%Straight Lines [id:da5089201346271957] 
\draw [color={rgb, 255:red, 80; green, 227; blue, 194 }  ,draw opacity=1 ][line width=1.5]    (169,108) -- (169,158) ;
%Straight Lines [id:da12680120462207456] 
\draw [color={rgb, 255:red, 80; green, 227; blue, 194 }  ,draw opacity=1 ][line width=1.5]    (169,108) -- (219,108) ;

% Text Node
\draw (167,83) node [anchor=east] [inner sep=0.75pt]   [align=left] {1};
% Text Node
\draw (167,133) node [anchor=east] [inner sep=0.75pt]   [align=left] {2};
% Text Node
\draw (194,105) node [anchor=south] [inner sep=0.75pt]   [align=left] {3};


\end{tikzpicture}

    \caption{Smooth boundary}
    \label{fig:smooth}
\end{figure}
Here we consider some edge states of a toric-code mode. The \concept{smooth} boundary is shown in 
\prettyref{fig:smooth}, where the boundary Hamiltonian is the sum of terms like $\sigma^x_1 \sigma^x_2 \sigma^x_3$.
We can see that an $e$-particle on the boundary cost energy, while an $m$-particle does not cost energy.
Therefore, there is a $m$-particle condensation in the smooth boundary.
The same works for $e$-particles. Consider the \concept{rough} boundary in \prettyref{fig:rough},
where the boundary Hamiltonian is the sum of $\sigma^z_1 \sigma^z_2 \sigma^z_3$, and we find an $m$-particle
on the boundary costs energy, while an $e$-particle does not, so there is an $e$-particle condensation.

\begin{figure}
    \centering
    

\tikzset{every picture/.style={line width=0.75pt}} %set default line width to 0.75pt        

\begin{tikzpicture}[x=0.75pt,y=0.75pt,yscale=-1,xscale=1]
%uncomment if require: \path (0,300); %set diagram left start at 0, and has height of 300

%Shape: Square [id:dp24607464006803847] 
\draw   (189,78) -- (239,78) -- (239,128) -- (189,128) -- cycle ;
%Shape: Square [id:dp6780792283236707] 
\draw   (239,78) -- (289,78) -- (289,128) -- (239,128) -- cycle ;
%Shape: Square [id:dp15774568684430879] 
\draw   (189,128) -- (239,128) -- (239,178) -- (189,178) -- cycle ;
%Shape: Square [id:dp5960520642729539] 
\draw   (239,128) -- (289,128) -- (289,178) -- (239,178) -- cycle ;
%Shape: Square [id:dp839187073580361] 
\draw   (289,78) -- (339,78) -- (339,128) -- (289,128) -- cycle ;
%Shape: Square [id:dp45190217994573056] 
\draw   (289,128) -- (339,128) -- (339,178) -- (289,178) -- cycle ;
%Shape: Square [id:dp573156382774133] 
\draw   (189,178) -- (239,178) -- (239,228) -- (189,228) -- cycle ;
%Shape: Square [id:dp03367077258628659] 
\draw   (239,178) -- (289,178) -- (289,228) -- (239,228) -- cycle ;
%Shape: Square [id:dp5299266714388193] 
\draw   (289,178) -- (339,178) -- (339,228) -- (289,228) -- cycle ;
%Straight Lines [id:da4855956972903335] 
\draw [color={rgb, 255:red, 74; green, 144; blue, 226 }  ,draw opacity=1 ][line width=1.5]    (139,178) -- (189,178) ;
%Straight Lines [id:da7481938919216213] 
\draw [color={rgb, 255:red, 74; green, 144; blue, 226 }  ,draw opacity=1 ][line width=1.5]    (189,128) -- (189,178) ;
%Straight Lines [id:da40103705479920326] 
\draw [color={rgb, 255:red, 74; green, 144; blue, 226 }  ,draw opacity=1 ][line width=1.5]    (139,128) -- (189,128) ;

% Text Node
\draw (164,180) node [anchor=north] [inner sep=0.75pt]   [align=left] {1};
% Text Node
\draw (187,153) node [anchor=east] [inner sep=0.75pt]   [align=left] {2};
% Text Node
\draw (164,125) node [anchor=south] [inner sep=0.75pt]   [align=left] {3};


\end{tikzpicture}

    \caption{Rough boundary}
    \label{fig:rough}
\end{figure}

We are still unable to determine the correspondence between the two any condensation phases and the two quantum
phases of transverse Ising chain. This can be more easily found in Wen-Plaquette model, which can be mapped 
precisely into the toric code model. 

\begin{figure}
    \centering
    

\tikzset{every picture/.style={line width=0.75pt}} %set default line width to 0.75pt        

\begin{tikzpicture}[x=0.75pt,y=0.75pt,yscale=-1,xscale=1]
%uncomment if require: \path (0,300); %set diagram left start at 0, and has height of 300

%Straight Lines [id:da900500795342539] 
\draw    (91,240) -- (128.56,240) ;
\draw [shift={(128.56,240)}, rotate = 0] [color={rgb, 255:red, 0; green, 0; blue, 0 }  ][fill={rgb, 255:red, 0; green, 0; blue, 0 }  ][line width=0.75]      (0, 0) circle [x radius= 3.35, y radius= 3.35]   ;
%Straight Lines [id:da07809278182037516] 
\draw    (128.56,240) -- (166.11,240) ;
\draw [shift={(166.11,240)}, rotate = 360] [color={rgb, 255:red, 0; green, 0; blue, 0 }  ][fill={rgb, 255:red, 0; green, 0; blue, 0 }  ][line width=0.75]      (0, 0) circle [x radius= 3.35, y radius= 3.35]   ;
%Straight Lines [id:da5259149614011949] 
\draw    (166.11,240) -- (203.67,240) ;
\draw [shift={(203.67,240)}, rotate = 0] [color={rgb, 255:red, 0; green, 0; blue, 0 }  ][fill={rgb, 255:red, 0; green, 0; blue, 0 }  ][line width=0.75]      (0, 0) circle [x radius= 3.35, y radius= 3.35]   ;
%Straight Lines [id:da6979630179453218] 
\draw    (203.67,240) -- (241.22,240) ;
\draw [shift={(241.22,240)}, rotate = 360] [color={rgb, 255:red, 0; green, 0; blue, 0 }  ][fill={rgb, 255:red, 0; green, 0; blue, 0 }  ][line width=0.75]      (0, 0) circle [x radius= 3.35, y radius= 3.35]   ;
%Straight Lines [id:da25229411786434075] 
\draw    (241.22,240) -- (278.78,240) ;
\draw [shift={(278.78,240)}, rotate = 0] [color={rgb, 255:red, 0; green, 0; blue, 0 }  ][fill={rgb, 255:red, 0; green, 0; blue, 0 }  ][line width=0.75]      (0, 0) circle [x radius= 3.35, y radius= 3.35]   ;
%Straight Lines [id:da32720370682371613] 
\draw    (278.78,240) -- (316.33,240) ;
\draw [shift={(316.33,240)}, rotate = 0] [color={rgb, 255:red, 0; green, 0; blue, 0 }  ][fill={rgb, 255:red, 0; green, 0; blue, 0 }  ][line width=0.75]      (0, 0) circle [x radius= 3.35, y radius= 3.35]   ;
%Straight Lines [id:da2145285782145716] 
\draw    (316.33,240) -- (353.89,240) ;
\draw [shift={(353.89,240)}, rotate = 0] [color={rgb, 255:red, 0; green, 0; blue, 0 }  ][fill={rgb, 255:red, 0; green, 0; blue, 0 }  ][line width=0.75]      (0, 0) circle [x radius= 3.35, y radius= 3.35]   ;
%Straight Lines [id:da35616913345634704] 
\draw    (353.89,240) -- (391.44,240) ;
\draw [shift={(391.44,240)}, rotate = 0] [color={rgb, 255:red, 0; green, 0; blue, 0 }  ][fill={rgb, 255:red, 0; green, 0; blue, 0 }  ][line width=0.75]      (0, 0) circle [x radius= 3.35, y radius= 3.35]   ;
%Straight Lines [id:da9260101918071779] 
\draw    (391.44,240) -- (429,240) ;
%Straight Lines [id:da4415721168750564] 
\draw    (91,203) -- (128.56,203) ;
\draw [shift={(128.56,203)}, rotate = 0] [color={rgb, 255:red, 0; green, 0; blue, 0 }  ][fill={rgb, 255:red, 0; green, 0; blue, 0 }  ][line width=0.75]      (0, 0) circle [x radius= 3.35, y radius= 3.35]   ;
%Straight Lines [id:da6907678781075219] 
\draw    (128.56,203) -- (166.11,203) ;
\draw [shift={(166.11,203)}, rotate = 0] [color={rgb, 255:red, 0; green, 0; blue, 0 }  ][fill={rgb, 255:red, 0; green, 0; blue, 0 }  ][line width=0.75]      (0, 0) circle [x radius= 3.35, y radius= 3.35]   ;
%Straight Lines [id:da1626471171615138] 
\draw    (166.11,203) -- (203.67,203) ;
\draw [shift={(203.67,203)}, rotate = 0] [color={rgb, 255:red, 0; green, 0; blue, 0 }  ][fill={rgb, 255:red, 0; green, 0; blue, 0 }  ][line width=0.75]      (0, 0) circle [x radius= 3.35, y radius= 3.35]   ;
%Straight Lines [id:da6761019473785617] 
\draw    (203.67,203) -- (241.22,203) ;
\draw [shift={(241.22,203)}, rotate = 0] [color={rgb, 255:red, 0; green, 0; blue, 0 }  ][fill={rgb, 255:red, 0; green, 0; blue, 0 }  ][line width=0.75]      (0, 0) circle [x radius= 3.35, y radius= 3.35]   ;
%Straight Lines [id:da279954881926445] 
\draw    (241.22,203) -- (278.78,203) ;
\draw [shift={(278.78,203)}, rotate = 0] [color={rgb, 255:red, 0; green, 0; blue, 0 }  ][fill={rgb, 255:red, 0; green, 0; blue, 0 }  ][line width=0.75]      (0, 0) circle [x radius= 3.35, y radius= 3.35]   ;
%Straight Lines [id:da09272944788144843] 
\draw    (278.78,203) -- (316.33,203) ;
\draw [shift={(316.33,203)}, rotate = 0] [color={rgb, 255:red, 0; green, 0; blue, 0 }  ][fill={rgb, 255:red, 0; green, 0; blue, 0 }  ][line width=0.75]      (0, 0) circle [x radius= 3.35, y radius= 3.35]   ;
%Straight Lines [id:da39137282415896446] 
\draw    (316.33,203) -- (353.89,203) ;
\draw [shift={(353.89,203)}, rotate = 0] [color={rgb, 255:red, 0; green, 0; blue, 0 }  ][fill={rgb, 255:red, 0; green, 0; blue, 0 }  ][line width=0.75]      (0, 0) circle [x radius= 3.35, y radius= 3.35]   ;
%Straight Lines [id:da47286308300982927] 
\draw    (353.89,203) -- (391.44,203) ;
\draw [shift={(391.44,203)}, rotate = 0] [color={rgb, 255:red, 0; green, 0; blue, 0 }  ][fill={rgb, 255:red, 0; green, 0; blue, 0 }  ][line width=0.75]      (0, 0) circle [x radius= 3.35, y radius= 3.35]   ;
%Straight Lines [id:da04014278551459549] 
\draw    (391.44,203) -- (429,203) ;
%Straight Lines [id:da39837824321908766] 
\draw    (91,165) -- (128.56,165) ;
\draw [shift={(128.56,165)}, rotate = 0] [color={rgb, 255:red, 0; green, 0; blue, 0 }  ][fill={rgb, 255:red, 0; green, 0; blue, 0 }  ][line width=0.75]      (0, 0) circle [x radius= 3.35, y radius= 3.35]   ;
%Straight Lines [id:da5419352787026661] 
\draw    (128.56,165) -- (166.11,165) ;
\draw [shift={(166.11,165)}, rotate = 0] [color={rgb, 255:red, 0; green, 0; blue, 0 }  ][fill={rgb, 255:red, 0; green, 0; blue, 0 }  ][line width=0.75]      (0, 0) circle [x radius= 3.35, y radius= 3.35]   ;
%Straight Lines [id:da032590854490881505] 
\draw    (166.11,165) -- (203.67,165) ;
\draw [shift={(203.67,165)}, rotate = 0] [color={rgb, 255:red, 0; green, 0; blue, 0 }  ][fill={rgb, 255:red, 0; green, 0; blue, 0 }  ][line width=0.75]      (0, 0) circle [x radius= 3.35, y radius= 3.35]   ;
%Straight Lines [id:da4521483114588589] 
\draw    (203.67,165) -- (241.22,165) ;
\draw [shift={(241.22,165)}, rotate = 0] [color={rgb, 255:red, 0; green, 0; blue, 0 }  ][fill={rgb, 255:red, 0; green, 0; blue, 0 }  ][line width=0.75]      (0, 0) circle [x radius= 3.35, y radius= 3.35]   ;
%Straight Lines [id:da6717991605748379] 
\draw    (241.22,165) -- (278.78,165) ;
\draw [shift={(278.78,165)}, rotate = 0] [color={rgb, 255:red, 0; green, 0; blue, 0 }  ][fill={rgb, 255:red, 0; green, 0; blue, 0 }  ][line width=0.75]      (0, 0) circle [x radius= 3.35, y radius= 3.35]   ;
%Straight Lines [id:da7617012386805049] 
\draw    (278.78,165) -- (316.33,165) ;
\draw [shift={(316.33,165)}, rotate = 0] [color={rgb, 255:red, 0; green, 0; blue, 0 }  ][fill={rgb, 255:red, 0; green, 0; blue, 0 }  ][line width=0.75]      (0, 0) circle [x radius= 3.35, y radius= 3.35]   ;
%Straight Lines [id:da7164411388462681] 
\draw    (316.33,165) -- (353.89,165) ;
\draw [shift={(353.89,165)}, rotate = 0] [color={rgb, 255:red, 0; green, 0; blue, 0 }  ][fill={rgb, 255:red, 0; green, 0; blue, 0 }  ][line width=0.75]      (0, 0) circle [x radius= 3.35, y radius= 3.35]   ;
%Straight Lines [id:da6319129263634178] 
\draw    (353.89,165) -- (391.44,165) ;
\draw [shift={(391.44,165)}, rotate = 0] [color={rgb, 255:red, 0; green, 0; blue, 0 }  ][fill={rgb, 255:red, 0; green, 0; blue, 0 }  ][line width=0.75]      (0, 0) circle [x radius= 3.35, y radius= 3.35]   ;
%Straight Lines [id:da42211669433613674] 
\draw    (391.44,165) -- (429,165) ;
%Straight Lines [id:da131081771580299] 
\draw    (91,128) -- (128.56,128) ;
\draw [shift={(128.56,128)}, rotate = 0] [color={rgb, 255:red, 0; green, 0; blue, 0 }  ][fill={rgb, 255:red, 0; green, 0; blue, 0 }  ][line width=0.75]      (0, 0) circle [x radius= 3.35, y radius= 3.35]   ;
%Straight Lines [id:da541541209031867] 
\draw    (128.56,128) -- (166.11,128) ;
\draw [shift={(166.11,128)}, rotate = 0] [color={rgb, 255:red, 0; green, 0; blue, 0 }  ][fill={rgb, 255:red, 0; green, 0; blue, 0 }  ][line width=0.75]      (0, 0) circle [x radius= 3.35, y radius= 3.35]   ;
%Straight Lines [id:da3534257343684195] 
\draw    (166.11,128) -- (203.67,128) ;
\draw [shift={(203.67,128)}, rotate = 0] [color={rgb, 255:red, 0; green, 0; blue, 0 }  ][fill={rgb, 255:red, 0; green, 0; blue, 0 }  ][line width=0.75]      (0, 0) circle [x radius= 3.35, y radius= 3.35]   ;
%Straight Lines [id:da5018507617665542] 
\draw    (203.67,128) -- (241.22,128) ;
\draw [shift={(241.22,128)}, rotate = 0] [color={rgb, 255:red, 0; green, 0; blue, 0 }  ][fill={rgb, 255:red, 0; green, 0; blue, 0 }  ][line width=0.75]      (0, 0) circle [x radius= 3.35, y radius= 3.35]   ;
%Straight Lines [id:da133906003212513] 
\draw    (241.22,128) -- (278.78,128) ;
\draw [shift={(278.78,128)}, rotate = 0] [color={rgb, 255:red, 0; green, 0; blue, 0 }  ][fill={rgb, 255:red, 0; green, 0; blue, 0 }  ][line width=0.75]      (0, 0) circle [x radius= 3.35, y radius= 3.35]   ;
%Straight Lines [id:da3771903229375777] 
\draw    (278.78,128) -- (316.33,128) ;
\draw [shift={(316.33,128)}, rotate = 0] [color={rgb, 255:red, 0; green, 0; blue, 0 }  ][fill={rgb, 255:red, 0; green, 0; blue, 0 }  ][line width=0.75]      (0, 0) circle [x radius= 3.35, y radius= 3.35]   ;
%Straight Lines [id:da43601395810878296] 
\draw    (316.33,128) -- (353.89,128) ;
\draw [shift={(353.89,128)}, rotate = 0] [color={rgb, 255:red, 0; green, 0; blue, 0 }  ][fill={rgb, 255:red, 0; green, 0; blue, 0 }  ][line width=0.75]      (0, 0) circle [x radius= 3.35, y radius= 3.35]   ;
%Straight Lines [id:da19117298716746367] 
\draw    (353.89,128) -- (391.44,128) ;
\draw [shift={(391.44,128)}, rotate = 0] [color={rgb, 255:red, 0; green, 0; blue, 0 }  ][fill={rgb, 255:red, 0; green, 0; blue, 0 }  ][line width=0.75]      (0, 0) circle [x radius= 3.35, y radius= 3.35]   ;
%Straight Lines [id:da5533989457281352] 
\draw    (391.44,128) -- (429,128) ;
%Straight Lines [id:da1709898707085964] 
\draw    (47,240) -- (47,104.4) ;
\draw [shift={(47,102.4)}, rotate = 90] [fill={rgb, 255:red, 0; green, 0; blue, 0 }  ][line width=0.08]  [draw opacity=0] (12,-3) -- (0,0) -- (12,3) -- cycle    ;
%Straight Lines [id:da007996348946977871] 
\draw [color={rgb, 255:red, 245; green, 166; blue, 35 }  ,draw opacity=1 ][line width=1.5]    (128.56,240) -- (166.11,240) ;
%Straight Lines [id:da2646806651057385] 
\draw [color={rgb, 255:red, 245; green, 166; blue, 35 }  ,draw opacity=1 ][line width=1.5]    (166.11,203) -- (203.67,203) ;
%Straight Lines [id:da45111264809783846] 
\draw [color={rgb, 255:red, 245; green, 166; blue, 35 }  ,draw opacity=1 ][line width=1.5]    (203.67,165) -- (241.22,165) ;
%Straight Lines [id:da18627084996838716] 
\draw [color={rgb, 255:red, 245; green, 166; blue, 35 }  ,draw opacity=1 ][line width=1.5]    (203.67,128) -- (241.22,128) ;
%Curve Lines [id:da46771619745544113] 
\draw [color={rgb, 255:red, 248; green, 231; blue, 28 }  ,draw opacity=1 ]   (133,252.4) .. controls (173,222.4) and (233,154.4) .. (231,112.4) ;
\draw [shift={(195.36,190.01)}, rotate = 128.43] [fill={rgb, 255:red, 248; green, 231; blue, 28 }  ,fill opacity=1 ][line width=0.08]  [draw opacity=0] (12,-3) -- (0,0) -- (12,3) -- cycle    ;

% Text Node
\draw (34,79) node [anchor=north west][inner sep=0.75pt]   [align=left] {time};


\end{tikzpicture}

    \caption{A string defect in an (1+1)-dimensional system. The orange bonds are flipped, and the yellow line is the defect in the spacetime.}
    \label{fig:defect-string-1d}
\end{figure}

Now we consider the generalized case in conformal field theories. We can impose \emph{defects} into the theory,
or in other words we can impose certain kind of external potential field. It is natural to impose a 
\emph{string defect} in an (1+1)-dimensional theory. For example, in a spin chain, we can flip the $J$ coefficient
on a certain bond, or in other words turning it from antiferromagnetic into ferromagnetic. At different 
time steps we can flip different bonds, so what we are doing is to impose a string defect in the spacetime 
(see \prettyref{fig:defect-string-1d}). Note that, however, that a closed string defect is trivial, because we can 
just flip all spins inside the closed string, and we find in this way the defect does not effect the outer world.
Therefore, only open string defects matter, and open strings defects with the same edges are equivalent. 
So what really matters are the edges of open strings, and in other words in (1+1)-dimensional CFTs, what matters
are pairs of point excitations.

This approach to identify low energy degrees of freedom can be generalized into a (2+1)-dimensional CFT, 
where we can impose defect \emph{surfaces}, and again closed defect surfaces are trivial, and open 
surfaces with the same edge are equivalent (because they only differ with a trivial closed surface), 
so what matter are the edge of open surfaces, and we have \emph{line excitations}. People usually call 
this \concept{defect string operators}. Note that they are objects in a (2+1)-dimensional CFT and do not 
have direct relation with the string defects in an (1+1)-dimensional CFT.

The properties of these excitations are unknown, but note that we can always put an CFT at the boundary of a 
gapped system - usually a topological order - by this duality we can already know a lot about the gapless 
state on the boundary.

This is precisely the case in a two-dimensional transverse field Ising model, which is considered as the boundary 
of a (3+1)-dimensional toric code model, and the two phases in the TFIM correspond to one anyon condensation phase 
and another line excitation condensation phase.  

\end{document}