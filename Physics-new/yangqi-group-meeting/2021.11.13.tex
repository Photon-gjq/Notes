\documentclass[hyperref, a4paper]{article}

\usepackage{geometry}
\usepackage{float}
\usepackage{titling}
\usepackage{titlesec}
% No longer needed, since we will use enumitem package
% \usepackage{paralist}
\usepackage{enumitem}
\usepackage{footnote}
% This package causes conflict.
% \usepackage{enumerate}
\usepackage{amsmath, amssymb, amsthm}
\usepackage{mathtools}
\usepackage{bbm}
\usepackage{cite}
\usepackage{graphicx}
\usepackage{subfigure}
\usepackage{physics}
\usepackage{tensor}
\usepackage{siunitx}
\usepackage{booktabs}
\usepackage[version=4]{mhchem}
\usepackage{tikz}
\usepackage{xcolor}
\usepackage{listings}
\usepackage{autobreak}
\usepackage[ruled, vlined, linesnumbered]{algorithm2e}
\usepackage{xr-hyper}
\usepackage[colorlinks,unicode]{hyperref} % , linkcolor=black, anchorcolor=black, citecolor=black, urlcolor=black, filecolor=black
\usepackage{prettyref}

% Page style
\geometry{left=3.18cm,right=3.18cm,top=2.54cm,bottom=2.54cm}
\titlespacing{\paragraph}{0pt}{1pt}{10pt}[20pt]
\setlength{\droptitle}{-5em}
\preauthor{\vspace{-10pt}\begin{center}}
\postauthor{\par\end{center}}

% More compact lists 
\setlist[itemize]{itemindent=17pt, leftmargin=1pt}

% Math operators
\DeclareMathOperator{\timeorder}{T}
\DeclareMathOperator{\diag}{diag}
\DeclareMathOperator{\legpoly}{P}
\DeclareMathOperator{\primevalue}{P}
\DeclareMathOperator{\sgn}{sgn}
\newcommand*{\ii}{\mathrm{i}}
\newcommand*{\ee}{\mathrm{e}}
\newcommand*{\const}{\mathrm{const}}
\newcommand*{\suchthat}{\quad \text{s.t.} \quad}
\newcommand*{\argmin}{\arg\min}
\newcommand*{\argmax}{\arg\max}
\newcommand*{\normalorder}[1]{: #1 :}
\newcommand*{\pair}[1]{\langle #1 \rangle}
\newcommand*{\fd}[1]{\mathcal{D} #1}
\DeclareMathOperator{\bigO}{\mathcal{O}}

% TikZ setting
\usetikzlibrary{arrows,shapes,positioning}
\usetikzlibrary{arrows.meta}
\usetikzlibrary{decorations.markings}
\tikzstyle arrowstyle=[scale=1]
\tikzstyle directed=[postaction={decorate,decoration={markings,
    mark=at position .5 with {\arrow[arrowstyle]{stealth}}}}]
\tikzstyle ray=[directed, thick]
\tikzstyle dot=[anchor=base,fill,circle,inner sep=1pt]

% Algorithm setting
% Julia-style code
\SetKwIF{If}{ElseIf}{Else}{if}{}{elseif}{else}{end}
\SetKwFor{For}{for}{}{end}
\SetKwFor{While}{while}{}{end}
\SetKwProg{Function}{function}{}{end}
\SetArgSty{textnormal}

\newcommand*{\concept}[1]{{\textbf{#1}}}

% Embedded codes
\lstset{basicstyle=\ttfamily,
  showstringspaces=false,
  commentstyle=\color{gray},
  keywordstyle=\color{blue}
}

% Math operators used in this lecture
\DeclareMathOperator{\cone}{C}
\DeclareMathOperator{\suspension}{S}
\DeclareMathOperator{\sigmasus}{\Sigma}
\DeclareMathOperator{\loopspace}{\Omega}

\title{$\omega$-spectrum and SPT by Tian Yuan}
\author{Jinyuan Wu}

\begin{document}

\maketitle

Suppose $X \in \mathsf{Top}$. We define the \concept{cone} of $X$ as 
\begin{equation}
    \cone X \coloneqq X \times I / X \times \{1\}.
\end{equation}
The meaning of ``cone'' is quite clear: we first make a cylinder with cross section $X$ and then make it into a cone.
Similarly we define the \concept{suspension} of $X$ as 
\begin{equation}
    \suspension X = X \times I / (X \times \{1\} \cup X \times \{0\}).
\end{equation}
For pointed space $X \in \mathsf{Top}_*$ we have 
\begin{equation}
    \sigmasus X = \suspension X / x_0 \times I.
\end{equation}

Now suppose we have two topological spaces $A$ and $X$, and $A$ is included in $X$.
We have the following sequence 
\begin{equation}
    A \hookrightarrow X \hookrightarrow \underbrace{X \cup \cone A}_{\simeq X / A} 
    \hookrightarrow  \underbrace{(X \cup \cone A) \cup \cone X}_{\simeq \suspension A} 
    \hookrightarrow \underbrace{((X \cup \cone A) \cup \cone X) \cup \cone (X \cup \cone A)}_{\simeq \suspension X} 
    \hookrightarrow \cdots.
\end{equation}
Here the union operation is the ``geometric'' one and uses the natural identifications like 
the one of $A$ and $A \times \{0\}$. Continuing this sequence we have 
\begin{equation}
    \suspension A \hookrightarrow \suspension X \hookrightarrow 
    \underbrace{\suspension X / \suspension A}_{\suspension(X / A)}
    \hookrightarrow \suspension^2 A \hookrightarrow \suspension^2 X \hookrightarrow \cdots.
    \label{eq:seq-1}
\end{equation}
We also have a version of \eqref{eq:seq-1} for pointed spaces, which can be obtained by replacing $\suspension$ with $\sigmasus$.

\eqref{eq:seq-1} and its pointed space version give us a strong sense of cohomology. 
Here we define the \concept{generalized cohomology}. 
This is defined by a functor $h^n : (\mathsf{hCW}_*)^\text{op} \to \mathsf{Ab}$ where $\mathsf{hSW}_*$ is the category of CW 
complexes in which we have modded out homotopy equivalence, and the following conditions
(called the \concept{axioms of generalized cohomology}) hold:
\begin{enumerate}
    \item We have the following exact sequence 
    \begin{equation}
        \cdots \to h^n(X / A) \stackrel{q^*}{\to} h^n(X) \stackrel{i^*}{\to} h^n(A) \stackrel{\delta}{\to} h^{n+1}(X / A) \to \cdots,
    \end{equation}
    where $\delta$ is a boundary operator (the choice of which is not specified by the axioms of generalized cohomology),
    and we have the commutative diagram 
    \begin{equation}
        \begin{matrix}
            h^n(A) & \stackrel{\delta}{\to} & h^{n+1}(X / A) \\
            \downarrow & & \downarrow \\
            h^n(B) & \stackrel{\delta}{\to} & h^{n+1}(Y / B).
        \end{matrix}
    \end{equation}
    \item we have 
    \begin{equation}
        h^n(V_\alpha X_\alpha) \stackrel{\prod_\alpha i^*_\alpha}{\longrightarrow} \prod_\alpha h^n(X_\alpha). 
    \end{equation}
\end{enumerate}

Condition 1 is equivalent to the condition that $h^n(X)$ is naturally equivalent to $h^{n+1}(\sigmasus X)$ and we have a 
short exact sequence
\begin{equation}
    h^n(X / A) \to h^n(X) \to h^n(A).
\end{equation}

Define the \concept{loop space} $\loopspace X$ of $X$ as the set of all possible $I \to X$ with good enough properties.
It can be verified that $\loopspace X$ has a topological structure.
Define $\langle A, B \rangle$ to be homotopy classes of (basepoint preserving?) maps from $A$ to $B$.
We have group structure on $\langle A, B \rangle$.
We have 
\begin{equation}
    \langle \sigmasus X, Y \rangle \simeq \langle X, \loopspace Y \rangle,
    \label{eq:iso-1}
\end{equation}
and we know that $x_0$ in $\sigmasus X$ is mapped to the trivial loop because it is contractable.
Note that from \eqref{eq:iso-1} we have 
\begin{equation}
    \pi_{n+1}(Y) \simeq \pi_n(\Omega Y).
\end{equation}
It can also be proved that $\langle \cdot, \loopspace^2 \cdot \rangle$ is always an Abelian group.

The definition of \concept{$\Omega$-spectrum} can now be given. $\{K_n\}_{n \in \mathbb{Z}}$ is an $\Omega$-spectrum if 
\begin{equation}
    \loopspace K_{n+1} \simeq K_n.
\end{equation}

From this definition we find a realization of generalized cohomology. The functor is 
\begin{equation}
    h^n(\cdot) = \langle \cdot, K_n \rangle.
    \label{eq:cohom-imp}
\end{equation}
Since $\langle \cdot, \loopspace^2 \cdot \rangle$ is always an Abelian group,
we find 
\[
    \langle X, K_n \rangle \simeq \langle X, \loopspace K_{n+1} \rangle \simeq \langle X, \loopspace^2 K_{n+2} \rangle \in \mathsf{Ab},
\]
so we see \eqref{eq:cohom-imp} is definitely to $\mathsf{Ab}$.

The proof of the exact sequence condition 1 can be finished with the help of \eqref{eq:seq-1}.

We finish this introduction with the \concept{Brown representation}, which says that every generalized cohomology can be implemented 
by an $\Omega$-spectrum in that we can always find a series $\{K_n\}_{n \in \mathbb{Z}}$ such that 
\begin{equation}
    \langle X, K_n \rangle \simeq h^n(X).
\end{equation}


\end{document}