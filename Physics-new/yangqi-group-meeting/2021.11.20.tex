\documentclass[hyperref, a4paper]{article}

\usepackage{geometry}
\usepackage{float}
\usepackage{titling}
\usepackage{titlesec}
% No longer needed, since we will use enumitem package
% \usepackage{paralist}
\usepackage{enumitem}
\usepackage{footnote}
% This package causes conflict.
% \usepackage{enumerate}
\usepackage{amsmath, amssymb, amsthm}
\usepackage{mathtools}
\usepackage{bbm}
\usepackage{cite}
\usepackage{graphicx}
\usepackage{subfigure}
\usepackage{physics}
\usepackage{tensor}
\usepackage{siunitx}
\usepackage{booktabs}
\usepackage[version=4]{mhchem}
\usepackage{tikz}
\usepackage{xcolor}
\usepackage{listings}
\usepackage{autobreak}
\usepackage[ruled, vlined, linesnumbered]{algorithm2e}
\usepackage{xr-hyper}
\usepackage[colorlinks,unicode]{hyperref} % , linkcolor=black, anchorcolor=black, citecolor=black, urlcolor=black, filecolor=black
\usepackage{prettyref}

% Page style
\geometry{left=3.18cm,right=3.18cm,top=2.54cm,bottom=2.54cm}
\titlespacing{\paragraph}{0pt}{1pt}{10pt}[20pt]
\setlength{\droptitle}{-5em}
\preauthor{\vspace{-10pt}\begin{center}}
\postauthor{\par\end{center}}

% More compact lists 
\setlist[itemize]{itemindent=17pt, leftmargin=1pt}

% Math operators
\DeclareMathOperator{\timeorder}{T}
\DeclareMathOperator{\diag}{diag}
\DeclareMathOperator{\legpoly}{P}
\DeclareMathOperator{\primevalue}{P}
\DeclareMathOperator{\sgn}{sgn}
\newcommand*{\ii}{\mathrm{i}}
\newcommand*{\ee}{\mathrm{e}}
\newcommand*{\const}{\mathrm{const}}
\newcommand*{\suchthat}{\quad \text{s.t.} \quad}
\newcommand*{\argmin}{\arg\min}
\newcommand*{\argmax}{\arg\max}
\newcommand*{\normalorder}[1]{: #1 :}
\newcommand*{\pair}[1]{\langle #1 \rangle}
\newcommand*{\fd}[1]{\mathcal{D} #1}
\DeclareMathOperator{\bigO}{\mathcal{O}}

% TikZ setting
\usetikzlibrary{arrows,shapes,positioning}
\usetikzlibrary{arrows.meta}
\usetikzlibrary{decorations.markings}
\tikzstyle arrowstyle=[scale=1]
\tikzstyle directed=[postaction={decorate,decoration={markings,
    mark=at position .5 with {\arrow[arrowstyle]{stealth}}}}]
\tikzstyle ray=[directed, thick]
\tikzstyle dot=[anchor=base,fill,circle,inner sep=1pt]

% Algorithm setting
% Julia-style code
\SetKwIF{If}{ElseIf}{Else}{if}{}{elseif}{else}{end}
\SetKwFor{For}{for}{}{end}
\SetKwFor{While}{while}{}{end}
\SetKwProg{Function}{function}{}{end}
\SetArgSty{textnormal}

\newcommand*{\concept}[1]{{\textbf{#1}}}

% Embedded codes
\lstset{basicstyle=\ttfamily,
  showstringspaces=false,
  commentstyle=\color{gray},
  keywordstyle=\color{blue}
}

% Math operators used in this lecture
\DeclareMathOperator{\cone}{C}
\DeclareMathOperator{\suspension}{S}
\DeclareMathOperator{\sigmasus}{\Sigma}
\DeclareMathOperator{\loopspace}{\Omega}

\title{Atomic Magnetism by Yao Wang}
\author{Jinyuan Wu}

\begin{document}

\maketitle

\section{Magnetism of a single atom}

Note that both the total spin magnetic moment and the orbital magnetic moment of a closed shell are zero, 
and the magnetism shown in an atom solely comes from open shells.
We use the following Hamiltonian to describe open shells:
\begin{equation}
    H = \sum_i \frac{\vb*{p}_i^2}{2m} + \sum_i V(\vb*{r}_i) + 
    \sum_{i < j} \frac{e^2}{4 \pi \epsilon_0 \abs*{\vb*{r}_{ij}}} + \sum_i \xi \vb*{l}_i \cdot \vb*{s}_i.
\end{equation}
The (pseudo)potential $V(\vb*{r})$ is assumed to only be dependent to $\abs*{\vb*{r}}_i$, and in this case symmetry guarantees 
it is spherical. This is not the case when we do \emph{ab initio} calculation and make quantitative comparison between 
computational and experimental results. 

The Coulomb interaction between electrons means that there is no rotational invariance for a single electron, but the atom 
as a whole still has rotational invariance. Therefore, if we just consider the Coulomb interaction, then the good quantum 
numbers of the system include the total orbital angular momentum $\vb*{L}$ and the total spin momentum $\vb*{S}$.
Taking the spin-orbital coupling into account, only $\vb*{J} = \vb*{L} + \vb*{S}$ is a good quantum number.

The relative magnitudes of the two interaction term decide whether we should use \concept{$L$-$S$ coupling} 
or \concept{$J$-$J$ coupling}. For elements that are usually encountered in condensed matter physics, $L$-$S$ coupling
is the correct choice, where Coulomb repulsion is much stronger than spin-orbital coupling.
In this way we have the \concept{Hund's rule}: due to Coulomb repulsion the total spin should be maximized in the ground state,
and then due to spin-orbital coupling the total orbital angular momentum can be determined, and finally we obtain $J$ and 
can then derive the Landé $g$-factor.

Adding a megnetic field $\vb*{B} = (0, 0, B_0)$, we have 
\begin{equation}
    \sum_i \frac{(\vb*{p}_i + e \vb*{A})^2}{2m} = \sum_i \frac{\vb*{p}_i^2}{2m} 
    - \frac{e B_0}{2m} \sum_i \vb*{l}_i \cdot \vu*{e}_z + \frac{e^2 B_0^2}{4m} \sum_i (x_i^2 + y_i^2).
\end{equation}
The magnetic moment of the atom is therefore 
\begin{equation}
    \vb*{\mu} = \vb*{L} + % TODO
\end{equation}
and the atom-magnetic field coupling Hamiltonian is $- \vb*{\mu} \cdot \vb*{B}$.
We can see that if the atom has a total orbital angular momentum, then it shows \concept{paramagnetism}, 
otherwise it shows \concept{diamagnetism}.

\section{Crystal field theory}

The magnetic moment derived for isolated atoms often does not work well for atoms in a crystal. 
The reason is clear, since an electron is actually surrounded by other atoms in the crystal.
We call this correction of $V(\vb*{r})$ the \concept{crystal field}.
Usually for a $L$-$S$ coupling atom, we have 
\[
    \text{energy gap between different $l$} > \text{exchange energy} > \text{crystal field} > \text{spin-orbital coupling},
\]
and therefore the crystal field can be seen as a perturbation acting on orbitals that have no spin-orbital coupling and mixing only 
orbitals with the same $l$.

Group representation theory can be used to see the structure of energy splitting: 
We are actually splitting a irreducible representation of $SO(3)$ (with a definite $l$) into several 
irreducible representations of the point group $G$ after introducing the crystal field.
Orbitals in the same irreducible representation of $G$ share the same energy.
We use the symbols for irreducible representation of point groups to label the orbitals after energy splitting.
For example, orbitals that are in a $T_2$ representation are called $t_{2g}$ orbitals, where $g$ means 
``inversion even'' (on the contrary, subscript $u$ means ``inversion odd'').

We can also take the spin-orbital coupling back. Finally, what we are doing is to split a representation 
of $SO^\text{D}(3)$ into irreducible representations of $G^\text{D}$. $G^\text{D}$ has several more 
irreducible representation than $G$, the dimensions of which are always even, which comes from Kramers degeneracy.

The energy splitting caused by the crystal field is the underlying reason why crystals have colors. 
Hopping between split energy levels from orbitals with the same $l$ usually corresponds to a photon within 
the visible light spectrum, and hence the crystal absorbs visible light and therefore has the complementary 
color of the absorbed frequency. 

We already have automated tools for determining the structure of energy splitting from the crystal structure. 
With pesudopotentials, the energy levels can also be found.
\texttt{PyCrystalField}, for example, is such a tool in Python. \url{https://github.com/asche1/PyCrystalField}
Usually the number of low energy orbitals is limited and they often have overwhelmingly dominant $j$ components.
That is why Ising-like degrees of freedom are frequently used to model magnetism in crystals.
Note that crystal field theory is a single-electron model and is unable to determine the spin coupling.
There are various methods to obtain parameters in the spin model for the crystal
We can put different magnetic moments to different atoms and use DFT to find the energy, 
and thus we can obtain the parameters. This approach is frequently used, though the fact that it completely
ignores quantum fluctuation makes it dubious. Another approach is to write down a spin model and fit the experimental data.
The effective model is usually \emph{not} a Heisenberg model because spin-orbital coupling destroys the $SU(2)$ 
symmetry for spins. People often use Ising model to highlight the anisotropic nature of the system.
It should also be noted that the ``spins'' used in the effective models are not real spins.
The magnetic structure of these pseudo-spins may not be detected by experiments that are designed to detect ordinary
magnetic orders. We may need to use NMR or neutron scattering experiments to detect them.

It should be noted that in the crystal field theory we are only dealing with localized electrons. 
For systems with itinerant electrons the approximation fails.
Also, tools like \texttt{PyCrystalField} regard atoms as \emph{point charges}, which is of course not 
correct, so energies calculated are usually quantitatively unreliable.
What we are sure are only eigenstates. For quantitative investigation, we can use the eigenstates to obtain 
an ansatz of scattering experiments and then fit the experiment data to obtain energy splitting.

\end{document}