\documentclass[hyperref, a4paper]{article}

\usepackage{geometry}
\usepackage{titling}
\usepackage{titlesec}
% No longer needed, since we will use enumitem package
% \usepackage{paralist}
\usepackage{enumitem}
\usepackage{footnote}
% Conflicts with enumitem
%\usepackage{enumerate}
\usepackage{amsmath, amssymb, amsthm}
\usepackage{mathtools}
\usepackage{bbm}
\usepackage{cite}
\usepackage{graphicx}
\usepackage{subfigure}
\usepackage{physics}
\usepackage{tensor}
\usepackage{siunitx}
\usepackage[version=4]{mhchem}
\usepackage{tikz}
\usepackage{xcolor}
\usepackage{listings}
\usepackage{autobreak}
\usepackage[ruled, vlined, linesnumbered]{algorithm2e}
\usepackage{nameref,zref-xr}
\zxrsetup{toltxlabel}
\zexternaldocument*[optics-]{../optics/optics}[optics.pdf]
\zexternaldocument*[solid-]{../solid/solid}[solid.pdf]
\usepackage[colorlinks,unicode]{hyperref} % , linkcolor=black, anchorcolor=black, citecolor=black, urlcolor=black, filecolor=black
\usepackage[most]{tcolorbox}
\usepackage{prettyref}

% Page style
\geometry{left=3.18cm,right=3.18cm,top=2.54cm,bottom=2.54cm}
\titlespacing{\paragraph}{0pt}{1pt}{10pt}[20pt]
\setlength{\droptitle}{-5em}
\preauthor{\vspace{-10pt}\begin{center}}
\postauthor{\par\end{center}}

% More compact lists 
\setlist[itemize]{
    itemindent=17pt, 
    leftmargin=1pt,
    listparindent=\parindent,
    parsep=0pt,
}

% Math operators
\DeclareMathOperator{\timeorder}{\mathcal{T}}
\DeclareMathOperator{\diag}{diag}
\DeclareMathOperator{\legpoly}{P}
\DeclareMathOperator{\primevalue}{P}
\DeclareMathOperator{\sgn}{sgn}
\newcommand*{\ii}{\mathrm{i}}
\newcommand*{\ee}{\mathrm{e}}
\newcommand*{\const}{\mathrm{const}}
\newcommand*{\suchthat}{\quad \text{s.t.} \quad}
\newcommand*{\argmin}{\arg\min}
\newcommand*{\argmax}{\arg\max}
\newcommand*{\normalorder}[1]{: #1 :}
\newcommand*{\pair}[1]{\langle #1 \rangle}
\newcommand*{\fd}[1]{\mathcal{D} #1}
\DeclareMathOperator{\bigO}{\mathcal{O}}

% TikZ setting
\usetikzlibrary{arrows,shapes,positioning}
\usetikzlibrary{calc}
\usetikzlibrary{arrows.meta}
\usetikzlibrary{decorations.markings}
\tikzstyle arrowstyle=[scale=1]
\tikzstyle directed=[postaction={decorate,decoration={markings,
    mark=at position .5 with {\arrow[arrowstyle]{stealth}}}}]
\tikzstyle ray=[directed, thick]
\tikzstyle dot=[anchor=base,fill,circle,inner sep=1pt]

% Algorithm setting
% Julia-style code
\SetKwIF{If}{ElseIf}{Else}{if}{}{elseif}{else}{end}
\SetKwFor{For}{for}{}{end}
\SetKwFor{While}{while}{}{end}
\SetKwProg{Function}{function}{}{end}
\SetArgSty{textnormal}

\newcommand*{\concept}[1]{{\textbf{#1}}}

% Embedded codes
\lstset{basicstyle=\ttfamily,
  showstringspaces=false,
  commentstyle=\color{gray},
  keywordstyle=\color{blue}
}

% Reference formatting
\newrefformat{fig}{Figure~\ref{#1} on page~\pageref{#1}}

% Color boxes
\tcbuselibrary{skins, breakable, theorems}
\newtcbtheorem[number within=section]{warning}{Warning}%
  {colback=orange!5,colframe=orange!65,fonttitle=\bfseries, breakable}{warn}
\newtcbtheorem[number within=section]{note}{Note}%
  {colback=green!5,colframe=green!65,fonttitle=\bfseries, breakable}{note}
\newtcbtheorem[number within=section]{info}{Info}%
  {colback=blue!5,colframe=blue!65,fonttitle=\bfseries, breakable}{info}

\title{Jinxiang Hao on KT fluctuations on Superconductor Films}
\author{Jinyuan Wu}

\begin{document}

\maketitle

Today's representation is about how to use a 2D Coulomb gas model to explain the behavior of oscillating magnetic 
susceptibility for certain kind of superconducting films.

A 2D Coulomb gas consists of particles with positive and negative charges of equal magnitude. 
The Hamiltonian is 
\begin{equation}
    H = \sum_{\pair{i, j}} V(r_{ij}) q_i q_j,
\end{equation}
where $V(r)$ is the $d$-dimensional Coulomb potential, and in 2D it is  
\begin{equation}
    V(r_{ij}) = - \ln r_{ij} / \lambda.
\end{equation}
We already know that it is equivalent to a sine-Gordon model after a Hubbard-Stratonovich transformation, and 
it is the effective theory of vortices in the XY model. Therefore it undergoes a KT phase transition at 
a critical temperature $T_\text{c}$ when the numbers of positive and negative charged particles are the same. 

Now we discuss a \emph{non-neutral} Coulomb gas. We introduce a parameter $\Delta E$ to give positive and 
negative charges different energies, or in other words, difference in chemical potentials of positive and 
negative charges. When $\Delta E = 0$, there is no KT phase transition (just like the case in the Ising model:
when there is a magnetic field, there is no second order phase transition), but as $\Delta E$ changes 
there may be some non-trivial phenomena (again like the case in the Ising model).

So what is the relation between KT phase transition and superconductivity? We know the complex order parameter 
in superconductivity itself is in a superfluid phase, and only its phase $\theta$ is important. Note that the 
target space of $\theta$ is $S^{1}$, and we have vortices in the system. We have discussed this
\href{../quantum-fluid/wen-boson.pdf}{here}. Also, note that the magnetic field in superconductivity is just 
$\Delta E$ in a Coulomb gas. The last thing we need to map from superconductivity to the Coulomb gas is the 
boundary, which corresponds to $\vb*{n} \cdot \grad \theta = 0$ and can be modeled as the boundary of a 
ideal conductor and solved by the image charge method.

Now we can calculate observables from the Coulomb gas model and try to explain some behaviors of a 2D 
superconducting film. We do not need to simulate the Coulomb gas model numerically from the beginning.
Rather, we can use the idea of effective interaction and screening and measure the ``dielectric constant''
from experiment and calculate the susceptibility. The oscillation of the susceptibility can be intuitively 
explained as follows: as the magnetic field increases, at first it is weak and there is no vortex, so 
the susceptibility is zero; when the magnetic field is able to excite a vortex, the current increases quickly,
so there is a peak in the susceptibility; then the magnetic field is able to excite a cortex but unable to 
excite two vortices, and again the susceptibility is down to zero. These steps occur over and over again,
therefore inducing oscillating a magnetic susceptibility.

\end{document}