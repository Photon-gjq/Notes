\documentclass[hyperref, a4paper]{article}

\usepackage{geometry}
\usepackage{titling}
\usepackage{titlesec}
% No longer needed, since we will use enumitem package
% \usepackage{paralist}
\usepackage{enumitem}
\usepackage{footnote}
\usepackage{enumerate}
\usepackage{amsmath, amssymb, amsthm}
\usepackage{mathtools}
\usepackage{bbm}
\usepackage{cite}
\usepackage{graphicx}
\usepackage{subfigure}
\usepackage{physics}
\usepackage{tensor}
\usepackage{siunitx}
\usepackage[version=4]{mhchem}
\usepackage{tikz}
\usepackage{xcolor}
\usepackage{listings}
\usepackage{autobreak}
\usepackage[ruled, vlined, linesnumbered]{algorithm2e}
\usepackage{nameref,zref-xr}
\zxrsetup{toltxlabel}
\zexternaldocument*[optics-]{../optics/optics}[optics.pdf]
\zexternaldocument*[solid-]{../solid/solid}[solid.pdf]
\usepackage[colorlinks,unicode]{hyperref} % , linkcolor=black, anchorcolor=black, citecolor=black, urlcolor=black, filecolor=black
\usepackage[most]{tcolorbox}
\usepackage{prettyref}

% Page style
\geometry{left=3.18cm,right=3.18cm,top=2.54cm,bottom=2.54cm}
\titlespacing{\paragraph}{0pt}{1pt}{10pt}[20pt]
\setlength{\droptitle}{-5em}
\preauthor{\vspace{-10pt}\begin{center}}
\postauthor{\par\end{center}}

% More compact lists 
\setlist[itemize]{
    itemindent=17pt, 
    leftmargin=1pt,
    listparindent=\parindent,
    parsep=0pt,
}

% Math operators
\DeclareMathOperator{\timeorder}{\mathcal{T}}
\DeclareMathOperator{\diag}{diag}
\DeclareMathOperator{\legpoly}{P}
\DeclareMathOperator{\primevalue}{P}
\DeclareMathOperator{\sgn}{sgn}
\newcommand*{\ii}{\mathrm{i}}
\newcommand*{\ee}{\mathrm{e}}
\newcommand*{\const}{\mathrm{const}}
\newcommand*{\suchthat}{\quad \text{s.t.} \quad}
\newcommand*{\argmin}{\arg\min}
\newcommand*{\argmax}{\arg\max}
\newcommand*{\normalorder}[1]{: #1 :}
\newcommand*{\pair}[1]{\langle #1 \rangle}
\newcommand*{\fd}[1]{\mathcal{D} #1}
\DeclareMathOperator{\bigO}{\mathcal{O}}

% TikZ setting
\usetikzlibrary{arrows,shapes,positioning}
\usetikzlibrary{calc}
\usetikzlibrary{arrows.meta}
\usetikzlibrary{decorations.markings}
\tikzstyle arrowstyle=[scale=1]
\tikzstyle directed=[postaction={decorate,decoration={markings,
    mark=at position .5 with {\arrow[arrowstyle]{stealth}}}}]
\tikzstyle ray=[directed, thick]
\tikzstyle dot=[anchor=base,fill,circle,inner sep=1pt]

% Algorithm setting
% Julia-style code
\SetKwIF{If}{ElseIf}{Else}{if}{}{elseif}{else}{end}
\SetKwFor{For}{for}{}{end}
\SetKwFor{While}{while}{}{end}
\SetKwProg{Function}{function}{}{end}
\SetArgSty{textnormal}

\newcommand*{\concept}[1]{{\textbf{#1}}}

% Embedded codes
\lstset{basicstyle=\ttfamily,
  showstringspaces=false,
  commentstyle=\color{gray},
  keywordstyle=\color{blue}
}

% Reference formatting
\newrefformat{fig}{Figure~\ref{#1} on page~\pageref{#1}}

% Color boxes
\tcbuselibrary{skins, breakable, theorems}
\newtcbtheorem[number within=section]{warning}{Warning}%
  {colback=orange!5,colframe=orange!65,fonttitle=\bfseries, breakable}{warn}
\newtcbtheorem[number within=section]{note}{Note}%
  {colback=green!5,colframe=green!65,fonttitle=\bfseries, breakable}{note}

\title{TQD Model by Prof. Wang}
\author{Jinyuan Wu}
\date{November 26, 2021}

\begin{document}

\maketitle

\section{The TQD model}

The \concept{twisted quantum double (TQD)} models are a type of models with intrinsic topological order in 2D. 
We know quantum double models are generalization of the toric code, and TQDs are generalization of quantum 
double models. Details may be found in PRB 87, 125114 (2013).

We consider a triangulated 2D plane. We index each sites with a natural number. In this way the sites are 
\emph{ordered}. We place an arrow on each bond, the direction of which is from the site with a large index
to the site with a small index. This lattice is the playground of the model.
Note that if we view the arrows on the edges as oriented paths, then no closed loop will form on any triangle.

The input data of TQD on such a lattice $\Gamma$ with $E$ edges is as follows:
\begin{itemize}
    \item A group $G$. We place an element $g_i$ of $G$ on each edge $i$. 
    When $G = \mathbb{Z}_2$, for example, we have $g_i = \pm 1$. 
    These group elements are all degrees of freedom in the model, and the Hilbert space is given. 

    We use $[a b]$ to denote the group element on the edge between site $a$ and site $b$, and usually we 
    assume $a < b$. We further define 
    \begin{equation}
        [ab] = [ba]^{-1}.
    \end{equation}

    We have the inner product relation 
    \begin{equation}
    \braket{\tikzset{every picture/.style={line width=0.75pt}} %set default line width to 0.75pt        
\begin{tikzpicture}[x=0.75pt,y=0.75pt,yscale=-0.75,xscale=0.75, baseline=(XXXX.south) ]
\path (0,100);\path (100,0);\draw    ($(current bounding box.center)+(0,0.3em)$) node [anchor=south] (XXXX) {};
%Shape: Triangle [id:dp9949650978799922] 
\draw   (53.82,24.98) -- (82.67,79.93) -- (24.98,79.93) -- cycle ;
% Text Node
\draw (22.98,81.63) node [anchor=north east] [inner sep=0.75pt]    {$a$};
% Text Node
\draw (84.67,81.63) node [anchor=north west][inner sep=0.75pt]    {$b$};
% Text Node
\draw (53.82,23.28) node [anchor=south] [inner sep=0.75pt]    {$c$};
\end{tikzpicture}
}{\tikzset{every picture/.style={line width=0.75pt}} %set default line width to 0.75pt        
\begin{tikzpicture}[x=0.75pt,y=0.75pt,yscale=-0.75,xscale=0.75, baseline=(XXXX.south) ]
\path (0,100);\path (100,0);\draw    ($(current bounding box.center)+(0,0.3em)$) node [anchor=south] (XXXX) {};
%Shape: Triangle [id:dp26755800404916497] 
\draw   (52.15,23.05) -- (81,78) -- (23.31,78) -- cycle ;
% Text Node
\draw (21.31,79.7) node [anchor=north east] [inner sep=0.75pt]    {$a'$};
% Text Node
\draw (83,79.7) node [anchor=north west][inner sep=0.75pt]    {$b'$};
% Text Node
\draw (52.15,21.35) node [anchor=south] [inner sep=0.75pt]    {$c'$};
\end{tikzpicture}
} =\delta_{[ ab][ a'b']} \delta_{[bc][b'c']} \delta_{[ac][a'c']},
\end{equation} 
    where we assume the orders of $a, b, c$ and $a,', b', c'$ are the same.

    \item A normalized 3-cocycle $\alpha \in H^3[G, U(1)]$. We require 
    \begin{equation}
        \var{\alpha}(g_0, g_1 g_2, g_3) = \alpha(g_1, g_2, g_3) \alpha^{-1}(g_0 g_1, g_2, g_3) 
        \alpha(g_0, g_1 g_2, g_3),
        \label{eq:cocycle-1}
    \end{equation}
    \begin{equation}
        \alpha^{-1}(g_0, g_1, g_2 g_3) \alpha(g_0, g_1, g_2) = 1,
        \label{eq:cocycle-2}
    \end{equation}
    and the normalization condition is 
    \begin{equation}
        \alpha(1, g, h) = \alpha(g, 1, h) = \alpha(g, h, 1) = 1.
    \end{equation}

    These 3-cocycles can be placed on a subgraph of the lattice like the follows:
    \begin{equation}
    \begin{gathered}
        \begin{tikzpicture}[x=0.75pt,y=0.75pt,yscale=-1,xscale=1]
            %uncomment if require: \path (0,300); %set diagram left start at 0, and has height of 300
            
            %Shape: Triangle [id:dp7098951064257721] 
            \draw   (176.51,109) -- (253.02,240.38) -- (100,240.38) -- cycle ;
            %Straight Lines [id:da8162486890668672] 
            \draw    (176.51,109) -- (176.51,186.38) ;
            %Straight Lines [id:da2843979428920951] 
            \draw    (176.51,186.38) -- (100,240.38) ;
            %Straight Lines [id:da2477682584878509] 
            \draw    (253.02,240.38) -- (176.51,186.38) ;
            
            % Text Node
            \draw (98,243.78) node [anchor=north east] [inner sep=0.75pt]    {$v_{1}$};
            % Text Node
            \draw (176.51,189.78) node [anchor=north] [inner sep=0.75pt]    {$v_{2}$};
            % Text Node
            \draw (255.02,243.78) node [anchor=north west][inner sep=0.75pt]    {$v_{3}$};
            % Text Node
            \draw (176.51,105.6) node [anchor=south] [inner sep=0.75pt]    {$v_{4}$};
            \end{tikzpicture}                
    \end{gathered}.
    \label{eq:subgraph}
\end{equation}
    The subgraph is actually a flattened tetrahedron, and it can be oriented.
    We see the ascending order of $v_1, v_2, v_3, v_4$ as the orientation.
    For positive orientation we assign $\alpha([v_1 v_2], [v_2 v_3], [v_3 v_4])$ to the subgraph,
    and for negative orientation we assign $\alpha([v_1 v_2], [v_2 v_3], [v_3 v_4])^{-1}$.
    It can be seen that \eqref{eq:subgraph} is $\alpha([v_1 v_2], [v_2 v_3], [v_3 v_4])^{\epsilon(v_1, v_2, v_3, v_4)}$.

    \item Now the Hamiltonian can be defined. We define 
    \begin{equation}
        H = - \sum_{v} A_v - \sum_f B_f,
        \label{eq:hamiltonian}
    \end{equation}
    where $v$ runs over all vertices and $f$ runs over all triangles.
    We have
    \begin{equation}
    B_{f}\ket{\tikzset{every picture/.style={line width=0.75pt}} %set default line width to 0.75pt        
\begin{tikzpicture}[x=0.75pt,y=0.75pt,yscale=-1,xscale=1, baseline=(XXXX.south) ]
\path (0,100);\path (100,0);\draw    ($(current bounding box.center)+(0,0.3em)$) node [anchor=south] (XXXX) {};
%Shape: Triangle [id:dp8774781724037415] 
\draw   (52.15,21.05) -- (81,76) -- (23.31,76) -- cycle ;
% Text Node
\draw (21.31,77.7) node [anchor=north east] [inner sep=0.75pt]    {$a$};
% Text Node
\draw (83,77.7) node [anchor=north west][inner sep=0.75pt]    {$b$};
% Text Node
\draw (52.15,19.35) node [anchor=south] [inner sep=0.75pt]    {$c$};
% Text Node
\draw (52.15,48.52) node    {$f$};
\end{tikzpicture}
} =\delta _{[ ab][ bc][ ca]}\ket{\tikzset{every picture/.style={line width=0.75pt}} %set default line width to 0.75pt        
\begin{tikzpicture}[x=0.75pt,y=0.75pt,yscale=-1,xscale=1, baseline=(XXXX.south) ]
\path (0,100);\path (100,0);\draw    ($(current bounding box.center)+(0,0.3em)$) node [anchor=south] (XXXX) {};
%Shape: Triangle [id:dp5571637338137228] 
\draw   (52.15,21.05) -- (81,76) -- (23.31,76) -- cycle ;
% Text Node
\draw (21.31,77.7) node [anchor=north east] [inner sep=0.75pt]    {$a$};
% Text Node
\draw (83,77.7) node [anchor=north west][inner sep=0.75pt]    {$b$};
% Text Node
\draw (52.15,19.35) node [anchor=south] [inner sep=0.75pt]    {$c$};
% Text Node
\draw (52.15,48.52) node    {$f$};
\end{tikzpicture}
}.
\end{equation}
    This quantity is unity when 
    \begin{equation}
        [a b] [b c] = [a c].
        \label{eq:flatness}
    \end{equation}
    This is called the \concept{flatness condition}, and $B_f$'s presence in the Hamiltonian means it holds 
    in the ground state. It means there is no flux in the ground state, if we regard $G$ to be a gauge group
    and $\{g_i\}$ a gauge field. Sometimes \eqref{eq:flatness} is also called the \concept{chain rule}.
    The definition of $A$ is 
    \begin{equation}
        A_v = \frac{1}{\abs*{G}} \sum_{[v'v] = g \in G} A_v^g,
    \end{equation}
    where 
    \begin{equation}
    A_{3}^{g}\ket{\tikzset{every picture/.style={line width=0.75pt}} %set default line width to 0.75pt        
\begin{tikzpicture}[x=0.75pt,y=0.75pt,yscale=-1,xscale=1, baseline=(XXXX.south) ]
\path (0,112);\path (105.97916412353516,0);\draw    ($(current bounding box.center)+(0,0.3em)$) node [anchor=south] (XXXX) {};
%Shape: Triangle [id:dp08920175392146201] 
\draw   (49.79,23.67) -- (81.6,84.67) -- (17.98,84.67) -- cycle ;
%Straight Lines [id:da7633652277024112] 
\draw    (49.79,66.67) -- (49.79,23.67) ;
%Straight Lines [id:da6132549336406812] 
\draw    (17.98,84.67) -- (49.79,66.67) ;
%Straight Lines [id:da8094238355730792] 
\draw    (49.79,66.67) -- (81.6,84.67) ;
% Text Node
\draw (15.98,86.17) node [anchor=north east] [inner sep=0.75pt]   [align=left] {1};
% Text Node
\draw (83.6,86.17) node [anchor=north west][inner sep=0.75pt]   [align=left] {2};
% Text Node
\draw (49.79,22.17) node [anchor=south] [inner sep=0.75pt]   [align=left] {4};
% Text Node
\draw (48.79,64.67) node [anchor=north] [inner sep=0.75pt]   [align=left] {3};
\end{tikzpicture}
} =\delta _{[ 3'3]} \alpha ([ 12] ,[ 23'] ,[ 3'3]) \alpha ([ 23'] ,[ 3'3] ,[ 34]) \alpha ([ 13'] ,[ 3'3] ,[ 34])^{-1}\ket{\tikzset{every picture/.style={line width=0.75pt}} %set default line width to 0.75pt        
\begin{tikzpicture}[x=0.75pt,y=0.75pt,yscale=-1,xscale=1, baseline=(XXXX.south) ]
\path (0,112);\path (105.97916412353516,0);\draw    ($(current bounding box.center)+(0,0.3em)$) node [anchor=south] (XXXX) {};
%Shape: Triangle [id:dp8593774011924047] 
\draw   (49.79,23.67) -- (81.6,84.67) -- (17.98,84.67) -- cycle ;
%Straight Lines [id:da516453776915506] 
\draw    (49.79,66.67) -- (49.79,23.67) ;
%Straight Lines [id:da9005001479262773] 
\draw    (17.98,84.67) -- (49.79,66.67) ;
%Straight Lines [id:da12343627082151754] 
\draw    (49.79,66.67) -- (81.6,84.67) ;
% Text Node
\draw (15.98,86.17) node [anchor=north east] [inner sep=0.75pt]   [align=left] {1};
% Text Node
\draw (83.6,86.17) node [anchor=north west][inner sep=0.75pt]   [align=left] {2};
% Text Node
\draw (49.79,22.17) node [anchor=south] [inner sep=0.75pt]   [align=left] {4};
% Text Node
\draw (48.79,64.67) node [anchor=north] [inner sep=0.75pt]   [align=left] {3'};
\end{tikzpicture}
}.
\label{eq:a-v-def}
\end{equation}
    $A_v$ induces time evolution. Under the assume that $1 < 2 < 3' < 3 < 4$, 
    applying \eqref{eq:flatness} to relevant triangles, we have 
    \begin{equation}
        [13]=[13'][3'3], \quad [23]=[23'][3'3], \quad [3'4]=[3'3][34],
    \end{equation}
    or in other words 
    \begin{equation}
        [13'] = [13][33'], \quad [23'] = [23][33'], \quad [3'4]=[3'3][34].
        \label{eq:gauge-convention}
    \end{equation}
    This can be seen as a gauge transformation: for a given vertex, applying gauge transformation $g$ to it means 
    to replace the group element $g_0$ on an incoming edge by $g g_0$ and to replace the group element $g_0$ on an 
    outgoing edge by $g_0 g^{-1}$. 
    The topological order in \eqref{eq:hamiltonian} is actually the same as a Dijkgraaf-Witten TQFT.
    A TQFT does not have a Hamiltonian in the sense of Legendre transformation, and \eqref{eq:hamiltonian}
    can be seen as a Hamiltonian extension of the TQFT. % ???
\end{itemize}

Note that for a given $\alpha$ we can find an equivalent 3-cocycle
\begin{equation}
    \alpha' = \alpha \cdot \var{\beta},
\end{equation}
and we have 
\[
    \alpha'(g_0, g_1, g_2) = \underbrace{\frac{\beta(g_1, g_2) \beta(g_0, g_1 g_2)}{\beta(g_0 g_1, g_2) \beta(g_0, g_1)}}_\text{3-coboundary} \alpha(g_0, g_1, g_2),
\]
and we will find all the additional $\beta$ factors in \eqref{eq:a-v-def} give rise to a equation in which 
each $\beta$ factor corresponds to a triangle. So by redefinition states and absorb the $\beta$ factors into 
the phase factors, \eqref{eq:a-v-def} remains the same. So in the end, the input data include an equivalent 
class $[\alpha]$, a group $G$ and a lattice.

\section{The topological order in a TQD model}

\eqref{eq:hamiltonian} is similar to the toric code, but since the operators are much more complicated 
and the algebra is not clear, we are unable to write down all ground states directly. 

We can verify that again all $B_f$ operators and $A_v$ operators are commuting projectors. 
Hence projector to the ground state subspace is 
\begin{equation}
    P^0_\Gamma = \prod_f B_f \prod_v A_v.
\end{equation}
The ground state subspace $\mathcal{H}^0_\Gamma$ is invariant under $P^0_\Gamma$. The ground state degeneracy is 
\begin{equation}
    \text{GSD} = \trace P^0_\Gamma.
\end{equation}
Since the lattice is complicated, there is no easy way to evaluate it directly from the Hamiltonian.
Note, however, that we expect \eqref{eq:hamiltonian} to host a topological order, and if so, the GSD 
is independent of the details of triangulation. We consider the following \concept{Pachner moves}:
\begin{equation}
    f_{1} :\tikzset{every picture/.style={line width=0.75pt}} %set default line width to 0.75pt        
\begin{tikzpicture}[x=0.75pt,y=0.75pt,yscale=-1,xscale=1, baseline=(XXXX.south) ]
\path (0,100);\path (100,0);\draw    ($(current bounding box.center)+(0,0.3em)$) node [anchor=south] (XXXX) {};
%Shape: Triangle [id:dp7236680361346721] 
\draw   (13.04,50.87) -- (51.81,24.76) -- (51.81,76.98) -- cycle ;
%Shape: Triangle [id:dp1498239049527148] 
\draw   (90.58,50.87) -- (51.81,24.76) -- (51.81,76.98) -- cycle ;
\end{tikzpicture}
\longrightarrow \tikzset{every picture/.style={line width=0.75pt}} %set default line width to 0.75pt        
\begin{tikzpicture}[x=0.75pt,y=0.75pt,yscale=-1,xscale=1, baseline=(XXXX.south) ]
\path (0,100);\path (100,0);\draw    ($(current bounding box.center)+(0,0.3em)$) node [anchor=south] (XXXX) {};
%Shape: Triangle [id:dp3397092331323992] 
\draw   (50.23,13.12) -- (76.34,51.89) -- (24.12,51.89) -- cycle ;
%Shape: Triangle [id:dp4842761070850914] 
\draw   (50.23,90.66) -- (76.34,51.89) -- (24.12,51.89) -- cycle ;
\end{tikzpicture},
\end{equation}
\begin{equation}
    f_{2} :\tikzset{every picture/.style={line width=0.75pt}} %set default line width to 0.75pt        
\begin{tikzpicture}[x=0.75pt,y=0.75pt,yscale=-1,xscale=1, baseline=(XXXX.south) ]
\path (0,100);\path (100,0);\draw    ($(current bounding box.center)+(0,0.3em)$) node [anchor=south] (XXXX) {};
%Shape: Triangle [id:dp4682276834162664] 
\draw   (55.21,18.41) -- (82.64,71.89) -- (27.78,71.89) -- cycle ;
\end{tikzpicture}
\longrightarrow \tikzset{every picture/.style={line width=0.75pt}} %set default line width to 0.75pt        
\begin{tikzpicture}[x=0.75pt,y=0.75pt,yscale=-1,xscale=1, baseline=(XXXX.south) ]
\path (0,100);\path (100,0);\draw    ($(current bounding box.center)+(0,0.3em)$) node [anchor=south] (XXXX) {};
%Shape: Triangle [id:dp017881963294507974] 
\draw   (54.19,15.5) -- (86,76.5) -- (22.38,76.5) -- cycle ;
%Straight Lines [id:da39267319512323784] 
\draw    (54.19,53.5) -- (54.19,15.5) ;
%Straight Lines [id:da9805352217840213] 
\draw    (22.38,76.5) -- (54.19,53.5) ;
%Straight Lines [id:da3198178775450793] 
\draw    (54.19,53.5) -- (86,76.5) ;
\end{tikzpicture},
\end{equation}
\begin{equation}
    f_{3} :\tikzset{every picture/.style={line width=0.75pt}} %set default line width to 0.75pt        
\begin{tikzpicture}[x=0.75pt,y=0.75pt,yscale=-1,xscale=1, baseline=(XXXX.south) ]
\path (0,100);\path (100,0);\draw    ($(current bounding box.center)+(0,0.3em)$) node [anchor=south] (XXXX) {};
%Shape: Triangle [id:dp18559420993217102] 
\draw   (58.19,20.5) -- (90,81.5) -- (26.38,81.5) -- cycle ;
%Straight Lines [id:da07722772461882488] 
\draw    (58.19,58.5) -- (58.19,20.5) ;
%Straight Lines [id:da027229217227217717] 
\draw    (26.38,81.5) -- (58.19,58.5) ;
%Straight Lines [id:da17520120455579513] 
\draw    (58.19,58.5) -- (90,81.5) ;
\end{tikzpicture}
\longrightarrow \tikzset{every picture/.style={line width=0.75pt}} %set default line width to 0.75pt        
\begin{tikzpicture}[x=0.75pt,y=0.75pt,yscale=-1,xscale=1, baseline=(XXXX.south) ]
\path (0,100);\path (100,0);\draw    ($(current bounding box.center)+(0,0.3em)$) node [anchor=south] (XXXX) {};
%Shape: Triangle [id:dp7867947544020701] 
\draw   (53.21,22.41) -- (80.64,75.89) -- (25.78,75.89) -- cycle ;
\end{tikzpicture}.
\end{equation}
Note that we are working on a quantum theory and we need to find three operators on states to implement 
these moves. A natural choice is 
\begin{equation}
    T_{1}\ket{\tikzset{every picture/.style={line width=0.75pt}} %set default line width to 0.75pt        
\begin{tikzpicture}[x=0.75pt,y=0.75pt,yscale=-1,xscale=1, baseline=(XXXX.south) ]
\path (0,100);\path (100,0);\draw    ($(current bounding box.center)+(0,0.3em)$) node [anchor=south] (XXXX) {};
%Shape: Triangle [id:dp1804347637670367] 
\draw   (13.04,50.87) -- (51.81,24.76) -- (51.81,76.98) -- cycle ;
%Shape: Triangle [id:dp06220783033656718] 
\draw   (90.58,50.87) -- (51.81,24.76) -- (51.81,76.98) -- cycle ;
% Text Node
\draw (4,31.5) node [anchor=north west][inner sep=0.75pt]   [align=left] {1};
% Text Node
\draw (46,76.5) node [anchor=north west][inner sep=0.75pt]   [align=left] {2};
% Text Node
\draw (81,26.5) node [anchor=north west][inner sep=0.75pt]   [align=left] {3};
% Text Node
\draw (44,0.5) node [anchor=north west][inner sep=0.75pt]   [align=left] {4};
\end{tikzpicture}
} =\sum _{[ 13] \in G} \alpha ([ 12] ,[ 23] ,[ 34])\ket{\tikzset{every picture/.style={line width=0.75pt}} %set default line width to 0.75pt        
\begin{tikzpicture}[x=0.75pt,y=0.75pt,yscale=-1,xscale=1, baseline=(XXXX.south) ]
\path (0,100);\path (100,0);\draw    ($(current bounding box.center)+(0,0.3em)$) node [anchor=south] (XXXX) {};
%Shape: Triangle [id:dp16302403077155359] 
\draw   (50.23,13.12) -- (76.34,51.89) -- (24.12,51.89) -- cycle ;
%Shape: Triangle [id:dp8815634085946111] 
\draw   (50.23,90.66) -- (76.34,51.89) -- (24.12,51.89) -- cycle ;
% Text Node
\draw (8,35.5) node [anchor=north west][inner sep=0.75pt]   [align=left] {1};
% Text Node
\draw (44,62.5) node [anchor=north west][inner sep=0.75pt]   [align=left] {2};
% Text Node
\draw (81,38.5) node [anchor=north west][inner sep=0.75pt]   [align=left] {3};
% Text Node
\draw (39,-3.5) node [anchor=north west][inner sep=0.75pt]   [align=left] {4};
\end{tikzpicture}
}
\end{equation}
\begin{equation}
    T_{2}\ket{\tikzset{every picture/.style={line width=0.75pt}} %set default line width to 0.75pt        
\begin{tikzpicture}[x=0.75pt,y=0.75pt,yscale=-1,xscale=1, baseline=(XXXX.south) ]
\path (0,100);\path (100,0);\draw    ($(current bounding box.center)+(0,0.3em)$) node [anchor=south] (XXXX) {};
%Shape: Triangle [id:dp38897603533023006] 
\draw   (51.57,21.52) -- (79,75) -- (24.14,75) -- cycle ;
% Text Node
\draw (10,65.5) node [anchor=north west][inner sep=0.75pt]   [align=left] {1};
% Text Node
\draw (80,65.5) node [anchor=north west][inner sep=0.75pt]   [align=left] {2};
% Text Node
\draw (45,1.5) node [anchor=north west][inner sep=0.75pt]   [align=left] {3};
\end{tikzpicture}
} =\sum _{[ g1] ,[ g2] ,[ g3] \in G} \alpha ([ g1] ,[ 12] ,[ 23])\ket{\tikzset{every picture/.style={line width=0.75pt}} %set default line width to 0.75pt        
\begin{tikzpicture}[x=0.75pt,y=0.75pt,yscale=-1,xscale=1, baseline=(XXXX.south) ]
\path (0,100);\path (100,0);\draw    ($(current bounding box.center)+(0,0.3em)$) node [anchor=south] (XXXX) {};
%Shape: Triangle [id:dp3690877216895365] 
\draw   (51.57,21.52) -- (79,75) -- (24.14,75) -- cycle ;
%Straight Lines [id:da6398205080389376] 
\draw    (51.57,55.26) -- (51.57,22.26) ;
%Straight Lines [id:da8605324297784194] 
\draw    (24.14,75) -- (51.57,55.26) ;
%Straight Lines [id:da39965404281810923] 
\draw    (51.57,55.26) -- (79,75) ;
% Text Node
\draw (10,65.5) node [anchor=north west][inner sep=0.75pt]   [align=left] {1};
% Text Node
\draw (80,65.5) node [anchor=north west][inner sep=0.75pt]   [align=left] {2};
% Text Node
\draw (45,1.5) node [anchor=north west][inner sep=0.75pt]   [align=left] {3};
% Text Node
\draw (44,52.7) node [anchor=north west][inner sep=0.75pt]    {$g$};
\end{tikzpicture}
},
\end{equation}
\begin{equation}
    T_{3}\ket{\tikzset{every picture/.style={line width=0.75pt}} %set default line width to 0.75pt        
\begin{tikzpicture}[x=0.75pt,y=0.75pt,yscale=-1,xscale=1, baseline=(XXXX.south) ]
\path (0,100);\path (100,0);\draw    ($(current bounding box.center)+(0,0.3em)$) node [anchor=south] (XXXX) {};
%Shape: Triangle [id:dp6507395375068188] 
\draw   (51.57,21.52) -- (79,75) -- (24.14,75) -- cycle ;
%Straight Lines [id:da3986798837927681] 
\draw    (51.57,55.26) -- (51.57,22.26) ;
%Straight Lines [id:da33432836321267856] 
\draw    (24.14,75) -- (51.57,55.26) ;
%Straight Lines [id:da057398586142625385] 
\draw    (51.57,55.26) -- (79,75) ;
% Text Node
\draw (10,65.5) node [anchor=north west][inner sep=0.75pt]   [align=left] {1};
% Text Node
\draw (80,65.5) node [anchor=north west][inner sep=0.75pt]   [align=left] {2};
% Text Node
\draw (45,1.5) node [anchor=north west][inner sep=0.75pt]   [align=left] {3};
% Text Node
\draw (44,53.7) node [anchor=north west][inner sep=0.75pt]    {$4$};
\end{tikzpicture}
} =\alpha ([ 12] ,[ 23] ,[ 34])\ket{\tikzset{every picture/.style={line width=0.75pt}} %set default line width to 0.75pt        
\begin{tikzpicture}[x=0.75pt,y=0.75pt,yscale=-1,xscale=1, baseline=(XXXX.south) ]
\path (0,100);\path (100,0);\draw    ($(current bounding box.center)+(0,0.3em)$) node [anchor=south] (XXXX) {};
%Shape: Triangle [id:dp48691023742141204] 
\draw   (51.57,21.52) -- (79,75) -- (24.14,75) -- cycle ;
% Text Node
\draw (10,65.5) node [anchor=north west][inner sep=0.75pt]   [align=left] {1};
% Text Node
\draw (80,65.5) node [anchor=north west][inner sep=0.75pt]   [align=left] {2};
% Text Node
\draw (45,1.5) node [anchor=north west][inner sep=0.75pt]   [align=left] {3};
\end{tikzpicture}
}.
\end{equation}
It can be proved that if $\ket*{\phi} \in \mathcal{H}^0_\Gamma$, we also 
have $T \ket*{\phi} \in \mathcal{H}^0_{\Gamma'}$. It can also be proved that $T$'s are unitary.

Now we consider the example of a torus, which may be considered as a square with its left and right 
bonds identified, and its upper and lower bonds identified. We consider the following triangulation:


\tikzset{every picture/.style={line width=0.75pt}} %set default line width to 0.75pt        

\begin{tikzpicture}[x=0.75pt,y=0.75pt,yscale=-1,xscale=1]
%uncomment if require: \path (0,300); %set diagram left start at 0, and has height of 300

%Shape: Ellipse [id:dp10888654123535657] 
\draw   (91,139) .. controls (91,113.59) and (129.73,93) .. (177.5,93) .. controls (225.27,93) and (264,113.59) .. (264,139) .. controls (264,164.41) and (225.27,185) .. (177.5,185) .. controls (129.73,185) and (91,164.41) .. (91,139) -- cycle ;
%Shape: Arc [id:dp9067895132467914] 
\draw  [draw opacity=0] (235.63,133.04) .. controls (235.69,133.42) and (235.71,133.81) .. (235.71,134.19) .. controls (235.67,146.12) and (209.35,155.7) .. (176.93,155.59) .. controls (144.5,155.48) and (118.25,145.73) .. (118.29,133.81) .. controls (118.29,133.42) and (118.32,133.05) .. (118.37,132.67) -- (177,134) -- cycle ; \draw   (235.63,133.04) .. controls (235.69,133.42) and (235.71,133.81) .. (235.71,134.19) .. controls (235.67,146.12) and (209.35,155.7) .. (176.93,155.59) .. controls (144.5,155.48) and (118.25,145.73) .. (118.29,133.81) .. controls (118.29,133.42) and (118.32,133.05) .. (118.37,132.67) ;
%Shape: Arc [id:dp7282595977083575] 
\draw  [draw opacity=0] (230.99,142.4) .. controls (224.37,131.23) and (202.8,123.1) .. (177.25,123.18) .. controls (151.77,123.27) and (130.31,131.5) .. (123.69,142.66) -- (177.34,149.79) -- cycle ; \draw   (230.99,142.4) .. controls (224.37,131.23) and (202.8,123.1) .. (177.25,123.18) .. controls (151.77,123.27) and (130.31,131.5) .. (123.69,142.66) ;

%Shape: Arc [id:dp4865327224674516] 
\draw  [draw opacity=0] (167.5,184) .. controls (167.5,184) and (167.5,184) .. (167.5,184) .. controls (162.25,184) and (158,177.73) .. (158,170) .. controls (158,162.27) and (162.25,156) .. (167.5,156) -- (167.5,170) -- cycle ; \draw  [color={rgb, 255:red, 155; green, 155; blue, 155 }  ,draw opacity=1 ] (167.5,184) .. controls (167.5,184) and (167.5,184) .. (167.5,184) .. controls (162.25,184) and (158,177.73) .. (158,170) .. controls (158,162.27) and (162.25,156) .. (167.5,156) ;
%Shape: Arc [id:dp049149640047466026] 
\draw  [draw opacity=0][dash pattern={on 4.5pt off 4.5pt}] (167.5,184) .. controls (167.5,184) and (167.5,184) .. (167.5,184) .. controls (172.75,184) and (177,177.73) .. (177,170) .. controls (177,162.27) and (172.75,156) .. (167.5,156) -- (167.5,170) -- cycle ; \draw  [color={rgb, 255:red, 155; green, 155; blue, 155 }  ,draw opacity=1 ][dash pattern={on 4.5pt off 4.5pt}] (167.5,184) .. controls (167.5,184) and (167.5,184) .. (167.5,184) .. controls (172.75,184) and (177,177.73) .. (177,170) .. controls (177,162.27) and (172.75,156) .. (167.5,156) ;

%Curve Lines [id:da7562826546508461] 
\draw [color={rgb, 255:red, 155; green, 155; blue, 155 }  ,draw opacity=1 ]   (179,103.5) .. controls (136.5,97) and (39.5,148.5) .. (167.5,170) .. controls (308,180) and (241,106) .. (179,103.5) -- cycle ;




\end{tikzpicture}
where the boundary condition requires 
\[
    [13]=[24] = g, \quad [12] = [34] = h.
\]
The flatness conditions requires 
\[
    [14] = gh = hg, 
\]
so in $\mathcal{H}^0_\mathbb{T}$ we have $gh = hg$. 

% TODO: details on the constraint imposed by A

Now we find \eqref{eq:cocycle-1} and \eqref{eq:cocycle-2} are actually two Pachner moves. 

\begin{equation}
    A^x \ket*{g, h} = 
\end{equation}

\begin{equation}
    \begin{aligned}
        \text{GSD} &= \trace \left( \frac{1}{\abs*{G}} \sum_x A^x \right) \\
        &= \frac{1}{\abs*{G}} \sum_{g, h} \sum_x \delta_{gh, hg} \expval*{A^x}{gh} \\
        &= \frac{1}{\abs*{G}} \alpha(g, hx^{-1}, x) \alpha
    \end{aligned}
\end{equation}

\end{document}