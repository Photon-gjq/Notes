\documentclass[hyperref, a4paper]{article}

\usepackage{geometry}
\usepackage{float}
\usepackage{titling}
\usepackage{titlesec}
% No longer needed, since we will use enumitem package
% \usepackage{paralist}
\usepackage{enumitem}
\usepackage{footnote}
\usepackage{enumerate}
\usepackage{amsmath, amssymb, amsthm}
\usepackage{mathtools}
\usepackage{bbm}
\usepackage{cite}
\usepackage{graphicx}
\usepackage{subfigure}
\usepackage{physics}
\usepackage{tensor}
\usepackage{siunitx}
\usepackage{booktabs}
\usepackage[version=4]{mhchem}
\usepackage{tikz}
\usepackage{xcolor}
\usepackage{listings}
\usepackage{autobreak}
\usepackage[ruled, vlined, linesnumbered]{algorithm2e}
\usepackage{xr-hyper}
\usepackage[colorlinks,unicode]{hyperref} % , linkcolor=black, anchorcolor=black, citecolor=black, urlcolor=black, filecolor=black
\usepackage{prettyref}

% Page style
\geometry{left=3.18cm,right=3.18cm,top=2.54cm,bottom=2.54cm}
\titlespacing{\paragraph}{0pt}{1pt}{10pt}[20pt]
\setlength{\droptitle}{-5em}
\preauthor{\vspace{-10pt}\begin{center}}
\postauthor{\par\end{center}}

% More compact lists 
\setlist[itemize]{itemindent=17pt, leftmargin=1pt}

% Math operators
\DeclareMathOperator{\timeorder}{T}
\DeclareMathOperator{\diag}{diag}
\DeclareMathOperator{\legpoly}{P}
\DeclareMathOperator{\primevalue}{P}
\DeclareMathOperator{\sgn}{sgn}
\newcommand*{\ii}{\mathrm{i}}
\newcommand*{\ee}{\mathrm{e}}
\newcommand*{\const}{\mathrm{const}}
\newcommand*{\suchthat}{\quad \text{s.t.} \quad}
\newcommand*{\argmin}{\arg\min}
\newcommand*{\argmax}{\arg\max}
\newcommand*{\normalorder}[1]{: #1 :}
\newcommand*{\pair}[1]{\langle #1 \rangle}
\newcommand*{\fd}[1]{\mathcal{D} #1}
\DeclareMathOperator{\bigO}{\mathcal{O}}

% TikZ setting
\usetikzlibrary{arrows,shapes,positioning}
\usetikzlibrary{arrows.meta}
\usetikzlibrary{decorations.markings}
\tikzstyle arrowstyle=[scale=1]
\tikzstyle directed=[postaction={decorate,decoration={markings,
    mark=at position .5 with {\arrow[arrowstyle]{stealth}}}}]
\tikzstyle ray=[directed, thick]
\tikzstyle dot=[anchor=base,fill,circle,inner sep=1pt]

% Algorithm setting
% Julia-style code
\SetKwIF{If}{ElseIf}{Else}{if}{}{elseif}{else}{end}
\SetKwFor{For}{for}{}{end}
\SetKwFor{While}{while}{}{end}
\SetKwProg{Function}{function}{}{end}
\SetArgSty{textnormal}

\newcommand*{\concept}[1]{{\textbf{#1}}}

% Embedded codes
\lstset{basicstyle=\ttfamily,
  showstringspaces=false,
  commentstyle=\color{gray},
  keywordstyle=\color{blue}
}

\title{SPT by Prof. Yang Qi}
\author{Jinyuan Wu}

\begin{document}

\maketitle

This article is a lecture note of Prof. Yang Qi's lecture on the topological states of matter seminar on October 8, 2021.

\section{One dimensional AKLT chain}

\subsection{The AKLT state and the AKLT model}

The 1D AKLT chain is first proposed by Affleck, Kennedy, Lieb and Tasaki in \cite{Affleck_1987}.
The Haldane conjecture states that an integer spin chain is gapped while a half-integer spin chain is gapless.
Since symmetry breaking is impossible in an integer spin chain, it was traditionally regarded as a trivial ferromagnetic system with gapped excitations (magnon).
This is definitely true in the bulk - but the possibility that exotic boundary modes exist cannot be ruled out.

Consider a chain, each site of which has a spin-1 degree of freedom on it. 
The \concept{AKLT state} is a wave function of the spin chain defined on the following way.
To make things more understandable we decompose one spin-1 degree of freedom into two spin-$1/2$ degrees of freedom, i.e. to put two spin-$1/2$ degrees of freedom on one site, and then use a projector to eliminate redundant Hilbert subspaces. 
We know that 
\[
    \frac{1}{2} \otimes \frac{1}{2} = 0 \otimes 1,
\]
so we need to project the state into \emph{spin triplet subspaces} and throw away the spin singlet states.
Suppose we decompose the spin-1 degree of freedom on site $i$ into two spin-$1/2$ degrees of freedom labeled as A and B, and the projector is visualized as 
\begin{equation}
    \begin{gathered}
        \begin{tikzpicture}[x=0.75pt,y=0.75pt,yscale=-1,xscale=1]
            %uncomment if require: \path (0,300); %set diagram left start at 0, and has height of 300
            
            %Straight Lines [id:da9495322085331372] 
            \draw    (88.5,105) -- (119.15,105) ;
            \draw [shift={(121.5,105)}, rotate = 0] [color={rgb, 255:red, 0; green, 0; blue, 0 }  ][line width=0.75]      (0, 0) circle [x radius= 3.35, y radius= 3.35]   ;
            %Straight Lines [id:da018779805704167263] 
            \draw    (162.85,105) -- (193.5,105) ;
            \draw [shift={(160.5,105)}, rotate = 0] [color={rgb, 255:red, 0; green, 0; blue, 0 }  ][line width=0.75]      (0, 0) circle [x radius= 3.35, y radius= 3.35]   ;
            %Shape: Ellipse [id:dp6737778885888006] 
            \draw   (106,104) .. controls (106,92.95) and (121.67,84) .. (141,84) .. controls (160.33,84) and (176,92.95) .. (176,104) .. controls (176,115.05) and (160.33,124) .. (141,124) .. controls (121.67,124) and (106,115.05) .. (106,104) -- cycle ;
            
            % Text Node
            \draw (136,128.4) node [anchor=north west][inner sep=0.75pt]    {$i$};
            % Text Node
            \draw (114,86) node [anchor=north west][inner sep=0.75pt]   [align=left] {A};
            % Text Node
            \draw (154,86) node [anchor=north west][inner sep=0.75pt]   [align=left] {B};
            \end{tikzpicture} 
    \end{gathered} = P = \dyad{+}{\uparrow \uparrow} + \dyad{-}{\downarrow \downarrow} + \frac{1}{\sqrt{2}} \ket*{0} (\bra*{\uparrow \downarrow} + \bra*{\downarrow \uparrow}).
    \label{eq:aklt-projector}
\end{equation}
Now we let $i \text{B}$ and $i+1, \text{A}$ get entangled in the following ``valence bond''%
\footnote{
    When two spins are in a singlet state, we call it \emph{valence bond state} because the spin part of the two-electron wavefunction in a valence bond is just a singlet state.
    The two electrons come close in the orbital space, so the spin part of the wave function have to be a singlet to make the whole wave function antisymmetric. 
    If two spins in a singlet states can be directly traced back to two electrons, then it is highly likely that the two electrons are indeed in a valence bond.
}%
way:
\begin{equation}
    \begin{gathered}
        \begin{tikzpicture}[x=0.75pt,y=0.75pt,yscale=-1,xscale=1]
            %uncomment if require: \path (0,300); %set diagram left start at 0, and has height of 300
            
            %Straight Lines [id:da008217126970560651] 
            \draw    (182.85,125) -- (225.65,125) ;
            \draw [shift={(228,125)}, rotate = 0] [color={rgb, 255:red, 0; green, 0; blue, 0 }  ][line width=0.75]      (0, 0) circle [x radius= 3.35, y radius= 3.35]   ;
            \draw [shift={(180.5,125)}, rotate = 0] [color={rgb, 255:red, 0; green, 0; blue, 0 }  ][line width=0.75]      (0, 0) circle [x radius= 3.35, y radius= 3.35]   ;
            
            % Text Node
            \draw (176.5,125) node [anchor=east] [inner sep=0.75pt]    {$i,\text{B}$};
            % Text Node
            \draw (232,125) node [anchor=west] [inner sep=0.75pt]    {$i+1,\text{A}$};
            \end{tikzpicture}
    \end{gathered}
     = \frac{1}{\sqrt{2}} (\ket*{\uparrow \downarrow} - \ket*{\downarrow \uparrow} ),
\end{equation}
and then apply the projector \eqref{eq:aklt-projector}, and now we get an explicit definition of the AKLT state as  
\begin{equation}
    \ket*{\Psi} \coloneqq \prod_{i} P_i \ket*{\Psi_0}, \quad \ket*{\Psi_0} \coloneqq \prod_i \frac{1}{\sqrt{2}} (\ket*{\uparrow}_{i \text{B}} \ket*{\downarrow}_{i+1, \text{A}} - \ket*{\downarrow}_{i \text{B}} \ket*{\uparrow}_{i+1, \text{A}}),
    \label{eq:aklt-state}
\end{equation}
or in a visualized way as 
\begin{equation}
    \begin{gathered}
        \begin{tikzpicture}[x=0.75pt,y=0.75pt,yscale=-1,xscale=1]
            %uncomment if require: \path (0,300); %set diagram left start at 0, and has height of 300
            
            %Straight Lines [id:da6245821686017994] 
            \draw    (108.5,125) -- (139.15,125) ;
            \draw [shift={(141.5,125)}, rotate = 0] [color={rgb, 255:red, 0; green, 0; blue, 0 }  ][line width=0.75]      (0, 0) circle [x radius= 3.35, y radius= 3.35]   ;
            %Straight Lines [id:da9728888558043336] 
            \draw    (182.85,125) -- (213.5,125) ;
            \draw [shift={(180.5,125)}, rotate = 0] [color={rgb, 255:red, 0; green, 0; blue, 0 }  ][line width=0.75]      (0, 0) circle [x radius= 3.35, y radius= 3.35]   ;
            %Shape: Ellipse [id:dp40247130481984605] 
            \draw   (126,124) .. controls (126,112.95) and (141.67,104) .. (161,104) .. controls (180.33,104) and (196,112.95) .. (196,124) .. controls (196,135.05) and (180.33,144) .. (161,144) .. controls (141.67,144) and (126,135.05) .. (126,124) -- cycle ;
            %Straight Lines [id:da5535258332135746] 
            \draw    (213.5,125) -- (244.15,125) ;
            \draw [shift={(246.5,125)}, rotate = 0] [color={rgb, 255:red, 0; green, 0; blue, 0 }  ][line width=0.75]      (0, 0) circle [x radius= 3.35, y radius= 3.35]   ;
            %Straight Lines [id:da2347934449839162] 
            \draw    (286.85,125) -- (317.5,125) ;
            \draw [shift={(284.5,125)}, rotate = 0] [color={rgb, 255:red, 0; green, 0; blue, 0 }  ][line width=0.75]      (0, 0) circle [x radius= 3.35, y radius= 3.35]   ;
            %Shape: Ellipse [id:dp5388082527929081] 
            \draw   (230,124) .. controls (230,112.95) and (245.67,104) .. (265,104) .. controls (284.33,104) and (300,112.95) .. (300,124) .. controls (300,135.05) and (284.33,144) .. (265,144) .. controls (245.67,144) and (230,135.05) .. (230,124) -- cycle ;
            %Straight Lines [id:da7687316396415951] 
            \draw    (317.5,125) -- (348.15,125) ;
            \draw [shift={(350.5,125)}, rotate = 0] [color={rgb, 255:red, 0; green, 0; blue, 0 }  ][line width=0.75]      (0, 0) circle [x radius= 3.35, y radius= 3.35]   ;
            %Straight Lines [id:da5902566541675356] 
            \draw    (390.85,125) -- (421.5,125) ;
            \draw [shift={(388.5,125)}, rotate = 0] [color={rgb, 255:red, 0; green, 0; blue, 0 }  ][line width=0.75]      (0, 0) circle [x radius= 3.35, y radius= 3.35]   ;
            %Shape: Ellipse [id:dp7785523629855451] 
            \draw   (334,124) .. controls (334,112.95) and (349.67,104) .. (369,104) .. controls (388.33,104) and (404,112.95) .. (404,124) .. controls (404,135.05) and (388.33,144) .. (369,144) .. controls (349.67,144) and (334,135.05) .. (334,124) -- cycle ;
            
            % Text Node
            \draw (106.5,125) node [anchor=east] [inner sep=0.75pt]    {$\cdots $};
            % Text Node
            \draw (423.5,125) node [anchor=west] [inner sep=0.75pt]    {$\cdots $};
            \end{tikzpicture}            
    \end{gathered} \eqqcolon \ket*{\Psi}.
\end{equation}
Though \eqref{eq:aklt-state} is constructed with $2N$ $1/2$-spins, after the projection it is already a wave function of a spin-1 model with $N$ sites.

Now we try to find a Hamiltonian with \eqref{eq:aklt-state} as its ground state. %
\footnote{
    As is often the case when we investigate topological states of matter,  we often write down a wave function \emph{first} and then try to find a model with the wave function as the ground state, or even forget about the model.
}%
Note that the sum of two nearest spin-1 degrees of freedom cannot be 2, because in doing so, we require $S^z_i = S^z_{i+1} = 1$, which, in turn, require that the two $1/2$-spins in $S_i$ are all upward and so are the two $1/2$-spins in $S_{i+1}$, which is not possible since the $1/2$-spin on $i, \text{B}$ and the $1/2$-spin on $i+1, \text{A}$ form a spin singlet. 
Due to the spin rotational symmetry, suppose 
\[
    \vb*{S} = \vb*{S}_i + \vb*{S}_{i+1},
\]
then $S \neq 2$.
So we can give a state where $S = 2$ an energy penalty, in an attempt to make \eqref{eq:aklt-state} a ground state.
We may consider a Hamiltonian in the following form: 
\begin{equation}
    H = \sum_{i} P_2 (\vb*{S}_i + \vb*{S}_{i+1}) , 
    \label{eq:aklt-model-original}
\end{equation}
where $P_2$ is an \emph{operator function} which returns $1$ when $S = 2$ and returns $0$ otherwise.
Note that all spins are spin-$1$ - or otherwise the $(\vb*{S}_i \cdot \vb*{S}_{i+1})^2$ terms in the following discussion make no sense.

\begin{table}[H]
    \caption{The eigenstates and eigenvalues of $(\vb*{S}_i + \vb*{S}_{i+1})^2$ and $P_2$}
    \label{tab:s-2-eigs}
    \centering
    \label{table1}
    \begin{tabular}[c]{ccc}
        \toprule
        $s$ & $(\vb*{S}_i + \vb*{S}_{i+1})^2 \eqqcolon X$ & $P_2$ \\ 
        \midrule
        0 & 1 & 0 \\
        1 & 2 & 0 \\
        2 & 6 & 1 \\
        \bottomrule
    \end{tabular}
\end{table} 

We should evaluate the explicit form of $P_2$. Note that the spectrum of $\vb*{S}_i + \vb*{S}_{i+1}$ carries three irreducible representation of $SU(2)$, namely $s = 0, 1, 2$, and since 
\[
    (\vb*{S}_i + \vb*{S}_{i+1})^2 = s(s+1)
\]
we have \prettyref{tab:s-2-eigs}, and by curve fitting we find 
\[
    P_2 = \frac{1}{24} X (X-2).
\]
So the Hamiltonian \eqref{eq:aklt-model-original} is 
\begin{equation}
    H = \sum_i \frac{1}{24} (\vb*{S}_i + \vb*{S}_{i+1})^2 ((\vb*{S}_i + \vb*{S}_{i+1})^2 -2).
\end{equation}
Using the formula
\[
    (\vb*{S}_i + \vb*{S}_{i+1})^2 = 4 + 2 \vb*{S}_i \cdot \vb*{S}_{i+1},
\]
we have 
\[
    \begin{aligned}
        H &= \sum_i \frac{1}{24} (\vb*{S}_i + \vb*{S}_{i+1})^2 ((\vb*{S}_i + \vb*{S}_{i+1})^2 -2) \\
        &= \sum_i \frac{1}{24} (4 + 2 \vb*{S}_i \cdot \vb*{S}_{i+1}) (2 + 2 \vb*{S}_i \cdot \vb*{S}_{i+1}) \\
        &= \sum_i \left( \frac{1}{2} \vb*{S}_i \cdot \vb*{S}_{i+1} + \frac{1}{6} (\vb*{S}_i \cdot \vb*{S}_{i+1})^2 + \frac{1}{3} \right),
    \end{aligned}
\]
and since \eqref{eq:aklt-model-original} is just a penalty function we are free to multiply a coefficient onto it and hence we have a much prettier Hamiltonian
\begin{equation}
    H = \sum_i \left( \vb*{S}_{i} \cdot \vb*{S}_{i+1} + \frac{1}{3} (\vb*{S}_i \cdot \vb*{S}_{i+1})^2 \right),
    \label{eq:aklt-model}
\end{equation}
where we ignore the constant term. 

We can see that \eqref{eq:aklt-state} is an eigenstate of \eqref{eq:aklt-model}, because the projector \eqref{eq:aklt-projector} commutes with $(\vb*{S}_i + \vb*{S}_{i+1})^2$, 

\subsection{The edge states}

The spin-$1/2$ degrees of freedom can be viewed as fractionalized degrees of freedom in \eqref{eq:aklt-model}, since they do not appear in the basic degrees of freedom in \eqref{eq:aklt-model} but they do help to understand what is going on in its ground state.

Note that \eqref{eq:aklt-model} differs with the Heisenberg model with only one $\sum_i (\vb*{S}_i \cdot \vb*{S}_{i+1})^2$ term. 
It has been numerically demonstrated that the model
\begin{equation}
    H = \sum_i (\vb*{S}_{i} \cdot \vb*{S}_{i+1} + \lambda (\vb*{S}_i \cdot \vb*{S}_{i+1})^2)
\end{equation}
has no quantum phase transition as $\lambda$ goes from $1/3$ to $0$, and therefore the Heisenberg model is somehow in one quantum phase with \eqref{eq:aklt-model}. 

\subsection{Symmetry of the AKLT model and SPT}

The AKLT state is actually an example of \concept{symmetry protected topological phases} - or one may call them symmetry protected \emph{trivial} phases, since there are no anyons in them as is the case in \emph{intrinsic} topological phases. 
The occurrence of gapless boundary states is quite similar to the Kramers degeneracy.
In the case of the AKLT model, the symmetry involved is $SO(3)$ (not $SU(2)$ - since we are talking about an integer spin system) and the time reversal symmetry $\mathbb{Z}_2^T$.

We consider a subgroup of $SO(3)$:
\begin{equation}
    D_2 \simeq \mathbb{Z}_2 \times \mathbb{Z}_2.
\end{equation}
The group can be generated by two elements satisfying
\begin{equation}
    x^2 = 1, \quad y^2 = 1, \quad xy = yx,
\end{equation}
where $x$ and $y$ correspond to \SI{180}{\degree} rotations around the $x$ and $y$ axises, respectively.
The spin-$1/2$ spins do not carry an ordinary (or \emph{linear}) representation of $D_2$, but rather a \emph{projective} one: for a spin-$1/2$ degree of freedom we have 
\begin{equation}
    x^2 = y^2 = -1, \quad xy = - yx.
\end{equation}

\section{Classification of SPT states and group cohomology}

\begin{itemize}
    \item All SPT states can be characterized by topological field theories.
    \item Topological field theories can be classified by group cohomology.
    \item Group cohomology can also be used to classified projective representations.
\end{itemize}

We consider the general theory of projective representations.
\begin{equation}
    \omega_2(g, h) \omega_2(gh, k) = \prescript{g}{}{\omega_2(h, k)} \omega_2(g, hk)
\end{equation}
\begin{equation}
    H^2(\mathbb{Z}_2 \times \mathbb{Z}_2, U(1)) = \mathbb{Z}_2
\end{equation}

\section{Group cohomology}

\bibliographystyle{plain}
\bibliography{spin-chain} 

\end{document}