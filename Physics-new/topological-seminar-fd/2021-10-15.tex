\documentclass[hyperref, a4paper]{article}

\usepackage{geometry}
\usepackage{float}
\usepackage{titling}
\usepackage{titlesec}
% No longer needed, since we will use enumitem package
% \usepackage{paralist}
\usepackage{enumitem}
\usepackage{footnote}
\usepackage{enumerate}
\usepackage{amsmath, amssymb, amsthm}
\usepackage{mathtools}
\usepackage{bbm}
\usepackage{cite}
\usepackage{graphicx}
\usepackage{subfigure}
\usepackage{physics}
\usepackage{tensor}
\usepackage{siunitx}
\usepackage{booktabs}
\usepackage[version=4]{mhchem}
\usepackage{tikz}
\usepackage{xcolor}
\usepackage{listings}
\usepackage{autobreak}
\usepackage[ruled, vlined, linesnumbered]{algorithm2e}
\usepackage{xr-hyper}
\usepackage[colorlinks,unicode]{hyperref} % , linkcolor=black, anchorcolor=black, citecolor=black, urlcolor=black, filecolor=black
\usepackage{prettyref}

% Page style
\geometry{left=3.18cm,right=3.18cm,top=2.54cm,bottom=2.54cm}
\titlespacing{\paragraph}{0pt}{1pt}{10pt}[20pt]
\setlength{\droptitle}{-5em}
\preauthor{\vspace{-10pt}\begin{center}}
\postauthor{\par\end{center}}

% More compact lists 
\setlist[itemize]{itemindent=17pt, leftmargin=1pt}

% Math operators
\DeclareMathOperator{\timeorder}{T}
\DeclareMathOperator{\diag}{diag}
\DeclareMathOperator{\legpoly}{P}
\DeclareMathOperator{\primevalue}{P}
\DeclareMathOperator{\sgn}{sgn}
\newcommand*{\ii}{\mathrm{i}}
\newcommand*{\ee}{\mathrm{e}}
\newcommand*{\const}{\mathrm{const}}
\newcommand*{\suchthat}{\quad \text{s.t.} \quad}
\newcommand*{\argmin}{\arg\min}
\newcommand*{\argmax}{\arg\max}
\newcommand*{\normalorder}[1]{: #1 :}
\newcommand*{\pair}[1]{\langle #1 \rangle}
\newcommand*{\fd}[1]{\mathcal{D} #1}
\DeclareMathOperator{\bigO}{\mathcal{O}}

% TikZ setting
\usetikzlibrary{arrows,shapes,positioning}
\usetikzlibrary{arrows.meta}
\usetikzlibrary{decorations.markings}
\tikzstyle arrowstyle=[scale=1]
\tikzstyle directed=[postaction={decorate,decoration={markings,
    mark=at position .5 with {\arrow[arrowstyle]{stealth}}}}]
\tikzstyle ray=[directed, thick]
\tikzstyle dot=[anchor=base,fill,circle,inner sep=1pt]

% Algorithm setting
% Julia-style code
\SetKwIF{If}{ElseIf}{Else}{if}{}{elseif}{else}{end}
\SetKwFor{For}{for}{}{end}
\SetKwFor{While}{while}{}{end}
\SetKwProg{Function}{function}{}{end}
\SetArgSty{textnormal}

\newcommand*{\concept}[1]{{\textbf{#1}}}

% Embedded codes
\lstset{basicstyle=\ttfamily,
  showstringspaces=false,
  commentstyle=\color{gray},
  keywordstyle=\color{blue}
}

\title{Topological order by Prof. Yidun Wan}
\author{Jinyuan Wu}

\begin{document}

\maketitle

This article is a lecture not of Prof. Yidun Wang's lecture on the topological states of matter seminar on October 15, 2021.
It has been revised after the lecture to complete uncovered yet important information, details in computation etc.

\section{Basic facts about (intrinsic) topological orders}

In \href{./2021-10-8}{this document} SPTs are discussed. 
They show an order which cannot be described by Landau-Ginzburg paradigm, as all the phases share the same symmetry. 
In this article we will discuss \emph{intrinsic} topological orders.

We usually discuss topological orders in 2D where we have braiding structure. 
This is not necessary as topological orders can also exist in higher dimensions, using concepts in category theories.
Topological orders are gapped in the sense that the energy difference between ground states and first excited states is independent of the system size in the thermodynamic limit.
The low energy states of a topological ordered system are usually \emph{anyons}, 
and they have a ground state degeneracy depending only on the type of anyons and the topological features of the lattice.%
\footnote{
    Note that the lattice may not be a compact manifold. Topological orders with boundary conditions have been investigated.
}%
For examples, an Abelian topological ordered state has ground state degeneracy $K^g$, where $g$ is the genus of the lattice and $K$ the number of anyon types.

An example of topological order is $\nu = 1 / m$ FQHE (fractional quantum Hall effect). There are $m$ anyon species which we label as $1, a_1, \ldots, a_{m-1}$, where 1 denotes the ``trivial'' anyon.
The statistics of anyons are:
\begin{itemize}
    \item Exchange two $j$-type anyons and we get a phase factor $\ee^{\ii \pi \frac{j^2}{m}}$.
    \item Exchange a $j$-type anyon and a $k$-type anyon and we get a phase factor $\ee^{\ii \pi \frac{jk}{m}}$.
\end{itemize}
The phase factors can be introduced in a topological quantum field theory (which is an effective theory for the anyons only, ignoring the details about how they are generated), via the A-B effect brought by the gauge field.

\section{The toric code model}

A \concept{$\mathbb{Z}_2$ toric code} is defined by the following data:
\begin{itemize}
    \item A square lattice with opposite sides identified, or in other words the lattice is on a torus.
    \item We put a spin-$1/2$ degree of freedom on each bond of the lattice.
    \item The Hamiltonian is 
    \begin{equation}
        H = - \sum_{s} A_{s} - \sum_{p} B_p,
        \label{eq:toric-code}
    \end{equation}
    where $s$ goes through all crosses (i.e. four bonds attached to the same site) in the lattice and $p$ goes through all plaquettes in the lattice, and 
    \begin{equation}
        A_s = \prod_{b \in s} \sigma^x_s, \quad B_p = \prod_{b \in p} \sigma^z_b.
    \end{equation}
\end{itemize}

The $A_{\vb*{i}}$ operators and the $B_{p}$ operators can be verified to commute with each other, and we have 
\begin{equation}
    A_s = \pm 1, \quad B_p = \pm 1.
\end{equation}
Since the $A_s$ operators and $B_p$ operators commute with the Hamiltonian, 
a ground state of the Hamiltonian is also an eigenstate of these operators, 
and from the Hamiltonian it is clear that 
\begin{equation}
    A_s \ket*{\text{ground}} = \ket*{\text{ground}}, \quad B_p \ket*{\text{ground}} = \ket*{\text{ground}}.
    \label{eq:ground-state-toric-code-condition}
\end{equation}
A state is a ground state if and only if it is in the subspace where \eqref{eq:ground-state-toric-code-condition} holds.

We can project an arbitrary state into the ground subspace. 
Note that $\ket*{\psi} = \ket*{\sigma^z = \uparrow}$ is a state where $B_p = 1$ holds, 
and $(1 + A_s)/{2}$ is a projector operator to the subspace where $A_s = 1$, the state 
\begin{equation}
    \ket*{\psi_\text{loop}} = \prod_s \frac{1 + A_s}{2} \ket*{\sigma^z = \uparrow}
    \label{eq:ground-state-1}
\end{equation}
is a ground state. \eqref{eq:ground-state-1} is a so-called string condensation state, 
which is the superposition of a series states where $\sigma^z = \downarrow$ bonds form loops.
\eqref{eq:ground-state-1} is definitely not the only ground state. 
The state
\begin{equation}
    \ket*{\psi_\text{loop}'} = \prod_p \frac{1 + B_p}{2} \ket*{\sigma^x = \rightarrow},
    \label{eq:ground-state-2}
\end{equation}
for example, is another ground state, 
where $\ket*{\sigma^x = \rightarrow}$ is a eigenstate of $A_s$ operators and we project it into the $B_p = 1$ subspace.

Now we try to determine the size of ground subspace. 
Suppose there are $N$ sites, so there are $2N$ bonds, so we have $2N$ degrees of freedom.%
\footnote{
    Note that $\sigma^x$ and $\sigma^z$ are \emph{not} two degrees of freedom, as they do not commute.
    In classical statistical mechanics sometimes we consider $x$ and $p$ to be two degrees of freedom, but in the quantum context we say that ``a 1D Newtonian particle has one degree of freedom, which is its position'', and view $p$ as something that provides dynamics for the particle.
    The same rule applies here, where we may view $\sigma^z$ or $\sigma^x$ as \emph{the degree of freedom} on one bond, but not both.
}
The fact that there are $N$ sites also indicates that the numbers of $A_s$ operators and $B_p$ operators are both $N$.
Therefore there seems to be $2N$ constraints given by \eqref{eq:ground-state-toric-code-condition}, 
so the ground subspace has no degrees of freedom left and the dimension is $1$.
However, on a torus we have 
\[
    \prod_s A_s = \prod_p B_p = 1,
\]
which is a consequence of the boundary periodic condition (or in other words, the definition of a torus).
Therefore there are actually only $2N - 2$ constraints.
There are, therefore, 2 degrees of freedom in the ground subspace, and the dimension is $2^2 = 4$.
We have four independent ground states.

What, then, are these degrees of freedom in the ground subspace? 
They are not local loop operators, since a ground state is a string condensation state and applying one loop operator to a ground state just move one of its components into another, permuting these components and in the end everything stays the same.
Note that, however, that loop components in \eqref{eq:ground-state-1} or \eqref{eq:ground-state-2} are all ``within'' the square.
Or, in the language of the torus, these loops are all contractible: they are generated from nothing by ``local'' loop operators $A_s$ and $B_p$, and therefore they can be gradually shrunk back to nothing.

The ground state degeneracy labeled by the non-contractible loops actually lead to \emph{long range entanglement} in the toric code. 
The leading terms of entanglement entropy are
\begin{equation}
    S = - \ln 2 + \alpha L + \cdots,
\end{equation}
and we see that apart from the usual area law entanglement entropy, we have a $- \ln 2$ term, which is the \concept{topological entanglement entropy}.
Generally, we have 
\begin{equation}
    S_\text{topo} = - \ln D,
\end{equation}
where $D$ is a parameter dubbed \concept{the quantum dimension}.

Now we investigate what are excitations of the toric code. Note that $A_s$ is actually the generator of $\mathbb{Z}_2$ gauge transformation, so $A_s = -1$ means we have a $\mathbb{Z}_2$ charge on the site of $s$ - we say an \concept{$e$ particle} exists there.
Also, in a $\mathbb{Z}_2$ gauge theory $\sigma^z$ serves as the hopping term of fermions coupled to the gauge field,
and therefore $B_p$ is the $\mathbb{Z}_2$ flux of plaquette $p$, and $B_p = -1$ means there is a $\mathbb{Z}_2$ flux in the plaquette $p$ - we say an \concept{$m$ particle} resides there.

We can soon use string operators to study the self statistics of $e$ and $m$ are the same as bosons, but they have nontrivial mutual exchange statistics, and therefore we have 
\begin{equation}
    \epsilon = e \times m
\end{equation}
as fermions.
The particle species $e$ and $m$, therefore, are called \concept{semions} in that they are ``halves of a fermion'.'

\section{Generalization: quantum double models}

In an ordinary $\mathbb{Z}_2$ gauge theory we only care about $\mathbb{Z}_2$ charges.
In the toric code, however, both the charge and the flux are important.
We say the toric code is a $\mathbb{Z}_2 \times \mathbb{Z}_2$ theory, in that the four anyon species are irreducible representations of $\mathbb{Z}_2 \times \mathbb{Z}_2$.

We can also say the toric code is a $D[\mathbb{Z}_2]$ theory, where $D[G]$ is the \concept{quantum double} of $G$, defined as
\begin{equation}
    D[G] = C[G] \times F[G],
\end{equation}
where $C[G]$ is the group algebra of $G$ and $F[G]$ is the \emph{function space} on $G$, which is the space of functions from $G$ to $\mathbb{C}$ and consists of the matrices of irreducible representations of $G$.
For Abelian models we have 
\begin{equation}
    D[G] = G \times G.
\end{equation}

Like a toric code model, a \concept{quantum double model} is defined by the following data:
\begin{itemize}
    \item A square lattice 
    \item A group $G$, the group members of which are operators defined on bonds of the lattice.
    \item The Hamiltonian is 
    \begin{equation}
        H = - \sum_\nu A_\nu - \sum_p B_p,
    \end{equation}
    where 
    \begin{equation}
        A_\nu = \frac{1}{\abs*{G}} \sum_{g \in G} A_\nu^g,
    \end{equation}
\end{itemize}

\section{Modular matrices of the toric code}

Now we consider how can a system transit from a ground state of \eqref{eq:toric-code} to another one.
What we need is to turn one non-contractible loop on a torus to another.
For a torus, $\mathrm{SL}(2, \mathbb{Z})$ can handle the job well.
$S$ rotates the square lattice by \SI{90}{\degree}, and therefore change the direction of the loops.
$T$ is the shear operation for the square lattice.
Note that the ground subspace is four dimensional and we need to find a 4D representation of $\mathrm{SL}(2, \mathbb{Z})$.
It turns out that 
\begin{equation}
    T = \diag(1, 1, 1, -1)
\end{equation}
and 
\begin{equation}
    S = \frac{1}{2} \pmqty{1 & 1 & 1 & 1 \\ 1 & 1 & -1 & -1 \\ 1 & -1 & 1 & -1 \\ }
\end{equation}
We find $T$ is the self statistics of the anyons and $S$ is the mutual statistics.
This is not coincidence. The following argument % TODO

We find that SPTs and group-based intrinsic topological orders can both be classified by Dijkgraaf-Witten TQFTs.
This is not coincidence because there is indeed duality between SPTs and group-based intrinsic topological orders, and they can be both characterized by group cohomology.

\end{document}