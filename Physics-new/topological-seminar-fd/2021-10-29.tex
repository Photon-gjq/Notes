\documentclass[hyperref, a4paper]{article}

\usepackage{geometry}
\usepackage{float}
\usepackage{titling}
\usepackage{titlesec}
% No longer needed, since we will use enumitem package
% \usepackage{paralist}
\usepackage{enumitem}
\usepackage{footnote}
\usepackage{enumerate}
\usepackage{amsmath, amssymb, amsthm}
\usepackage{mathtools}
\usepackage{bbm}
\usepackage{cite}
\usepackage{graphicx}
\usepackage{subcaption}
\usepackage{physics}
\usepackage{tensor}
\usepackage{siunitx}
\usepackage{booktabs}
\usepackage[version=4]{mhchem}
\usepackage{tikz}
\usepackage{xcolor}
\usepackage{listings}
\usepackage{autobreak}
\usepackage[ruled, vlined, linesnumbered]{algorithm2e}
\usepackage{xr-hyper}
\usepackage[colorlinks,unicode]{hyperref} % , linkcolor=black, anchorcolor=black, citecolor=black, urlcolor=black, filecolor=black
\usepackage{prettyref}

% Page style
\geometry{left=3.18cm,right=3.18cm,top=2.54cm,bottom=2.54cm}
\titlespacing{\paragraph}{0pt}{1pt}{10pt}[20pt]
\setlength{\droptitle}{-5em}
\preauthor{\vspace{-10pt}\begin{center}}
\postauthor{\par\end{center}}

% More compact lists 
\setlist[itemize]{itemindent=17pt, leftmargin=1pt}

% Math operators
\DeclareMathOperator{\timeorder}{T}
\DeclareMathOperator{\diag}{diag}
\DeclareMathOperator{\legpoly}{P}
\DeclareMathOperator{\primevalue}{P}
\DeclareMathOperator{\sgn}{sgn}
\newcommand*{\ii}{\mathrm{i}}
\newcommand*{\ee}{\mathrm{e}}
\newcommand*{\const}{\mathrm{const}}
\newcommand*{\suchthat}{\quad \text{s.t.} \quad}
\newcommand*{\argmin}{\arg\min}
\newcommand*{\argmax}{\arg\max}
\newcommand*{\normalorder}[1]{: #1 :}
\newcommand*{\pair}[1]{\langle #1 \rangle}
\newcommand*{\fd}[1]{\mathcal{D} #1}
\DeclareMathOperator{\bigO}{\mathcal{O}}
\DeclareMathOperator{\object}{Ob}
\DeclareMathOperator{\morphism}{Hom}

% TikZ setting
\usetikzlibrary{arrows,shapes,positioning}
\usetikzlibrary{arrows.meta}
\usetikzlibrary{decorations.markings}
\tikzstyle arrowstyle=[scale=1]
\tikzstyle directed=[postaction={decorate,decoration={markings,
    mark=at position .5 with {\arrow[arrowstyle]{stealth}}}}]
\tikzstyle ray=[directed, thick]
\tikzstyle dot=[anchor=base,fill,circle,inner sep=1pt]

% Algorithm setting
% Julia-style code
\SetKwIF{If}{ElseIf}{Else}{if}{}{elseif}{else}{end}
\SetKwFor{For}{for}{}{end}
\SetKwFor{While}{while}{}{end}
\SetKwProg{Function}{function}{}{end}
\SetArgSty{textnormal}

\newcommand*{\concept}[1]{{\textbf{#1}}}

\newrefformat{fig}{Figure~\ref{#1}}

% Embedded codes
\lstset{basicstyle=\ttfamily,
  showstringspaces=false,
  commentstyle=\color{gray},
  keywordstyle=\color{blue}
}

% Disable unsupported commands in bookmark titles 
\pdfstringdefDisableCommands{%
  \def\\{}%
  \def\texttt#1{<#1>}%
  \def\mathbb#1{#1}%
}
\pdfstringdefDisableCommands{\def\eqref#1{(\ref{#1})}}

\makeatletter
\pdfstringdefDisableCommands{\let\HyPsd@CatcodeWarning\@gobble}
\makeatother

\title{Braiding and Fusion in Topological Orders by Prof. Yidun Wan}
\author{Jinyuan Wu}

\begin{document}

\maketitle

This article is a lecture note of Prof. Yidun Wang's lecture on the topological states of matter seminar on October 29, 2021.

\section{Fusion algebra in topological orders and the quantum dimension}

We know anyons in a topological order have fusion and braiding, which may be understood as two types of interaction channels of anyons.
We, however, investigate only the \emph{algebraic} features of braiding and fusion, ignoring the dynamics.

A \concept{fusion albegra} has the following data:
\begin{itemize}
    \item A set of anyon types $a, b, c, \ldots$.
    \item A set of fusion rules in the form of 
    \begin{equation}
        a \times b = \sum_c N^c_{ab} c, \quad N^c_{ab} \in \mathbb{N}.
    \end{equation}
\end{itemize}

Fusion is kind of like the ``fusion'' of spins, for example the famous singlet-triplet decomposition 
\[
    \frac{1}{2} \otimes \frac{1}{2} = 0 \oplus 1.
\]
The decomposition can be important with certain terms in the Hamiltonian. For example, if we have a $\sum_{\pair{\vb*{i}, \vb*{j}}} \vb*{S}_{\vb*{i}} \cdot \vb*{S}_{\vb*{j}}$ term in the Hamiltonian, then the ground state tends to contain spin singlets instead of spin triplets.
The decomposition itself, however, has nothing to do with the dynamics; to make it ``activated'' we need certain Hamiltonians. 

Another concept can be defined on anyons is the \concept{quantum dimension}.
The quantum dimension $d_a$ of anyon species $a$ is defined as the \emph{asymptotic} dimension of the internal Hilbert space of a type-$a$ anyon.
Suppose the Hilbert space $\mathcal{H}_{a}(N)$ containing all possible states of a system with $N$ type-$a$ anyons, then we have 
\begin{equation}
    d_a^N \simeq \dim \mathcal{H}_a(N) \quad \text{as $N \to \infty$}.
\end{equation}
It can be verified that $d_a \geq 1$, and for the vacuum we have $d = 1$. Note that $d_a$ may not be an integer.
The fusion rules can be used to derive the quantum dimensions, because we have 
\begin{equation}
    d_a d_b = \sum_c N_{ab}^c d_c.
\end{equation}

In an Abelian topological order, for all $a$ and $b$ there exists an anyon species $c$ such that 
\begin{equation}
    a \times b = c,
\end{equation}
and thus all quantum dimensions are $1$. 
We find that an anyon in an Abelian topological order is quite a ``local'' or ``simple'' object, the only labels of which are the position and the topological spin, without complicated inner states.

The \concept{Ising topological order} is defined with the following data: there are three types of anyons, namely $1, \psi, \sigma$, and we have 
\begin{equation}
    1 \times \sigma = \sigma, \quad 1 \times \psi = \psi, \quad  \sigma \times \sigma = 1 + \psi, \quad \sigma \times \psi = \sigma, \quad \psi \times \psi = 1,
\end{equation}
from which we have 
\begin{equation}
    d_\sigma = \sqrt{2}.
\end{equation}
This is an example of non-rational quantum dimension in non-Abelian topological orders.

We can also define the concept of \concept{fusion space}. Suppose $N_{ab}^c \geq 1$, additional labels are required to indicate \emph{how} $a$ and $b$ fuse into $c$.
These labels are the labels of the fusion space, and the dimension of the fusion space $V_{ab}^c$ - the subspace of the Hilbert space spanned by $a \times b$ in which the fusion result is restricted to $c$ but the additional labels specifying the exact fusion channel may vary - is 
\begin{equation}
    \dim V_{ab}^c = N_{ab}^c.
\end{equation}

A good example where the fusion space occurs is a sphere with several anyons placed on it.
We require the fusion of all anyons on a sphere to be the trivial one, or with a little jargons, that the total topological charges on a sphere is zero.
Therefore, the Hilbert space of a spherical system with anyons $a_1, a_2, \ldots, a_N$ is the Hilbert space from $a_1 \times a_2 \times \cdots a_N$ to the trivial anyon $1$.

Consider a \concept{3-punctured sphere}, which is a sphere with three anyons placed on it, namely $a, b$ and $c$.

In this way we can predict that $1 \notin V_{ab}^c$, or otherwise we have 
\[
    a \times b \times c = c + \cdots,
\]
which is obviously not the trivial charge.

Note that 
\begin{equation}
    V_{abc}^e = \bigoplus_{c} V_{ab}^c V_{cd}^e,
\end{equation}
A \concept{4-punctured sphere}, which is a sphere with four anyons placed on it, can always be constructed by two 3-punctured spheres.

Anyon fusion algebra on a torus: 
\begin{equation}
    \otimes_a 
\end{equation}

Now we consider an \concept{$n$-punctured sphere}, which may also be understood as a $(n-1)$-punctured disk with one anyon placed on $\infty$.
The Hilbert space is 
\begin{equation}
    \mathcal{H} = V_{a a \cdots a}^1 = \bigoplus_{\substack{\{b_i\}}} V_{aa}^{b_1} V_{a b_1}^{b_2} \cdots V_{a b_{n-3}}^a,
\end{equation}
and the dimension of the Hilbert space is 
\[
    \dim \mathcal{H} = \sum_{\{b_i\}} N_{aa}^{b_1} N_{a b_1}^{b_2} \cdots N_{a b_{n-3}}^a.
\]
We denote the matrix $[N_{ab}^c]_{bc}$ as $N_a$, and we have 
\[
    \dim \mathcal{H} = \Trace (N_a)^{n-2} \simeq \lambda^{n-2} \simeq \lambda^n \quad \text{as $n \to \infty$},
\]
where $\lambda$ is the largest eigenvalue of $N_a$.
Now we see the hidden meaning of the quantum dimension: it is the largest eigenvalue of $N_a$.

A non-integer quantum dimension means anyons are not local objects: it is \emph{impossible} to write down the Hilbert space of the whole system as the tensor product of the Hilbert space of each anyon.
Indeed that is what we need for \emph{topological quantum computation}.
Abelian topological orders cannot be used for topological quantum computation because any operation on one anyon only results in a change of the phase of that particular anyon without influencing other parts of the system.

\section{The $F$-symbol and basis transformation in fusion spaces}

What we have done are all based on the assumption that fusion is associative.
The associative condition means
\begin{equation}
    a \times (b \times c) \simeq (a \times b) \times c.
    \label{eq:fusion-assoc-cond}
\end{equation}
The letters $a, b$ and $c$ here are abstract notion of anyons.
The \emph{basis} of $a \times (b \times c)$ and $(a \times b) \times c$ may be different, however, and in this way \eqref{eq:fusion-assoc-cond} induces a basis transformation.

% TODO: fusion tree

We sometimes call the $F$-symbol the \concept{6-$j$ symbol}. A representation transformation guided by the $F$-symbol is also called an \concept{$F$-move}.

Here an issue on consistency occurs. If there are more than three anyons, do different fusion orders still give the same Hilbert space?
The \concept{coherence theorem} assures that if the order of the fusion of four anyons do not affect the Hilbert space, than so does the fusion of arbitrary anyons.
In other words, we just need to check the coherence of four anyons.

We call this condition \concept{the pentagon identity}. 
A set of fusion rules must satisfy this identity, or otherwise different orders of the fusion of four anyons result in different Hilbert spaces, which is ridiculous.

The additional constraint on the fusion algebra makes it a \concept{fusion category}.

As an example we consider a topological order where there are two anyons, namely 0 and 1.
We may assign an Abelian fusion algebra
\begin{equation}
    1 \times 1 = 0, \quad 0 \times 1 = 1,
\end{equation}
and then by solving the pentagon identity we find two solutions, corresponding to two consistent topological orders.
One is a chiral $\mathbb{Z}_2$ topological order, where the anyons are $1$ and $e$.
This is half of the toric code topological order. Chiral topological orders currently cannot be realized in lattice models.
Another solution has $1$ and $s$ where $s$ is a \concept{semion} or ``half of a fermion''.
We may also assign a non-Abelian fusion algebra 
\begin{equation}
    0 \times 1 = 1, \quad 1 \times 1 = 0 + 1,
\end{equation}
which is called the \concept{Fibonacci topological order}, a highly non-trivial one that can be used to realize universal quantum computing.

\section{Braiding}

A braiding is in the form of 
\begin{equation}
    \begin{gathered}
        \begin{tikzpicture}[x=0.75pt,y=0.75pt,yscale=-1,xscale=1]
            %uncomment if require: \path (0,300); %set diagram left start at 0, and has height of 300
            
            %Curve Lines [id:da6782864008586009] 
            \draw    (206,166.67) .. controls (154,163.67) and (164,99.67) .. (242,80.67) ;
            %Shape: Circle [id:dp6017914867716119] 
            \draw  [draw opacity=0][fill={rgb, 255:red, 255; green, 255; blue, 255 }  ,fill opacity=1 ] (198,95.5) .. controls (198,90.81) and (201.81,87) .. (206.5,87) .. controls (211.19,87) and (215,90.81) .. (215,95.5) .. controls (215,100.19) and (211.19,104) .. (206.5,104) .. controls (201.81,104) and (198,100.19) .. (198,95.5) -- cycle ;
            %Curve Lines [id:da06524248613459038] 
            \draw    (206,166.67) .. controls (258,163.67) and (248,99.67) .. (170,80.67) ;
            %Straight Lines [id:da7473603270752556] 
            \draw    (206,166.67) -- (206,227.33) ;
            
            % Text Node
            \draw (154,68.4) node [anchor=north west][inner sep=0.75pt]    {$a$};
            % Text Node
            \draw (251,68.4) node [anchor=north west][inner sep=0.75pt]    {$b$};
            % Text Node
            \draw (206,230.73) node [anchor=north] [inner sep=0.75pt]    {$c$};
            \end{tikzpicture}            
    \end{gathered} = R^{ab}_c \begin{gathered}
        \begin{tikzpicture}[x=0.75pt,y=0.75pt,yscale=-1,xscale=1]
            %uncomment if require: \path (0,300); %set diagram left start at 0, and has height of 300
            
            %Straight Lines [id:da8721179915382755] 
            \draw    (226,141.67) -- (226,202.33) ;
            %Straight Lines [id:da08458411844613889] 
            \draw    (189,95) -- (226,141.67) ;
            %Straight Lines [id:da7904698341446577] 
            \draw    (226,141.67) -- (258,96) ;
            
            % Text Node
            \draw (183,73.4) node [anchor=north west][inner sep=0.75pt]    {$a$};
            % Text Node
            \draw (254,74.4) node [anchor=north west][inner sep=0.75pt]    {$b$};
            % Text Node
            \draw (226,205.73) node [anchor=north] [inner sep=0.75pt]    {$c$};
            \end{tikzpicture}            
    \end{gathered}.
\end{equation}
After the introduction of braiding, with only three anyons we have a coherence issue. There is an additional coherence condition: \concept{the hexagon identity}.

A fusion category equipped with an additional braiding structure, namely the $R$-matrix and the hexagon identity, is called a \concept{braided fusion category}.

\end{document}