\documentclass[a4paper]{article}

\usepackage{geometry}
\usepackage{caption}
\usepackage{subcaption}
\usepackage{abstract}
\usepackage{amsmath, amssymb}
\usepackage{qtree}
\usepackage{gb4e}
\usepackage[colorlinks, urlcolor=cyan]{hyperref}
\usepackage{prettyref}

\geometry{left=3.18cm,right=3.18cm,top=2.54cm,bottom=2.54cm}

\title{Laudate, Laudate Dominum}
\author{Jinyuan Wu}

\begin{document}

\maketitle

\section{The Text}

\emph{Laudate, Laudate Dominum} is a hymn by Christopher Walker, a famous Catholic composer, and is written for the new Cathedral of Our Lady of the Angels, Los Angeles, California \cite{ocp-preview}.
The music is played in several videos available on Youtube \cite{youtube-1,youtube-2}, in one of which it was used as the entrance song in a mass held by Pope Francis. 

The lyrics can be found in OCP's website \cite{ocp-preview}. Since the material is copyrighted, we do not put a full copy here. You can download a preview from the link above.
The lyrics is written in Latin, English and Spanish.

Here is the Latin part of the lyrics:
\begin{quotation}
    Laudate, laudate Dominum, omnes gentes, laudate Dominum.

    Exsultate, jubilate per annos Domini, omnes gentes.
\end{quotation}

\section{Translation}

The Latin part can be glossed as follows:
\begin{exe} 
    \sn 
    \gll Laudate,               laudate                Dominum,     omnes        gentes,       laudate                Dominum. \\
         praise-IMP.PRS.ACT.2PL praise-IMP.PRS.ACT.2PL Lord-SG.ACC  every-PL.VOC people-PL.VOC praise-IMP.PRS.ACT.2PL Lord-SG.ACC \\
    \glt `Let us praise, Let us praise the Lord, O all people, let us praise the Lord.'
    \sn 
    \gll Exsultate,              jubilate                per      annos             Domini,     omnes        gentes. \\
         rejoice-IMP.PRS.ACT.2PL cheer-IMP.PRS.ACT.2PL   through  time-PL.ACC       Lord-SG.ACC every-PL.VOC people-PL.VOC \\
    \glt `Let us rejoice, let us in the Lord's time, O all people.'
\end{exe}   

\bibliographystyle{plain}
\bibliography{laudate} 

\end{document}