\documentclass[a4paper]{article}

\usepackage[utf8]{inputenc}
\usepackage{geometry}
\usepackage{caption}
\usepackage{subcaption}
\usepackage{abstract}
% \usepackage{paralist}
\usepackage{amsmath, amssymb}
\usepackage{qtree}
\usepackage{gb4e}
\usepackage[colorlinks, urlcolor=cyan]{hyperref}
\usepackage{prettyref}

\geometry{left=3.18cm,right=3.18cm,top=2.54cm,bottom=2.54cm}

% \renewenvironment{itemize}{\begin{compactitem}}{\end{compactitem}}
% \renewenvironment{enumerate}{\begin{compactenum}}{\end{compactenum}}

\newcommand*{\concept}[1]{{\textbf{#1}}}

\title{Translation of Adeste Fideles}
\author{Jinyuan Wu}

\begin{document}

\maketitle

\section{The origin and the text of \emph{Adeste Fideles}}

\emph{Adeste Fideles} is a Christmas carol by anonymous. 
It has been attributed to various authors, but the tru origin is not clear. 
Anyway, it origins in English Catholic Church as a Latin hymn, 
and the most widely accepted English version of it is translated by the English Catholic priest Frederick Oakeley,
and is widely spread in the English-speaking countries \cite{wiki}.

Several versions can be found on Youtube. 

\section{Literal translation of the original lyrics}

\begin{exe}
    
\sn 
\gll Adeste                  fideles                    laeti           triumphantes, \\
     arrive-IMP.PRS.ACT.2PL  faithful.people-PL.VOC(?)  happy-PL.NOM.M  rejoice-PRS.ACT.PTCP-PL.NOM.M \\ 
\glt ``The faithful, O come, happy and rejoicing,''

\sn
\gll Venite,              venite                in Bethlehem. \\
     come-IMP.PRS.ACT.2PL  come-IMP.PRS.ACT.2PL in Bethlehem \\
\glt ``O come, O come into Bethlehem.''

\sn
\gll Natum                     videte,              Regem         angelorum: \\
    born-PRF.PASS.PTCP-SG.ACC  look-IMP.PRS.ACT.2PL king-SG.ACC   angel-PL.GEN  \\
\glt ``O look the king of angels having been born:''

\sn
\gll Venite                adoremus               Dominum. \\
     come-IMP.PRS.ACT.2PL  adore-SBJ.PRS.ACT.1PL  lord-SG.ACC \\
\glt ``O come we should adore the Lord.''

\end{exe}

\paragraph{Notes} \begin{itemize}
    \item In both Latin and English, the imperative is by default second person. Therefore \emph{fideles} should be 
    parsed as vocative.
    \item \emph{Bethlehem} is indeclinable. It has a declinable counterpart, \emph{Bethlehemum}. 
    \item The word \emph{adoremus} is a so-called \concept{jussive}, a subjunctive used to express a command or a suggestion: ``We should ...''
\end{itemize}

\begin{exe}

\sn
\gll Deum        de     Deo,         lumen         de    lumine,  \\
    god-SG.ACC   from   god-SG.ABL   light-SG.ACC  from  light-SG.ABL \\
\glt ``God from God, light from light,''

\sn
\gll gestant                puellae      viscera.               \\
     carry-IND.PRS.ACT.3PL  girl-SG.GEN  internal.organ-PL.NOM  \\
\glt ``The inner organs of the girl persists.''

\sn
\gll Deum        verum,   genitum                    non  factum. \\
     god-SG.ACC  truly    beget-PRF.PASS.PTCP-SG.ACC not  make-PRF.PASS.PTCP-SG.ACC \\
\glt ``Truly God, begotten not made.''

\sn
\gll Venite                adoremus               Dominum. \\
     come-IMP.PRS.ACT.2PL  adore-SBJ.PRS.ACT.1PL  lord-SG.ACC \\
\glt ``O come we should adore the Lord.''

\end{exe}

\paragraph{Notes} \begin{itemize}
    \item The word \emph{lumen} can be nominative, accusative or vocative, 
    but we find that the status of \emph{lumen} is the same as \emph{Deum}, so their cases must be the same.
    \item These lines involve some doctrine in Christianity, or more precisely Catholicism. 
    The clause \emph{gestant puellae viscera} means the \concept{Immaculate Conception}, that Mary remains a virgin even after giving birth to Jesus. 
    The clause \emph{genitum non factum} is a sentence from the Nicene Creed, that Jesus is \emph{beget} directly from the Holy Father, 
    and not \emph{made} (in the same way other things are made) by God.
\end{itemize}

\begin{exe}

\sn
\gll Cantet nunc io, chorus angelorum;

\sn
\gll Cantet nunc aula caelestium,

\sn
\gll Gloria, gloria in excelsis Deo,

\end{exe}

Ergo qui natus die hodierna.
Jesu, tibi sit gloria,
Patris aeterni Verbum caro factum.

\section{The usual English version}

\begin{quotation}
    O come, all ye faithful, joyful and triumphant!

    O come ye, O come ye to Bethlehem;

    Come and behold him

    Born the King of Angels:
    
    O come, let us adore Him, (3 times)
    
    Christ the Lord.
\end{quotation}

\begin{quotation}
    God of God, light of light,

    Lo, he abhors not the Virgin's womb;
    
    Very God, begotten, not created:
    
    O come, let us adore Him, (3 times)
    
    Christ the Lord.
\end{quotation}

\begin{quotation}
    Sing, choirs of angels, sing in exultation,
    
    Sing, all ye citizens of Heaven above!
    
    Glory to God, glory in the highest:
    
    O come, let us adore Him, (3 times)
    
    Christ the Lord.
\end{quotation}

\begin{quotation}
    Yea, Lord, we greet thee, born this happy morning;
    
    Jesus, to thee be glory given!
    
    Word of the Father, now in flesh appearing!
    
    O come, let us adore Him, (3 times)
    
    Christ the Lord.
\end{quotation}

\bibliographystyle{plain}
\bibliography{adeste-fideles}

\end{document}