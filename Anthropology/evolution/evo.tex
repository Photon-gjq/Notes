\documentclass[hyperref, UTF8, a4paper, oneside]{ctexart}

\usepackage{geometry}
\usepackage{titling}
\usepackage{titlesec}
\usepackage{paralist}
\usepackage{footnote}
\usepackage{enumerate}
\usepackage{autobreak}
\usepackage{amsmath, amssymb, amsthm}
\usepackage{mathtools}
\usepackage{bbm}
\usepackage{cite}
\usepackage{graphicx}
\usepackage{subfigure}
\usepackage{physics}
\usepackage{siunitx}
\usepackage{tikz}
\usepackage[compat=1.1.0]{tikz-feynhand}
\usepackage[ruled, vlined, linesnumbered, noend]{algorithm2e}
\usepackage[colorlinks, linkcolor=black, anchorcolor=black, citecolor=black]{hyperref}
\usepackage[most]{tcolorbox}
\usepackage{caption}
\usepackage{prettyref}

\geometry{left=3.18cm,right=3.18cm,top=2.54cm,bottom=2.54cm}
\titlespacing{\paragraph}{0pt}{1pt}{10pt}[20pt]
\setlength{\droptitle}{-5em}
\preauthor{\vspace{-10pt}\begin{center}}
\postauthor{\par\end{center}}

\DeclareMathOperator{\timeorder}{\mathcal{T}}
\DeclareMathOperator{\diag}{diag}
\DeclareMathOperator{\legpoly}{P}
\DeclareMathOperator{\primevalue}{P}
\DeclareMathOperator{\sgn}{sgn}
\newcommand*{\ii}{\mathrm{i}}
\newcommand*{\ee}{\mathrm{e}}
\newcommand*{\const}{\mathrm{const}}
\newcommand*{\suchthat}{\quad \text{s.t.} \quad}
\newcommand*{\argmin}{\arg\min}
\newcommand*{\argmax}{\arg\max}
\newcommand*{\normalorder}[1]{: #1 :}
\newcommand*{\pair}[1]{\langle #1 \rangle}
\newcommand*{\fd}[1]{\mathcal{D} #1}

\newrefformat{chap}{第\ref{#1}章}
\newrefformat{sec}{第\ref{#1}节}
\newrefformat{note}{注\ref{#1}}
\newrefformat{fig}{图\ref{#1}}
\newrefformat{alg}{算法\ref{#1}}
\newrefformat{back}{背景知识\ref{#1}}
\newrefformat{info}{资料框\ref{#1}}
\newrefformat{warn}{注意事项\ref{#1}}
\renewcommand{\autoref}{\prettyref}

\usetikzlibrary{arrows,shapes,positioning}
\usetikzlibrary{arrows.meta}
\usetikzlibrary{decorations.markings}
\tikzstyle arrowstyle=[scale=1]
\tikzstyle directed=[postaction={decorate,decoration={markings,
    mark=at position .5 with {\arrow[arrowstyle]{stealth}}}}]
\tikzstyle ray=[directed, thick]
\tikzstyle dot=[anchor=base,fill,circle,inner sep=1pt]

% Algorithm setting
\renewcommand{\algorithmcfname}{算法}
% Python-style code
\SetKwIF{If}{ElseIf}{Else}{if}{:}{elif:}{else:}{}
\SetKwFor{For}{for}{:}{}
\SetKwFor{While}{while}{:}{}
\SetKwInput{KwData}{输入}
\SetKwInput{KwResult}{输出}
\SetArgSty{textnormal}

\tcbuselibrary{skins, breakable, theorems}

\newtcbtheorem[number within=section]{back}{背景知识}%
  {colback=blue!5,colframe=blue!65,fonttitle=\bfseries, breakable}{back}
\newtcbtheorem[number within=section]{info}{资料框}%
  {colback=blue!5,colframe=blue!65,fonttitle=\bfseries, breakable}{info}
\newtcbtheorem[number within=section]{warning}{注意事项}%
  {colback=orange!5,colframe=orange!65,fonttitle=\bfseries, breakable}{warn}

\renewcommand{\emph}[1]{\textbf{#1}}
\newcommand*{\concept}[1]{\underline{\textbf{#1}}}

\numberwithin{equation}{section}

\newcommand{\hmn}[1]{% Hermann-Maguin notation
  \ensuremath{\begingroup\setupHMN #1\endgroup}%
}

\newcommand{\setupHMN}{%
  \doHMN{-}{\HMNoverline}%
  \doHMN{*}{\HMNminverse}%
  \doHMN{i}{\infty}
}

\newcommand{\doHMN}[2]{%
  \begingroup\lccode`~=`#1
  \lowercase{\endgroup\let~}#2%
  \mathcode`#1="8000
}

\newcommand{\HMNminverse}[1]{\frac{#1}{m}}
\newcommand{\HMNoverline}[1]{\mkern1mu\overline{\mkern-1mu#1\mkern-1mu}\mkern1mu}

\newcommand{\Ztwo}{$\mathbb{Z}_2$}

\newcommand{\bigO}[1]{\mathcal{O}(#1)}

\title{进化论}
\author{吴晋渊}

\begin{document}

\maketitle

\section{基本观点}

\subsection{生命从何而来}

关于生命从何而来,各个民族都有他们自己的\concept{创世说},即神创造天地万物的故事。
然而神创论并不是一个特别有趣的理论,因为从它出发预测不出什么东西来。

\concept{天外起源说}认为早期生命可能来自地球以外。支持这一说法的证据包括陨石上发现了氨基酸和核糖核酸,以及地球上的生物分子具有单一的手性。

\subsection{演化学说的历史}

“物种是在变化的”这个说法最早可以追溯到古希腊的一些思想家,他们假定生命来自于某种海边的淤泥。
基督宗教对“七天创世”的字面解释导致这一类说法几乎完全被遗忘。启蒙运动之后的一些思想家——比如布丰——也提出过进化思想,不过由于避免触怒宗教人士等原因,往往对此不再深究。
生物分类学先驱林奈——双名法的创始人——就将他的双名法当成对永恒不变的“自然的体系”的一种描述。
拉马克的进化论可能是第一个认真的、并且获得广泛传播的进化理论,他认为生物同时受到两种力量的驱动:一种是“适应力”,即用进废退,一种则是“复杂性力”,即生物有从简单到复杂的倾向。
这两个机制在后世都受到批判:仅仅依靠多使用不能改变基因,因此用进废退的说法是错误的,的确,表观遗传是一种将后天获得的特性传递给后代的机制,但是且不说它并不能用“表观遗传”概括;至于复杂性力,这基本上就是神创论的一种翻版了。

如今进化的概念不仅仅适用于物种;作为一个典型的复杂适应系统,它还能用于描述社会行为、文化、商业模式、流行病学,以及其它现象。
进化论不再是生物学的一个分支,甚至也不是它的一个框架,而是一个覆盖广泛的“进化科学”。

\subsection{生物进化是一个事实}

虽然不少人出于一些原因——可能是宗教或是特定的哲学观念——声称进化论无法给出可预测的语言、只是一种说法等等,仍然有大量证据强烈暗示了进化的存在,并且如果不使用进化论,无法系统地解释它们。

我们有关于生物个体的证据。通常这能够证明\concept{宏观进化},即进化导致全新物种的产生:
\begin{itemize}
  \item 化石证据:现今较复杂的生物结构有古代的不那么复杂的前体。
  \item 同源器官证据:功能完全不同的器官存在共同的祖先。
  \item 比较胚胎学和发育证据。
\end{itemize}

我们也有生态学上的证据。通常这能够证明有\concept{微进化},即种群中的一些基因的频率发生变化:
\begin{itemize}
  \item 渔业的自然实验:捕鱼业发达的地区,鱼的体型会快速下降。
  这是因为一个地区的渔业通常使用的那种渔网的网眼施加了一个强烈的选择压力,从而只有基因型为“长得较小”的鱼才能够留存下来。
  当更加严格的渔业资源保护规定施加之后,鱼的体型恢复得很慢,预示着种群的基因型分布确实发生了明显变化,而不是因为恶劣的生存环境而后天地、获得性地减小了体型。
  \item 澳大利亚的有毒蟾蜍和蛇:当一类有毒蟾蜍引入澳大利亚后,当地的一种蛇的口部能张开的角度明显减小,因为嘴巴大而能够吃下蟾蜍的蛇死了。
  \item 育种可能是能够人为操纵的最典型例子。
\end{itemize}

我们甚至还有生物医学的证据。其中很大一部分是所谓的\concept{超微进化},它们的主体是生物体内部的一些东西,如:
\begin{itemize}
  \item 艾滋病是最为典型——也是最为令人头疼的案例。艾滋病病毒能够快速变异,而目前的艾滋病病毒已经具有多种药物的耐药性。
  目前的方案是鸡尾酒疗法,即同时使用多种药物,并且一段时间后轮换药物。
  较短时间内不能产生能够同时抵抗多种药物的毒株,而轮换药物能够让选择压力发生持续变化,从而能够同时抵抗多种药物的毒株即使出现,也无法占据主导。
  \item 癌细胞是另一个令人头疼的案例。肿瘤细胞具有明显的异质性,这就是癌症治疗耐药性的根源:持续使用一种药物只会筛选出能够抵抗它的细胞株,随后该细胞株占据主导。甚至还有癌症的生态学:例如,心脏很少出现肿瘤。
\end{itemize}
人类短短的历史中发生的进化能够解释很多其它疾病:
\begin{itemize}
  \item 早期人类的身体适应于长时间奔跑,代谢问题
\end{itemize}

\section{第一个生物}

应当指出进化论虽然已经广泛接受为正确了,但并非完全没有困难。这类似于凝聚态物理中第一性原理计算被相信能够解释一切,但是并不是所有现象都能够被\emph{ab initio}地解释清楚。
目前最大的困难可能就是早期生命是如何产生的。

中心法则似乎和化学演化产生生物的假定矛盾

是否RNA可能是第一个生物大分子?

然而RNA是非常不稳定的。

DNA和RNA可以同时产生

自催化网络,自催化

总之,目前无法确定第一个生命体到底是怎么产生的。

\end{document}